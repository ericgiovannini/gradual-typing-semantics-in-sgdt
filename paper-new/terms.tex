\section{A Simple Denotational Semantics for the Terms of GTLC}\label{sec:gtlc-terms}

In this section, we introduce the term syntax for the gradually-typed
lambda calculus (GTLC) and give a set-theoretic denotational semantics
using tools from SGDT.

In the following section, we will extend the denotational semantics to accommodate
the type and term precision orderings.


\subsection{Syntax}\label{sec:term-syntax}

Our syntax is based on fine-grained call by value, and as such it has
separate value and producer terms and typing judgments for each.

% Given a term $M$ of type $A$, the term $\bind{x}{M}{N}$ should be thought of as
% running $M$ to a value $V$ and then continuing as $N$, with $V$ in place of $x$.


\begin{align*}
  &\text{Types } A := \nat \alt \,\dyn \alt (A \ra A') \\
  &\text{Contexts } \Gamma := \cdot \alt (\Gamma, x : A) \\
  &\text{Values } V :=  \zro \alt \suc\, V \alt \lda{x}{M} \alt \up{A}{B} V \\ 
  &\text{Producers } M, N := \err_B \alt \matchnat {V} {M} {n} {M'} \\ 
  &\quad\quad \alt \ret {V} \alt \bind{x}{M}{N} \alt V_f\, V_x \alt \dn{A}{B} M 
\end{align*}


The value typing judgment is written $\vhasty{\Gamma}{V}{A}$ and 
the producer typing judgment is written $\phasty{\Gamma}{M}{A}$.

The typing rules are as expected, with a cast between $A$ to $B$ allowed only when $A \ltdyn B$.
The precise rules for $A \ltdyn B$ will be given below.
Notice that the upcast of a value is a value, since it always succeeds, while the downcast
of a value is a producer, since it may fail.

\begin{mathpar}
    % Var
    \inferrule*{ }{\vhasty {\cdot, \Gamma, x : A, \Gamma'} x A}

    % Err
    \inferrule*{ }{\phasty {\cdot, \Gamma} {\err_A} A} 
  
    % Zero and suc
    \inferrule*{ }{\vhasty \Gamma \zro \nat}
  
    \inferrule*{\vhasty \Gamma V \nat} {\vhasty \Gamma {\suc\, V} \nat}

    % Match-nat
    \inferrule*
    {\vhasty \Gamma V \nat \and 
     \phasty \Delta M A \and \phasty {\Gamma, n : \nat} {M'} A}
    {\phasty \Gamma {\matchnat {V} {M} {n} {M'}} A}
  
    % Lambda
    \inferrule* 
    {\phasty {\Gamma, x : A} M {A'}} 
    {\vhasty \Gamma {\lda x M} {A \ra A'}}
  
    % App
    \inferrule*
    {\vhasty \Gamma {V_f} {A \ra A'} \and \vhasty \Gamma {V_x} A}
    {\phasty {\Gamma} {V_f \, V_x} {A'}}

    % Ret
    \inferrule*
    {\vhasty \Gamma V A}
    {\phasty {\Gamma} {\ret\, V} {A}}

    % Bind
    \inferrule*
    {\phasty \Gamma M {A} \and \phasty{\Gamma, x : A}{N}{B} } % Need x : A in context
    {\phasty {\Gamma} {\bind{x}{M}{N}} {B}}

    % Upcast
    \inferrule*
    {A \ltdyn A' \and \vhasty \Gamma V A}
    {\vhasty \Gamma {\up A {A'} V} {A'} }

    \inferrule* % TODO is this correct?
    {A \ltdyn A' \and \phasty {\Gamma} {M} {A'}}
    {\phasty {\Gamma} {\dn A {A'} M} {A}}

\end{mathpar}


In the equational theory, we have $\beta$ and $\eta$ laws for function type,
as well a $\beta$ and $\eta$ law for bind.

\begin{mathpar}
  % Function Beta and Eta
  \inferrule*
  {\phasty {\Gamma, x : A} M {B} \and \vhasty \Gamma V A}
  {(\lda x M)\, V = M[V/x]}

  \inferrule*
  {\vhasty \Gamma V {A \ra A}}
  {\Gamma \vdash V = \lda x {V\, x}}

  % Ret Beta and Eta
  \inferrule*
  {}
  {(\bind{x}{\ret\, V}{N}) = N[V/x]}

  \inferrule*
  {\phasty {\Gamma} {M} {B}}
  {\bind{x}{M}{\ret x} = M}

  % Match-nat Beta
  \inferrule*
  {\phasty \Delta M A \and \phasty {\Gamma, n : \nat} {M'} A}
  {\matchnat{\zro}{M}{n}{M'} = M}

  \inferrule*
  {\vhasty \Gamma V \nat \and 
   \phasty \Gamma M B \and \phasty {\Gamma, n : \nat} {M'} B}
  {\matchnat{\suc\, V}{M}{n}{M'} = M'}

  % Match-nat Eta
  % This doesn't build in substitution
  \inferrule*
  {\hasty {\Gamma , x : \nat} M A}
  {M = \matchnat{x} {M[\zro / x]} {n} {M[(\suc\, n) / x]}}

\end{mathpar}

\subsubsection{Type Precision}\label{sec:type-precision}

The type precision rules specify what it means for a type $A$ to be more precise than $A'$.
We have reflexivity rules for $\dyn$ and $\nat$, as well as rules that $\nat$ is more precise than $\dyn$
and $\dyn \ra \dyn$ is more precise than $\dyn$.
We also have a congruence rule for function types stating that given $A_i \ltdyn A'_i$ and $A_o \ltdyn A'_o$, we can prove
$A_i \ra A_o \ltdyn A'_i \ra A'_o$. Note that precision is covariant in both the domain and codomain.
Finally, we can lift a relation on value types $A \ltdyn A'$ to a relation $\Ret A \ltdyn \Ret A'$ on
computation types.

\begin{mathpar}
  \inferrule*[right = \dyn]
    { }{\dyn \ltdyn\, \dyn}

  \inferrule*[right = \nat]
    { }{\nat \ltdyn \nat}

  \inferrule*[right = $\ra$]
    {A_i \ltdyn A'_i \and A_o \ltdyn A'_o }
    {(A_i \ra A_o) \ltdyn (A'_i \ra A'_o)}

  \inferrule*[right = $\textsf{Inj}_\nat$]
    { }{\nat \ltdyn\, \dyn}

  \inferrule*[right = $\textsf{Inj}_{\ra}$]
    { }
    {(\dyn \ra \dyn) \ltdyn\, \dyn}

  \inferrule*[right = $\injarr{}$]
    {(R \ra S) \ltdyn\, (\dyn \ra \dyn)}
    {(R \ra S) \ltdyn\, \dyn}

  
\end{mathpar}

We can prove that transitivity of type precision is admissible, i.e.,
if $A \ltdyn B$ and $B \ltdyn C$, then $A \ltdyn C$.

% Type precision derivations
Note that as a consequence of this presentation of the type precision rules, we
have that if $A \ltdyn A'$, there is a unique precision derivation that witnesses this.
As in previous work, we go a step farther and make these derivations first-class objects,
known as \emph{type precision derivations}.
Specifically, for every $A \ltdyn A'$, we have a derivation $c : A \ltdyn A'$ that is constructed
using the rules above. For instance, there is a derivation $\dyn : \dyn \ltdyn \dyn$, and a derivation
$\nat : \nat \ltdyn \nat$, and if $c_i : A_i \ltdyn A_i$ and $c_o : A_o \ltdyn A'_o$, then
there is a derivation $c_i \ra c_o : (A_i \ra A_o) \ltdyn (A'_i \ra A'_o)$. Likewise for
the remaining rules. The benefit to making these derivations explicit in the syntax is that we
can perform induction over them.
Note also that for any type $A$, we use $A$ to denote the reflexivity derivation that $A \ltdyn A$,
i.e., $A : A \ltdyn A$.
Finally, observe that for type precision derivations $c : A \ltdyn A'$ and $c' : A' \ltdyn A''$, we
can define their composition $c \relcomp c' : A \ltdyn A''$.
This notion will be used below in the statement of transitivity of the term precision relation.

\subsection{Removing Casts as Primitives}

% We now observe that all casts, except those between $\nat$ and $\dyn$
% and between $\dyn \ra \dyn$ and $\dyn$, are admissible, in the sense that
% we can start from $\extlcm$, remove casts except the aforementioned ones,
% and in the resulting language we will be able to derive the other casts.

We now observe that all casts, except those between $\nat$ and $\dyn$
and between $\dyn \ra \dyn$ and $\dyn$, are admissible.
That is, consider a new language ($\extlcprime$) in which
instead of having arbitrary casts, we have injections from $\nat$ and
$\dyn \ra \dyn$ into $\dyn$, and a case inspection on $\dyn$.
We claim that in $\extlcprime$, all of the casts present in $\extlc$ are derivable.
It will suffice to verify that casts for function type are derivable.
This holds because function casts are constructed inductively from the casts
of their domain and codomain. The base case is one of the casts involving $\nat$
or $\dyn \ra \dyn$ which are present in $\extlcprime$ as injections and case inspections.


The resulting calculus $\extlcprime$ now lacks arbitrary casts as a primitive notion:

%%%%%%%%%%%%%%%%%%%%%%%%%%%%%%%%%%%%%%%%%%%%%%
% TODO update

\begin{align*}
  &\text{Types } A := \nat \alt \dyn \alt (A \ra A') \\
  &\text{Contexts } \Gamma := \cdot \alt (\Gamma, x : A) \\
  &\text{Values } V :=  \zro \alt \suc\, V \alt \lda{x}{M} \alt \injnat V \alt \injarr V \\ 
  &\text{Producers } M, N := \err_B \alt \ret {V} \alt \bind{x}{M}{N}
    \alt V_f\, V_x \alt
    \\ & \quad\quad \casenat{V}{M_{no}}{n}{M_{yes}} 
    \alt \casearr{V}{M_{no}}{f}{M_{yes}}
\end{align*}


% New rules
Figure \ref{fig:extlc-minus-minus-typing} shows the new typing rules,
and Figure \ref{fig:extlc-minus-minus-eqns} shows the equational rules
for case-nat (the rules for case-arrow are analogous).

\begin{figure}
  \begin{mathpar}
      % inj-nat
      \inferrule*
      {\hasty \Gamma M \nat}
      {\hasty \Gamma {\injnat M} \dyn}

      % inj-arr 
      \inferrule*
      {\hasty \Gamma M (\dyn \ra \dyn)}
      {\hasty \Gamma {\injarr M} \dyn}

      % Case dyn
      \inferrule*
      {\hasty{\Delta|_V}{V}{\dyn} \and
        \hasty{\Delta , x : \nat }{M_{nat}}{B} \and 
        \hasty{\Delta , x : (\dyn \ra \dyn) }{M_{fun}}{B}
      }
      {\hasty {\Delta} {\casedyn{V}{n}{M_{nat}}{f}{M_{fun}}} {B}}
  \end{mathpar}
  \caption{New typing rules for $\extlcmm$.}
  \label{fig:extlc-minus-minus-typing}
\end{figure}


\begin{figure}
  \begin{mathpar}
     % Case-dyn Beta
     \inferrule*
     {\hasty \Gamma V \nat}
     {\casedyn {\injnat {V}} {n} {M_{nat}} {f} {M_{fun}} = M_{nat}[V/n]}

     \inferrule*
     {\hasty \Gamma V {\dyn \ra \dyn} }
     {\casedyn {\injarr {V}} {n} {M_{nat}} {f} {M_{fun}} = M_{fun}[V/f]}

     % Case-dyn Eta
     \inferrule*
     {}
     {\Gamma , x :\, \dyn \vdash M = \casedyn{x}{n}{M[(\injnat{n}) / x]}{f}{M[(\injarr{f}) / x]} }


  \end{mathpar}
  \caption{New equational rules for $\extlcprime$ (rules for case-arrow are analogous
           and hence are omitted).}
  \label{fig:extlc-minus-minus-eqns}
\end{figure}




\section{Term Semantics}\label{sec:term-semantics}

\subsection{Domain-Theoretic Constructions}\label{sec:domain-theory}

In this section, we discuss the fundamental objects of the model into which we will embed
the gradual lambda calculus.
It is important to remember that the constructions in this section are entirely
independent of the syntax described in the previous section; the notions defined 
here exist in their own right as purely mathematical constructs.
In Section \ref{sec:term-interpretation}, we will link the syntax and semantics
via a semantic interpretation function.


\subsection{Modeling the Dynamic Type}

When modeling the dynamic type $\dyn$, we need a semantic object $D$ that satisfies the
isomorphism

\[ D \cong \Nat + (D \to (D + 1)). \]

where the $D + 1$ represents the fact that in the function case, the function may return an error.
Unfortunately, this equation does not have inductive or coinductive solutions. The usual way of
dealing with such equations is via domain theory, by which we can obtain an exact solution.
However, the heavy machinery of domain theory can be difficult for language designers to learn
and apply in the mechanized setting.
Instead, we will leverage the tools of guarded type theory, considering instead the following
similar-looking equation:

\[ D \cong \Nat + \later (D \to (D + 1)). \]

Since the negative occurrence of $D$ is guarded under a later, this equation has a (guarded) solution.
Specifically, we consider the following function $f$ of type
$\later \type \to \type$:

\[ \lambda (D' : \later \type) . \Nat + \later_t (D'_t \to (D'_t + 1)). \]

(Recall that the tick $t : \tick$ is evidence that time has passed, and since
$D'$ has type $\later \type$, i.e. $\tick \to \type$, then $D'_t$ has type $\type$.)

Then we define 

\[ D = \fix f. \]

% TODO explain better
As it turns out, this definition is not quite correct; let us try to understand why.
The price we pay for such a simple solution to the above equation is that we have
introduced a notion of ``time'', i.e., we must now consider the stepping behavior of terms.
% Therefore, the semantics of terms will need to allow for terms that potentially
% do not terminate. This is accomplished using an extension of the guarded lift monad,
% which we describe in the next section.
Thus, in the equation for the semantics of $\dyn$, the result of the function should be
a computation that may return a value, error, \emph{or} take an observable step.
We model such computations using an extension of the so-called guarded lift monad
\cite{mogelberg-paviotti2016} which we describe in the next section.
We will then discuss the correct definition of the semantics of the dynamic type.

\subsubsection{The Lift + Error Monad}

% TODO ensure the previous section flows into this one
When thinking about how to model gradually-typed programs where we track the steps they take,
we should consider their possible behaviors. On the one hand, we have \emph{failure}: a program may fail
at run-time because of a type error. In addition to this, a program may ``think'',
i.e., take a step of computation. If a program thinks forever, then it never returns a value,
so we can think of the idea of thinking as a way of intensionally modelling \emph{partiality}.

With this in mind, we can describe a semantic object that models these behaviors: a monad
for embedding computations that has cases for failure and ``thinking''.
Previous work has studied such a construct in the setting of the latter, called the lift
monad \cite{mogelberg-paviotti2016}; here, we augment it with the additional effect of failure.

For a type $A$, we define the \emph{lift monad with failure} $\li A$, which we will just call
the \emph{lift monad}, as the following datatype:

\begin{align*}
  \li A &:= \\
  &\eta \colon A \to \li A \\
  &\mho \colon \li A \\
  &\theta \colon \later (\li A) \to \li A
\end{align*}

Unless otherwise mentioned, all constructs involving $\later$ or $\fix$
are understood to be with respect to a fixed clock $k$. So for the above, we really have for each
clock $k$ a type $\li^k A$ with respect to that clock.

Formally, the lift monad $\li A$ is defined as the solution to the guarded recursive type equation

\[ \li A \cong A + 1 + \later \li A. \]

An element of $\li A$ should be viewed as a computation that can either (1) return a value (via $\eta$),
(2) raise an error and stop (via $\mho$), or (3) think for a step (via $\theta$).
%
Notice there is a computation $\fix \theta$ of type $\li A$. This represents a computation
that runs forever and never returns a value.

Since we claimed that $\li A$ is a monad, we need to define the monadic operations
and show that they respect the monadic laws. The return is just $\eta$, and extend
is defined via guarded recursion by cases on the input.
% It is instructive to give at least one example of a use of guarded recursion, so
% we show below how to define extend:
% TODO
%
%
Verifying that the monadic laws hold uses \lob-induction and is straightforward.

%\subsubsection{Model-Theoretic Description}
%We can describe the lift monad in the topos of trees model as follows.

\subsubsection{Revisiting the Dynamic Type}
Now we can state the correct definition of the semantics for the dynamic type.
The set $D$ is defined to be the solution to the guarded equation

\[ D \cong \Nat + \later (D \to \textcolor{red}{\li} D). \]


\subsection{Interpretation}\label{sec:term-interpretation}

We can now give a semantics to the gradual lambda calculus we defined
in Section \ref{sec:step-sensitive-lc}.
%
Much of the semantics is similar to a normal call-by-value denotational semantics;
we highlight the differences.
We will interpret types as sets, and terms as functions.
Contexts $\Gamma = x_1 \colon A_1, \dots, x_n \colon A_n$
will be interpreted as the product $\sem{A_1} \times \cdots \times \sem{A_n}$.


The semantics of the dynamic type $\dyn$ is the set $\Dyn$ introduced in Section
\ref{sec:predomains}.
%
The interpretation of a value $\vhasty {\Gamma} V A$ will be a monotone function from
$\sem{\Gamma}$ to $\sem{A}$. Likewise, a term $\phasty {\Gamma} M {{A}}$ will be interpreted
as a monotone function from $\sem{\Gamma}$ to $\li \sem{A}$.

Recall that $\Dyn$ is isomorphic to $\Nat\, + \later (\Dyn \to \li \Dyn)$.
Thus, the semantics of $\injnat{\cdot}$ is simply $\inl$ and the semantics
of $\injarr{\cdot}$ is simply $\inr \circ \next$.
The semantics of case inspection on dyn performs a case analysis on the sum.

The interpretation of $\lda{x}{M}$ works as follows. Recall by the typing rule for
lambda that $\phasty {\Gamma, x : A_i} M {{A_o}}$, so the interpretation of $M$
has type $(\sem{\Gamma} \times \sem{A_i})$ to $\li \sem{A_o}$.
The interpretation of lambda is thus a function (in the ambient type theory) that takes
a value $a$ representing the argument and applies it (along with $\gamma$) as argument to
the interpretation of $M$.
%
The interpretation of bind and of application both make use the monadic extend function on $\li A$.
%
The interpretation of case-nat and case-arrow is simply a case inspection on the
interpretation of the scrutinee, which has type $\Dyn$.


\vspace{2ex}


\noindent Types:
\begin{align*}
  \sem{\nat} &= \Nat \\
  \sem{\dyn} &= \Dyn \\
  \sem{A \ra A'} &= \sem{A} \To \li \sem{A'} \\
\end{align*}

% Contexts:

% TODO check these, especially the semantics of bind, case-nat, and case-arr
% with respect to their context argument
\noindent Values and terms:
\begin{align*}
  \sem{\zro}         &= \lambda \gamma . 0 \\
  \sem{\suc\, V}     &= \lambda \gamma . (\sem{V}\, \gamma) + 1 \\
  \sem{x \in \Gamma} &= \lambda \gamma . \gamma(x) \\
  \sem{\lda{x}{M}}   &= \lambda \gamma . \lambda a . \sem{M}\, (*,\, (\gamma , a))  \\
  \sem{\injnat{V_n}} &= \lambda \gamma . \inl\, (\sem{V_n}\, \gamma) \\
  \sem{\injarr{V_f}} &= \lambda \gamma . \inr\, (\sem{V_f}\, \gamma) \\[2ex]
  \sem{\nxt\, V}     &= \lambda \gamma . \nxt (\sem{V}\, \gamma) \\
  \sem{\theta}       &= \lambda \gamma . \theta \\
%
  \sem{\err_B}         &= \lambda \delta . \mho \\
  \sem{\ret\, V}       &= \lambda \gamma . \eta\, \sem{V} \\
  \sem{\bind{x}{M}{N}} &= \lambda \delta . \ext {(\lambda x . \sem{N}\, (\delta, x))} {\sem{M}\, \delta} \\
  \sem{V_f\, V_x}      &= \lambda \gamma . {({(\sem{V_f}\, \gamma)} \, {(\sem{V_x}\, \gamma)})} \\
  \sem{\casedyn{V}{n}{M_{nat}}{\tilde{f}}{M_{fun}}} &=
    \lambda \delta . \text{case $(\sem{V}\, \delta)$ of} \\ 
    &\quad\quad\quad\quad \alt \inl(n) \to \sem{M_{nat}}(n) \\
    &\quad\quad\quad\quad \alt \inr(\tilde{f}) \to \sem{M_{fun}}(\tilde{f})
\end{align*}

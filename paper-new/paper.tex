\documentclass[acmsmall,screen]{acmart}
\let\Bbbk\relax

\usepackage{quiver}
\usepackage{mathpartir}
% \usepackage{tikz-cd}
\usepackage{enumitem}
\usepackage{wrapfig}
\usepackage{fancyvrb}
\usepackage{comment}



%% Rights management information.  This information is sent to you
%% when you complete the rights form.  These commands have SAMPLE
%% values in them; it is your responsibility as an author to replace
%% the commands and values with those provided to you when you
%% complete the rights form.
\setcopyright{acmcopyright}
\copyrightyear{2018}
\acmYear{2018}
\acmDOI{10.1145/1122445.1122456}

%% These commands are for a PROCEEDINGS abstract or paper.
\acmConference[Woodstock '18]{Woodstock '18: ACM Symposium on Neural
  Gaze Detection}{June 03--05, 2018}{Woodstock, NY}
\acmBooktitle{Woodstock '18: ACM Symposium on Neural Gaze Detection,
  June 03--05, 2018, Woodstock, NY}
\acmPrice{15.00}
\acmISBN{978-1-4503-XXXX-X/18/06}

\newcommand{\To}{\Rightarrow}
\newcommand{\inl}{\mathsf{inl}}
\newcommand{\inr}{\mathsf{inr}}
\newcommand{\alt}{\mathrel{\bf \,\mid\,}}


\newcommand{\extlc}{\text{Ext-}\lambda}
\newcommand{\extlcm}{\text{Ext-}\lambda^{-\text{trans}}}
\newcommand{\extlcmm}{\text{Ext-}\lambda^{-\text{trans}-\text{cast}}}
\newcommand{\extlcprime}{\text{Ext-}\lambda'}
\newcommand{\intlc}{\text{Int-}\lambda}
\newcommand{\intlcbisim}{\text{Int$_\approx$-}\lambda}
\newcommand{\erase}[1]{\lfloor {#1} \rfloor}


\newcommand{\uarrowl}{\mathrel{\rotatebox[origin=c]{-30}{$\leftarrowtail$}}}
\newcommand{\uarrowr}{\mathrel{\rotatebox[origin=c]{60}{$\leftarrowtail$}}}
\newcommand{\darrowl}{\mathrel{\rotatebox[origin=c]{30}{$\twoheadleftarrow$}}}
\newcommand{\darrowr}{\mathrel{\rotatebox[origin=c]{120}{$\twoheadleftarrow$}}}
\newcommand{\vuarrow}{\mathrel{\rotatebox[origin=c]{-90}{$\leftarrowtail$}}}
\newcommand{\vdarrow}{\mathrel{\rotatebox[origin=c]{90}{$\twoheadleftarrow$}}}

% Types, terms, and precision 

\newcommand{\dyn}{?}
\newcommand{\nat}{\text{Nat}}
\newcommand{\bool}{\text{Bool}}
\newcommand{\ra}{\rightharpoonup}
\newcommand{\Ret}[1]{\mathsf{Ret\,}{#1}}
\newcommand{\hole}[1]{\bullet \colon {#1}}
\newcommand{\dyntodyn}{\dyn \ra\, \dyn}


\newcommand{\up}[2]{\langle{#2}\uarrowl{#1}\rangle}
\newcommand{\dn}[2]{\langle{#1}\darrowl{#2}\rangle}

\newcommand{\upc}[2]{\text{up}\,{#1}\,{#2}}
\newcommand{\dnc}[2]{\text{dn}\,{#1}\,{#2}}


\newcommand{\ret}{\mathsf{ret}}
\newcommand{\err}{\mho}
\newcommand{\zro}{\textsf{zro}}
\newcommand{\suc}{\textsf{suc}}
\newcommand{\lda}[2]{\lambda {#1} . {#2}}
\newcommand{\injarr}[1]{\textsf{Inj}_\ra ({#1})}
\newcommand{\injnat}[1]{\textsf{Inj}_\text{nat} ({#1})}
\newcommand{\casenat}[4]{\text{Case}_\text{nat} ({#1}) \{ \text{no} \to {#2} \alt \text{nat}({#3}) \to {#4} \}}
\newcommand{\casearr}[4]{\text{Case}_\ra ({#1}) \{ \text{no} \to {#2} \alt \text{fun}({#3}) \to {#4} \}}
\newcommand{\casedyn}[5]{\text{Case}_\Dyn ({#1}) \{ \text{nat}({#2}) \to {#3} \alt \text{fun}({#4}) \to {#5} \}}
\newcommand{\bind}[3]{\text{var } {#1} = {#2} \text{ in } {#3}}
\newcommand{\matchnat}[4]{\text{match } {#1} \text{ with } \{ \textsf {zero} \Rightarrow {#2} \alt \textsf{ suc } {#3} \Rightarrow {#4} \}}

\newcommand{\refl}{\text{refl}}

\newcommand{\Lift}{\text{Lift}}

%\newcommand{\rel}{\circ\hspace{-4px}-\hspace{-4px}\bullet}
\newcommand{\rel}{\mathrel{\circ\mkern-6mu-\mkern-6mu\bullet}}
\newcommand{\ltdyn}{\sqsubseteq}
\newcommand{\gtdyn}{\sqsupseteq}
\newcommand{\equidyn}{\mathrel{\gtdyn\ltdyn}}
\newcommand{\gamlt}{\Gamma^\ltdyn}
\newcommand{\deltalt}{\Delta^\ltdyn}
\newcommand{\relcomp}{\odot}

\newcommand{\hasty}[3]{{#1} \vdash {#2} \colon {#3}}
\newcommand{\vhasty}[3]{{#1} \vdash^v {#2} \colon {#3}}
\newcommand{\phasty}[3]{{#1} \vdash^p {#2} \colon {#3}}
\newcommand{\etmprec}[4]{{#1} \vdash {#2} \ltdyn_e {#3} \colon {#4}}
\newcommand{\itmprec}[4]{{#1} \vdash {#2} \ltdyn_i {#3} \colon {#4}}
\newcommand{\etmequidyn}[4]{{#1} \vdash {#2} \equidyn_e {#3} \colon {#4}}
\newcommand{\itmequidyn}[4]{{#1} \vdash {#2} \equidyn_i {#3} \colon {#4}}
\newcommand{\synbisim}{\approx_{\text{syn}}}

\newcommand{\Dwn}{\Downarrow}
\newcommand{\qte}[1]{\text{quote}({#1})}

\newcommand{\elab}[1]{\text{Elab}({#1})}

% Perturbations
\newcommand{\pertp}{\text{Pert}^\text{P}}
\newcommand{\perte}{\text{Pert}^\text{E}}
\newcommand{\pertdyn}[2]{\text{pert-dyn}({#1}, {#2})}
\newcommand{\delaypert}[1]{\text{delay-pert}({#1})}

\newcommand{\pertc}{\text{Pert}_{\text{C}}}
\newcommand{\pertv}{\text{Pert}_{\text{V}}}


% SGDT and Intensional Stuff

\newcommand{\later}{\vartriangleright}
\newcommand{\type}{\texttt{Type}}
\newcommand{\lob}{\text{L\"{o}b}}
\newcommand{\tick}{\mathsf{tick}}
\newcommand{\nxt}{\mathsf{next}}
\newcommand{\fix}{\mathsf{fix}}
\newcommand{\kpa}{\kappa}

% Model-related stuff
\newcommand{\calC}{\mathcal{C}}
\newcommand{\Set}{\mathsf{Set}}
\newcommand{\Yo}{\mathsf{Yo}}
\newcommand{\Hom}{\mathsf{Hom}}
\newcommand{\calS}{\mathcal{S}}
\newcommand{\gfix}{\texttt{gfix}}
\newcommand{\calU}{\mathcal{U}}
\newcommand{\laterhat}{\widehat{\later}}
\newcommand{\El}{\mathsf{El}}
\newcommand{\Clock}{\mathsf{Clock}}

\newcommand{\Machine}[1]{\mathsf{Machine}\, {#1}}


% Predomains and EP pairs
\newcommand{\Nat}{\mathsf{Nat}}
\newcommand{\Dyn}{\mathsf{Dyn}}
\newcommand{\ty}[1]{\langle {#1} \rangle}
\newcommand{\li}{L_\mho}
\newcommand{\liclk}[1]{L_\mho [{#1}]}

\newcommand{\ext}[2]{\text{ext}\,{#1}\,{#2}}
\newcommand{\map}[2]{\text{map}\,{#1}\,{#2}}

\newcommand{\ltls}{\lesssim}
\newcommand{\bisim}{\approx}
\newcommand{\semlt}{\le}
\newcommand{\semltbad}{\lesssim}

%\newcommand{\injarr}{\textsf{Inj}_\to}
%\newcommand{\injnat}{\textsf{Inj}_\mathbb{N}}

\newcommand{\id}{\mathsf{id}}
\newcommand{\ep}{\leadsto}

\newcommand{\emb}[2]{\mathsf{emb}_{#1}({#2})}
\newcommand{\proj}[2]{\mathsf{proj}_{#1}({#2})}

\newcommand{\monto}{\to_m}


% Notation for wait functions
\newcommand{\wre}{w_r^e}
\newcommand{\wle}{w_l^e}
\newcommand{\wrp}{w_r^p}
\newcommand{\wlp}{w_l^p}

\newcommand{\sem}[1]{\llbracket {#1} \rrbracket}
\newcommand{\semgl}[1]{{\sem{#1}}^\text{gl}}


% Denotational model
\newcommand{\ptb}{\text{ptb}}
\newcommand{\push}{\text{push}}
\newcommand{\pull}{\text{pull}}



\begin{document}

\title{Mechanized Denotational Semantics of Gradual Typing using Synthetic Guarded Domain Theory}
\author{Eric Giovannini}
\author{Max S. New}

\begin{abstract}
  Gradually typed programming languages, which allow for soundly
  mixing static and dynamically typed programming styles, present a
  strong challenge for metatheorists. Even the simplest sound
  gradually typed languages feature at least recursion and errors,
  with realistic languages featuring furthermore runtime allocation of
  memory locations and dynamic type tags. Further, the desired
  metatheoretic properties of gradually typed languages have become
  increasingly sophisticated: validity of typed based equational
  reasoning as well as the relational property known as the gradual
  guarantee or graduality. Many recent works have tackled verifying
  these properties, but the resulting mathematical developments are
  highly repetitive and tedious, with few reusable theorems persisting
  across different developments.

  In this work, we present a new denotational account of gradual
  typing semantics developed using guarded domain theory. Guarded
  domain theory combines the expressive power of step-indexed logical
  relations for modeling recursive features with the modularity and
  reusability of denotational semantics. Further, recent extensions to
  cubical Agda mean that synthetic guarded domain theory is readily
  mechanized in a proof assistant. We demonstrate the feasibility of
  this approach with a model of gradually typed lambda calculus and
  prove the validity of beta-eta equality and the graduality theorem
  for the denotational model. This model should provide the basis for
  a reusable mathematical theory of gradually typed program semantics.
%%     We develop a denotational semantics for a simple gradually typed language
%%     that is adequate and proves the graduality theorem.
%%     %
%%     The denotational semantics is constructed using \emph{synthetic
%%     guarded domain theory} working in a type theory with a later
%%     modality and clock quantification.
%%     %
%%     This provides a remarkably simple presentation of the semantics,
%%     where gradual types are interpreted as ordinary types in our ambient
%%     type theory equipped with an ordinary preorder structure to model
%%     the error ordering.
%%     %
%%     This avoids the complexities of classical domain-theoretic models
%%     (New and Licata) or logical relations models using explicit
%%     step-indexing (New and Ahmed).
%%     %
%%     In particular, we avoid a major technical complexity of New and
%%     Ahmed that requires two logical relations to prove the graduality
%%     theorem.
  
%%     By working synthetically we can treat the domains in which gradual
%%     types are interpreted as if they were ordinary sets. This allows us
%%     to give a ``na\"ive'' presentation of gradual typing where each
%%     gradual type is modeled as a well-behaved subset of the universal
%%     domain used to model the dynamic type, and type precision is modeled
%%     as simply a subset relation.
%%     %
  \end{abstract}

\maketitle

% Outline

% 1. Intro: What do we want out of gradually typed languages and why
%    is it hard to prove? Explanation: Gradual Typing inherently
%    involves recursive types, multiple effects, relational properties.
%    We argue that the increasing complexity of the metatheory of
%    gradual typing makes it a good candidate for 

% 2. Extensional Dream Semantics: double categorical

% 3. The Problem with Step-indexing

% 4. A Compromise

% 5. Formalization in Guarded Cubical Agda


\section{Introduction}
  
% gradual typing, graduality
\subsection{Gradual Typing and Graduality}
In programming language design, there is a tension between \emph{static} typing
and \emph{dynamic} typing disciplines.
With static typing, the code is type-checked at compile time, while in dynamic typing,
the type checking is deferred to run-time. Both approaches have benefits: with static 
typing, the programmer is assured that if the program passes the type-checker, their
program is free of type errors, and moreover, soundness of the equational theory implies
that program optimizations are valid.
Meanwhile, dynamic typing allows the programmer to rapidly prototype
their application code without needing to commit to fixed type signatures for their
functions. Most languages choose between static or dynamic typing and as a result, programmers that initially write their code in a dynamically typed language
in order to benefit from faster prototyping and development time need to rewrite
some or all of their codebase in a static language if they would like to receive the benefits of static
typing once their codebase has matured.

\emph{Gradually-typed languages} \cite{siek-taha06, tobin-hochstadt06} seek to resolve this tension
by allowing for both static and dynamic typing disciplines to be used in the same codebase,
and by supporting smooth interoperability between statically-typed and dynamically-typed code.
This flexibility allows programmers to begin their projects in a dynamic style and
enjoy the benefits of dynamic typing related to rapid prototyping and easy modification
while the codebase ``solidifies''. Over time, as parts of the code become more mature
and the programmer is more certain of what the types should be, the code can be
\emph{gradually} migrated to a statically typed style without needing to
rewrite the project in a completely different language.

%Gradually-typed languages should satisfy two intuitive properties.
% The following two properties have been identified as useful for gradually typed languages.

In order for this to work as expected, gradually-typed languages should allow for
different parts of the codebase to be in different places along the spectrum from
dynamic to static, and allow for those different parts to interact with one another.
Moreover, gradually-typed languages should support the smooth migration from
dynamic typing to static typing, in that the programmer can initially leave off the
typing annotations and provide them later without altering the meaning of the program.
% Sound gradual typing
Furthermore, the parts of the program that are written in a dynamic
style should soundly interoperate with the parts that are written in a
static style.  That is, the interaction between the static and dynamic
components of the codebase should preserve, to the extent possible,
the guarantees made by the static types.  In particular, while
statically-typed code can error at runtime in a gradually-typed
language, such an error can always be traced back to a
dynamically-typed term that violated the typing contract imposed by
statically typed code. Further, static type assertions are sound in
the static portion, and should enable type-based reasoning and
optimization.

% Moreover, gradually-typed languages should allow for
% different parts of the codebase to be in different places along the spectrum from
% dynamic to static, and allow for those different parts to interact with one another.
% In a \emph{sound} gradually-typed language,
% this interaction should respect the guarantees made by the static types.

% Graduality property
One of the fundamental theorems for gradually typed languages is
\emph{graduality}, also known as the \emph{dynamic gradual guarantee}
\emph{dynamic gradual guarantee}, originally defined by Siek,
Vitousek, Cimini, and Boyland \cite{siek_et_al:LIPIcs:2015:5031,
  new-ahmed2018}.
%
Informally, graduality says that going from a dynamic to static style should only allow for the introduction of static or dynamic type errors, and not otherwise change the behavior of the program.
%
This is a way to capture programmer intuition that increasing type
precision corresponds to a generalized form of runtime assertions in
that there are no observable behavioral changes up to the point of the
first dynamic type error\footnote{once a dynamic type error is raised,
in languages where the type error can be caught, program behavior may
then further diverge, but this is typically not modeled in gradual
calculi.}.
%
Fundamentally, this property comes down to the behavior of
\emph{runtime type casts}\max{TODO: introduce casts more thoroughly as
  they are important}.

Additionally, gradually typed languages should offer some of the
benefits of static typing. While classical type soundness, that
well-typed programs are free from runtime errors, is not compatible
with runtime type errors, we can instead focus on \emph{type-based
reasoning}. For instance, while dynamically typed $\lambda$ calculi
only satisfy $\beta$ equality for their type formers, statically typed
$\lambda$ calculi additionally satisfy type-dependent $\eta$
properties that ensure that functions are determined by their behavior
under application and that pattern matching on data types
is safe and exhaustive.

More concretely, consider a gradually typed language whose only
effects are gradual type errors and divergence. Then if we fix a
result type of natural numbers, a whole program semantics is a partial
function from closed programs to either natural numbers or errors:
\[ -\Downarrow : \{M \,|\, \cdot \vdash M : \nat \} \rightharpoonup \mathbb{N} \cup {\mho} \]
where $\mho$ is notation for a runtime type error. We write $M
\Downarrow n$ and $M\Downarrow \mho$ to mean this semantics is defined
as a number or error, and $M\Uparrow$ to mean the semantics is
undefined, representing divergence.
%
A well-behaved semantics should then satisfy several properties. First, it
should be adequate: natural number constants should step to
themselves. Second it should validate type based reasoning. To
formalize type based reasoning we give a typed equational theory for
terms of the language $M \cong N$ for when two terms should be
considered equivalent. Then we want to verify that the big step
semantics respects this equational theory: if closed programs $M \cong
N$ are equivalent in the equational theory then they have the same
semantics, $M \Downarrow n \iff N \downarrow n$ and $M\Uparrow \iff N
\Uparrow$ and $M \Downarrow \mho \iff N \Downarrow \mho$.
%
Lastly, the graduality property is defined by giving an
\emph{inequational} theory called term precision, where $M \ltdyn N$
roughly means that $M$ and $N$ have the same type erasure and $M$ has
at each point in the program a more precise/static type than $N$.
%
Then, the graduality property states that if $M \ltdyn N$ are whole
programs then $M$ must either have the same behavior as $N$ or error:
Either $M\Downarrow \mho$ or $M \Downarrow n $ and $N \Downarrow n$ or
$M \Uparrow $ and $N \Uparrow$\footnote{we use a slightly more complex
definition of this relation in our technical development below that is
classically equivalent but constructively weaker}.

\subsection{Denotational Semantics in Guarded Domain Theory}

Our goal in this work is to provide an \emph{expressive},
\emph{reusable}, \emph{compositional} semantic framework for defining
such well-behaved semantics of programs.
%
Our approach to achieving this goal is to provide a compositional
\emph{denotational semantics}, mapping types to a kind of semantic
domain, terms to functions and relations such as term precision to
proofs of semantic relations between the denoted functions.
%
Since the denotational constructions are all syntax-independent, the
constructions we provide may be reused for similar languages. Since it
is compositional, components can be mixed and matched depending on
what source language features are present.
%
Providing this semantics for gradual typing is inherently complicated
in that it involves: (1) recursion and recursive types through the
presence of dynamic types, (2) effects in the form of divergence and
errors (3) relational models in capturing the graduality
property. Recursion and recursive types must be handled using some
flavor of domain theory. Effects can be modeled using monads in the
style of Moggi, or adjunctions in the style of Levy. Relational
properties and their verification lead naturally to the use of
reflexive graph categories or double categories.

The only prior denotational semantics for gradual typing was given by
New and Licata and is based on \emph{domain theory}
\cite{newlicata2019}. The fundamental idea is to equip $\omega$-CPOs
with an additional ``error ordering'' $\ltdyn$ which models the
graduality ordering, and for casts to arise from
\emph{embedding-projection pairs}. Then the graduality property
follows as long as all language constructs can be interpreted using
constructions that are monotone with respect to the error ordering.
%
This framework has the benefit of being compositional, and was
expressive enough to be extended to model dependently typed gradual
typing \cite{dependentgradualtyping}.
%
However, an approach based on classical domain theory has fundamental
limitations: domain theory is incapable of modeling certain ``highly
recursive'' features of programming languages such as dynamic type tag
generation and higher-order references, which are commonplace in
real-world gradually typed systems as well as gradual calculi
\cite{examples-of-gradual-stuff}. Our goal in this work is to develop
an approach that will eventually scale up to these advanced features,
though for a first work we focus on adapting their results for a more
basic semantics.

The main denotational alternative to classical domain theory that can
successfully model these advanced features is \emph{guarded domain
theory}, which we adopt in this work. While classical domain theory is
based on modeling types as ordered sets with certain joins, guarded
domain theory is based on an entirely different foundations, sometimes
(ultra)metric spaces but more commonly as ``step-indexed sets'', i.e.,
objects in the ``topos of trees'', i.e., presheaves on the poset of
natural numbers. We think of these as modeling a kind of ``sequence of
approximations'' to the domain being modeled.
%
Key to guarded domain theory is that there is a ``later'' operator
$\triangleright$ on types. Thinking of types as a sequence of
approximations, the later operator delays the approximation by one
step. Then the crucial axiom of guarded domain theory is that any
guarded equation $X \cong F(\triangleright X)$ has a unique
solution. This allows guarded domain theory to model essentially
\emph{any} recursive concept, with the caveat that it is guarded by a
later.
%
This caveat is the main source of difficulty in adapting a semantic
approach based on classical domain theory to guarded domain theory:
classical domain theory has limitations in what it can model, but it
provides \emph{exact} solutions to domain equations when it
applies. When adapting the New-Licata approach to guarded domain
theory, this means we must work with an \emph{intensional} semantics,
one where unfolding the dynamic type takes an observable runtime step.

\subsection{Synthetic Guarded Domain Theory}

While guarded domain theory can be presented analytically using
ultrametric spaces or the topos of trees, in practice it is
considerably simpler to work \emph{synthetically} by working in a
non-standard foundation such as guarded type theory where the later
modality is simply an operation on our basic notion of type, and we
take as an axiom that guarded domain equations have a (necessarily
unique) solution. This has the added benefit that providing a
denotational semantics of gradually typed terms is as simple as an
semantics of an effectful language using a monad on the category of
sets. Not only does this make on-paper reasoning about guarded domain
theory easier, it also enables a simpler avenue to verification in a
proof assistant. Whereas formalizing analytic guarded domain theory
would require significant theory of presheaves and making sure that
all constructions are functors on categories of presheaves,
formalizing synthetic guarded domain theory can be done by directly
adding the later modality and the guarded fixed point property
axiomatically.

\max{TODO: more here}

\subsection{Contributions}

The main contribution of this work is a compositional denotational
semantics for gradually typed languages that validates $\beta\eta$
equality and satisfies a graduality theorem. A great deal of the work
has further been verified in guarded cubical Agda, demonstrating that
the semantics is readily mechanizable.

\begin{enumerate}
\item First, we give a simple concrete term semantics where we show
  how to model the dynamic type as a solution to a guarded domain equation

\item Next, we identify where prior work on classical domain theoretic
  semantics of gradual typing breaks down when using guarded semantics
  of recursive types.

\item We develop a key new concept of \emph{syntactic perturbations},
  which allow us to recover enough extensional reasoning to model the
  graduality property compositionally

\item We combine this insight together with an abstract categorical
  model of gradual typing using reflexive graph categories and
  call-by-push-value to give a compositional construction of our
  denotational model.
\item We prove that the resulting denotational model provides a
  well-behaved semantics as defined above by proving \emph{adequacy},
  respect for an equational theory and the graduality property.
\end{enumerate}

The paper is laid out as follows:
\begin{enumerate}
\item In Section\ref{sec:syntax} we fix our input language, a fairly
  typical gradually typed cast calculus.
\item In Section\ref{sec:concrete-terms} we develop a first
  denotational semantics in synthetic guarded domain theory that
  satisfies adequacy and respects the equational theory, but does not
  validate graduality. We use this to introduce some of our main
  technical tools: modeling recursive types in guarded type theory and
  modeling effects using call-by-push-value.
\item In Section\ref{sec:relational-issues} we show where the
  New-Licata classical domain theoretic approach fails to adapt
  cleanly to the guarded setting and explore the difficulties of
  proving graduality in an intensional model.
\item In Section\ref{sec:relational-semantics} we
\item In Section\ref{sec:discussion} we discuss prior work on proving
  graduality, our partial mechanization of these results and discuss
  future directions for denotational semantics of gradual typing.
\end{enumerate}

\subsection{Limitations of Prior Work}

We give an overview of current approaches to proving graduality of
languages and why they do not meet our criteria of a reusable semantic
framework.

\subsubsection{From Static to Gradual}

Current approaches to constructing languages that satisfy the
graduality property include the methods of Abstracting Gradual Typing
\cite{garcia-clark-tanter2016} and the formal tools of the Gradualizer
\cite{cimini-siek2016}.  These allow the language developer to start
with a statically typed language and derive a gradually typed language
that satisfies the gradual guarantee. The main downside to these
approaches lies in their inflexibility: since the process in entirely
mechanical, the language designer must adhere to the predefined
framework.  Many gradually typed languages do not fit into either
framework, e.g., Typed Racket \cite{tobin-hochstadt06,
  tobin-hochstadt08} and the semantics produced is not always the
desired one.
%
Furthermore, while these frameworks do prove graduality of the
resulting languages, they do not show the correctness of the
equational theory, which is equally important to sound gradual typing.

%% For example, programmers often refactor their code, and in so doing they rely
%% implicitly on the validity of the laws in the equational theory.
%% Similarly, correctness of compiler optimizations rests on the validity of the
%% corresponding equations from the equational theory. It is therefore important
%% that the languages that claim to be gradually typed have provably correct
%% equational theories.

% The approaches are too inflexible... the fact that the resulting semantics are too lazy
% is a consequence of that inflexibility.
% The validity of the equational theory captures the programmer's intuitive thinking when they refactor their code

%The downside is that
%not all gradually typed languages can be derived from these frameworks, and moreover, in both
%approaches the semantics is derived from the static type system as opposed to the alternative
%in which the semantics determines the type checking. Without a clear semantic interpretation of type
%dynamism, it becomes difficult to extend these techniques to new language features such as polymorphism.


% [Eric] I moved the next two paragraphs from the technical background section
% to here in the intro.
\subsubsection{Double Categorical Semantics}

New and Licata \cite{new-licata18} developed an axiomatic account of
the graduality relation on a call-by-name cast calculus terms and
showed that the graduality proof could be modeled using semantics in
certain kinds of \emph{double categories}, categories internal to the
category of categories. A double category extends a category with a
second notion of morphism, often a notion of ``relation'' to be paired
with the notion of functional morphism, as well as a notion of
functional morphisms preserving relations. In gradual typing the
notion of relation models type precision and the squares model the
term precision relation. This approach was influenced by the semantics
of parametricity using reflexive graph categories
\cite{ma-reynolds,dunphythesis,reynoldsprogramme}: reflexive graph
categories are essentially double categories without a notion of
relational composition. In addition to capturing the notions of type
and term precision, the double categorical approach allows for a
\emph{universal property} for casts: upcasts are the \emph{universal}
way to turn a relation arrow into a function in a forward direction
and downcasts are the dual universal arrow.  Later, New, Licata and
Ahmed \cite{new-licata-ahmed2019} extended this axiomatic treatment from
call-by-name to call-by-value as well by giving an axiomatic theory of
type and term precision in call-by-push-value. This left implicit any
connection to a ``double call-by-push-value'', which we make explicit
in Section~\ref{sec:cbpv}.

With this notion of abstract categorical model in hand, denotational
semantics is then the work of constructing concrete models that
exhibit the categorical construction. New and Licata
\cite{new-licata18} present such a model using categories of
$\omega$-CPOs, and this model was extended by Lennon-Bertrand,
Maillard, Tabareau and Tanter to prove graduality of a gradual
dependently typed calculus $\textrm{CastCIC}^{\mathcal G}$. This
domain-theoretic approach meets our criteria of being a semantic
framework for proving graduality, but suffers from the limitations of
classical domain theory: the inability to model viciously
self-referential structures such as higher-order extensible state and
similar features such as runtime-extensible dynamic types. Since these
features are quite common in dynamically typed languages, we seek a
new denotational framework that can model these type system features.

The standard alternative to domain theory that scales to essentially
arbitrary self-referential definitions is \emph{step-indexing} or its
synthetic form of \emph{guarded recursion}. A series of works
\cite{new-ahmed2018, new-licata-ahmed2019, new-jamner-ahmed19}
developed step-indexed logical relations models of gradually typed
languages based on operational semantics. Unlike classical domain
theory, such step-indexed techniques are capable of modeling
higher-order store and runtime-extensible dynamic types
\cite{appelmcallester01,ahmed06,neis09,new-jamner-ahmed19}. However,
their proof developments are highly repetitive and technical, with
each development formulating a logical relation from first-principles
and proving many of the same tedious lemmas without reusable
mathematical abstractions. Our goal in the current work is to extract
these reusable mathematical principles from these explicit
step-indexed to make formalization of realistic gradual languages
tractible.


% Alternative phrasing:
\begin{comment}
\subsubsection{Embedding-Projection Pairs}

The series of works by New, Licata, and Ahmed \cite{new-licata18,new-ahmed2018,new-licata-ahmed2019}
develop an axiomatic account of gradual typing involving \emph{embedding-projection pairs}.
This allows for a particularly elegant formulation of the gradual guarantee.
Moreover, their axiomatic account of program equivalence allows for type-based reasoning about program equivalences.

In \cite{new-licata18}, New and Licata construct a denotational model of the axioms of
gradual typing using tools from classical domain theory. The benefit to this approach is that it is a reusable mathematical theory:
general semantic theorems about gradual domains can be developed independent of any particular syntax and then reused in many different denotational models.
Unfortunately, however, it is unclear how to extend these domain-theoretic techniques to incorporate more advanced language features such as higher-order store, a standard feature of realistic gradually typed languages such as Typed
Racket.
Thus, if we want a reusable mathematical theory of gradual typing that can scale to realistic programming languages, we need to look to different techniques.

In \cite{new-ahmed2018,new-licata-ahmed2019}, a more traditional operational semantics for gradual typing is derived from the axioms.
% The axioms are then proven to be sound with respect to this operational semantics by constructing a logical relation.
A logical relations model is constructed and used to prove that the axioms are sound with respect to the operational semantics.
The downside of this approach is that each new language requires a different logical relation
to prove graduality, even though the logical relations for different languages end up sharing many commonalities.
Furthermore, the logical relations tend to be quite complicated due to a technical requirement known as \emph{step-indexing},
where the stepping behavior of terms must be accounted for in the logical relation.
As a result, developments using this approach tend to require vast effort, with the
corresponding technical reports having 50+ pages of proofs.

% In this approach, a logical relation is constructed and used to show that the equational theory
% is sound with respect to the operational semantics.

% Additionally, while the axioms of gradual type theory are compositional at a ``global'' level,
% they do not compose in the step-indexed setting. One of the main goals of the present work
% is to formulate a composable theory of gradual typing in a setting where the stepping behavior
% is tracked.
\end{comment}

An alternative approach, which we investigate in this paper, is provided by
\emph{synthetic guarded domain
theory}\cite{birkedal-mogelberg-schwinghammer-stovring2011}, abbreviated as
SGDT. The techniques of SGDT allow us to internalize the step-indexed reasoning
normally required in logical relations proofs of graduality, ultimately allowing
us to specify the logical relation in a manner that looks nearly identical to a
typical, non-step-indexed logical relation. In fact, guarded domain theory goes
further, allowing us to define step-indexed \emph{denotational semantics} not
just step-indexed relations, just as easily as constructing an ordinary
set-theoretic semantics.

In this paper, we develop an adequate denotational semantics that satisfies
graduality and soundness of the equational theory of cast calculi using the
techniques of SGDT.  

\eric{This is out of date: the mechanization is now an artifact of the current paper.}
Our longer-term goal is to mechanize these proofs in a reusable way in the
Guarded Cubical Agda \cite{veltri-vezzosi2020} proof assistant, thereby
providing a framework to more easily and conveniently prove that existing
languages satisfy graduality and have sound equational theories.

Moreover, the aim is for designers of new languages to utilize the framework to
facilitate the design of new provably-correct gradually-typed languages with
more complex features.

\subsection{Contributions}

The main contribution of this work is a categorical and denotational semantics
for gradually typed langauges that models not just the term language but the
graduality property as well.
\begin{enumerate}
\item First, we give a simple abstract categorical model of Gradual Type Theory
using CBPV double categories.
\item Next, we modify this semantics to develop reflexive graph- and double
  categorical models that abstract over the details of step-indexed models, and
  provide a method for constructing such models.
\item We instantiate the abstract construction to provide a concrete semantics
  in informal guarded type theory.
\item We prove that the resulting denotational model is \emph{adequate} for the
  graduality property: a closed term precision $M \ltdyn N : \nat$ has the
  expected semantics, that $M$ errors or $M$ and $N$ have the same extensional
  behavior.
\end{enumerate}

% TODO is it okay to refer to our cast calculus as *the* gradually-typed lambda
% calculus?
\eric{Where should we put the description of the big-step semantics and adequacy
of the semantics? I can put it in the outline of the remainder of the paper, or
I can postpone it until the subsequent section on the syntax of gradual typing.
Adequacy seems to be better introduced there since at that point we have
introduced the notion of term precision.}

The remainder of the paper is organized as follows. In Section
\ref{sec:technical-background}, we provide technical background on synthetic
guarded domain theory and some important details about the specific variant of
SGDT we employ as our ambient language.
%
In Section \ref{sec:GTLC}, we describe the syntactic theory of a standard cast
calculus for gradual typing, which we also refer to as the \emph{gradually-typed
lambda calculus}. We do this to establish a common syntax that will be
interpreted into the models we subsequently construct.
%
In Section \ref{sec:cbpv}, we develop \emph{abstract} categorical
semantics of gradually typed languages with the goal of organizing our
construction of denotational models of gradual typing.
%
In Sections \ref{sec:concrete-term-model} and
\ref{sec:towards-relational-model}, we set out to instantiate the abstract
categorical semantics by constructing a concrete model in a suitable double
category. First, in Section \ref{sec:concrete-term-model}, we give a concrete
model for the \emph{terms} of the gradually-typed lambda calculus. This involves
using the tools of SGDT to solve a domain equation to give the semantics of the
dynamic type. We explain how to \emph{globalize} this model to obtain a model in
the usual (non-guarded) dependent type theory. In the end, we obtain a
\emph{big-step semantics} for closed terms of type $\nat$:
%
\[ \sem{\cdot} \colon (\cdot \vdash M : \nat) \to (\mathbb{N} \to (\sem{\nat} + 1)) \] 
%
where a closed term $M$ of type $\nat$ denotes a \emph{function} from
$\mathbb{N}$ to $\sem{\nat} + 1$ which given input $i$ returns $\inl\, m$ if $M$
terminates with value $m$ in $i$ or fewer steps, and returns $\inr\, *$ if $M$
fails to terminate in $i$ or fewer steps. This is already a significant result,
as it serves to verify the $\beta$ and $\eta$ laws of the equational theory.
%
Then in Section \ref{sec:towards-relational-model}, we attempt to extend the
methodology of the previous section to give a denotational semantics that models
the relational theory of the gradually-typed lambda calculus in addition to the
terms. Interesting issues arise here arising from the \emph{intensional} nature
of SGDT, that is, the fact that \emph{steps are observable}. We then develop
novel solutions to address these issues. We do not complete the construction of
the concrete relational model in this section. Instead, we choose to reformulate
the definition of abstract categorical model to take into account the solutions
to the issues discussed here.

% Interesting issues arise here, and the novel solutions we employ to
% address these issues inform the definition of the abstract models introduced in
% the subsequent section.

With the lessons from the concrete setting at our disposal, in Section
\ref{sec:abstract-models}, we describe revised categorical models of gradual
typing. We actually introduce two separate notions of model: \emph{intensional},
where the stepping behavior of terms is observable, and \emph{extensional},
where we consider terms that differ only in their stepping behavior to be equal.
The extensional model is what we will ultimately use to interpret the syntax and
relational axioms of gradually-typed languages. The principal benefit of the
notion of intensional model is that it admits more compositional reasoning. It
therefore serves as a convenient intermediate setting that enjoys many of the
benefits of the usual double-categorical models of gradual typing.
% We give the definition of intensional model as a sequence of steps each
% building on the previous. 
We then outline how to build an extensional model in a sequence of steps beginning
with a significantly simpler ``input'' model. The construction proceeds in
steps, first building up an intensional model, and then from there extracting an
extensional model.
%We are able to leverage the compositional reasoning afforded by the intensional
%model to more easily meet the requirements of an extensional model.
This gives us a more tractable way to construct models.
Moreover, the abstract definitions and constructions in the definitions make no
mention of SGDT, so that the places where SGDT techniques are employed are
conveniently limited to the data passed as input to the model construction.
%
%In addition to organizing and facilitating the construction of a model, the abstract constructions are described independently of SGDT
%
%we also isolate the parts of the definition the notion of model without any explicit mention of SGDT.
%
Then in Section \ref{sec:concrete-model} we return to the task of defining a
concrete model. Equipped now with the abstract constructions of the previous
section, we provide only the data needed as an input to the construction. Many
of the relevant definitions have already been introduced in Section
\ref{sec:towards-relational-model} and can be provided to the construction
as-is; the remainder of the necessary data is defined in this section.
%
We end the section by sketching a proof of the \emph{adequacy} of our concrete
model with respect to graduality.
%
Finally, in Section \ref{sec:discussion} we conclude with discussion: comparison
to related work, the mechanization of our concrete model in Guarded Cubical
Agda, and directions for future research.





%% \subsection{Outline of Remainder of Paper}

%% \max{this is entirely out of date, update later}

%% % In Section \ref{sec:overview}, we give an overview of the gradually-typed lambda
%% % calculus and the graduality theorem.
%% %
%% In Section \ref{sec:technical-background}, we provide technical background on synthetic guarded domain theory.
%% % 
%% In Section \ref{sec:gtlc-terms}, we formally introduce the gradually-typed cast calculus
%% for which we will prove graduality. We give a semantics to the terms using
%% synthetic guarded domain theory.
%% % Important here are the notions of syntactic
%% % type precision and term precision. For reasons we describe below, we
%% % introduce two related calculi, one in which the stepping behavior of terms is
%% % implicit (an \emph{extensional} calculus, $\extlc$), and another where this behavior
%% % is made explicit (an \emph{intensional} calculus, $\intlc$).
%% %

%% In Section \ref{sec:gtlc-precision}, we define the term precision ordering for
%% the gradually-typed lambda calculus and describe our approach to assigning a
%% denotational semantics to this ordering.
%% This approach builds on the semantics constructed in the previous section,
%% but extending it to the term ordering presents new challenges.


%% % In Section \ref{sec:domain-theory}, we define several fundamental constructions
%% % internal to SGDT that will be needed when we give a denotational semantics to
%% % the gradual lambda calculus.
%% %This includes the Lift monad, predomains and error domains.
%% %
%% % In Section \ref{sec:semantics}, we define the denotational semantics for the
%% % intensional gradually-typed lambda calculus using the domain theoretic
%% % constructions defined in the previous section.
%% %
%% In Section \ref{sec:graduality}, we outline in more detail the proof of graduality for the
%% gradual lambda calculus.
%% %
%% In Section \ref{sec:discussion}, we discuss the benefits and drawbacks to our approach in comparison
%% to the traditional step-indexing approach, as well as possibilities for future work.

%% \subsection{Overview of Results}\label{sec:overview}

%% % This section used to be part of the intro.
%% % \subsection{Proving Graduality in SGDT}
%% % TODO This section should probably be moved to after the relevant background has been introduced?

%% % TODO introduce the idea of cast calculus and explicit casts?

%% In this paper, we will utilize SGDT techniques to prove graduality for a particularly
%% simple gradually-typed cast calculus, the gradually-typed lambda calculus.
%% Before stating the graduality theorem, we introduce some definitions related to gradual typing.

%% % Cast calculi
%% % TODO move this to earlier?
%% Gradually typed languages are generally elaborated to a \emph{cast calculus}, in which run-time type checking
%% that is made explicit. Elaboration inserts \emph{type casts} at the boundaries between static and dynamic code.
%% In particular, cast insertion happens at the elimination forms of the language (e.g., pattern matching, field reference, etc.).
%% The original gradually typed language that is elaborated to a cast calculus is called the \emph{surface language}.

%% % Up and down casts
%% In a cast calculus, there is a relation $\ltdyn$ on types such that $A \ltdyn B$ means that
%% $A$ is a \emph{more precise} type than $B$.
%% There a distinguished type $\dyn$, the \emph{dynamic type}, with the property that $A \ltdyn\, \dyn$ for all $A$.
%% The parts of the code that are dynamically typed will be given type $\dyn$ in the cast calculus.
%% %
%% If $A \ltdyn B$, a term $M$ of type $A$ may be \emph{up}casted to $B$, written $\up A B M$,
%% and a term $N$ of type $B$ may be \emph{down}casted to $A$, written $\dn A B N$.
%% Upcasts always succeed, while downcasts may fail at runtime, resulting in a type error.
%% Cast calculi have a term $\mho$ representing a run-time type error for any type $A$.

%% % Syntactic term precision
%% We also have a notion of \emph{syntactic term precision}.
%% If $A \ltdyn B$, and $M$ and $N$ are terms of type $A$ and $B$ respectively, we write
%% $M \ltdyn N : A \ltdyn B$ to mean that
%% $M$ is ``syntactically more precise'' than $N$, or equivalently, $N$ is 
%% ``more dynamic'' than $M$. Intuitively, this models the situation where $M$ and $N$
%% behave the same except that $M$ may error more.
%% Term precision is defined by a set of axioms that capture this intuitive notion.
%% The specific rules for term precision depend on the specifics of the language, but
%% many of the rules can be derived from the typing rules in a straightforward way.
%% % Cast calculi also have a term $\mho$ representing a run-time type error for any type $A$,
%% % and since
%% Since $\mho$ always errors, there is a term precision rule stating that $\mho \ltdyn M$ for all $M$.

%% % GTLC
%% The gradually-typed lambda calculus is the usual simply-typed lambda calculus with a dynamic
%% type $\dyn$ such that $A \ltdyn\, \dyn$ for all types $A$, as well as upcasts and downcasts
%% between any types $A$ and $B$ such that $A \ltdyn B$. The complete definition will be provided in
%% Section \ref{sec:gtlc-precision}.

%% With these definitions, we can state the graduality theorem for the gradually-typed lambda calculus.

%% % \begin{theorem}[Graduality]
%% %   If $M \ltdyn N : \nat$, then either:
%% %   \begin{enumerate}
%% %     \item $M = \mho$
%% %     \item $M = N = \zro$
%% %     \item $M = N = \suc n$ for some $n$
%% %     \item $M$ and $N$ diverge
%% %   \end{enumerate}
%% % \end{theorem}

%% \begin{theorem}[Graduality]
%%   Let $\cdot \vdash M \ltdyn N : \nat$. 
%%   If $M \Downarrow v_?$ and $N \Downarrow v'_?$, then either $v_? = \mho$, or $v_? = v'_?$.
%% \end{theorem}

%% Note:
%% \begin{itemize}

%%   \item $\cdot \Downarrow$ is a relation between terms that is defined such that $M \Downarrow$ means
%%   that $M$ terminates, either with a run-time error or a value $n$ of type $\nat$.

%%   \item $v_?$ is shorthand for the syntactic representation of a term that is either equal to
%%   $\mho$, or equal to the syntactic representation of a value $n$ of type $\nat$.
%% \end{itemize}

%% % We also should be able to show that $\mho$, $\zro$, and $\suc\, N$ are not equal.

%% To prove graduality and validate the equational theory, we construct a model of the types
%% and terms and show that all of the axioms for term precision and for equality of terms
%% hold in this model. Modeling the dynamic type presents a challenge in the presence of a
%% language with functions: we want the dynamic type to represent a sum of all possible types
%% in the language, so we write down an recursive equation that the semantic object modeling
%% dynamic type should satisfy. When the language includes function types, this equation involves a
%% negative occurrence of the variable for which we are solving, and so the equation 
%% does not have inductive or coinductive solutions.
%% %
%% To model the dynamic type, we therefore use guarded recursion to define a suitable
%% semantic object that satisfies the unfolding isomorphism expected of the dynamic type.
%% The key is that we do not actually get an exact solution to the equation in the style
%% of traditional domain theory; rather, we get a ``guarded'' solution that holds ``up to a time step''.
%% %
%% That is, we introduce a notion of ``time'' and in the equation for the dynamic type,
%% we guard the negative occurrences of the variable by a special operator that
%% specifies that its argument is available ``later''.
%% %This can be seen as a logic that internalizes the notion of step-indexing.
%% See Section \ref{sec:technical-background} for more details on guarded type theory.

%% At a high level, the key parts of our proof are as follows:

%% % TODO revise this
%% \begin{itemize}
%%   \item Our first step toward proving graduality is to formulate a \emph{step-sensitive},
%%   or \emph{intensional}, gradual lambda calculus, which we call $\intlc$, in which the
%%   computation steps taken by a term are made explicit.
%%   The ``normal'' gradual lambda calculus for which we want to prove graduality will be called the
%%   \emph{surface}, \emph{step-insensitive}, or \emph{extensional}, gradual lambda calculus,
%%   denoted $\extlc$.

%%   \item We define a translation from the surface syntax to the intensional syntax, and
%%   prove a theorem relating the term precision in the surface to term precision in the
%%   intensional syntax.
  
%%   \item We define a semantics for the intensional syntax in guarded type theory, for both the
%%   terms and for the term precision ordering $\ltdyn$.

%% \end{itemize}


\section{Technical Background}\label{sec:technical-background}

\subsection{Gradual Typing}

% Cast calculi
In a gradually-typed language, the mixing of static and dynamic code is seamless, in that
the dynamically typed parts are checked at runtime. This type checking occurs at the elimination
forms of the language (e.g., pattern matching, field reference, etc.).
Gradual languages are generally elaborated to a \emph{cast calculus}, in which the dynamic
type checking is made explicit through the insertion of \emph{type casts}.

% Up and down casts
In a cast calculus, there is a relation $\ltdyn$ on types such that $A \ltdyn B$ means that
$A$ is a \emph{more precise} type than $B$.
There a dynamic type $\dyn$ with the property that $A \ltdyn\, \dyn$ for all $A$.
%
If $A \ltdyn B$, a term $M$ of type $A$ may be \emph{up}casted to $B$, written $\up A B M$,
and a term $N$ of type $B$ may be \emph{down}casted to $A$, written $\dn A B N$.
Upcasts always succeed, while downcasts may fail at runtime.
%
% Syntactic term precision
We also have a notion of \emph{syntactic term precision}.
If $A \ltdyn B$, and $M$ and $N$ are terms of type $A$ and $B$ respectively, we write
$M \ltdyn N : A \ltdyn B$ to mean that $M$ is more precise than $N$, i.e., $M$ and $N$
behave the same except that $M$ may error more.

% Modeling the dynamic type as a recursive sum type?
% Observational equivalence and approximation?

% synthetic guarded domain theory, denotational semantics therein

\subsection{Difficulties in Prior Semantics}
  % Difficulties in prior semantics

  In this work, we compare our approach to proving graduality to the approach
  introduced by New and Ahmed \cite{new-ahmed2018} which constructs a step-indexed
  logical relations model and shows that this model is sound with respect to their
  notion of contextual error approximation.

  Because the dynamic type is modeled as a non-well-founded
  recursive type, their logical relation needs to be paramterized by natural numbers
  to restore well-foundedness. This technique is known as a \emph{step-indexed logical relation}.
  Reasoning about step-indexed logical relations
  can be tedious and error-prone, and there are some very subtle aspects that must
  be taken into account in the proofs. Figure \ref{TODO} shows an example of a step-indexed logical
  relation for the gradually-typed lambda calculus.

  In particular, the prior approach of New and Ahmed requires two separate logical
  relations for terms, one in which the steps of the left-hand term are counted,
  and another in which the steps of the right-hand term are counted.
  Then two terms $M$ and $N$ are related in the ``combined'' logical relation if they are
  related in both of the one-sided logical relations. Having two separate logical relations
  complicates the statement of the lemmas used to prove graduality, becasue any statement that
  involves a term stepping needs to take into account whether we are counting steps on the left
  or the right. Some of the differences can be abstracted over, but difficulties arise for properties %/results
  as fundamental and seemingly straightforward as transitivty.

  Specifically, for transitivity, we would like to say that if $M$ is related to $N$ at
  index $i$ and $N$ is related to $P$ at index $i$, then $M$ is related to $P$ at $i$.
  But this does not actually hold: we requrie that one of the two pairs of terms
  be related ``at infinity'', i.e., that they are related at $i$ for all $i \in \mathbb{N}$.
  Which pair is required to satisfy this depends on which logical relation we are considering,
  (i.e., is it counting steps on the left or on the right),
  and so any argument that uses transitivity needs to consider two cases, one
  where $M$ and $N$ must be shown to be related for all $i$, and another where $N$ and $P$ must
  be related for all $i$. % This may not even be possible to show in some scenarios!

  % These complications introduced by step-indexing lead one to wonder whether there is a
  % way of proving graduality without relying on tedious arguments involving natural numbers.
  % An alternative approach, which we investigate in this paper, is provided by
  % \emph{synthetic guarded domain theory}, as discussed below.
  % Synthetic guarded domain theory allows the resulting logical relation to look almost
  % identical to a typical, non-step-indexed logical relation.

\subsection{Synthetic Guarded Domain Theory}
One way to avoid the tedious reasoning associated with step-indexing is to work
axiomatically inside of a logical system that can reason about non-well-founded recursive
constructions while abstracting away the specific details of step-indexing required
if we were working analytically.
The system that proves useful for this purpose is called \emph{synthetic guarded
domain theory}, or SGDT for short. We provide a brief overview here, but more
details can be found in \cite{birkedal-mogelberg-schwinghammer-stovring2011}.

SGDT offers a synthetic approach to domain theory that allows for guarded recursion
to be expressed syntactically via a type constructor $\later : \type \to \type$ 
(pronounced ``later''). The use of a modality to express guarded recursion
was introduced by Nakano \cite{Nakano2000}.
%
Given a type $A$, the type $\later A$ represents an element of type $A$
that is available one time step later. There is an operator $\nxt : A \to\, \later A$
that ``delays'' an element available now to make it available later.
We will use a tilde to denote a term of type $\later A$, e.g., $\tilde{M}$.

% TODO later is an applicative functor, but not a monad

There is a \emph{guarded fixpoint} operator

\[
  \fix : \forall T, (\later T \to T) \to T.
\]

That is, to construct a term of type $T$, it suffices to assume that we have access to
such a term ``later'' and use that to help us build a term ``now''.
This operator satisfies the axiom that $\fix f = f (\nxt (\fix f))$.
In particular, this axiom applies to propositions $P : \texttt{Prop}$; proving
a statement in this manner is known as $\lob$-induction.

The operators $\later$, $\next$, and $\fix$ described above can be indexed by objects
called \emph{clocks}. A clock serves as a reference relative to which steps are counted.
For instance, given a clock $k$ and type $T$, the type $\later^k T$ represents a value of type
$T$ one unit of time in the future according to clock $k$.
If we only ever had one clock, then we would not need to bother defining this notion.
However, the notion of \emph{clock quantification} is crucial for encoding coinductive types using guarded
recursion, an idea first introduced by Atkey and McBride \cite{atkey-mcbride2013}.


% Clocked Cubical Type Theory
\subsubsection{Ticked Cubical Type Theory}
% TODO motivation for Clocked Cubical Type Theory, e.g., delayed substitutions?

In Ticked Cubical Type Theory \cite{TODO}, there is an additional sort
called \emph{ticks}. Given a clock $k$, a tick $t : \tick k$ serves
as evidence that one unit of time has passed according to the clock $k$.
The type $\later A$ is represented as a function from ticks of a clock $k$ to $A$.
The type $A$ is allowed to depend on $t$, in which case we write $\later^k_t A$
to emphasize the dependence.

% TODO next as a function that ignores its input tick argument?

The rules for tick abstraction and application are similar to those of dependent
$\Pi$ types. 
In particular, if we have a term $M$ of type $\later^k A$, and we
have available in the context a tick $t' : \tick k$, then we can apply the tick to $M$ to get
a term $M[t'] : A[t'/t]$. We will also write tick application as $M_t$.
Conversely, if in a context $\Gamma, t : \tick k$ we have that $M$ has type $A$,
then in context $\Gamma$ we have $\lambda t.M$ has type $\later A$. % TODO dependent version?

The statements in this paper have been formalized in a variant of Agda called
Guarded Cubical Agda \cite{TODO}, an implementation of Clocked Cubical Type Theory.


% TODO axioms (clock irrelevance, tick irrelevance)?


\subsubsection{The Topos of Trees Model}

The topos of trees model provides a useful intuition for reasoning in SGDT 
\cite{birkedal-mogelberg-schwinghammer-stovring2011}.
This section presupposes knowledge of category theory and can be safely skipped without
disrupting the continuity.

The topos of trees $\calS$ is the presheaf category $\Set^{\omega^o}$. 
%
We assume a universe $\calU$ of types, with encodings for operations such
as sum types (written as $\widehat{+}$). There is also an operator 
$\laterhat \colon \later \calU \to \calU$ such that 
$\El(\laterhat(\nxt A)) =\,\, \later \El(A)$, where $\El$ is the type corresponding
to the code $A$.

An object $X$ is a family $\{X_i\}$ of sets indexed by natural numbers,
along with restriction maps $r^X_i \colon X_{i+1} \to X_i$ (see Figure \ref{fig:topos-of-trees-object}).
These should be thought of as ``sets changing over time", where $X_i$ is the view of the set if we have $i+1$
time steps to reason about it.
We can also think of an ongoing computation, with $X_i$ representing the potential results of the computation
after it has run for $i+1$ steps.

\begin{figure}
  % https://q.uiver.app/?q=WzAsNSxbMiwwLCJYXzIiXSxbMCwwLCJYXzAiXSxbMSwwLCJYXzEiXSxbMywwLCJYXzMiXSxbNCwwLCJcXGRvdHMiXSxbMiwxLCJyXlhfMCIsMl0sWzAsMiwicl5YXzEiLDJdLFszLDAsInJeWF8yIiwyXSxbNCwzLCJyXlhfMyIsMl1d
\[\begin{tikzcd}[ampersand replacement=\&]
	{X_0} \& {X_1} \& {X_2} \& {X_3} \& \dots
	\arrow["{r^X_0}"', from=1-2, to=1-1]
	\arrow["{r^X_1}"', from=1-3, to=1-2]
	\arrow["{r^X_2}"', from=1-4, to=1-3]
	\arrow["{r^X_3}"', from=1-5, to=1-4]
\end{tikzcd}\]
  \caption{An example of an object in the topos of trees.}
  \label{fig:topos-of-trees-object}
\end{figure}

A morphism from $\{X_i\}$ to $\{Y_i\}$ is a family of functions $f_i \colon X_i \to Y_i$
that commute with the restriction maps in the obvious way, that is,
$f_i \circ r^X_i = r^Y_i \circ f_{i+1}$ (see Figure \ref{fig:topos-of-trees-morphism}).

\begin{figure}
% https://q.uiver.app/?q=WzAsMTAsWzIsMCwiWF8yIl0sWzAsMCwiWF8wIl0sWzEsMCwiWF8xIl0sWzMsMCwiWF8zIl0sWzQsMCwiXFxkb3RzIl0sWzAsMSwiWV8wIl0sWzEsMSwiWV8xIl0sWzIsMSwiWV8yIl0sWzMsMSwiWV8zIl0sWzQsMSwiXFxkb3RzIl0sWzIsMSwicl5YXzAiLDJdLFswLDIsInJeWF8xIiwyXSxbMywwLCJyXlhfMiIsMl0sWzQsMywicl5YXzMiLDJdLFs2LDUsInJeWV8wIiwyXSxbNyw2LCJyXllfMSIsMl0sWzgsNywicl5ZXzIiLDJdLFs5LDgsInJeWV8zIiwyXSxbMSw1LCJmXzAiLDJdLFsyLDYsImZfMSIsMl0sWzAsNywiZl8yIiwyXSxbMyw4LCJmXzMiLDJdXQ==
\[\begin{tikzcd}[ampersand replacement=\&]
	{X_0} \& {X_1} \& {X_2} \& {X_3} \& \dots \\
	{Y_0} \& {Y_1} \& {Y_2} \& {Y_3} \& \dots
	\arrow["{r^X_0}"', from=1-2, to=1-1]
	\arrow["{r^X_1}"', from=1-3, to=1-2]
	\arrow["{r^X_2}"', from=1-4, to=1-3]
	\arrow["{r^X_3}"', from=1-5, to=1-4]
	\arrow["{r^Y_0}"', from=2-2, to=2-1]
	\arrow["{r^Y_1}"', from=2-3, to=2-2]
	\arrow["{r^Y_2}"', from=2-4, to=2-3]
	\arrow["{r^Y_3}"', from=2-5, to=2-4]
	\arrow["{f_0}"', from=1-1, to=2-1]
	\arrow["{f_1}"', from=1-2, to=2-2]
	\arrow["{f_2}"', from=1-3, to=2-3]
	\arrow["{f_3}"', from=1-4, to=2-4]
\end{tikzcd}\]
  \caption{An example of a morphism in the topos of trees.}
  \label{fig:topos-of-trees-morphism}
\end{figure}


The type operator $\later$ is defined on an object $X$ by
$(\later X)_0 = 1$ and $(\later X)_{i+1} = X_i$.
The restriction maps are given by $r^\later_0 =\, !$, where $!$ is the
unique map into $1$, and $r^\later_{i+1} = r^X_i$.
The morphism $\nxt^X \colon X \to \later X$ is defined pointwise by
$\nxt^X_0 =\, !$, and $\nxt^X_{i+1} = r^X_i$. It is easily checked that
this satisfies the commutativity conditions required of a morphism in $\calS$.
%
Given a morphism $f \colon \later X \to X$, i.e., a family of functions
$f_i \colon (\later X)_i \to X_i$ that commute with the restrictions in the appropriate way,
we define $\fix(f) \colon 1 \to X$ pointwise
by $\fix(f)_i = f_{i} \circ \dots \circ f_0$.
This can be visualized as a diagram in the category of sets as shown in
Figure \ref{fig:topos-of-trees-fix}.
% Observe that the fixpoint is a \emph{global element} in the topos of trees.
% Global elements allow us to view the entire computation on a global level.


% https://q.uiver.app/?q=WzAsOCxbMSwwLCJYXzAiXSxbMiwwLCJYXzEiXSxbMywwLCJYXzIiXSxbMCwwLCIxIl0sWzAsMSwiWF8wIl0sWzEsMSwiWF8xIl0sWzIsMSwiWF8yIl0sWzMsMSwiWF8zIl0sWzMsNCwiZl8wIl0sWzAsNSwiZl8xIl0sWzEsNiwiZl8yIl0sWzIsNywiZl8zIl0sWzAsMywiISIsMl0sWzEsMCwicl5YXzAiLDJdLFsyLDEsInJeWF8xIiwyXSxbNSw0LCJyXlhfMCJdLFs2LDUsInJeWF8xIl0sWzcsNiwicl5YXzIiXV0=
% \[\begin{tikzcd}[ampersand replacement=\&]
% 	1 \& {X_0} \& {X_1} \& {X_2} \\
% 	{X_0} \& {X_1} \& {X_2} \& {X_3}
% 	\arrow["{f_0}", from=1-1, to=2-1]
% 	\arrow["{f_1}", from=1-2, to=2-2]
% 	\arrow["{f_2}", from=1-3, to=2-3]
% 	\arrow["{f_3}", from=1-4, to=2-4]
% 	\arrow["{!}"', from=1-2, to=1-1]
% 	\arrow["{r^X_0}"', from=1-3, to=1-2]
% 	\arrow["{r^X_1}"', from=1-4, to=1-3]
% 	\arrow["{r^X_0}", from=2-2, to=2-1]
% 	\arrow["{r^X_1}", from=2-3, to=2-2]
% 	\arrow["{r^X_2}", from=2-4, to=2-3]
% \end{tikzcd}\]

% \begin{figure}
%   % https://q.uiver.app/?q=WzAsMTksWzEsMiwiWF8wIl0sWzIsMywiWF8xIl0sWzMsMSwiMSJdLFswLDEsIjEiXSxbMCwyLCJYXzAiXSxbMSwzLCJYXzEiXSxbMSwxLCIxIl0sWzIsMSwiMSJdLFsyLDIsIlhfMCJdLFsyLDQsIlhfMiJdLFszLDIsIlhfMCJdLFszLDMsIlhfMSJdLFszLDQsIlhfMiJdLFszLDUsIlhfMyJdLFs0LDIsIlxcY2RvdHMiXSxbMCwwLCJcXGZpeChmKV8wIl0sWzEsMCwiXFxmaXgoZilfMSJdLFsyLDAsIlxcZml4KGYpXzIiXSxbMywwLCJcXGZpeChmKV8zIl0sWzMsNCwiZl8wIl0sWzAsNSwiZl8xIl0sWzYsMCwiZl8wIl0sWzcsOCwiZl8wIl0sWzgsMSwiZl8xIl0sWzEsOSwiZl8yIl0sWzIsMTAsImZfMCJdLFsxMCwxMSwiZl8xIl0sWzExLDEyLCJmXzIiXSxbMTIsMTMsImZfMyJdXQ==
%   \[\begin{tikzcd}[ampersand replacement=\&]
%     {\fix(f)_0} \& {\fix(f)_1} \& {\fix(f)_2} \& {\fix(f)_3} \\
%     1 \& 1 \& 1 \& 1 \\
%     {X_0} \& {X_0} \& {X_0} \& {X_0} \& \cdots \\
%     \& {X_1} \& {X_1} \& {X_1} \\
%     \&\& {X_2} \& {X_2} \\
%     \&\&\& {X_3}
%     \arrow["{f_0}", from=2-1, to=3-1]
%     \arrow["{f_1}", from=3-2, to=4-2]
%     \arrow["{f_0}", from=2-2, to=3-2]
%     \arrow["{f_0}", from=2-3, to=3-3]
%     \arrow["{f_1}", from=3-3, to=4-3]
%     \arrow["{f_2}", from=4-3, to=5-3]
%     \arrow["{f_0}", from=2-4, to=3-4]
%     \arrow["{f_1}", from=3-4, to=4-4]
%     \arrow["{f_2}", from=4-4, to=5-4]
%     \arrow["{f_3}", from=5-4, to=6-4]
%   \end{tikzcd}\]
%   \caption{The first few approximations to the guarded fixpoint of $f$.}
%   \label{fig:topos-of-trees-fix-approx}
% \end{figure}


\begin{figure}
  % https://q.uiver.app/?q=WzAsNixbMywwLCIxIl0sWzAsMiwiWF8wIl0sWzIsMiwiWF8xIl0sWzQsMiwiWF8yIl0sWzYsMiwiWF8zIl0sWzgsMiwiXFxkb3RzIl0sWzIsMSwicl5YXzAiXSxbNCwzLCJyXlhfMiJdLFswLDEsIlxcZml4KGYpXzAiLDFdLFswLDIsIlxcZml4KGYpXzEiLDFdLFswLDMsIlxcZml4KGYpXzIiLDFdLFswLDQsIlxcZml4KGYpXzMiLDFdLFswLDUsIlxcY2RvdHMiLDMseyJzdHlsZSI6eyJib2R5Ijp7Im5hbWUiOiJub25lIn0sImhlYWQiOnsibmFtZSI6Im5vbmUifX19XSxbMywyLCJyXlhfMSJdLFs1LDQsInJeWF8zIl1d
  \[\begin{tikzcd}[ampersand replacement=\&,column sep=2.25em]
    \&\&\& 1 \\
    \\
    {X_0} \&\& {X_1} \&\& {X_2} \&\& {X_3} \&\& \dots
    \arrow["{r^X_0}", from=3-3, to=3-1]
    \arrow["{r^X_2}", from=3-7, to=3-5]
    \arrow["{\fix(f)_0}"{description}, from=1-4, to=3-1]
    \arrow["{\fix(f)_1}"{description}, from=1-4, to=3-3]
    \arrow["{\fix(f)_2}"{description}, from=1-4, to=3-5]
    \arrow["{\fix(f)_3}"{description}, from=1-4, to=3-7]
    \arrow["\cdots"{marking}, draw=none, from=1-4, to=3-9]
    \arrow["{r^X_1}", from=3-5, to=3-3]
    \arrow["{r^X_3}", from=3-9, to=3-7]
  \end{tikzcd}\]
  \caption{The guarded fixpoint of $f$.}
  \label{fig:topos-of-trees-fix}
\end{figure}

% TODO global elements?

% \section{Syntactic Theory of Gradually Typed Lambda Calculus}\label{sec:GTLC}

Here we give an overview of a fairly standard cast calculus for
gradual typing along with its (in-)equational theory that capture our
desired notion of type-based reasoning and graduality. The main
departure from prior work is our explicit treatment of type precision
derivations and an equational theory of those derivations.

We give the basic syntax and select typing rules in
Figure~\ref{fig:gtlc-syntax}. We include a dynamic type, a type of
numbers, the call-by-value function type $A \ra A'$ and products.
%
We include a syntax for \emph{type precision} derivations $c : A
\ltdyn A'$; the typing is given in Figure~\ref{fig:gtlc-syntax}.
%
Any type precision derivation $c : A \ltdyn A'$ induces a pair of
casts, the upcast $\upc c : A \ra A'$ and the downcast $\dnc c : A' \ra
A$.
%
The syntactic intuition is that $c$ is a proof that $A$ is ``less
dynamic'' than $A'$. Semantically, this gives us coercions back and
forth where the upcast is (to a first-order) a pure function whereas
the downcast can fail.
%
These casts are inserted automatically in an elaboration from a
surface language. In this work, we are focused on semantic aspects and
so elide these standard details.
%
The syntax of precision derivations includes reflexivity $r(A)$ and
transitivity $cc'$ as well as monotonicity $c \ra c'$ and $c \times
c'$ that are \emph{covariant} in all arguments and finally generators
$\inat,\iarr,\itimes$ that correspond to the type tags of our dynamic
type.
%
We additionally impose an equational theory $c \equiv c'$ on the
derivations that implies that the corresponding casts are weakly
bisimilar in the semantics.
%
We impose category axioms for the reflexivity and
transitivity and functoriality for the monotonicity rules.
%
We note the following two admissible principles: any two derivations
$c,c' : A \ltdyn A'$ of the same fact are equivalent $c \equiv c'$ and
for any $A$, there is a derivation $\textrm{dyn}(A): A \ltdyn\dyn$. That is, $\dyn$ is the ``most dynamic'' type.

There is a more common set of rules for type precision where reflexivity and
transitivity are admissible, and whenever $A \ltdyn A'$, there is a unique
precision derivation witnessing this. These rules are shown in Section
\ref{sec:appendix-gtlc-syntax} in the Appendix. The reason for choosing our
system rather than that one is that in our semantics, equivalent type precision
derivations do not denote \emph{equal} relations. Instead, they denote
relations that are \emph{quasi-equivalent}, i.e., if two terms are related by
one then they are related also by the other up to insertion of delays (see
Definition \ref{def:quasi-equivalent} for the details).
%
However, because all type precision derivations in our system are equivalent, it
is straightforward to define a translation from the more standard system of type
and term precision into ours, so ultimately our graduality proofs can still be
applied to the standard formulation.

\begin{figure}
  \begin{mathpar}
    \begin{array}{rcl}
    \text{Types } A &::=& \nat \alt \,\dyn \alt A \ra A' \alt A \times A'\\
    \text{Type Precision } c &::=& r(A) \alt c c' \alt \iarr \alt \inat \alt \itimes \alt c \ra c' \alt c \times c'\\
    \text{Values } V &::=& x \alt \upc c V \alt \zro \alt \suc\, V \alt \lda{x}{M} \alt (V,V') \\ 
    \text{Terms } M,N &::=& \err\alt \upc c M \alt \dnc c M \alt \zro \alt \suc\, M \alt \lda{x}{M} \\ 
     &&\alt M\, N \alt (M,N) \alt \textrm{let } (x,y) = M \textrm{ in } N\\
    \text{Contexts } \Gamma &::= &\cdot \alt \Gamma, x : A \\
    \text{Ctx Precision } \Delta &::=& \cdot\alt \Delta,x:c
  \end{array}

  \inferrule
  {\Gamma \vdash M : A \and c : A \ltdyn A'}
  {\Gamma \vdash \upc c M : A'}

  \inferrule
  {\Gamma \vdash N : A' \and c : A \ltdyn A'}
  {\Gamma \vdash \dnc c N : A}

  \inferrule{}{\Gamma \vdash \mho : A}
  \end{mathpar}
  \begin{mathpar}
    \inferrule{}{r(A) : A \ltdyn A}\and
    \inferrule{c : A \ltdyn A' \and c' : A' \ltdyn A''}{cc' : A \ltdyn A''}\and
    \inferrule{}{\iarr \colon \dyn \ra \dyn \ltdyn \dyn}\and
    \inferrule{}{\inat \colon \nat \ltdyn \dyn}\and
    \inferrule{}{\itimes \colon \dyn \times \dyn \ltdyn \dyn}\and
    \inferrule{c_i : A_i \ltdyn A_i' \and c_o : A_o \ltdyn A_o'}{c_i \ra c_o : (A_i \ra A_o) \ltdyn (A_i' \ra A_o')}\and
    \inferrule{c_1 : A_1 \ltdyn A_1' \and c_2 : A_2 \ltdyn A_2'}{c_1 \times c_2 : (A_1 \times A_2) \ltdyn (A_1' \times A_2')}\and
     r(A)c \equiv c\and
     c \equiv cr(A')\and
     c(c'c'') \equiv (cc')c''\and
     r(A_i \ra A_o) \equiv r(A_i) \ra r(A_o)\and
     r(A_1\times A_2) \equiv r(A_1) \times r(A_2)\and
     (c_i \ra c_o)(c_i' \ra c_o')\equiv (c_ic_i' \ra c_oc_o') \and
     (c_1\times c_2)(c_1'\times c_2')\equiv (c_1c_1' \times c_2c_2')
  \end{mathpar}
  \caption{GTLC Cast Calculus Syntax, Type Precision Derivations and Precision Equivalence}
  \label{fig:gtlc-syntax}
\end{figure}

Next, we consider the axiomatic (in)equational reasoning principles
for terms: $\beta\eta$ equality and term precision in
Figure~\ref{fig:term-prec}.
%
We include standard CBV $\beta\eta$ rules for function and product
types, as well as equations stating that casts are given functorially.
%
Next, we have \emph{term} precision, an extension of
type precision to terms.
%
The form of the term precision rule is $\Delta \vdash M \ltdyn M' : c$
where $\Delta$ is a context where variables are assigned to type
precision derivations.
%
The judgment is only well formed when every use of $x : c$ for $c : A
\ltdyn A'$ is used with type $A$ in $M$ and $A'$ in $M'$ and similarly
the output types match $c$.
%
We elide the congruence rules for every type constructor, e.g., that
$M \ltdyn M'$ and $N \ltdyn N'$ that $M\,N \ltdyn M'\,N'$.
%
With such congruence rules, reflexivity $M \ltdyn M$ is
admissible. Transitivity, on the other hand, is intentionally not
taken as a primitive rule, matching the original formulation of the
dynamic gradual guarantee \cite{siek_et_al:LIPIcs:2015:5031}.
%
We include a rule that says that equivalent type precision derivations
$c \equiv c'$ are equivalent for the purposes of term precision.
%
% Removed retraction
%
% The next rule is the \emph{retraction} principle, which states that a
% downcast after an upcast is equivalent to doing nothing at all, since
% intuitively the upcasted value should already satisfy the type. Here
% $\equidyn$ means we require each is $\ltdyn$ the other, with
% reflexivity precision derivations.
%
Finally, we include 4 rules for reasoning about casts. Intuitively
these say that the upcast is a kind of \emph{least upper bound} and
dually that the downcast is a \emph{greatest lower bound}.

As a higher-order gradually typed language, we inherently have to deal
with two effects: errors and divergence. Errors arise from failing
casts, e.g. casting a number to dynamic to a function
$\dnc{\iarr}\upc{\inat} x$. Divergence arises because our dynamic type
allows us to encode untyped lambda calculus, and so we can encode the
$\Omega$ term with the help of casts $\Omega = (\lambda
x:\dyn. (\dnc{\iarr} x)x)(\upc{\iarr}(\lambda x:\dyn. (\dnc{\iarr}
x)x))$.

\begin{figure}
  \begin{mathpar}
  (\lambda x. M)(V) = M[V/x] \and (V : A \ra A') = \lambda x. V\,x\\

   \textrm{let } (x,y) = (V,V') \textrm{ in } N = N[V/x,V'/y] \and
   M[V:A\times A'/p] = \textrm{let } (x,y) = V \textrm{ in } M[(x,y)/p]

  % Removed these as they are not needed. The next rule implies that they are quasi-equivalent.
  % \upc{(r(A))}M = M \and
  % \upc{c'}\upc{c}M = \upc{cc'}M \and
  % \dnc{(r(A))}M = M \and
  % \dnc{c}\dnc{c'}M = \dnc{cc'}M

  \inferrule*[right=EquivTyPrec]
  {\Delta\vdash M \ltdyn M' : c \and c \equiv c'}
  {\Delta\vdash M \ltdyn M' : c'}

  \inferrule*[right=ErrBot]
  {}
  {\Delta \vdash \mho \ltdyn M : c}

  % Removed retraction
  % \inferrule
  % {}
  % {\dnc {c} \upc {c} M \equidyn M}

  \inferrule*[right=UpL]
  {M \ltdyn M' : cc_r}
  {\upc {c} M \ltdyn M' : c_r}

  \inferrule*[right=UpR]
  {M \ltdyn M' : c_l}
  {M \ltdyn \upc {c} M' : c_lc}

  \inferrule*[right=DnL]
  {M \ltdyn M' : c_r}
  {\dnc {c} M \ltdyn M' : cc_r}

  \inferrule*[right=DnR]
  {M \ltdyn M' : c_lc}
  {M \ltdyn \dnc {c} M' : c_l}
  \end{mathpar}
  \caption{Equality and Term Precision Rules (Selected)}
  \label{fig:term-prec}
\end{figure}

Our goal in the remainder of this work is to develop compositional
denotational semantics of types, terms, type and term precision from
which we can easily extract a big step semantics that satisfies
graduality and respects the equational theory of the calculus.

%% Here we describe the syntax and typing for the gradually-typed lambda calculus.
%% We also give the rules for syntactic type and term precision.
%% % We define four separate calculi: the normal gradually-typed lambda calculus, which we
%% % call the extensional or \emph{step-insensitive} lambda calculus ($\extlc$),
%% % as well as an \emph{intensional} lambda calculus
%% % ($\intlc$) whose syntax makes explicit the steps taken by a program.

%% Before diving into the details, let us give a brief overview of what we will define.
%% We begin with a gradually-typed lambda calculus $(\extlc)$, which is similar to
%% the normal call-by-value gradually-typed lambda calculus, but differs in that it
%% is actually a fragment of call-by-push-value specialized such that there are no
%% non-trivial computation types. We do this for convenience, as either way
%% we would need a distinction between values and effectful terms; the framework of
%% of call-by-push-value gives us a convenient language to define what we need.

%% We then show that composition of type precision derivations is admissible, as is
%% heterogeneous transitivity for term precision, so it will suffice to consider a new
%% language ($\extlcm$) in which we don't have composition of type precision derivations
%% or heterogeneous transitivity of term precision.

%% We then observe that all casts, except those between $\nat$ and $\dyn$
%% and between $\dyn \ra \dyn$ and $\dyn$, are admissible.
%% % (we can define the cast of a function type functorially using the casts for its domain and codomain).
%% This means it will be sufficient to consider a new language ($\extlcmm$) in which
%% instead of having arbitrary casts, we have injections from $\nat$ and
%% $\dyn \ra \dyn$ into $\dyn$, and case inspections from $\dyn$ to $\nat$ and
%% $\dyn$ to $\dyn \ra \dyn$.

%% From here, we define a \emph{step-sensitive} (also called \emph{intensional}) GSTLC,
%% so-named because it makes the intensional stepping behavior of programs explicit in the syntax.
%% This is accomplished by adding a syntactic ``later'' type and a
%% syntactic $\theta$ that maps terms of type later $A$ to terms of type $A$.
%% Finally, we define a \emph{quotiented} version of the step-sensitive language where
%% we add a rule that equates terms that are the same up to their stepping behavior.

%% % ---------------------------------------------------------------------------------------
%% % ---------------------------------------------------------------------------------------

%% \subsection{Syntax}

%% The language is based on Call-By-Push-Value \cite{levy01:phd}, and as such it has two kinds of types:
%% \emph{value types}, representing pure values, and \emph{computation types}, representing
%% potentially effectful computations.
%% In the language, all computation types have the form $\Ret A$ for some value type $A$.
%% Given a value $V$ of type $A$, the term $\ret V$ views $V$ as a term of computation type $\Ret A$.
%% Given a term $M$ of computation type $B$, the term $\bind{x}{M}{N}$ should be thought of as
%% running $M$ to a value $V$ and then continuing as $N$, with $V$ in place of $x$.


%% We also have value contexts and computation contexts, where the latter can be viewed
%% as a pair consisting of (1) a stoup $\Sigma$, which is either empty or a hole of type $B$,
%% and (2) a (potentially empty) value context $\Gamma$.

%% \begin{align*} % TODO is hole a term?
%%   &\text{Value Types } A := \nat \alt \,\dyn \alt (A \ra A') \\
%%   &\text{Computation Types } B := \Ret A \\
%%   &\text{Value Contexts } \Gamma := \cdot \alt (\Gamma, x : A) \\
%%   &\text{Computation Contexts } \Delta := \cdot \alt \hole B \alt \Delta , x : A \\
%%   &\text{Values } V :=  \zro \alt \suc\, V \alt \lda{x}{M} \alt \up{A}{B} V \\ 
%%   &\text{Terms } M, N := \err_B \alt \matchnat {V} {M} {n} {M'} \\ 
%%   &\quad\quad \alt \ret {V} \alt \bind{x}{M}{N} \alt V_f\, V_x \alt \dn{A}{B} M 
%% \end{align*}

%% The value typing judgment is written $\hasty{\Gamma}{V}{A}$ and 
%% the computation typing judgment is written $\hasty{\Delta}{M}{B}$.

%% \begin{comment}
%% We define substitution for value contexts by the following rules:

%% \begin{mathpar}
%%   \inferrule*
%%   { \gamma : \Gamma' \to \Gamma \and 
%%     \hasty{\Gamma'}{V}{A}}
%%   { (\gamma , V/x ) \colon \Gamma' \to \Gamma , x : A }

%%   \inferrule*
%%   {}
%%   {\cdot \colon \cdot \to \cdot}
%% \end{mathpar}

%% We define substitution for computation contexts by the following rules:

%% \begin{mathpar}
%%     \inferrule*
%%     { \delta : \Delta' \to \Delta \and 
%%       \hasty{\Delta'|_V}{V}{A}}
%%     { (\delta , V/x ) \colon \Delta' \to \Delta , x : A }

%%     \inferrule*
%%     {}
%%     {\cdot \colon \cdot \to \cdot}

%%     \inferrule*
%%     {\hasty{\Delta'}{M}{B}}
%%     {M \colon \Delta' \to \hole{B}}
%% \end{mathpar}
%% \end{comment}

%% The typing rules are as expected, with a cast between $A$ to $B$ allowed only when $A \ltdyn B$.
%% Notice that the upcast of a value is a value, since it always succeeds, while the downcast
%% of a value is a computation, since it may fail.

%% \begin{mathpar}
%%     % Var
%%     \inferrule*{ }{\hasty {\cdot, \Gamma, x : A, \Gamma'} x A}

%%     % Err
%%     \inferrule*{ }{\hasty {\cdot, \Gamma} {\err_B} B} 
  
%%     % Zero and suc
%%     \inferrule*{ }{\hasty \Gamma \zro \nat}
  
%%     \inferrule*{\hasty \Gamma V \nat} {\hasty \Gamma {\suc\, V} \nat}

%%     % Match-nat
%%     \inferrule*
%%     {\hasty \Gamma V \nat \and 
%%      \hasty \Delta M B \and \hasty {\Delta, n : \nat} {M'} B}
%%     {\hasty \Delta {\matchnat {V} {M} {n} {M'}} B}
  
%%     % Lambda
%%     \inferrule* 
%%     {\hasty {\cdot, \Gamma, x : A} M {\Ret A'}} 
%%     {\hasty \Gamma {\lda x M} {A \ra A'}}
  
%%     % App
%%     \inferrule*
%%     {\hasty \Gamma {V_f} {A \ra A'} \and \hasty \Gamma {V_x} A}
%%     {\hasty {\cdot , \Gamma} {V_f \, V_x} {\Ret A'}}

%%     % Ret
%%     \inferrule*
%%     {\hasty \Gamma V A}
%%     {\hasty {\cdot , \Gamma} {\ret\, V} {\Ret A}}
%%     % TODO should this involve a Delta?

%%     % Bind
%%     \inferrule*
%%     {\hasty \Delta M {\Ret A} \and \hasty{\cdot , \Delta|_V , x : A}{N}{B} } % Need x : A in context
%%     {\hasty {\Delta} {\bind{x}{M}{N}} {B}}

%%     % Upcast
%%     \inferrule*
%%     {A \ltdyn A' \and \hasty \Gamma V A}
%%     {\hasty \Gamma {\up A {A'} V} {A'} }

%%     % Downcast
%%     % \inferrule*
%%     % {A \ltdyn A' \and \hasty {\Gamma} V {A'}}
%%     % {\hasty {\cdot, \Gamma} {\dn A {A'} V} {\Ret A}}

%%     \inferrule* % TODO is this correct?
%%     {B \ltdyn B' \and \hasty {\Delta} {M} {B'}}
%%     {\hasty {\Delta} {\dn B {B'} M} {B}}

%% \end{mathpar}


%% In the equational theory, we have $\beta$ and $\eta$ laws for function type,
%% as well a $\beta$ and $\eta$ law for $\Ret A$.

%% % TODO do we need to add a substitution rule here?
%% \begin{mathpar}
%%   % Function Beta and Eta
%%   \inferrule*
%%   {\hasty {\cdot, \Gamma, x : A} M {\Ret A'} \and \hasty \Gamma V A}
%%   {(\lda x M)\, V = M[V/x]}

%%   \inferrule*
%%   {\hasty \Gamma V {A \ra A}}
%%   {\Gamma \vdash V = \lda x {V\, x}}

%%   % Ret Beta and Eta
%%   \inferrule*
%%   {}
%%   {(\bind{x}{\ret\, V}{N}) = N[V/x]}

%%   \inferrule*
%%   {\hasty {\hole{\Ret A} , \Gamma} {M} {B}}
%%   {\hole{\Ret A}, \Gamma \vdash M = (\bind{x}{\bullet}{M[\ret\, x]})}

%%   % Match-nat Beta
%%   \inferrule*
%%   {\hasty \Delta M B \and \hasty {\Delta, n : \nat} {M'} B}
%%   {\matchnat{\zro}{M}{n}{M'} = M}

%%   \inferrule*
%%   {\hasty \Gamma V \nat \and 
%%    \hasty \Delta M B \and \hasty {\Delta, n : \nat} {M'} B}
%%   {\matchnat{\suc\, V}{M}{n}{M'} = M'}

%%   % Match-nat Eta
%%   % This doesn't build in substitution
%%   \inferrule*
%%   {\hasty {\Delta , x : \nat} M A}
%%   {M = \matchnat{x} {M[\zro / x]} {n} {M[(\suc\, n) / x]}}



%% \end{mathpar}

%% % ---------------------------------------------------------------------------------------
%% % ---------------------------------------------------------------------------------------

%% \subsection{Type Precision}

%% The type precision rules specify what it means for a type $A$ to be more precise than $A'$.
%% We have reflexivity rules for $\dyn$ and $\nat$, as well as rules that $\nat$ is more precise than $\dyn$
%% and $\dyn \ra \dyn$ is more precise than $\dyn$.
%% We also have a transitivity rule for composition of type precision,
%% and also a rule for function types stating that given $A_i \ltdyn A'_i$ and $A_o \ltdyn A'_o$, we can prove
%% $A_i \ra A_o \ltdyn A'_i \ra A'_o$.
%% Finally, we can lift a relation on value types $A \ltdyn A'$ to a relation $\Ret A \ltdyn \Ret A'$ on
%% computation types.

%% \begin{mathpar}
%%   \inferrule*[right = \dyn]
%%     { }{\dyn \ltdyn\, \dyn}

%%   \inferrule*[right = \nat]
%%     { }{\nat \ltdyn \nat}

%%   \inferrule*[right = $\ra$]
%%     {A_i \ltdyn A'_i \and A_o \ltdyn A'_o }
%%     {(A_i \ra A_o) \ltdyn (A'_i \ra A'_o)}

%%   \inferrule*[right = $\textsf{Inj}_\nat$]
%%     { }{\nat \ltdyn\, \dyn}

%%   \inferrule*[right=$\textsf{Inj}_{\ra}$]
%%     { }
%%     {(\dyn \ra \dyn) \ltdyn\, \dyn}

%%   \inferrule*[right=ValTrans]
%%     {A \ltdyn A' \and A' \ltdyn A''}
%%     {A \ltdyn A''}

%%   \inferrule*[right=CompTrans]
%%     {B \ltdyn B' \and B' \ltdyn B''}
%%     {B \ltdyn B''}

%%   \inferrule*[right=$\Ret{}$]
%%     {A \ltdyn A'}
%%     {\Ret {A} \ltdyn \Ret {A'}}

%%     % TODO are there other rules needed for computation types?

  
%% \end{mathpar}

%% % Type precision derivations
%% Note that as a consequence of this presentation of the type precision rules, we
%% have that if $A \ltdyn A'$, there is a unique precision derivation that witnesses this.
%% As in previous work, we go a step farther and make these derivations first-class objects,
%% known as \emph{type precision derivations}.
%% Specifically, for every $A \ltdyn A'$, we have a derivation $c : A \ltdyn A'$ that is constructed
%% using the rules above. For instance, there is a derivation $\dyn : \dyn \ltdyn \dyn$, and a derivation
%% $\nat : \nat \ltdyn \nat$, and if $c_i : A_i \ltdyn A_i$ and $c_o : A_o \ltdyn A'_o$, then
%% there is a derivation $c_i \ra c_o : (A_i \ra A_o) \ltdyn (A'_i \ra A'_o)$. Likewise for
%% the remaining rules. The benefit to making these derivations explicit in the syntax is that we
%% can perform induction over them.
%% Note also that for any type $A$, we use $A$ to denote the reflexivity derivation that $A \ltdyn A$,
%% i.e., $A : A \ltdyn A$.
%% Finally, observe that for type precision derivations $c : A \ltdyn A'$ and $c' : A' \ltdyn A''$, we
%% can define (via the rule ValComp) their composition $c \relcomp c' : A \ltdyn A''$.
%% The same holds for computation type precision derivations.
%% This notion will be used below in the statement of transitivity of the term precision relation.

%% % ---------------------------------------------------------------------------------------
%% % ---------------------------------------------------------------------------------------

%% \subsection{Term Precision}

%% We allow for a \emph{heterogeneous} term precision judgment on terms values $V$ of type
%% $A$ and $V'$ of type $A'$ provided that $A \ltdyn A'$ holds. Likewise, for computation
%% types $B \ltdyn B'$, if $M$ has type $B$ and $M'$ has type $B'$, we can form the judgment
%% that $M \ltdyn M'$.

%% % Type precision contexts
%% % TODO should we include the formal definitions of value and computation type precision contexts?
%% In order to deal with open terms, we will need the notion of a type precision \emph{context}, which we denote
%% $\gamlt$. This is similar to a normal context but instead of mapping variables to types,
%% it maps variables $x$ to related types $A \ltdyn A'$, where $x$ has type $A$ in the left-hand term
%% and $B$ in the right-hand term. We may also write $x : d$ where $d : A \ltdyn A'$ to indicate this.
%% Similarly, we have computation type precision contexts $\Delta^\ltdyn$. Similar to ``normal'' computation
%% type precision contexts $\Delta$, these consist of (1) a stoup $\Sigma$ which is either empty or
%% has a hole $\hole{d}$ for some computation type precision derivation $d$, and (2) a value type precision context
%% $\Gamma^\ltdyn$.

%% % An equivalent way of thinking of type precision contexts is as a pair of ``normal" typing
%% % contexts $\Gamma, \Gamma'$ with the same domain such that $\Gamma(x) \ltdyn \Gamma'(x)$ for
%% % each $x$ in the domain.
%% % We will write $\gamlt : \Gamma \ltdyn \Gamma'$ when we want to emphasize the pair of contexts.
%% % Conversely, if we are given $\gamlt$, we write $\gamlt_l$ and $\gamlt_r$ for the normal typing contexts on each side.

%% An equivalent way of thinking of a type precision context $\gamlt$ is as a
%% pair of ``normal" typing contexts, $\gamlt_l$ and $\gamlt_r$, with the same
%% domain and such that $\gamlt_l(x) \ltdyn \gamlt_r(x)$ for each $x$ in the domain.
%% We will write $\gamlt : \gamlt_l \ltdyn \gamlt_r$ when we want to emphasize the pair of contexts.

%% As with type precision derivations, we write $\Gamma$ to mean the ``reflexivity" type precision context
%% $\Gamma : \Gamma \ltdyn \Gamma$.
%% Concretely, this consists of reflexivity type precision derivations $\Gamma(x) \ltdyn \Gamma(x)$ for
%% each $x$ in the domain of $\Gamma$.
%% Similarly, we also have reflexivity for computation type precision contexts.
%% %
%% Furthermore, we write $\gamlt_1 \relcomp \gamlt_2$ to denote the ``composition'' of $\gamlt_1$ and $\gamlt_2$
%% --- that is, the precision context whose value at $x$ is the type precision derivation
%% $\gamlt_1(x) \relcomp \gamlt_2(x)$. This of course assumes that each of the type precision
%% derivations is composable, i.e., that the RHS of $\gamlt_1(x)$ is the same as the left-hand side of $\gamlt_2(x)$.
%% We define the same for computation type precision contexts $\deltalt_1$ and $\deltalt_2$,
%% provided that both the computation type precision contexts have the same ``shape'', which is defined as
%% (1) either the stoup is empty in both, or the stoup has a hole in both, say $\hole{d}$ and $\hole{d'}$
%% where $d$ and $d'$ are composable, and (2) their value type precision contexts are composable as described above.

%% The rules for term precision come in two forms. We first have the \emph{congruence} rules,
%% one for each term constructor. These assert that the term constructors respect term precision.
%% The congruence rules are as follows:

%% \begin{mathpar}

%%   \inferrule*[right = Var]
%%     { c : A \ltdyn B \and \gamlt(x) = (A, B) } 
%%     { \etmprec {\gamlt} x x c }

%%   \inferrule*[right = Zro]
%%     { } {\etmprec \gamlt \zro \zro \nat }

%%   \inferrule*[right = Suc]
%%     { \etmprec \gamlt V {V'} \nat } {\etmprec \gamlt {\suc\, V} {\suc\, V'} \nat}

%%   \inferrule*[right = MatchNat]
%%   {\etmprec \gamlt V {V'} \nat \and 
%%     \etmprec \deltalt M {M'} d \and \etmprec {\deltalt, n : \nat} {N} {N'} d}
%%   {\etmprec \deltalt {\matchnat {V} {M} {n} {N}} {\matchnat {V'} {M'} {n} {N'}} d}

%%   \inferrule*[right = Lambda]
%%     { c_i : A_i \ltdyn A'_i \and 
%%       c_o : A_o \ltdyn A'_o \and 
%%       \etmprec {\cdot , \gamlt , x : c_i} {M} {M'} {\Ret c_o} } 
%%     { \etmprec \gamlt {\lda x M} {\lda x {M'}} {(c_i \ra c_o)} }

%%   \inferrule*[right = App]
%%     { c_i : A_i \ltdyn A'_i \and
%%       c_o : A_o \ltdyn A'_o \\\\
%%       \etmprec \gamlt {V_f} {V_f'} {(c_i \ra c_o)} \and
%%       \etmprec \gamlt {V_x} {V_x'} {c_i}
%%     } 
%%     { \etmprec {\cdot , \gamlt} {V_f\, V_x} {V_f'\, V_x'} {\Ret {c_o}}}

%%   \inferrule*[right = Ret]
%%     {\etmprec {\gamlt} V {V'} c}
%%     {\etmprec {\cdot , \gamlt} {\ret\, V} {\ret\, V'} {\Ret c}}

%%   \inferrule*[right = Bind]
%%     {\etmprec {\deltalt} {M} {M'} {\Ret c} \and 
%%      \etmprec {\cdot , \deltalt|_V , x : c} {N} {N'} {d} }
%%     {\etmprec {\deltalt} {\bind {x} {M} {N}} {\bind {x} {M'} {N'}} {d}}
%% \end{mathpar}

%% We then have additional equational axioms, including transitivity, $\beta$ and $\eta$ laws, and
%% rules characterizing upcasts as least upper bounds, and downcasts as greatest lower bounds.

%% We write $M \equidyn N$ to mean that both $M \ltdyn N$ and $N \ltdyn M$.

%% % TODO adapt these for value/computation distinction
%% % TODO substitution rules for values and terms?
%% \begin{mathpar}
%%   \inferrule*[right = $\err$]
%%     { \hasty {\deltalt_l} M B }
%%     {\etmprec {\Delta} {\err_B} M B}

%%   \inferrule*[right = Transitivity]
%%     { d : B \ltdyn B' \and d' : B' \ltdyn B'' \\\\
%%      \etmprec {\deltalt_1} {M} {M'} {d} \and
%%      \etmprec {\deltalt_2} {M'} {M''} {d'} } 
%%     {\etmprec {\deltalt_1 \relcomp \deltalt_2} {M} {M''} {d \relcomp d'} }


%%   \inferrule*[right = $\beta$-fun]
%%     { \hasty {\cdot, \Gamma, x : A_i} M {\Ret A_o} \and
%%       \hasty {\Gamma} V {A_i} } 
%%     { \etmequidyn {\cdot, \Gamma} {(\lda x M)\, V} {M[V/x]} {\Ret A_o} }

%%   \inferrule*[right = $\eta$-fun]
%%     { \hasty {\Gamma} {V} {A_i \ra A_o} } 
%%     { \etmequidyn \Gamma {\lda x (V\, x)} V {A_i \ra A_o} }

%%   % Match-nat beta and eta



%%   \inferrule*[right = $\beta$-ret]
%%     {}
%%     {\bind{x}{\ret\, V}{N} \equidyn N[V/x]}

%%   \inferrule*[right = $\eta$-ret]
%%     {\hasty {\hole{\Ret A} , \Gamma} {M} {B}}
%%     {\hole{\Ret A}, \Gamma \vdash M \equidyn \bind{x}{\bullet}{M[\ret\, x]}}
    

%%   % Could specify \gamlt : \Gamma \ltdyn \Gamma'
%%   % and then we wouldn't need to say l and r

%%   \inferrule*[right = UpR]
%%     { d : A \ltdyn A' \and 
%%       \hasty {\Delta} {M} {A} } 
%%     { \etmprec {\Delta} {M} {\up {A} {A'} M} {d}  }

%%   \inferrule*[right = UpL]
%%     { d : A \ltdyn A' \and
%%       \etmprec {\deltalt} {M} {N} {d} } 
%%     { \etmprec {\deltalt} {\up {A} {A'} M} {N} {A'} }

%%   \inferrule*[right = DnL]
%%     { d : B \ltdyn B' \and 
%%       \hasty {\Delta} {M} {B'} } 
%%     { \etmprec {\Delta} {\dn {B} {B'} M} {M} {d} }

%%   \inferrule*[right = DnR]
%%     { d : B \ltdyn B' \and
%%       \etmprec {\deltalt} {M} {N} {d} } 
%%     { \etmprec {\deltalt} {M} {\dn {B} {B'} N} {B} }
%% \end{mathpar}

%% % TODO explain the least upper bound/greatest lower bound rules
%% The rules UpR, UpL, DnL, and DnR were introduced in \cite{new-licata18} as a means
%% of cleanly axiomatizing the intended behavior of casts in a way that
%% doesn't depend on the specific constructs of the language.
%% Intuitively, rule UpR says that the upcast of $M$ is an upper bound for $M$
%% in that $M$ may error more, and UpL says that the upcast is the \emph{least}
%% such upper bound, in that it errors more than any other upper bound for $M$.
%% Conversely, DnL says that the downcast of $M$ is a lower bound, and DnR says
%% that it is the \emph{greatest} lower bound.
%% % These rules provide a clean axiomatization of the behavior of casts that doesn't
%% % depend on the specific constructs of the language.

%% % ---------------------------------------------------------------------------------------
%% % ---------------------------------------------------------------------------------------
%% \subsection{Removing Transitivity as a Primitive}

%% The first observation we make is that transitivity of type precision, and heterogeneous
%% transitivity of term precision, are admissible. That is, consider a related language which
%% is the same as $\extlc$ except that we have removed the composition rule for type precision and
%% the heterogeneous transitivity rule for type precision. Denote this language by $\extlcm$.
%% We claim that in this new language, the rules we removed are derivable from the remaining rules.

%% To see this, suppose $\gamlt : \Gamma \ltdyn \Gamma'$ and $d : A \ltdyn A'$, and that
%%  $\etmprec {\gamlt} {V} {V'} {d}$, as shown in the diagram below:

%% % https://q.uiver.app/?q=WzAsNCxbMCwwLCJcXEdhbW1hIl0sWzAsMSwiXFxHYW1tYSciXSxbMSwwLCJBIl0sWzEsMSwiQSciXSxbMCwxLCJcXGx0ZHluIiwzLHsic3R5bGUiOnsiYm9keSI6eyJuYW1lIjoibm9uZSJ9LCJoZWFkIjp7Im5hbWUiOiJub25lIn19fV0sWzIsMywiXFxsdGR5biIsMyx7InN0eWxlIjp7ImJvZHkiOnsibmFtZSI6Im5vbmUifSwiaGVhZCI6eyJuYW1lIjoibm9uZSJ9fX1dLFswLDIsIlYiXSxbMSwzLCJWJyJdLFs2LDcsIlxcbHRkeW4iLDMseyJzaG9ydGVuIjp7InNvdXJjZSI6MjAsInRhcmdldCI6MjB9LCJzdHlsZSI6eyJib2R5Ijp7Im5hbWUiOiJub25lIn0sImhlYWQiOnsibmFtZSI6Im5vbmUifX19XV0=
%% \[\begin{tikzcd}[ampersand replacement=\&]
%% 	\Gamma \& A \\
%% 	{\Gamma'} \& {A'}
%% 	\arrow["\ltdyn"{marking}, draw=none, from=1-1, to=2-1]
%% 	\arrow["\ltdyn"{marking}, draw=none, from=1-2, to=2-2]
%% 	\arrow[""{name=0, anchor=center, inner sep=0}, "V", from=1-1, to=1-2]
%% 	\arrow[""{name=1, anchor=center, inner sep=0}, "{V'}", from=2-1, to=2-2]
%% 	\arrow["\ltdyn"{marking}, draw=none, from=0, to=1]
%% \end{tikzcd}\]

%% Now note that this is equivalent, by the cast rule UpL, to
%% $\etmprec {\Gamma'} {\up{A}{A'} V} {V'} {A'}$,
%% where as noted above, $\Gamma'$ refers to the context $\Gamma'$ viewed as a reflexivity
%% precision context and likewise the $A'$ at the end refers to the reflexivity derivation $A' \ltdyn A'$.

%% % https://q.uiver.app/?q=WzAsMixbMCwwLCJcXEdhbW1hJyJdLFsxLDAsIkEnIl0sWzAsMSwiXFx1cCB7QX0ge0EnfSBWIiwwLHsiY3VydmUiOi0yfV0sWzAsMSwiViciLDIseyJjdXJ2ZSI6Mn1dLFsyLDMsIlxcbHRkeW4iLDMseyJzaG9ydGVuIjp7InNvdXJjZSI6MjAsInRhcmdldCI6MjB9LCJzdHlsZSI6eyJib2R5Ijp7Im5hbWUiOiJub25lIn0sImhlYWQiOnsibmFtZSI6Im5vbmUifX19XV0=
%% \[\begin{tikzcd}[ampersand replacement=\&]
%% 	{\Gamma'} \& {A'}
%% 	\arrow[""{name=0, anchor=center, inner sep=0}, "{\up {A} {A'} V}", curve={height=-12pt}, from=1-1, to=1-2]
%% 	\arrow[""{name=1, anchor=center, inner sep=0}, "{V'}"', curve={height=12pt}, from=1-1, to=1-2]
%% 	\arrow["\ltdyn"{marking}, draw=none, from=0, to=1]
%% \end{tikzcd}\]

%% Now consider the situation shown below:

%% % https://q.uiver.app/?q=WzAsNixbMCwwLCJcXEdhbW1hIl0sWzAsMSwiXFxHYW1tYSciXSxbMCwyLCJcXEdhbW1hJyciXSxbMiwwLCJBIl0sWzIsMSwiQSciXSxbMiwyLCJBJyciXSxbMiw1LCJWJyciXSxbMSw0LCJWJyJdLFswLDMsIlYiXSxbMyw0LCJcXGx0ZHluIiwzLHsic3R5bGUiOnsiYm9keSI6eyJuYW1lIjoibm9uZSJ9LCJoZWFkIjp7Im5hbWUiOiJub25lIn19fV0sWzQsNSwiXFxsdGR5biIsMyx7InN0eWxlIjp7ImJvZHkiOnsibmFtZSI6Im5vbmUifSwiaGVhZCI6eyJuYW1lIjoibm9uZSJ9fX1dLFswLDEsIlxcbHRkeW4iLDMseyJzdHlsZSI6eyJib2R5Ijp7Im5hbWUiOiJub25lIn0sImhlYWQiOnsibmFtZSI6Im5vbmUifX19XSxbMSwyLCIiLDEseyJzdHlsZSI6eyJib2R5Ijp7Im5hbWUiOiJub25lIn0sImhlYWQiOnsibmFtZSI6Im5vbmUifX19XSxbMSwyLCJcXGx0ZHluIiwzLHsic3R5bGUiOnsiYm9keSI6eyJuYW1lIjoibm9uZSJ9LCJoZWFkIjp7Im5hbWUiOiJub25lIn19fV0sWzgsNywiXFxsdGR5biIsMyx7InNob3J0ZW4iOnsic291cmNlIjoyMCwidGFyZ2V0IjoyMH0sInN0eWxlIjp7ImJvZHkiOnsibmFtZSI6Im5vbmUifSwiaGVhZCI6eyJuYW1lIjoibm9uZSJ9fX1dLFs3LDYsIlxcbHRkeW4iLDMseyJzaG9ydGVuIjp7InNvdXJjZSI6MjAsInRhcmdldCI6MjB9LCJzdHlsZSI6eyJib2R5Ijp7Im5hbWUiOiJub25lIn0sImhlYWQiOnsibmFtZSI6Im5vbmUifX19XV0=
%% \[\begin{tikzcd}[ampersand replacement=\&]
%% 	\Gamma \&\& A \\
%% 	{\Gamma'} \&\& {A'} \\
%% 	{\Gamma''} \&\& {A''}
%% 	\arrow[""{name=0, anchor=center, inner sep=0}, "{V''}", from=3-1, to=3-3]
%% 	\arrow[""{name=1, anchor=center, inner sep=0}, "{V'}", from=2-1, to=2-3]
%% 	\arrow[""{name=2, anchor=center, inner sep=0}, "V", from=1-1, to=1-3]
%% 	\arrow["\ltdyn"{marking}, draw=none, from=1-3, to=2-3]
%% 	\arrow["\ltdyn"{marking}, draw=none, from=2-3, to=3-3]
%% 	\arrow["\ltdyn"{marking}, draw=none, from=1-1, to=2-1]
%% 	\arrow[draw=none, from=2-1, to=3-1]
%% 	\arrow["\ltdyn"{marking}, draw=none, from=2-1, to=3-1]
%% 	\arrow["\ltdyn"{marking}, draw=none, from=2, to=1]
%% 	\arrow["\ltdyn"{marking}, draw=none, from=1, to=0]
%% \end{tikzcd}\]


%% Using the above observation, we have that the above is equivalent to

%% % https://q.uiver.app/?q=WzAsNCxbMCwwLCJcXEdhbW1hJyJdLFswLDEsIlxcR2FtbWEnJyJdLFsyLDAsIkEnIl0sWzIsMSwiQScnIl0sWzAsMiwiXFx1cCB7QX0ge0EnfSBWIiwwLHsiY3VydmUiOi0yfV0sWzAsMiwiViciLDIseyJjdXJ2ZSI6Mn1dLFsxLDMsIlYnJyIsMix7ImN1cnZlIjoyfV0sWzAsMSwiXFxsdGR5biIsMyx7InN0eWxlIjp7ImJvZHkiOnsibmFtZSI6Im5vbmUifSwiaGVhZCI6eyJuYW1lIjoibm9uZSJ9fX1dLFsyLDMsIlxcbHRkeW4iLDMseyJzdHlsZSI6eyJib2R5Ijp7Im5hbWUiOiJub25lIn0sImhlYWQiOnsibmFtZSI6Im5vbmUifX19XSxbNCw1LCJcXGx0ZHluIiwzLHsic2hvcnRlbiI6eyJzb3VyY2UiOjIwLCJ0YXJnZXQiOjIwfSwic3R5bGUiOnsiYm9keSI6eyJuYW1lIjoibm9uZSJ9LCJoZWFkIjp7Im5hbWUiOiJub25lIn19fV0sWzUsNiwiXFxsdGR5biIsMyx7InNob3J0ZW4iOnsic291cmNlIjoyMCwidGFyZ2V0IjoyMH0sInN0eWxlIjp7ImJvZHkiOnsibmFtZSI6Im5vbmUifSwiaGVhZCI6eyJuYW1lIjoibm9uZSJ9fX1dXQ==
%% \[\begin{tikzcd}[ampersand replacement=\&]
%% 	{\Gamma'} \&\& {A'} \\
%% 	{\Gamma''} \&\& {A''}
%% 	\arrow[""{name=0, anchor=center, inner sep=0}, "{\up {A} {A'} V}", curve={height=-12pt}, from=1-1, to=1-3]
%% 	\arrow[""{name=1, anchor=center, inner sep=0}, "{V'}"', curve={height=12pt}, from=1-1, to=1-3]
%% 	\arrow[""{name=2, anchor=center, inner sep=0}, "{V''}"', curve={height=12pt}, from=2-1, to=2-3]
%% 	\arrow["\ltdyn"{marking}, draw=none, from=1-1, to=2-1]
%% 	\arrow["\ltdyn"{marking}, draw=none, from=1-3, to=2-3]
%% 	\arrow["\ltdyn"{marking}, draw=none, from=0, to=1]
%% 	\arrow["\ltdyn"{marking}, draw=none, from=1, to=2]
%% \end{tikzcd}\]

%% % TODO finish the explanation
  

%% % ---------------------------------------------------------------------------------------
%% % ---------------------------------------------------------------------------------------

%% \subsection{Removing Casts as Primitives}

%% % We now observe that all casts, except those between $\nat$ and $\dyn$
%% % and between $\dyn \ra \dyn$ and $\dyn$, are admissible, in the sense that
%% % we can start from $\extlcm$, remove casts except the aforementioned ones,
%% % and in the resulting language we will be able to derive the other casts.

%% We now observe that all casts, except those between $\nat$ and $\dyn$
%% and between $\dyn \ra \dyn$ and $\dyn$, are admissible.
%% That is, consider a new language ($\extlcmm$) in which
%% instead of having arbitrary casts, we have injections from $\nat$ and
%% $\dyn \ra \dyn$ into $\dyn$, and case inspections from $\dyn$ to $\nat$ and
%% $\dyn$ to $\dyn \ra \dyn$. We claim that in $\extlcmm$, all of the casts
%% present in $\extlcm$ are derivable.
%% It will suffice to verify that casts for function type are derivable.
%% This holds because function casts are constructed inductively from the casts
%% of their domain and codomain. The base case is one of the casts involving $\nat$
%% or $\dyn \ra \dyn$ which are present in $\extlcmm$ as injections and case inspections.


%% The resulting calculus $\extlcmm$ now lacks transitivity of type precision,
%% heterogeneous transitivity of term precision, and arbitrary casts as primitive
%% notions.

%% \begin{align*}
%%   &\text{Value Types } A := \nat \alt \dyn \alt (A \ra A') \\
%%   &\text{Computation Types } B := \Ret A \\
%%   &\text{Value Contexts } \Gamma := \cdot \alt (\Gamma, x : A) \\
%%   &\text{Computation Contexts } \Delta := \cdot \alt \hole B \alt \Delta , x : A \\
%%   &\text{Values } V :=  \zro \alt \suc\, V \alt \lda{x}{M} \alt \injnat V \alt \injarr V \\ 
%%   &\text{Terms } M, N := \err_B \alt \ret {V} \alt \bind{x}{M}{N}
%%     \alt V_f\, V_x \alt
%%     \\ & \quad\quad \casenat{V}{M_{no}}{n}{M_{yes}} 
%%     \alt \casearr{V}{M_{no}}{f}{M_{yes}}
%% \end{align*}

%% In this setting, rather than type precision, it makes more sense to
%% speak of arbitrary \emph{monotone relations} on types, which we denote by $A \rel A'$.
%% We have relations on value types, as well as on computation types. We also have
%% value relation contexts and computation relation contexts, analogous to the value type
%% precision contexts and computation type precision contexts from before.

%% \begin{align*}
%%   &\text{Value Relations } R := \nat \alt \dyn \alt (R \ra R) \alt\, \dyn\, R(V_1, V_2)\\
%%   &\text{Computation Relations } S := \li R \\
%%   &\text{Value Relation Contexts } \Gamma^{\rel} := \cdot \alt \Gamma^{\rel} , A^{\rel} (x_l : A_l , x_r : A_r)\\
%%   &\text{Computation Relation Contexts } \Delta^{\rel} := \cdot \alt \hole{B^{\rel}} \alt 
%%     \Delta^{\rel} , A^{\rel} (x_l : A_l , x_r : A_r)   \\
%% \end{align*}

%% % TODO rules for relations
%% The forms for relations are as follows:

%% \begin{align*}
%%   A^{\rel}      &\colon A_l      \rel A_r \\
%%   \Gamma^{\rel} &\colon \Gamma_l \rel \Gamma_r \\
%%   B^{\rel}      &\colon B_l      \rel B_r \\
%%   \Delta^{\rel} &\colon \Delta_l \rel \Delta_r
%% \end{align*}



%% Figure \ref{fig:relation-rules} shows the rules for relations. We show only those for value types;
%% the corresponding computation type relation rules are analogous.
%% The rules for relations are as follows. First, we require relations to be reflexive.
%% We also require that they are \emph{profunctorial}, in the sense that a relation between
%% $A$ and $A'$ is closed under the ``homogeneous'' relations on both sides.
%% We also require that they satisfy a substitution principle.

%% \begin{figure}
%%   \begin{mathpar}
%%     \inferrule*[right = Reflexivity]
%%     {\hasty \Gamma V A}
%%     {\refl(\Gamma) \vdash \refl(A)(V, V)}

%%     \inferrule*[right = Profunctoriality]
%%     { \refl(\Gamma^{\rel}_l) \vdash  \refl(A^{\rel}_l) (V_l' , V_l) \\\\ 
%%         \Gamma^{\rel}    \vdash    A^{\rel}    (V_l  , V_r) \\\\
%%       \refl(\Gamma^{\rel}_r) \vdash  \refl(A^{\rel}_r) (V_r  , V_r')
%%     }
%%     {\Gamma^{\rel} \vdash A^{\rel} (V_l', V_r')}

%%     \inferrule*[right = Subst]
%%     { \Gamma'^{\rel} \vdash \Gamma^{\rel} (\gamma_l, \gamma_r) \\\\
%%       \Gamma^{\rel} A^{\rel} (V_l, V_r)
%%     }
%%     {\Gamma'^{\rel} \vdash A^{\rel} (V_l[\gamma_l] , V_r[\gamma_r]) }

%%     % \inferrule*[right = TermSubst]
%%     % { \Delta'^{\rel} \vdash \Delta^{\rel} (\delta_l, \delta_r) \\\\
%%     %   \Delta^{\rel} B^{\rel} (M_l, M_r)
%%     % }
%%     % {\Delta'^{\rel} \vdash B^{\rel} (M_l[\delta_l] , M_r[\delta_r]) }

%%   \end{mathpar}
%%   \caption{Rules for value type relations. The rules for computation type relations are analogous.}
%%   \label{fig:relation-rules}
%% \end{figure}

%% We also have a rule for the restriction of a relation along a function,
%% and we have a rule characterizing relation at function type. The latter states that
%% if under the assumption that $x$ is related to $x'$ by $A^{\rel}$, we can show that $M$
%% is related to $M'$ by $\li A'^{\rel}$, then we have that $\lda{x}{M}$ is related to
%% $\lda{x'}{M'}$ by $A^{\rel} \ra A'^{\rel}$.

%% \begin{mathpar}
%%   \mprset{fraction={===}}

%%   % \inferrule*[]
%%   % { A^{\rel}  (x_l, x_r) \vdash A^{\rel} (V_l, V_r) }
%%   % { A'^{\rel} (x_l, x_r) \vdash A^{\rel} (V_l, V_r)(x_l, x_r) }

%%   \inferrule*[right = Restriction]
%%   { \Gamma^{\rel} \vdash A^{\rel} (V_l (V_l'), V_r (V_r')) }
%%   { \Gamma^{\rel} \vdash (A^{\rel} (V_l, V_r)) (V_l', V_r') }

%%   \inferrule*[right = $\text{Rel}_\ra$]
%%   { A^{\rel} (x, x') \vdash (\li A'^{\rel})(M , M') }
%%   {  \vdash (A^{\rel} \ra A'^{\rel}) (\lda{x}{M}) , (\lda{x'}{M'})}

%% \end{mathpar}



%% % New rules
%% Figure \ref{fig:extlc-minus-minus-typing} shows the new typing rules,
%% and Figure \ref{fig:extlc-minus-minus-eqns} shows the equational rules
%% for case-nat (the rules for case-arrow are analogous).

%% \begin{figure}
%%   \begin{mathpar}
%%       % inj-nat
%%       \inferrule*
%%       {\hasty \Gamma M \nat}
%%       {\hasty \Gamma {\injnat M} \dyn}

%%       % inj-arr 
%%       \inferrule*
%%       {\hasty \Gamma M (\dyn \ra \dyn)}
%%       {\hasty \Gamma {\injarr M} \dyn}

%%       % Case nat
%%       \inferrule*
%%       {\hasty{\Delta|_V}{V}{\dyn} \and 
%%         \hasty{\Delta , x : \nat }{M_{yes}}{B} \and 
%%         \hasty{\Delta}{M_{no}}{B}}
%%       {\hasty {\Delta} {\casenat{V}{M_{no}}{n}{M_{yes}}} {B}}
    
%%       % Case arr
%%       \inferrule*
%%       {\hasty{\Delta|_V}{V}{\dyn} \and 
%%         \hasty{\Delta , x : (\dyn \ra \dyn) }{M_{yes}}{B} \and 
%%         \hasty{\Delta}{M_{no}}{B}}
%%       {\hasty {\Delta} {\casearr{V}{M_{no}}{f}{M_{yes}}} {B}}
%%   \end{mathpar}
%%   \caption{New typing rules for $\extlcmm$.}
%%   \label{fig:extlc-minus-minus-typing}
%% \end{figure}


%% \begin{figure}
%%   \begin{mathpar}
%%      % Case-nat Beta
%%      \inferrule*
%%      {\hasty \Gamma V \nat}
%%      {\casenat {\injnat {V}} {M_{no}} {n} {M_{yes}} = M_{yes}[V/n]}

%%      \inferrule*
%%      {\hasty \Gamma V {\dyn \ra \dyn} }
%%      {\casenat {\injarr {V}} {M_{no}} {n} {M_{yes}} = M_{no}}

%%      % Case-nat Eta
%%      \inferrule*
%%      {}
%%      {\Gamma , x :\, \dyn \vdash M = \casenat{x}{M}{n}{M[(\injnat{n}) / x]} }


%%      % Case-arr Beta


%%      % Case-arr Eta


%%   \end{mathpar}
%%   \caption{New equational rules for $\extlcmm$ (rules for case-arrow are analogous
%%            and hence are omitted).}
%%   \label{fig:extlc-minus-minus-eqns}
%% \end{figure}



%% % TODO : Updated term precision rules



%% \subsection{The Step-Sensitive Lambda Calculus}\label{sec:step-sensitive-lc}

%% % \textbf{TODO: Subject to change!}

%% Rather than give a semantics to $\extlcmm$ directly, we first introduce another intermediary
%% language, a \emph{step-sensitive} (also called \emph{intensional}) calculus.
%% As mentioned, this language makes the intensional stepping behavior of programs
%% explicit in the syntax. We do this by adding a syntactic ``later'' type and a
%% syntactic $\theta$ that takes terms of type later $A$ to terms of type $A$.

%% % In the step-sensitive syntax, we add a type constructor for later, as well as a
%% % syntactic $\theta$ term and a syntactic $\nxt$ term.
%% We add rules for these new constructs, and also modify the rules for inj-arr and
%% case-arrow, since now the function is not $\Dyn \ra \Dyn$ but rather $\later (\Dyn \ra \Dyn)$.
%% We also add congruence relations for $\later$ and $\nxt$.

%% % TODO show changes

%% \noindent Modified syntax:
%% \begin{align*}
%%   &\text{Value Types } A := \nat \alt \dyn \alt (A \ra A') \alt {\color{red} \later A} \\
%%   &\text{Values } V :=  \zro \alt \suc\, V \alt \lda{x}{M} \alt \injnat V \alt \injarr V 
%%     \alt {\color{red} \nxt\, V} \alt {\color{red} \mathbf{\theta}}
%% \end{align*}

%% \noindent Additional typing rules:
%% \begin{mathpar}
%%   \inferrule
%%   {\hasty \Gamma V A}
%%   {\hasty \Gamma {\nxt\, V} {\later A}}

%%   \inferrule
%%   {}
%%   {\hasty \Gamma \theta {\later A \ra A}}

%%   % \theta(\nxt x) = \theta(y); \texttt{ret}\, x
%% \end{mathpar}

%% \noindent Modified typing rules:
%% \begin{mathpar}

%%   % inj-arr 
%%   \inferrule*
%%   {\hasty \Gamma M {\color{red} \later (\dyn \ra \dyn)}}
%%   {\hasty \Gamma {\injarr M} \dyn}

%%   % Case arr
%%   % TODO if the extensional version is incorrect and needs to change, make
%%   % sure to change this one accordingly
%%   \inferrule*
%%   {\hasty{\Delta|_V}{V}{\dyn} \and 
%%     \hasty{\Delta , x \colon {\color{red} \later (\dyn \ra \dyn)} }{M_{yes}}{B} \and 
%%     \hasty{\Delta}{M_{no}}{B}}
%%   {\hasty {\Delta} {\casearr{V}{M_{no}}{\tilde{f}}{M_{yes}}} {B}}  
%% \end{mathpar}

%% \noindent Additional relations:
%% \begin{mathpar}
%%   \inferrule*[]
%%   {A^{\rel} : A_l \rel A_r}
%%   {\later A^{\rel} : \later A_l \rel \later A_r}

%%   \inferrule*[]
%%   {A^{\rel} (V_l, V_r)}
%%   {\later A^{\rel} (\nxt\, V_l, \nxt\, V_r)}

%% \end{mathpar}

%% % TODO what about the relation for theta? Or is that automatic since it's a function symbol?

%% % TODO beta rule for theta

%% We define the term $\delta$ to be the function $\lda {x} {\theta\, (\nxt\, x)}$.

%% % We define an erasure function from step-sensitive syntax to step-insensitive syntax
%% % by induction on the step-sensitive types and terms.
%% % The basic idea is that the syntactic type $\later A$ erases to $A$,
%% % and $\nxt$ and $\theta$ erase to the identity.



%% \subsection{Quotienting by Syntactic Bisimilarity}

%% We now define a quotiented variant of the above step-sensitive calculus,
%% which we denote by $\intlcbisim$.
%% In this syntax, we add a rule saying, roughly speaking, that 
%% $\theta \circ \nxt$ is the identity. This causes terms that differ only in
%% their intensional behavior to become equal.
%% Note that a priori, this is not the same language as the step-insensitive
%% calculus on which we based the insensitive calculus.

%% Formally, the equational theory for the quotiented syntax is the same as
%% that of the original step-sensitive language, with the addition of the following
%% rule:

%% % TODO is this correct?
%% \begin{mathpar}
%%   \inferrule*
%%   { }
%%   { \theta\, (\nxt\, x) = \bind{y}{(\theta\, V')}{\ret\, x}  }
%% \end{mathpar}

%% This states that the application of $\theta$ to $\nxt\, x$ is equivalent to
%% the computation that applies $\theta$ to $V'$ to obtain a variable $y$, and
%% then simply returns $x$.




% \section{Denotational Semantics}

First, we define a denotational semantics of types and terms of the
cast calculus by giving a standard monadic denotational semantics in
the cartesian closed category of preorders and monotone functions,
extended to model the primitives of gradual typing: the dynamic type,
errors and type casts. The most interesting part of this semantics is
the construction of the monad and the dynamic type.



\section{Domain-Theoretic Constructions}\label{sec:domain-theory}

In this section, we discuss the fundamental objects of the model into which we will embed
the step-sensitive lambda calculus $\intlc$ and inequational theory. It is important to remember that
the constructions in this section are entirely independent of the syntax described in the
previous section; the notions defined here exist in their own right as purely mathematical
constructs. In the next section, we will link the syntax and semantics via a semantic interpretation
function.

\subsection{The Lift Monad}

When thinking about how to model intensional gradually-typed programs, we should consider
their possible behaviors. On the one hand, we have \emph{failure}: a program may fail
at run-time because of a type error. In addition to this, a program may ``think'',
i.e., take a step of computation. If a program thinks forever, then it never returns a value,
so we can think of the idea of thinking as a way of intensionally modelling \emph{partiality}.

With this in mind, we can describe a semantic object that models these behaviors: a monad
for embedding computations that has cases for failure and ``thinking''.
Previous work has studied such a construct in the setting of the latter, called the lift
monad \cite{mogelberg-paviotti2016}; here, we augment it with the additional effect of failure.

For a type $A$, we define the \emph{lift monad with failure} $\li A$, which we will just call
the \emph{lift monad}, as the following datatype:

\begin{align*}
  \li A &:= \\
  &\eta \colon A \to \li A \\
  &\mho \colon \li A \\
  &\theta \colon \later (\li A) \to \li A
\end{align*}

Unless otherwise mentioned, all constructs involving $\later$ or $\fix$
are understood to be with respect to a fixed clock $k$. So for the above, we really have for each
clock $k$ a type $\li^k A$ with respect to that clock.

Formally, the lift monad $\li A$ is defined as the solution to the guarded recursive type equation

\[ \li A \cong A + 1 + \later \li A. \]

An element of $\li A$ should be viewed as a computation that can either (1) return a value (via $\eta$),
(2) raise an error and stop (via $\mho$), or (3) think for a step (via $\theta$).
%
Notice there is a computation $\fix \theta$ of type $\li A$. This represents a computation
that thinks forever and never returns a value.

Since we claimed that $\li A$ is a monad, we need to define the monadic operations
and show that they respect the monadic laws. The return is just $\eta$, and extend
is defined via by guarded recursion by cases on the input.
% It is instructive to give at least one example of a use of guarded recursion, so
% we show below how to define extend:
% TODO
%
%
Verifying that the monadic laws hold requires \lob-induction and is straightforward.

\begin{comment}
% Since this mentions the ordering on B, it should be introduced after introducing predomains.
The lift monad has the following universal property. Let $f$ be a function from $A$ to $B$,
where $B$ is a $\later$-algebra, i.e., there is $\theta_B \colon \later B \to B$.
Further suppose that $B$ is also an ``error-algebra'', that is, there is an error element
$\mho_B$ such that $\mho_B \le_B y$ for all $y \in B$.

% Further suppose that $B$ is also an algebra of the
% constant functor $1 \colon \text{Type} \to \text{Type}$ mapping all types to Unit.
% This latter statement amounts to saying that there is a map $\text{Unit} \to B$, so $B$ has a
% distinguished ``error element" $\mho_B \colon B$ such that $\mho_B \le_B y$ for all $y \in B$.

Then there is a unique homomorphism of algebras $f' \colon \li A \to B$ such that
$f' \circ \eta = f$. The function $f'(l)$ is defined via guarded fixpoint by cases on $l$. 
In the $\mho$ case, we simply return $\mho_B$.
In the $\theta(\tilde{l})$ case, we will return

\[\theta_B (\lambda t . (f'_t \, \tilde{l}_t)). \]

Recalling that $f'$ is a guarded fixpoint, it is available ``later'' and by
applying the tick we get a function we can apply ``now''; for the argument,
we apply the tick to $\tilde{l}$ to get a term of type $\li A$.
\end{comment}

%\subsubsection{Model-Theoretic Description}
%We can describe the lift monad in the topos of trees model as follows.


\subsection{Predomains}\label{sec:predomains}

The next important construction is that of a \emph{predomain}. A predomain is intended to
model the notion of error ordering that we want terms to have. Thus, we define a predomain $A$
as a partially-ordered set, which consists of a type which we denote $\ty{A}$ and a reflexive,
transitive, and antisymmetric relation $\le_P$ on $A$.

We define monotone functions between predomain as expected. Given predomains
$A$ and $B$, we write $f \colon A_i \monto A_o$ to indicate that $f$ is a monotone
function from $A_i$ to $A_o$, i.e, for all $a_1 \le_{A_i} a_2$, we have $f(a_1) \le_{A_o} f(a_2)$.
We also define an ordering on monotone functions as
$f \le g$ if for all $a$ in $\ty{A_i}$, we have $f(a) \le_{A_o} g(a)$.

For each type we want to represent, we define a predomain for the corresponding semantic
type. For instance, we define a predomain for natural numbers, a predomain for the
dynamic type, a predomain for functions, and a predomain for the lift monad. We
describe each of these below.

\begin{itemize}
  \item There is a predomain $\Nat$ for natural numbers, where the ordering is equality.
  
  \item There is a predomain $\Dyn$ to represent the dynamic type. The underlying type
  for this predomain is defined by guarded fixpoint to be such that
  $\ty{\Dyn} \cong \mathbb{N}\, +\, \later (\ty{\Dyn} \monto \li \ty{\Dyn})$.
  This definition is valid because the occurrences of Dyn are guarded by the $\later$.
  The ordering is defined via guarded recursion by cases on the argument, using the
  ordering on $\mathbb{N}$ and the ordering on monotone functions described above.

  \item For a predomain $A$, there is a predomain $\li A$ for the ``lift'' of $A$
  using the lift monad. We use the same notation for $\li A$ when $A$ is a type
  and $A$ is a predomain, since the context should make clear which one we are referring to.
  The underling type of $\li A$ is simply $\li \ty{A}$, i.e., the lift of the underlying
  type of $A$.
  The ordering on $\li A$ is the ``step-sensitive error-ordering'' which we describe in
  \ref{subsec:lock-step}.

  \item For predomains $A_i$ and $A_o$, we form the predomain of monotone functions
  from $A_i$ to $A_o$, which we denote by $A_i \To A_o$.

  \item Given a preomain $A$, we can form the predomain $\later A$ whose underlying
  type is $\later \ty{A}$. We define $\tilde{x} \le_{\later A} \tilde{y}$ to be
  $\later_t (\tilde{x}_t \le_A \tilde{y}_t)$.
\end{itemize}



\subsection{Step-Sensitive Error Ordering}\label{subsec:lock-step}

As mentioned, the ordering on the lift of a predomain $A$ is called the
\emph{step-sensitive error-ordering} (also called ``lock-step error ordering''),
the idea being that two computations $l$ and $l'$ are related if they are in
lock-step with regard to their intensional behavior, up to $l$ erroring.
Formally, we define this ordering as follows:

\begin{itemize}
  \item 	$\eta\, x \ltls \eta\, y$ if $x \le_A y$.
  \item 	$\mho \ltls l$ for all $l$ 
  \item   $\theta\, \tilde{r} \ltls \theta\, \tilde{r'}$ if
          $\later_t (\tilde{r}_t \ltls \tilde{r'}_t)$
\end{itemize}

We also define a heterogeneous version of this ordering between the lifts of two
different predomains $A$ and $B$, parameterized by a relation $R$ between $A$ and $B$.

\subsection{Step-Insensitive Bisimilarity Relation}

We define another ordering on $\li A$, called ``step-insensitive bisimilarity"
or ``weak bisimilarity" written $l \bisim l'$.
Intuitively, we say $l \bisim l'$ if they are equivalent ``up to delays''.
We introduce the notation $x \sim_A y$ to mean $x \le_A y$ and $y \le_A x$.
% TODO if A is a poset, then we can just say that x = y
%
The step-insensitive bisimilarity relation is defined by guarded fixpoint as follows:

\begin{align*}
  &\mho \bisim \mho \\
%
  &\eta\, x \bisim \eta\, y \text{ if } 
    x \sim_A y \\
%		
  &\theta\, \tilde{x} \bisim \theta\, \tilde{y} \text{ if } 
    \later_t (\tilde{x}_t \bisim \tilde{y}_t) \\
%	
  &\theta\, \tilde{x} \bisim \mho \text{ if } 
    \theta\, \tilde{x} = \delta^n(\mho) \text { for some $n$ } \\
%	
  &\theta\, \tilde{x} \bisim \eta\, y \text{ if }
    (\theta\, \tilde{x} = \delta^n(\eta\, x))
  \text { for some $n$ and $x : \ty{A}$ such that $x \sim_A y$ } \\
%
  &\mho \bisim \theta\, \tilde{y} \text { if } 
    \theta\, \tilde{y} = \delta^n(\mho) \text { for some $n$ } \\
%	
  &\eta\, x \bisim \theta\, \tilde{y} \text { if }
    (\theta\, \tilde{y} = \delta^n (\eta\, y))
  \text { for some $n$ and $y : \ty{A}$ such that $x \sim_A y$ }
\end{align*}

When both sides are $\eta$, then we ensure that the underlying values are related.
When one side is a $\theta$ and the other is $\eta x$ (i.e., one side steps),
we stipulate that the $\theta$-term runs to $\eta y$ where $x$ is related to $y$.
Similarly when one side is $\theta$ and the other $\mho$.
If both sides step, then we allow one time step to pass and compare the resulting terms.
In this way, the definition captures the intuition of terms being equivalent up to
delays.

It can be shown (by \lob-induction) that the step-sensitive relation is symmetric.
However, it can also be shown that this relation is \emph{not} transitive:
One can prove within Clocked Cubical Type Theory
that if this relation were transitive, then in fact it would be trivial in that
$l \bisim l'$ for all $l, l'$.
This issue will be resolved when we consider the relation's \emph{globalization}.



\subsection{Error Domains}

While value types will be interpreted as predomains, we also need a semantics
for computation types. This will be in the form of \emph{error domains}, of which the
Lift monad is a prototypical example. For a fixed clock $k$, an error domain $A$
consists of a predomain (which we also denote by $A$ when there is no risk of confusion),
along with a bottom element $\mho_A$ and a $\later$-algebra $\theta_A \colon \later^k A \monto A$.

% TODO function space as error domain?

\subsection{Globalization}\label{sec:globalization}

Recall that in the above definitions, any occurrences of $\later$ were with
respect to a fixed clock $k$. Intuitively, this corresponds to a step-indexed set.
It will be necessary to consider the ``globalization'' of these definitions,
i.e., the ``global'' behavior of the type over all potential time steps.
This is accomplished in the type theory by \emph{clock quantification} \cite{atkey-mcbride2013},
whereby given a type $X$ parameterized by a clock $k$, we consider the type
$\forall k. X[k]$. This corresponds to leaving the step-indexed world and passing to
the usual semantics in the category of sets.


\section{Semantics}\label{sec:semantics}

\subsection{Step-indexed Semantics}

We give a semantics to the step-sensitive lambda calculus $\intlc$ we defined
in Section \ref{sec:step-sensitive-lc}.
%
Much of the semantics is similar to a normal call-by-value denotational semantics;
we highlight the differences.
Recall that we will interpret value types as predomains, and computation types
as error domains. Value type contexts $\Gamma = x_1 \colon A_1, \dots, x_n \colon A_n$
will be interpreted as the product $\sem{A_1} \times \cdots \times \sem{A_n}$, and
computation type contexts $\Delta = \Delta_\Sigma , \Delta|_V$ will be interpreted as a pair
$(\delta_\Sigma, \delta_V)$ where $\delta_\Sigma$ is either empty or $\sem{B}$.


The semantics of the dynamic type $\dyn$ is the predomain $\Dyn$ introduced in Section
\ref{sec:predomains}.
%
The interpretation of a value $\hasty {\Gamma} V A$ will be a monotone function from
$\sem{\Gamma}$ to $\sem{A}$. Likewise, a term $\hasty {\Delta} M {\Ret{A}}$ will be interpreted
as a monotone function from $\sem{\Delta}$ to $\sem{\Ret{A}} = \li \sem{A}$.

Recall that $\Dyn$ is isomorphic to $\Nat\, + \later (\Dyn \monto \li \Dyn)$.
Thus, the semantics of $\injnat{\cdot}$ and $\injarr{\cdot}$ are simply the
injections $\inl$ and $\inr$.

The interpretation of $\lda{x}{M}$ works as follows. Recall by the typing rule for
lambda that $\hasty {\cdot, \Gamma, x : A_i} M {\Ret {A_o}}$, so the interpretation of $M$
has type $\{*\} \times (\sem{\Gamma} \times \sem{A_i})$ to $\sem{A_o}$.
The interpretation of lambda is thus a function (in the ambient type theory) that takes
a value $a$ representing the argument and applies it (along with $\gamma$) as argument to
the interpretation of $M$.
%
The interpretation of bind and of application both make use the monadic extend function on $\li A$.
%
The interpretation of case-nat and case-arrow is simply a case inspection on the
interpretation of the scrutinee, which has type $\Dyn$.


\vspace{2ex}


\noindent Types:
\begin{align*}
  \sem{\nat} &= \Nat \\
  \sem{\dyn} &= \Dyn \\
  \sem{A \ra A'} &= \sem{A} \To \sem{A'} \\
  \sem{\later A} &=\, \later \sem{A} \\
  \sem{\Ret A} &= \li \sem{A}
\end{align*}

% Contexts:

% TODO check these, especially the semantics of bind, case-nat, and case-arr
% with respect to their context argument
\noindent Values and terms:
\begin{align*}
  \sem{\zro}         &= \lambda \gamma . 0 \\
  \sem{\suc\, V}     &= \lambda \gamma . (\sem{V}\, \gamma) + 1 \\
  \sem{x \in \Gamma} &= \lambda \gamma . \gamma(x) \\
  \sem{\lda{x}{M}}   &= \lambda \gamma . \lambda a . \sem{M}\, (*,\, (\gamma , a))  \\
  \sem{\injnat{V_n}} &= \lambda \gamma . \inl\, (\sem{V_n}\, \gamma) \\
  \sem{\injarr{V_f}} &= \lambda \gamma . \inr\, (\sem{V_f}\, \gamma) \\[2ex]
  \sem{\nxt\, V}     &= \lambda \gamma . \nxt (\sem{V}\, \gamma) \\
  \sem{\theta}       &= \lambda \gamma . \theta \\
%
  \sem{\err_B}         &= \lambda \delta . \mho \\
  \sem{\ret\, V}       &= \lambda \gamma . \eta\, \sem{V} \\
  \sem{\bind{x}{M}{N}} &= \lambda \delta . \ext {(\lambda x . \sem{N}\, (\delta, x))} {\sem{M}\, \delta} \\
  \sem{V_f\, V_x}      &= \lambda \gamma . \ext {(\lambda f . (\ext {f} {\sem{V_x}\, \gamma}))} {(\sem{V_f}\, \gamma)} \\
  \sem{\casenat{V}{M_{no}}{n}{M_{yes}}}         &= 
    \lambda \delta . \text{case $(\sem{V}\, \delta)$ of} \\ 
    &\quad\quad\quad\quad \alt \inl(n) \to \sem{M_{yes}}(n) \\
    &\quad\quad\quad\quad \alt \inr(\tilde{f}) \to \sem{M_{no}} \\
  \sem{\casearr{V}{M_{no}}{\tilde{f}}{M_{yes}}} &= 
    \lambda \delta . \text{case $(\sem{V}\, \delta)$ of} \\ 
    &\quad\quad\quad\quad \alt \inl(n) \to \sem{M_{no}} \\
    &\quad\quad\quad\quad \alt \inr(\tilde{f}) \to \sem{M_{yes}}(\tilde{f})
\end{align*}

% TODO
% \noindent Relations:
% \begin{align*}
% %
% \end{align*}


\begin{comment}
\subsection{Global Semantics}

Having defined the above step-indexed semantics, we now pass to a ``global''
semantics that does not involve any step-indexing. The resulting semantics is still
intensional in that terms that produce the same value in a different number of steps
will be distinct.
We define 

\[ \semgl{\cdot} = \forall (\kpa \colon \Clock) .\, \sem{\cdot}[\kpa]. \]

Note that for a term $M$ of type $\Ret{A}$, the semantics has type
$\sem{M} \colon \sem{\Delta} \monto \sem{\Ret{A}} = \sem{\Delta} \monto \li \sem{A}$.
In the case where $\Delta$ is the empty context, i.e., when $M$ is a closed term,
then this is equivalent to $\li \sem{A}$.
Then the global semantics in this case is $\forall \kappa . \liclk {\kappa} \sem{A}$.
We can show in Clocked Cubical Type Theory this type satisfies a coinductive unfolding property

\[ \forall \kappa . \li \sem{A} \cong \sem{A} + 1\, + (\forall \kappa.\li \sem{A}). \]
\end{comment}


% Machines

% We then define a relation $\Dwn^n$ between terms of type $T$ and $\Machine {\sem{T}}$ by

% \subsection{Extensional Collapse}


% \subsection{Relational Semantics}

% \subsubsection{Term Precision via the Step-Sensitive Error Ordering}
% Homogeneous vs heterogeneous term precision

% \subsection{Logical Relations Semantics}


\section{A Simple Denotational Semantics for the Terms of GTLC}\label{sec:gtlc-terms}

In this section, we introduce the term syntax for the gradually-typed
lambda calculus (GTLC) and give a set-theoretic denotational semantics
using tools from SGDT. This serves two purposes: First, it is a simple setting
in which to employ the tools of SGDT.
Second, constructing this semantic model establishes the validity of the
beta and eta principles for the gradually-typed lambda calculus.

In Section \ref{sec:gtlc-precision}, we will discuss how to extend the denotational
semantics to accommodate the type and term precision orderings.


\subsection{Syntax}\label{sec:term-syntax}

Our syntax is based on fine-grained call by value, and as such it has
separate value and producer terms and typing judgments for each.

% Given a term $M$ of type $A$, the term $\bind{x}{M}{N}$ should be thought of as
% running $M$ to a value $V$ and then continuing as $N$, with $V$ in place of $x$.


\begin{align*}
  &\text{Types } A := \nat \alt \,\dyn \alt (A \ra A') \\
  &\text{Contexts } \Gamma := \cdot \alt (\Gamma, x : A) \\
  &\text{Values } V :=  \zro \alt \suc\, V \alt \lda{x}{M} \alt \up{A}{B} V \\ 
  &\text{Producers } M, N := \err_B \alt \matchnat {V} {M} {n} {M'} \\ 
  &\quad\quad \alt \ret {V} \alt \bind{x}{M}{N} \alt V_f\, V_x \alt \dn{A}{B} M 
\end{align*}


The value typing judgment is written $\vhasty{\Gamma}{V}{A}$ and 
the producer typing judgment is written $\phasty{\Gamma}{M}{A}$.

The typing rules are as expected, with a cast between $A$ to $B$ allowed only when $A \ltdyn B$.
The precise rules for $A \ltdyn B$ will be given below.
Notice that the upcast of a value is a value, since it always succeeds, while the downcast
of a value is a producer, since it may fail.

\begin{mathpar}
    % Var
    \inferrule*{ }{\vhasty {\cdot, \Gamma, x : A, \Gamma'} x A}

    % Err
    \inferrule*{ }{\phasty {\cdot, \Gamma} {\err_A} A} 
  
    % Zero and suc
    \inferrule*{ }{\vhasty \Gamma \zro \nat}
  
    \inferrule*{\vhasty \Gamma V \nat} {\vhasty \Gamma {\suc\, V} \nat}

    % Match-nat
    \inferrule*
    {\vhasty \Gamma V \nat \and 
     \phasty \Delta M A \and \phasty {\Gamma, n : \nat} {M'} A}
    {\phasty \Gamma {\matchnat {V} {M} {n} {M'}} A}
  
    % Lambda
    \inferrule* 
    {\phasty {\Gamma, x : A} M {A'}} 
    {\vhasty \Gamma {\lda x M} {A \ra A'}}
  
    % App
    \inferrule*
    {\vhasty \Gamma {V_f} {A \ra A'} \and \vhasty \Gamma {V_x} A}
    {\phasty {\Gamma} {V_f \, V_x} {A'}}

    % Ret
    \inferrule*
    {\vhasty \Gamma V A}
    {\phasty {\Gamma} {\ret\, V} {A}}

    % Bind
    \inferrule*
    {\phasty \Gamma M {A} \and \phasty{\Gamma, x : A}{N}{B} } % Need x : A in context
    {\phasty {\Gamma} {\bind{x}{M}{N}} {B}}

    % Upcast
    \inferrule*
    {A \ltdyn A' \and \vhasty \Gamma V A}
    {\vhasty \Gamma {\up A {A'} V} {A'} }

    \inferrule* % TODO is this correct?
    {A \ltdyn A' \and \phasty {\Gamma} {M} {A'}}
    {\phasty {\Gamma} {\dn A {A'} M} {A}}

\end{mathpar}


In the equational theory, we have $\beta$ and $\eta$ laws for function type,
as well a $\beta$ and $\eta$ law for bind.

\begin{mathpar}
  % Function Beta and Eta
  \inferrule*
  {\phasty {\Gamma, x : A} M {B} \and \vhasty \Gamma V A}
  {(\lda x M)\, V = M[V/x]}

  \inferrule*
  {\vhasty \Gamma V {A \ra A}}
  {\Gamma \vdash V = \lda x {V\, x}}

  % Ret Beta and Eta
  \inferrule*
  {}
  {(\bind{x}{\ret\, V}{N}) = N[V/x]}

  \inferrule*
  {\phasty {\Gamma} {M} {B}}
  {\bind{x}{M}{\ret x} = M}

  % Match-nat Beta
  \inferrule*
  {\phasty \Delta M A \and \phasty {\Gamma, n : \nat} {M'} A}
  {\matchnat{\zro}{M}{n}{M'} = M}

  \inferrule*
  {\vhasty \Gamma V \nat \and 
   \phasty \Gamma M B \and \phasty {\Gamma, n : \nat} {M'} B}
  {\matchnat{\suc\, V}{M}{n}{M'} = M'}

  % Match-nat Eta
  % This doesn't build in substitution
  \inferrule*
  {\hasty {\Gamma , x : \nat} M A}
  {M = \matchnat{x} {M[\zro / x]} {n} {M[(\suc\, n) / x]}}

\end{mathpar}

\subsubsection{Type Precision}\label{sec:type-precision}

The type precision rules specify what it means for a type $A$ to be more precise than $A'$.
We have reflexivity rules for $\dyn$ and $\nat$, as well as rules that $\nat$ is more precise than $\dyn$
and $\dyntodyn$ is more precise than $\dyn$.
We also have a congruence rule for function types stating that given $A_i \ltdyn A'_i$ and $A_o \ltdyn A'_o$, we can prove
$A_i \ra A_o \ltdyn A'_i \ra A'_o$. Note that precision is covariant in both the domain and codomain.
Finally, we can lift a relation on value types $A \ltdyn A'$ to a relation $\Ret A \ltdyn \Ret A'$ on
computation types.

\begin{mathpar}
  \inferrule*[right = \dyn]
    { }{\dyn \ltdyn\, \dyn}

  \inferrule*[right = \nat]
    { }{\nat \ltdyn \nat}

  \inferrule*[right = $\ra$]
    {A_i \ltdyn A'_i \and A_o \ltdyn A'_o }
    {(A_i \ra A_o) \ltdyn (A'_i \ra A'_o)}

  \inferrule*[right = $\textsf{Inj}_\nat$]
    { }{\nat \ltdyn\, \dyn}

  \inferrule*[right = $\textsf{Inj}_{\ra}$]
    { }
    {(\dyntodyn) \ltdyn\, \dyn}

  \inferrule*[right = $\injarr{}$]
    {(R \ra S) \ltdyn\, (\dyntodyn)}
    {(R \ra S) \ltdyn\, \dyn}

  
\end{mathpar}

We can prove that transitivity of type precision is admissible, i.e.,
if $A \ltdyn B$ and $B \ltdyn C$, then $A \ltdyn C$.

% Type precision derivations
Note that as a consequence of this presentation of the type precision rules, we
have that if $A \ltdyn A'$, there is a unique precision derivation that witnesses this.
As in previous work, we go a step farther and make these derivations first-class objects,
known as \emph{type precision derivations}.
Specifically, for every $A \ltdyn A'$, we have a derivation $c : A \ltdyn A'$ that is constructed
using the rules above. For instance, there is a derivation $\dyn : \dyn \ltdyn \dyn$, and a derivation
$\nat : \nat \ltdyn \nat$, and if $c_i : A_i \ltdyn A_i$ and $c_o : A_o \ltdyn A'_o$, then
there is a derivation $c_i \ra c_o : (A_i \ra A_o) \ltdyn (A'_i \ra A'_o)$. Likewise for
the remaining rules. The benefit to making these derivations explicit in the syntax is that we
can perform induction over them.
Note also that for any type $A$, we use $A$ to denote the reflexivity derivation that $A \ltdyn A$,
i.e., $A : A \ltdyn A$.
Finally, observe that for type precision derivations $c : A \ltdyn A'$ and $c' : A' \ltdyn A''$, we
can define their composition $c \relcomp c' : A \ltdyn A''$.
This notion will be used below in the statement of transitivity of the term precision relation.


\begin{comment}
\subsection{Removing Casts as Primitives}

% We now observe that all casts, except those between $\nat$ and $\dyn$
% and between $\dyntodyn$ and $\dyn$, are admissible, in the sense that
% we can start from $\extlcm$, remove casts except the aforementioned ones,
% and in the resulting language we will be able to derive the other casts.

We now observe that all casts, except those between $\nat$ and $\dyn$
and between $\dyntodyn$ and $\dyn$, are admissible.
That is, consider a new language ($\extlcprime$) in which
instead of having arbitrary casts, we have injections from $\nat$ and
$\dyntodyn$ into $\dyn$, and a case inspection on $\dyn$.
We claim that in $\extlcprime$, all of the casts present in $\extlc$ are derivable.
It will suffice to verify that casts for function type are derivable.
This holds because function casts are constructed inductively from the casts
of their domain and codomain. The base case is one of the casts involving $\nat$
or $\dyntodyn$ which are present in $\extlcprime$ as injections and case inspections.


The resulting calculus $\extlcprime$ now lacks arbitrary casts as a primitive notion:

%%%%%%%%%%%%%%%%%%%%%%%%%%%%%%%%%%%%%%%%%%%%%%
% TODO update

\begin{align*}
  &\text{Types } A := \nat \alt \dyn \alt (A \ra A') \\
  &\text{Contexts } \Gamma := \cdot \alt (\Gamma, x : A) \\
  &\text{Values } V :=  \zro \alt \suc\, V \alt \lda{x}{M} \alt \injnat V \alt \injarr V \\ 
  &\text{Producers } M, N := \err_B \alt \ret {V} \alt \bind{x}{M}{N}
    \alt V_f\, V_x \alt
    \\ & \quad\quad \casenat{V}{M_{no}}{n}{M_{yes}} 
    \alt \casearr{V}{M_{no}}{f}{M_{yes}}
\end{align*}


% New rules
Figure \ref{fig:extlc-minus-minus-typing} shows the new typing rules,
and Figure \ref{fig:extlc-minus-minus-eqns} shows the equational rules
for case-nat (the rules for case-arrow are analogous).

\begin{figure}
  \begin{mathpar}
      % inj-nat
      \inferrule*
      {\hasty \Gamma M \nat}
      {\hasty \Gamma {\injnat M} \dyn}

      % inj-arr 
      \inferrule*
      {\hasty \Gamma M (\dyntodyn)}
      {\hasty \Gamma {\injarr M} \dyn}

      % Case dyn
      \inferrule*
      {\hasty{\Delta|_V}{V}{\dyn} \and
        \hasty{\Delta , x : \nat }{M_{nat}}{B} \and 
        \hasty{\Delta , x : (\dyntodyn) }{M_{fun}}{B}
      }
      {\hasty {\Delta} {\casedyn{V}{n}{M_{nat}}{f}{M_{fun}}} {B}}
  \end{mathpar}
  \caption{New typing rules for $\extlcmm$.}
  \label{fig:extlc-minus-minus-typing}
\end{figure}


\begin{figure}
  \begin{mathpar}
     % Case-dyn Beta
     \inferrule*
     {\hasty \Gamma V \nat}
     {\casedyn {\injnat {V}} {n} {M_{nat}} {f} {M_{fun}} = M_{nat}[V/n]}

     \inferrule*
     {\hasty \Gamma V {\dyntodyn} }
     {\casedyn {\injarr {V}} {n} {M_{nat}} {f} {M_{fun}} = M_{fun}[V/f]}

     % Case-dyn Eta
     \inferrule*
     {}
     {\Gamma , x :\, \dyn \vdash M = \casedyn{x}{n}{M[(\injnat{n}) / x]}{f}{M[(\injarr{f}) / x]} }


  \end{mathpar}
  \caption{New equational rules for $\extlcprime$ (rules for case-arrow are analogous
           and hence are omitted).}
  \label{fig:extlc-minus-minus-eqns}
\end{figure}

\end{comment}


% \section{Term Semantics}\label{sec:term-semantics}

\subsection{Semantic Constructions}\label{sec:domain-theory}

In this section, we discuss the fundamental objects of the model into which we will embed
the terms of the gradually-typed lambda calculus.
It is important to remember that the constructions in this section are entirely
independent of the syntax described in the previous section; the notions defined 
here exist in their own right as purely mathematical constructs.
In Section \ref{sec:term-interpretation}, we will link the syntax and semantics
via a semantic interpretation function.


\subsection{Modeling the Dynamic Type}

When modeling the dynamic type $\dyn$, we need a semantic object $D$ that satisfies the
isomorphism

\[ D \cong \Nat + (D \to (D + 1)). \]

where the $D + 1$ represents the fact that in the function case, the function may return an error.
Unfortunately, this equation does not have inductive or coinductive solutions. The usual way of
dealing with such equations is via domain theory, by which we can obtain an exact solution.
However, the heavy machinery of domain theory can be difficult for language designers to learn
and apply in the mechanized setting.
Instead, we will leverage the tools of guarded type theory, considering instead the following
similar-looking equation:

\[ D \cong \Nat + \later (D \to (D + 1)). \]

Since the negative occurrence of $D$ is guarded under a later, this equation has a (guarded) solution.
Specifically, we consider the following function $f$ of type
$\later \type \to \type$:

\[ \lambda (D' : \later \type) . \Nat + \later_t (D'_t \to (D'_t + 1)). \]

(Recall that the tick $t : \tick$ is evidence that time has passed, and since
$D'$ has type $\later \type$, i.e. $\tick \to \type$, then $D'_t$ has type $\type$.)

Then we define 

\[ D = \fix f. \]

% TODO explain better
As it turns out, this definition is not quite correct, as it doesn't provide a way to
model functions that are potentially non-terminating.
Another way to think about this is that by using guarded recursion to solve the
equation for the dynamic type, the solution to the equation involves a notion of
``time" or ``steps".
So, in addition to returning a value or erroring, programs may now take one or
more observable steps of computation, and this possibility must be reflected in
in the equation for the dynamic type.

% Therefore, the semantics of terms will need to allow for terms that potentially
% do not terminate. This is accomplished using an extension of the guarded lift monad,
% which we describe in the next section.
More specifically, in the equation for the semantics of $\dyn$, the result of the
function should be a computation that may return a value, error, \emph{or} take an observable step.
We model such computations using an extension of the so-called guarded lift monad
\cite{mogelberg-paviotti2016} which we describe in the next section.
We will then use this to give the correct definition of the semantics of the dynamic type.

\subsubsection{The Lift + Error Monad}\label{sec:lift-monad}

% TODO ensure the previous section flows into this one
When thinking about how to model gradually-typed programs where we track the steps they take,
we should consider their possible behaviors. On the one hand, we have \emph{failure}: a program may fail
at run-time because of a type error. In addition to this, a program may take one or more steps of computation.
If a program steps forever, then it never returns a value,
so we can think of the idea of stepping as a way of intensionally modelling \emph{partiality}.

With this in mind, we can describe a semantic object that models these behaviors: a monad
for embedding computations that has cases for failure and ``stepping''.
Previous work has studied such a construct in the setting of the latter, called the lift
monad \cite{mogelberg-paviotti2016}; here, we augment it with the additional effect of failure.

For a type $A$, we define the \emph{lift monad with failure} $\li A$, which we will just call
the \emph{lift monad}, as the following datatype:

\begin{align*}
  \li A &:= \\
  &\eta \colon A \to \li A \\
  &\mho \colon \li A \\
  &\theta \colon \later (\li A) \to \li A
\end{align*}

Unless otherwise mentioned, all constructs involving $\later$ or $\fix$
are understood to be with respect to a fixed clock $k$. So for the above, we really have for each
clock $k$ a type $\li^k A$ with respect to that clock.

Formally, the lift monad $\li A$ is defined as the solution to the guarded recursive type equation

\[ \li A \cong A + 1 + \later \li A. \]

An element of $\li A$ should be viewed as a computation that can either (1) return a value (via $\eta$),
(2) raise an error and stop (via $\mho$), or (3) take a step (via $\theta$).
%
Notice there is a computation $\fix \theta$ of type $\li A$. This represents a computation
that runs forever and never returns a value.

Since we claimed that $\li A$ is a monad, we need to define the monadic operations
and show that they respect the monadic laws. The return is just $\eta$, and extend
is defined via guarded recursion by cases on the input.
% It is instructive to give at least one example of a use of guarded recursion, so
% we show below how to define extend:
% TODO
%
%
Verifying that the monadic laws hold uses \lob-induction and is straightforward.

%\subsubsection{Model-Theoretic Description}
%We can describe the lift monad in the topos of trees model as follows.

\subsubsection{Revisiting the Dynamic Type}
Now we can state the correct definition of the semantics for the dynamic type.
The set $D$ is defined to be the solution to the guarded equation

\[ D \cong \Nat + \later (D \to \textcolor{red}{\li} D). \]


\subsection{Interpretation}\label{sec:term-interpretation}

We can now give a semantics to the \emph{terms} of the gradual lambda calculus we defined
above. The full definition is given in Figure \ref{fig:term-semantics}.
%
Much of the semantics is similar to a normal call-by-value denotational semantics:
We will interpret types as sets, and terms as functions.
Contexts $\Gamma = x_1 \colon A_1, \dots, x_n \colon A_n$
will be interpreted as the product $\sem{A_1} \times \cdots \times \sem{A_n}$.


The semantics of the dynamic type $\dyn$ is the set $\Dyn$ introduced in Section
\ref{sec:predomains}.
%
The interpretation of a value $\vhasty {\Gamma} V A$ will be a function from
$\sem{\Gamma}$ to $\sem{A}$. Likewise, a term $\phasty {\Gamma} M {{A}}$ will be interpreted
as a function from $\sem{\Gamma}$ to $\li \sem{A}$.

Recall that $\Dyn$ is isomorphic to $\Nat\, + \later (\Dyn \to \li \Dyn)$.
Thus, the semantics of $\injnat{\cdot}$ is simply $\inl$ and the semantics
of $\injarr{\cdot}$ is simply $\inr \circ \nxt$.
The semantics of case inspection on dyn performs a case analysis on the sum.

The interpretation of $\lda{x}{M}$ works as follows. Recall by the typing rule for
lambda that $\phasty {\Gamma, x : A_i} M {{A_o}}$, so the interpretation of $M$
has type $(\sem{\Gamma} \times \sem{A_i})$ to $\li \sem{A_o}$.
The interpretation of lambda is thus a function (in the ambient type theory) that takes
a value $a$ representing the argument and applies it (along with $\gamma$) as argument to
the interpretation of $M$.
%
The interpretation of bind makes use the monadic extend function on $\li A$.
%
The interpretation of case-nat and case-arrow is simply a case inspection on the
interpretation of the scrutinee, which has type $\Dyn$.


\vspace{2ex}


\begin{figure*}
  \noindent Types:
  \begin{align*}
    \sem{\nat} &= \Nat \\
    \sem{\dyn} &= \Dyn \\
    \sem{A \ra A'} &= \sem{A} \To \li \sem{A'} \\
  \end{align*}

  % Contexts:

  % TODO check these, especially the semantics of bind, case-nat, and case-arr
  % with respect to their context argument
  \noindent Values and terms:
  \begin{align*}
    \sem{\zro}         &= \lambda \gamma . 0 \\
    \sem{\suc\, V}     &= \lambda \gamma . (\sem{V}\, \gamma) + 1 \\
    \sem{x \in \Gamma} &= \lambda \gamma . \gamma(x) \\
    \sem{\lda{x}{M}}   &= \lambda \gamma . \lambda a . \sem{M}\, (*,\, (\gamma , a))  \\
    \sem{\injnat{V_n}} &= \lambda \gamma . \inl\, (\sem{V_n}\, \gamma) \\
    \sem{\injarr{V_f}} &= \lambda \gamma . \inr\, (\sem{V_f}\, \gamma) \\[2ex]
    % \sem{\nxt\, V}     &= \lambda \gamma . \nxt (\sem{V}\, \gamma) \\
    % \sem{\theta}       &= \lambda \gamma . \theta \\
  %
    \sem{\err_B}         &= \lambda \delta . \mho \\
    \sem{\ret\, V}       &= \lambda \gamma . \eta\, \sem{V} \\
    \sem{\bind{x}{M}{N}} &= \lambda \delta . \ext {(\lambda x . \sem{N}\, (\delta, x))} {\sem{M}\, \delta} \\
    \sem{V_f\, V_x}      &= \lambda \gamma . {({(\sem{V_f}\, \gamma)} \, {(\sem{V_x}\, \gamma)})} \\
    \sem{\casedyn{V}{n}{M_{nat}}{\tilde{f}}{M_{fun}}} &=
      \lambda \delta . \text{case $(\sem{V}\, \delta)$ of} \\ 
      &\quad\quad\quad\quad \alt \inl(n) \to \sem{M_{nat}}(n) \\
      &\quad\quad\quad\quad \alt \inr(\tilde{f}) \to \sem{M_{fun}}(\tilde{f})
  \end{align*}

  \caption{Term semantics for the gradually-typed lambda calculus.}
  \label{fig:term-semantics}
\end{figure*}

\section{Extending the Semantics to Precision}\label{sec:gtlc-precision}

In this section, we extend the set-theoretic semantics for terms given in
the previous section to a semantics for the type and term precision relations
of the gradually-typed lambda calculus. We first introduce the type and term precision
relations, then show how to give them a semantics using SGDT.

% TODO mention intensional syntax


\subsection{Term Precision for GTLC}\label{sec:gtlc-term-precision-axioms}

% ---------------------------------------------------------------------------------------
% ---------------------------------------------------------------------------------------

%\subsubsection{Term Precision}\label{sec:term-precision}

We allow for a \emph{heterogeneous} term precision judgment on values $V$ of type
$A$ and $V'$ of type $A'$ provided that $A \ltdyn A'$ holds. Likewise, for producers,
if $M$ has type $A$ and $M'$ has type $A'$, we can form the judgment that $M \ltdyn M'$.
We use the same notation for the precision relation on both values and producers.

% Type precision contexts
In order to deal with open terms, we will need the notion of a type precision \emph{context}, which we denote
$\gamlt$. This is similar to a normal context but instead of mapping variables to types,
it maps variables $x$ to related types $A \ltdyn A'$, where $x$ has type $A$ in the left-hand term
and $A'$ in the right-hand term. We may also write $x : d$ where $d : A \ltdyn A'$ to indicate this.

% An equivalent way of thinking of type precision contexts is as a pair of ``normal" typing
% contexts $\Gamma, \Gamma'$ with the same domain such that $\Gamma(x) \ltdyn \Gamma'(x)$ for
% each $x$ in the domain.
% We will write $\gamlt : \Gamma \ltdyn \Gamma'$ when we want to emphasize the pair of contexts.
% Conversely, if we are given $\gamlt$, we write $\gamlt_l$ and $\gamlt_r$ for the normal typing contexts on each side.

An equivalent way of thinking of a type precision context $\gamlt$ is as a
pair of ``normal" typing contexts, $\gamlt_l$ and $\gamlt_r$, with the same
domain and such that $\gamlt_l(x) \ltdyn \gamlt_r(x)$ for each $x$ in the domain.
We will write $\gamlt : \gamlt_l \ltdyn \gamlt_r$ when we want to emphasize the pair of contexts.

As with type precision derivations, we write $\Gamma$ to mean the ``reflexivity" type precision context
$\Gamma : \Gamma \ltdyn \Gamma$.
Concretely, this consists of reflexivity type precision derivations $\Gamma(x) \ltdyn \Gamma(x)$ for
each $x$ in the domain of $\Gamma$.

Furthermore, we write $\gamlt_1 \relcomp \gamlt_2$ to denote the ``composition'' of $\gamlt_1$ and $\gamlt_2$
--- that is, the precision context whose value at $x$ is the type precision derivation
$\gamlt_1(x) \relcomp \gamlt_2(x)$. This of course assumes that each of the type precision
derivations is composable, i.e., that the RHS of $\gamlt_1(x)$ is the same as the left-hand side of $\gamlt_2(x)$.

% We define the same for computation type precision contexts $\deltalt_1$ and $\deltalt_2$,
% provided that both the computation type precision contexts have the same ``shape'', which is defined as
% (1) either the stoup is empty in both, or the stoup has a hole in both, say $\hole{d}$ and $\hole{d'}$
% where $d$ and $d'$ are composable, and (2) their value type precision contexts are composable as described above.

The rules for term precision come in two forms. We first have the \emph{congruence} rules,
one for each term constructor. These assert that the term constructors respect term precision.
The congruence rules are as follows:

\begin{mathpar}

  \inferrule*[right = Var]
    { c : A \ltdyn B \and \gamlt(x) = (A, B) } 
    { \etmprec {\gamlt} x x c }

  \inferrule*[right = Zro]
    { } {\etmprec \gamlt \zro \zro \nat }

  \inferrule*[right = Suc]
    { \etmprec \gamlt V {V'} \nat } {\etmprec \gamlt {\suc\, V} {\suc\, V'} \nat}

  \inferrule*[right = MatchNat]
  {\etmprec \gamlt V {V'} \nat \and 
    \etmprec \deltalt M {M'} d \and \etmprec {\deltalt, n : \nat} {N} {N'} d}
  {\etmprec \deltalt {\matchnat {V} {M} {n} {N}} {\matchnat {V'} {M'} {n} {N'}} d}

  \inferrule*[right = Lambda]
    { c_i : A_i \ltdyn A'_i \and 
      c_o : A_o \ltdyn A'_o \and 
      \etmprec {\gamlt, x : c_i} {M} {M'} {c_o} } 
    { \etmprec \gamlt {\lda x M} {\lda x {M'}} {(c_i \ra c_o)} }

  \inferrule*[right = App]
    { c_i : A_i \ltdyn A'_i \and
      c_o : A_o \ltdyn A'_o \\\\
      \etmprec \gamlt {V_f} {V_f'} {(c_i \ra c_o)} \and
      \etmprec \gamlt {V_x} {V_x'} {c_i}
    } 
    { \etmprec {\gamlt} {V_f\, V_x} {V_f'\, V_x'} {{c_o}}}

  \inferrule*[right = Ret]
    {\etmprec {\gamlt} V {V'} c}
    {\etmprec {\gamlt} {\ret\, V} {\ret\, V'} {c}}

  \inferrule*[right = Bind]
    {\etmprec {\gamlt} {M} {M'} {c} \and 
     \etmprec {\gamlt, x : c} {N} {N'} {d} }
    {\etmprec {\gamlt} {\bind {x} {M} {N}} {\bind {x} {M'} {N'}} {d}}
\end{mathpar}

We then have additional equational axioms, including $\beta$ and $\eta$ laws, and
rules characterizing upcasts as least upper bounds, and downcasts as greatest lower bounds.
For the sake of familiarity, we formulate the cast rules using arbitrary casts; later we
will state the analogous versions for the version of the calculus without arbitrary casts.

We write $M \equidyn N$ to mean that both $M \ltdyn N$ and $N \ltdyn M$.

\begin{mathpar}
  \inferrule*[right = $\err$]
    {\phasty {\Gamma} M B }
    {\etmprec {\Gamma} {\err_B} M B}

  \inferrule*[right = $\beta$-fun]
    { \phasty {\Gamma, x : A_i} M {A_o} \and
      \vhasty {\Gamma} V {A_i} } 
    { \etmequidyn {\Gamma} {(\lda x M)\, V} {M[V/x]} {A_o} }

  \inferrule*[right = $\eta$-fun]
    { \vhasty {\Gamma} {V} {A_i \ra A_o} } 
    { \etmequidyn \Gamma {\lda x (V\, x)} V {A_i \ra A_o} }

  \inferrule*[right = UpR]
    { c : A \ltdyn B \and d : B \ltdyn C \and 
      \etmprec {\gamlt} {M} {N} {c} } 
    { \etmprec {\gamlt} {M} {\up {B} {C} N} {c \circ d}  }

  \inferrule*[right = UpL]
    { c : A \ltdyn B \and d : B \ltdyn C \and
      \etmprec {\gamlt} {M} {N} {c \circ d} } 
    { \etmprec {\gamlt} {\up {A} {B} M} {N} {d} }

  \inferrule*[right = DnL]
    { c : A \ltdyn B \and d : B \ltdyn C \and
      \etmprec {\gamlt} {M} {N} {d} } 
    { \etmprec {\gamlt} {\dn {A} {B} M} {N} {c \circ d} }

  \inferrule*[right = DnR]
    { c : A \ltdyn B \and d : B \ltdyn C \and
      \etmprec {\gamlt} {M} {N} {c \circ d} } 
    { \etmprec {\gamlt} {M} {\dn {B} {C} N} {c} }
\end{mathpar}

% TODO explain the least upper bound/greatest lower bound rules
The rules UpR, UpL, DnL, and DnR were introduced in \cite{new-licata18} as a means
of cleanly axiomatizing the intended behavior of casts in a way that
doesn't depend on the specific constructs of the language.
Intuitively, rule UpR says that the upcast of $M$ is an upper bound for $M$
in that $M$ may error more, and UpL says that the upcast is the \emph{least}
such upper bound, in that it errors more than any other upper bound for $M$.
Conversely, DnL says that the downcast of $M$ is a lower bound, and DnR says
that it is the \emph{greatest} lower bound.
% These rules provide a clean axiomatization of the behavior of casts that doesn't
% depend on the specific constructs of the language.



\subsection{Semantics for Precision}

As a first attempt at giving a semantics to the ordering, we could try to model types as
sets equipped with an ordering that models term precision. Since term precision is reflexive
and transitive, and since we identify terms that are equi-precise, we choose to model types
as partially-ordered sets. We model the term precision ordering $M \ltdyn N : A \ltdyn B$ as an
ordering relation between the posets denoted by $A$ and $B$.

However, it turns out that modeling term precision by a relation defined by guarded fixpoint
is not as straightforward as one might hope.
A first attempt might be to define an ordering $\semltbad$ between $\li X$ and $\li Y$
that allows for computations that may take different numbers of steps to be related.
The relation is parameterized by a relation $\le$ between $X$ and $Y$, and is defined
by guarded fixpoint as follows:
% simultaneously captures the notions of error approximation and equivalence up to stepping behavior:

\begin{align*}
  &\eta\, x \semltbad \eta\, y \text{ if } 
    x \semlt y \\
%		
  &\mho \semltbad l \\
%
  &\theta\, \tilde{l} \semltbad \theta\, \tilde{l'} \text{ if } 
    \later_t (\tilde{l}_t \semltbad \tilde{l'}_t) \\
%	
  &\theta\, \tilde{l} \semltbad \mho \text{ if } 
    \theta\, \tilde{l} = \delta^n(\mho) \text { for some $n$ } \\
%	
  &\theta\, \tilde{l} \semltbad \eta\, y \text{ if }
    (\theta\, \tilde{l} = \delta^n(\eta\, x))
  \text { for some $n$ and $x : \ty{X}$ such that $x \le y$ } \\
%
  &\mho \semltbad \theta\, \tilde{l'} \text { if } 
    \theta\, \tilde{l'} = \delta^n(\mho) \text { for some $n$ } \\
%	
  &\eta\, x \semltbad \theta\, \tilde{l'} \text { if }
    (\theta\, \tilde{l'} = \delta^n (\eta\, y))
  \text { for some $n$ and $y : \ty{Y}$ such that $x \le y$ }
\end{align*}

Two computations that immediately return $(\eta)$ are related if the underlying
values are related in the underlying ordering. 
%
The computation that errors $(\mho)$ is below everything else.
%
If both sides step (i.e., both sides are $\theta$),
then we allow one time step to pass and compare the resulting terms.
(This is where use the relation defined ``later''.)
%
Lastly, if one side steps and the other returns a value, the side that steps should
terminate with a value in some finite number of steps $n$, and that value should
be related to the value returned by the other side.
Likewise, if one side steps and the other errors, then the side that steps
should terminate with error.

The problem with this definition is that the resulting relation is \emph{provably} not
transitive: it can be shown (in Clocked Cubical Type Theory) that if $R$ is a
relation on $\li X$ satisfying three specific properties, one of which is
transitivity, then that relation is trivial.
(The other two properties are that the relation is a congruence with respect to $\theta$,
and that the relation is closed under delays $\delta = \theta \circ \nxt$ on either side.)
Since the above relation \emph{does} satisfy the other two properties, we conclude
that it must not be transitive.

%But having a non-transitive relation to model term precision presents a problem
%for...

We are therefore led to wonder whether we can formulate a version of the relation
that \emph{is} transitive.
It turns out that we can, by sacrificing another of the three properties from
the above lemma. Namely, we give up on closure under delays. Doing so, we end up
with a \emph{lock-step} error ordering, where, roughly speaking, in order for
computations to be related, they must have the same stepping behavior.
%
We then formulate a separate relation, \emph{weak bisimilarity}, that relates computations
that are extensionally equal and may only differ in their stepping behavior.

% As a result, we instead separate the semantics of term precision into two relations:
% an intensional, step-sensitive \emph{error ordering} and a \emph{bisimilarity relation}.


\subsubsection{Double Posets}\label{sec:predomains}

As discussed above, there are two relations that we would like to define
in the semantics: a step-sensitive error ordering, and weak bisimilarity of computations.
%
The semantic objects that interpret our types should therefore be equipped with
two relations. We call these objects ``double posets''.
A double poset $A$ is a set with two relations: an partial order $\semlt_A$ on $A$, and
a reflexive, symmetric relation $\bisim_A$ on $A$.
We write the underling set of $A$ as $\ty{A}$.

We define morphisms of double posets as functions that preserve both
the ordering and the bisimilarity relation. Given double posets
$A$ and $B$, we write $f \colon A \monto B$ to indicate that $f$ is a morphism
from $A$ to $B$, i.e, the following hold:
(1) for all $a_1 \semlt_A a_2$, we have $f(a_1) \semlt_{B} f(a_2)$, and
(2) for all $a_1 \bisim_A a_2$, we have $f(a_1) \bisim_{B} f(a_2)$.


%%%%% RESUME HERE

We define an ordering on morphisms of double posets as
$f \le g$ if for all $a$ in $\ty{A_i}$, we have $f(a) \le_{A_o} g(a)$,
and similarly bisimilarity extends to morphisms via
$f \bisim g$ if for all $a$ in $\ty{A_i}$, we have $f(a) \bisim_{A_o} g(a)$.

For each type we want to represent, we define a double poset for the corresponding semantic
type. For instance, we define a double poset for natural numbers, for the
dynamic type, for functions, and for the lift monad. We
describe each of these below.

\begin{itemize}
  \item There is a double poset $\Nat$ for natural numbers, where the ordering and the
  bisimilarity relations are both equality.
  
  % TODO explain that there is a theta operator for posets?
  \item There is a double poset $\Dyn$ to represent the dynamic type. The underlying type
  for this double poset is defined by guarded fixpoint to be such that
  $\ty{\Dyn} \cong \mathbb{N}\, +\, \later (\ty{\Dyn} \monto \li \ty{\Dyn})$.
  This definition is valid because the occurrences of Dyn are guarded by the $\later$.
  The ordering is defined via guarded recursion by cases on the argument, using the
  ordering on $\mathbb{N}$ and the ordering on monotone functions described above.

  \item For a double poset $A$, there is a double poset $\li A$ for the ``lift'' of $A$
  using the lift monad. We use the same notation for $\li A$ when $A$ is a type
  and $A$ is a double poset, since the context should make clear which one we are referring to.
  The underling type of $\li A$ is simply $\li \ty{A}$, i.e., the lift of the underlying
  type of $A$.
  The ordering on $\li A$ is the ``lock-step error-ordering'' which we describe in
  \ref{subsec:lock-step}. The bismilarity relation is the ``weak bisimilarity''
  described in Section \ref{}

  \item For double posets $A_i$ and $A_o$, we form the double poset of monotone functions
  from $A_i$ to $A_o$, which we denote by $A_i \To A_o$.

  \item Given a double poset $A$, we can form the double poset $\later A$ whose underlying
  type is $\later \ty{A}$. We define $\tilde{x} \le_{\later A} \tilde{y}$ to be
  $\later_t (\tilde{x}_t \le_A \tilde{y}_t)$.
\end{itemize}

\subsubsection{Step-Sensitive Error Ordering}\label{subsec:lock-step}

As mentioned, the ordering on the lift of a double poset $A$ is called the
\emph{step-sensitive error-ordering} (also called ``lock-step error ordering''),
the idea being that two computations $l$ and $l'$ are related if they are in
lock-step with regard to their intensional behavior, up to $l$ erroring.
Formally, we define this ordering as follows:

\begin{itemize}
  \item 	$\eta\, x \ltls \eta\, y$ if $x \le_A y$.
  \item 	$\mho \ltls l$ for all $l$ 
  \item   $\theta\, \tilde{r} \ltls \theta\, \tilde{r'}$ if
          $\later_t (\tilde{r}_t \ltls \tilde{r'}_t)$
\end{itemize}

We also define a heterogeneous version of this ordering between the lifts of two
different double posets $A$ and $B$, parameterized by a relation $R$ between $A$ and $B$.

\subsubsection{Weak Bisimilarity Relation}

For a double poset $A$, we define a relation on $\li A$, called ``weak bisimilarity",
written $l \bisim l'$. Intuitively, we say $l \bisim l'$ if they are equivalent
``up to delays''.
% We introduce the notation $x \sim_A y$ to mean $x \le_A y$ and $y \le_A x$.
% TODO if A is a poset, then we can just say that x = y
%
The weak bisimilarity relation is defined by guarded fixpoint as follows:

\begin{align*}
  &\mho \bisim \mho \\
%
  &\eta\, x \bisim \eta\, y \text{ if } 
    x \bisim_A y \\
%		
  &\theta\, \tilde{x} \bisim \theta\, \tilde{y} \text{ if } 
    \later_t (\tilde{x}_t \bisim \tilde{y}_t) \\
%	
  &\theta\, \tilde{x} \bisim \mho \text{ if } 
    \theta\, \tilde{x} = \delta^n(\mho) \text { for some $n$ } \\
%	
  &\theta\, \tilde{x} \bisim \eta\, y \text{ if }
    (\theta\, \tilde{x} = \delta^n(\eta\, x))
  \text { for some $n$ and $x : \ty{A}$ such that $x \sim_A y$ } \\
%
  &\mho \bisim \theta\, \tilde{y} \text { if } 
    \theta\, \tilde{y} = \delta^n(\mho) \text { for some $n$ } \\
%	
  &\eta\, x \bisim \theta\, \tilde{y} \text { if }
    (\theta\, \tilde{y} = \delta^n (\eta\, y))
  \text { for some $n$ and $y : \ty{A}$ such that $x \sim_A y$ }
\end{align*}

When both sides are $\eta$, then we ensure that the underlying values are bisimilar
in the underlying bisimilarity relation on $A$.
When one side is a $\theta$ and the other is $\eta x$ (i.e., one side steps),
we stipulate that the $\theta$-term runs to $\eta y$ where $x$ is related to $y$.
Similarly when one side is $\theta$ and the other $\mho$.
If both sides step, then we allow one time step to pass and compare the resulting terms.
In this way, the definition captures the intuition of terms being equivalent up to
delays.

It can be shown (by \lob-induction) that the step-sensitive relation is symmetric.
However, it can also be shown that this relation is \emph{not} transitive:
The argument is the same as that used to show that the step-insensitive error
ordering $\semltbad$ described above is not transitive. Namely, we show that
if it were transitive, then it would have to be trivial in that $l \bisim l'$ for all $l, l'$.
that if this relation were transitive, then in fact it would be trivial in that
%This issue will be resolved when we consider the relation's \emph{globalization}.

\subsection{The Cast Rules}

Unfortunately, the four cast rules defined above do not hold in
the intensional setting where we are tracking the steps taken by terms.
The source of the problem is that the downcast from the dynamic type to
a function involves a delay, i.e., a $\theta$.
So in order to keep the other term in lock-step, we need to insert a ``delay"
that is extensionally equivalent to the identity function.
More concretely, consider a simplified version of the DnL rule shown below:

\begin{mathpar}
  \inferrule*{M \ltdyn_i N : B}
             {\dnc{c}{M} \ltdyn_i N : c}
\end{mathpar}

If $c$ is inj-arr, then when we downcast $M$ from $dyn$ to $\dyntodyn$,
semantically this will involve a $\theta$ because the value of type $dyn$
in the semantics will contain a \emph{later} function $\tilde{f}$.
Thus, in order for the right-hand side to be related to the downcast,
we need to insert a delay on the right.
%
The need for delays affects the cast rules involving upcasts as well, because
the upcast for functions involves a downcast on the domain:

\[ \up{A_i \ra A_o}{B_i \ra B_o}{M} \equiv \lambda (x : B_i). \up{A_o}{B_o}(M\, (\dn {A_i}{B_i} x)). \]

Thus, the correct versions of the cast rules involve delays on the side that was not casted.


% Delays for function types and for inj-arr(c)


\subsubsection{Perturbations}

We can describe precisely how the delays are inserted for any type precision
derivation $c$.

To do so, we first define simultaneously an inductive type of \emph{perturbations}
for embeddings $\perte$ and for projections $\pertp$ by the following rules:

\begin{mathpar}

\inferrule{}{\id : \perte A}

\inferrule{}{\id : \pertp A}

\inferrule
  {\delta_c : \pertp A \and \delta_d : \perte B}
  {\delta_c \ra \delta_d : \perte (A \ra B)}

\inferrule
  {\delta_c : \perte A \and \delta_d : \pertp B}
  {\delta_c \ra \delta_d : \pertp (A \ra B)}

\inferrule
  {\delta_\nat : \perte \nat \and \delta_f : \perte (\dyntodyn)}
  {\pertdyn{\delta_\nat}{\delta_f} : \perte \dyn}

\inferrule
  {\delta_\nat : \pertp \nat \and \delta_f : \pertp (\dyntodyn)}
  {\pertdyn{\delta_\nat}{\delta_f} : \pertp \dyn}

\end{mathpar}

The structure of embedding perturbations is designed to follow the structure
of the corresponding embeddings, and likewise for the projection perturbations.
Thus, in the function case, an embedding perturbation consists of a \emph{projection}
perturbation for the domain and an \emph{embedding} perturbation for the codomain.
The opposite holds for the projection perturbation for functions.

Another way in which the two kinds of perturbations differ is that there is an additional
projection perturbation for delaying $\delaypert{\delta}$.
This corresponds to the actual delay term $\delta = \theta \circ \nxt$ in the semantics,
and it is the generator/source of all non-trivial perturbations.

Given a perturbation $\delta$, we can turn it into a term, which we also write as
$\delta$ unless there is opportunity for confusion.



\section{Unary Canonicity}
Before discussing graduality, we seek to prove its ``unary'' analogue.
Namely, instead of considering inequality between terms, we start by considering equality.


\section{Graduality}\label{sec:graduality}
The main theorem we would like to prove is the following:

\begin{theorem}[Graduality]
  If $\cdot \vdash M \ltdyn N : \nat$, then
  \begin{enumerate}
    \item If $N = \mho$, then $M = \mho$
    \item If $N = `n$, then $M = \mho$ or $M = `n$
    \item If $M = V$, then $N = V$
  \end{enumerate}
\end{theorem}




% \section{Discussion}\label{sec:discussion}

% \subsection{Synthetic Ordering}

% While the use of synthetic guarded domain theory allows us to very
% conveniently work with non-well-founded recursive constructions while
% abstracting away the precise details of step-indexing, we do work with
% the error ordering in a mostly analytic fashion in that gradual types
% are interpreted as sets equipped with an ordering relation, and all
% terms must be proven to be monotone.
% %
% It is possible that a combination of synthetic guarded domain theory
% with \emph{directed} type theory would allow for an a synthetic
% treatment of the error ordering as well.


\bibliographystyle{ACM-Reference-Format}
\bibliography{references}

\end{document}

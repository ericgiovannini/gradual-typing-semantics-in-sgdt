\documentclass[acmsmall,screen]{acmart}
\usepackage{mathpartir}
\usepackage{tikz-cd}
\usepackage{enumitem}
\usepackage{wrapfig}
\usepackage{fancyvrb}
\usepackage{comment}

\usepackage{ stmaryrd }


%% Rights management information.  This information is sent to you
%% when you complete the rights form.  These commands have SAMPLE
%% values in them; it is your responsibility as an author to replace
%% the commands and values with those provided to you when you
%% complete the rights form.
\setcopyright{acmcopyright}
\copyrightyear{2018}
\acmYear{2018}
\acmDOI{10.1145/1122445.1122456}

%% These commands are for a PROCEEDINGS abstract or paper.
\acmConference[Woodstock '18]{Woodstock '18: ACM Symposium on Neural
  Gaze Detection}{June 03--05, 2018}{Woodstock, NY}
\acmBooktitle{Woodstock '18: ACM Symposium on Neural Gaze Detection,
  June 03--05, 2018, Woodstock, NY}
\acmPrice{15.00}
\acmISBN{978-1-4503-XXXX-X/18/06}

\newcommand{\To}{\Rightarrow}
\newcommand{\inl}{\mathsf{inl}}
\newcommand{\inr}{\mathsf{inr}}
\newcommand{\alt}{\mathrel{\bf \,\mid\,}}

\newcommand{\ob}{\text{Ob}}


\newcommand{\extlc}{\text{Ext-}\lambda}
\newcommand{\extlcm}{\text{Ext-}\lambda^{-\text{trans}}}
\newcommand{\extlcmm}{\text{Ext-}\lambda^{-\text{trans}-\text{cast}}}
\newcommand{\extlcprime}{\text{Ext-}\lambda'}
\newcommand{\intlc}{\text{Int-}\lambda}
\newcommand{\intlcbisim}{\text{Int$_\approx$-}\lambda}
\newcommand{\erase}[1]{\lfloor {#1} \rfloor}


\newcommand{\uarrowl}{\mathrel{\rotatebox[origin=c]{-30}{$\leftarrowtail$}}}
\newcommand{\uarrowr}{\mathrel{\rotatebox[origin=c]{60}{$\leftarrowtail$}}}
\newcommand{\darrowl}{\mathrel{\rotatebox[origin=c]{30}{$\twoheadleftarrow$}}}
\newcommand{\darrowr}{\mathrel{\rotatebox[origin=c]{120}{$\twoheadleftarrow$}}}
\newcommand{\vuarrow}{\mathrel{\rotatebox[origin=c]{-90}{$\leftarrowtail$}}}
\newcommand{\vdarrow}{\mathrel{\rotatebox[origin=c]{90}{$\twoheadleftarrow$}}}

% Types, terms, and precision 

\newcommand{\dyn}{{?}}
\newcommand{\nat}{\text{Nat}}
\newcommand{\bool}{\text{Bool}}
\newcommand{\ra}{\rightharpoonup}
\newcommand{\Ret}[1]{\mathsf{Ret\,}{#1}}
\newcommand{\hole}[1]{\bullet \colon {#1}}
\newcommand{\dyntodyn}{\dyn \ra\, \dyn}


\newcommand{\up}[2]{\langle{#2}\uarrowl{#1}\rangle}
\newcommand{\dn}[2]{\langle{#1}\darrowl{#2}\rangle}

\newcommand{\upc}[1]{\text{up}\,{#1}\,}
\newcommand{\dnc}[1]{\text{dn}\,{#1}\,}


% \newcommand{\ret}{\mathsf{ret}}
\newcommand{\err}{\mho}
\newcommand{\zro}{\textsf{zro}}
\newcommand{\suc}{\textsf{suc}}
\newcommand{\lda}[2]{\lambda {#1} . {#2}}

\newcommand{\injarr}[1]{\textsf{Inj}_\ra ({#1})}
\newcommand{\injnat}[1]{\textsf{Inj}_\text{nat} ({#1})}
\newcommand{\casenat}[4]{\text{Case}_\text{nat} ({#1}) \{ \text{no} \to {#2} \alt \text{nat}({#3}) \to {#4} \}}
\newcommand{\casearr}[4]{\text{Case}_\ra ({#1}) \{ \text{no} \to {#2} \alt \text{fun}({#3}) \to {#4} \}}
\newcommand{\casedyn}[5]{\text{Case}_\Dyn ({#1}) \{ \text{nat}({#2}) \to {#3} \alt \text{fun}({#4}) \to {#5} \}}
%\newcommand{\bind}[3]{\text{var } {#1} = {#2} \text{ in } {#3}}
\newcommand{\matchnat}[4]{\text{match } {#1} \text{ with } \{ \textsf {zero} \Rightarrow {#2} \alt \textsf{ suc } {#3} \Rightarrow {#4} \}}

\newcommand{\refl}{\text{refl}}

\newcommand{\Lift}{\text{Lift}}

%\newcommand{\rel}{\circ\hspace{-4px}-\hspace{-4px}\bullet}
\newcommand{\rel}{\mathrel{\circ\mkern-6mu-\mkern-6mu\bullet}}
% \newcommand{\wand}{\mathrel{-\mkern-6mu*}}
\newcommand{\wand}{\mathrel{\multimap}}
\newcommand{\ltdyn}{\sqsubseteq}
\newcommand{\gtdyn}{\sqsupseteq}
\newcommand{\equidyn}{\mathrel{\gtdyn\ltdyn}}
\newcommand{\gamlt}{\Gamma^\ltdyn}
\newcommand{\deltalt}{\Delta^\ltdyn}
\newcommand{\relcomp}{\odot}

\newcommand{\hasty}[3]{{#1} \vdash {#2} \colon {#3}}
\newcommand{\vhasty}[3]{{#1} \vdash^v {#2} \colon {#3}}
\newcommand{\phasty}[3]{{#1} \vdash^c {#2} \colon {#3}}
\newcommand{\etmprec}[4]{{#1} \vdash {#2} \ltdyn_e {#3} \colon {#4}}
\newcommand{\itmprec}[4]{{#1} \vdash {#2} \ltdyn_i {#3} \colon {#4}}
\newcommand{\etmequidyn}[4]{{#1} \vdash {#2} \equidyn_e {#3} \colon {#4}}
\newcommand{\itmequidyn}[4]{{#1} \vdash {#2} \equidyn_i {#3} \colon {#4}}
\newcommand{\synbisim}{\approx_{\text{syn}}}

\newcommand{\Dwn}{\Downarrow}
\newcommand{\qte}[1]{\text{quote}({#1})}

\newcommand{\elab}[1]{\text{Elab}({#1})}

% Perturbations
\newcommand{\pertp}{\text{Pert}^\text{P}}
\newcommand{\perte}{\text{Pert}^\text{E}}
\newcommand{\pertdyn}[2]{\text{pert-dyn}({#1}, {#2})}
\newcommand{\delaypert}[1]{\text{delay-pert}({#1})}

\newcommand{\pertc}{\text{Pert}_{\text{C}}}
\newcommand{\pertv}{\text{Pert}_{\text{V}}}


% SGDT and Intensional Stuff

\newcommand{\later}{{\vartriangleright}}
\newcommand{\laterhs}{{\later}}
\newcommand{\type}{\texttt{Type}}
\newcommand{\lob}{\text{L\"{o}b}}
\newcommand{\tick}{\mathsf{tick}}
\newcommand{\nxt}{\mathsf{next}}
\newcommand{\fix}{\mathsf{fix}}
\newcommand{\kpa}{\kappa}

% Model-related stuff
\newcommand{\calV}{\mathcal{V}}
\newcommand{\calE}{\mathcal{E}}
\newcommand{\calVk}{\mathcal{V}_k}
\newcommand{\calEk}{\mathcal{E}_k}
\newcommand{\tok}{\mathrel{\mathop{\to}\limits^{\textrm{\footnotesize k}}}}
\newcommand{\timesk}{\mathrel{\mathop{\times}\limits^{\textrm{k}}}}
\newcommand{\Fk}{\mathrel{\mathop{F}\limits^{\textrm{k}}}}
\newcommand{\Uk}{\mathrel{\mathop{U}\limits^{\textrm{k}}}}

\newcommand{\op}[1]{{#1}^{\textrm{op}}}
\newcommand{\calC}{\mathcal{C}}
\newcommand{\Set}{\mathsf{Set}}
\newcommand{\ErrDom}{\mathsf{ErrDom}}
\newcommand{\Yo}{\mathsf{Yo}}
\newcommand{\Hom}{\mathsf{Hom}}
\newcommand{\calS}{\mathcal{S}}
\newcommand{\gfix}{\texttt{gfix}}
\newcommand{\qfix}{\texttt{qfix}}
\newcommand{\calU}{\mathcal{U}}
\newcommand{\laterhat}{\widehat{\later}}
\newcommand{\El}{\mathsf{El}}
\newcommand{\Clock}{\mathsf{Clock}}

\newcommand{\Machine}[1]{\mathsf{Machine}\, {#1}}


% Predomains and EP pairs
\newcommand{\Nat}{\mathsf{Nat}}
\newcommand{\Dyn}{\mathsf{Dyn}}
\newcommand{\ty}[1]{\langle {#1} \rangle}
\newcommand{\li}{L_\mho}
\newcommand{\liclk}[1]{L_\mho [{#1}]}

\newcommand{\ext}[2]{\text{ext}\,{#1}\,{#2}}
\newcommand{\map}[2]{\text{map}\,{#1}\,{#2}}

\newcommand{\ltls}{\ltdyn}
\newcommand{\bisim}{\approx}
\newcommand{\semlt}{\le}
\newcommand{\semltbad}{\lesssim}

%\newcommand{\injarr}{\textsf{Inj}_\to}
%\newcommand{\injnat}{\textsf{Inj}_\mathbb{N}}

\newcommand{\id}{\mathsf{id}}
\newcommand{\ep}{\leadsto}

\newcommand{\emb}[2]{\mathsf{emb}_{#1}({#2})}
\newcommand{\proj}[2]{\mathsf{proj}_{#1}({#2})}

\newcommand{\monto}{\to_m}


% Notation for wait functions
\newcommand{\wre}{w_r^e}
\newcommand{\wle}{w_l^e}
\newcommand{\wrp}{w_r^p}
\newcommand{\wlp}{w_l^p}

\newcommand{\sem}[1]{\llbracket {#1} \rrbracket}
\newcommand{\semgl}[1]{{\sem{#1}}^\text{gl}}


% Denotational model
\newcommand{\errdom}{\textsf{ErrDom}}

\newcommand{\ptb}{\text{ptb}}
\newcommand{\push}{\text{push}}
\newcommand{\pull}{\text{pull}}

\newcommand{\pv}{P^{\mathcal{V}}}
\newcommand{\pe}{P^{\mathcal{E}}}
\newcommand{\ptbv}{\text{ptb}^\mathcal{V}}
\newcommand{\ptbe}{\text{ptb}^\mathcal{E}}

\newcommand{\Ppure}{P}
\newcommand{\Pk}{P^K}
\newcommand{\ptbk}{\text{ptb}^K}


\newcommand{\upf}{\text{up}}
\newcommand{\dnf}{\text{dn}
}
\newcommand{\arr}{\to}
\newcommand{\comp}{}

\newcommand{\ltsq}[2]{\mathrel{\ltdyn^{#1}_{#2}}}
\newcommand{\ltsqbisim}[2]{\mathrel{{\widetilde{\ltdyn}}^{#1}_{#2}}}
\newcommand{\ltbisim}{\mathrel{\widetilde{\ltdyn}}}
\newcommand{\vf}{\mathcal{V}_f}
\newcommand{\vsq}{\mathcal{V}_{sq}}
\newcommand{\ef}{\mathcal{E}_f}
\newcommand{\esq}{\mathcal{E}_{sq}}

\newcommand{\morbisimid}[1]{\text{Endo}_\bisim({#1})}


\newcommand{\sv}{s_{\mathcal{V}}}
\newcommand{\tv}{t_{\mathcal{V}}}
\newcommand{\rv}{r_{\mathcal{V}}}
\newcommand{\se}{s_{\mathcal{E}}}
\newcommand{\te}{t_{\mathcal{E}}}
\newcommand{\re}{r_{\mathcal{E}}}

\newcommand{\Ff}{F_f}
\newcommand{\Fsq}{F_{sq}}
\newcommand{\Uf}{U_f}
\newcommand{\Usq}{U_{sq}}
\newcommand{\arrf}{\arr_f}
\newcommand{\arrsq}{\arr_{sq}}

\newcommand{\vsim}{\mathcal{V}_{\bisim}}
\newcommand{\esim}{\mathcal{E}_{\bisim}}
\newcommand{\svsim}{s_{\mathcal{V}}^\bisim}
\newcommand{\tvsim}{t_{\mathcal{V}}^\bisim}
\newcommand{\rvsim}{r_{\mathcal{V}}^\bisim}
\newcommand{\sesim}{s_{\mathcal{E}}^\bisim}
\newcommand{\tesim}{t_{\mathcal{E}}^\bisim}
\newcommand{\resim}{r_{\mathcal{E}}^\bisim}



\newcommand{\ve}{\mathcal{V}_e}
\newcommand{\ee}{\mathcal{E}_e}

\newcommand{\vr}{\mathcal{V}_r}
\newcommand{\er}{\mathcal{E}_r}

\newcommand{\relatedin}[3]{{#1}\, {#3}\, {#2}}
\newcommand{\binrel}[1]{\mathbin{#1}}


\newcommand{\da}{\downarrow}

\newcommand{\ret}{\text{ret}}
\newcommand{\bind}[3]{{#1} <- {#2}\, ; \, {#3}}

\newcommand{\piv}{\Pi^\mathcal{V}}
\newcommand{\pie}{\Pi^\mathcal{E}}

\newcommand{\upl}{\textsc{UpL}}
\newcommand{\upr}{\textsc{UpR}}
\newcommand{\dnl}{\textsc{DnL}}
\newcommand{\dnr}{\textsc{DnR}}

\newcommand{\delre}{\delta^{r,e}}
\newcommand{\delle}{\delta^{l,e}}
\newcommand{\delrp}{\delta^{r,p}}
\newcommand{\dellp}{\delta^{l,p}}

\newcommand{\qordeq}{\bisim}

\newcommand{\inat}{\text{Inj}_\mathbb{N}}
\newcommand{\itimes}{\text{Inj}_\times}
\newcommand{\iarr}{\text{Inj}_\to}

\newcommand{\tnat}{\mathsf{nat}}
\newcommand{\ttimes}{\mathsf{times}}
\newcommand{\tfun}{\mathsf{fun}}

\newcommand{\cased}[7]{
    \begin{align*}
    \text{Case}_D ({#1}) &\text{ of } \{ \\
        &\alt \text{nat}({#2}) \to {#3}  \\
        &\alt \text{times}({#4}) \to {#5}  \\
        &\alt \text{fun}({#6}) \to {#7} \}
    \end{align*}
}

\newcommand{\inone}{\mathsf{in}_1}
\newcommand{\intwo}{\mathsf{in}_2}
\newcommand{\inthree}{\mathsf{in}_3}
\newcommand{\infour}{\mathsf{in}_4}
\newcommand{\infive}{\mathsf{in}_5}


\newcommand{\delay}{\text{Delay}}
\newcommand{\ledelay}{\le^{\text{Del}}}
\newcommand{\bisimdelay}{\bisim^{\text{Del}}}
\newcommand{\tnow}{\mathsf{now}}
\newcommand{\tlater}{\mathsf{later}}
\newcommand{\Da}{\Downarrow}



\begin{document}

\title{Denotational Semantics for Gradual Typing in Synthetic Guarded Domain Theory}
\author{Eric Giovannini}
\author{Max S. New}


\begin{abstract}
    We develop a denotational semantics for a gradually typed language
    with effects that is adequate and proves the graduality theorem.
    %
    The denotational semantics is constructed using \emph{synthetic
    guarded domain theory} working in a type theory with a later
    modality and clock quantification.
    %
    This provides a remarkably simple presentation of the semantics,
    where gradual types are interpreted as ordinary types in our ambient
    type theory equipped with an ordinary preorder structure to model
    the error ordering.
    %
    This avoids the complexities of classical domain-theoretic models
    (New and Licata) or logical relations models using explicit
    step-indexing (New and Ahmed).
    %
    In particular, we avoid a major technical complexity of New and
    Ahmed that requires two logical relations to prove the graduality
    theorem.
  
    By working synthetically we can treat the domains in which gradual
    types are interpreted as if they were ordinary sets. This allows us
    to give a ``na\"ive'' presentation of gradual typing where each
    gradual type is modeled as a well-behaved subset of the universal
    domain used to model the dynamic type, and type precision is modeled
    as simply a subset relation.
    %
  \end{abstract}

  \maketitle
  
  \section{Introduction}
  
  % gradual typing, graduality
  \subsection{Gradual Typing and Graduality}
  Gradual typing allows a language to have both statically-typed and dynamically-typed terms;
  the statically-typed terms are type checked at compile time, while type checking for the 
  dynamically-typed terms is deferred to runtime.

  Gradually-typed languages should satisfy two intuitive properties.
  First, the interaction
  between the static and dynamic components of the codebase should be safe -- i.e., should
  preserve the guarantees made by the static types.
  In other words, in the static portions of the codebase, type soundness must be preserved.
  Second, gradual langugaes should support the smooth migration from dynamic typing
  to static typing, in that the programmer can initially leave off the
  typing annotations and provide them later without altering the meaning of the
  program.

  % Graduality property
  Formally speaking, gradually typed languages should satisfy the 
  \emph{dynamic gradual guarantee}, originally defined by Siek, Vitousek, Cimini,
  and Boyland \cite{siek_et_al:LIPIcs:2015:5031}.
  This property is also referred to as \emph{graduality}, by analogy with parametricity.
  Intuitively, graduality says that in going from a dynamic to static style should not
  introduce changes in the meaning of the program.
  More specifically, making the types more precise by adding annotations
  % (replacing $\dyn$ with a more specific type
  %such as integers)
  will either result in the same behavior as the less precise program,
  or result in a type error.

  
  \subsection{Current Approaches}
  Current approaches to proving graduality include the methods of Abstracting Gradual Typing
  \cite{garcia-clark-tanter2016} and the formal tools of the Gradualier \cite{cimini-siek2016}.
  These allow the language developer to start with a statically typed langauge and derive a
  gradually typed language that satisfies the gradual guarantee. The downside is that
  not all gradually typed languages can be derived from these frameworks, and moreover, in both
  approaches the semantics is derived from the static type system as opposed to the alternative
  in which the semantics determines the type checking. Without a clear semantic interpretation of type
  dynamism, it becomes difficult to extend these techniques to new language features such as polymorphism.

  % Proving graduality via LR
  New and Ahmed \cite{new-ahmed2018}
  have developed a semantic approach to specifying type dynamism in terms of
  \emph{embedding-projection pairs}, which allows for a particularly elegant formulation of the
  gradual guarantee.
  Moreover, their axiomatic account of program equivalence allows for type-based reasoning
  about program equivalences.
%
  In this approach, a logical relation is constructed and shown to be sound with respect to
  the notion of observational approximation that specifies when one program is more precise than another.
  The downside of this approach is that each new language requires a different logical relation
  to prove graduality. Furthermore, the logical relations tend to be quite complicated due
  to a technical requirement known as \emph{step-indexing}.
  As a result, developments using this approach tend to require vast effort, with the
  corresponding technical reports having 50+ pages of proofs.

  An alternative approach, which we investigate in this paper, is provided by
  \emph{synthetic guarded domain theory}.
  The tecnhiques of synthetic guarded domain theory allow us to internalize the
  step-index reasoning normally required in logical relations proofs of graduality,
  ultimately allowing us to specify the logical relation in a manner that looks nearly
  identical to a typical, non-step-indexed logical relation.

  In this paper, we report on work we have done to mechanize graduality proofs
  using SGDT techniques in Agda.
  Our hope in this work is that by mechanizing a graduality proof in a reusable way,
  we will provide a framework for other language designers to use to more easily and
  conveniently prove that their languages satsify graduality.


  \subsection{Contributions}
  Our main contribution is a framework in Guarded Cubical Agda for proving graduality of a cast calculus.
  To demonstrate the feasability and utility of our approach, we have used the framework to prove
  graduality for the simply-typed gradual lambda calculus.Along the way, we have developed an ``intensional" theory
  of graduality that is of independent interest.


  \subsection{Proving Graduality in SGDT}
  \textbf{TODO:} This section should probably be moved to after the relevant background has been introduced.

  
  In this paper, we will utilize SGDT techniques to prove graduality for a particularly
  simple gradually-typed cast calculus, the gradually-typed lambda calculus. This is just
  the usual simply-typed lambda calculus with a dynamic type $\dyn$ such that $A \ltdyn\, \dyn$ for all types $A$,
  as well as upcasts and downcasts between any types $A$ and $B$ such that $A \ltdyn B$.
  The complete definition will be provided in Section \ref{sec:GTLC}.

  Our main theorem is the following:

  
  % \begin{theorem}[Graduality]
  %   If $M \ltdyn N : \nat$, then either:
  %   \begin{enumerate}
  %     \item $M = \mho$
  %     \item $M = N = \zro$
  %     \item $M = N = \suc n$ for some $n$
  %     \item $M$ and $N$ diverge
  %   \end{enumerate}
  % \end{theorem}
  

  \begin{theorem}[Graduality]
    If $\cdot \vdash M \ltdyn N : \nat$, then
    \begin{enumerate}
      \item If $N = \mho$, then $M = \mho$
      \item If $N = `n$, then $M = \mho$ or $M = `n$
      \item If $M = V$, then $N = V$
    \end{enumerate}
  \end{theorem}

  We also should be able to show that $\mho$, $\zro$, and $\suc\, N$ are not equal.

  Our first step toward proving graduality is to formulate an \emph{intensional} gradual lambda calculus,
  which we call $\intlc$, in which the computation steps taken by a term are made explicit.
  The ``normal'' graudal lambda calculus for which we want to prove graduality will be called the
  \emph{extensional} gradual lambda calculus, denoted $\extlc$. We will define an erasure function
  $\erase{\cdot} : \intlc \to \extlc$ which takes a program in the intensional lambda calculus
  and ``forgets'' the syntactic information about the steps to produce a term in the extensional calculus.

  Every term $M_e$ in $\extlc$ will have a corresponding program $M_i$ in $\intlc$ such that
  $\erase{M_i} = M_e$. Moreover, we will show that if $M_e \ltdyn_e M_e'$ in the extensional theory,
  then there exists terms $M_i$ and $M_i'$ such that $\erase{M_i}=M_e$, $\erase{M_i'}=M_e'$ and
  $M_i \ltdyn_i M_i'$ in the intensional theory.

  We formulate and prove an analogous graduality theorem for the intensional lambda calculus.
  We define an interpretation of the intensional lambda calculus into a model in which we prove
  various results. Using the observation above, given $M_e \ltdyn M_e' : \nat$, we can find
  intensional programs $M_i$ and $M_i'$ that erase to them and are such that $M_i \ltdyn M_i'$.
  We will then apply the intensional graduality theorem to $M_i$ and $M_i'$, and translate the result
  back to $M_e$ and $M_e'$.


  \subsection{Overview of Remainder of Paper}

  In Section \ref{sec:technical-background}, we provide technical background on gradually typed languages and
  on synthetic guarded domain theory.
 % 
  In Section \ref{sec:GTLC}, we introduce the gradually-typed cast calculus
  for which we will prove graduality. Important here are the notions of syntactic
  type precision and term precision. We introduce both the \emph{extensional} gradual lambda calculus
  ($\extlc$) and the \emph{intensional} gradual lambda calculus ($\intlc$).
%
  In Section \ref{sec:domain-theory}, we define several fundamental constructions
  internal to SGDT that will be needed when we give a denotational semantics to
  our intensional lambda calculus.
  This includes the notion of Predomains as well as the concept
  of EP-Pairs.
%
  In Section \ref{sec:semantics}, we define the denotational semantics for the
  intensional gradually-typed lambda calculus using the domain theoretic constructions in the
  previous section.
%
  In Section \ref{sec:graduality}, we outline in more detail the proof of graduality for the
  extensional gradual lambda calculus, which will make use of prove properties we prove about
   the intensional gradual lambda calculus.
%
  In Section \ref{sec:discussion}, we discuss the benefits and drawbacks to our approach in comparison
  to the traditional step-indexing approach, as well as possibilities for future work.


\section{Technical Background}\label{sec:technical-background}

\subsection{Gradual Typing}



% Cast calculi
In a gradually-typed language, the mixing of static and dynamic code is seamless, in that
the dynamically typed parts are checked at runtime. This type checking occurs at the elimination
forms of the language (e.g., pattern matching, field reference, etc.).
Gradual languages are generally elaborated to a \emph{cast calculus}, in which the dynamic
type checking is made explicit through the insertion of \emph{type casts}.

% Up and down casts
In a cast calculus, there is a relation $\ltdyn$ on types such that $A \ltdyn B$ means that
$A$ is a \emph{more precise} type than $B$.
There a dynamic type $\dyn$ with the property that $A \ltdyn \dyn$ for all $A$.
%
If $A \ltdyn B$, a term $M$ of type $A$ may be \emph{up}casted to $B$, written $\up A B M$,
and a term $N$ of type $B$ may be \emph{down}casted to $A$, written $\dn A B N$.
Upcasts always succeed, while downcasts may fail at runtime.
%
% Syntactic term precision
We also have a notion of \emph{syntatcic term precision}.
If $A \ltdyn B$, and $M$ and $N$ are terms of type $A$ and $B$ respectively, we write
$M \ltdyn N : A \ltdyn B$ to mean that $M$ is more precise than $N$, i.e., $M$ and $N$
behave the same except that $M$ may error more.

% Modeling the dynamic type as a recursive sum type?
% Observational equivalence and approximation?

% synthetic guarded domain theory, denotational semantics therein

\subsection{Difficulties in Prior Semantics}

 % TODO move to another section?
  % Difficulties in prior semantics
  % - two separate logical relations for expressions
  % - transitivity

  In this work, we compare our approach to proving graduality to the approach
  introduced by New and Ahmed \cite{new-ahmed2018} which constructs a step-indexed
  logical relations model and shows that this model is sound with respect to their
  notion of contextual error approximation.

  Because the dynamic type is modeled as a non-well-founded
  recursive type, their logical relation needs to be paramterized by natural numbers
  to restore well-foundedness. This technique is known as a \emph{step-indexed logical relation}.
  Reasoning about step-indexed logical relations
  can be tedious and error-prone, and there are some very subtle aspects that must
  be taken into account in the proofs. Figure \ref{TODO} shows an example of a step-indexed logical
  relation for the gradually-typed lambda calculus.

  In particular, the prior approach of New and Ahmed requires two separate logical
  relations for terms, one in which the steps of the left-hand term are counted,
  and another in which the steps of the right-hand term are counted.
  Then two terms $M$ and $N$ are related in the ``combined'' logical relation if they are
  related in both of the one-sided logical relations. Having two separate logical relations
  complicates the statement of the lemmas used to prove graduality, becasue any statement that
  involves a term stepping needs to take into account whether we are counting steps on the left
  or the right. Some of the differences can be abstracted over, but difficulties arise for properties %/results
  as fundamental and seemingly straightforward as transitivty.

  Specifically, for transitivity, we would like to say that if $M$ is related to $N$ at
  index $i$ and $N$ is related to $P$ at index $i$, then $M$ is related to $P$ at $i$.
  But this does not actually hold: we requrie that one of the two pairs of terms
  be related ``at infinity'', i.e., that they are related at $i$ for all $i \in \mathbb{N}$.
  Which pair is required to satisfy this depends on which logical relation we are considering,
  (i.e., is it counting steps on the left or on the right),
  and so any argument that uses transitivity needs to consider two cases, one
  where $M$ and $N$ must be shown to be related for all $i$, and another where $N$ and $P$ must
  be related for all $i$. This may not even be possible to show in some scenarios!

  % These complications introduced by step-indexing lead one to wonder whether there is a
  % way of proving graduality without relying on tedious arguments involving natural numbers.
  % An alternative approach, which we investigate in this paper, is provided by
  % \emph{synthetic guarded domain theory}, as discussed below.
  % Synthetic guarded domain theory allows the resulting logical relation to look almost
  % identical to a typical, non-step-indexed logical relation.

\subsection{Synthetic Guarded Domain Theory}
One way to avoid the tedious reasoning associated with step-indexing is to work
axiomatically inside of a logical system that can reason about non-well-founded recursive
constructions while abstracting away the specific details of step-indexing required
if we were working analytically.
The system that proves useful for this purpose is called \emph{synthetic guarded
domain theory}, or SGDT for short. We provide a brief overview here, but more
details can be found in \cite{birkedal-mogelberg-schwinghammer-stovring2011}.

SGDT offers a synthetic approach to domain theory that allows for guarded recursion
to be expressed syntactically via a type constructor $\later : \type \to \type$ 
(pronounced ``later''). The use of a modality to express guarded recursion
was introduced by Nakano \cite{Nakano2000}.
%
Given a type $A$, the type $\later A$ represents an element of type $A$
that is available one time step later. There is an operator $\nxt : A \to\, \later A$
that ``delays'' an element available now to make it available later.
We will use a tilde to denote a term of type $\later A$, e.g., $\tilde{M}$.

% TODO later is an applicative functor, but not a monad

There is a \emph{guarded fixpoint} operator

\[
  \fix : \forall T, (\later T \to T) \to T.
\]

That is, to construct a term of type $T$, it suffices to assume that we have access to
such a term ``later'' and use that to help us build a term ``now''.
This operator satisfies the axiom that $\fix f = f (\nxt (\fix f))$.
In particular, this axiom applies to propositions $P : \texttt{Prop}$; proving
a statement in this manner is known as $\lob$-induction.

The operators $\later$, $\next$, and $\fix$ described above can be indexed by objects
called \emph{clocks}. A clock serves as a reference relative to which steps are counted.
For instance, given a clock $k$ and type $T$, the type $\later^k T$ represents a value of type
$T$ one unit of time in the future according to clock $k$.
If we only ever had one clock, then we would not need to bother defining this notion.
However, the notion of \emph{clock quantification} is crucial for encoding coinductive types using guarded
recursion, an idea first introduced by Atkey and McBride \cite{atkey-mcbride2013}.


% Clocked Cubical Type Theory
\subsubsection{Ticked Cubical Type Theory}
% TODO motivation for Clocked Cubical Type Theory, e.g., delayed substitutions?

In Ticked Cubical Type Theory \cite{TODO}, there is an additional sort
called \emph{ticks}. Given a clock $k$, a tick $t : \tick k$ serves
as evidence that one unit of time has passed according to the clock $k$.
The type $\later A$ is represented as a function from ticks of a clock $k$ to $A$.
The type $A$ is allowed to depend on $t$, in which case we write $\later^k_t A$
to emphasize the dependence.

% TODO next as a function that ignores its input tick argument?

The rules for tick abstraction and application are similar to those of dependent
$\Pi$ types. 
In particular, if we have a term $M$ of type $\later^k A$, and we
have available in the context a tick $t' : \tick k$, then we can apply the tick to $M$ to get
a term $M[t'] : A[t'/t]$. We will also write tick application as $M_t$.
Conversely, if in a context $\Gamma, t : \tick k$ we have that $M$ has type $A$,
then in context $\Gamma$ we have $\lambda t.M$ has type $\later A$. % TODO dependent version?

The statements in this paper have been formalized in a variant of Agda called
Guarded Cubical Agda \cite{TODO}, an implementation of Clocked Cubical Type Theory.


% TODO axioms (clock irrelevance, tick irrelevance)?


\subsubsection{The Topos of Trees Model}

The topos of trees model provides a useful intuition for reasoning in SGDT 
\cite{birkedal-mogelberg-schwinghammer-stovring2011}.
This section presupposes knowledge of category theory and can be safely skipped without
disrupting the continuity.

The topos of trees $\calS$ is the presheaf category $\Set^{\omega^o}$. 
%
We assume a universe $\calU$ of types, with encodings for operations such
as sum types (written as $\widehat{+}$). There is also an operator 
$\laterhat \colon \later \calU \to \calU$ such that 
$\El(\laterhat(\nxt A)) =\,\, \later \El(A)$, where $\El$ is the type corresponding
to the code $A$.

An object $X$ is a family $\{X_i\}$ of sets indexed by natural numbers,
along with restriction maps $r^X_i \colon X_{i+1} \to X_i$ (see Figure \ref{fig:topos-of-trees-object}).
These should be thought of as ``sets changing over time", where $X_i$ is the view of the set if we have $i+1$
time steps to reason about it.
We can also think of an ongoing computation, with $X_i$ representing the potential results of the computation
after it has run for $i+1$ steps.

\begin{figure}
  % https://q.uiver.app/?q=WzAsNSxbMiwwLCJYXzIiXSxbMCwwLCJYXzAiXSxbMSwwLCJYXzEiXSxbMywwLCJYXzMiXSxbNCwwLCJcXGRvdHMiXSxbMiwxLCJyXlhfMCIsMl0sWzAsMiwicl5YXzEiLDJdLFszLDAsInJeWF8yIiwyXSxbNCwzLCJyXlhfMyIsMl1d
\[\begin{tikzcd}[ampersand replacement=\&]
	{X_0} \& {X_1} \& {X_2} \& {X_3} \& \dots
	\arrow["{r^X_0}"', from=1-2, to=1-1]
	\arrow["{r^X_1}"', from=1-3, to=1-2]
	\arrow["{r^X_2}"', from=1-4, to=1-3]
	\arrow["{r^X_3}"', from=1-5, to=1-4]
\end{tikzcd}\]
  \caption{An example of an object in the topos of trees.}
  \label{fig:topos-of-trees-object}
\end{figure}

A morphism from $\{X_i\}$ to $\{Y_i\}$ is a family of functions $f_i \colon X_i \to Y_i$
that commute with the restriction maps in the obvious way, that is,
$f_i \circ r^X_i = r^Y_i \circ f_{i+1}$ (see Figure \ref{fig:topos-of-trees-morphism}).

\begin{figure}
% https://q.uiver.app/?q=WzAsMTAsWzIsMCwiWF8yIl0sWzAsMCwiWF8wIl0sWzEsMCwiWF8xIl0sWzMsMCwiWF8zIl0sWzQsMCwiXFxkb3RzIl0sWzAsMSwiWV8wIl0sWzEsMSwiWV8xIl0sWzIsMSwiWV8yIl0sWzMsMSwiWV8zIl0sWzQsMSwiXFxkb3RzIl0sWzIsMSwicl5YXzAiLDJdLFswLDIsInJeWF8xIiwyXSxbMywwLCJyXlhfMiIsMl0sWzQsMywicl5YXzMiLDJdLFs2LDUsInJeWV8wIiwyXSxbNyw2LCJyXllfMSIsMl0sWzgsNywicl5ZXzIiLDJdLFs5LDgsInJeWV8zIiwyXSxbMSw1LCJmXzAiLDJdLFsyLDYsImZfMSIsMl0sWzAsNywiZl8yIiwyXSxbMyw4LCJmXzMiLDJdXQ==
\[\begin{tikzcd}[ampersand replacement=\&]
	{X_0} \& {X_1} \& {X_2} \& {X_3} \& \dots \\
	{Y_0} \& {Y_1} \& {Y_2} \& {Y_3} \& \dots
	\arrow["{r^X_0}"', from=1-2, to=1-1]
	\arrow["{r^X_1}"', from=1-3, to=1-2]
	\arrow["{r^X_2}"', from=1-4, to=1-3]
	\arrow["{r^X_3}"', from=1-5, to=1-4]
	\arrow["{r^Y_0}"', from=2-2, to=2-1]
	\arrow["{r^Y_1}"', from=2-3, to=2-2]
	\arrow["{r^Y_2}"', from=2-4, to=2-3]
	\arrow["{r^Y_3}"', from=2-5, to=2-4]
	\arrow["{f_0}"', from=1-1, to=2-1]
	\arrow["{f_1}"', from=1-2, to=2-2]
	\arrow["{f_2}"', from=1-3, to=2-3]
	\arrow["{f_3}"', from=1-4, to=2-4]
\end{tikzcd}\]
  \caption{An example of a morphism in the topos of trees.}
  \label{fig:topos-of-trees-morphism}
\end{figure}


The type operator $\later$ is defined on an object $X$ by
$(\later X)_0 = 1$ and $(\later X)_{i+1} = X_i$.
The restriction maps are given by $r^\later_0 =\, !$, where $!$ is the
unique map into $1$, and $r^\later_{i+1} = r^X_i$.
The morphism $\nxt^X \colon X \to \later X$ is defined pointwise by
$\nxt^X_0 =\, !$, and $\nxt^X_{i+1} = r^X_i$. It is easily checked that
this satisfies the commutativity conditions required of a morphism in $\calS$.
%
Given a morphism $f \colon \later X \to X$, i.e., a family of functions
$f_i \colon (\later X)_i \to X_i$ that commute with the restrictions in the appropriate way,
we define $\fix(f) \colon 1 \to X$ pointwise
by $\fix(f)_i = f_{i} \circ \dots \circ f_0$.
This can be visualized as a diagram in the category of sets as shown in
Figure \ref{fig:topos-of-trees-fix}.
% Observe that the fixpoint is a \emph{global element} in the topos of trees.
% Global elements allow us to view the entire computation on a global level.


% https://q.uiver.app/?q=WzAsOCxbMSwwLCJYXzAiXSxbMiwwLCJYXzEiXSxbMywwLCJYXzIiXSxbMCwwLCIxIl0sWzAsMSwiWF8wIl0sWzEsMSwiWF8xIl0sWzIsMSwiWF8yIl0sWzMsMSwiWF8zIl0sWzMsNCwiZl8wIl0sWzAsNSwiZl8xIl0sWzEsNiwiZl8yIl0sWzIsNywiZl8zIl0sWzAsMywiISIsMl0sWzEsMCwicl5YXzAiLDJdLFsyLDEsInJeWF8xIiwyXSxbNSw0LCJyXlhfMCJdLFs2LDUsInJeWF8xIl0sWzcsNiwicl5YXzIiXV0=
% \[\begin{tikzcd}[ampersand replacement=\&]
% 	1 \& {X_0} \& {X_1} \& {X_2} \\
% 	{X_0} \& {X_1} \& {X_2} \& {X_3}
% 	\arrow["{f_0}", from=1-1, to=2-1]
% 	\arrow["{f_1}", from=1-2, to=2-2]
% 	\arrow["{f_2}", from=1-3, to=2-3]
% 	\arrow["{f_3}", from=1-4, to=2-4]
% 	\arrow["{!}"', from=1-2, to=1-1]
% 	\arrow["{r^X_0}"', from=1-3, to=1-2]
% 	\arrow["{r^X_1}"', from=1-4, to=1-3]
% 	\arrow["{r^X_0}", from=2-2, to=2-1]
% 	\arrow["{r^X_1}", from=2-3, to=2-2]
% 	\arrow["{r^X_2}", from=2-4, to=2-3]
% \end{tikzcd}\]

% \begin{figure}
%   % https://q.uiver.app/?q=WzAsMTksWzEsMiwiWF8wIl0sWzIsMywiWF8xIl0sWzMsMSwiMSJdLFswLDEsIjEiXSxbMCwyLCJYXzAiXSxbMSwzLCJYXzEiXSxbMSwxLCIxIl0sWzIsMSwiMSJdLFsyLDIsIlhfMCJdLFsyLDQsIlhfMiJdLFszLDIsIlhfMCJdLFszLDMsIlhfMSJdLFszLDQsIlhfMiJdLFszLDUsIlhfMyJdLFs0LDIsIlxcY2RvdHMiXSxbMCwwLCJcXGZpeChmKV8wIl0sWzEsMCwiXFxmaXgoZilfMSJdLFsyLDAsIlxcZml4KGYpXzIiXSxbMywwLCJcXGZpeChmKV8zIl0sWzMsNCwiZl8wIl0sWzAsNSwiZl8xIl0sWzYsMCwiZl8wIl0sWzcsOCwiZl8wIl0sWzgsMSwiZl8xIl0sWzEsOSwiZl8yIl0sWzIsMTAsImZfMCJdLFsxMCwxMSwiZl8xIl0sWzExLDEyLCJmXzIiXSxbMTIsMTMsImZfMyJdXQ==
%   \[\begin{tikzcd}[ampersand replacement=\&]
%     {\fix(f)_0} \& {\fix(f)_1} \& {\fix(f)_2} \& {\fix(f)_3} \\
%     1 \& 1 \& 1 \& 1 \\
%     {X_0} \& {X_0} \& {X_0} \& {X_0} \& \cdots \\
%     \& {X_1} \& {X_1} \& {X_1} \\
%     \&\& {X_2} \& {X_2} \\
%     \&\&\& {X_3}
%     \arrow["{f_0}", from=2-1, to=3-1]
%     \arrow["{f_1}", from=3-2, to=4-2]
%     \arrow["{f_0}", from=2-2, to=3-2]
%     \arrow["{f_0}", from=2-3, to=3-3]
%     \arrow["{f_1}", from=3-3, to=4-3]
%     \arrow["{f_2}", from=4-3, to=5-3]
%     \arrow["{f_0}", from=2-4, to=3-4]
%     \arrow["{f_1}", from=3-4, to=4-4]
%     \arrow["{f_2}", from=4-4, to=5-4]
%     \arrow["{f_3}", from=5-4, to=6-4]
%   \end{tikzcd}\]
%   \caption{The first few approximations to the guarded fixpoint of $f$.}
%   \label{fig:topos-of-trees-fix-approx}
% \end{figure}


\begin{figure}
  % https://q.uiver.app/?q=WzAsNixbMywwLCIxIl0sWzAsMiwiWF8wIl0sWzIsMiwiWF8xIl0sWzQsMiwiWF8yIl0sWzYsMiwiWF8zIl0sWzgsMiwiXFxkb3RzIl0sWzIsMSwicl5YXzAiXSxbNCwzLCJyXlhfMiJdLFswLDEsIlxcZml4KGYpXzAiLDFdLFswLDIsIlxcZml4KGYpXzEiLDFdLFswLDMsIlxcZml4KGYpXzIiLDFdLFswLDQsIlxcZml4KGYpXzMiLDFdLFswLDUsIlxcY2RvdHMiLDMseyJzdHlsZSI6eyJib2R5Ijp7Im5hbWUiOiJub25lIn0sImhlYWQiOnsibmFtZSI6Im5vbmUifX19XSxbMywyLCJyXlhfMSJdLFs1LDQsInJeWF8zIl1d
  \[\begin{tikzcd}[ampersand replacement=\&,column sep=2.25em]
    \&\&\& 1 \\
    \\
    {X_0} \&\& {X_1} \&\& {X_2} \&\& {X_3} \&\& \dots
    \arrow["{r^X_0}", from=3-3, to=3-1]
    \arrow["{r^X_2}", from=3-7, to=3-5]
    \arrow["{\fix(f)_0}"{description}, from=1-4, to=3-1]
    \arrow["{\fix(f)_1}"{description}, from=1-4, to=3-3]
    \arrow["{\fix(f)_2}"{description}, from=1-4, to=3-5]
    \arrow["{\fix(f)_3}"{description}, from=1-4, to=3-7]
    \arrow["\cdots"{marking}, draw=none, from=1-4, to=3-9]
    \arrow["{r^X_1}", from=3-5, to=3-3]
    \arrow["{r^X_3}", from=3-9, to=3-7]
  \end{tikzcd}\]
  \caption{The guarded fixpoint of $f$.}
  \label{fig:topos-of-trees-fix}
\end{figure}

% TODO global elements?


\section{GTLC}\label{sec:GTLC}

Here we describe the syntax and typing for the graudally-typed lambda calculus.
We also give the rules for syntactic type and term precision.
We define two separate calculi: the normal gradually-typed lambda calculus, which we
call the extensional lambda calculus ($\extlc$), as well as an \emph{intensional} lambda calculus
($\intlc$) whose syntax makes explicit the steps taken by a program.

\subsection{Syntax}


\begin{align*}
  &\text{Types } A, B := \nat, \, \dyn, \, (A \To B) \\
  &\text{Terms } M, N := \err_A, \, \zro, \,  \suc\, M, \, (\lda{x}{M}), \, (M\, N), \, (\up{A}{B} M), \, (\dn{A}{B} M) \\
  &\text{Contexts } \Gamma := \cdot, \, (\Gamma, x : A)
\end{align*}

The typing rules are as expected, with a cast between $A$ to $B$ allowed only when $A \ltdyn B$.

\begin{mathpar}
  \inferrule*{ }{\hasty \Gamma {\err_A} A}

  \inferrule*{ }{\hasty \Gamma \zro \nat}

  \inferrule {\hasty \Gamma M \nat} {\hasty \Gamma {\suc\, M} \nat}

  \inferrule* 
    {\hasty {\Gamma, x : A} M B} 
    {\hasty \Gamma {\lda x M} {A \To B}}

  \inferrule *
    {\hasty \Gamma M {A \To B} \and \hasty \Gamma N A}
    {\hasty \Gamma {M \, N} B}

  \inferrule*
    {A \ltdyn B \and \hasty \Gamma M A}
    {\hasty \Gamma {\up A B M} B }

  \inferrule*
    {A \ltdyn B \and \hasty \Gamma M B}
    {\hasty \Gamma {\dn A B M} A}
\end{mathpar}

The equational theory is also as expected, with $\beta$ and $\eta$ laws for function type.

\subsection{Type Precision}

The rules for type precision are as follows:

\begin{mathpar}
  \inferrule*[right = \dyn]
    { }{\dyn \ltdyn \dyn}

  \inferrule*[right = \nat]
    { }{\nat \ltdyn \nat}

  \inferrule*[right = $Inj_\nat$]
    { }{\nat \ltdyn \dyn}

  \inferrule*[right = $\To$]
    {A_i \ltdyn B_i \and A_o \ltdyn B_o }
    {(A_i \To A_o) \ltdyn (B_i \To B_o)}

  \inferrule*[right=$Inj_{\To}$]
    {(A_i \to A_o) \ltdyn (\dyn \To \dyn)}
    {(A_i \to A_o) \ltdyn\, \dyn}

  
\end{mathpar}

% Type precision derivations
Note that as a consequence of this presentation of the type precision rules, we
have that if $A \ltdyn B$, there is a unique precision derivation that witnesses this.
As in previous work, we go a step farther and make these derivations first-class objects,
known as \emph{type precision derivations}.
Specifically, for every $A \ltdyn B$, we have a derivation $d : A \ltdyn B$ that is constructed
using the rules above. For instance, there is a derivation $\dyn : \dyn \ltdyn \dyn$, and a derivation
$\nat : \nat \ltdyn \nat$, and if $d_i : A_i \ltdyn B_i$ and $d_o : A_o \ltdyn B_o$, then
there is a derivation $d_i \To d_o : (A_i \To A_o) \ltdyn (B_i \To B_o)$. Likewise for
the remaining two rules. The benefit to making these derivations explicit in the syntax is that we
can perform induction over them.
Note also that for any type $A$, we use $A$ to denote the reflexivity derivation that $A \ltdyn A$,
i.e., $A : A \ltdyn A$.
Finally, observe that for type precision derivations $d : A \ltdyn B$ and $d' : B \ltdyn C$, we
can define their composition $d' \circ d : A \ltdyn C$. This will be used in the statement
of transitivity of the term precision relation.

\subsection{Term Precision}

We allow for a \emph{heterogeneous} term precision judgment on terms $M$ of type $A$ and $N$ of type $B$,
provided that $A \ltdyn B$ holds.

% Type precision contexts
In order to deal with open terms, we will need the notion of a type precision \emph{context}, which we denote
$\gamlt$. This is similar to a normal context but instead of mapping variables to types,
it maps variables $x$ to related types $A \ltdyn B$, where $x$ has type $A$ in the left-hand term
and $B$ in the right-hand term. We may also write $x : d$ where $d : A \ltdyn B$ to indicate this.
Another way of thinking of type precision contexts is as a zipped pair of contexts $\Gamma^l, \Gamma^r$
with the same domain such that $\Gamma^l(x) \ltdyn \Gamma^r(x)$ for each $x$ in the domain.

The rules for term precision come in two forms. We first have the \emph{congruence} rules,
one for each term constructor. These assert that the term constructors respect term precision.
The congruence rules are as follows:


\begin{mathpar}
  \inferrule*[right = $\err$]
    { \hasty \Gamma M A }
    {\etmprec {\gamlt} {\err_A} M A}

  \inferrule*[right = Var]
    { d : A \ltdyn B \and \gamlt(x) = (A, B) } { \etmprec {\gamlt} x x d  }

  \inferrule*[right = Zro]
    { } {\etmprec \gamlt \zro \zro \nat }

  \inferrule*[right = Suc]
    { \etmprec \gamlt M N \nat } {\etmprec \gamlt {\suc\, M} {\suc\, N} \nat}

  \inferrule*[right = Lambda]
    { d_i : A_i \ltdyn B_i \and 
      d_o : A_o \ltdyn B_o \and 
      \etmprec {\gamlt , x : d_i} M N {d_o} } 
    { \etmprec \gamlt {\lda x M} {\lda x N} (d_i \To d_o) }

  \inferrule*[right = App]
    { d_i : A_i \ltdyn B_i \and
      d_o : A_o \ltdyn B_o \\\\
      \etmprec \gamlt {M} {M'} (d_i \To d_o) \and
      \etmprec \gamlt {N} {N'} d_i
    } 
    { \etmprec \gamlt {M\, N} {M'\, N'} {d_o}}
\end{mathpar}

We then have additional equational axioms, including transitivity, $\beta$ and $\eta$ laws, and
rules characterizing upcasts as least upper bounds, and downcasts as greatest lower bounds.

We write $M \equidyn N$ to mean that both $M \ltdyn N$ and $N \ltdyn M$.

% TODO
\begin{mathpar}
  \inferrule*[right = Transitivity]
    {\etmprec \gamlt M N d \and
     \etmprec \gamlt N P {d'} } 
    {\etmprec \gamlt M P {d' \circ d} }

  \inferrule*[right = $\beta$]
    { \hasty {\Gamma, x : A_i} M {A_o} \and
      \hasty {\Gamma} V {A_i} } 
    { \etmequidyn \gamlt {(\lda x M)\, V} {M[V/x]} {A_o} }

  \inferrule*[right = $\eta$]
    { \hasty {\Gamma} {V} {A_i \To A_o} } 
    { \etmequidyn \gamlt {\lda x (V\, x)} V {A_i \To A_o} }

  \inferrule*[right = UpR]
    { d : A \ltdyn B \and 
      \hasty \Gamma M A } 
    { \etmprec \gamlt M {\up A B M} d  }

  \inferrule*[right = UpL]
    { d : A \ltdyn B \and
      \etmprec \gamlt M N d } 
    { \etmprec \gamlt {\up A B M} N B }

  \inferrule*[right = DnL]
    { d : A \ltdyn B \and 
      \hasty \Gamma M B } 
    { \etmprec \gamlt {\dn A B M} M d }

  \inferrule*[right = DnR]
    { d : A \ltdyn B \and
      \etmprec \gamlt M N d } 
    { \etmprec \gamlt M {\dn A B N} A }
\end{mathpar}

% TODO explain the least upper bound/greatest lower bound rules
The rules UpR, UpL, DnL, and DnR were introduced in \cite{TODO} as a means
of cleanly axiomatizing the intended behavior of casts in a way that
doesn't depend on the specific constructs of the language.
Intuitively, rule UpR says that the upcast of $M$ is an upper bound for $M$
in that $M$ may error more, and UpL says that the upcast is the \emph{least}
such upper bound, in that it errors more than any other upper bound for $M$.
Conversely, DnL says that the downcast of $M$ is a lower bound, and DnR says
that it is the \emph{greatest} lower bound.
% These rules provide a clean axiomatization of the behavior of casts that doesn't
% depend on the specific constructs of the language.


\subsection{The Intensional Lambda Calculus}

\textbf{TODO: Subject to change!}

Now that we have described the syntax along with the type and term precision judgments for
$\extlc$, we can now do the same for $\intlc$. One key difference between the two calculi
is that we define $\intlc$ using the constructs available to us in the language of synthetic
guarded domain theory, e.g., we use the $\later$ operator. Whereas when we defined the syntax
of the extensional lambda calculus we were working in the category of sets, when we define the
syntax of the intensional lambda calculus we will be working in the topos of trees. 
% in a presheaf category known as the \emph{topos of trees}.

More specifically, in $\intlc$, we not only have normal terms, but also terms available
``later'', which we denote by $\tilde{M}$.
We have a term constructor $\theta_s$ which takes a later-term and turns it into
a term avaialble now. 
The typing and precision rules for $\theta_s$ involve the $\later$ operator, as shown below.
Observe that $\theta_s$ is a syntactic analogue to the $\theta$ constructor of the lifting monad
that we will define in the section on domain theoretic constructions (Section \ref{sec:domain-theory}),
but it is important to note that $\theta_s (\tilde{M})$ is an actual term in $\intlc$, whereas the
$\theta$ constructor is a purely semantic construction. These will be connected when we discuss the
interpretation of $\intlc$ into the semantic model.

% Most aspects of the two calculi are the same, except that in $\intlc$ we add a new term constructor
% $\theta_s$ which represents a syntactic way of delaying a term by one computation step.

To better understand this situation, note that $\lda{x}{(\cdot)}$ can be viewed as a function
(at the level of the ambient type theory) from terms of type $B$ under context $\Gamma, x : A$
to terms of type $A \To B$ under context $\Gamma$. Similarly, we can view $\theta_s(\cdot)$ as a
function (in the ambient type theory, which is sythetic guarded domain theory) taking terms
$\tilde{M}$ of type $A$ avaiable \emph{later} to terms of type $A$ available \emph{now}.

Notice that the notion of a ``term available later'' does not need to be part of the syntax
of the intensional lambda calculus, because this can be expressed in the ambient theory.
Similarly, we do not need a syntactic form to delay a term, because we can simply use $\nxt$.


\begin{align*}
  % &\text{Later Terms } \tilde{M}, \tilde{N}
  &\text{Terms } M, N := \err_A, \, \dots \, \theta_s (\tilde{M}) \\
\end{align*}

\begin{mathpar}
  \inferrule*[]
    { \later_t (\hasty \Gamma {M_t} A) }
    { \hasty \Gamma {\theta_s M} {A} }
\end{mathpar}

\begin{mathpar}
  \inferrule*[]
    { \later_t (\itmprec \gamlt {M_t} {N_t} d) }
    { \itmprec \gamlt {\theta_s M} {\theta_s N} d }
\end{mathpar}

Recall that $\later_t$ is a dependent form of $\later$ where the arugment is allowed
to mention $t$. In particular, here we apply the tick $t$ to the later-terms $M$ and $N$ to get 
``now"-terms $M_t$ and $N_t$.

Formally speaking, the term precision relation must be defined as a guarded fixpoint,
i.e., we assume that the function is defined later, and use it to construct a definition
that is defined ``now''. This involves applying a tick to the later-function to shift it
to a now-function. Indeed, this is what we do in the formal Agda development, but in this
paper, we will elide these issues as they are not relevant to the main ideas.

% TODO equational theory


\section{Domain-Theoretic Constructions}\label{sec:domain-theory}

In this section, we discuss the fundamental objects of the model into which we will embed
the intensional lambda calculus and inequational theory. It is important to remember that
the constructions in this section are entirely independent of the syntax described in the
previous section; the notions defined here exist in their own right as purely mathematical
constructs. In the next section, we will link the syntax and semantics via an interpretation
function.

\subsection{The Lift Monad}

When thinking about how to model intensional gradually-typed programs, we must consider
their possible behaviors. On the one hand, we have \emph{failure}: a program may fail
at run-time because of a type error. In addition to this, a program may ``think'',
i.e., take a step of computation. If a program thinks forever, then it never returns a value,
so we can think of the idea of thinking as a way of intensionally modelling \emph{partiality}.

With this in mind, we can describe a semantic object that models these behaviors: a monad
for embedding computations that has cases for failure and ``thinking''.
Previous work has studied such a construct in the setting of the latter only, called the lift
monad \cite{mogelberg-paviotti2016}; here, we add the additional effect of failure.

For a type $A$, we define the \emph{lift monad with failure} $\li A$, which we will just call
the \emph{lift monad}, as the following datatype:

\begin{align*}
  \li A &:= \\
  &\eta \colon A \to \li A \\
  &\mho \colon \li A \\
  &\theta \colon \later (\li A) \to \li A
\end{align*}

Formally, the lift monad $\li A$ is defined as the solution to the guarded recursive type equation

\[ \li A \cong A + 1 + \later \li A. \]

An element of $\li A$ should be viewed as a computation that can either (1) return a value (via $\eta$),
(2) raise an error and stop (via $\mho$), or (3) think for a step (via $\theta$).
%
Notice there is a computation $\fix \theta$ of type $\li A$. This represents a computation
that thinks forever and never returns a value.

Since we claimed that $\li A$ is a monad, we need to define the monadic operations
and show that they repect the monadic laws. The return is just $\eta$, and extend
is defined via by guarded recursion by cases on the input.
% It is instructive to give at least one example of a use of guarded recursion, so
% we show below how to define extend:
% TODO
%
%
Verifying that the monadic laws hold requires \lob-induction and is straightforward.

The lift monad has the following universal property. Let $f$ be a function from $A$ to $B$,
where $B$ is a $\later$-algebra, i.e., there is $\theta_B \colon \later B \to B$.
Further suppose that $B$ is also an ``error-algebra'', that is, an algebra of the
constant functor $1 \colon \text{Type} \to \text{Type}$ mapping all types to Unit.
This latter statement amounts to saying that there is a map $\text{Unit} \to B$, so $B$ has a
distinguished ``error element" $\mho_B \colon B$.

Then there is a unique homomorphism of algebras $f' \colon \li A \to B$ such that
$f' \circ \eta = f$. The function $f'(l)$ is defined via guarded fixpoint by cases on $l$. 
In the $\mho$ case, we simply return $\mho_B$.
In the $\theta(\tilde{l})$ case, we will return

\[\theta_B (\lambda t . (f'_t \, \tilde{l}_t)). \]

Recalling that $f'$ is a guaded fixpoint, it is available ``later'' and by
applying the tick we get a function we can apply ``now''; for the argument,
we apply the tick to $\tilde{l}$ to get a term of type $\li A$.


\subsubsection{Model-Theoretic Description}
We can describe the lift monad in the topos of trees model as follows.

% TODO

\subsection{Predomains}

The next important construction is that of a \emph{predomain}. A predomain is intended to
model the notion of error ordering that we want terms to have. Thus, we define a predomain $A$
as a partially-ordered set, which consists of a type which we denote $\ty{A}$ and a reflexive,
transitive, and antisymmetric relation $\le_P$ on $A$.

For each type we want to represent, we define a predomain for the corresponding semantic
type. For instance, we define a predomain for natural numbers, a predomain for the
dynamic type, a predomain for functions, and a predomain for the lift monad. We
describe each of these below.

We define monotone functions between predomain as expected. Given predomains
$A$ and $B$, we write $f \colon A \monto B$ to indicate that $f$ is a monotone
function from $A$ to $B$, i.e, for all $a_1 \le_A a_2$, we have $f(a_1) \le_B f(a_2)$.

% TODO
\begin{itemize}
  \item There is a predomain Nat for natural numbers, where the ordering is equality.
  
  \item There is a predomain Dyn to represent the dynamic type. The underlying type
  for this predomain is defined by guarded fixpoint to be such that
  $\ty{\Dyn} \cong \mathbb{N}\, +\, \later (\ty{\Dyn} \monto \ty{\Dyn})$.
  This definition is valid because the occurrences of Dyn are guarded by the $\later$.
  The ordering is defined via guarded recursion by cases on the argument, using the
  ordering on $\mathbb{N}$ and the ordering on monotone functions described below.

  \item For a predomain $A$, there is a predomain $\li A$ that is the ``lift'' of $A$
  using the lift monad. We use the same notation for $\li A$ when $A$ is a type
  and $A$ is a predomain, since the context should make clear which one we are referring to.
  The underling type of $\li A$ is simply $\li \ty{A}$, i.e., the lift of the underlying
  type of $A$.
  The ordering on $\li A$ is the ``lock-step error-ordering'' which we describe in
  \ref{subsec:lock-step}.

  \item For predomains $A_i$ and $A_o$, we form the predomain of monotone functions
  from $A_i$ to $A_o$, which we denote by $A_i \To A_o$.
  The ordering is such that $f$ is below $g$ if for all $a$ in $\ty{A_i}$, we have
  $f(a)$ is below $g(a)$ in the ordering for $A_o$. 
\end{itemize}



\subsection{Lock-step Error Ordering}\label{subsec:lock-step}

As mentioned, the ordering on the lift of a predomain $A$
is called the \emph{lock-step error-ordering}, the idea being that two computations
$l$ and $l'$ are related if they are in lock-step with regard to their
intensional behavior, up to $l$ erroring.
Formally, we define this ordering as follows:

\begin{itemize}
  \item 	$\eta\, x \ltls \eta\, y$ if $x \le_A y$.
  \item 	$\mho \ltls l$ for all $l$ 
  \item   $\theta\, \tilde{r} \ltls \theta\, \tilde{r'}$ if
          $\later_t (\tilde{r}_t \ltls \tilde{r'}_t)$
\end{itemize}

We also define a heterogeneous version of this ordering between the lifts of two
different predomains $A$ and $B$, parameterized by a relation $R$ between $A$ and $B$.

\subsection{Weak Bisimilarity Relation}

We also define another ordering on $\li A$, called ``weak bisimilarity'',
written $l \bisim l'$.
Intuitively, we say $l \bisim l'$ if they are equivalent ``up to delays''.
We introduce the notation $x \sim_A y$ to mean $x \le_A y$ and $y \le_A x$.

The weak bisimilarity relation is defined by guarded fixpoint as follows:

\begin{align*}
  &\mho \bisim \mho \\
%
  &\eta\, x \bisim \eta\, y \text{ if } 
    x \sim_A y \\
%		
  &\theta\, \tilde{x} \bisim \theta\, \tilde{y} \text{ if } 
    \later_t (\tilde{x}_t \bisim \tilde{y}_t) \\
%	
  &\theta\, \tilde{x} \bisim \mho \text{ if } 
    \theta\, \tilde{x} = \delta^n(\mho) \text { for some $n$ } \\
%	
  &\theta\, \tilde{x} \bisim \eta\, y \text{ if }
    (\theta\, \tilde{x} = \delta^n(\eta\, x))
  \text { for some $n$ and $x : \ty{A}$ such that $x \sim_A y$ } \\
%
  &\mho \bisim \theta\, \tilde{y} \text { if } 
    \theta\, \tilde{y} = \delta^n(\mho) \text { for some $n$ } \\
%	
  &\eta\, x \bisim \theta\, \tilde{y} \text { if }
    (\theta\, \tilde{y} = \delta^n (\eta\, y))
  \text { for some $n$ and $y : \ty{A}$ such that $x \sim_A y$ }
\end{align*}


\subsection{EP-Pairs}

Here, we adapt the notion of embedding-projection pair as used to study
gradual typing (\cite{new-ahmed2018}) to the setting of intensional denotaional semantics.





\section{Semantics}\label{sec:semantics}

\subsection{Relational Semantics}

\subsubsection{Types as Predomains}


\subsubsection{Terms as Monotone Functions}


\subsubsection{Type Precision as EP-Pairs}


\subsubsection{Term Precision via the Lock-Step Error Ordering}
% Homogeneous vs heterogeneous term precision

\subsection{Logical Relations Semantics}


\section{Graduality}\label{sec:graduality}
The main theorem we would like to prove is the following:

\begin{theorem}[Graduality]
  If $\cdot \vdash M \ltdyn N : \nat$, then
  \begin{enumerate}
    \item If $N = \mho$, then $M = \mho$
    \item If $N = `n$, then $M = \mho$ or $M = `n$
    \item If $M = V$, then $N = V$
  \end{enumerate}
\end{theorem}


\subsection{Outline}

\subsection{Extensional to Intensional}

\subsection{Intensional Results}

\subsection{Adequacy}

\subsection{Putting it Together}



\section{Discussion}\label{sec:discussion}

\subsection{Synthetic Ordering}

While the use of synthetic guarded domain theory allows us to very
conveniently work with non-well-founded recursive constructions while
abstracting away the precise details of step-indexing, we do work with
the error ordering in a mostly analytic fashion in that gradual types
are interpreted as sets equipped with an ordering relation, and all
terms must be proven to be monotone.
%
It is possible that a combination of synthetic guarded domain theory
with \emph{directed} type theory would allow for an a synthetic
treatment of the error ordering as well.


\bibliographystyle{ACM-Reference-Format}
\bibliography{references}

\end{document}
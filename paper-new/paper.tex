\documentclass[acmsmall]{acmart}
\let\Bbbk\relax

\usepackage{quiver}
\usepackage{mathpartir}
% \usepackage{tikz-cd}
\usepackage{enumitem}
\usepackage{wrapfig}
\usepackage{fancyvrb}
\usepackage{comment}
\usepackage{array}

\theoremstyle{remark}
\newtheorem{remark}{Remark}
\newenvironment{DIFnomarkup}{}{}
% \pagestyle{empty}
% \settopmatter{printfolios=false}


%% CCS Concepts
\begin{CCSXML}
  <ccs2012>
     <concept>
         <concept_id>10003752.10010124.10010131.10010133</concept_id>
         <concept_desc>Theory of computation~Denotational semantics</concept_desc>
         <concept_significance>500</concept_significance>
         </concept>
     <concept>
         <concept_id>10011007.10011006.10011039.10011311</concept_id>
         <concept_desc>Software and its engineering~Semantics</concept_desc>
         <concept_significance>300</concept_significance>
         </concept>
     <concept>
         <concept_id>10011007.10011006.10011008.10011009.10011012</concept_id>
         <concept_desc>Software and its engineering~Functional languages</concept_desc>
         <concept_significance>300</concept_significance>
         </concept>
   </ccs2012>
\end{CCSXML}
  
\ccsdesc[500]{Theory of computation~Denotational semantics}
\ccsdesc[300]{Software and its engineering~Semantics}
\ccsdesc[300]{Software and its engineering~Functional languages}

%% Additional Key Words and Phrases
\keywords{Gradual Typing, Denotational Semantics, Synthetic Guarded Domain Theory, Guarded Type Theory}


%%% The following is specific to POPL '25 and the paper
%%% 'Denotational Semantics of Gradual Typing using Synthetic Guarded Domain Theory'
%%% by Eric Giovannini, Tingting Ding, and Max S. New.
%%%
\setcopyright{rightsretained}
\acmDOI{10.1145/3704863}
\acmYear{2025}
\acmJournal{PACMPL}
\acmVolume{9}
\acmNumber{POPL}
\acmArticle{27}
\acmMonth{1}
\received{2024-07-11}
\received[accepted]{2024-11-07}



%% Rights management information.  This information is sent to you
%% when you complete the rights form.  These commands have SAMPLE
%% values in them; it is your responsibility as an author to replace
%% the commands and values with those provided to you when you
%% complete the rights form.

% \setcopyright{acmcopyright}
% \copyrightyear{2018}
% \acmYear{2018}
% \acmDOI{10.1145/1122445.1122456}


%% These commands are for a PROCEEDINGS abstract or paper.
% \acmConference[Woodstock '18]{Woodstock '18: ACM Symposium on Neural
%   Gaze Detection}{June 03--05, 2018}{Woodstock, NY}
% \acmBooktitle{Woodstock '18: ACM Symposium on Neural Gaze Detection,
%   June 03--05, 2018, Woodstock, NY}
% \acmPrice{15.00}
% \acmISBN{978-1-4503-XXXX-X/18/06}

\newcommand{\To}{\Rightarrow}
\newcommand{\inl}{\mathsf{inl}}
\newcommand{\inr}{\mathsf{inr}}
\newcommand{\alt}{\mathrel{\bf \,\mid\,}}


\newcommand{\extlc}{\text{Ext-}\lambda}
\newcommand{\extlcm}{\text{Ext-}\lambda^{-\text{trans}}}
\newcommand{\extlcmm}{\text{Ext-}\lambda^{-\text{trans}-\text{cast}}}
\newcommand{\extlcprime}{\text{Ext-}\lambda'}
\newcommand{\intlc}{\text{Int-}\lambda}
\newcommand{\intlcbisim}{\text{Int$_\approx$-}\lambda}
\newcommand{\erase}[1]{\lfloor {#1} \rfloor}


\newcommand{\uarrowl}{\mathrel{\rotatebox[origin=c]{-30}{$\leftarrowtail$}}}
\newcommand{\uarrowr}{\mathrel{\rotatebox[origin=c]{60}{$\leftarrowtail$}}}
\newcommand{\darrowl}{\mathrel{\rotatebox[origin=c]{30}{$\twoheadleftarrow$}}}
\newcommand{\darrowr}{\mathrel{\rotatebox[origin=c]{120}{$\twoheadleftarrow$}}}
\newcommand{\vuarrow}{\mathrel{\rotatebox[origin=c]{-90}{$\leftarrowtail$}}}
\newcommand{\vdarrow}{\mathrel{\rotatebox[origin=c]{90}{$\twoheadleftarrow$}}}

% Types, terms, and precision 

\newcommand{\dyn}{?}
\newcommand{\nat}{\text{Nat}}
\newcommand{\bool}{\text{Bool}}
\newcommand{\ra}{\rightharpoonup}
\newcommand{\Ret}[1]{\mathsf{Ret\,}{#1}}
\newcommand{\hole}[1]{\bullet \colon {#1}}
\newcommand{\dyntodyn}{\dyn \ra\, \dyn}


\newcommand{\up}[2]{\langle{#2}\uarrowl{#1}\rangle}
\newcommand{\dn}[2]{\langle{#1}\darrowl{#2}\rangle}

\newcommand{\upc}[2]{\text{up}\,{#1}\,{#2}}
\newcommand{\dnc}[2]{\text{dn}\,{#1}\,{#2}}


\newcommand{\ret}{\mathsf{ret}}
\newcommand{\err}{\mho}
\newcommand{\zro}{\textsf{zro}}
\newcommand{\suc}{\textsf{suc}}
\newcommand{\lda}[2]{\lambda {#1} . {#2}}
\newcommand{\injarr}[1]{\textsf{Inj}_\ra ({#1})}
\newcommand{\injnat}[1]{\textsf{Inj}_\text{nat} ({#1})}
\newcommand{\casenat}[4]{\text{Case}_\text{nat} ({#1}) \{ \text{no} \to {#2} \alt \text{nat}({#3}) \to {#4} \}}
\newcommand{\casearr}[4]{\text{Case}_\ra ({#1}) \{ \text{no} \to {#2} \alt \text{fun}({#3}) \to {#4} \}}
\newcommand{\casedyn}[5]{\text{Case}_\Dyn ({#1}) \{ \text{nat}({#2}) \to {#3} \alt \text{fun}({#4}) \to {#5} \}}
\newcommand{\bind}[3]{\text{var } {#1} = {#2} \text{ in } {#3}}
\newcommand{\matchnat}[4]{\text{match } {#1} \text{ with } \{ \textsf {zero} \Rightarrow {#2} \alt \textsf{ suc } {#3} \Rightarrow {#4} \}}

\newcommand{\refl}{\text{refl}}

\newcommand{\Lift}{\text{Lift}}

%\newcommand{\rel}{\circ\hspace{-4px}-\hspace{-4px}\bullet}
\newcommand{\rel}{\mathrel{\circ\mkern-6mu-\mkern-6mu\bullet}}
\newcommand{\ltdyn}{\sqsubseteq}
\newcommand{\gtdyn}{\sqsupseteq}
\newcommand{\equidyn}{\mathrel{\gtdyn\ltdyn}}
\newcommand{\gamlt}{\Gamma^\ltdyn}
\newcommand{\deltalt}{\Delta^\ltdyn}
\newcommand{\relcomp}{\odot}

\newcommand{\hasty}[3]{{#1} \vdash {#2} \colon {#3}}
\newcommand{\vhasty}[3]{{#1} \vdash^v {#2} \colon {#3}}
\newcommand{\phasty}[3]{{#1} \vdash^p {#2} \colon {#3}}
\newcommand{\etmprec}[4]{{#1} \vdash {#2} \ltdyn_e {#3} \colon {#4}}
\newcommand{\itmprec}[4]{{#1} \vdash {#2} \ltdyn_i {#3} \colon {#4}}
\newcommand{\etmequidyn}[4]{{#1} \vdash {#2} \equidyn_e {#3} \colon {#4}}
\newcommand{\itmequidyn}[4]{{#1} \vdash {#2} \equidyn_i {#3} \colon {#4}}
\newcommand{\synbisim}{\approx_{\text{syn}}}

\newcommand{\Dwn}{\Downarrow}
\newcommand{\qte}[1]{\text{quote}({#1})}

\newcommand{\elab}[1]{\text{Elab}({#1})}

% Perturbations
\newcommand{\pertp}{\text{Pert}^\text{P}}
\newcommand{\perte}{\text{Pert}^\text{E}}
\newcommand{\pertdyn}[2]{\text{pert-dyn}({#1}, {#2})}
\newcommand{\delaypert}[1]{\text{delay-pert}({#1})}

\newcommand{\pertc}{\text{Pert}_{\text{C}}}
\newcommand{\pertv}{\text{Pert}_{\text{V}}}


% SGDT and Intensional Stuff

\newcommand{\later}{\vartriangleright}
\newcommand{\type}{\texttt{Type}}
\newcommand{\lob}{\text{L\"{o}b}}
\newcommand{\tick}{\mathsf{tick}}
\newcommand{\nxt}{\mathsf{next}}
\newcommand{\fix}{\mathsf{fix}}
\newcommand{\kpa}{\kappa}

% Model-related stuff
\newcommand{\calC}{\mathcal{C}}
\newcommand{\Set}{\mathsf{Set}}
\newcommand{\Yo}{\mathsf{Yo}}
\newcommand{\Hom}{\mathsf{Hom}}
\newcommand{\calS}{\mathcal{S}}
\newcommand{\gfix}{\texttt{gfix}}
\newcommand{\calU}{\mathcal{U}}
\newcommand{\laterhat}{\widehat{\later}}
\newcommand{\El}{\mathsf{El}}
\newcommand{\Clock}{\mathsf{Clock}}

\newcommand{\Machine}[1]{\mathsf{Machine}\, {#1}}


% Predomains and EP pairs
\newcommand{\Nat}{\mathsf{Nat}}
\newcommand{\Dyn}{\mathsf{Dyn}}
\newcommand{\ty}[1]{\langle {#1} \rangle}
\newcommand{\li}{L_\mho}
\newcommand{\liclk}[1]{L_\mho [{#1}]}

\newcommand{\ext}[2]{\text{ext}\,{#1}\,{#2}}
\newcommand{\map}[2]{\text{map}\,{#1}\,{#2}}

\newcommand{\ltls}{\lesssim}
\newcommand{\bisim}{\approx}
\newcommand{\semlt}{\le}
\newcommand{\semltbad}{\lesssim}

%\newcommand{\injarr}{\textsf{Inj}_\to}
%\newcommand{\injnat}{\textsf{Inj}_\mathbb{N}}

\newcommand{\id}{\mathsf{id}}
\newcommand{\ep}{\leadsto}

\newcommand{\emb}[2]{\mathsf{emb}_{#1}({#2})}
\newcommand{\proj}[2]{\mathsf{proj}_{#1}({#2})}

\newcommand{\monto}{\to_m}


% Notation for wait functions
\newcommand{\wre}{w_r^e}
\newcommand{\wle}{w_l^e}
\newcommand{\wrp}{w_r^p}
\newcommand{\wlp}{w_l^p}

\newcommand{\sem}[1]{\llbracket {#1} \rrbracket}
\newcommand{\semgl}[1]{{\sem{#1}}^\text{gl}}


% Denotational model
\newcommand{\ptb}{\text{ptb}}
\newcommand{\push}{\text{push}}
\newcommand{\pull}{\text{pull}}



\begin{document}

\title{Denotational Semantics of Gradual Typing using Synthetic Guarded Domain Theory}
\author{Eric Giovannini}
\affiliation{
  \department{Electrical Engineering and Computer Science}
  \institution{University of Michigan}
  \country{USA}
}
\email{ericgio@umich.edu}
\orcid{0009-0003-6871-1714}

\author{Tingting Ding}
\affiliation{
  \department{Electrical Engineering and Computer Science}
  \institution{University of Michigan}
  \country{USA}
}
\email{tingtind@umich.edu}
\orcid{0009-0000-5676-1886}

\author{Max S. New}
\affiliation{
  \department{Electrical Engineering and Computer Science}
  \institution{University of Michigan}
  \country{USA}
}
\email{maxsnew@umich.edu}
\orcid{0000-0001-8141-195X}

\begin{abstract}
  Gradually typed programming languages, which allow for soundly
  mixing static and dynamically typed programming styles, present a
  strong challenge for metatheorists. Even the simplest sound
  gradually typed languages feature at least recursion and errors,
  with realistic languages featuring furthermore runtime allocation of
  memory locations and dynamic type tags. Further, the desired
  metatheoretic properties of gradually typed languages have become
  increasingly sophisticated: validity of type-based equational
  reasoning as well as the relational property known as
  graduality. Many recent works have tackled verifying these
  properties, but the resulting mathematical developments are highly
  repetitive and tedious, with few reusable theorems persisting across
  different developments.

  In this work, we present a new denotational semantics for gradual
  typing developed using guarded domain theory. Guarded domain theory
  combines the generality of step-indexed logical relations for
  modeling advanced programming features with the modularity and
  reusability of denotational semantics. We demonstrate the
  feasibility of this approach with a model of a simple gradually
  typed lambda calculus and prove the validity of beta-eta equality
  and the graduality theorem for the denotational model. This model
  should provide the basis for a reusable mathematical theory of
  gradually typed program semantics. Finally, we have mechanized most
  of the core theorems of our development in Guarded Cubical Agda, a
  recent extension of Agda with support for the guarded recursive
  constructions we use.
\end{abstract}

\maketitle

% Outline

% 1. Intro: What do we want out of gradually typed languages and why
%    is it hard to prove? Explanation: Gradual Typing inherently
%    involves recursive types, multiple effects, relational properties.
%    We argue that the increasing complexity of the metatheory of
%    gradual typing makes it a good candidate for 

% 2. Extensional Dream Semantics: double categorical

% 3. The Problem with Step-indexing

% 4. A Compromise

% 5. Formalization in Guarded Cubical Agda

\newif\ifdraft
\drafttrue
\renewcommand{\max}[1]{\ifdraft{\color{blue}[{\bf Max}: #1]}\fi}
\newcommand{\eric}[1]{\ifdraft{\color{orange}[{\bf Eric}: #1]}\fi}
\newcommand{\tingting}[1]{\ifdraft{\color{red}[{\bf Tingting}: #1]}\fi}

\section{Introduction}
  
% gradual typing, graduality
\subsection{Gradual Typing and Graduality}
In programming language design, there is a tension between \emph{static} typing
and \emph{dynamic} typing disciplines.
With static typing, the code is type-checked at compile time, while in dynamic typing,
the type checking is deferred to run-time. Both approaches have benefits: with static 
typing, the programmer is assured that if the program passes the type-checker, their
program is free of type errors, and moreover, soundness of the equational theory implies
that program optimizations are valid.
Meanwhile, dynamic typing allows the programmer to rapidly prototype
their application code without needing to commit to fixed type signatures for their
functions. Most languages choose between static or dynamic typing and as a result, programmers that initially write their code in a dynamically typed language
in order to benefit from faster prototyping and development time need to rewrite
some or all of their codebase in a static language if they would like to receive the benefits of static
typing once their codebase has matured.

\emph{Gradually-typed languages} \cite{siek-taha06, tobin-hochstadt06} seek to resolve this tension
by allowing for both static and dynamic typing disciplines to be used in the same codebase,
and by supporting smooth interoperability between statically-typed and dynamically-typed code.
This flexibility allows programmers to begin their projects in a dynamic style and
enjoy the benefits of dynamic typing related to rapid prototyping and easy modification
while the codebase ``solidifies''. Over time, as parts of the code become more mature
and the programmer is more certain of what the types should be, the code can be
\emph{gradually} migrated to a statically typed style without needing to
rewrite the project in a completely different language.

%Gradually-typed languages should satisfy two intuitive properties.
% The following two properties have been identified as useful for gradually typed languages.

In order for this to work as expected, gradually-typed languages should allow for
different parts of the codebase to be in different places along the spectrum from
dynamic to static, and allow for those different parts to interact with one another.
Moreover, gradually-typed languages should support the smooth migration from
dynamic typing to static typing, in that the programmer can initially leave off the
typing annotations and provide them later without altering the meaning of the program.
% Sound gradual typing
Furthermore, the parts of the program that are written in a dynamic
style should soundly interoperate with the parts that are written in a
static style.  That is, the interaction between the static and dynamic
components of the codebase should preserve, to the extent possible,
the guarantees made by the static types.  In particular, while
statically-typed code can error at runtime in a gradually-typed
language, such an error can always be traced back to a
dynamically-typed term that violated the typing contract imposed by
statically typed code. Further, static type assertions are sound in
the static portion, and should enable type-based reasoning and
optimization.

% Moreover, gradually-typed languages should allow for
% different parts of the codebase to be in different places along the spectrum from
% dynamic to static, and allow for those different parts to interact with one another.
% In a \emph{sound} gradually-typed language,
% this interaction should respect the guarantees made by the static types.

% Graduality property
One of the fundamental theorems for gradually typed languages is
\emph{graduality}, also known as the \emph{dynamic gradual guarantee}
\emph{dynamic gradual guarantee}, originally defined by Siek,
Vitousek, Cimini, and Boyland \cite{siek_et_al:LIPIcs:2015:5031,
  new-ahmed2018}.
%
Informally, graduality says that going from a dynamic to static style should only allow for the introduction of static or dynamic type errors, and not otherwise change the behavior of the program.
%
This is a way to capture programmer intuition that increasing type
precision corresponds to a generalized form of runtime assertions in
that there are no observable behavioral changes up to the point of the
first dynamic type error\footnote{once a dynamic type error is raised,
in languages where the type error can be caught, program behavior may
then further diverge, but this is typically not modeled in gradual
calculi.}.
%
Fundamentally, this property comes down to the behavior of
\emph{runtime type casts}\max{TODO: introduce casts more thoroughly as
  they are important}.

Additionally, gradually typed languages should offer some of the
benefits of static typing. While classical type soundness, that
well-typed programs are free from runtime errors, is not compatible
with runtime type errors, we can instead focus on \emph{type-based
reasoning}. For instance, while dynamically typed $\lambda$ calculi
only satisfy $\beta$ equality for their type formers, statically typed
$\lambda$ calculi additionally satisfy type-dependent $\eta$
properties that ensure that functions are determined by their behavior
under application and that pattern matching on data types
is safe and exhaustive.

More concretely, consider a gradually typed language whose only
effects are gradual type errors and divergence. Then if we fix a
result type of natural numbers, a whole program semantics is a partial
function from closed programs to either natural numbers or errors:
\[ -\Downarrow : \{M \,|\, \cdot \vdash M : \nat \} \rightharpoonup \mathbb{N} \cup {\mho} \]
where $\mho$ is notation for a runtime type error. We write $M
\Downarrow n$ and $M\Downarrow \mho$ to mean this semantics is defined
as a number or error, and $M\Uparrow$ to mean the semantics is
undefined, representing divergence.
%
A well-behaved semantics should then satisfy several properties. First, it
should be adequate: natural number constants should step to
themselves. Second it should validate type based reasoning. To
formalize type based reasoning we give a typed equational theory for
terms of the language $M \cong N$ for when two terms should be
considered equivalent. Then we want to verify that the big step
semantics respects this equational theory: if closed programs $M \cong
N$ are equivalent in the equational theory then they have the same
semantics, $M \Downarrow n \iff N \downarrow n$ and $M\Uparrow \iff N
\Uparrow$ and $M \Downarrow \mho \iff N \Downarrow \mho$.
%
Lastly, the graduality property is defined by giving an
\emph{inequational} theory called term precision, where $M \ltdyn N$
roughly means that $M$ and $N$ have the same type erasure and $M$ has
at each point in the program a more precise/static type than $N$.
%
Then, the graduality property states that if $M \ltdyn N$ are whole
programs then $M$ must either have the same behavior as $N$ or error:
Either $M\Downarrow \mho$ or $M \Downarrow n $ and $N \Downarrow n$ or
$M \Uparrow $ and $N \Uparrow$\footnote{we use a slightly more complex
definition of this relation in our technical development below that is
classically equivalent but constructively weaker}.

\subsection{Denotational Semantics in Guarded Domain Theory}

Our goal in this work is to provide an \emph{expressive},
\emph{reusable}, \emph{compositional} semantic framework for defining
such well-behaved semantics of programs.
%
Our approach to achieving this goal is to provide a compositional
\emph{denotational semantics}, mapping types to a kind of semantic
domain, terms to functions and relations such as term precision to
proofs of semantic relations between the denoted functions.
%
Since the denotational constructions are all syntax-independent, the
constructions we provide may be reused for similar languages. Since it
is compositional, components can be mixed and matched depending on
what source language features are present.
%
Providing this semantics for gradual typing is inherently complicated
in that it involves: (1) recursion and recursive types through the
presence of dynamic types, (2) effects in the form of divergence and
errors (3) relational models in capturing the graduality
property. Recursion and recursive types must be handled using some
flavor of domain theory. Effects can be modeled using monads in the
style of Moggi, or adjunctions in the style of Levy. Relational
properties and their verification lead naturally to the use of
reflexive graph categories or double categories.

The only prior denotational semantics for gradual typing was given by
New and Licata and is based on \emph{domain theory}
\cite{newlicata2019}. The fundamental idea is to equip $\omega$-CPOs
with an additional ``error ordering'' $\ltdyn$ which models the
graduality ordering, and for casts to arise from
\emph{embedding-projection pairs}. Then the graduality property
follows as long as all language constructs can be interpreted using
constructions that are monotone with respect to the error ordering.
%
This framework has the benefit of being compositional, and was
expressive enough to be extended to model dependently typed gradual
typing \cite{dependentgradualtyping}.
%
However, an approach based on classical domain theory has fundamental
limitations: domain theory is incapable of modeling certain ``highly
recursive'' features of programming languages such as dynamic type tag
generation and higher-order references, which are commonplace in
real-world gradually typed systems as well as gradual calculi
\cite{examples-of-gradual-stuff}. Our goal in this work is to develop
an approach that will eventually scale up to these advanced features,
though for a first work we focus on adapting their results for a more
basic semantics.

The main denotational alternative to classical domain theory that can
successfully model these advanced features is \emph{guarded domain
theory}, which we adopt in this work. While classical domain theory is
based on modeling types as ordered sets with certain joins, guarded
domain theory is based on an entirely different foundations, sometimes
(ultra)metric spaces but more commonly as ``step-indexed sets'', i.e.,
objects in the ``topos of trees'', i.e., presheaves on the poset of
natural numbers. We think of these as modeling a kind of ``sequence of
approximations'' to the domain being modeled.
%
Key to guarded domain theory is that there is a ``later'' operator
$\triangleright$ on types. Thinking of types as a sequence of
approximations, the later operator delays the approximation by one
step. Then the crucial axiom of guarded domain theory is that any
guarded equation $X \cong F(\triangleright X)$ has a unique
solution. This allows guarded domain theory to model essentially
\emph{any} recursive concept, with the caveat that it is guarded by a
later.
%
This caveat is the main source of difficulty in adapting a semantic
approach based on classical domain theory to guarded domain theory:
classical domain theory has limitations in what it can model, but it
provides \emph{exact} solutions to domain equations when it
applies. When adapting the New-Licata approach to guarded domain
theory, this means we must work with an \emph{intensional} semantics,
one where unfolding the dynamic type takes an observable runtime step.

\subsection{Synthetic Guarded Domain Theory}

While guarded domain theory can be presented analytically using
ultrametric spaces or the topos of trees, in practice it is
considerably simpler to work \emph{synthetically} by working in a
non-standard foundation such as guarded type theory where the later
modality is simply an operation on our basic notion of type, and we
take as an axiom that guarded domain equations have a (necessarily
unique) solution. This has the added benefit that providing a
denotational semantics of gradually typed terms is as simple as an
semantics of an effectful language using a monad on the category of
sets. Not only does this make on-paper reasoning about guarded domain
theory easier, it also enables a simpler avenue to verification in a
proof assistant. Whereas formalizing analytic guarded domain theory
would require significant theory of presheaves and making sure that
all constructions are functors on categories of presheaves,
formalizing synthetic guarded domain theory can be done by directly
adding the later modality and the guarded fixed point property
axiomatically.

\max{TODO: more here}

\subsection{Contributions}

The main contribution of this work is a compositional denotational
semantics for gradually typed languages that validates $\beta\eta$
equality and satisfies a graduality theorem. A great deal of the work
has further been verified in guarded cubical Agda, demonstrating that
the semantics is readily mechanizable.

\begin{enumerate}
\item First, we give a simple concrete term semantics where we show
  how to model the dynamic type as a solution to a guarded domain equation

\item Next, we identify where prior work on classical domain theoretic
  semantics of gradual typing breaks down when using guarded semantics
  of recursive types.

\item We develop a key new concept of \emph{syntactic perturbations},
  which allow us to recover enough extensional reasoning to model the
  graduality property compositionally

\item We combine this insight together with an abstract categorical
  model of gradual typing using reflexive graph categories and
  call-by-push-value to give a compositional construction of our
  denotational model.
\item We prove that the resulting denotational model provides a
  well-behaved semantics as defined above by proving \emph{adequacy},
  respect for an equational theory and the graduality property.
\end{enumerate}

The paper is laid out as follows:
\begin{enumerate}
\item In Section\ref{sec:syntax} we fix our input language, a fairly
  typical gradually typed cast calculus.
\item In Section\ref{sec:concrete-terms} we develop a first
  denotational semantics in synthetic guarded domain theory that
  satisfies adequacy and respects the equational theory, but does not
  validate graduality. We use this to introduce some of our main
  technical tools: modeling recursive types in guarded type theory and
  modeling effects using call-by-push-value.
\item In Section\ref{sec:relational-issues} we show where the
  New-Licata classical domain theoretic approach fails to adapt
  cleanly to the guarded setting and explore the difficulties of
  proving graduality in an intensional model.
\item In Section\ref{sec:relational-semantics} we
\item In Section\ref{sec:discussion} we discuss prior work on proving
  graduality, our partial mechanization of these results and discuss
  future directions for denotational semantics of gradual typing.
\end{enumerate}

\subsection{Limitations of Prior Work}

We give an overview of current approaches to proving graduality of
languages and why they do not meet our criteria of a reusable semantic
framework.

\subsubsection{From Static to Gradual}

Current approaches to constructing languages that satisfy the
graduality property include the methods of Abstracting Gradual Typing
\cite{garcia-clark-tanter2016} and the formal tools of the Gradualizer
\cite{cimini-siek2016}.  These allow the language developer to start
with a statically typed language and derive a gradually typed language
that satisfies the gradual guarantee. The main downside to these
approaches lies in their inflexibility: since the process in entirely
mechanical, the language designer must adhere to the predefined
framework.  Many gradually typed languages do not fit into either
framework, e.g., Typed Racket \cite{tobin-hochstadt06,
  tobin-hochstadt08} and the semantics produced is not always the
desired one.
%
Furthermore, while these frameworks do prove graduality of the
resulting languages, they do not show the correctness of the
equational theory, which is equally important to sound gradual typing.

%% For example, programmers often refactor their code, and in so doing they rely
%% implicitly on the validity of the laws in the equational theory.
%% Similarly, correctness of compiler optimizations rests on the validity of the
%% corresponding equations from the equational theory. It is therefore important
%% that the languages that claim to be gradually typed have provably correct
%% equational theories.

% The approaches are too inflexible... the fact that the resulting semantics are too lazy
% is a consequence of that inflexibility.
% The validity of the equational theory captures the programmer's intuitive thinking when they refactor their code

%The downside is that
%not all gradually typed languages can be derived from these frameworks, and moreover, in both
%approaches the semantics is derived from the static type system as opposed to the alternative
%in which the semantics determines the type checking. Without a clear semantic interpretation of type
%dynamism, it becomes difficult to extend these techniques to new language features such as polymorphism.


% [Eric] I moved the next two paragraphs from the technical background section
% to here in the intro.
\subsubsection{Double Categorical Semantics}

New and Licata \cite{new-licata18} developed an axiomatic account of
the graduality relation on a call-by-name cast calculus terms and
showed that the graduality proof could be modeled using semantics in
certain kinds of \emph{double categories}, categories internal to the
category of categories. A double category extends a category with a
second notion of morphism, often a notion of ``relation'' to be paired
with the notion of functional morphism, as well as a notion of
functional morphisms preserving relations. In gradual typing the
notion of relation models type precision and the squares model the
term precision relation. This approach was influenced by the semantics
of parametricity using reflexive graph categories
\cite{ma-reynolds,dunphythesis,reynoldsprogramme}: reflexive graph
categories are essentially double categories without a notion of
relational composition. In addition to capturing the notions of type
and term precision, the double categorical approach allows for a
\emph{universal property} for casts: upcasts are the \emph{universal}
way to turn a relation arrow into a function in a forward direction
and downcasts are the dual universal arrow.  Later, New, Licata and
Ahmed \cite{new-licata-ahmed2019} extended this axiomatic treatment from
call-by-name to call-by-value as well by giving an axiomatic theory of
type and term precision in call-by-push-value. This left implicit any
connection to a ``double call-by-push-value'', which we make explicit
in Section~\ref{sec:cbpv}.

With this notion of abstract categorical model in hand, denotational
semantics is then the work of constructing concrete models that
exhibit the categorical construction. New and Licata
\cite{new-licata18} present such a model using categories of
$\omega$-CPOs, and this model was extended by Lennon-Bertrand,
Maillard, Tabareau and Tanter to prove graduality of a gradual
dependently typed calculus $\textrm{CastCIC}^{\mathcal G}$. This
domain-theoretic approach meets our criteria of being a semantic
framework for proving graduality, but suffers from the limitations of
classical domain theory: the inability to model viciously
self-referential structures such as higher-order extensible state and
similar features such as runtime-extensible dynamic types. Since these
features are quite common in dynamically typed languages, we seek a
new denotational framework that can model these type system features.

The standard alternative to domain theory that scales to essentially
arbitrary self-referential definitions is \emph{step-indexing} or its
synthetic form of \emph{guarded recursion}. A series of works
\cite{new-ahmed2018, new-licata-ahmed2019, new-jamner-ahmed19}
developed step-indexed logical relations models of gradually typed
languages based on operational semantics. Unlike classical domain
theory, such step-indexed techniques are capable of modeling
higher-order store and runtime-extensible dynamic types
\cite{appelmcallester01,ahmed06,neis09,new-jamner-ahmed19}. However,
their proof developments are highly repetitive and technical, with
each development formulating a logical relation from first-principles
and proving many of the same tedious lemmas without reusable
mathematical abstractions. Our goal in the current work is to extract
these reusable mathematical principles from these explicit
step-indexed to make formalization of realistic gradual languages
tractible.


% Alternative phrasing:
\begin{comment}
\subsubsection{Embedding-Projection Pairs}

The series of works by New, Licata, and Ahmed \cite{new-licata18,new-ahmed2018,new-licata-ahmed2019}
develop an axiomatic account of gradual typing involving \emph{embedding-projection pairs}.
This allows for a particularly elegant formulation of the gradual guarantee.
Moreover, their axiomatic account of program equivalence allows for type-based reasoning about program equivalences.

In \cite{new-licata18}, New and Licata construct a denotational model of the axioms of
gradual typing using tools from classical domain theory. The benefit to this approach is that it is a reusable mathematical theory:
general semantic theorems about gradual domains can be developed independent of any particular syntax and then reused in many different denotational models.
Unfortunately, however, it is unclear how to extend these domain-theoretic techniques to incorporate more advanced language features such as higher-order store, a standard feature of realistic gradually typed languages such as Typed
Racket.
Thus, if we want a reusable mathematical theory of gradual typing that can scale to realistic programming languages, we need to look to different techniques.

In \cite{new-ahmed2018,new-licata-ahmed2019}, a more traditional operational semantics for gradual typing is derived from the axioms.
% The axioms are then proven to be sound with respect to this operational semantics by constructing a logical relation.
A logical relations model is constructed and used to prove that the axioms are sound with respect to the operational semantics.
The downside of this approach is that each new language requires a different logical relation
to prove graduality, even though the logical relations for different languages end up sharing many commonalities.
Furthermore, the logical relations tend to be quite complicated due to a technical requirement known as \emph{step-indexing},
where the stepping behavior of terms must be accounted for in the logical relation.
As a result, developments using this approach tend to require vast effort, with the
corresponding technical reports having 50+ pages of proofs.

% In this approach, a logical relation is constructed and used to show that the equational theory
% is sound with respect to the operational semantics.

% Additionally, while the axioms of gradual type theory are compositional at a ``global'' level,
% they do not compose in the step-indexed setting. One of the main goals of the present work
% is to formulate a composable theory of gradual typing in a setting where the stepping behavior
% is tracked.
\end{comment}

An alternative approach, which we investigate in this paper, is provided by
\emph{synthetic guarded domain
theory}\cite{birkedal-mogelberg-schwinghammer-stovring2011}, abbreviated as
SGDT. The techniques of SGDT allow us to internalize the step-indexed reasoning
normally required in logical relations proofs of graduality, ultimately allowing
us to specify the logical relation in a manner that looks nearly identical to a
typical, non-step-indexed logical relation. In fact, guarded domain theory goes
further, allowing us to define step-indexed \emph{denotational semantics} not
just step-indexed relations, just as easily as constructing an ordinary
set-theoretic semantics.

In this paper, we develop an adequate denotational semantics that satisfies
graduality and soundness of the equational theory of cast calculi using the
techniques of SGDT.  

\eric{This is out of date: the mechanization is now an artifact of the current paper.}
Our longer-term goal is to mechanize these proofs in a reusable way in the
Guarded Cubical Agda \cite{veltri-vezzosi2020} proof assistant, thereby
providing a framework to more easily and conveniently prove that existing
languages satisfy graduality and have sound equational theories.

Moreover, the aim is for designers of new languages to utilize the framework to
facilitate the design of new provably-correct gradually-typed languages with
more complex features.

\subsection{Contributions}

The main contribution of this work is a categorical and denotational semantics
for gradually typed langauges that models not just the term language but the
graduality property as well.
\begin{enumerate}
\item First, we give a simple abstract categorical model of Gradual Type Theory
using CBPV double categories.
\item Next, we modify this semantics to develop reflexive graph- and double
  categorical models that abstract over the details of step-indexed models, and
  provide a method for constructing such models.
\item We instantiate the abstract construction to provide a concrete semantics
  in informal guarded type theory.
\item We prove that the resulting denotational model is \emph{adequate} for the
  graduality property: a closed term precision $M \ltdyn N : \nat$ has the
  expected semantics, that $M$ errors or $M$ and $N$ have the same extensional
  behavior.
\end{enumerate}

% TODO is it okay to refer to our cast calculus as *the* gradually-typed lambda
% calculus?
\eric{Where should we put the description of the big-step semantics and adequacy
of the semantics? I can put it in the outline of the remainder of the paper, or
I can postpone it until the subsequent section on the syntax of gradual typing.
Adequacy seems to be better introduced there since at that point we have
introduced the notion of term precision.}

The remainder of the paper is organized as follows. In Section
\ref{sec:technical-background}, we provide technical background on synthetic
guarded domain theory and some important details about the specific variant of
SGDT we employ as our ambient language.
%
In Section \ref{sec:GTLC}, we describe the syntactic theory of a standard cast
calculus for gradual typing, which we also refer to as the \emph{gradually-typed
lambda calculus}. We do this to establish a common syntax that will be
interpreted into the models we subsequently construct.
%
In Section \ref{sec:cbpv}, we develop \emph{abstract} categorical
semantics of gradually typed languages with the goal of organizing our
construction of denotational models of gradual typing.
%
In Sections \ref{sec:concrete-term-model} and
\ref{sec:towards-relational-model}, we set out to instantiate the abstract
categorical semantics by constructing a concrete model in a suitable double
category. First, in Section \ref{sec:concrete-term-model}, we give a concrete
model for the \emph{terms} of the gradually-typed lambda calculus. This involves
using the tools of SGDT to solve a domain equation to give the semantics of the
dynamic type. We explain how to \emph{globalize} this model to obtain a model in
the usual (non-guarded) dependent type theory. In the end, we obtain a
\emph{big-step semantics} for closed terms of type $\nat$:
%
\[ \sem{\cdot} \colon (\cdot \vdash M : \nat) \to (\mathbb{N} \to (\sem{\nat} + 1)) \] 
%
where a closed term $M$ of type $\nat$ denotes a \emph{function} from
$\mathbb{N}$ to $\sem{\nat} + 1$ which given input $i$ returns $\inl\, m$ if $M$
terminates with value $m$ in $i$ or fewer steps, and returns $\inr\, *$ if $M$
fails to terminate in $i$ or fewer steps. This is already a significant result,
as it serves to verify the $\beta$ and $\eta$ laws of the equational theory.
%
Then in Section \ref{sec:towards-relational-model}, we attempt to extend the
methodology of the previous section to give a denotational semantics that models
the relational theory of the gradually-typed lambda calculus in addition to the
terms. Interesting issues arise here arising from the \emph{intensional} nature
of SGDT, that is, the fact that \emph{steps are observable}. We then develop
novel solutions to address these issues. We do not complete the construction of
the concrete relational model in this section. Instead, we choose to reformulate
the definition of abstract categorical model to take into account the solutions
to the issues discussed here.

% Interesting issues arise here, and the novel solutions we employ to
% address these issues inform the definition of the abstract models introduced in
% the subsequent section.

With the lessons from the concrete setting at our disposal, in Section
\ref{sec:abstract-models}, we describe revised categorical models of gradual
typing. We actually introduce two separate notions of model: \emph{intensional},
where the stepping behavior of terms is observable, and \emph{extensional},
where we consider terms that differ only in their stepping behavior to be equal.
The extensional model is what we will ultimately use to interpret the syntax and
relational axioms of gradually-typed languages. The principal benefit of the
notion of intensional model is that it admits more compositional reasoning. It
therefore serves as a convenient intermediate setting that enjoys many of the
benefits of the usual double-categorical models of gradual typing.
% We give the definition of intensional model as a sequence of steps each
% building on the previous. 
We then outline how to build an extensional model in a sequence of steps beginning
with a significantly simpler ``input'' model. The construction proceeds in
steps, first building up an intensional model, and then from there extracting an
extensional model.
%We are able to leverage the compositional reasoning afforded by the intensional
%model to more easily meet the requirements of an extensional model.
This gives us a more tractable way to construct models.
Moreover, the abstract definitions and constructions in the definitions make no
mention of SGDT, so that the places where SGDT techniques are employed are
conveniently limited to the data passed as input to the model construction.
%
%In addition to organizing and facilitating the construction of a model, the abstract constructions are described independently of SGDT
%
%we also isolate the parts of the definition the notion of model without any explicit mention of SGDT.
%
Then in Section \ref{sec:concrete-model} we return to the task of defining a
concrete model. Equipped now with the abstract constructions of the previous
section, we provide only the data needed as an input to the construction. Many
of the relevant definitions have already been introduced in Section
\ref{sec:towards-relational-model} and can be provided to the construction
as-is; the remainder of the necessary data is defined in this section.
%
We end the section by sketching a proof of the \emph{adequacy} of our concrete
model with respect to graduality.
%
Finally, in Section \ref{sec:discussion} we conclude with discussion: comparison
to related work, the mechanization of our concrete model in Guarded Cubical
Agda, and directions for future research.





%% \subsection{Outline of Remainder of Paper}

%% \max{this is entirely out of date, update later}

%% % In Section \ref{sec:overview}, we give an overview of the gradually-typed lambda
%% % calculus and the graduality theorem.
%% %
%% In Section \ref{sec:technical-background}, we provide technical background on synthetic guarded domain theory.
%% % 
%% In Section \ref{sec:gtlc-terms}, we formally introduce the gradually-typed cast calculus
%% for which we will prove graduality. We give a semantics to the terms using
%% synthetic guarded domain theory.
%% % Important here are the notions of syntactic
%% % type precision and term precision. For reasons we describe below, we
%% % introduce two related calculi, one in which the stepping behavior of terms is
%% % implicit (an \emph{extensional} calculus, $\extlc$), and another where this behavior
%% % is made explicit (an \emph{intensional} calculus, $\intlc$).
%% %

%% In Section \ref{sec:gtlc-precision}, we define the term precision ordering for
%% the gradually-typed lambda calculus and describe our approach to assigning a
%% denotational semantics to this ordering.
%% This approach builds on the semantics constructed in the previous section,
%% but extending it to the term ordering presents new challenges.


%% % In Section \ref{sec:domain-theory}, we define several fundamental constructions
%% % internal to SGDT that will be needed when we give a denotational semantics to
%% % the gradual lambda calculus.
%% %This includes the Lift monad, predomains and error domains.
%% %
%% % In Section \ref{sec:semantics}, we define the denotational semantics for the
%% % intensional gradually-typed lambda calculus using the domain theoretic
%% % constructions defined in the previous section.
%% %
%% In Section \ref{sec:graduality}, we outline in more detail the proof of graduality for the
%% gradual lambda calculus.
%% %
%% In Section \ref{sec:discussion}, we discuss the benefits and drawbacks to our approach in comparison
%% to the traditional step-indexing approach, as well as possibilities for future work.

%% \subsection{Overview of Results}\label{sec:overview}

%% % This section used to be part of the intro.
%% % \subsection{Proving Graduality in SGDT}
%% % TODO This section should probably be moved to after the relevant background has been introduced?

%% % TODO introduce the idea of cast calculus and explicit casts?

%% In this paper, we will utilize SGDT techniques to prove graduality for a particularly
%% simple gradually-typed cast calculus, the gradually-typed lambda calculus.
%% Before stating the graduality theorem, we introduce some definitions related to gradual typing.

%% % Cast calculi
%% % TODO move this to earlier?
%% Gradually typed languages are generally elaborated to a \emph{cast calculus}, in which run-time type checking
%% that is made explicit. Elaboration inserts \emph{type casts} at the boundaries between static and dynamic code.
%% In particular, cast insertion happens at the elimination forms of the language (e.g., pattern matching, field reference, etc.).
%% The original gradually typed language that is elaborated to a cast calculus is called the \emph{surface language}.

%% % Up and down casts
%% In a cast calculus, there is a relation $\ltdyn$ on types such that $A \ltdyn B$ means that
%% $A$ is a \emph{more precise} type than $B$.
%% There a distinguished type $\dyn$, the \emph{dynamic type}, with the property that $A \ltdyn\, \dyn$ for all $A$.
%% The parts of the code that are dynamically typed will be given type $\dyn$ in the cast calculus.
%% %
%% If $A \ltdyn B$, a term $M$ of type $A$ may be \emph{up}casted to $B$, written $\up A B M$,
%% and a term $N$ of type $B$ may be \emph{down}casted to $A$, written $\dn A B N$.
%% Upcasts always succeed, while downcasts may fail at runtime, resulting in a type error.
%% Cast calculi have a term $\mho$ representing a run-time type error for any type $A$.

%% % Syntactic term precision
%% We also have a notion of \emph{syntactic term precision}.
%% If $A \ltdyn B$, and $M$ and $N$ are terms of type $A$ and $B$ respectively, we write
%% $M \ltdyn N : A \ltdyn B$ to mean that
%% $M$ is ``syntactically more precise'' than $N$, or equivalently, $N$ is 
%% ``more dynamic'' than $M$. Intuitively, this models the situation where $M$ and $N$
%% behave the same except that $M$ may error more.
%% Term precision is defined by a set of axioms that capture this intuitive notion.
%% The specific rules for term precision depend on the specifics of the language, but
%% many of the rules can be derived from the typing rules in a straightforward way.
%% % Cast calculi also have a term $\mho$ representing a run-time type error for any type $A$,
%% % and since
%% Since $\mho$ always errors, there is a term precision rule stating that $\mho \ltdyn M$ for all $M$.

%% % GTLC
%% The gradually-typed lambda calculus is the usual simply-typed lambda calculus with a dynamic
%% type $\dyn$ such that $A \ltdyn\, \dyn$ for all types $A$, as well as upcasts and downcasts
%% between any types $A$ and $B$ such that $A \ltdyn B$. The complete definition will be provided in
%% Section \ref{sec:gtlc-precision}.

%% With these definitions, we can state the graduality theorem for the gradually-typed lambda calculus.

%% % \begin{theorem}[Graduality]
%% %   If $M \ltdyn N : \nat$, then either:
%% %   \begin{enumerate}
%% %     \item $M = \mho$
%% %     \item $M = N = \zro$
%% %     \item $M = N = \suc n$ for some $n$
%% %     \item $M$ and $N$ diverge
%% %   \end{enumerate}
%% % \end{theorem}

%% \begin{theorem}[Graduality]
%%   Let $\cdot \vdash M \ltdyn N : \nat$. 
%%   If $M \Downarrow v_?$ and $N \Downarrow v'_?$, then either $v_? = \mho$, or $v_? = v'_?$.
%% \end{theorem}

%% Note:
%% \begin{itemize}

%%   \item $\cdot \Downarrow$ is a relation between terms that is defined such that $M \Downarrow$ means
%%   that $M$ terminates, either with a run-time error or a value $n$ of type $\nat$.

%%   \item $v_?$ is shorthand for the syntactic representation of a term that is either equal to
%%   $\mho$, or equal to the syntactic representation of a value $n$ of type $\nat$.
%% \end{itemize}

%% % We also should be able to show that $\mho$, $\zro$, and $\suc\, N$ are not equal.

%% To prove graduality and validate the equational theory, we construct a model of the types
%% and terms and show that all of the axioms for term precision and for equality of terms
%% hold in this model. Modeling the dynamic type presents a challenge in the presence of a
%% language with functions: we want the dynamic type to represent a sum of all possible types
%% in the language, so we write down an recursive equation that the semantic object modeling
%% dynamic type should satisfy. When the language includes function types, this equation involves a
%% negative occurrence of the variable for which we are solving, and so the equation 
%% does not have inductive or coinductive solutions.
%% %
%% To model the dynamic type, we therefore use guarded recursion to define a suitable
%% semantic object that satisfies the unfolding isomorphism expected of the dynamic type.
%% The key is that we do not actually get an exact solution to the equation in the style
%% of traditional domain theory; rather, we get a ``guarded'' solution that holds ``up to a time step''.
%% %
%% That is, we introduce a notion of ``time'' and in the equation for the dynamic type,
%% we guard the negative occurrences of the variable by a special operator that
%% specifies that its argument is available ``later''.
%% %This can be seen as a logic that internalizes the notion of step-indexing.
%% See Section \ref{sec:technical-background} for more details on guarded type theory.

%% At a high level, the key parts of our proof are as follows:

%% % TODO revise this
%% \begin{itemize}
%%   \item Our first step toward proving graduality is to formulate a \emph{step-sensitive},
%%   or \emph{intensional}, gradual lambda calculus, which we call $\intlc$, in which the
%%   computation steps taken by a term are made explicit.
%%   The ``normal'' gradual lambda calculus for which we want to prove graduality will be called the
%%   \emph{surface}, \emph{step-insensitive}, or \emph{extensional}, gradual lambda calculus,
%%   denoted $\extlc$.

%%   \item We define a translation from the surface syntax to the intensional syntax, and
%%   prove a theorem relating the term precision in the surface to term precision in the
%%   intensional syntax.
  
%%   \item We define a semantics for the intensional syntax in guarded type theory, for both the
%%   terms and for the term precision ordering $\ltdyn$.

%% \end{itemize}


%\section{Technical Background}\label{sec:technical-background}

\subsection{Gradual Typing}

% Cast calculi
In a gradually-typed language, the mixing of static and dynamic code is seamless, in that
the dynamically typed parts are checked at runtime. This type checking occurs at the elimination
forms of the language (e.g., pattern matching, field reference, etc.).
Gradual languages are generally elaborated to a \emph{cast calculus}, in which the dynamic
type checking is made explicit through the insertion of \emph{type casts}.

% Up and down casts
In a cast calculus, there is a relation $\ltdyn$ on types such that $A \ltdyn B$ means that
$A$ is a \emph{more precise} type than $B$.
There a dynamic type $\dyn$ with the property that $A \ltdyn\, \dyn$ for all $A$.
%
If $A \ltdyn B$, a term $M$ of type $A$ may be \emph{up}casted to $B$, written $\up A B M$,
and a term $N$ of type $B$ may be \emph{down}casted to $A$, written $\dn A B N$.
Upcasts always succeed, while downcasts may fail at runtime.
%
% Syntactic term precision
We also have a notion of \emph{syntactic term precision}.
If $A \ltdyn B$, and $M$ and $N$ are terms of type $A$ and $B$ respectively, we write
$M \ltdyn N : A \ltdyn B$ to mean that $M$ is more precise than $N$, i.e., $M$ and $N$
behave the same except that $M$ may error more.

% Modeling the dynamic type as a recursive sum type?
% Observational equivalence and approximation?

% synthetic guarded domain theory, denotational semantics therein

\subsection{Difficulties in Prior Semantics}
  % Difficulties in prior semantics

  In this work, we compare our approach to proving graduality to the approach
  introduced by New and Ahmed \cite{new-ahmed2018} which constructs a step-indexed
  logical relations model and shows that this model is sound with respect to their
  notion of contextual error approximation.

  Because the dynamic type is modeled as a non-well-founded
  recursive type, their logical relation needs to be paramterized by natural numbers
  to restore well-foundedness. This technique is known as a \emph{step-indexed logical relation}.
  Reasoning about step-indexed logical relations
  can be tedious and error-prone, and there are some very subtle aspects that must
  be taken into account in the proofs. Figure \ref{TODO} shows an example of a step-indexed logical
  relation for the gradually-typed lambda calculus.

  In particular, the prior approach of New and Ahmed requires two separate logical
  relations for terms, one in which the steps of the left-hand term are counted,
  and another in which the steps of the right-hand term are counted.
  Then two terms $M$ and $N$ are related in the ``combined'' logical relation if they are
  related in both of the one-sided logical relations. Having two separate logical relations
  complicates the statement of the lemmas used to prove graduality, becasue any statement that
  involves a term stepping needs to take into account whether we are counting steps on the left
  or the right. Some of the differences can be abstracted over, but difficulties arise for properties %/results
  as fundamental and seemingly straightforward as transitivty.

  Specifically, for transitivity, we would like to say that if $M$ is related to $N$ at
  index $i$ and $N$ is related to $P$ at index $i$, then $M$ is related to $P$ at $i$.
  But this does not actually hold: we requrie that one of the two pairs of terms
  be related ``at infinity'', i.e., that they are related at $i$ for all $i \in \mathbb{N}$.
  Which pair is required to satisfy this depends on which logical relation we are considering,
  (i.e., is it counting steps on the left or on the right),
  and so any argument that uses transitivity needs to consider two cases, one
  where $M$ and $N$ must be shown to be related for all $i$, and another where $N$ and $P$ must
  be related for all $i$. % This may not even be possible to show in some scenarios!

  % These complications introduced by step-indexing lead one to wonder whether there is a
  % way of proving graduality without relying on tedious arguments involving natural numbers.
  % An alternative approach, which we investigate in this paper, is provided by
  % \emph{synthetic guarded domain theory}, as discussed below.
  % Synthetic guarded domain theory allows the resulting logical relation to look almost
  % identical to a typical, non-step-indexed logical relation.

\subsection{Synthetic Guarded Domain Theory}
One way to avoid the tedious reasoning associated with step-indexing is to work
axiomatically inside of a logical system that can reason about non-well-founded recursive
constructions while abstracting away the specific details of step-indexing required
if we were working analytically.
The system that proves useful for this purpose is called \emph{synthetic guarded
domain theory}, or SGDT for short. We provide a brief overview here, but more
details can be found in \cite{birkedal-mogelberg-schwinghammer-stovring2011}.

SGDT offers a synthetic approach to domain theory that allows for guarded recursion
to be expressed syntactically via a type constructor $\later : \type \to \type$ 
(pronounced ``later''). The use of a modality to express guarded recursion
was introduced by Nakano \cite{Nakano2000}.
%
Given a type $A$, the type $\later A$ represents an element of type $A$
that is available one time step later. There is an operator $\nxt : A \to\, \later A$
that ``delays'' an element available now to make it available later.
We will use a tilde to denote a term of type $\later A$, e.g., $\tilde{M}$.

% TODO later is an applicative functor, but not a monad

There is a \emph{guarded fixpoint} operator

\[
  \fix : \forall T, (\later T \to T) \to T.
\]

That is, to construct a term of type $T$, it suffices to assume that we have access to
such a term ``later'' and use that to help us build a term ``now''.
This operator satisfies the axiom that $\fix f = f (\nxt (\fix f))$.
In particular, this axiom applies to propositions $P : \texttt{Prop}$; proving
a statement in this manner is known as $\lob$-induction.

The operators $\later$, $\next$, and $\fix$ described above can be indexed by objects
called \emph{clocks}. A clock serves as a reference relative to which steps are counted.
For instance, given a clock $k$ and type $T$, the type $\later^k T$ represents a value of type
$T$ one unit of time in the future according to clock $k$.
If we only ever had one clock, then we would not need to bother defining this notion.
However, the notion of \emph{clock quantification} is crucial for encoding coinductive types using guarded
recursion, an idea first introduced by Atkey and McBride \cite{atkey-mcbride2013}.


% Clocked Cubical Type Theory
\subsubsection{Ticked Cubical Type Theory}
% TODO motivation for Clocked Cubical Type Theory, e.g., delayed substitutions?

In Ticked Cubical Type Theory \cite{TODO}, there is an additional sort
called \emph{ticks}. Given a clock $k$, a tick $t : \tick k$ serves
as evidence that one unit of time has passed according to the clock $k$.
The type $\later A$ is represented as a function from ticks of a clock $k$ to $A$.
The type $A$ is allowed to depend on $t$, in which case we write $\later^k_t A$
to emphasize the dependence.

% TODO next as a function that ignores its input tick argument?

The rules for tick abstraction and application are similar to those of dependent
$\Pi$ types. 
In particular, if we have a term $M$ of type $\later^k A$, and we
have available in the context a tick $t' : \tick k$, then we can apply the tick to $M$ to get
a term $M[t'] : A[t'/t]$. We will also write tick application as $M_t$.
Conversely, if in a context $\Gamma, t : \tick k$ we have that $M$ has type $A$,
then in context $\Gamma$ we have $\lambda t.M$ has type $\later A$. % TODO dependent version?

The statements in this paper have been formalized in a variant of Agda called
Guarded Cubical Agda \cite{TODO}, an implementation of Clocked Cubical Type Theory.


% TODO axioms (clock irrelevance, tick irrelevance)?


\subsubsection{The Topos of Trees Model}

The topos of trees model provides a useful intuition for reasoning in SGDT 
\cite{birkedal-mogelberg-schwinghammer-stovring2011}.
This section presupposes knowledge of category theory and can be safely skipped without
disrupting the continuity.

The topos of trees $\calS$ is the presheaf category $\Set^{\omega^o}$. 
%
We assume a universe $\calU$ of types, with encodings for operations such
as sum types (written as $\widehat{+}$). There is also an operator 
$\laterhat \colon \later \calU \to \calU$ such that 
$\El(\laterhat(\nxt A)) =\,\, \later \El(A)$, where $\El$ is the type corresponding
to the code $A$.

An object $X$ is a family $\{X_i\}$ of sets indexed by natural numbers,
along with restriction maps $r^X_i \colon X_{i+1} \to X_i$ (see Figure \ref{fig:topos-of-trees-object}).
These should be thought of as ``sets changing over time", where $X_i$ is the view of the set if we have $i+1$
time steps to reason about it.
We can also think of an ongoing computation, with $X_i$ representing the potential results of the computation
after it has run for $i+1$ steps.

\begin{figure}
  % https://q.uiver.app/?q=WzAsNSxbMiwwLCJYXzIiXSxbMCwwLCJYXzAiXSxbMSwwLCJYXzEiXSxbMywwLCJYXzMiXSxbNCwwLCJcXGRvdHMiXSxbMiwxLCJyXlhfMCIsMl0sWzAsMiwicl5YXzEiLDJdLFszLDAsInJeWF8yIiwyXSxbNCwzLCJyXlhfMyIsMl1d
\[\begin{tikzcd}[ampersand replacement=\&]
	{X_0} \& {X_1} \& {X_2} \& {X_3} \& \dots
	\arrow["{r^X_0}"', from=1-2, to=1-1]
	\arrow["{r^X_1}"', from=1-3, to=1-2]
	\arrow["{r^X_2}"', from=1-4, to=1-3]
	\arrow["{r^X_3}"', from=1-5, to=1-4]
\end{tikzcd}\]
  \caption{An example of an object in the topos of trees.}
  \label{fig:topos-of-trees-object}
\end{figure}

A morphism from $\{X_i\}$ to $\{Y_i\}$ is a family of functions $f_i \colon X_i \to Y_i$
that commute with the restriction maps in the obvious way, that is,
$f_i \circ r^X_i = r^Y_i \circ f_{i+1}$ (see Figure \ref{fig:topos-of-trees-morphism}).

\begin{figure}
% https://q.uiver.app/?q=WzAsMTAsWzIsMCwiWF8yIl0sWzAsMCwiWF8wIl0sWzEsMCwiWF8xIl0sWzMsMCwiWF8zIl0sWzQsMCwiXFxkb3RzIl0sWzAsMSwiWV8wIl0sWzEsMSwiWV8xIl0sWzIsMSwiWV8yIl0sWzMsMSwiWV8zIl0sWzQsMSwiXFxkb3RzIl0sWzIsMSwicl5YXzAiLDJdLFswLDIsInJeWF8xIiwyXSxbMywwLCJyXlhfMiIsMl0sWzQsMywicl5YXzMiLDJdLFs2LDUsInJeWV8wIiwyXSxbNyw2LCJyXllfMSIsMl0sWzgsNywicl5ZXzIiLDJdLFs5LDgsInJeWV8zIiwyXSxbMSw1LCJmXzAiLDJdLFsyLDYsImZfMSIsMl0sWzAsNywiZl8yIiwyXSxbMyw4LCJmXzMiLDJdXQ==
\[\begin{tikzcd}[ampersand replacement=\&]
	{X_0} \& {X_1} \& {X_2} \& {X_3} \& \dots \\
	{Y_0} \& {Y_1} \& {Y_2} \& {Y_3} \& \dots
	\arrow["{r^X_0}"', from=1-2, to=1-1]
	\arrow["{r^X_1}"', from=1-3, to=1-2]
	\arrow["{r^X_2}"', from=1-4, to=1-3]
	\arrow["{r^X_3}"', from=1-5, to=1-4]
	\arrow["{r^Y_0}"', from=2-2, to=2-1]
	\arrow["{r^Y_1}"', from=2-3, to=2-2]
	\arrow["{r^Y_2}"', from=2-4, to=2-3]
	\arrow["{r^Y_3}"', from=2-5, to=2-4]
	\arrow["{f_0}"', from=1-1, to=2-1]
	\arrow["{f_1}"', from=1-2, to=2-2]
	\arrow["{f_2}"', from=1-3, to=2-3]
	\arrow["{f_3}"', from=1-4, to=2-4]
\end{tikzcd}\]
  \caption{An example of a morphism in the topos of trees.}
  \label{fig:topos-of-trees-morphism}
\end{figure}


The type operator $\later$ is defined on an object $X$ by
$(\later X)_0 = 1$ and $(\later X)_{i+1} = X_i$.
The restriction maps are given by $r^\later_0 =\, !$, where $!$ is the
unique map into $1$, and $r^\later_{i+1} = r^X_i$.
The morphism $\nxt^X \colon X \to \later X$ is defined pointwise by
$\nxt^X_0 =\, !$, and $\nxt^X_{i+1} = r^X_i$. It is easily checked that
this satisfies the commutativity conditions required of a morphism in $\calS$.
%
Given a morphism $f \colon \later X \to X$, i.e., a family of functions
$f_i \colon (\later X)_i \to X_i$ that commute with the restrictions in the appropriate way,
we define $\fix(f) \colon 1 \to X$ pointwise
by $\fix(f)_i = f_{i} \circ \dots \circ f_0$.
This can be visualized as a diagram in the category of sets as shown in
Figure \ref{fig:topos-of-trees-fix}.
% Observe that the fixpoint is a \emph{global element} in the topos of trees.
% Global elements allow us to view the entire computation on a global level.


% https://q.uiver.app/?q=WzAsOCxbMSwwLCJYXzAiXSxbMiwwLCJYXzEiXSxbMywwLCJYXzIiXSxbMCwwLCIxIl0sWzAsMSwiWF8wIl0sWzEsMSwiWF8xIl0sWzIsMSwiWF8yIl0sWzMsMSwiWF8zIl0sWzMsNCwiZl8wIl0sWzAsNSwiZl8xIl0sWzEsNiwiZl8yIl0sWzIsNywiZl8zIl0sWzAsMywiISIsMl0sWzEsMCwicl5YXzAiLDJdLFsyLDEsInJeWF8xIiwyXSxbNSw0LCJyXlhfMCJdLFs2LDUsInJeWF8xIl0sWzcsNiwicl5YXzIiXV0=
% \[\begin{tikzcd}[ampersand replacement=\&]
% 	1 \& {X_0} \& {X_1} \& {X_2} \\
% 	{X_0} \& {X_1} \& {X_2} \& {X_3}
% 	\arrow["{f_0}", from=1-1, to=2-1]
% 	\arrow["{f_1}", from=1-2, to=2-2]
% 	\arrow["{f_2}", from=1-3, to=2-3]
% 	\arrow["{f_3}", from=1-4, to=2-4]
% 	\arrow["{!}"', from=1-2, to=1-1]
% 	\arrow["{r^X_0}"', from=1-3, to=1-2]
% 	\arrow["{r^X_1}"', from=1-4, to=1-3]
% 	\arrow["{r^X_0}", from=2-2, to=2-1]
% 	\arrow["{r^X_1}", from=2-3, to=2-2]
% 	\arrow["{r^X_2}", from=2-4, to=2-3]
% \end{tikzcd}\]

% \begin{figure}
%   % https://q.uiver.app/?q=WzAsMTksWzEsMiwiWF8wIl0sWzIsMywiWF8xIl0sWzMsMSwiMSJdLFswLDEsIjEiXSxbMCwyLCJYXzAiXSxbMSwzLCJYXzEiXSxbMSwxLCIxIl0sWzIsMSwiMSJdLFsyLDIsIlhfMCJdLFsyLDQsIlhfMiJdLFszLDIsIlhfMCJdLFszLDMsIlhfMSJdLFszLDQsIlhfMiJdLFszLDUsIlhfMyJdLFs0LDIsIlxcY2RvdHMiXSxbMCwwLCJcXGZpeChmKV8wIl0sWzEsMCwiXFxmaXgoZilfMSJdLFsyLDAsIlxcZml4KGYpXzIiXSxbMywwLCJcXGZpeChmKV8zIl0sWzMsNCwiZl8wIl0sWzAsNSwiZl8xIl0sWzYsMCwiZl8wIl0sWzcsOCwiZl8wIl0sWzgsMSwiZl8xIl0sWzEsOSwiZl8yIl0sWzIsMTAsImZfMCJdLFsxMCwxMSwiZl8xIl0sWzExLDEyLCJmXzIiXSxbMTIsMTMsImZfMyJdXQ==
%   \[\begin{tikzcd}[ampersand replacement=\&]
%     {\fix(f)_0} \& {\fix(f)_1} \& {\fix(f)_2} \& {\fix(f)_3} \\
%     1 \& 1 \& 1 \& 1 \\
%     {X_0} \& {X_0} \& {X_0} \& {X_0} \& \cdots \\
%     \& {X_1} \& {X_1} \& {X_1} \\
%     \&\& {X_2} \& {X_2} \\
%     \&\&\& {X_3}
%     \arrow["{f_0}", from=2-1, to=3-1]
%     \arrow["{f_1}", from=3-2, to=4-2]
%     \arrow["{f_0}", from=2-2, to=3-2]
%     \arrow["{f_0}", from=2-3, to=3-3]
%     \arrow["{f_1}", from=3-3, to=4-3]
%     \arrow["{f_2}", from=4-3, to=5-3]
%     \arrow["{f_0}", from=2-4, to=3-4]
%     \arrow["{f_1}", from=3-4, to=4-4]
%     \arrow["{f_2}", from=4-4, to=5-4]
%     \arrow["{f_3}", from=5-4, to=6-4]
%   \end{tikzcd}\]
%   \caption{The first few approximations to the guarded fixpoint of $f$.}
%   \label{fig:topos-of-trees-fix-approx}
% \end{figure}


\begin{figure}
  % https://q.uiver.app/?q=WzAsNixbMywwLCIxIl0sWzAsMiwiWF8wIl0sWzIsMiwiWF8xIl0sWzQsMiwiWF8yIl0sWzYsMiwiWF8zIl0sWzgsMiwiXFxkb3RzIl0sWzIsMSwicl5YXzAiXSxbNCwzLCJyXlhfMiJdLFswLDEsIlxcZml4KGYpXzAiLDFdLFswLDIsIlxcZml4KGYpXzEiLDFdLFswLDMsIlxcZml4KGYpXzIiLDFdLFswLDQsIlxcZml4KGYpXzMiLDFdLFswLDUsIlxcY2RvdHMiLDMseyJzdHlsZSI6eyJib2R5Ijp7Im5hbWUiOiJub25lIn0sImhlYWQiOnsibmFtZSI6Im5vbmUifX19XSxbMywyLCJyXlhfMSJdLFs1LDQsInJeWF8zIl1d
  \[\begin{tikzcd}[ampersand replacement=\&,column sep=2.25em]
    \&\&\& 1 \\
    \\
    {X_0} \&\& {X_1} \&\& {X_2} \&\& {X_3} \&\& \dots
    \arrow["{r^X_0}", from=3-3, to=3-1]
    \arrow["{r^X_2}", from=3-7, to=3-5]
    \arrow["{\fix(f)_0}"{description}, from=1-4, to=3-1]
    \arrow["{\fix(f)_1}"{description}, from=1-4, to=3-3]
    \arrow["{\fix(f)_2}"{description}, from=1-4, to=3-5]
    \arrow["{\fix(f)_3}"{description}, from=1-4, to=3-7]
    \arrow["\cdots"{marking}, draw=none, from=1-4, to=3-9]
    \arrow["{r^X_1}", from=3-5, to=3-3]
    \arrow["{r^X_3}", from=3-9, to=3-7]
  \end{tikzcd}\]
  \caption{The guarded fixpoint of $f$.}
  \label{fig:topos-of-trees-fix}
\end{figure}

% TODO global elements?

\section{Syntactic Theory of Gradually Typed Lambda Calculus}\label{sec:GTLC}

Here we give an overview of a fairly standard cast calculus for
gradual typing along with its (in-)equational theory that capture our
desired notion of type-based reasoning and graduality. The main
departure from prior work is our explicit treatment of type precision
derivations and an equational theory of those derivations.

We give the basic syntax and select typing rules in
Figure~\ref{fig:gtlc-syntax}. We include a dynamic type, a type of
numbers, the call-by-value function type $A \ra A'$ and products.
%
We include a syntax for \emph{type precision} derivations $c : A
\ltdyn A'$; the typing is given in Figure~\ref{fig:gtlc-syntax}.
%
Any type precision derivation $c : A \ltdyn A'$ induces a pair of
casts, the upcast $\upc c : A \ra A'$ and the downcast $\dnc c : A' \ra
A$.
%
The syntactic intuition is that $c$ is a proof that $A$ is ``less
dynamic'' than $A'$. Semantically, this gives us coercions back and
forth where the upcast is (to a first-order) a pure function whereas
the downcast can fail.
%
These casts are inserted automatically in an elaboration from a
surface language. In this work, we are focused on semantic aspects and
so elide these standard details.
%
The syntax of precision derivations includes reflexivity $r(A)$ and
transitivity $cc'$ as well as monotonicity $c \ra c'$ and $c \times
c'$ that are \emph{covariant} in all arguments and finally generators
$\inat,\iarr,\itimes$ that correspond to the type tags of our dynamic
type.
%
We additionally impose an equational theory $c \equiv c'$ on the
derivations that implies that the corresponding casts are weakly
bisimilar in the semantics.
%
We impose category axioms for the reflexivity and
transitivity and functoriality for the monotonicity rules.
%
We note the following two admissible principles: any two derivations
$c,c' : A \ltdyn A'$ of the same fact are equivalent $c \equiv c'$ and
for any $A$, there is a derivation $\textrm{dyn}(A): A \ltdyn\dyn$. That is, $\dyn$ is the ``most dynamic'' type.

There is a more common set of rules for type precision where reflexivity and
transitivity are admissible, and whenever $A \ltdyn A'$, there is a unique
precision derivation witnessing this. These rules are shown in Section
\ref{sec:appendix-gtlc-syntax} in the Appendix. The reason for choosing our
system rather than that one is that in our semantics, equivalent type precision
derivations do not denote \emph{equal} relations. Instead, they denote
relations that are \emph{quasi-equivalent}, i.e., if two terms are related by
one then they are related also by the other up to insertion of delays (see
Definition \ref{def:quasi-equivalent} for the details).
%
However, because all type precision derivations in our system are equivalent, it
is straightforward to define a translation from the more standard system of type
and term precision into ours, so ultimately our graduality proofs can still be
applied to the standard formulation.

\begin{figure}
  \begin{mathpar}
    \begin{array}{rcl}
    \text{Types } A &::=& \nat \alt \,\dyn \alt A \ra A' \alt A \times A'\\
    \text{Type Precision } c &::=& r(A) \alt c c' \alt \iarr \alt \inat \alt \itimes \alt c \ra c' \alt c \times c'\\
    \text{Values } V &::=& x \alt \upc c V \alt \zro \alt \suc\, V \alt \lda{x}{M} \alt (V,V') \\ 
    \text{Terms } M,N &::=& \err\alt \upc c M \alt \dnc c M \alt \zro \alt \suc\, M \alt \lda{x}{M} \\ 
     &&\alt M\, N \alt (M,N) \alt \textrm{let } (x,y) = M \textrm{ in } N\\
    \text{Contexts } \Gamma &::= &\cdot \alt \Gamma, x : A \\
    \text{Ctx Precision } \Delta &::=& \cdot\alt \Delta,x:c
  \end{array}

  \inferrule
  {\Gamma \vdash M : A \and c : A \ltdyn A'}
  {\Gamma \vdash \upc c M : A'}

  \inferrule
  {\Gamma \vdash N : A' \and c : A \ltdyn A'}
  {\Gamma \vdash \dnc c N : A}

  \inferrule{}{\Gamma \vdash \mho : A}
  \end{mathpar}
  \begin{mathpar}
    \inferrule{}{r(A) : A \ltdyn A}\and
    \inferrule{c : A \ltdyn A' \and c' : A' \ltdyn A''}{cc' : A \ltdyn A''}\and
    \inferrule{}{\iarr \colon \dyn \ra \dyn \ltdyn \dyn}\and
    \inferrule{}{\inat \colon \nat \ltdyn \dyn}\and
    \inferrule{}{\itimes \colon \dyn \times \dyn \ltdyn \dyn}\and
    \inferrule{c_i : A_i \ltdyn A_i' \and c_o : A_o \ltdyn A_o'}{c_i \ra c_o : (A_i \ra A_o) \ltdyn (A_i' \ra A_o')}\and
    \inferrule{c_1 : A_1 \ltdyn A_1' \and c_2 : A_2 \ltdyn A_2'}{c_1 \times c_2 : (A_1 \times A_2) \ltdyn (A_1' \times A_2')}\and
     r(A)c \equiv c\and
     c \equiv cr(A')\and
     c(c'c'') \equiv (cc')c''\and
     r(A_i \ra A_o) \equiv r(A_i) \ra r(A_o)\and
     r(A_1\times A_2) \equiv r(A_1) \times r(A_2)\and
     (c_i \ra c_o)(c_i' \ra c_o')\equiv (c_ic_i' \ra c_oc_o') \and
     (c_1\times c_2)(c_1'\times c_2')\equiv (c_1c_1' \times c_2c_2')
  \end{mathpar}
  \caption{GTLC Cast Calculus Syntax, Type Precision Derivations and Precision Equivalence}
  \label{fig:gtlc-syntax}
\end{figure}

Next, we consider the axiomatic (in)equational reasoning principles
for terms: $\beta\eta$ equality and term precision in
Figure~\ref{fig:term-prec}.
%
We include standard CBV $\beta\eta$ rules for function and product
types, as well as equations stating that casts are given functorially.
%
Next, we have \emph{term} precision, an extension of
type precision to terms.
%
The form of the term precision rule is $\Delta \vdash M \ltdyn M' : c$
where $\Delta$ is a context where variables are assigned to type
precision derivations.
%
The judgment is only well formed when every use of $x : c$ for $c : A
\ltdyn A'$ is used with type $A$ in $M$ and $A'$ in $M'$ and similarly
the output types match $c$.
%
We elide the congruence rules for every type constructor, e.g., that
$M \ltdyn M'$ and $N \ltdyn N'$ that $M\,N \ltdyn M'\,N'$.
%
With such congruence rules, reflexivity $M \ltdyn M$ is
admissible. Transitivity, on the other hand, is intentionally not
taken as a primitive rule, matching the original formulation of the
dynamic gradual guarantee \cite{siek_et_al:LIPIcs:2015:5031}.
%
We include a rule that says that equivalent type precision derivations
$c \equiv c'$ are equivalent for the purposes of term precision.
%
% Removed retraction
%
% The next rule is the \emph{retraction} principle, which states that a
% downcast after an upcast is equivalent to doing nothing at all, since
% intuitively the upcasted value should already satisfy the type. Here
% $\equidyn$ means we require each is $\ltdyn$ the other, with
% reflexivity precision derivations.
%
Finally, we include 4 rules for reasoning about casts. Intuitively
these say that the upcast is a kind of \emph{least upper bound} and
dually that the downcast is a \emph{greatest lower bound}.

As a higher-order gradually typed language, we inherently have to deal
with two effects: errors and divergence. Errors arise from failing
casts, e.g. casting a number to dynamic to a function
$\dnc{\iarr}\upc{\inat} x$. Divergence arises because our dynamic type
allows us to encode untyped lambda calculus, and so we can encode the
$\Omega$ term with the help of casts $\Omega = (\lambda
x:\dyn. (\dnc{\iarr} x)x)(\upc{\iarr}(\lambda x:\dyn. (\dnc{\iarr}
x)x))$.

\begin{figure}
  \begin{mathpar}
  (\lambda x. M)(V) = M[V/x] \and (V : A \ra A') = \lambda x. V\,x\\

   \textrm{let } (x,y) = (V,V') \textrm{ in } N = N[V/x,V'/y] \and
   M[V:A\times A'/p] = \textrm{let } (x,y) = V \textrm{ in } M[(x,y)/p]

  % Removed these as they are not needed. The next rule implies that they are quasi-equivalent.
  % \upc{(r(A))}M = M \and
  % \upc{c'}\upc{c}M = \upc{cc'}M \and
  % \dnc{(r(A))}M = M \and
  % \dnc{c}\dnc{c'}M = \dnc{cc'}M

  \inferrule*[right=EquivTyPrec]
  {\Delta\vdash M \ltdyn M' : c \and c \equiv c'}
  {\Delta\vdash M \ltdyn M' : c'}

  \inferrule*[right=ErrBot]
  {}
  {\Delta \vdash \mho \ltdyn M : c}

  % Removed retraction
  % \inferrule
  % {}
  % {\dnc {c} \upc {c} M \equidyn M}

  \inferrule*[right=UpL]
  {M \ltdyn M' : cc_r}
  {\upc {c} M \ltdyn M' : c_r}

  \inferrule*[right=UpR]
  {M \ltdyn M' : c_l}
  {M \ltdyn \upc {c} M' : c_lc}

  \inferrule*[right=DnL]
  {M \ltdyn M' : c_r}
  {\dnc {c} M \ltdyn M' : cc_r}

  \inferrule*[right=DnR]
  {M \ltdyn M' : c_lc}
  {M \ltdyn \dnc {c} M' : c_l}
  \end{mathpar}
  \caption{Equality and Term Precision Rules (Selected)}
  \label{fig:term-prec}
\end{figure}

Our goal in the remainder of this work is to develop compositional
denotational semantics of types, terms, type and term precision from
which we can easily extract a big step semantics that satisfies
graduality and respects the equational theory of the calculus.

%% Here we describe the syntax and typing for the gradually-typed lambda calculus.
%% We also give the rules for syntactic type and term precision.
%% % We define four separate calculi: the normal gradually-typed lambda calculus, which we
%% % call the extensional or \emph{step-insensitive} lambda calculus ($\extlc$),
%% % as well as an \emph{intensional} lambda calculus
%% % ($\intlc$) whose syntax makes explicit the steps taken by a program.

%% Before diving into the details, let us give a brief overview of what we will define.
%% We begin with a gradually-typed lambda calculus $(\extlc)$, which is similar to
%% the normal call-by-value gradually-typed lambda calculus, but differs in that it
%% is actually a fragment of call-by-push-value specialized such that there are no
%% non-trivial computation types. We do this for convenience, as either way
%% we would need a distinction between values and effectful terms; the framework of
%% of call-by-push-value gives us a convenient language to define what we need.

%% We then show that composition of type precision derivations is admissible, as is
%% heterogeneous transitivity for term precision, so it will suffice to consider a new
%% language ($\extlcm$) in which we don't have composition of type precision derivations
%% or heterogeneous transitivity of term precision.

%% We then observe that all casts, except those between $\nat$ and $\dyn$
%% and between $\dyn \ra \dyn$ and $\dyn$, are admissible.
%% % (we can define the cast of a function type functorially using the casts for its domain and codomain).
%% This means it will be sufficient to consider a new language ($\extlcmm$) in which
%% instead of having arbitrary casts, we have injections from $\nat$ and
%% $\dyn \ra \dyn$ into $\dyn$, and case inspections from $\dyn$ to $\nat$ and
%% $\dyn$ to $\dyn \ra \dyn$.

%% From here, we define a \emph{step-sensitive} (also called \emph{intensional}) GSTLC,
%% so-named because it makes the intensional stepping behavior of programs explicit in the syntax.
%% This is accomplished by adding a syntactic ``later'' type and a
%% syntactic $\theta$ that maps terms of type later $A$ to terms of type $A$.
%% Finally, we define a \emph{quotiented} version of the step-sensitive language where
%% we add a rule that equates terms that are the same up to their stepping behavior.

%% % ---------------------------------------------------------------------------------------
%% % ---------------------------------------------------------------------------------------

%% \subsection{Syntax}

%% The language is based on Call-By-Push-Value \cite{levy01:phd}, and as such it has two kinds of types:
%% \emph{value types}, representing pure values, and \emph{computation types}, representing
%% potentially effectful computations.
%% In the language, all computation types have the form $\Ret A$ for some value type $A$.
%% Given a value $V$ of type $A$, the term $\ret V$ views $V$ as a term of computation type $\Ret A$.
%% Given a term $M$ of computation type $B$, the term $\bind{x}{M}{N}$ should be thought of as
%% running $M$ to a value $V$ and then continuing as $N$, with $V$ in place of $x$.


%% We also have value contexts and computation contexts, where the latter can be viewed
%% as a pair consisting of (1) a stoup $\Sigma$, which is either empty or a hole of type $B$,
%% and (2) a (potentially empty) value context $\Gamma$.

%% \begin{align*} % TODO is hole a term?
%%   &\text{Value Types } A := \nat \alt \,\dyn \alt (A \ra A') \\
%%   &\text{Computation Types } B := \Ret A \\
%%   &\text{Value Contexts } \Gamma := \cdot \alt (\Gamma, x : A) \\
%%   &\text{Computation Contexts } \Delta := \cdot \alt \hole B \alt \Delta , x : A \\
%%   &\text{Values } V :=  \zro \alt \suc\, V \alt \lda{x}{M} \alt \up{A}{B} V \\ 
%%   &\text{Terms } M, N := \err_B \alt \matchnat {V} {M} {n} {M'} \\ 
%%   &\quad\quad \alt \ret {V} \alt \bind{x}{M}{N} \alt V_f\, V_x \alt \dn{A}{B} M 
%% \end{align*}

%% The value typing judgment is written $\hasty{\Gamma}{V}{A}$ and 
%% the computation typing judgment is written $\hasty{\Delta}{M}{B}$.

%% \begin{comment}
%% We define substitution for value contexts by the following rules:

%% \begin{mathpar}
%%   \inferrule*
%%   { \gamma : \Gamma' \to \Gamma \and 
%%     \hasty{\Gamma'}{V}{A}}
%%   { (\gamma , V/x ) \colon \Gamma' \to \Gamma , x : A }

%%   \inferrule*
%%   {}
%%   {\cdot \colon \cdot \to \cdot}
%% \end{mathpar}

%% We define substitution for computation contexts by the following rules:

%% \begin{mathpar}
%%     \inferrule*
%%     { \delta : \Delta' \to \Delta \and 
%%       \hasty{\Delta'|_V}{V}{A}}
%%     { (\delta , V/x ) \colon \Delta' \to \Delta , x : A }

%%     \inferrule*
%%     {}
%%     {\cdot \colon \cdot \to \cdot}

%%     \inferrule*
%%     {\hasty{\Delta'}{M}{B}}
%%     {M \colon \Delta' \to \hole{B}}
%% \end{mathpar}
%% \end{comment}

%% The typing rules are as expected, with a cast between $A$ to $B$ allowed only when $A \ltdyn B$.
%% Notice that the upcast of a value is a value, since it always succeeds, while the downcast
%% of a value is a computation, since it may fail.

%% \begin{mathpar}
%%     % Var
%%     \inferrule*{ }{\hasty {\cdot, \Gamma, x : A, \Gamma'} x A}

%%     % Err
%%     \inferrule*{ }{\hasty {\cdot, \Gamma} {\err_B} B} 
  
%%     % Zero and suc
%%     \inferrule*{ }{\hasty \Gamma \zro \nat}
  
%%     \inferrule*{\hasty \Gamma V \nat} {\hasty \Gamma {\suc\, V} \nat}

%%     % Match-nat
%%     \inferrule*
%%     {\hasty \Gamma V \nat \and 
%%      \hasty \Delta M B \and \hasty {\Delta, n : \nat} {M'} B}
%%     {\hasty \Delta {\matchnat {V} {M} {n} {M'}} B}
  
%%     % Lambda
%%     \inferrule* 
%%     {\hasty {\cdot, \Gamma, x : A} M {\Ret A'}} 
%%     {\hasty \Gamma {\lda x M} {A \ra A'}}
  
%%     % App
%%     \inferrule*
%%     {\hasty \Gamma {V_f} {A \ra A'} \and \hasty \Gamma {V_x} A}
%%     {\hasty {\cdot , \Gamma} {V_f \, V_x} {\Ret A'}}

%%     % Ret
%%     \inferrule*
%%     {\hasty \Gamma V A}
%%     {\hasty {\cdot , \Gamma} {\ret\, V} {\Ret A}}
%%     % TODO should this involve a Delta?

%%     % Bind
%%     \inferrule*
%%     {\hasty \Delta M {\Ret A} \and \hasty{\cdot , \Delta|_V , x : A}{N}{B} } % Need x : A in context
%%     {\hasty {\Delta} {\bind{x}{M}{N}} {B}}

%%     % Upcast
%%     \inferrule*
%%     {A \ltdyn A' \and \hasty \Gamma V A}
%%     {\hasty \Gamma {\up A {A'} V} {A'} }

%%     % Downcast
%%     % \inferrule*
%%     % {A \ltdyn A' \and \hasty {\Gamma} V {A'}}
%%     % {\hasty {\cdot, \Gamma} {\dn A {A'} V} {\Ret A}}

%%     \inferrule* % TODO is this correct?
%%     {B \ltdyn B' \and \hasty {\Delta} {M} {B'}}
%%     {\hasty {\Delta} {\dn B {B'} M} {B}}

%% \end{mathpar}


%% In the equational theory, we have $\beta$ and $\eta$ laws for function type,
%% as well a $\beta$ and $\eta$ law for $\Ret A$.

%% % TODO do we need to add a substitution rule here?
%% \begin{mathpar}
%%   % Function Beta and Eta
%%   \inferrule*
%%   {\hasty {\cdot, \Gamma, x : A} M {\Ret A'} \and \hasty \Gamma V A}
%%   {(\lda x M)\, V = M[V/x]}

%%   \inferrule*
%%   {\hasty \Gamma V {A \ra A}}
%%   {\Gamma \vdash V = \lda x {V\, x}}

%%   % Ret Beta and Eta
%%   \inferrule*
%%   {}
%%   {(\bind{x}{\ret\, V}{N}) = N[V/x]}

%%   \inferrule*
%%   {\hasty {\hole{\Ret A} , \Gamma} {M} {B}}
%%   {\hole{\Ret A}, \Gamma \vdash M = (\bind{x}{\bullet}{M[\ret\, x]})}

%%   % Match-nat Beta
%%   \inferrule*
%%   {\hasty \Delta M B \and \hasty {\Delta, n : \nat} {M'} B}
%%   {\matchnat{\zro}{M}{n}{M'} = M}

%%   \inferrule*
%%   {\hasty \Gamma V \nat \and 
%%    \hasty \Delta M B \and \hasty {\Delta, n : \nat} {M'} B}
%%   {\matchnat{\suc\, V}{M}{n}{M'} = M'}

%%   % Match-nat Eta
%%   % This doesn't build in substitution
%%   \inferrule*
%%   {\hasty {\Delta , x : \nat} M A}
%%   {M = \matchnat{x} {M[\zro / x]} {n} {M[(\suc\, n) / x]}}



%% \end{mathpar}

%% % ---------------------------------------------------------------------------------------
%% % ---------------------------------------------------------------------------------------

%% \subsection{Type Precision}

%% The type precision rules specify what it means for a type $A$ to be more precise than $A'$.
%% We have reflexivity rules for $\dyn$ and $\nat$, as well as rules that $\nat$ is more precise than $\dyn$
%% and $\dyn \ra \dyn$ is more precise than $\dyn$.
%% We also have a transitivity rule for composition of type precision,
%% and also a rule for function types stating that given $A_i \ltdyn A'_i$ and $A_o \ltdyn A'_o$, we can prove
%% $A_i \ra A_o \ltdyn A'_i \ra A'_o$.
%% Finally, we can lift a relation on value types $A \ltdyn A'$ to a relation $\Ret A \ltdyn \Ret A'$ on
%% computation types.

%% \begin{mathpar}
%%   \inferrule*[right = \dyn]
%%     { }{\dyn \ltdyn\, \dyn}

%%   \inferrule*[right = \nat]
%%     { }{\nat \ltdyn \nat}

%%   \inferrule*[right = $\ra$]
%%     {A_i \ltdyn A'_i \and A_o \ltdyn A'_o }
%%     {(A_i \ra A_o) \ltdyn (A'_i \ra A'_o)}

%%   \inferrule*[right = $\textsf{Inj}_\nat$]
%%     { }{\nat \ltdyn\, \dyn}

%%   \inferrule*[right=$\textsf{Inj}_{\ra}$]
%%     { }
%%     {(\dyn \ra \dyn) \ltdyn\, \dyn}

%%   \inferrule*[right=ValTrans]
%%     {A \ltdyn A' \and A' \ltdyn A''}
%%     {A \ltdyn A''}

%%   \inferrule*[right=CompTrans]
%%     {B \ltdyn B' \and B' \ltdyn B''}
%%     {B \ltdyn B''}

%%   \inferrule*[right=$\Ret{}$]
%%     {A \ltdyn A'}
%%     {\Ret {A} \ltdyn \Ret {A'}}

%%     % TODO are there other rules needed for computation types?

  
%% \end{mathpar}

%% % Type precision derivations
%% Note that as a consequence of this presentation of the type precision rules, we
%% have that if $A \ltdyn A'$, there is a unique precision derivation that witnesses this.
%% As in previous work, we go a step farther and make these derivations first-class objects,
%% known as \emph{type precision derivations}.
%% Specifically, for every $A \ltdyn A'$, we have a derivation $c : A \ltdyn A'$ that is constructed
%% using the rules above. For instance, there is a derivation $\dyn : \dyn \ltdyn \dyn$, and a derivation
%% $\nat : \nat \ltdyn \nat$, and if $c_i : A_i \ltdyn A_i$ and $c_o : A_o \ltdyn A'_o$, then
%% there is a derivation $c_i \ra c_o : (A_i \ra A_o) \ltdyn (A'_i \ra A'_o)$. Likewise for
%% the remaining rules. The benefit to making these derivations explicit in the syntax is that we
%% can perform induction over them.
%% Note also that for any type $A$, we use $A$ to denote the reflexivity derivation that $A \ltdyn A$,
%% i.e., $A : A \ltdyn A$.
%% Finally, observe that for type precision derivations $c : A \ltdyn A'$ and $c' : A' \ltdyn A''$, we
%% can define (via the rule ValComp) their composition $c \relcomp c' : A \ltdyn A''$.
%% The same holds for computation type precision derivations.
%% This notion will be used below in the statement of transitivity of the term precision relation.

%% % ---------------------------------------------------------------------------------------
%% % ---------------------------------------------------------------------------------------

%% \subsection{Term Precision}

%% We allow for a \emph{heterogeneous} term precision judgment on terms values $V$ of type
%% $A$ and $V'$ of type $A'$ provided that $A \ltdyn A'$ holds. Likewise, for computation
%% types $B \ltdyn B'$, if $M$ has type $B$ and $M'$ has type $B'$, we can form the judgment
%% that $M \ltdyn M'$.

%% % Type precision contexts
%% % TODO should we include the formal definitions of value and computation type precision contexts?
%% In order to deal with open terms, we will need the notion of a type precision \emph{context}, which we denote
%% $\gamlt$. This is similar to a normal context but instead of mapping variables to types,
%% it maps variables $x$ to related types $A \ltdyn A'$, where $x$ has type $A$ in the left-hand term
%% and $B$ in the right-hand term. We may also write $x : d$ where $d : A \ltdyn A'$ to indicate this.
%% Similarly, we have computation type precision contexts $\Delta^\ltdyn$. Similar to ``normal'' computation
%% type precision contexts $\Delta$, these consist of (1) a stoup $\Sigma$ which is either empty or
%% has a hole $\hole{d}$ for some computation type precision derivation $d$, and (2) a value type precision context
%% $\Gamma^\ltdyn$.

%% % An equivalent way of thinking of type precision contexts is as a pair of ``normal" typing
%% % contexts $\Gamma, \Gamma'$ with the same domain such that $\Gamma(x) \ltdyn \Gamma'(x)$ for
%% % each $x$ in the domain.
%% % We will write $\gamlt : \Gamma \ltdyn \Gamma'$ when we want to emphasize the pair of contexts.
%% % Conversely, if we are given $\gamlt$, we write $\gamlt_l$ and $\gamlt_r$ for the normal typing contexts on each side.

%% An equivalent way of thinking of a type precision context $\gamlt$ is as a
%% pair of ``normal" typing contexts, $\gamlt_l$ and $\gamlt_r$, with the same
%% domain and such that $\gamlt_l(x) \ltdyn \gamlt_r(x)$ for each $x$ in the domain.
%% We will write $\gamlt : \gamlt_l \ltdyn \gamlt_r$ when we want to emphasize the pair of contexts.

%% As with type precision derivations, we write $\Gamma$ to mean the ``reflexivity" type precision context
%% $\Gamma : \Gamma \ltdyn \Gamma$.
%% Concretely, this consists of reflexivity type precision derivations $\Gamma(x) \ltdyn \Gamma(x)$ for
%% each $x$ in the domain of $\Gamma$.
%% Similarly, we also have reflexivity for computation type precision contexts.
%% %
%% Furthermore, we write $\gamlt_1 \relcomp \gamlt_2$ to denote the ``composition'' of $\gamlt_1$ and $\gamlt_2$
%% --- that is, the precision context whose value at $x$ is the type precision derivation
%% $\gamlt_1(x) \relcomp \gamlt_2(x)$. This of course assumes that each of the type precision
%% derivations is composable, i.e., that the RHS of $\gamlt_1(x)$ is the same as the left-hand side of $\gamlt_2(x)$.
%% We define the same for computation type precision contexts $\deltalt_1$ and $\deltalt_2$,
%% provided that both the computation type precision contexts have the same ``shape'', which is defined as
%% (1) either the stoup is empty in both, or the stoup has a hole in both, say $\hole{d}$ and $\hole{d'}$
%% where $d$ and $d'$ are composable, and (2) their value type precision contexts are composable as described above.

%% The rules for term precision come in two forms. We first have the \emph{congruence} rules,
%% one for each term constructor. These assert that the term constructors respect term precision.
%% The congruence rules are as follows:

%% \begin{mathpar}

%%   \inferrule*[right = Var]
%%     { c : A \ltdyn B \and \gamlt(x) = (A, B) } 
%%     { \etmprec {\gamlt} x x c }

%%   \inferrule*[right = Zro]
%%     { } {\etmprec \gamlt \zro \zro \nat }

%%   \inferrule*[right = Suc]
%%     { \etmprec \gamlt V {V'} \nat } {\etmprec \gamlt {\suc\, V} {\suc\, V'} \nat}

%%   \inferrule*[right = MatchNat]
%%   {\etmprec \gamlt V {V'} \nat \and 
%%     \etmprec \deltalt M {M'} d \and \etmprec {\deltalt, n : \nat} {N} {N'} d}
%%   {\etmprec \deltalt {\matchnat {V} {M} {n} {N}} {\matchnat {V'} {M'} {n} {N'}} d}

%%   \inferrule*[right = Lambda]
%%     { c_i : A_i \ltdyn A'_i \and 
%%       c_o : A_o \ltdyn A'_o \and 
%%       \etmprec {\cdot , \gamlt , x : c_i} {M} {M'} {\Ret c_o} } 
%%     { \etmprec \gamlt {\lda x M} {\lda x {M'}} {(c_i \ra c_o)} }

%%   \inferrule*[right = App]
%%     { c_i : A_i \ltdyn A'_i \and
%%       c_o : A_o \ltdyn A'_o \\\\
%%       \etmprec \gamlt {V_f} {V_f'} {(c_i \ra c_o)} \and
%%       \etmprec \gamlt {V_x} {V_x'} {c_i}
%%     } 
%%     { \etmprec {\cdot , \gamlt} {V_f\, V_x} {V_f'\, V_x'} {\Ret {c_o}}}

%%   \inferrule*[right = Ret]
%%     {\etmprec {\gamlt} V {V'} c}
%%     {\etmprec {\cdot , \gamlt} {\ret\, V} {\ret\, V'} {\Ret c}}

%%   \inferrule*[right = Bind]
%%     {\etmprec {\deltalt} {M} {M'} {\Ret c} \and 
%%      \etmprec {\cdot , \deltalt|_V , x : c} {N} {N'} {d} }
%%     {\etmprec {\deltalt} {\bind {x} {M} {N}} {\bind {x} {M'} {N'}} {d}}
%% \end{mathpar}

%% We then have additional equational axioms, including transitivity, $\beta$ and $\eta$ laws, and
%% rules characterizing upcasts as least upper bounds, and downcasts as greatest lower bounds.

%% We write $M \equidyn N$ to mean that both $M \ltdyn N$ and $N \ltdyn M$.

%% % TODO adapt these for value/computation distinction
%% % TODO substitution rules for values and terms?
%% \begin{mathpar}
%%   \inferrule*[right = $\err$]
%%     { \hasty {\deltalt_l} M B }
%%     {\etmprec {\Delta} {\err_B} M B}

%%   \inferrule*[right = Transitivity]
%%     { d : B \ltdyn B' \and d' : B' \ltdyn B'' \\\\
%%      \etmprec {\deltalt_1} {M} {M'} {d} \and
%%      \etmprec {\deltalt_2} {M'} {M''} {d'} } 
%%     {\etmprec {\deltalt_1 \relcomp \deltalt_2} {M} {M''} {d \relcomp d'} }


%%   \inferrule*[right = $\beta$-fun]
%%     { \hasty {\cdot, \Gamma, x : A_i} M {\Ret A_o} \and
%%       \hasty {\Gamma} V {A_i} } 
%%     { \etmequidyn {\cdot, \Gamma} {(\lda x M)\, V} {M[V/x]} {\Ret A_o} }

%%   \inferrule*[right = $\eta$-fun]
%%     { \hasty {\Gamma} {V} {A_i \ra A_o} } 
%%     { \etmequidyn \Gamma {\lda x (V\, x)} V {A_i \ra A_o} }

%%   % Match-nat beta and eta



%%   \inferrule*[right = $\beta$-ret]
%%     {}
%%     {\bind{x}{\ret\, V}{N} \equidyn N[V/x]}

%%   \inferrule*[right = $\eta$-ret]
%%     {\hasty {\hole{\Ret A} , \Gamma} {M} {B}}
%%     {\hole{\Ret A}, \Gamma \vdash M \equidyn \bind{x}{\bullet}{M[\ret\, x]}}
    

%%   % Could specify \gamlt : \Gamma \ltdyn \Gamma'
%%   % and then we wouldn't need to say l and r

%%   \inferrule*[right = UpR]
%%     { d : A \ltdyn A' \and 
%%       \hasty {\Delta} {M} {A} } 
%%     { \etmprec {\Delta} {M} {\up {A} {A'} M} {d}  }

%%   \inferrule*[right = UpL]
%%     { d : A \ltdyn A' \and
%%       \etmprec {\deltalt} {M} {N} {d} } 
%%     { \etmprec {\deltalt} {\up {A} {A'} M} {N} {A'} }

%%   \inferrule*[right = DnL]
%%     { d : B \ltdyn B' \and 
%%       \hasty {\Delta} {M} {B'} } 
%%     { \etmprec {\Delta} {\dn {B} {B'} M} {M} {d} }

%%   \inferrule*[right = DnR]
%%     { d : B \ltdyn B' \and
%%       \etmprec {\deltalt} {M} {N} {d} } 
%%     { \etmprec {\deltalt} {M} {\dn {B} {B'} N} {B} }
%% \end{mathpar}

%% % TODO explain the least upper bound/greatest lower bound rules
%% The rules UpR, UpL, DnL, and DnR were introduced in \cite{new-licata18} as a means
%% of cleanly axiomatizing the intended behavior of casts in a way that
%% doesn't depend on the specific constructs of the language.
%% Intuitively, rule UpR says that the upcast of $M$ is an upper bound for $M$
%% in that $M$ may error more, and UpL says that the upcast is the \emph{least}
%% such upper bound, in that it errors more than any other upper bound for $M$.
%% Conversely, DnL says that the downcast of $M$ is a lower bound, and DnR says
%% that it is the \emph{greatest} lower bound.
%% % These rules provide a clean axiomatization of the behavior of casts that doesn't
%% % depend on the specific constructs of the language.

%% % ---------------------------------------------------------------------------------------
%% % ---------------------------------------------------------------------------------------
%% \subsection{Removing Transitivity as a Primitive}

%% The first observation we make is that transitivity of type precision, and heterogeneous
%% transitivity of term precision, are admissible. That is, consider a related language which
%% is the same as $\extlc$ except that we have removed the composition rule for type precision and
%% the heterogeneous transitivity rule for type precision. Denote this language by $\extlcm$.
%% We claim that in this new language, the rules we removed are derivable from the remaining rules.

%% To see this, suppose $\gamlt : \Gamma \ltdyn \Gamma'$ and $d : A \ltdyn A'$, and that
%%  $\etmprec {\gamlt} {V} {V'} {d}$, as shown in the diagram below:

%% % https://q.uiver.app/?q=WzAsNCxbMCwwLCJcXEdhbW1hIl0sWzAsMSwiXFxHYW1tYSciXSxbMSwwLCJBIl0sWzEsMSwiQSciXSxbMCwxLCJcXGx0ZHluIiwzLHsic3R5bGUiOnsiYm9keSI6eyJuYW1lIjoibm9uZSJ9LCJoZWFkIjp7Im5hbWUiOiJub25lIn19fV0sWzIsMywiXFxsdGR5biIsMyx7InN0eWxlIjp7ImJvZHkiOnsibmFtZSI6Im5vbmUifSwiaGVhZCI6eyJuYW1lIjoibm9uZSJ9fX1dLFswLDIsIlYiXSxbMSwzLCJWJyJdLFs2LDcsIlxcbHRkeW4iLDMseyJzaG9ydGVuIjp7InNvdXJjZSI6MjAsInRhcmdldCI6MjB9LCJzdHlsZSI6eyJib2R5Ijp7Im5hbWUiOiJub25lIn0sImhlYWQiOnsibmFtZSI6Im5vbmUifX19XV0=
%% \[\begin{tikzcd}[ampersand replacement=\&]
%% 	\Gamma \& A \\
%% 	{\Gamma'} \& {A'}
%% 	\arrow["\ltdyn"{marking}, draw=none, from=1-1, to=2-1]
%% 	\arrow["\ltdyn"{marking}, draw=none, from=1-2, to=2-2]
%% 	\arrow[""{name=0, anchor=center, inner sep=0}, "V", from=1-1, to=1-2]
%% 	\arrow[""{name=1, anchor=center, inner sep=0}, "{V'}", from=2-1, to=2-2]
%% 	\arrow["\ltdyn"{marking}, draw=none, from=0, to=1]
%% \end{tikzcd}\]

%% Now note that this is equivalent, by the cast rule UpL, to
%% $\etmprec {\Gamma'} {\up{A}{A'} V} {V'} {A'}$,
%% where as noted above, $\Gamma'$ refers to the context $\Gamma'$ viewed as a reflexivity
%% precision context and likewise the $A'$ at the end refers to the reflexivity derivation $A' \ltdyn A'$.

%% % https://q.uiver.app/?q=WzAsMixbMCwwLCJcXEdhbW1hJyJdLFsxLDAsIkEnIl0sWzAsMSwiXFx1cCB7QX0ge0EnfSBWIiwwLHsiY3VydmUiOi0yfV0sWzAsMSwiViciLDIseyJjdXJ2ZSI6Mn1dLFsyLDMsIlxcbHRkeW4iLDMseyJzaG9ydGVuIjp7InNvdXJjZSI6MjAsInRhcmdldCI6MjB9LCJzdHlsZSI6eyJib2R5Ijp7Im5hbWUiOiJub25lIn0sImhlYWQiOnsibmFtZSI6Im5vbmUifX19XV0=
%% \[\begin{tikzcd}[ampersand replacement=\&]
%% 	{\Gamma'} \& {A'}
%% 	\arrow[""{name=0, anchor=center, inner sep=0}, "{\up {A} {A'} V}", curve={height=-12pt}, from=1-1, to=1-2]
%% 	\arrow[""{name=1, anchor=center, inner sep=0}, "{V'}"', curve={height=12pt}, from=1-1, to=1-2]
%% 	\arrow["\ltdyn"{marking}, draw=none, from=0, to=1]
%% \end{tikzcd}\]

%% Now consider the situation shown below:

%% % https://q.uiver.app/?q=WzAsNixbMCwwLCJcXEdhbW1hIl0sWzAsMSwiXFxHYW1tYSciXSxbMCwyLCJcXEdhbW1hJyciXSxbMiwwLCJBIl0sWzIsMSwiQSciXSxbMiwyLCJBJyciXSxbMiw1LCJWJyciXSxbMSw0LCJWJyJdLFswLDMsIlYiXSxbMyw0LCJcXGx0ZHluIiwzLHsic3R5bGUiOnsiYm9keSI6eyJuYW1lIjoibm9uZSJ9LCJoZWFkIjp7Im5hbWUiOiJub25lIn19fV0sWzQsNSwiXFxsdGR5biIsMyx7InN0eWxlIjp7ImJvZHkiOnsibmFtZSI6Im5vbmUifSwiaGVhZCI6eyJuYW1lIjoibm9uZSJ9fX1dLFswLDEsIlxcbHRkeW4iLDMseyJzdHlsZSI6eyJib2R5Ijp7Im5hbWUiOiJub25lIn0sImhlYWQiOnsibmFtZSI6Im5vbmUifX19XSxbMSwyLCIiLDEseyJzdHlsZSI6eyJib2R5Ijp7Im5hbWUiOiJub25lIn0sImhlYWQiOnsibmFtZSI6Im5vbmUifX19XSxbMSwyLCJcXGx0ZHluIiwzLHsic3R5bGUiOnsiYm9keSI6eyJuYW1lIjoibm9uZSJ9LCJoZWFkIjp7Im5hbWUiOiJub25lIn19fV0sWzgsNywiXFxsdGR5biIsMyx7InNob3J0ZW4iOnsic291cmNlIjoyMCwidGFyZ2V0IjoyMH0sInN0eWxlIjp7ImJvZHkiOnsibmFtZSI6Im5vbmUifSwiaGVhZCI6eyJuYW1lIjoibm9uZSJ9fX1dLFs3LDYsIlxcbHRkeW4iLDMseyJzaG9ydGVuIjp7InNvdXJjZSI6MjAsInRhcmdldCI6MjB9LCJzdHlsZSI6eyJib2R5Ijp7Im5hbWUiOiJub25lIn0sImhlYWQiOnsibmFtZSI6Im5vbmUifX19XV0=
%% \[\begin{tikzcd}[ampersand replacement=\&]
%% 	\Gamma \&\& A \\
%% 	{\Gamma'} \&\& {A'} \\
%% 	{\Gamma''} \&\& {A''}
%% 	\arrow[""{name=0, anchor=center, inner sep=0}, "{V''}", from=3-1, to=3-3]
%% 	\arrow[""{name=1, anchor=center, inner sep=0}, "{V'}", from=2-1, to=2-3]
%% 	\arrow[""{name=2, anchor=center, inner sep=0}, "V", from=1-1, to=1-3]
%% 	\arrow["\ltdyn"{marking}, draw=none, from=1-3, to=2-3]
%% 	\arrow["\ltdyn"{marking}, draw=none, from=2-3, to=3-3]
%% 	\arrow["\ltdyn"{marking}, draw=none, from=1-1, to=2-1]
%% 	\arrow[draw=none, from=2-1, to=3-1]
%% 	\arrow["\ltdyn"{marking}, draw=none, from=2-1, to=3-1]
%% 	\arrow["\ltdyn"{marking}, draw=none, from=2, to=1]
%% 	\arrow["\ltdyn"{marking}, draw=none, from=1, to=0]
%% \end{tikzcd}\]


%% Using the above observation, we have that the above is equivalent to

%% % https://q.uiver.app/?q=WzAsNCxbMCwwLCJcXEdhbW1hJyJdLFswLDEsIlxcR2FtbWEnJyJdLFsyLDAsIkEnIl0sWzIsMSwiQScnIl0sWzAsMiwiXFx1cCB7QX0ge0EnfSBWIiwwLHsiY3VydmUiOi0yfV0sWzAsMiwiViciLDIseyJjdXJ2ZSI6Mn1dLFsxLDMsIlYnJyIsMix7ImN1cnZlIjoyfV0sWzAsMSwiXFxsdGR5biIsMyx7InN0eWxlIjp7ImJvZHkiOnsibmFtZSI6Im5vbmUifSwiaGVhZCI6eyJuYW1lIjoibm9uZSJ9fX1dLFsyLDMsIlxcbHRkeW4iLDMseyJzdHlsZSI6eyJib2R5Ijp7Im5hbWUiOiJub25lIn0sImhlYWQiOnsibmFtZSI6Im5vbmUifX19XSxbNCw1LCJcXGx0ZHluIiwzLHsic2hvcnRlbiI6eyJzb3VyY2UiOjIwLCJ0YXJnZXQiOjIwfSwic3R5bGUiOnsiYm9keSI6eyJuYW1lIjoibm9uZSJ9LCJoZWFkIjp7Im5hbWUiOiJub25lIn19fV0sWzUsNiwiXFxsdGR5biIsMyx7InNob3J0ZW4iOnsic291cmNlIjoyMCwidGFyZ2V0IjoyMH0sInN0eWxlIjp7ImJvZHkiOnsibmFtZSI6Im5vbmUifSwiaGVhZCI6eyJuYW1lIjoibm9uZSJ9fX1dXQ==
%% \[\begin{tikzcd}[ampersand replacement=\&]
%% 	{\Gamma'} \&\& {A'} \\
%% 	{\Gamma''} \&\& {A''}
%% 	\arrow[""{name=0, anchor=center, inner sep=0}, "{\up {A} {A'} V}", curve={height=-12pt}, from=1-1, to=1-3]
%% 	\arrow[""{name=1, anchor=center, inner sep=0}, "{V'}"', curve={height=12pt}, from=1-1, to=1-3]
%% 	\arrow[""{name=2, anchor=center, inner sep=0}, "{V''}"', curve={height=12pt}, from=2-1, to=2-3]
%% 	\arrow["\ltdyn"{marking}, draw=none, from=1-1, to=2-1]
%% 	\arrow["\ltdyn"{marking}, draw=none, from=1-3, to=2-3]
%% 	\arrow["\ltdyn"{marking}, draw=none, from=0, to=1]
%% 	\arrow["\ltdyn"{marking}, draw=none, from=1, to=2]
%% \end{tikzcd}\]

%% % TODO finish the explanation
  

%% % ---------------------------------------------------------------------------------------
%% % ---------------------------------------------------------------------------------------

%% \subsection{Removing Casts as Primitives}

%% % We now observe that all casts, except those between $\nat$ and $\dyn$
%% % and between $\dyn \ra \dyn$ and $\dyn$, are admissible, in the sense that
%% % we can start from $\extlcm$, remove casts except the aforementioned ones,
%% % and in the resulting language we will be able to derive the other casts.

%% We now observe that all casts, except those between $\nat$ and $\dyn$
%% and between $\dyn \ra \dyn$ and $\dyn$, are admissible.
%% That is, consider a new language ($\extlcmm$) in which
%% instead of having arbitrary casts, we have injections from $\nat$ and
%% $\dyn \ra \dyn$ into $\dyn$, and case inspections from $\dyn$ to $\nat$ and
%% $\dyn$ to $\dyn \ra \dyn$. We claim that in $\extlcmm$, all of the casts
%% present in $\extlcm$ are derivable.
%% It will suffice to verify that casts for function type are derivable.
%% This holds because function casts are constructed inductively from the casts
%% of their domain and codomain. The base case is one of the casts involving $\nat$
%% or $\dyn \ra \dyn$ which are present in $\extlcmm$ as injections and case inspections.


%% The resulting calculus $\extlcmm$ now lacks transitivity of type precision,
%% heterogeneous transitivity of term precision, and arbitrary casts as primitive
%% notions.

%% \begin{align*}
%%   &\text{Value Types } A := \nat \alt \dyn \alt (A \ra A') \\
%%   &\text{Computation Types } B := \Ret A \\
%%   &\text{Value Contexts } \Gamma := \cdot \alt (\Gamma, x : A) \\
%%   &\text{Computation Contexts } \Delta := \cdot \alt \hole B \alt \Delta , x : A \\
%%   &\text{Values } V :=  \zro \alt \suc\, V \alt \lda{x}{M} \alt \injnat V \alt \injarr V \\ 
%%   &\text{Terms } M, N := \err_B \alt \ret {V} \alt \bind{x}{M}{N}
%%     \alt V_f\, V_x \alt
%%     \\ & \quad\quad \casenat{V}{M_{no}}{n}{M_{yes}} 
%%     \alt \casearr{V}{M_{no}}{f}{M_{yes}}
%% \end{align*}

%% In this setting, rather than type precision, it makes more sense to
%% speak of arbitrary \emph{monotone relations} on types, which we denote by $A \rel A'$.
%% We have relations on value types, as well as on computation types. We also have
%% value relation contexts and computation relation contexts, analogous to the value type
%% precision contexts and computation type precision contexts from before.

%% \begin{align*}
%%   &\text{Value Relations } R := \nat \alt \dyn \alt (R \ra R) \alt\, \dyn\, R(V_1, V_2)\\
%%   &\text{Computation Relations } S := \li R \\
%%   &\text{Value Relation Contexts } \Gamma^{\rel} := \cdot \alt \Gamma^{\rel} , A^{\rel} (x_l : A_l , x_r : A_r)\\
%%   &\text{Computation Relation Contexts } \Delta^{\rel} := \cdot \alt \hole{B^{\rel}} \alt 
%%     \Delta^{\rel} , A^{\rel} (x_l : A_l , x_r : A_r)   \\
%% \end{align*}

%% % TODO rules for relations
%% The forms for relations are as follows:

%% \begin{align*}
%%   A^{\rel}      &\colon A_l      \rel A_r \\
%%   \Gamma^{\rel} &\colon \Gamma_l \rel \Gamma_r \\
%%   B^{\rel}      &\colon B_l      \rel B_r \\
%%   \Delta^{\rel} &\colon \Delta_l \rel \Delta_r
%% \end{align*}



%% Figure \ref{fig:relation-rules} shows the rules for relations. We show only those for value types;
%% the corresponding computation type relation rules are analogous.
%% The rules for relations are as follows. First, we require relations to be reflexive.
%% We also require that they are \emph{profunctorial}, in the sense that a relation between
%% $A$ and $A'$ is closed under the ``homogeneous'' relations on both sides.
%% We also require that they satisfy a substitution principle.

%% \begin{figure}
%%   \begin{mathpar}
%%     \inferrule*[right = Reflexivity]
%%     {\hasty \Gamma V A}
%%     {\refl(\Gamma) \vdash \refl(A)(V, V)}

%%     \inferrule*[right = Profunctoriality]
%%     { \refl(\Gamma^{\rel}_l) \vdash  \refl(A^{\rel}_l) (V_l' , V_l) \\\\ 
%%         \Gamma^{\rel}    \vdash    A^{\rel}    (V_l  , V_r) \\\\
%%       \refl(\Gamma^{\rel}_r) \vdash  \refl(A^{\rel}_r) (V_r  , V_r')
%%     }
%%     {\Gamma^{\rel} \vdash A^{\rel} (V_l', V_r')}

%%     \inferrule*[right = Subst]
%%     { \Gamma'^{\rel} \vdash \Gamma^{\rel} (\gamma_l, \gamma_r) \\\\
%%       \Gamma^{\rel} A^{\rel} (V_l, V_r)
%%     }
%%     {\Gamma'^{\rel} \vdash A^{\rel} (V_l[\gamma_l] , V_r[\gamma_r]) }

%%     % \inferrule*[right = TermSubst]
%%     % { \Delta'^{\rel} \vdash \Delta^{\rel} (\delta_l, \delta_r) \\\\
%%     %   \Delta^{\rel} B^{\rel} (M_l, M_r)
%%     % }
%%     % {\Delta'^{\rel} \vdash B^{\rel} (M_l[\delta_l] , M_r[\delta_r]) }

%%   \end{mathpar}
%%   \caption{Rules for value type relations. The rules for computation type relations are analogous.}
%%   \label{fig:relation-rules}
%% \end{figure}

%% We also have a rule for the restriction of a relation along a function,
%% and we have a rule characterizing relation at function type. The latter states that
%% if under the assumption that $x$ is related to $x'$ by $A^{\rel}$, we can show that $M$
%% is related to $M'$ by $\li A'^{\rel}$, then we have that $\lda{x}{M}$ is related to
%% $\lda{x'}{M'}$ by $A^{\rel} \ra A'^{\rel}$.

%% \begin{mathpar}
%%   \mprset{fraction={===}}

%%   % \inferrule*[]
%%   % { A^{\rel}  (x_l, x_r) \vdash A^{\rel} (V_l, V_r) }
%%   % { A'^{\rel} (x_l, x_r) \vdash A^{\rel} (V_l, V_r)(x_l, x_r) }

%%   \inferrule*[right = Restriction]
%%   { \Gamma^{\rel} \vdash A^{\rel} (V_l (V_l'), V_r (V_r')) }
%%   { \Gamma^{\rel} \vdash (A^{\rel} (V_l, V_r)) (V_l', V_r') }

%%   \inferrule*[right = $\text{Rel}_\ra$]
%%   { A^{\rel} (x, x') \vdash (\li A'^{\rel})(M , M') }
%%   {  \vdash (A^{\rel} \ra A'^{\rel}) (\lda{x}{M}) , (\lda{x'}{M'})}

%% \end{mathpar}



%% % New rules
%% Figure \ref{fig:extlc-minus-minus-typing} shows the new typing rules,
%% and Figure \ref{fig:extlc-minus-minus-eqns} shows the equational rules
%% for case-nat (the rules for case-arrow are analogous).

%% \begin{figure}
%%   \begin{mathpar}
%%       % inj-nat
%%       \inferrule*
%%       {\hasty \Gamma M \nat}
%%       {\hasty \Gamma {\injnat M} \dyn}

%%       % inj-arr 
%%       \inferrule*
%%       {\hasty \Gamma M (\dyn \ra \dyn)}
%%       {\hasty \Gamma {\injarr M} \dyn}

%%       % Case nat
%%       \inferrule*
%%       {\hasty{\Delta|_V}{V}{\dyn} \and 
%%         \hasty{\Delta , x : \nat }{M_{yes}}{B} \and 
%%         \hasty{\Delta}{M_{no}}{B}}
%%       {\hasty {\Delta} {\casenat{V}{M_{no}}{n}{M_{yes}}} {B}}
    
%%       % Case arr
%%       \inferrule*
%%       {\hasty{\Delta|_V}{V}{\dyn} \and 
%%         \hasty{\Delta , x : (\dyn \ra \dyn) }{M_{yes}}{B} \and 
%%         \hasty{\Delta}{M_{no}}{B}}
%%       {\hasty {\Delta} {\casearr{V}{M_{no}}{f}{M_{yes}}} {B}}
%%   \end{mathpar}
%%   \caption{New typing rules for $\extlcmm$.}
%%   \label{fig:extlc-minus-minus-typing}
%% \end{figure}


%% \begin{figure}
%%   \begin{mathpar}
%%      % Case-nat Beta
%%      \inferrule*
%%      {\hasty \Gamma V \nat}
%%      {\casenat {\injnat {V}} {M_{no}} {n} {M_{yes}} = M_{yes}[V/n]}

%%      \inferrule*
%%      {\hasty \Gamma V {\dyn \ra \dyn} }
%%      {\casenat {\injarr {V}} {M_{no}} {n} {M_{yes}} = M_{no}}

%%      % Case-nat Eta
%%      \inferrule*
%%      {}
%%      {\Gamma , x :\, \dyn \vdash M = \casenat{x}{M}{n}{M[(\injnat{n}) / x]} }


%%      % Case-arr Beta


%%      % Case-arr Eta


%%   \end{mathpar}
%%   \caption{New equational rules for $\extlcmm$ (rules for case-arrow are analogous
%%            and hence are omitted).}
%%   \label{fig:extlc-minus-minus-eqns}
%% \end{figure}



%% % TODO : Updated term precision rules



%% \subsection{The Step-Sensitive Lambda Calculus}\label{sec:step-sensitive-lc}

%% % \textbf{TODO: Subject to change!}

%% Rather than give a semantics to $\extlcmm$ directly, we first introduce another intermediary
%% language, a \emph{step-sensitive} (also called \emph{intensional}) calculus.
%% As mentioned, this language makes the intensional stepping behavior of programs
%% explicit in the syntax. We do this by adding a syntactic ``later'' type and a
%% syntactic $\theta$ that takes terms of type later $A$ to terms of type $A$.

%% % In the step-sensitive syntax, we add a type constructor for later, as well as a
%% % syntactic $\theta$ term and a syntactic $\nxt$ term.
%% We add rules for these new constructs, and also modify the rules for inj-arr and
%% case-arrow, since now the function is not $\Dyn \ra \Dyn$ but rather $\later (\Dyn \ra \Dyn)$.
%% We also add congruence relations for $\later$ and $\nxt$.

%% % TODO show changes

%% \noindent Modified syntax:
%% \begin{align*}
%%   &\text{Value Types } A := \nat \alt \dyn \alt (A \ra A') \alt {\color{red} \later A} \\
%%   &\text{Values } V :=  \zro \alt \suc\, V \alt \lda{x}{M} \alt \injnat V \alt \injarr V 
%%     \alt {\color{red} \nxt\, V} \alt {\color{red} \mathbf{\theta}}
%% \end{align*}

%% \noindent Additional typing rules:
%% \begin{mathpar}
%%   \inferrule
%%   {\hasty \Gamma V A}
%%   {\hasty \Gamma {\nxt\, V} {\later A}}

%%   \inferrule
%%   {}
%%   {\hasty \Gamma \theta {\later A \ra A}}

%%   % \theta(\nxt x) = \theta(y); \texttt{ret}\, x
%% \end{mathpar}

%% \noindent Modified typing rules:
%% \begin{mathpar}

%%   % inj-arr 
%%   \inferrule*
%%   {\hasty \Gamma M {\color{red} \later (\dyn \ra \dyn)}}
%%   {\hasty \Gamma {\injarr M} \dyn}

%%   % Case arr
%%   % TODO if the extensional version is incorrect and needs to change, make
%%   % sure to change this one accordingly
%%   \inferrule*
%%   {\hasty{\Delta|_V}{V}{\dyn} \and 
%%     \hasty{\Delta , x \colon {\color{red} \later (\dyn \ra \dyn)} }{M_{yes}}{B} \and 
%%     \hasty{\Delta}{M_{no}}{B}}
%%   {\hasty {\Delta} {\casearr{V}{M_{no}}{\tilde{f}}{M_{yes}}} {B}}  
%% \end{mathpar}

%% \noindent Additional relations:
%% \begin{mathpar}
%%   \inferrule*[]
%%   {A^{\rel} : A_l \rel A_r}
%%   {\later A^{\rel} : \later A_l \rel \later A_r}

%%   \inferrule*[]
%%   {A^{\rel} (V_l, V_r)}
%%   {\later A^{\rel} (\nxt\, V_l, \nxt\, V_r)}

%% \end{mathpar}

%% % TODO what about the relation for theta? Or is that automatic since it's a function symbol?

%% % TODO beta rule for theta

%% We define the term $\delta$ to be the function $\lda {x} {\theta\, (\nxt\, x)}$.

%% % We define an erasure function from step-sensitive syntax to step-insensitive syntax
%% % by induction on the step-sensitive types and terms.
%% % The basic idea is that the syntactic type $\later A$ erases to $A$,
%% % and $\nxt$ and $\theta$ erase to the identity.



%% \subsection{Quotienting by Syntactic Bisimilarity}

%% We now define a quotiented variant of the above step-sensitive calculus,
%% which we denote by $\intlcbisim$.
%% In this syntax, we add a rule saying, roughly speaking, that 
%% $\theta \circ \nxt$ is the identity. This causes terms that differ only in
%% their intensional behavior to become equal.
%% Note that a priori, this is not the same language as the step-insensitive
%% calculus on which we based the insensitive calculus.

%% Formally, the equational theory for the quotiented syntax is the same as
%% that of the original step-sensitive language, with the addition of the following
%% rule:

%% % TODO is this correct?
%% \begin{mathpar}
%%   \inferrule*
%%   { }
%%   { \theta\, (\nxt\, x) = \bind{y}{(\theta\, V')}{\ret\, x}  }
%% \end{mathpar}

%% This states that the application of $\theta$ to $\nxt\, x$ is equivalent to
%% the computation that applies $\theta$ to $V'$ to obtain a variable $y$, and
%% then simply returns $x$.




% \section{Idealized Double Categorical Models of Graduality}
\label{sec:cbpv}

In order to organize our construction of denotational models, we first
develop sufficient \emph{abstract} categorical semantics of gradually
typed languages. We start by modeling the type and term structure of
gradual typing and then extend this to type and term precision.
%
Gradually typed languages inherently involve computational effects of
errors and non-termination and typically in practice many others such
as mutable state and I/O.
%
To model this cleanly categorically, we follow New, Licata and Ahmed's
GTT calculus and base our models on Levy's Call-by-push-value
(CBPV) calculus which is a standard model of effectful programming
\cite{levy99}.
%
There are several notions of model of CBPV from the literature with
varying requirements of which connectives are present
\cite{levy99,cfm2016,eec}. We will use a variant which models precisely
the connectives we require and no more
($1,\times,F,U,\to$)\footnote{It is essential in this case that we do
not require a cartesian closed category of values as there is no way
to implement casts for an exponential in general.}.

\begin{enumerate}
\item A cartesian category $\mathcal V$ and a category $\mathcal E$.
\item An action of $\mathcal V^{op}$ (with the $\mathcal V$ cartesian
  product as monoidal structure) on $\mathcal E$. We write this with
  an arrow $A \arr B$.
  This means we have natural isomorphisms
  $\alpha : {A_1 \times A_2} \arr B \cong A_2 \arr (A_1 \arr B)$ and $i : 1 \arr B \cong B$ satisfying pentagon and triangle identities\cite{action}.
\item $F \dashv U$ where $U : \mathcal E \to \mathcal V$ such that $U$ ``preserves
  powering'' in that every $U(A \arr B)$ is an exponential of $UB$ by $A$
  and that $U\alpha$ and $Ui$ are mapped to the canonical isomorphisms
  for exponentials.
\end{enumerate}

\begin{example}
  Given a strong monad $T$ on a bicartesian closed category $\mathcal
  V$, we can extend this to a CBPV model by defining $\mathcal E$ to
  be the category $\mathcal V^T$ of algebras of the monad, defining $A
  \to B$ as the powering of algebras, $F$ as the free algebra functor and $U$
  as the underlying object functor.
\end{example}

To additionally model the error terms, we add a requirement that there
is a natural transformation $\mho : 1 \Rightarrow U$. The naturality
requirement encodes that strict morphisms (e.g., the denotations of
evaluation contexts) preserve errors.

We can then model CBV terms and types in a straightforward adaptation
of Levy's interpretation of CBV in CBPV. We interpret types $A$ and
contexts $\Gamma$ as objects $A,\Gamma \in \mathcal V$ and CBV terms
$\Gamma \vdash M : A$ as morphisms of any of the equivalent forms
$\mathcal E(F\Gamma, FA) \cong \mathcal V(\Gamma , UFA) \cong
\mathcal E(F(1), \Gamma \to F(A))$. The most interesting type
translation is the CBV function type: $A \ra A' = U(A \to F A')$.
%
Such a model validates all type-based equational reasoning, i.e.,
$\beta\eta$ equality, and models the introduction and elimination
rules for CBV.
%
Thus a CBPV model is sufficient to interpret the CBV term language. We
will require additional structure to interpret the precision and type
casts.

\subsection{Double Categorical Semantics of Graduality}

New and Licata \cite{new-licata18} modeled the graduality and type casts for call-by-name
gradual typing using \emph{double categories}, which are defined to be
categories internal to the category of categories. That is, a double
category $\mathcal C$ consists of a category $\mathcal C_o$ of
``objects and function morphisms'' and a category $\mathcal C_{sq}$ of
``relation morphisms and squares'' with functors (reflexive relation)
$r : C_o \to C_{sq}$ and (source and target) $s,t : C_{sq} \to C_o$
satisfying $sr = tr = \id$ as well as a composition operation $c :
C_{sq} \times_{s,t} C_{sq} \to C_{sq}$ respecting source and
target. This models an abstract notion of functions and relations. For
notation, we write function morphisms as $f : A \to B$ and relation
morphisms as $c : A \rel B$ where $c \in C_{sq}$ and $s(c) = A$ and
$t(c) = B$. Finally, a morphism $\alpha$ from $c$ to $d$ with
$s(\alpha) = f$ and $s(\beta) = g$ is visualized as
% https://q.uiver.app/#q=WzAsNCxbMCwwLCJBIl0sWzEsMCwiQiJdLFswLDEsIkEnIl0sWzEsMSwiQiciXSxbMCwyLCJmIiwyXSxbMSwzLCJnIl0sWzAsMSwiYyIsMCx7InN0eWxlIjp7ImJvZHkiOnsibmFtZSI6ImJhcnJlZCJ9LCJoZWFkIjp7Im5hbWUiOiJub25lIn19fV0sWzIsMywiZCIsMix7InN0eWxlIjp7ImJvZHkiOnsibmFtZSI6ImJhcnJlZCJ9LCJoZWFkIjp7Im5hbWUiOiJub25lIn19fV1d
\[\begin{tikzcd}[ampersand replacement=\&]
	A \& B \\
	{A'} \& {B'}
	\arrow["f"', from=1-1, to=2-1]
	\arrow["g", from=1-2, to=2-2]
	\arrow["c", "\shortmid"{marking}, no head, from=1-1, to=1-2]
	\arrow["d"', "\shortmid"{marking}, no head, from=2-1, to=2-2]
\end{tikzcd}\]
Such a morphism is thought of as an abstraction of the notion of relatedness of
functions: functions take related inputs to related outputs. The
composition operations and functoriality give us a notion of
composition of relations as well as functions and vertical and
horizontal composition of squares. In this work we will be chiefly
interested in \emph{locally thin} double categories, that is, double
categories where there is at most one square for any $f,c,g,d$. In
this case we use the notation $f \leq_{c,d} g$ to mean that a square
like the above exists.

New, Licata and Ahmed \cite{new-licata-ahmed2019} extended the axiomatic
syntax to call-by-push-value but did not analyze the structure
categorically.
We fill in this missing analysis now, extending the double categorical semantics
given in \cite{new-licata18} from cartesian closed categories to CBPV models.
A model of the congruence rules of the CBPV system can be given by a locally thin
``double CBPV model'', which we define as a category internal to the
category of CBPV models and \emph{strict} homomorphisms of CBPV
models\footnote{it may be possible to also define this as a notion of
CBPV model internal to some structured $2$-category of categories, but
the authors are not aware of any such definition of an internal CBPV
model}. A strict homomorphism of CBPV models from $(\mathcal
V,\mathcal E,\ldots)$ to $(\mathcal V', \mathcal E',\ldots)$ consists
of functors $G_v : \mathcal V \to \mathcal V'$ and $G_e : \mathcal E
\to \mathcal E'$ that strictly preserve all CBPV constructions, see
the appendix for a more detailed definition. We call this a strict morphism
in contrast to a \emph{lax} morphism, which only preserves CBPV constructions
up to transformation.
Some of the data of a double CBPV model can be visualized as follows:
% https://q.uiver.app/#q=WzAsNCxbMCwwLCJcXHZzcSJdLFsyLDAsIlxcZXNxIl0sWzAsMiwiXFx2ZiJdLFsyLDIsIlxcZWYiXSxbMiwzLCJcXEZmIiwwLHsiY3VydmUiOi0yfV0sWzMsMiwiXFxVZiIsMCx7ImN1cnZlIjotMn1dLFswLDEsIlxcRnNxIiwwLHsiY3VydmUiOi0yfV0sWzEsMCwiXFxVc3EiLDAseyJjdXJ2ZSI6LTJ9XSxbMiwwLCJcXHJ2Il0sWzAsMiwiXFxzdiIsMCx7ImN1cnZlIjotMn1dLFswLDIsIlxcdHYiLDIseyJjdXJ2ZSI6Mn1dLFsxLDMsIlxcc2UiLDAseyJjdXJ2ZSI6LTJ9XSxbMSwzLCJcXHRlIiwyLHsiY3VydmUiOjJ9XSxbMywxLCJcXHJlIl0sWzQsNSwiXFxib3QiLDEseyJzaG9ydGVuIjp7InNvdXJjZSI6MjAsInRhcmdldCI6MjB9LCJzdHlsZSI6eyJib2R5Ijp7Im5hbWUiOiJub25lIn0sImhlYWQiOnsibmFtZSI6Im5vbmUifX19XSxbNiw3LCJcXGJvdCIsMSx7InNob3J0ZW4iOnsic291cmNlIjoyMCwidGFyZ2V0IjoyMH0sInN0eWxlIjp7ImJvZHkiOnsibmFtZSI6Im5vbmUifSwiaGVhZCI6eyJuYW1lIjoibm9uZSJ9fX1dXQ==
\[\begin{tikzcd}[ampersand replacement=\&]
	\vsq \&\& \esq \\
	\\
	\vf \&\& \ef
	\arrow[""{name=0, anchor=center, inner sep=0}, "\Ff", curve={height=-12pt}, from=3-1, to=3-3]
	\arrow[""{name=1, anchor=center, inner sep=0}, "\Uf", curve={height=-12pt}, from=3-3, to=3-1]
	\arrow[""{name=2, anchor=center, inner sep=0}, "\Fsq", curve={height=-12pt}, from=1-1, to=1-3]
	\arrow[""{name=3, anchor=center, inner sep=0}, "\Usq", curve={height=-12pt}, from=1-3, to=1-1]
	\arrow["\rv", from=3-1, to=1-1]
	\arrow["\sv", curve={height=-12pt}, from=1-1, to=3-1]
	\arrow["\tv"', curve={height=12pt}, from=1-1, to=3-1]
	\arrow["\se", curve={height=-12pt}, from=1-3, to=3-3]
	\arrow["\te"', curve={height=12pt}, from=1-3, to=3-3]
	\arrow["\re", from=3-3, to=1-3]
	\arrow["\bot"{description}, draw=none, from=0, to=1]
	\arrow["\bot"{description}, draw=none, from=2, to=3]
\end{tikzcd}\]
Type precision $A \ltdyn A'$ is interpreted as a relation morphism
$c_A : A \rel A'$ in $\mathcal V_{sq}$, and term precision $\Gamma
\ltdyn \Gamma' \vdash M \ltdyn M' : A \ltdyn A'$ is interpreted as a
square $M \ltdyn_{c_\Gamma,UF c_A} M'$. The fact that $t,r$ and the
composition are all given by strict CBPV homomorphisms says that all
the type constructors lift to precision (monotonicity of type
constructors) as well as all term constructors (congruence). Further,
$r$ and composition being strict homomorphisms implies that all type
constructors strictly preserve the identity relation (identity
extension) and composition.

Next, to model type casts, their model further requires that every
value relation $c : A \rel A'$ is \emph{left representable} by a
function $u_c : A \to A'$ and every computation relation $d : B \rel
B'$ is \emph{right representable} by a function $d_c : B' \to B$. In a locally thin double category, these are defined as follows:
\begin{definition}
  $c : A \rel B$ is left representable by $f : A \to B$ if $f \ltsq{c}{r(B)} \id$ and $\id \ltsq{r(A)}{c} f $.

  Dually, $c : A \rel B$ is right representable by $g : B \to A$ if
  $\id \ltsq{c}{r(A)} g$ and $g \ltsq{r(B)}{c} \id$.
\end{definition}
These rules are sufficient to model the UpL/UpR/DnL/DnR rules for
casts. Additionally, since representable morphisms compose and so the
compositionality of casts comes for free. However, the retraction
property must be added as an additional axiom to the model.  To model
the error being a least element we add the requirement that $\mho
\circ {!} \ltsq{r(A)}{r(UB)} f$ holds for all $f : \mathcal V(A,B)$.
Finally, the dynamic type can be modeled as an arbitrary value type $D$ with
arbitrary relations $\nat \rel D$ and $D \ra D \rel D$ and $D \times D
\rel D$ (or whatever basic type cases are required).

\begin{example}
  (Adapted from \cite{new-licata18}): Define a double CBPV model where
  $\mathcal V$ is the category of predomain preorders: sets with an
  $\omega$-CPO structure $\leq$ as well as a poset structure
  $\ltdyn$. Functional morphisms are given by $\leq$-continuous and
  $\ltdyn$-monotone functions. Then define $\mathcal E$ to be the
  category of pointed domain preorders which are domain preorders with
  least elements $\bot$ for $\leq$ and $\mho$ for $\ltdyn$ such that
  $\bot$ is $\ltdyn$-maximal, and morphisms are as before but preserve
  $\bot$ and $\mho$. This can be extended to a CBPV model with
  forgetful functor $U : \mathcal E \to \mathcal V$. $D$ can be
  defined by solving a domain equation.

  This can be extended to a double CBPV model by defining a value
  relation $A \rel A'$ to be a $\ltdyn$-\emph{embedding}: a morphism
  $e : A \to A'$ that is injective and such that $F e : F A \to F A'$
  has a right adjoint (with respect to $\ltdyn$) and a square $f
  \ltsq{e}{e'} f' = f \circ e \ltdyn e' \circ f$. Similarly
  computation relations $B \rel B'$ are defined to be
  \emph{projections}: morphisms $p : B' \to B$ that are surjective and
  $Up$ has a left adjoint, with squares defined similarly. A suitable
  dynamic type can be constructed by solving a domain equation $D
  \cong \nat + U(D \to FD) + (D \times D)$.
\end{example}




\section{A Denotational Semantics for Types and Terms}\label{sec:concrete-term-model}

As a first step to giving a well-behaved semantics of GTLC, in this
section we define a simpler semantics that validates all of our
desired properties except for graduality: that is, it is adequate and
validates $\beta\eta$ equality. Our relational semantics will then be
a refinement of this simpler design. In the process we will introduce
some of the technical tools we will use in the refined semantics:
guarded type theory and call-by-push-value. While these tools can be
somewhat technical we emphasize that due to the use of synthetic
guarded domain theory, this initial semantics is essentially just a
``na\"\i ve'' denotational semantics of an effectful language arising
from a monad on sets. But since we are working in the non-standard
foundation of guarded type theory our notion of sets allows for the
construction of more sophisticated sets and monads than classical
logic allows.

% In Section \ref{sec:towards-relational-model}, we will discuss how to extend the
% denotational semantics to accommodate the type and term precision orderings. 
%
%necessitate the abstract components that are found in the definition of the model.

\subsection{Guarded Type Theory}\label{sec:guarded-type-theory}

% \max{TODO: discuss this wording with Eric. Are we working in guarded
%   type theory, SGDT or Ticked Cubical Type Theory in the informal work
%   in the paper?}
We begin with a brief overview of the background theory that we use in this work.
\emph{Synthetic guarded domain theory}, or SGDT for
short, is an axiomatic framework in which we can reason about non-well-founded
recursive constructions while abstracting away the specific details of
step-indexing that we would need to track if we were working analytically. This
allows us to avoid the tedious reasoning associated with traditional
step-indexing techniques. We provide a brief overview here; more details can be
found in \cite{birkedal-mogelberg-schwinghammer-stovring2011}.

SGDT offers a synthetic approach to domain theory that allows for guarded
recursion to be expressed syntactically via a type constructor $\later \colon
\type \to \type$ (pronounced ``later''). The use of a modality to express
guarded recursion was introduced by Nakano \cite{Nakano2000}.
%
Given a type $A$, the type $\later A$ represents an element of type $A$ that is
available one time step later. There is an operator $\nxt : A \to\, \later A$
that ``delays'' an element available now to make it available later. We will use
a tilde to denote a term of type $\later A$, e.g., $\tilde{M}$.

% TODO later is an applicative functor, but not a monad

There is a \emph{guarded fixpoint} operator
\[ \fix : \forall T, (\later T \to T) \to T. \]
That is, to construct a term of type $T$, it suffices to assume that we have
access to such a term ``later'' and use that to help us build a term ``now''.
This operator satisfies the axiom that $\fix f = f (\nxt (\fix f))$.
%
In particular, $\fix$ applies when type $T$ is instantiated to a proposition 
$P : \texttt{Prop}$; in that case, it corresponds to $\lob$-induction.

% Clocked Cubical Type Theory
\subsubsection{Ticked Cubical Type Theory}
% TODO motivation for Clocked Cubical Type Theory, e.g., delayed substitutions?

Ticked Cubical Type Theory \cite{mogelberg-veltri2019} is an extension of
Cubical Type Theory \cite{CohenCoquandHuberMortberg2017} that has an additional
sort called \emph{ticks}. Ticks were originally introduced in
\cite{bahr-grathwohl-bugge-mogelberg2017}. A tick $t : \tick$ serves as evidence
that one unit of time has passed. In Ticked Cubical Type Theory, the type
$\later A$ is the type of dependent functions from ticks to $A$. The type $A$ is
allowed to depend on $t$, in which case we write $\later_t A$ to emphasize the
dependence.

% TODO next as a function that ignores its input tick argument?

% TODO include a figure with some of the rules for ticks

The rules for tick abstraction and application are similar to those of ordinary
$\Pi$ types. A context now consists of ordinary variables $x : A$ as well as
tick variables $t : \tick$. The presence of the tick variable $t$ in context
$\Gamma, (t : \tick), \Gamma'$ intuitively means that the values of the
variables in $\Gamma$ arrive ``first'', then one time step occurs, and then the
values of the variables in $\Gamma'$ arrive.
%
The abstraction rule for ticks states that if in context $\Gamma, t : \tick$
the term $M$ has type $A$, then in context $\Gamma$ the term $\lambda t.M$ has
type $\later_t A$.
%
Conversely, if we have a term $M$ of type $\later A$, and we have available in
the context a tick $t' : \tick$, then we can apply the tick to $M$ to get a
term $M[t'] : A[t'/t]$. However, there is an important restriction on when we
are allowed to apply ticks: To apply $M$ to tick $t$, we require $M$ to be
well-typed in the prefix of the context occurring before the tick $t$. That is,
all variables mentioned in $M$ must be available before $t$. This ensures that
we cannot, for example, define a term of type $\later \laterhs A \to\, \laterhs
A$ via repeated tick application.
% TODO restriction on tick application
%
For the sake of brevity, we will also write tick application as $M_t$.
The constructions we carry out in this paper will take place in a guarded type
theory with ticks.
\subsection{A Call-by-push-value Model}

% TODO there aren't actually computation types in the syntax. They
% arise because we are embedding call-by-value into CBPV.

Our denotational model of GTLC must account for the fact that the
language is call-by-value with non-trivial effects (divergence and
errors).  To do this, we formulate the model using the categorical
structures of Levy's \emph{call-by-push-value}
\cite{levy99}. Call-by-push-value is a categorical model of effects
that is a refinement of Moggi's monadic semantics
\cite{MOGGI199155}. While Moggi works with a (strong) monad $T$ on a
category $\mathcal V$ of pure morphisms, call-by-push-value works
instead with a decomposition of such a monad into an adjunction
between a category $\mathcal V$ of ``value types'' and pure morphisms
and a category $\mathcal E$ of ``computation types'' and homomorphisms.
The monad $T$ on $\calV$ is decomposed into a left adjoint
$F : \calV \to \calE$ constructing the type $F A$ of computations that
return $A$ values and right adjoint $U : \calE \to \calV$
constructing the type $U B$ of values that are first-class suspended
computations of type $B$. The monad $T$ can then be reconstituted as
$T = UF$, but the main advantage is that many constructions on
effectful programs naturally decompose into constructions on
value/computation types. Most strikingly, the CBV function type $A
\rightharpoonup A'$ decomposes into the composition of three
constructions $U(A \to F A')$ where $\arr : \calV^{op} \times \calE
\to \calE$ is a kind of mixed kinded function type. This decomposition
considerably simplifies treatment of the CBV function
type. Additionally, prior work on gradual typing semantics has shown
that the casts of gradual typing are naturally formulated in terms of
pairs of a pure morphism and a homomorphism, as we explain below
\cite{new-licata-ahmed2019}.

For our simple denotational semantics, the category $\calV$ is simply
the category $\Set$ of sets and functions. Then because GTLC is
call-by-value, all types will be interpreted as objects of this
category, i.e., sets. More interesting are the computation
types. These will be interpreted in a category of sets equipped with
algebraic\footnote{The $\theta$ structure is not algebraic in the
strictest sense since it does not have finite arity.} structure with
morphisms being functions that are homomorphic in this structure.
This algebraic structure needs to model the effects present in our
language, in this case these are errors as well as taking
computational steps (potentially diverging by taking computational
steps forever).
%
We call these algebraic structures simple error domains to distinguish
them from the error domains we define later to model graduality:
\begin{definition}[Simple Error Domains]
  A (simple) error domain $B$ consists of
  \begin{enumerate}
  \item A carrier set $UB$
  \item An element $\mho_{B} : UB$ representing error
  \item A function $\theta_B : \laterhs UB \to UB$ modeling a computational step
  \end{enumerate}
  A homomorphism of error domains $\phi : B \multimap B'$ is a
  function $\phi : UB \to UB'$ that preserves $\mho$ and $\theta$, i.e.,
  $\phi(\mho_B) = \mho_{B'}$ and $\phi(\theta_{B}(\tilde{x})) =
  \theta_{B'}(\later (\phi(\tilde{x})))$.  Simple error domains and
  homomorphisms assemble into a category $\ErrDom$ with a forgetful
  functor $U : \ErrDom \to \Set$ taking the underlying set and
  function.
\end{definition}
Here is our first point where we utilize guarded type theory: rather
than simply being a function $UB \to UB$, the ``think'' map $\theta$
takes an element \emph{later}. This makes a major difference, because
the structure of a think map combined with the guarded fixpoint
operator allows us to define recursive elements of $UB$ in that any
function $f : UB \to UB$ has a ``quasi-fixed point'' $\textrm{qfix}(f)
= \fix(f \circ \theta_B)$ satisfying the quasi-fixed point property:
\[ \qfix(f) = f(\delta_B(\qfix f)) \]
where $\delta_B = \theta_B \circ \nxt$ is a map we call the ``delay''
map which trivially delays an element now to be available later.  As
an example, we can define $\Omega_B = \qfix(\id) : UB$, the
``diverging element'' that ``thinks forever'' in that $\Omega_B =
\delta_B(\Omega_B)$. We call this a ``quasi'' fixed point because it is
\emph{nearly} a fixed point except for the presence of the delay map
$\delta_B$, which is irrelevant from an extensional point of view
where we would prefer to ignore differences in the number of steps
that computations take.

% \max{move this}
% The functor $\arr : \Set^{op} \times \errdom \to \errdom$ is defined on objects
% as follows. Given a set $A$ and error domain $B$, $A \arr B$ is the error domain
% whose underlying set is the set of functions $A \to UB$. The error element is
% the constant error function, i.e., $\lambda x. \mho_B$. The map $\theta :
% (\laterhs (A \to UB)) \to (A \to UB)$ is defined by 
% $\theta(\tilde{f}) = \lambda x . \theta_B (\tilde{f}_t)$.
% %
% Given a morphism $f : A_o \to A_i$ and $\phi : B_i \to B_o$, the morphism $f
% \arr \phi : (A_i \arr B_i) \to (A_o \arr B_o)$ is given by pre-composing by $f$
% and post-composing by $\phi$.

Then to model effectful programs, we need to construct a left adjoint
$\li : \Set \to \ErrDom$ to $U$ which constructs the ``free error
domain'' on a set, which models an effectful CBV computation.
%% \subsection{The Lift + Error Monad}\label{sec:lift-monad}
We base the construction on the \emph{guarded
lift monad} described in \cite{mogelberg-paviotti2016}. Here, we augment the
guarded lift monad to accommodate the additional effect of failure.
\begin{definition}[Free error domain]
  For a set $A$, we define the (carrier of) \emph{free error domain} $U(\li A)$ as the unique solution to the guarded domain equation:
  \[ U(\li A) \cong A + 1\, + \laterhs U(\li A). \]
  For which we use the following notation for the three constructors:
  \begin{enumerate}
  \item $\eta \colon A \to U(\li A)$
  \item $\mho \colon U(\li A)$
  \item $\theta \colon \laterhs U(\li A) \to U(\li A)$
  \end{enumerate}
  Here $\mho, \theta$ provide the error domain structure for $\li A$,
  and this has the universal property of being free in that any
  function $f : U(\li A) \to UB$ uniquely extends to a homomorphism
  $f^\dagger : \li A \multimap B$ satisfying $Uf^\dagger \circ \eta =
  f$. In other words, the free error monad is left adjoint $\li \dashv
  U$ to the forgetful functor, with $\eta$ the unit of the adjunction.
  %
  We will write $\theta_t(\dots)$ to mean $\theta (\lambda t. \dots)$.
\end{definition}
%% %
%% Unless otherwise mentioned, all constructs involving $\later$ or $\fix$ are
%% understood to be with respect to a fixed clock $k$. So for the above, we really
%% have for each clock $k$ a type $\li^k A$ with respect to that clock.
%% %
%% Formally, the lift monad $\li A$ is defined as the solution to the guarded
%% recursive type equation
%% %
%
An element of $\li A$ should be viewed as a computation that can either (1)
return a value (via $\eta$), (2) raise an error and stop (via $\mho$), or (3)
take a computational step (via $\theta$).
%
The operation $f^\dagger$ can be defined by guarded recursion in the
apparent way, and proven to satisfy the freeness property by L\"ob
induction.
% It is instructive to give at least one example of a use of guarded recursion, so
% we show below how to define extend:
% TODO
%
%
%% Verifying that the monadic laws hold uses \lob-induction and is straightforward.

%% % TODO mention that error domains are algebras of the lift monad?
%% We observe that the guarded lift monad applied to a set $A$ is the underlying
%% set of the free error domain on $A$ $A$, i.e., we have $\li = UF$ where $F :
%% \Set \to \errdom$ and $U : \errdom \to \Set$ and $F \dashv U$. Given a set $A$
%% and an error domain $B$, the set of functions $A \to UB$ is naturally isomorphic
%% to the set of error domain morphisms $FA \to B$.

%% Lastly, we define the function $\delta : \li A \to \li A$ by $\delta = \theta_A
%% \circ \nxt$.

% The function type A \ra A' will be modeled as \sem{A} \to \li \sem{A'}

The free error domain is essential to modeling CBV computation: a term
$\Gamma \vdash M : A$ in GTLC will be modeled as a function $M :
\sem{\Gamma} \to U\li \sem{A}$ from the set denoted by $\Gamma$ to the
underlying set of the free error domain on the set denoted by $A$,
just as in a monadic semantics with monad $U\li$.

\subsection{Modeling the type structure}\label{sec:dynamic-type}
Much of the type structure of GTLC is simple to model: $\nat$ denotes
the set of natural numbers and $\times$ denotes the Cartesian product
of sets. To model the CBV function type $A \rightharpoonup A'$ we use
the CBPV decomposition $U(\sem{A} \to \li \sem{A'})$ where $A \to B$
is the obvious point-wise algebraic structure on the set of functions
from $A$ to $UB$.

The final type to model is the the dynamic type $\dyn$.  In the style
of Scott, a classical domain theoretic model for $\dyn$ would be given
by solving a domain equation
%
\[ D \cong \Nat + (D \times D) + U(D \to \li D), \]
%
representing the fact that dynamically typed values can be numbers,
pairs or CBV functions. This equation does not have inductive or
coinductive solutions due to the negative occurrence of $D$ in the
domain of the function type. In classical domain theory this is
avoided by instead taking the initial algebra in a category of
embedding-projection pairs, by which we obtain an exact solution to
this equation. However, in guarded domain theory we instead replace this domain equation with a guarded domain equation.
Consider the following similar-looking equation:
%
\begin{equation}\label{eq:dyn}
D \cong \Nat + (D \times D)\, + \laterhs U(D \to \li D).
\end{equation}
%
Since the negative occurrence of $D$ is guarded under a later, we can
solve this equation by a mixture of inductive types and guarded fixed
points. Specifically, consider the parameterized inductive type
%
\[ D'\, X := \mu T. \Nat + (T \times T) + X. \]
%
Then we can construct $D$ as the unique solution to the guarded domain equation
\[ D \cong D'(\laterhs U(D \to \li D)) \]
which is a guarded equation in that all occurrences of $D$ on the right hand side are guarded by the $\laterhs$.
%
Expanding out the guarded fixed point and least fixed point property
gives us that $D$ satisfies Equation $\ref{eq:dyn}$. 
% \max{give that equation a number}.
%
We then refer to the three injections for numbers, pairs and functions
as $\inat, \itimes, \iarr$ respectively.

\subsection{Term Semantics}\label{sec:term-interpretation}

We now extend our semantics of types to a semantics of terms of
GTLC. As mentioned above, terms $x:A_1,\ldots \vdash M : A$ are
modeled as effectful functions $\sem{M}: \sem{A_1}\times \cdots \to U\li\sem{A}$
and values $x:A_1,\ldots \vdash V : A$ are also modeled as pure
functions $\sem{V} : \sem{A_1}\times\cdots \to \sem{A}$.
%
The interpretation of the typical CBV features of GTLC into a model of CBPV is
standard \cite{levy99}; the only additional part we need to account for
is the semantics of casts, shown in Figure \ref{fig:term-semantics}.
%
%% Much of the semantics is similar to a normal call-by-value denotational
%% semantics, so we focus only on the cast semantics. We interpret types as sets;
%% the interpretation of types is given in Figure \ref{fig:type-interpretation}.
%% Contexts $\Gamma = x_1 \colon A_1, \dots, x_n \colon A_n$ are interpreted as the
%% product $\sem{A_1} \times \cdots \times \sem{A_n}$. The semantics of the dynamic
%% type $\dyn$ is the set $D$ introduced in Section \ref{sec:dynamic-type}. The
%% product type $A \times A'$ is interpreted as the Cartesian product of the
%% denotations of $A$ and $A'$. The function type $A \ra A'$ is interpreted as the
%% set of functions from $\sem{A}$ to $\li (\sem {A'})$.
%
%% The interpretation of a value $\hasty {\Gamma} V A$ is a function from
%% $\sem{\Gamma}$ to $\sem{A}$. Likewise, a term $\hasty {\Gamma} M {{A}}$ is
%% interpreted as a function from $\sem{\Gamma}$ to $\li \sem{A}$.
Following the CBPV treatment of gradual typing given by
New-Licata-Ahmed, an upcast $\upc c$ where $c : A \ltdyn A'$ is
interpreted as a pure function $\sem{\upc c} :\sem{A} \to \sem{A'}$ whereas a
downcast $\dnc c$ is interpreted as a homomorphism $\sem{\dnc c} :
\li\sem{A'} \multimap \li\sem{A}$.
%% By the universal property of $\li$, this is determined by a function $\sem{A'} \to U\li\sem{A}$ and
%% We use $\text{ext}$ to go from the former to the latter in the definitions of the downcasts.
We define the semantics of upcasts and downcasts simultaneously by
recursion over type precision derivations.
%
The reflexivity casts are given by identities, and the transitivity
casts by composition. The upcasts for products, and the downcasts for
functions are given by the functorial actions of the type constructors
on the casts for the subformulae.
%
The \emph{up}casts for functions and similarly the \emph{down}casts
for products, on the other hand, are more complex.
%
For these, we use not the ordinary functorial action of type
constructors, but an action we call the \emph{Kleisli} action.

The Kleisli action of type constructors can be defined in any CBPV
model. First, we can define the Kleisli category of value types
$\calV_k$ to have value types as objects, but have morphisms
$\calV_k(A,A') = \calE(FA,FA')$. Similarly, we can define a Kleisli
category of computation types $\calE_k$ to have computation types as
objects, but have morphisms $\calE_k(B,B') = \calV(UB,UB')$. The
intuition is that these are the ``effectful'' morphisms between
value/computation types, whereas the morphisms of $\calV$ and $\calE$ are
pure morphisms and homomorphisms, respectively. Then every CBPV type constructor $F,U,\times,\to$
has in addition to its usual functorial action on pure morphisms/homomorphisms
an action in each argument on Kleisli morphisms. So for
instance, the function construction is functorial in pure morphisms/homomorphisms
$\to : \op{\calV} \times \calE \to \calE$ but on Kleisli
morphisms what we have is for each computation type $B$ a functorial
action $-\to B : \calV_k^{op} \to \calE_k$ given by $-\tok B$ on
objects and for each value type a functorial action $A \tok - :
\calE_k \to \calE_k$. Similarly we get two actions $A \timesk -$ and
$-\timesk A'$ on Kleisli morphisms between value types. In these two
cases we do not get a bifunctorial action because the two Kleisli
actions do not commute past each other, i.e., in general $(A_l'
\timesk f_r \circ f_l \timesk A_r) \neq (f_l \timesk A_r' \circ A_l
\timesk f_r)$.
%
% EXTENDED VERSION 
% The full definitions of the Kleisli actions are included in Appendix
% \ref{sec:kleisli-actions}.
%
% CONFERENCE VERSION
%
The full definitions of the Kleisli actions are included in the appendix of the
extended version of this paper \cite{gtt-sgdt-extended-version}.

Lastly, we have the upcasts and downcasts for the injections into the dynamic type, which
are the core primitive casts. The upcasts are simply the injections
themselves, except the function case which must include a $\nxt$ to
account for the fact that the functions are under a later in the
dynamic type. The downcasts are similar in that on values
they pattern match on the input and return it if it is of the correct
type, otherwise erroring. Again, the function case is slightly
different in that if the input is in the function case, then it is
actually only a function available later, and so we must insert a
``think'' in order to return it.

% The upcast for $c_i \ra c_o$ and the downcast for $c_1 \times c_2$ make use of
% \emph{Kleisli functors} $\tok$ and $\timesk$.

%% \eric{Introduce Kleisli categories and functors}

%% Recall the definition of type precision derivations given in Figure
%% \ref{fig:typrec}. The semantics of the up- and downcasts for a type precision
%% derivation $c$ are defined by induction on $c$.
%% %
%% For $r(A)$ the up- and downcasts are simply the identity function. 
%% %
%% For the composition $c \comp c'$ we compose the cast for $c$ and the cast for
%% $c'$ as functions.
%% %
%% For $c_i \ra c_o$ we apply the casts for $c_i$ and $c_o$ in the domain and
%% codomain respectively. More concretely, for the upcast we are given a function
%% $V_f : A_i \to \li A_o$ and we must define a function $A_i' \to \li A_o'$. The
%% function first downcasts its argument according to $c_i$, resulting in an
%% element of $\li A_i$. It then applies $\ext{V_f}{}$ to this value to obtain an
%% element of $\li A_o$. Finally, it applies the upcast of $c_o$ via the functorial
%% action of the lift monad, obtaining an element of $\li A_o'$ as needed. For the
%% downcast from $A_i' \to \li A_o'$ to $A_i \to \li A_o$, the resulting function
%% first applies the upcast by $c_i$ to its argument, then applies the original
%% function, and finally applies the downcast by $c_o$ to the result.
%% %
%% Likewise, in the upcast for $c_1 \times c_2$, we apply the cast for $c_1$ on the
%% left and the cast for $c_2$ on the right.

%% The ``base cases'' for the casts are $\inat$, $\itimes$, and $\iarr$. Recall
%% that $D$ is isomorphic to $\Nat\, + (D \times D)\, + \laterhs (D \to \li D)$
%% (say that the sum is left-associative).
%% %
%% For $\inat$, the upcast is simply $\inl \circ \inl$. The downcast has type $D
%% \to \li D$ and performs a case inspection on the sum type. If it is a natural
%% number $n$, then we return $\eta\, n$. Otherwise, we return $\mho$, modeling
%% the fact that a run-time error occurs.
%% %
%% For $\itimes$, the upcast is $\inl \circ \inr$, and the downcast performs a case
%% inspection analogously to the natural number downcast.
%% %
%% % TODO explain this further
%% For $\iarr$, the upcast is given by $\inr \circ \nxt$. The downcast performs a
%% case inspection, and in the $\inr$ case, it uses a $\theta$ in order to gain
%% access to the function under the $\later$.


%
% TODO write this out explicitly?

% TODO should we include function casts?
%% \begin{figure*}
%%   \begin{align*}
%%     \sem{\nat} &= \Nat \\
%%     \sem{\dyn} &= D \\
%%     \sem{A \times A'} &= \sem{A} \times \sem{A'} \\ 
%%     \sem{A \ra A'} &= \sem{A} \To \li \sem{A'} \\
%%   \end{align*}
%%   \caption{Interpretation of types}
%%   \label{fig:type-interpretation}
%% \end{figure*}
\begin{figure*}
  \begin{minipage}{0.3\textwidth}
    \begin{footnotesize}
  \begin{align*}
    % upcasts
    \sem{\upc{r(A)}} &= \id_{\sem{A}} \\
    \sem{\upc{(c \comp c')}} &= \sem{\upc{c'}} \circ \sem{\upc{c}} \\
    \sem{\upc{(c_i \ra c_o)}} &= (\sem{\dnc {c_i}} \tok \li\sem{A_o'})\circ (\sem{A_i} \tok U\li(\sem{\upc {c_o}}))\\
    \sem{\upc{(c_1 \times c_2)}} &= \sem{\upc{c_1}} \times \sem{\upc{c_2}}\\
    \sem{\upc{\inat}} &= \inat\\
    \sem{\upc{\itimes}} &= \itimes\\
    \sem{\upc{\iarr}} &= \iarr \circ \nxt\\
    % \sem{\zro}         &= \lambda \gamma . 0 \\
    % \sem{\suc\, V}     &= \lambda \gamma . (\sem{V}\, \gamma) + 1 \\
    % \sem{x \in \Gamma} &= \lambda \gamma . \gamma(x) \\
    % \sem{\lda{x}{M}}   &= \lambda \gamma . \lambda a . \sem{M}\, (*,\, (\gamma , a))  \\
    % \sem{V_f\, V_x}    &= \lambda \gamma . {({(\sem{V_f}\, \gamma)} \, {(\sem{V_x}\, \gamma)})} \\
    % \sem{\err_B}       &= \lambda \gamma . \mho \\
    %
    % downcasts
    \sem{\dnc{r(A)}} &= \id_{\li \sem{A}} \\
    \sem{\dnc{(c \comp c')}} &= \sem{\dnc{c}} \circ \sem{\dnc{c'}} \\
    \sem{\dnc{(c_i \ra c_o)}} &= U(\sem{\upc {c_i}} \to \sem{\dnc {c_o}})\\
    \sem{\dnc{(c_1 \times c_2)}} &= (\sem{\dnc{c_1}} \timesk \sem{A_2}) \circ (\sem{A_1'} \timesk \sem{\dnc{c_2}}) \\
    \sem{\dnc{\inat}} &= (
      \lambda V_d . \text{case $({V_d})$ of }
      \{ \inat\,n \to \eta\, n
         \alt \text{otherwise} \to \mho \})^\dagger \\
    \sem{\dnc{\itimes}} &= (
      \lambda V_d . \text{case $({V_d})$ of }
      \{ \itimes(d_1, d_2) \to \eta\, (d_1, d_2)
         \alt \text{otherwise} \to \mho \})^\dagger \\
    \sem{\dnc{\iarr}} &= (
      \lambda V_d . \text{case $({V_d})$ of }
      \{ \iarr \tilde{f} \to \theta_t\, (\eta (\tilde{f}_t))
         \alt \text{otherwise} \to \mho \})^\dagger \\
    % \sem{\ret\, V}       &= \lambda \gamma . \eta\, \sem{V} \\
    % \sem{\bind{x}{M}{N}} &= \lambda \delta . \ext {(\lambda x . \sem{N}\, (\delta, x))} {\sem{M}\, \delta} \\
  \end{align*}
    \end{footnotesize}
  \end{minipage}
  \caption{Semantics of casts}
  \label{fig:term-semantics}
\end{figure*}

\subsection{Extracting a well-behaved big-step semantics}\label{sec:big-step-term-semantics}

% \max{TODO: Eric needs to add a discussion of adequacy/clock stuff here}
% \eric{Check this}

We now define from the above term model a ``big-step'' semantics. More
concretely, our goal is to define a partial function 
$-\Downarrow : \{M \,|\, \cdot \vdash M : \nat \} \rightharpoonup \mathbb{N} + {\mho}$.

To begin, we must first discuss another aspect of guarded type theory. The
constructions of guarded type theory ($\later$, $\nxt$, and $\fix$) we have been
using may be indexed by objects called \emph{clocks} \cite{atkey-mcbride2013}. A clock intuitively serves
as a reference relative to which steps are counted. For instance, given a clock
$k$ and type $X$, the type $\later^k X$ represents a value of type $X$ one unit
of time in the future according to clock $k$.
%
Up to this point, all uses of guarded constructs have been indexed by a single
clock $k$. In particular, we have for each clock $k$ a type $\li^k X$. The
notion of \emph{clock quantification} allows us to pass from the guarded setting
to the usual set-theoretic world. Clock quantification enables us to encode
coinductive types using guarded recursion. That is, given a type $A$ with a free
variable $k : \Clock$, we can form the type $A^{gl} := \forall k. A$ (here, $gl$ stands for ``global'').

A functor $F \colon \mathsf{Type} \to \mathsf{Type}$ is said to \emph{commute
with clock quantification} if for all $X : \Clock \to \mathsf{Type}$, the
canonical map $F(\forall k. X_k) \to \forall k. F(X_k)$ is an equivalence \cite{kristensen-mogelberg-vezzosi2022},
where we write $X_k$ for the application of clock $k$ to $X$.
%
Given a functor $F$ that commutes with clock quantification, if $A$ is a guarded
recursive type satisfying $A \cong F(\later^k A)$, then the type $A^{gl}$ has a
final $F$-coalgebra structure (see Theorem 4.3 of Kristensen et al. \cite{kristensen-mogelberg-vezzosi2022}). 

Consider the functor $F(X) = \mathbb{N} + {\mho} + X$. By the definition of the
guarded lift, we have that
% 
$\li^k \mathbb{N} 
  \cong \mathbb{N} + {\mho} + \later^k (\li \mathbb{N}) 
  \cong F(\later \li^k \mathbb{N})$.
%
Furthermore, it can be shown that the functor $F$ commutes with clock
quantification. Thus, the type $\li^{gl} \mathbb{N} := \forall k. \li^k
\mathbb{N}$ is a final $F$-coalgebra. In fact, $\li^{gl}$ is isomorphic to a
more familiar coinductive type.
%
Given a type $Y$, Capretta's \emph{delay monad} \cite{lmcs:2265},
which we write as $\text{Delay}(Y)$, is defined as the coinductive type
generated by $\tnow : Y \to \delay(Y)$ and $\tlater : \delay(Y) \to \delay(Y)$.
We have that $\delay (\mathbb{N} + {\mho})$ is a final coalgebra for the functor
$F$ defined above. This means $\li^{gl} \mathbb{N}$ is isomorphic to
$\delay(\mathbb{N} + {\mho})$. We write $\alpha$ to denote this isomorphism.

Given $d : \delay(\mathbb{N} + {\mho})$, we define a notion of termination in
$i$ steps with result $n_? \in \mathbb{N} + {\mho}$, written $d \da^i n_?$.
We define $d \da^i n_?$ inductively by the rules
$(\tnow\, n_?) \da^0 n_?$ and $(\tlater\, d) \da^{i+1} n_?$ if $d \da^i n_?$.
From this definition, it is clear that if there is an $i$
such that $d \da^i n_?$, then this $i$ is unique. 
% Furthermore, from the proof that
% $d$ terminates in $i$ steps, we can extract the result with which $d$
% terminates. This is defined by a straightforward induction on the proof that $d \da^i$.
Using this definition of termination, we define a partial function 
%
$-\Downarrow^\text{Delay} \colon \delay (\mathbb{N} + {\mho}) \rightharpoonup \mathbb{N} + {\mho}$
%
by existentially quantifying over $i$ for which $d \da^i n_?$. That is, we
define $d \Downarrow^\text{Delay} \,= n_?$ if $d \da^i n_?$ for some (necessarily
unique) $i$, and $d \Downarrow^\text{Delay}$ is undefined otherwise.

Now consider the global version of the term semantics, $\sem{\cdot}^{gl} \colon
\{M \,|\, \cdot \vdash M : \nat \} \to \li^{gl} \mathbb{N}$, defined by
$\sem{M}^{gl} := \lambda (k : \Clock) .\sem{M}$. Composing this with the
isomorphism $\alpha : \li^{gl} \mathbb{N} \cong \delay(\mathbb{N} + {\mho})$ and
the partial function $-\Downarrow^\text{Delay}$ yields the desired partial
function
%
$-\Downarrow \colon \{M \,|\, \cdot \vdash M : \nat \} \rightharpoonup \mathbb{N} + {\mho}$.
%
It is then straightforward to see that this big-step semantics respects the
equational theory and that the denotations of the diverging term $\Omega$, the
error term $\mho$, and a syntactic natural number value $n$ are all distinct.


\section{Challenges to a Guarded Model of Graduality}\label{sec:towards-relational-model}

% In the previous section, we gave a denotational semantics to the terms of the
% gradually-typed lambda calculus using the tools of SGDT. We defined a big-step
% semantics for closed terms of type $\nat$, as functions from $\mathbb{N} \to
% ((\mathbb{N} + 1) + 1)$.

Now that we have seen how to model the terms without regards to
graduality, we next turn to how to enhance this model to provide a
compositional semantics that additionally satisfies graduality. By a
compositional semantics, we mean that we want to provide a
compositional interpretation of \emph{type} and \emph{term precision}
such that we can extract the graduality relation from our model of
term precision.

\subsection{A first attempt: Modeling Term Precision with Posets}

In prior work, New and Licata model the term precision relation by
equipping $\omega$-CPOs with a second ordering $\ltdyn$ they call the
``error ordering'' such that a term precision relationship at a fixed
type $M \ltdyn M' : r(A)$ implies the error ordering $\sem{M} \ltdyn
\sem{M'}$ in the semantics. Then the error ordering on the free error
domain $\li A$ is based on the graduality property: essentially either
$\sem{M}$ errors or both terms diverge or both terms terminate with
values related by the ordering relation on $A$.
%
The obvious first try at a guarded semantics that models graduality
would be to similarly enhance our value types and computation types to
additionally carry a poset structure. For most type formers this
ordering can be defined as lifting poset structures on the
subformulae; the key design choice is in the ordering on the free
error domain $\li A$.

The ordering on $\li A$ should model the graduality property, but how
to interpret this is not so straightforward. The graduality property
as stated mentions divergence, but we only work indirectly with
diverging computations in the form of the think maps. The closest
direct encoding of the graduality property is an ordering we call the
\emph{step-insensitive error ordering}, defined by guarded recursion
in Figure~\ref{fig:step-insensitive-error-ordering}.
%
Firstly, to model graduality, $\mho$ is the least element. Next, two
returned values are related just when they are in the ordering $\ltdyn$
on $A$.
%
Two thinking elements are related just when they are related later.
%
The interesting cases are then those where one side is thinking and
the other has completed to either an error or a value.
%
Since the graduality property is \emph{extensional}, i.e., oblivious
to the number of steps taken, it is sensible to say that if one side
has terminated and the other side is thinking, that we must require
the thinking side to eventually terminate with a related behavior,
which is the content of the final three cases of the definition.
\begin{figure*}
      \begin{align*}
        \mho \semltbad l &\text{ iff } \top \\
        %
        \eta\, x \semltbad \eta\, y &\text{ iff } 
            x \ltdyn y \\
        %		
        \theta\, \tilde{l} \semltbad \theta\, \tilde{l'} &\text{ iff } 
            \later_t (\tilde{l}_t \semltbad \tilde{l'}_t) \\
        %	
        \theta\, \tilde{l} \semltbad \mho &\text{ iff } \exists n. \theta\, \tilde{l} = \delta^n(\mho) \\
        %	
        \theta\, \tilde{l} \semltbad \eta\, y &\text{ iff } \exists n. \exists x \ltdyn y.
            (\theta\, \tilde{l} = \delta^n(\eta\, x)) \\
        %	
        \eta\, x \semltbad \theta\, \tilde{l'} &\text { iff }
            \exists n. \exists y \gtdyn x. (\theta\, \tilde{l'} = \delta^n (\eta\, y))
    \end{align*}
    \caption{Step-insensitive error ordering}
    \label{fig:step-insensitive-error-ordering}
\end{figure*}

This definition of $\semltbad$ is sufficient to model the graduality
property in that if $l \semltbad l'$ for $l,l' : \li \mathbb{N}$ where
$\mathbb{N}$ is the flat poset whose ordering is equality, then the
big-step semantics we derive for $l, l'$ does in fact satisfy the
intended graduality relation.
%
However, there is a major issue with this definition: it's not a poset
at all: it is not anti-symmetric, but more importantly it is
\emph{not} transitive!
%
To see why, we observe the following undesirable property of any
relation that is transitive and ``step-insensitive'' on one side:
\begin{theorem}\label{thm:no-go}
  Let $R$ be a binary relation on the free error domain $U(\li A)$. If
  $R$ satisfies the following properties
  \begin{enumerate}
  \item Transitivity
  \item $\theta$-congruence: If $\later_t (\tilde{x}_t \binrel{R} \tilde{y}_t)$, then $\theta(\tilde{x}) \binrel{R} \theta(\tilde{y})$.
  \item Right step-insensitivity: If $x \binrel{R} y$ then $x \binrel{R} \delta y$.
  \end{enumerate}
  Then for any $l : U(\li A)$, we have $l \binrel{R} \Omega$. If $R$ is left
  step-insensitive instead then $\Omega \binrel{R} x$.
\end{theorem}
\begin{proof}
  By L\"ob induction, in the appendix.
\end{proof}
\begin{corollary}
  Let $R$ be a binary relation on $U(\li A)$. If $R$ satisfies
  transitivity, $\theta$-congruence and left and right
  step-insensitivity, then $R$ is the total relation: $\forall x, y. x
  \binrel{R} y$.
\end{corollary}
\begin{proof}
  $x \binrel{R} \Omega$ and $\Omega \binrel{R} y$ by the previous
  theorem. Then by transitivity $x \binrel R y$.
\end{proof}

This in turn implies $\semltbad$ is not transitive, since it is a
$\theta$-congruence and step-insensitive on both sides, but not
trivial: e.g., for the flat poset on $\mathbb N$, $\eta 0 \semltbad
\eta 1$ is false.
%
This shows definitively that we cannot provide a compositional
semantic graduality proof based on types as posets if we use this
step-insensitive ordering on $\li A$.

\subsection{Why transitivity is essential}

At this point we might try to weaken from our initial attempt at
providing a poset semantics to a semantics where types are equipped
with merely a reflexive relation, and continue using the
step-insensitive ordering. However, some level of transitivity is
absolutely essential to providing compositional reasoning, and
transitivity is used pervasively in the prior work by New and Licata.
%
The reason that it is essential becomes clear by examining the details of the
graduality proof, so first we expand on how we prove graduality compositionally.
First, we need to extend our compositional semantics of types and terms to a
compositional semantics of \emph{type precision derivations} and \emph{term
precision}.
%
Sticking to a poset-based semantics, if our value types are interpreted as
posets, then we want our precision derivations $c : A \ltdyn A'$ to be
interpreted as a \emph{relation} between $A$ and $A'$. It is also natural to
ask that the relation $c$ interact with the orderings on $A$ and $A'$. More
specifically, if $x' \ltdyn_{A} x$ and $x \mathrel{c} y$, then $x' \mathrel{c}
y$. Similarly, if $x \mathrel{c} y$ and $y \ltdyn_{A'} y'$, then $x \mathrel{c}
y'$. We summarize this requirement by saying $c$ is \emph{downward-closed} in
$A$ and $\emph{upward-closed}$ in $A'$. We will see below why this requirement
is necessary for the graduality proof.

% ...we then want our precision derivations $c : A \ltdyn A'$ to be \emph{poset relations} on $A,
% A'$, that is, relations between the underlying sets of $A, A'$ that are
% \emph{downward-closed} in $A$ and \emph{upward-closed} in $A'$.
% The former means that if $x' \ltdyn_{A} x$ and $x \mathrel{c} y$, then $x' \mathrel{c} y$,
% while the latter says that if $x \mathrel{c} y$ and $y \ltdyn_{A'} y'$, then $x \mathrel{c} y'$.

We call such a relation between $A$ and $A'$ a \emph{poset relation}, and we
write $c : A \rel A'$. The paradigmatic example of a poset relation is the poset
ordering itself, $\ltdyn_A : A \rel A$, which is a poset relation because it is
transitive.

%
Given two poset relations $c : A_1 \rel A_2$ and $c' : A_2 \rel A_3$, we can
form their \emph{composition} in the usual manner, i.e., $x \binrel{c \comp c'} y$ if and
only if there is an element $z : A_2$ with $x \binrel{c} z$ and $z \binrel{c'} y$.
%We write the composition of relations as $c \comp c'$.

% \max{TODO: Eric pick
%   up from here. Go into the squares for casts and how representability
%   gives us a compositional approach to proving them}

% \eric{Picked up here; Max, please read this.}

These relations are then used to model term precision. For closed programs $M \ltdyn M' :
c$ where $c : A \ltdyn A'$, the terms denote elements of the
free error domain $\sem{M},\sem{M'} : U\li \sem{A}$, and we model the term precision semantically as the relation holding between those denoted elements $\sem{M} \binrel{(U\li\sem{c})} \sem{M'}$ where
$U\li\sem{c}$ is a relational lifting of $U\circ\li$.
%
More generally, $M$ and $M'$ can be \emph{open} terms, in which case
they denote functions, not just elements. We model that as not simply
elements being related but in a typical logical-relations style as
\emph{preserving} relatedness: if they are passed in inputs related by
the domain relations then their output is related by the codomain
relation.
%
To capture this preservation of relations property, we use the notion of a
\emph{square}, which specifies a relation between two monotone functions. The
definition of square is as follows. Let $A_i, A_o, A_i', A_o'$ be
partially-ordered sets. Let $c_i : A_i \rel A_i'$ and $c_o : A_o \rel A_o'$ be
poset relations, and let $f : A_i \to A_o$ and $g : A_i' \to A_o'$ be monotone
functions. We say that $f \ltsq{c_i}{c_o} g$ if for all $x : A_i$ and $y : A_i'$
with $x \binrel{c_i} y$, we have $f(x) \binrel{c_o} g(y)$. We call this a
\emph{square} because we visualize this situation as follows:
%
% https://q.uiver.app/#q=WzAsNCxbMCwwLCJBX2kiXSxbMCwxLCJBX28iXSxbMSwwLCJBX2knIl0sWzEsMSwiQV9vJyJdLFsyLDMsImciXSxbMCwxLCJmIiwyXSxbMCwyLCJjX2kiLDAseyJzdHlsZSI6eyJib2R5Ijp7Im5hbWUiOiJiYXJyZWQifSwiaGVhZCI6eyJuYW1lIjoibm9uZSJ9fX1dLFsxLDMsImNfbyIsMix7InN0eWxlIjp7ImJvZHkiOnsibmFtZSI6ImJhcnJlZCJ9LCJoZWFkIjp7Im5hbWUiOiJub25lIn19fV1d
\[\begin{tikzcd}[ampersand replacement=\&]
  {A_i} \& {A_i'} \\
  {A_o} \& {A_o'}
  \arrow["{c_i}", "\shortmid"{marking}, no head, from=1-1, to=1-2]
  \arrow["f"', from=1-1, to=2-1]
  \arrow["g", from=1-2, to=2-2]
  \arrow["{c_o}"', "\shortmid"{marking}, no head, from=2-1, to=2-2]
\end{tikzcd}\]
%
Squares are then used to model the term precision ordering in that
$\Delta \vdash M \ltdyn N : c$ implies $\sem{M}
\ltsq{\sem{\Delta}}{U\li\sem{c}} \sem{N}$ where here $\sem{\Delta =
  x_1:c_1,\ldots}$ is the action of the product on the relations
$c_1,\ldots$.

%% % https://q.uiver.app/#q=WzAsNCxbMCwwLCJcXEdhbW1hIl0sWzAsMSwiVVxcbGkgQSJdLFsxLDAsIlxcR2FtbWEnIl0sWzEsMSwiVVxcbGkgQSciXSxbMiwzLCJcXHNlbXtOfSJdLFswLDEsIlxcc2Vte019IiwyXSxbMCwyLCJcXEdhbW1hXntcXGx0ZHlufSIsMCx7InN0eWxlIjp7ImJvZHkiOnsibmFtZSI6ImJhcnJlZCJ9LCJoZWFkIjp7Im5hbWUiOiJub25lIn19fV0sWzEsMywiVVxcbGkgYyIsMix7InN0eWxlIjp7ImJvZHkiOnsibmFtZSI6ImJhcnJlZCJ9LCJoZWFkIjp7Im5hbWUiOiJub25lIn19fV1d
%% \[\begin{tikzcd}[ampersand replacement=\&]
%% 	\Gamma \& {\Gamma'} \\
%% 	{U\li A} \& {U\li A'}
%% 	\arrow["{\Gamma^{\ltdyn}}", "\shortmid"{marking}, no head, from=1-1, to=1-2]
%% 	\arrow["{\sem{M}}"', from=1-1, to=2-1]
%% 	\arrow["{\sem{N}}", from=1-2, to=2-2]
%% 	\arrow["{U\li c}"', "\shortmid"{marking}, no head, from=2-1, to=2-2]
%% \end{tikzcd}\]

% \max{This needs to also talk about relational composition and horizontal pasting squares, since that's transitivity!}
We note that for any $c$, there is a ``vertical identity'' square $\id
\ltsq{c}{c} \id$, as can be seen by unfolding the
definition. Similarly, for any monotone function $f : A_i \to A_o$,
there is a ``horizontal identity'' square $f
\ltsq{\le_{A_i}}{\le_{A_o}} f$ arising from the fact that $f$ is
monotone.
%
We can \emph{compose} squares vertically, i.e., if we have $f
\ltsq{c_1}{c_2} f'$ and $g \ltsq{c_2}{c_3} g'$ then we obtain a square
$g \circ f \ltsq{c_1}{c_3} g' \circ f'$. Likewise, we can compose
squares horizontally: If $f \ltsq{c_i}{c_o} g$ and $g
\ltsq{c_i'}{c_o'} h$, then we obtain a square $f \ltsq{c_i \comp
  c_i'}{c_o \comp c_o'} h$. Horizontal composition of squares
corresponds to transitivity of term precision. While we do not dwell
on categorical abstractions in this work, we note that this structure
of functions, relations and squares forms a locally thin \emph{double
category} as used in previous work \cite{new-licata18}.

Now that we have an intended model of term precision in the form of
squares, we can see what is required to give a compositional proof of
graduality. Most of the cases of term precision are just congruence
and follow easily. The core of the proof of the graduality property is
in proving the validity of the \emph{cast rules} UpL, UpR, DnL and
DnR.
%
These rules specify a relationship between the semantics of type
precision derivations $c$ and the corresponding casts $\upc c, \dnc c$ (see Section
\ref{sec:GTLC}).
%
To prove graduality compositionally then, we will need to codify in an
extra property on $\sem{c}$ that the relation $\sem{c}$ has some
correspondence with the casts $\sem{\upc c}, \sem{\dnc c}$.
%
What property should we require?

Consider the UpL rule. It states that if
$c : A_1 \ltdyn A_2$ and $c' : A_2 \ltdyn A_3$ and $M : A_1$ is
related to $N : A_3$ via the composition $c \comp c'$, then the upcast
$\upc{c} M$ is related to $N$ via $c'$. Since the upcasts are pure
functions in the semantics this can be deduced from the existence of
the following square, where the relation on the top is $\sem{c} \comp \sem{c'}$:
% \max{the square doesn't make sense because we
%   haven't discussed horizontal composition of relations or that this
%   notation means horizontal composition of relations}
%
% https://q.uiver.app/#q=WzAsNSxbMCwwLCJBXzEiXSxbMSwwLCJBXzIiXSxbMiwwLCJBXzMiXSxbMCwxLCJBXzIiXSxbMiwxLCJBXzMiXSxbMyw0LCJjJyIsMix7InN0eWxlIjp7ImJvZHkiOnsibmFtZSI6ImJhcnJlZCJ9LCJoZWFkIjp7Im5hbWUiOiJub25lIn19fV0sWzAsMywiXFx1cGN7Y30iLDJdLFsyLDQsIlxcaWQiXSxbMCwxLCJjIiwwLHsic3R5bGUiOnsiYm9keSI6eyJuYW1lIjoiYmFycmVkIn0sImhlYWQiOnsibmFtZSI6Im5vbmUifX19XSxbMSwyLCJjJyIsMCx7InN0eWxlIjp7ImJvZHkiOnsibmFtZSI6ImJhcnJlZCJ9LCJoZWFkIjp7Im5hbWUiOiJub25lIn19fV1d
\[\begin{tikzcd}[ampersand replacement=\&]
  {\sem{A_1}} \& {\sem{A_2}} \& {\sem{A_3}} \\
  {\sem{A_2}} \&\& {\sem{A_3}}
  \arrow["\sem{c}", "\shortmid"{marking}, no head, from=1-1, to=1-2]
  \arrow["{\sem{\upc{c}}}"', from=1-1, to=2-1]
  \arrow["{\sem{c'}}", "\shortmid"{marking}, no head, from=1-2, to=1-3]
  \arrow["\id", from=1-3, to=2-3]
  \arrow["{\sem{c'}}"', "\shortmid"{marking}, no head, from=2-1, to=2-3]
\end{tikzcd}\]
While at first look this seems to be specifying a relationship between
$\sem{c}$ and $\sem{\upc c}$, there is a problem: it also quantifies
involves an arbitrary other relation $\sem{c'}$! This means we cannot
require the existence of this square as part of the definition of a
relation between value types, as it is self-referential. This would
seem to imply that we cannot give a compositional model for
graduality. However, New and Licata observed that in the presence of
transitivity, we can \emph{derive} the above squares from simpler ones
that do not involve composition of relations. Below are the simpler squares
corresponding to UpL, UpR, DnL, and DnR:
\begin{center}
  \begin{tabular}{ c | c | c | c} 
    % UpL
    % https://q.uiver.app/#q=WzAsNCxbMCwwLCJBXzEiXSxbMCwxLCJBXzIiXSxbMSwwLCJBXzIiXSxbMSwxLCJBXzIiXSxbMCwyLCJjIiwwLHsic3R5bGUiOnsiYm9keSI6eyJuYW1lIjoiYmFycmVkIn0sImhlYWQiOnsibmFtZSI6Im5vbmUifX19XSxbMSwzLCJcXGx0ZHluX3tBXzJ9IiwyLHsic3R5bGUiOnsiYm9keSI6eyJuYW1lIjoiYmFycmVkIn0sImhlYWQiOnsibmFtZSI6Im5vbmUifX19XSxbMCwxLCJcXHVwY3tjfSIsMl0sWzIsMywiXFxpZCJdLFs2LDcsIlxcdGV4dHtVcEx9IiwzLHsic2hvcnRlbiI6eyJzb3VyY2UiOjIwLCJ0YXJnZXQiOjIwfSwic3R5bGUiOnsiYm9keSI6eyJuYW1lIjoibm9uZSJ9LCJoZWFkIjp7Im5hbWUiOiJub25lIn19fV1d
    \begin{tikzcd}[ampersand replacement=\&]
      {A_1} \& {A_2} \\
      {A_2} \& {A_2}
      \arrow["c", "\shortmid"{marking}, no head, from=1-1, to=1-2]
      \arrow[""{name=0, anchor=center, inner sep=0}, "{\upc{c}}"', from=1-1, to=2-1]
      \arrow[""{name=1, anchor=center, inner sep=0}, "\id", from=1-2, to=2-2]
      \arrow["{\ltdyn_{A_2}}"', "\shortmid"{marking}, no head, from=2-1, to=2-2]
      \arrow["{\text{UpL}}"{marking, allow upside down}, draw=none, from=0, to=1]
  \end{tikzcd} &
    %
    % UpR
    % https://q.uiver.app/#q=WzAsNCxbMCwwLCJBXzEiXSxbMCwxLCJBXzEiXSxbMSwwLCJBXzEiXSxbMSwxLCJBXzIiXSxbMCwyLCJcXGx0ZHluX3tBXzF9IiwwLHsic3R5bGUiOnsiYm9keSI6eyJuYW1lIjoiYmFycmVkIn0sImhlYWQiOnsibmFtZSI6Im5vbmUifX19XSxbMSwzLCJjIiwyLHsic3R5bGUiOnsiYm9keSI6eyJuYW1lIjoiYmFycmVkIn0sImhlYWQiOnsibmFtZSI6Im5vbmUifX19XSxbMCwxLCJcXGlkIiwyXSxbMiwzLCJcXHVwYyBjIl0sWzYsNywiXFx0ZXh0e1VwUn0iLDMseyJzaG9ydGVuIjp7InNvdXJjZSI6MjAsInRhcmdldCI6MjB9LCJzdHlsZSI6eyJib2R5Ijp7Im5hbWUiOiJub25lIn0sImhlYWQiOnsibmFtZSI6Im5vbmUifX19XV0=
    \begin{tikzcd}[ampersand replacement=\&]
      {A_1} \& {A_1} \\
      {A_1} \& {A_2}
      \arrow["{\ltdyn_{A_1}}", "\shortmid"{marking}, no head, from=1-1, to=1-2]
      \arrow[""{name=0, anchor=center, inner sep=0}, "\id"', from=1-1, to=2-1]
      \arrow[""{name=1, anchor=center, inner sep=0}, "{\upc c}", from=1-2, to=2-2]
      \arrow["c"', "\shortmid"{marking}, no head, from=2-1, to=2-2]
      \arrow["{\text{UpR}}"{marking, allow upside down}, draw=none, from=0, to=1]
    \end{tikzcd} &
    %
    %
    %
    % DnL
    % https://q.uiver.app/#q=WzAsNCxbMCwwLCJBXzIiXSxbMCwxLCJBXzEiXSxbMSwwLCJBXzIiXSxbMSwxLCJBXzIiXSxbMCwyLCJcXGx0ZHluX3tBXzJ9IiwwLHsic3R5bGUiOnsiYm9keSI6eyJuYW1lIjoiYmFycmVkIn0sImhlYWQiOnsibmFtZSI6Im5vbmUifX19XSxbMSwzLCJjIiwyLHsic3R5bGUiOnsiYm9keSI6eyJuYW1lIjoiYmFycmVkIn0sImhlYWQiOnsibmFtZSI6Im5vbmUifX19XSxbMCwxLCJcXGRuYyBjIiwyXSxbMiwzLCJcXGlkIl0sWzYsNywiXFx0ZXh0e0RuTH0iLDMseyJzaG9ydGVuIjp7InNvdXJjZSI6MjAsInRhcmdldCI6MjB9LCJzdHlsZSI6eyJib2R5Ijp7Im5hbWUiOiJub25lIn0sImhlYWQiOnsibmFtZSI6Im5vbmUifX19XV0=
    \begin{tikzcd}[ampersand replacement=\&]
      {A_2} \& {A_2} \\
      {A_1} \& {A_2}
      \arrow["{\ltdyn_{A_2}}", "\shortmid"{marking}, no head, from=1-1, to=1-2]
      \arrow[""{name=0, anchor=center, inner sep=0}, "{\dnc c}"', from=1-1, to=2-1]
      \arrow[""{name=1, anchor=center, inner sep=0}, "\id", from=1-2, to=2-2]
      \arrow["c"', "\shortmid"{marking}, no head, from=2-1, to=2-2]
      \arrow["{\text{DnL}}"{marking, allow upside down}, draw=none, from=0, to=1]
    \end{tikzcd} &
    %
    % DnR
    % https://q.uiver.app/#q=WzAsNCxbMCwwLCJBXzEiXSxbMCwxLCJBXzEiXSxbMSwwLCJBXzIiXSxbMSwxLCJBXzEiXSxbMCwyLCJjIiwwLHsic3R5bGUiOnsiYm9keSI6eyJuYW1lIjoiYmFycmVkIn0sImhlYWQiOnsibmFtZSI6Im5vbmUifX19XSxbMSwzLCJcXGx0ZHluX3tBXzF9IiwyLHsic3R5bGUiOnsiYm9keSI6eyJuYW1lIjoiYmFycmVkIn0sImhlYWQiOnsibmFtZSI6Im5vbmUifX19XSxbMCwxLCJcXGlkIiwyXSxbMiwzLCJcXGRuYyBjIl0sWzYsNywiXFx0ZXh0e0RuUn0iLDMseyJzaG9ydGVuIjp7InNvdXJjZSI6MjAsInRhcmdldCI6MjB9LCJzdHlsZSI6eyJib2R5Ijp7Im5hbWUiOiJub25lIn0sImhlYWQiOnsibmFtZSI6Im5vbmUifX19XV0=
    \begin{tikzcd}[ampersand replacement=\&]
      {A_1} \& {A_2} \\
      {A_1} \& {A_1}
      \arrow["c", "\shortmid"{marking}, no head, from=1-1, to=1-2]
      \arrow[""{name=0, anchor=center, inner sep=0}, "\id"', from=1-1, to=2-1]
      \arrow[""{name=1, anchor=center, inner sep=0}, "{\dnc c}", from=1-2, to=2-2]
      \arrow["{\ltdyn_{A_1}}"', "\shortmid"{marking}, no head, from=2-1, to=2-2]
      \arrow["{\text{DnR}}"{marking, allow upside down}, draw=none, from=0, to=1]
    \end{tikzcd}
  \end{tabular}
\end{center}
%
These squares say that the relation $c$ is \emph{representable} by the upcast
morphism $\upc{c}$, essentially that the relation is a kind of \emph{graph} of
the function in that $x \binrel{\sem c} y$ if and only if $\sem{\upc c}(x)
\leq_{A_2} y$ \cite{shulman2007}.

We can use the simpler UpL square to derive the original UpL square by
horizontally composing with the identity square for $c'$:
%
% https://q.uiver.app/#q=WzAsNixbMCwwLCJBXzEiXSxbMSwwLCJBXzIiXSxbMiwwLCJBXzMiXSxbMCwxLCJBXzIiXSxbMiwxLCJBXzMiXSxbMSwxLCJBXzIiXSxbMCwxLCJjIiwwLHsic3R5bGUiOnsiYm9keSI6eyJuYW1lIjoiYmFycmVkIn0sImhlYWQiOnsibmFtZSI6Im5vbmUifX19XSxbMSwyLCJjJyIsMCx7InN0eWxlIjp7ImJvZHkiOnsibmFtZSI6ImJhcnJlZCJ9LCJoZWFkIjp7Im5hbWUiOiJub25lIn19fV0sWzMsNSwiXFxsdGR5bl97QV8yfSIsMix7InN0eWxlIjp7ImJvZHkiOnsibmFtZSI6ImJhcnJlZCJ9LCJoZWFkIjp7Im5hbWUiOiJub25lIn19fV0sWzUsNCwiYyciLDIseyJzdHlsZSI6eyJib2R5Ijp7Im5hbWUiOiJiYXJyZWQifSwiaGVhZCI6eyJuYW1lIjoibm9uZSJ9fX1dLFswLDMsIlxcdXBje2N9IiwyXSxbMiw0LCJcXGlkIl0sWzEsNSwiXFxpZCIsMl1d
\[\begin{tikzcd}[ampersand replacement=\&]
	{A_1} \& {A_2} \& {A_3} \\
	{A_2} \& {A_2} \& {A_3}
	\arrow["c", "\shortmid"{marking}, no head, from=1-1, to=1-2]
	\arrow["{\upc{c}}"', from=1-1, to=2-1]
	\arrow["{c'}", "\shortmid"{marking}, no head, from=1-2, to=1-3]
	\arrow["\id"', from=1-2, to=2-2]
	\arrow["\id", from=1-3, to=2-3]
	\arrow["{\ltdyn_{A_2}}"', "\shortmid"{marking}, no head, from=2-1, to=2-2]
	\arrow["{c'}"', "\shortmid"{marking}, no head, from=2-2, to=2-3]
\end{tikzcd}\]
%
Then to complete the proof, we observe that the composition of $\ltdyn_{A_2}$ with $c'$ is equal to
$c'$, because $c'$ is \emph{downward-closed} under the relation on $A_2$.
An analogous argument shows that we can derive the original UpR square from the simpler version of UpR,
this time using the fact that poset relations are \emph{upward-closed}.
% Thus, we
% obtain the original UpL square by vertically composing the square shown above
% with the square $\id \ltsq{\le_{A_2}\, c'}{c'} \id$ on the bottom.

To recap, we have shown that in order to carry out the graduality proof
compositionally, we need that the relations are closed under the ordering on
each side. In particular, this implies that the ordering relations $\ltdyn_{A}$
must be transitive. Thus, we cannot entirely drop posets from our semantics.

% The actual place where transitivity of the relations on the value types is
% needed in this argument occurs when $A_3 = A_2$ and we take $c'$ to be
% $\ltdyn_{A_2}$ (recall that $c'$ is universally quantified, so it can be any
% relation). The fact that $\ltdyn_{A_2}$ is downward-closed is equivalent to it
% being transitive. In other words, the relations $\ltdyn_A$ on value types must be
% poset relations in order to carry out the above compositional construction.

% So we see that in order to get a sensible condition that can give us a
% compositional definition of a well-behaved relation, some amount of
% transitive reasoning on squares is required, and so we cannot entirely
% drop posets from our semantics.

\subsection{Resolution: Splitting Error Ordering and Bisimilarity}\label{sec:lock-step-and-weak-bisim}

Given that some amount of transitive reasoning is essential for
modeling type and term precision compositionally, we need to revisit
our ordering relation on $\li A$ to get one that is
transitive. Theorem \ref{thm:no-go} tells us, however, that any non-trivial such
relation must either not be a congruence with respect to $\theta$, or
must not be left- and right-step-insensitive.
\footnote{Technically, the theorem only implies that we need to drop
one of left- or right-step-insensitivity. We might then attempt
to work with two separate ``step-semi-sensitive'' relations on $\li
A$, each closed under delays on the left and right respectively. This
has some similarities to prior work on logical relations models
\cite{new-licata-ahmed2019,new-giovannini-licata-2022}, but we do not take this
approach because the relations still do not seem to be provably
transitive, though we have not been able to prove that they are not
transitive either.}
% \max{Eric check if this footnote is correct}}
%
We cannot forego the property of being a $\theta$-congruence, as
without this we would not be able to prove basic properties of the
relation using \lob-induction, e.g., that the extension $f^\dagger$ is
\emph{monotone}, which is used pervasively in the proof of graduality.
%
Thus, we choose to sacrifice left and right step-insensitivity. That
is, we will define an ordering relation that requires terms to be in
``lock-step''. In order for two computations to be related in this
ordering, they must have the exact same stepping behavior
(unless/until the left-hand side results in an error).

More formally, we define the \emph{lock-step error-ordering}, with the idea
being that two computations $l$ and $l'$ are related if they are in lock-step
with regard to their intensional behavior, up to $l$ erroring. Figure
\ref{fig:lock-step-error-ordering} gives the definition of this relation.

\begin{figure}
  \begin{minipage}{0.5\textwidth}
    \fbox{$l_1 \ltls l_2$}
    \begin{align*}
        &\eta\, x \ltls \eta\, y \text{ if } 
            x \mathbin{\ltdyn_A} y \\
        %		
        &\mho \ltls l' \\
        %
        &\theta\, \tilde{l} \ltls \theta\, \tilde{l'} \text{ if } 
            \later_t (\tilde{l}_t \ltls \tilde{l'}_t)
  \end{align*}\end{minipage}\begin{minipage}{0.5\textwidth}
\fbox{$l_1 \bisim l_2$}
    \begin{align*}
        \mho \bisim \mho &\text{ iff } \top \\
      %
        \eta\, x \bisim \eta\, y &\text{ iff } x \bisim_A y \\
      %		
        \theta\, \tilde{x} \bisim \theta\, \tilde{y} &\text{ iff } \later_t (\tilde{x}_t \bisim \tilde{y}_t) \\
      %	
        \theta\, \tilde{x} \bisim \mho &\text{ iff } \exists n. \theta\, \tilde{x} = \delta^n(\mho)\\
      %	
        \theta\, \tilde{x} \bisim \eta\, y &\text{ iff } \exists n. \exists x \bisim_A y.
          (\theta\, \tilde{x} = \delta^n(\eta\, x))\\
      %
        \mho \bisim \theta\, \tilde{y} &\text{ iff } \exists n. \theta\, \tilde{y} = \delta^n(\mho) \\
      %	
        \eta\, x \bisim \theta\, \tilde{y} &\text { iff } \exists n. \exists y \bisim_A x. (\theta\, \tilde{y} = \delta^n (\eta\, y))
      \end{align*}
  \end{minipage}
    \caption{lock-step error ordering and weak bisimilarity}
    \label{fig:lock-step-error-ordering}
\end{figure}

When both sides are $\eta$, then we check that the returned values are related
in $\ltdyn_A$. The error term $\mho$ is the least element. Lastly, if both sides step
(i.e., are a $\theta$) then we compare the resulting computations one time step
later.
%
It is straightforward to prove using \lob-induction that this relation is
reflexive, transitive and anti-symmetric given that the underlying relation $R$
has those properties. The lock-step ordering is therefore the partial ordering
we will associate with $\li A$.
%
More generally we can define a heterogeneous version of this ordering that lifts
poset relation $c : A \rel A'$ to a poset relation $\li c : \li A \rel \li A'$.

However, we also cannot completely drop step-\emph{insensitivity} from
our model. The graduality property itself is not sensitive to steps,
and in fact there are terms related by term precision that take
differing numbers of steps.
%
The offending step arises precisely from the place where our
definition of $D$ uses a $\later$: the function case.
%
Consider the representability condition corresponding to the DnL rule
for $\iarr \colon \dyntodyn \ltdyn\, \dyn$. This is the square $\sem{\dnc{\iarr}} \ltsq{r({\li D})}{\li\sem{\iarr}} \id$.
%% % https://q.uiver.app/#q=WzAsNCxbMCwwLCJcXGxpIEQiXSxbMCwxLCJcXGxpIFUoRCBcXGFyciBcXGxpIEQpIl0sWzEsMCwiXFxsaSBEIl0sWzEsMSwiXFxsaSBEIl0sWzAsMiwiXFxsZXFfe1xcbGkgRH0iLDAseyJzdHlsZSI6eyJib2R5Ijp7Im5hbWUiOiJiYXJyZWQifSwiaGVhZCI6eyJuYW1lIjoibm9uZSJ9fX1dLFsxLDMsIlxcbGlcXGlhcnIiLDIseyJzdHlsZSI6eyJib2R5Ijp7Im5hbWUiOiJiYXJyZWQifSwiaGVhZCI6eyJuYW1lIjoibm9uZSJ9fX1dLFswLDEsIlxcZG5je1xcaWFycn0iLDJdLFsyLDMsIlxcaWQiXV0=
%% \[\begin{tikzcd}[ampersand replacement=\&]
%% 	{\li D} \& {\li D} \\
%% 	{\li U(D \arr \li D)} \& {\li D}
%% 	\arrow["{\leq_{\li D}}", "\shortmid"{marking}, no head, from=1-1, to=1-2]
%% 	\arrow["{\sem{\dnc{\iarr}}}"', from=1-1, to=2-1]
%% 	\arrow["\id", from=1-2, to=2-2]
%% 	\arrow["\li\sem{\iarr}"', "\shortmid"{marking}, no head, from=2-1, to=2-2]
%% \end{tikzcd}\]
For this square to be valid with the lock-step ordering the left and
right hand side would need to take the same number of steps, at least
when the left-hand side does not error.
%
However the right hand side \emph{always} takes $0$ steps, as it is
the identity function, whereas on the left, our definition of
$\sem{\dnc{\iarr}}$ had to take an observable step when its input is a
function: this was inherent to the fact that the function case of the
dynamic type is guarded by a later.

However, observe that we can remedy this particular situation by
replacing the $\id$ on the right hand side by an innocuous function
$\delta^* = (\delta \circ \eta)^\dagger$ that on an input value takes
a single computational step, but is otherwise the identity
function. If we were to ignore computational steps, then this function
\emph{would} be the identity function. We call such a function that is
the identity function except for the introduction of computational
steps a \emph{perturbation}.

We formalize this property of being equivalent ``except for steps''
with a second relation on the free error domain $\li A$: \emph{weak
bisimilarity}, defined in Figure~\ref{fig:lock-step-error-ordering}, which is
parameterized by a binary relation $\bisim_A$ on $A$
\cite{mogelberg-paviotti2016}.
% For a type $X$, we define a relation on $\li X$, called ``weak bisimilarity",
% written $l \bisim l'$. 
Two errors are bisimilar, and when both sides are $\eta$, we ensure
that the underlying values are bisimilar in the underlying
bisimilarity relation on $A$. When both sides are thinking, we ensure
the terms are bisimilar later.  Most importantly, when one side is
thinking but the other terminates at $\eta x$ (i.e., one side steps),
we stipulate that the $\theta$-term runs to $\eta y$ where $x$ is
related to $y$. And similarly, if one side is thinking and the other
errors, we ensure the thinking side eventually errors.

It can be shown (by \lob-induction) that weak bisimilarity is
reflexive and symmetric. Since it is non-trivial, a $\theta$
congruence and step-insensitive on both sides, by
Theorem~\ref{thm:no-go}, we also know that it is \emph{not}
transitive.
%
We will then require our denotations of types to be not just posets,
but posets additionally equipped with a reflexive symmetric relation
$\bisim_A$.

Then we can refine our denotation of term precision $\Delta \vdash M
\ltdyn N : c$ to mean not that $\sem{M}$ and $\sem{N}$ are necessarily
in a lock-step error ordering directly, but that they can be
``synchronized'' to do so, that is that there exist $f \bisim \sem{M}$
and $g \bisim \sem{N}$ such that $f \ltsq{\sem{\Delta}}{U\li\sem{c}}
g$.
%
Note that this in turn implies that $\sem{M}$ and $\sem{N}$ are
related in the original \emph{step-insensitive} error ordering that we
sought to prove!
%
We additionally weaken our representability squares so that our
upcasts and downcasts are not required to be in a lock-step ordering
with an identity but instead to be in a lock-step ordering with some
perturbation, that is a function \emph{bisimilar to} the identity. We
call this weakened form of representability
\emph{quasi}-representability.
%
Then if our relations are quasi-representable we can in fact prove the
validity of our cast rules.
%
The remainder of the proof of graduality then is to show that this
property of being quasi-representable is itself compositional: that
all of the constructions we have on type precision preserve the
property of being quasi-representable.
%
While this is true for representability, it is not quite true for
quasi-representability. The reason is that perturbations are not quite
as well behaved as \emph{actual} identity functions. To solve this
final issue, we attach one final piece of information to our types
$A$: a type of \emph{syntactic} perturbations that represent functions
bisimilar to the identity, but are presented as data so that we can
perform operations such as composition and functorial actions on them.
%
With this notion of syntactic perturbation, and additional
requirements that the relations interact well with perturbations, we
finally achieve our desired result: a compositional,
syntax-independent proof of graduality.


%% Importantly, we note that as with the original ordering $\semltbad$ defined at
%% the beginning of this section, the relation just defined is not transitive
%% (again as a result of the above no-go theorem). It may therefore seem that we
%% have not solved the original issue we faced. In a sense, this is true. The model
%% of gradual typing that we end up with will not support \emph{extensional}
%% compositional reasoning. However, by decomposing the denotation of term
%% precision in the above manner, we can employ \emph{intensional} compositional
%% reasoning by working with the lock-step error ordering. Thus, the tentative plan
%% going forward will be to carry out the proofs compositionally using the
%% lock-step ordering. We will then apply the closure under weak bisimilarity in
%% the denotation of term precision given above to handle the aspects involving
%% stepping.

%% Notice that this delay
%% function is weakly bisimilar to the identity function in that $\delta^*\, x
%% \bisim x$ for all $x$. \footnote{This follows from the fact that $\delta$ is
%% weakly bisimilar to the identity ($\delta\, x \bisim x$ for all $x$) and that
%% $-^\dagger$ preserves weak bisimilarity.} Thus, it will disappear when we apply
%% the bisimilarity-closure construction described above. That is, this delay makes
%% no difference in the extensional setting, but its presence is crucial in the
%% intensional setting.


%% The precise definition of $\iarr$ as a poset relation is not important
%% for this discussion.
%% %

%% \eric{We may want to introduce error domain relations and squares at this point,
%% since the squares for DnL and DnR involve lift.}
%
%% We now recall the definition of the semantics of a downcast given in Section
%% \ref{sec:term-interpretation}. The downcast on the left will insert a $\theta$
%% in the case where the value of type $D$ is a later-function $\tilde{f}$.
%% %
%% Thus, in order for the downcast to be related to the RHS in the lock-step error
%% ordering on $\li (D \to \li D)$, the RHS must be of the form $\theta(\dots)$,
%% and moreover, after one time step, the argument of $\theta$ on the LHS must be
%% related to the argument of $\theta$ on the RHS. As it stands now, this need not
%% be the case, e.g., if the RHS is of the form $\eta(\dots)$. Thus, we conclude
%% that the DnL rule does not in general hold under the lock-step error ordering.


%% \max{resume here}
%% To deal with
%% terms that take differing numbers of steps, we then define a separate
%% relation called \emph{weak bisimilarity} that relates terms that are
%% extensionally equal and may differ by a finite number of delays. Then,
%% the semantics of the error ordering for the guarded lift monad will
%% involve a combination of these two relations: a ``closure'' of the
%% lock-step error ordering under weak bisimilarity on both sides.
%% % Although the combined relation will not be transitive (for the same reason that
%% % $\semltbad$ is not transitive), this...
%% This decomposition has the advantage that we can recover some transitive
%% reasoning and push much of the reasoning about stepping to the margins of the
%% development.

%% % ...to carry out the proofs using the lock-step ordering and handle the stepping
%% % behavior separately via weak bisimilarity. In the end we combine everything
%% % together using the above denotation for term precision


%% \subsection{Extending the Model to Higher-Order Types}
%% \max{this subsection doesn't really say anything concrete. If we want to talk about perturbations, actually talk about them}

%% % At this point, we could carry out the full construction of a concrete relational
%% % model with these additional features. 

%% The above techniques give a semantic interpretation of term precision for closed
%% terms of base type. We now consider how to extend these constructions to
%% potentially-open terms of \emph{all} types. In particular, in order to model
%% higher-order data types (i.e., functions) we need to equip \emph{every} semantic
%% object with not only a partial ordering relation but also a ``bisimilarity''
%% relation that is reflexive and symmetric. We must similarly equip every object
%% with a structured set of delays that can be inserted and manipulated to ensure
%% the appropriate cast rules hold at all higher-order types. For example, the
%% upcast for a derivation $c_i \ra c_o$ involves the downcast corresponding to
%% $c_i$ and the upcast corresponding to $c_o$. We must be able to insert delays in
%% a functorial manner to mimic the structure of the casts. In the next section, we
%% will make the necessary definitions and describe the relevant constructions.

% For the sake of reusability and modularity, rather than carry out this
% construction in the concrete setting developed here, we will instead return to
% the abstract setting and adjust our definition of model to account for these
% requirements. We will break the construction into smaller steps and isolate the
% pieces that require the techniques of SGDT from those that do not. Then with
% this framework at our disposal, we will return to the construction of a concrete
% model in Section \ref{sec:concrete-model}.
% , taking as a starting point the definitions introduced in the current section.

% Thus in the next section we define revised notions of a model of
% gradual typing based on the lessons learned in the present section. 

% The lack of transitivity presents a major barrier towards giving a compositional
% semantics to gradual typing: if our relations are not transitive, then we cannot
% compose squares horizontally. But we have argued above that horizontal
% composition is essential for modelling the axioms of gradual typing in a
% syntax-independent manner.

% In particular, to model the UpL/UpR/DnL/DnR rules
% for casts in the presence of transitivity, it is sufficient to establish the
% validity of simpler rules that do not ``build in'' composition. We can then use
% transitivity to derive the original versions of the rules. On the other hand,
% without transitivity we must instead validate the cast rules in the model ``from
% scratch''. In fact, we cannot even define the semantic interpretation of type
% precision in a syntax-independent manner. The issue is that in the definition of
% relation, the requirement that it be representable now involves quantifying over
% all other relations, which is circular.
 
% So, although this approach would suffice for proving graduality, it lacks the
% compositionality that we seek in a reusable framework for the semantics of
% gradual typing.

%Because it is convenient to make use of transitive reasoning in proving graduality, ...


% Doing so, we define a \emph{lock-step} error ordering, where roughly speaking,
% in order for computations to be related, they must have the same stepping
% behavior. We then formulate a separate relation, \emph{weak bisimilarity}, that
% relates computations that are extensionally equal and may differ only in their
% stepping behavior. % up to a finite number of \delta's
% Finally, the semantics of term precision will involve a combination of these two
% relations, a sort of closure of the lock-step ordering under weak bisimilarity
% on both sides. This decomposition has the advantage that we can recover some
% transitive reasoning and push the parts involving stepping to the margins of the
% development.



\begin{comment}
\subsection{Modeling Type and Term Precision}
\max{TODO: figure out where this stuff goes}
% We will model type precision $c : A \ltdyn A'$ as a relation between the sets
% $\sem{A}$ and $\sem{A'}$. 

To model term precision, we begin by equipping the denotation of every type with
an ordering relation. Since term precision is reflexive and transitive, and
since we identify terms that are equi-precise, we model value types as
partially-ordered sets and values $\Gamma \vdash V : A$ as \emph{monotone}
functions. Analogously, every error domain is now equipped with a partial
ordering for which the error element is the bottom element, and the map
$\theta_B : \laterhs B \to B$ is now required to be monotone. 
%
Morphisms of error domains are morphisms of the underlying partially-ordered
sets that preserve the error element and $\theta$ map, as was the case in the
previous section.

We model a type precision relation $c : A \ltdyn A'$ as a \emph{monotone
relation}, i.e., a relation $c$ that is upward-closed under the relation on $A'$
and downward-closed under the relation on $A$. We denote such a relation between
$A$ and $A'$ by $c : A \rel A'$. The relation on the poset $A$ is denoted
$r(A)$.
% This extends to products $c_1 \times c_2$ in the obvious way.
Composition of relations on predomains is the usual relational composition
(which is truncated to be propositional).



\subsubsection{Modelling Term Precision}\label{sec:modeling-term-precision}




It remains to define $Fc$, i.e., the action of $F$ on relations.
%
% However, lifting a relation between $A$ and $A'$ to a relation between $\li A$
% and $\li A'$ proves problematic if we allow computations that take different
% numbers of steps to be related. To illustrate the issue, let us define an
% ordering $\semltbad$ between $\li A$ and $\li A'$; we call this the
% \emph{step-insensitive error ordering}. 
%
We note that the graduality property and the axioms of the inequational theory
are independent of the intensional stepping behavior of terms, so our ultimate
notion that interprets term precision will need to be oblivious to stepping as
well.
%
To that end, let us define a \emph{step-insensitive error ordering} $\semltbad$
between $\li A$ and $\li A'$; The ordering is parameterized by an ordering
relation $\le$ between $A$ and $A'$. The definition is by guarded recursion and
is shown in Figure \ref{fig:step-insensitive-error-ordering}. Recall that
$\delta : \li A \to \li A$ is defined by $\delta = \theta_A \circ \nxt$.

Two computations that immediately return $(\eta)$ are related if the returned
values are related in the underlying ordering. The computation that errors
$(\mho)$ is the least term in the ordering. If both sides step (i.e., both sides
are $\theta$), then we allow one time step to pass and compare the resulting
terms (this is where use the relation defined ``later'' and is why we employ
guarded recursion to define the relation).
%
Lastly, if one side steps and the other immediately returns a value, then in
order for these terms to be related, the side that steps must terminate with a
value in some finite number of steps $n$, and that value must be related to the
value returned by the other side. Likewise, if the LHS steps and the RHS
immediately errors, then in order to be related, the LHS must eventually
terminate with error.

\end{comment}


% \section{Denotational Semantics}

First, we define a denotational semantics of types and terms of the
cast calculus by giving a standard monadic denotational semantics in
the cartesian closed category of preorders and monotone functions,
extended to model the primitives of gradual typing: the dynamic type,
errors and type casts. The most interesting part of this semantics is
the construction of the monad and the dynamic type.



\section{Domain-Theoretic Constructions}\label{sec:domain-theory}

In this section, we discuss the fundamental objects of the model into which we will embed
the step-sensitive lambda calculus $\intlc$ and inequational theory. It is important to remember that
the constructions in this section are entirely independent of the syntax described in the
previous section; the notions defined here exist in their own right as purely mathematical
constructs. In the next section, we will link the syntax and semantics via a semantic interpretation
function.

\subsection{The Lift Monad}

When thinking about how to model intensional gradually-typed programs, we should consider
their possible behaviors. On the one hand, we have \emph{failure}: a program may fail
at run-time because of a type error. In addition to this, a program may ``think'',
i.e., take a step of computation. If a program thinks forever, then it never returns a value,
so we can think of the idea of thinking as a way of intensionally modelling \emph{partiality}.

With this in mind, we can describe a semantic object that models these behaviors: a monad
for embedding computations that has cases for failure and ``thinking''.
Previous work has studied such a construct in the setting of the latter, called the lift
monad \cite{mogelberg-paviotti2016}; here, we augment it with the additional effect of failure.

For a type $A$, we define the \emph{lift monad with failure} $\li A$, which we will just call
the \emph{lift monad}, as the following datatype:

\begin{align*}
  \li A &:= \\
  &\eta \colon A \to \li A \\
  &\mho \colon \li A \\
  &\theta \colon \later (\li A) \to \li A
\end{align*}

Unless otherwise mentioned, all constructs involving $\later$ or $\fix$
are understood to be with respect to a fixed clock $k$. So for the above, we really have for each
clock $k$ a type $\li^k A$ with respect to that clock.

Formally, the lift monad $\li A$ is defined as the solution to the guarded recursive type equation

\[ \li A \cong A + 1 + \later \li A. \]

An element of $\li A$ should be viewed as a computation that can either (1) return a value (via $\eta$),
(2) raise an error and stop (via $\mho$), or (3) think for a step (via $\theta$).
%
Notice there is a computation $\fix \theta$ of type $\li A$. This represents a computation
that thinks forever and never returns a value.

Since we claimed that $\li A$ is a monad, we need to define the monadic operations
and show that they respect the monadic laws. The return is just $\eta$, and extend
is defined via by guarded recursion by cases on the input.
% It is instructive to give at least one example of a use of guarded recursion, so
% we show below how to define extend:
% TODO
%
%
Verifying that the monadic laws hold requires \lob-induction and is straightforward.

\begin{comment}
% Since this mentions the ordering on B, it should be introduced after introducing predomains.
The lift monad has the following universal property. Let $f$ be a function from $A$ to $B$,
where $B$ is a $\later$-algebra, i.e., there is $\theta_B \colon \later B \to B$.
Further suppose that $B$ is also an ``error-algebra'', that is, there is an error element
$\mho_B$ such that $\mho_B \le_B y$ for all $y \in B$.

% Further suppose that $B$ is also an algebra of the
% constant functor $1 \colon \text{Type} \to \text{Type}$ mapping all types to Unit.
% This latter statement amounts to saying that there is a map $\text{Unit} \to B$, so $B$ has a
% distinguished ``error element" $\mho_B \colon B$ such that $\mho_B \le_B y$ for all $y \in B$.

Then there is a unique homomorphism of algebras $f' \colon \li A \to B$ such that
$f' \circ \eta = f$. The function $f'(l)$ is defined via guarded fixpoint by cases on $l$. 
In the $\mho$ case, we simply return $\mho_B$.
In the $\theta(\tilde{l})$ case, we will return

\[\theta_B (\lambda t . (f'_t \, \tilde{l}_t)). \]

Recalling that $f'$ is a guarded fixpoint, it is available ``later'' and by
applying the tick we get a function we can apply ``now''; for the argument,
we apply the tick to $\tilde{l}$ to get a term of type $\li A$.
\end{comment}

%\subsubsection{Model-Theoretic Description}
%We can describe the lift monad in the topos of trees model as follows.


\subsection{Predomains}\label{sec:predomains}

The next important construction is that of a \emph{predomain}. A predomain is intended to
model the notion of error ordering that we want terms to have. Thus, we define a predomain $A$
as a partially-ordered set, which consists of a type which we denote $\ty{A}$ and a reflexive,
transitive, and antisymmetric relation $\le_P$ on $A$.

We define monotone functions between predomain as expected. Given predomains
$A$ and $B$, we write $f \colon A_i \monto A_o$ to indicate that $f$ is a monotone
function from $A_i$ to $A_o$, i.e, for all $a_1 \le_{A_i} a_2$, we have $f(a_1) \le_{A_o} f(a_2)$.
We also define an ordering on monotone functions as
$f \le g$ if for all $a$ in $\ty{A_i}$, we have $f(a) \le_{A_o} g(a)$.

For each type we want to represent, we define a predomain for the corresponding semantic
type. For instance, we define a predomain for natural numbers, a predomain for the
dynamic type, a predomain for functions, and a predomain for the lift monad. We
describe each of these below.

\begin{itemize}
  \item There is a predomain $\Nat$ for natural numbers, where the ordering is equality.
  
  \item There is a predomain $\Dyn$ to represent the dynamic type. The underlying type
  for this predomain is defined by guarded fixpoint to be such that
  $\ty{\Dyn} \cong \mathbb{N}\, +\, \later (\ty{\Dyn} \monto \li \ty{\Dyn})$.
  This definition is valid because the occurrences of Dyn are guarded by the $\later$.
  The ordering is defined via guarded recursion by cases on the argument, using the
  ordering on $\mathbb{N}$ and the ordering on monotone functions described above.

  \item For a predomain $A$, there is a predomain $\li A$ for the ``lift'' of $A$
  using the lift monad. We use the same notation for $\li A$ when $A$ is a type
  and $A$ is a predomain, since the context should make clear which one we are referring to.
  The underling type of $\li A$ is simply $\li \ty{A}$, i.e., the lift of the underlying
  type of $A$.
  The ordering on $\li A$ is the ``step-sensitive error-ordering'' which we describe in
  \ref{subsec:lock-step}.

  \item For predomains $A_i$ and $A_o$, we form the predomain of monotone functions
  from $A_i$ to $A_o$, which we denote by $A_i \To A_o$.

  \item Given a preomain $A$, we can form the predomain $\later A$ whose underlying
  type is $\later \ty{A}$. We define $\tilde{x} \le_{\later A} \tilde{y}$ to be
  $\later_t (\tilde{x}_t \le_A \tilde{y}_t)$.
\end{itemize}



\subsection{Step-Sensitive Error Ordering}\label{subsec:lock-step}

As mentioned, the ordering on the lift of a predomain $A$ is called the
\emph{step-sensitive error-ordering} (also called ``lock-step error ordering''),
the idea being that two computations $l$ and $l'$ are related if they are in
lock-step with regard to their intensional behavior, up to $l$ erroring.
Formally, we define this ordering as follows:

\begin{itemize}
  \item 	$\eta\, x \ltls \eta\, y$ if $x \le_A y$.
  \item 	$\mho \ltls l$ for all $l$ 
  \item   $\theta\, \tilde{r} \ltls \theta\, \tilde{r'}$ if
          $\later_t (\tilde{r}_t \ltls \tilde{r'}_t)$
\end{itemize}

We also define a heterogeneous version of this ordering between the lifts of two
different predomains $A$ and $B$, parameterized by a relation $R$ between $A$ and $B$.

\subsection{Step-Insensitive Bisimilarity Relation}

We define another ordering on $\li A$, called ``step-insensitive bisimilarity"
or ``weak bisimilarity" written $l \bisim l'$.
Intuitively, we say $l \bisim l'$ if they are equivalent ``up to delays''.
We introduce the notation $x \sim_A y$ to mean $x \le_A y$ and $y \le_A x$.
% TODO if A is a poset, then we can just say that x = y
%
The step-insensitive bisimilarity relation is defined by guarded fixpoint as follows:

\begin{align*}
  &\mho \bisim \mho \\
%
  &\eta\, x \bisim \eta\, y \text{ if } 
    x \sim_A y \\
%		
  &\theta\, \tilde{x} \bisim \theta\, \tilde{y} \text{ if } 
    \later_t (\tilde{x}_t \bisim \tilde{y}_t) \\
%	
  &\theta\, \tilde{x} \bisim \mho \text{ if } 
    \theta\, \tilde{x} = \delta^n(\mho) \text { for some $n$ } \\
%	
  &\theta\, \tilde{x} \bisim \eta\, y \text{ if }
    (\theta\, \tilde{x} = \delta^n(\eta\, x))
  \text { for some $n$ and $x : \ty{A}$ such that $x \sim_A y$ } \\
%
  &\mho \bisim \theta\, \tilde{y} \text { if } 
    \theta\, \tilde{y} = \delta^n(\mho) \text { for some $n$ } \\
%	
  &\eta\, x \bisim \theta\, \tilde{y} \text { if }
    (\theta\, \tilde{y} = \delta^n (\eta\, y))
  \text { for some $n$ and $y : \ty{A}$ such that $x \sim_A y$ }
\end{align*}

When both sides are $\eta$, then we ensure that the underlying values are related.
When one side is a $\theta$ and the other is $\eta x$ (i.e., one side steps),
we stipulate that the $\theta$-term runs to $\eta y$ where $x$ is related to $y$.
Similarly when one side is $\theta$ and the other $\mho$.
If both sides step, then we allow one time step to pass and compare the resulting terms.
In this way, the definition captures the intuition of terms being equivalent up to
delays.

It can be shown (by \lob-induction) that the step-sensitive relation is symmetric.
However, it can also be shown that this relation is \emph{not} transitive:
One can prove within Clocked Cubical Type Theory
that if this relation were transitive, then in fact it would be trivial in that
$l \bisim l'$ for all $l, l'$.
This issue will be resolved when we consider the relation's \emph{globalization}.



\subsection{Error Domains}

While value types will be interpreted as predomains, we also need a semantics
for computation types. This will be in the form of \emph{error domains}, of which the
Lift monad is a prototypical example. For a fixed clock $k$, an error domain $A$
consists of a predomain (which we also denote by $A$ when there is no risk of confusion),
along with a bottom element $\mho_A$ and a $\later$-algebra $\theta_A \colon \later^k A \monto A$.

% TODO function space as error domain?

\subsection{Globalization}\label{sec:globalization}

Recall that in the above definitions, any occurrences of $\later$ were with
respect to a fixed clock $k$. Intuitively, this corresponds to a step-indexed set.
It will be necessary to consider the ``globalization'' of these definitions,
i.e., the ``global'' behavior of the type over all potential time steps.
This is accomplished in the type theory by \emph{clock quantification} \cite{atkey-mcbride2013},
whereby given a type $X$ parameterized by a clock $k$, we consider the type
$\forall k. X[k]$. This corresponds to leaving the step-indexed world and passing to
the usual semantics in the category of sets.


\section{Semantics}\label{sec:semantics}

\subsection{Step-indexed Semantics}

We give a semantics to the step-sensitive lambda calculus $\intlc$ we defined
in Section \ref{sec:step-sensitive-lc}.
%
Much of the semantics is similar to a normal call-by-value denotational semantics;
we highlight the differences.
Recall that we will interpret value types as predomains, and computation types
as error domains. Value type contexts $\Gamma = x_1 \colon A_1, \dots, x_n \colon A_n$
will be interpreted as the product $\sem{A_1} \times \cdots \times \sem{A_n}$, and
computation type contexts $\Delta = \Delta_\Sigma , \Delta|_V$ will be interpreted as a pair
$(\delta_\Sigma, \delta_V)$ where $\delta_\Sigma$ is either empty or $\sem{B}$.


The semantics of the dynamic type $\dyn$ is the predomain $\Dyn$ introduced in Section
\ref{sec:predomains}.
%
The interpretation of a value $\hasty {\Gamma} V A$ will be a monotone function from
$\sem{\Gamma}$ to $\sem{A}$. Likewise, a term $\hasty {\Delta} M {\Ret{A}}$ will be interpreted
as a monotone function from $\sem{\Delta}$ to $\sem{\Ret{A}} = \li \sem{A}$.

Recall that $\Dyn$ is isomorphic to $\Nat\, + \later (\Dyn \monto \li \Dyn)$.
Thus, the semantics of $\injnat{\cdot}$ and $\injarr{\cdot}$ are simply the
injections $\inl$ and $\inr$.

The interpretation of $\lda{x}{M}$ works as follows. Recall by the typing rule for
lambda that $\hasty {\cdot, \Gamma, x : A_i} M {\Ret {A_o}}$, so the interpretation of $M$
has type $\{*\} \times (\sem{\Gamma} \times \sem{A_i})$ to $\sem{A_o}$.
The interpretation of lambda is thus a function (in the ambient type theory) that takes
a value $a$ representing the argument and applies it (along with $\gamma$) as argument to
the interpretation of $M$.
%
The interpretation of bind and of application both make use the monadic extend function on $\li A$.
%
The interpretation of case-nat and case-arrow is simply a case inspection on the
interpretation of the scrutinee, which has type $\Dyn$.


\vspace{2ex}


\noindent Types:
\begin{align*}
  \sem{\nat} &= \Nat \\
  \sem{\dyn} &= \Dyn \\
  \sem{A \ra A'} &= \sem{A} \To \sem{A'} \\
  \sem{\later A} &=\, \later \sem{A} \\
  \sem{\Ret A} &= \li \sem{A}
\end{align*}

% Contexts:

% TODO check these, especially the semantics of bind, case-nat, and case-arr
% with respect to their context argument
\noindent Values and terms:
\begin{align*}
  \sem{\zro}         &= \lambda \gamma . 0 \\
  \sem{\suc\, V}     &= \lambda \gamma . (\sem{V}\, \gamma) + 1 \\
  \sem{x \in \Gamma} &= \lambda \gamma . \gamma(x) \\
  \sem{\lda{x}{M}}   &= \lambda \gamma . \lambda a . \sem{M}\, (*,\, (\gamma , a))  \\
  \sem{\injnat{V_n}} &= \lambda \gamma . \inl\, (\sem{V_n}\, \gamma) \\
  \sem{\injarr{V_f}} &= \lambda \gamma . \inr\, (\sem{V_f}\, \gamma) \\[2ex]
  \sem{\nxt\, V}     &= \lambda \gamma . \nxt (\sem{V}\, \gamma) \\
  \sem{\theta}       &= \lambda \gamma . \theta \\
%
  \sem{\err_B}         &= \lambda \delta . \mho \\
  \sem{\ret\, V}       &= \lambda \gamma . \eta\, \sem{V} \\
  \sem{\bind{x}{M}{N}} &= \lambda \delta . \ext {(\lambda x . \sem{N}\, (\delta, x))} {\sem{M}\, \delta} \\
  \sem{V_f\, V_x}      &= \lambda \gamma . \ext {(\lambda f . (\ext {f} {\sem{V_x}\, \gamma}))} {(\sem{V_f}\, \gamma)} \\
  \sem{\casenat{V}{M_{no}}{n}{M_{yes}}}         &= 
    \lambda \delta . \text{case $(\sem{V}\, \delta)$ of} \\ 
    &\quad\quad\quad\quad \alt \inl(n) \to \sem{M_{yes}}(n) \\
    &\quad\quad\quad\quad \alt \inr(\tilde{f}) \to \sem{M_{no}} \\
  \sem{\casearr{V}{M_{no}}{\tilde{f}}{M_{yes}}} &= 
    \lambda \delta . \text{case $(\sem{V}\, \delta)$ of} \\ 
    &\quad\quad\quad\quad \alt \inl(n) \to \sem{M_{no}} \\
    &\quad\quad\quad\quad \alt \inr(\tilde{f}) \to \sem{M_{yes}}(\tilde{f})
\end{align*}

% TODO
% \noindent Relations:
% \begin{align*}
% %
% \end{align*}


\begin{comment}
\subsection{Global Semantics}

Having defined the above step-indexed semantics, we now pass to a ``global''
semantics that does not involve any step-indexing. The resulting semantics is still
intensional in that terms that produce the same value in a different number of steps
will be distinct.
We define 

\[ \semgl{\cdot} = \forall (\kpa \colon \Clock) .\, \sem{\cdot}[\kpa]. \]

Note that for a term $M$ of type $\Ret{A}$, the semantics has type
$\sem{M} \colon \sem{\Delta} \monto \sem{\Ret{A}} = \sem{\Delta} \monto \li \sem{A}$.
In the case where $\Delta$ is the empty context, i.e., when $M$ is a closed term,
then this is equivalent to $\li \sem{A}$.
Then the global semantics in this case is $\forall \kappa . \liclk {\kappa} \sem{A}$.
We can show in Clocked Cubical Type Theory this type satisfies a coinductive unfolding property

\[ \forall \kappa . \li \sem{A} \cong \sem{A} + 1\, + (\forall \kappa.\li \sem{A}). \]
\end{comment}


% Machines

% We then define a relation $\Dwn^n$ between terms of type $T$ and $\Machine {\sem{T}}$ by

% \subsection{Extensional Collapse}


% \subsection{Relational Semantics}

% \subsubsection{Term Precision via the Step-Sensitive Error Ordering}
% Homogeneous vs heterogeneous term precision

% \subsection{Logical Relations Semantics}


%% \section{Extending the Semantics to Precision}\label{sec:gtlc-precision}

In this section, we extend the set-theoretic semantics for terms given in
the previous section to a semantics for the type and term precision relations
of the gradually-typed lambda calculus. We first introduce the type and term precision
relations, then show how to give them a semantics using SGDT.

% TODO mention intensional syntax


\subsection{Term Precision for GTLC}\label{sec:gtlc-term-precision-axioms}

% ---------------------------------------------------------------------------------------
% ---------------------------------------------------------------------------------------

%\subsubsection{Term Precision}\label{sec:term-precision}

We allow for a \emph{heterogeneous} term precision judgment on values $V$ of type
$A$ and $V'$ of type $A'$ provided that $A \ltdyn A'$ holds. Likewise, for producers,
if $M$ has type $A$ and $M'$ has type $A'$, we can form the judgment that $M \ltdyn M'$.
We use the same notation for the precision relation on both values and producers.

% Type precision contexts
In order to deal with open terms, we will need the notion of a type precision \emph{context}, which we denote
$\gamlt$. This is similar to a normal context but instead of mapping variables to types,
it maps variables $x$ to related types $A \ltdyn A'$, where $x$ has type $A$ in the left-hand term
and $A'$ in the right-hand term. We may also write $x : d$ where $d : A \ltdyn A'$ to indicate this.

% An equivalent way of thinking of type precision contexts is as a pair of ``normal" typing
% contexts $\Gamma, \Gamma'$ with the same domain such that $\Gamma(x) \ltdyn \Gamma'(x)$ for
% each $x$ in the domain.
% We will write $\gamlt : \Gamma \ltdyn \Gamma'$ when we want to emphasize the pair of contexts.
% Conversely, if we are given $\gamlt$, we write $\gamlt_l$ and $\gamlt_r$ for the normal typing contexts on each side.

An equivalent way of thinking of a type precision context $\gamlt$ is as a
pair of ``normal" typing contexts, $\gamlt_l$ and $\gamlt_r$, with the same
domain and such that $\gamlt_l(x) \ltdyn \gamlt_r(x)$ for each $x$ in the domain.
We will write $\gamlt : \gamlt_l \ltdyn \gamlt_r$ when we want to emphasize the pair of contexts.

As with type precision derivations, we write $\Gamma$ to mean the ``reflexivity" type precision context
$\Gamma : \Gamma \ltdyn \Gamma$.
Concretely, this consists of reflexivity type precision derivations $\Gamma(x) \ltdyn \Gamma(x)$ for
each $x$ in the domain of $\Gamma$.

Furthermore, we write $\gamlt_1 \relcomp \gamlt_2$ to denote the ``composition'' of $\gamlt_1$ and $\gamlt_2$
--- that is, the precision context whose value at $x$ is the type precision derivation
$\gamlt_1(x) \relcomp \gamlt_2(x)$. This of course assumes that each of the type precision
derivations is composable, i.e., that the RHS of $\gamlt_1(x)$ is the same as the left-hand side of $\gamlt_2(x)$.

% We define the same for computation type precision contexts $\deltalt_1$ and $\deltalt_2$,
% provided that both the computation type precision contexts have the same ``shape'', which is defined as
% (1) either the stoup is empty in both, or the stoup has a hole in both, say $\hole{d}$ and $\hole{d'}$
% where $d$ and $d'$ are composable, and (2) their value type precision contexts are composable as described above.

The rules for term precision come in two forms. We first have the \emph{congruence} rules,
one for each term constructor. These assert that the term constructors respect term precision.
The congruence rules are as follows:

\begin{mathpar}

  \inferrule*[right = Var]
    { c : A \ltdyn B \and \gamlt(x) = (A, B) } 
    { \etmprec {\gamlt} x x c }

  \inferrule*[right = Zro]
    { } {\etmprec \gamlt \zro \zro \nat }

  \inferrule*[right = Suc]
    { \etmprec \gamlt V {V'} \nat } {\etmprec \gamlt {\suc\, V} {\suc\, V'} \nat}

  \inferrule*[right = MatchNat]
  {\etmprec \gamlt V {V'} \nat \and 
    \etmprec \deltalt M {M'} d \and \etmprec {\deltalt, n : \nat} {N} {N'} d}
  {\etmprec \deltalt {\matchnat {V} {M} {n} {N}} {\matchnat {V'} {M'} {n} {N'}} d}

  \inferrule*[right = Lambda]
    { c_i : A_i \ltdyn A'_i \and 
      c_o : A_o \ltdyn A'_o \and 
      \etmprec {\gamlt, x : c_i} {M} {M'} {c_o} } 
    { \etmprec \gamlt {\lda x M} {\lda x {M'}} {(c_i \ra c_o)} }

  \inferrule*[right = App]
    { c_i : A_i \ltdyn A'_i \and
      c_o : A_o \ltdyn A'_o \\\\
      \etmprec \gamlt {V_f} {V_f'} {(c_i \ra c_o)} \and
      \etmprec \gamlt {V_x} {V_x'} {c_i}
    } 
    { \etmprec {\gamlt} {V_f\, V_x} {V_f'\, V_x'} {{c_o}}}

  \inferrule*[right = Ret]
    {\etmprec {\gamlt} V {V'} c}
    {\etmprec {\gamlt} {\ret\, V} {\ret\, V'} {c}}

  \inferrule*[right = Bind]
    {\etmprec {\gamlt} {M} {M'} {c} \and 
     \etmprec {\gamlt, x : c} {N} {N'} {d} }
    {\etmprec {\gamlt} {\bind {x} {M} {N}} {\bind {x} {M'} {N'}} {d}}
\end{mathpar}

We then have additional equational axioms, including $\beta$ and $\eta$ laws, and
rules characterizing upcasts as least upper bounds, and downcasts as greatest lower bounds.
For the sake of familiarity, we formulate the cast rules using arbitrary casts; later we
will state the analogous versions for the version of the calculus without arbitrary casts.

We write $M \equidyn N$ to mean that both $M \ltdyn N$ and $N \ltdyn M$.

\begin{mathpar}
  \inferrule*[right = $\err$]
    {\phasty {\Gamma} M B }
    {\etmprec {\Gamma} {\err_B} M B}

  \inferrule*[right = $\beta$-fun]
    { \phasty {\Gamma, x : A_i} M {A_o} \and
      \vhasty {\Gamma} V {A_i} } 
    { \etmequidyn {\Gamma} {(\lda x M)\, V} {M[V/x]} {A_o} }

  \inferrule*[right = $\eta$-fun]
    { \vhasty {\Gamma} {V} {A_i \ra A_o} } 
    { \etmequidyn \Gamma {\lda x (V\, x)} V {A_i \ra A_o} }

  \inferrule*[right = UpR]
    { c : A \ltdyn B \and d : B \ltdyn C \and 
      \etmprec {\gamlt} {M} {N} {c} } 
    { \etmprec {\gamlt} {M} {\up {B} {C} N} {c \circ d}  }

  \inferrule*[right = UpL]
    { c : A \ltdyn B \and d : B \ltdyn C \and
      \etmprec {\gamlt} {M} {N} {c \circ d} } 
    { \etmprec {\gamlt} {\up {A} {B} M} {N} {d} }

  \inferrule*[right = DnL]
    { c : A \ltdyn B \and d : B \ltdyn C \and
      \etmprec {\gamlt} {M} {N} {d} } 
    { \etmprec {\gamlt} {\dn {A} {B} M} {N} {c \circ d} }

  \inferrule*[right = DnR]
    { c : A \ltdyn B \and d : B \ltdyn C \and
      \etmprec {\gamlt} {M} {N} {c \circ d} } 
    { \etmprec {\gamlt} {M} {\dn {B} {C} N} {c} }
\end{mathpar}

% TODO explain the least upper bound/greatest lower bound rules
The rules UpR, UpL, DnL, and DnR were introduced in \cite{new-licata18} as a means
of cleanly axiomatizing the intended behavior of casts in a way that
doesn't depend on the specific constructs of the language.
Intuitively, rule UpR says that the upcast of $M$ is an upper bound for $M$
in that $M$ may error more, and UpL says that the upcast is the \emph{least}
such upper bound, in that it errors more than any other upper bound for $M$.
Conversely, DnL says that the downcast of $M$ is a lower bound, and DnR says
that it is the \emph{greatest} lower bound.
% These rules provide a clean axiomatization of the behavior of casts that doesn't
% depend on the specific constructs of the language.



\subsection{Semantics for Precision}

As a first attempt at giving a semantics to the ordering, we could try to model types as
sets equipped with an ordering that models term precision. Since term precision is reflexive
and transitive, and since we identify terms that are equi-precise, we choose to model types
as partially-ordered sets. We model the term precision ordering $M \ltdyn N : A \ltdyn B$ as an
ordering relation between the posets denoted by $A$ and $B$.

However, it turns out that modeling term precision by a relation defined by guarded fixpoint
is not as straightforward as one might hope.
A first attempt might be to define an ordering $\semltbad$ between $\li X$ and $\li Y$
that allows for computations that may take different numbers of steps to be related.
The relation is parameterized by a relation $\le$ between $X$ and $Y$, and is defined
by guarded fixpoint as follows:
% simultaneously captures the notions of error approximation and equivalence up to stepping behavior:

\begin{align*}
  &\eta\, x \semltbad \eta\, y \text{ if } 
    x \semlt y \\
%		
  &\mho \semltbad l \\
%
  &\theta\, \tilde{l} \semltbad \theta\, \tilde{l'} \text{ if } 
    \later_t (\tilde{l}_t \semltbad \tilde{l'}_t) \\
%	
  &\theta\, \tilde{l} \semltbad \mho \text{ if } 
    \theta\, \tilde{l} = \delta^n(\mho) \text { for some $n$ } \\
%	
  &\theta\, \tilde{l} \semltbad \eta\, y \text{ if }
    (\theta\, \tilde{l} = \delta^n(\eta\, x))
  \text { for some $n$ and $x : \ty{X}$ such that $x \le y$ } \\
%
  &\mho \semltbad \theta\, \tilde{l'} \text { if } 
    \theta\, \tilde{l'} = \delta^n(\mho) \text { for some $n$ } \\
%	
  &\eta\, x \semltbad \theta\, \tilde{l'} \text { if }
    (\theta\, \tilde{l'} = \delta^n (\eta\, y))
  \text { for some $n$ and $y : \ty{Y}$ such that $x \le y$ }
\end{align*}

Two computations that immediately return $(\eta)$ are related if the underlying
values are related in the underlying ordering. 
%
The computation that errors $(\mho)$ is below everything else.
%
If both sides step (i.e., both sides are $\theta$),
then we allow one time step to pass and compare the resulting terms.
(This is where use the relation defined ``later''.)
%
Lastly, if one side steps and the other returns a value, the side that steps should
terminate with a value in some finite number of steps $n$, and that value should
be related to the value returned by the other side.
Likewise, if one side steps and the other errors, then the side that steps
should terminate with error.

The problem with this definition is that the resulting relation is \emph{provably} not
transitive: it can be shown (in Clocked Cubical Type Theory) that if $R$ is a
relation on $\li X$ satisfying three specific properties, one of which is
transitivity, then that relation is trivial.
(The other two properties are that the relation is a congruence with respect to $\theta$,
and that the relation is closed under delays $\delta = \theta \circ \nxt$ on either side.)
Since the above relation \emph{does} satisfy the other two properties, we conclude
that it must not be transitive.

%But having a non-transitive relation to model term precision presents a problem
%for...

We are therefore led to wonder whether we can formulate a version of the relation
that \emph{is} transitive.
It turns out that we can, by sacrificing another of the three properties from
the above lemma. Namely, we give up on closure under delays. Doing so, we end up
with a \emph{lock-step} error ordering, where, roughly speaking, in order for
computations to be related, they must have the same stepping behavior.
%
We then formulate a separate relation, \emph{weak bisimilarity}, that relates computations
that are extensionally equal and may only differ in their stepping behavior.

% As a result, we instead separate the semantics of term precision into two relations:
% an intensional, step-sensitive \emph{error ordering} and a \emph{bisimilarity relation}.


\subsubsection{Double Posets}\label{sec:predomains}

As discussed above, there are two relations that we would like to define
in the semantics: a step-sensitive error ordering, and weak bisimilarity of computations.
%
The semantic objects that interpret our types should therefore be equipped with
two relations. We call these objects ``double posets''.
A double poset $A$ is a set with two relations: an partial order $\semlt_A$ on $A$, and
a reflexive, symmetric relation $\bisim_A$ on $A$.
We write the underling set of $A$ as $\ty{A}$.

We define morphisms of double posets as functions that preserve both
the ordering and the bisimilarity relation. Given double posets
$A$ and $B$, we write $f \colon A \monto B$ to indicate that $f$ is a morphism
from $A$ to $B$, i.e, the following hold:
(1) for all $a_1 \semlt_A a_2$, we have $f(a_1) \semlt_{B} f(a_2)$, and
(2) for all $a_1 \bisim_A a_2$, we have $f(a_1) \bisim_{B} f(a_2)$.


%%%%% RESUME HERE

We define an ordering on morphisms of double posets as
$f \le g$ if for all $a$ in $\ty{A_i}$, we have $f(a) \le_{A_o} g(a)$,
and similarly bisimilarity extends to morphisms via
$f \bisim g$ if for all $a$ in $\ty{A_i}$, we have $f(a) \bisim_{A_o} g(a)$.

For each type we want to represent, we define a double poset for the corresponding semantic
type. For instance, we define a double poset for natural numbers, for the
dynamic type, for functions, and for the lift monad. We
describe each of these below.

\begin{itemize}
  \item There is a double poset $\Nat$ for natural numbers, where the ordering and the
  bisimilarity relations are both equality.
  
  % TODO explain that there is a theta operator for posets?
  \item There is a double poset $\Dyn$ to represent the dynamic type. The underlying type
  for this double poset is defined by guarded fixpoint to be such that
  $\ty{\Dyn} \cong \mathbb{N}\, +\, \later (\ty{\Dyn} \monto \li \ty{\Dyn})$.
  This definition is valid because the occurrences of Dyn are guarded by the $\later$.
  The ordering is defined via guarded recursion by cases on the argument, using the
  ordering on $\mathbb{N}$ and the ordering on monotone functions described above.

  \item For a double poset $A$, there is a double poset $\li A$ for the ``lift'' of $A$
  using the lift monad. We use the same notation for $\li A$ when $A$ is a type
  and $A$ is a double poset, since the context should make clear which one we are referring to.
  The underling type of $\li A$ is simply $\li \ty{A}$, i.e., the lift of the underlying
  type of $A$.
  The ordering on $\li A$ is the ``lock-step error-ordering'' which we describe in
  \ref{subsec:lock-step}. The bismilarity relation is the ``weak bisimilarity''
  described in Section \ref{}

  \item For double posets $A_i$ and $A_o$, we form the double poset of monotone functions
  from $A_i$ to $A_o$, which we denote by $A_i \To A_o$.

  \item Given a double poset $A$, we can form the double poset $\later A$ whose underlying
  type is $\later \ty{A}$. We define $\tilde{x} \le_{\later A} \tilde{y}$ to be
  $\later_t (\tilde{x}_t \le_A \tilde{y}_t)$.
\end{itemize}

\subsubsection{Step-Sensitive Error Ordering}\label{subsec:lock-step}

As mentioned, the ordering on the lift of a double poset $A$ is called the
\emph{step-sensitive error-ordering} (also called ``lock-step error ordering''),
the idea being that two computations $l$ and $l'$ are related if they are in
lock-step with regard to their intensional behavior, up to $l$ erroring.
Formally, we define this ordering as follows:

\begin{itemize}
  \item 	$\eta\, x \ltls \eta\, y$ if $x \le_A y$.
  \item 	$\mho \ltls l$ for all $l$ 
  \item   $\theta\, \tilde{r} \ltls \theta\, \tilde{r'}$ if
          $\later_t (\tilde{r}_t \ltls \tilde{r'}_t)$
\end{itemize}

We also define a heterogeneous version of this ordering between the lifts of two
different double posets $A$ and $B$, parameterized by a relation $R$ between $A$ and $B$.

\subsubsection{Weak Bisimilarity Relation}

For a double poset $A$, we define a relation on $\li A$, called ``weak bisimilarity",
written $l \bisim l'$. Intuitively, we say $l \bisim l'$ if they are equivalent
``up to delays''.
% We introduce the notation $x \sim_A y$ to mean $x \le_A y$ and $y \le_A x$.
% TODO if A is a poset, then we can just say that x = y
%
The weak bisimilarity relation is defined by guarded fixpoint as follows:

\begin{align*}
  &\mho \bisim \mho \\
%
  &\eta\, x \bisim \eta\, y \text{ if } 
    x \bisim_A y \\
%		
  &\theta\, \tilde{x} \bisim \theta\, \tilde{y} \text{ if } 
    \later_t (\tilde{x}_t \bisim \tilde{y}_t) \\
%	
  &\theta\, \tilde{x} \bisim \mho \text{ if } 
    \theta\, \tilde{x} = \delta^n(\mho) \text { for some $n$ } \\
%	
  &\theta\, \tilde{x} \bisim \eta\, y \text{ if }
    (\theta\, \tilde{x} = \delta^n(\eta\, x))
  \text { for some $n$ and $x : \ty{A}$ such that $x \sim_A y$ } \\
%
  &\mho \bisim \theta\, \tilde{y} \text { if } 
    \theta\, \tilde{y} = \delta^n(\mho) \text { for some $n$ } \\
%	
  &\eta\, x \bisim \theta\, \tilde{y} \text { if }
    (\theta\, \tilde{y} = \delta^n (\eta\, y))
  \text { for some $n$ and $y : \ty{A}$ such that $x \sim_A y$ }
\end{align*}

When both sides are $\eta$, then we ensure that the underlying values are bisimilar
in the underlying bisimilarity relation on $A$.
When one side is a $\theta$ and the other is $\eta x$ (i.e., one side steps),
we stipulate that the $\theta$-term runs to $\eta y$ where $x$ is related to $y$.
Similarly when one side is $\theta$ and the other $\mho$.
If both sides step, then we allow one time step to pass and compare the resulting terms.
In this way, the definition captures the intuition of terms being equivalent up to
delays.

It can be shown (by \lob-induction) that the step-sensitive relation is symmetric.
However, it can also be shown that this relation is \emph{not} transitive:
The argument is the same as that used to show that the step-insensitive error
ordering $\semltbad$ described above is not transitive. Namely, we show that
if it were transitive, then it would have to be trivial in that $l \bisim l'$ for all $l, l'$.
that if this relation were transitive, then in fact it would be trivial in that
%This issue will be resolved when we consider the relation's \emph{globalization}.

\subsection{The Cast Rules}

Unfortunately, the four cast rules defined above do not hold in
the intensional setting where we are tracking the steps taken by terms.
The source of the problem is that the downcast from the dynamic type to
a function involves a delay, i.e., a $\theta$.
So in order to keep the other term in lock-step, we need to insert a ``delay"
that is extensionally equivalent to the identity function.
More concretely, consider a simplified version of the DnL rule shown below:

\begin{mathpar}
  \inferrule*{M \ltdyn_i N : B}
             {\dnc{c}{M} \ltdyn_i N : c}
\end{mathpar}

If $c$ is inj-arr, then when we downcast $M$ from $dyn$ to $\dyntodyn$,
semantically this will involve a $\theta$ because the value of type $dyn$
in the semantics will contain a \emph{later} function $\tilde{f}$.
Thus, in order for the right-hand side to be related to the downcast,
we need to insert a delay on the right.
%
The need for delays affects the cast rules involving upcasts as well, because
the upcast for functions involves a downcast on the domain:

\[ \up{A_i \ra A_o}{B_i \ra B_o}{M} \equiv \lambda (x : B_i). \up{A_o}{B_o}(M\, (\dn {A_i}{B_i} x)). \]

Thus, the correct versions of the cast rules involve delays on the side that was not casted.


% Delays for function types and for inj-arr(c)


\subsubsection{Perturbations}

We can describe precisely how the delays are inserted for any type precision
derivation $c$.

To do so, we first define simultaneously an inductive type of \emph{perturbations}
for embeddings $\perte$ and for projections $\pertp$ by the following rules:

\begin{mathpar}

\inferrule{}{\id : \perte A}

\inferrule{}{\id : \pertp A}

\inferrule
  {\delta_c : \pertp A \and \delta_d : \perte B}
  {\delta_c \ra \delta_d : \perte (A \ra B)}

\inferrule
  {\delta_c : \perte A \and \delta_d : \pertp B}
  {\delta_c \ra \delta_d : \pertp (A \ra B)}

\inferrule
  {\delta_\nat : \perte \nat \and \delta_f : \perte (\dyntodyn)}
  {\pertdyn{\delta_\nat}{\delta_f} : \perte \dyn}

\inferrule
  {\delta_\nat : \pertp \nat \and \delta_f : \pertp (\dyntodyn)}
  {\pertdyn{\delta_\nat}{\delta_f} : \pertp \dyn}

\end{mathpar}

The structure of embedding perturbations is designed to follow the structure
of the corresponding embeddings, and likewise for the projection perturbations.
Thus, in the function case, an embedding perturbation consists of a \emph{projection}
perturbation for the domain and an \emph{embedding} perturbation for the codomain.
The opposite holds for the projection perturbation for functions.

Another way in which the two kinds of perturbations differ is that there is an additional
projection perturbation for delaying $\delaypert{\delta}$.
This corresponds to the actual delay term $\delta = \theta \circ \nxt$ in the semantics,
and it is the generator/source of all non-trivial perturbations.

Given a perturbation $\delta$, we can turn it into a term, which we also write as
$\delta$ unless there is opportunity for confusion.



%% \section{Unary Canonicity}
Before discussing graduality, we seek to prove its ``unary'' analogue.
Namely, instead of considering inequality between terms, we start by considering equality.


\section{Graduality}\label{sec:graduality}
The main theorem we would like to prove is the following:

\begin{theorem}[Graduality]
  If $\cdot \vdash M \ltdyn N : \nat$, then
  \begin{enumerate}
    \item If $N = \mho$, then $M = \mho$
    \item If $N = `n$, then $M = \mho$ or $M = `n$
    \item If $M = V$, then $N = V$
  \end{enumerate}
\end{theorem}




% \section{Abstract Categorical Models of Graduality}

First, what is a categorical model of call-by-push-value? We will use
the following notion as our basic structure\footnote{There are models
that are closer to syntax e.g., which distinguish between value types
and contexts but this suffices for all of the models we consider}:
\begin{enumerate}
\item A cartesian category $\mathcal V$
\item A category $\mathcal E$
\item An action of $\mathcal V^{op}$ (with the $\mathcal V$ cartesian
  product as monoidal structure) on $\mathcal E$. We write this as
  exponentiation $B^A$.

  This means we have 
  \[ \alpha : B^{A_1 \times A_2} \cong (B^{A_1})^{A^2} \]
  and
  \[ i : B^1 \cong B \]
  satisfying coherence isomorphisms
\item A functor $U : \mathcal E \to \mathcal V$ that ``preserves
  powering'' in that every $U(B^A)$ is an exponential of $UB$ by $A$
  and that $U\alpha$ and $Ui$ are mapped to the canonical isomorphisms
  for exponentials.
\item A left adjoint $F \dashv U$
\item Distributive finite coproducts in $\mathcal V$
\end{enumerate}

\begin{example}
  Given a strong monad $T$ on a bicartesian closed category $\mathcal
  V$, we can extend this to a CBPV model by defining $\mathcal E$ to
  be the category $\mathcal V^T$ of algebras
\end{example}

The above definition can interpreted in any compact closed equipment
(if someone were to figure out a definition for a compact closed
equipment, that is,\ldots). Then we can get a model of a form of GTT
by taking a CBPV object in the equipment of \emph{reflexive graph
categories}. Since reflexive graphs form a topos we can get at this by
interpreting the above definition \emph{internally} to the topos of
reflexive graphs. Essentially what this means is that everything above
has a ``vertex'' component and an ``edge'' component, so we get a
cartesian category $\mathcal V_v$ which we think of as the value types
and pure functions but we also get a cartesian category $\mathcal V_e$
which we think of as the ``value edges'' and ``squares''.

We will work in ``locally thin'' models where there is at most one
square with a given boundary.

In addition to the ordinary universal properties above, when working
with reflexive graph models we also have access to new notions of
universal property that relate the ``function'' morphisms to the
``edges''.

Let $f : \mathcal V_v(A_i, A_o)$ and $R : \mathcal V_e(A_o,A')$ then a
\emph{left restriction} of $R$ along $f$ consists of
\begin{enumerate}
\item $R' : \mathcal V_e(A_i,A')$
\item $S \Rightarrow_{g,h} R'$ naturally isomorphic to $S \Rightarrow_{fg,h}R$
\end{enumerate}
In relational semantics, $R(a_o,a')$ is a relation and the left
restriction along $f$ is the relation $R(f(a_i), a')$.

Then we can formulate our value edges as follows:
\begin{enumerate}
\item We have a thin subcategory $\mathcal V_u$ of $\mathcal V_v$, of ``upcasts''
\item $\mathcal V$ has all left restrictions of edges along upcasts
\end{enumerate}

And our computation edges as follows:
\begin{enumerate}
\item We have a thin subcategory $\mathcal E_d$ of $\mathcal E_v$ of ``downcasts''
\item $\mathcal E$ has all right restrictions of edges along downcasts
\end{enumerate}

We also want something like
\[ F_c : \mathcal V_u^{op} \to \mathcal E_d \]
\[ U_c : \mathcal E_d^{op} \to \mathcal V_u \]
which ensures that if $R$ is a value edge equivalent to $A(u,-)$ then
\[ F(R) = F(A(u,-)) = (F A)(-,F u) \]



%\section{A Denotational Model for Intensional Gradual Typing}

In this section we model intensional gradual typing in a suitable double category.

We construct a (thin) double category $\mathsf{IGTT}$ as follows.

% First, we fix a set $D$ with a partial order $\le_D$ and bisimilarity relation $\bisim_D$.
% This is intended to model the dynamic type. Now we define the double category:

\begin{itemize}
  \item \textbf{Objects}: An object consists of the following data:
    \begin{itemize}
        \item A double poset $X$, i.e., a set $X$ equipped with a partial order $\le_X$
        and a reflexive, symmetric ``bisimilarity'' relation $\bisim_X$.
        \item Two commutative monoids of perturbations $P_V$ and $P_C$ with homomorphisms
        \begin{align*}
        \ptb_V &: P_V \to \{ f : X \to_m X \mid f \bisim \id \} \\
        \ptb_C &: P_C \to \{ f : X \to_m \li X \mid f \bisim \eta \}
        \end{align*}
        (where composition in the latter monoid of functions is given by Kleisli composition).

    \end{itemize}

  \item \textbf{Vertical arrows}: An vertical arrow from $(X, P_V^X, P_C^X)$ to $(Y, P_V^Y, P_C^Y)$ is
  a function $f : X \to Y$ that is \emph{monotone} (preserves ordering) and preserves the bisimilarity relation.
  % that preserves ordering and bisimilarity.
  
  \item \textbf{Horizontal arrows}: A horizontal arrow from $(X, P_V^X, P_C^X)$ to $(Y, P_V^Y, P_C^Y)$
  consists of:
  \begin{itemize}
    \item A relation $R : X \nrightarrow Y$ that is antitone with respect to $\le_X$ and
    monotone with respect to $\le_Y$.
    \item An embedding $e_{XY} : X \to_m Y$ preserving ordering and bisimilarity.
    \item A projection $p_{XY} : Y \to_m \li X$ preserving ordering and bisimilarity.
  \end{itemize}
  
  such that (1) $R$ is \emph{quasi-representable} by $e_{XY}$ and $p_{XY}$, and 
  (2) $R$ satisfies the \emph{push-pull} property.
  
  \vspace{3ex}
  The former means that there are distinguished elements $\delta^{l,e} \in P_V^X$, $\delta^{r,e} \in P_V^Y$, 
  $\delta^{l,p} \in P_C^X$ and $\delta^{r,p} \in P_C^Y$ such that the following squares commute:

  \begin{center}
    \begin{tabular}{ c | c } 
        \hline
        \hspace{3em} 
        % UpL
        % https://q.uiver.app/#q=WzAsNCxbMCwwLCJYIl0sWzAsMSwiWSJdLFsxLDAsIlkiXSxbMSwxLCJZIl0sWzAsMiwiUiIsMCx7InN0eWxlIjp7ImJvZHkiOnsibmFtZSI6ImJhcnJlZCJ9LCJoZWFkIjp7Im5hbWUiOiJub25lIn19fV0sWzEsMywiXFxsZV9ZIiwyLHsic3R5bGUiOnsiYm9keSI6eyJuYW1lIjoiYmFycmVkIn0sImhlYWQiOnsibmFtZSI6Im5vbmUifX19XSxbMCwxLCJlX3tYWX0iLDJdLFsyLDMsIlxccHRiX1ZeWShcXGRlbHRhXntyLGV9KSJdLFs2LDcsIlxcbHRkeW4iLDEseyJzaG9ydGVuIjp7InNvdXJjZSI6MjAsInRhcmdldCI6MjB9LCJzdHlsZSI6eyJib2R5Ijp7Im5hbWUiOiJub25lIn0sImhlYWQiOnsibmFtZSI6Im5vbmUifX19XV0=
        \begin{tikzcd}[ampersand replacement=\&]
            X \& Y \\
            Y \& Y
            \arrow["R", "\shortmid"{marking}, no head, from=1-1, to=1-2]
            \arrow["{\le_Y}"', "\shortmid"{marking}, no head, from=2-1, to=2-2]
            \arrow[""{name=0, anchor=center, inner sep=0}, "{e_{XY}}"', from=1-1, to=2-1]
            \arrow[""{name=1, anchor=center, inner sep=0}, "{\ptb_V^Y(\delta^{r,e})}", from=1-2, to=2-2]
            \arrow["\ltdyn"{description}, draw=none, from=0, to=1]
        \end{tikzcd} & 
        % UpR
        % https://q.uiver.app/#q=WzAsNCxbMCwwLCJYIl0sWzAsMSwiWCJdLFsxLDAsIlgiXSxbMSwxLCJZIl0sWzAsMSwiXFxwdGJfVl5YKFxcZGVsdGFee2wsZX0pIiwyXSxbMiwzLCJlX3tYWX0iXSxbMSwzLCJSIiwyLHsic3R5bGUiOnsiYm9keSI6eyJuYW1lIjoiYmFycmVkIn0sImhlYWQiOnsibmFtZSI6Im5vbmUifX19XSxbMCwyLCJcXGxlX1giLDAseyJzdHlsZSI6eyJib2R5Ijp7Im5hbWUiOiJiYXJyZWQifSwiaGVhZCI6eyJuYW1lIjoibm9uZSJ9fX1dLFs0LDUsIlxcbHRkeW4iLDEseyJzaG9ydGVuIjp7InNvdXJjZSI6MjAsInRhcmdldCI6MjB9LCJzdHlsZSI6eyJib2R5Ijp7Im5hbWUiOiJub25lIn0sImhlYWQiOnsibmFtZSI6Im5vbmUifX19XV0=
        \begin{tikzcd}[ampersand replacement=\&]
            X \& X \\
            X \& Y
            \arrow[""{name=0, anchor=center, inner sep=0}, "{\ptb_V^X(\delta^{l,e})}"', from=1-1, to=2-1]
            \arrow[""{name=1, anchor=center, inner sep=0}, "{e_{XY}}", from=1-2, to=2-2]
            \arrow["R"', "\shortmid"{marking}, no head, from=2-1, to=2-2]
            \arrow["{\le_X}", "\shortmid"{marking}, no head, from=1-1, to=1-2]
            \arrow["\ltdyn"{description}, draw=none, from=0, to=1]
        \end{tikzcd} \\
        \hline
        % DnR
        % https://q.uiver.app/#q=WzAsNCxbMCwwLCJYIl0sWzEsMCwiWSJdLFswLDEsIkxYIl0sWzEsMSwiTFgiXSxbMCwyLCJcXHB0Yl9DXlgoXFxkZWx0YV57bCxwfSkiLDJdLFsxLDMsInBfe1hZfSJdLFswLDEsIlIiLDAseyJzdHlsZSI6eyJib2R5Ijp7Im5hbWUiOiJiYXJyZWQifSwiaGVhZCI6eyJuYW1lIjoibm9uZSJ9fX1dLFsyLDMsIlxcbGVfe0xYfSIsMix7InN0eWxlIjp7ImJvZHkiOnsibmFtZSI6ImJhcnJlZCJ9LCJoZWFkIjp7Im5hbWUiOiJub25lIn19fV0sWzQsNSwiXFxsdGR5biIsMSx7InNob3J0ZW4iOnsic291cmNlIjoyMCwidGFyZ2V0IjoyMH0sInN0eWxlIjp7ImJvZHkiOnsibmFtZSI6Im5vbmUifSwiaGVhZCI6eyJuYW1lIjoibm9uZSJ9fX1dXQ==
        \begin{tikzcd}[ampersand replacement=\&]
            X \& Y \\
            LX \& LX
            \arrow[""{name=0, anchor=center, inner sep=0}, "{\ptb_C^X(\delta^{l,p})}"', from=1-1, to=2-1]
            \arrow[""{name=1, anchor=center, inner sep=0}, "{p_{XY}}", from=1-2, to=2-2]
            \arrow["R", "\shortmid"{marking}, no head, from=1-1, to=1-2]
            \arrow["{\le_{LX}}"', "\shortmid"{marking}, no head, from=2-1, to=2-2]
            \arrow["\ltdyn"{description}, draw=none, from=0, to=1]
        \end{tikzcd} & 
            \hspace{3em} 
            % DnL
            % https://q.uiver.app/#q=WzAsNCxbMCwwLCJZIl0sWzEsMCwiWSJdLFswLDEsIkxYIl0sWzEsMSwiTFkiXSxbMCwxLCJcXGxlX1kiLDAseyJzdHlsZSI6eyJib2R5Ijp7Im5hbWUiOiJiYXJyZWQifSwiaGVhZCI6eyJuYW1lIjoibm9uZSJ9fX1dLFswLDIsInBfe1hZfSIsMl0sWzEsMywiXFxwdGJfQ15ZKFxcZGVsdGFee3IscH0pIl0sWzIsMywiTFIiLDIseyJzdHlsZSI6eyJib2R5Ijp7Im5hbWUiOiJiYXJyZWQifSwiaGVhZCI6eyJuYW1lIjoibm9uZSJ9fX1dLFs1LDYsIlxcbHRkeW4iLDEseyJzaG9ydGVuIjp7InNvdXJjZSI6MjAsInRhcmdldCI6MjB9LCJzdHlsZSI6eyJib2R5Ijp7Im5hbWUiOiJub25lIn0sImhlYWQiOnsibmFtZSI6Im5vbmUifX19XV0=
            \begin{tikzcd}[ampersand replacement=\&]
                Y \& Y \\
                LX \& LY
                \arrow["{\le_Y}", "\shortmid"{marking}, no head, from=1-1, to=1-2]
                \arrow[""{name=0, anchor=center, inner sep=0}, "{p_{XY}}"', from=1-1, to=2-1]
                \arrow[""{name=1, anchor=center, inner sep=0}, "{\ptb_C^Y(\delta^{r,p})}", from=1-2, to=2-2]
                \arrow["LR"', "\shortmid"{marking}, no head, from=2-1, to=2-2]
                \arrow["\ltdyn"{description}, draw=none, from=0, to=1]
            \end{tikzcd} \\ 
        \hline
    \end{tabular}
    \end{center}
    (Here, $LR$ is the lock step lifting of the relation $R$.)

    \vspace{3ex}

    The push-pull property is defined as follows:
    \begin{itemize}
        \item Given any perturbation $\delta_X \in P_V^X$, we can \emph{push} it forward along $R$ to a
        perturbation $\push(\delta_X) \in P_V^Y$, such that $\ptb_V^X(\delta_X) \le \ptb_V^Y(\push(\delta_X))$.

        \item Conversely, given any perturbation $\delta_Y \in P_V^Y$, we can \emph{pull} it back along $R$
        to a perturbation $\pull(\delta_Y) \in P_V^X$, such that $\ptb_V^X(\pull(\delta_Y)) \le \ptb_V^Y(\delta_Y)$.

        \item Likewise, we can push any perturbation $\delta_X \in P_C^X$ along $LR$
        to get a perturbation $\push(\delta_X) \in P_C^Y$ such that
        $\ptb_C^X(\delta_X) \le \ptb_C^Y(\push(\delta_X))$.

        \item And similarly, we can pull a perturbation in $P_C^Y$ along $LR$ to a perturbation in $P_C^X$
        satisfying the analogous property.
    \end{itemize}

    \textbf{TODO: push and pull might need to be monoid homomorphisms}

  \item \textbf{Two-cells}: Let $f : W \to X$ and $g : Y \to Z$ and let $R : W \nrightarrow Y$ and 
  $S : X \nrightarrow Z$. We define $f \le g$ to mean for all $(w, y) \in R$, we have
  $(f(w), g(y)) \in S$. This is depicted in the square below:

  % https://q.uiver.app/#q=WzAsNCxbMCwwLCJXIl0sWzAsMSwiWCJdLFsxLDAsIlkiXSxbMSwxLCJaIl0sWzAsMiwiUiIsMCx7InN0eWxlIjp7ImJvZHkiOnsibmFtZSI6ImJhcnJlZCJ9LCJoZWFkIjp7Im5hbWUiOiJub25lIn19fV0sWzEsMywiUyIsMix7InN0eWxlIjp7ImJvZHkiOnsibmFtZSI6ImJhcnJlZCJ9LCJoZWFkIjp7Im5hbWUiOiJub25lIn19fV0sWzAsMSwiZiIsMl0sWzIsMywiZyJdLFs0LDUsIlxcc3FzdWJzZXRlcSIsMSx7InNob3J0ZW4iOnsic291cmNlIjoyMCwidGFyZ2V0IjoyMH0sInN0eWxlIjp7ImJvZHkiOnsibmFtZSI6Im5vbmUifSwiaGVhZCI6eyJuYW1lIjoibm9uZSJ9fX1dXQ==
\[\begin{tikzcd}[ampersand replacement=\&]
	W \& Y \\
	X \& Z
	\arrow[""{name=0, anchor=center, inner sep=0}, "R", "\shortmid"{marking}, no head, from=1-1, to=1-2]
	\arrow[""{name=1, anchor=center, inner sep=0}, "S"', "\shortmid"{marking}, no head, from=2-1, to=2-2]
	\arrow["f"', from=1-1, to=2-1]
	\arrow["g", from=1-2, to=2-2]
	\arrow["\sqsubseteq"{description}, draw=none, from=0, to=1]
\end{tikzcd}\]
  
\end{itemize}

The category satisfies the following additional properties:
\begin{itemize}
    \item \emph{Existence of Dyn}: There is an object $D$ with the property that for any
    object $X$, there is a horizontal arrow $X \nrightarrow D$.
    The underlying double poset is defined by guarded recursion as the solution to
    \[ D \cong \mathbb{N}\, + \later \hspace{-0.5ex} (D \to_m \li D). \]

    \textbf{TODO: define the perturbations for Dyn and show there is a horizontal arrow $X \nrightarrow D$ for all $X$.}
    
    \item \emph{Thinness}: There is at most one two-cell for any given square.

    % \item \emph{Push-Pull}: Let $X$ and $Y$ be objects, and let $R : X \nrightarrow Y$.
    % \begin{itemize}
    %     \item Given any perturbation $\delta_X \in P_V^X$, we can \emph{push} it forward along $R$ to a
    %     perturbation $\push(\delta_X) \in P_V^Y$, such that $\ptb_V^X(\delta_X) \le \ptb_V^Y(\push(\delta_X))$.

    %     \item Conversely, given any perturbation $\delta_Y \in P_V^Y$, we can \emph{pull} it back along $R$
    %     to a perturbation $\pull(\delta_Y) \in P_V^X$, such that $\ptb_V^X(\pull(\delta_Y)) \le \ptb_V^Y(\delta_Y)$.

    %     \item Likewise, we can push any perturbation $\delta_X \in P_C^X$ along $L\, R$
    %     (the lock-step lifting of the relation $R$) to get a perturbation $\push(\delta_X) \in P_C^Y$ such that
    %     $\ptb_C^X(\delta_X) \le \ptb_C^Y(\push(\delta_X))$.

    %     \item And similarly we can pull a perturbation in $P_C^Y$ to a perturbation in $P_C^X$
    %     satisfying the analogous property.
    % \end{itemize}
\end{itemize}

% Composability of embedding and projections

We need to verify that this forms a thin double category. 
\begin{itemize}
    \item \emph{Horizontal identity morphism}: 
    Let $X$ be an object. We take $R$ to be $\le_X$ (the ordering relation on $X$),
    which is trivially antitone and monotone with respect to itself.
    We let $e_{XX} = \id$ and $p_{XX} = \eta$. These clearly preserve the
    ordering and bisimilarity.

    \vspace{3ex}
    
    We first need to show that $R$ is quasi-representable.
    We prove the UpR rule; the others are similar.
    We need to specify a distinguished element $\delta^{l,e} \in P_V^X$ such that
    the following square commutes:
    
    % https://q.uiver.app/#q=WzAsNCxbMCwwLCJYIl0sWzAsMSwiWCJdLFsxLDAsIlgiXSxbMSwxLCJYIl0sWzAsMSwiXFxwdGJfVl5YKFxcZGVsdGFee2wsZX0pIiwyXSxbMiwzLCJlX3tYWH0gPSBcXGlkIl0sWzAsMiwiXFxsZV9YIiwwLHsic3R5bGUiOnsiYm9keSI6eyJuYW1lIjoiYmFycmVkIn0sImhlYWQiOnsibmFtZSI6Im5vbmUifX19XSxbMSwzLCJcXGxlX1giLDIseyJzdHlsZSI6eyJib2R5Ijp7Im5hbWUiOiJiYXJyZWQifSwiaGVhZCI6eyJuYW1lIjoibm9uZSJ9fX1dXQ==
    \[\begin{tikzcd}[ampersand replacement=\&]
        X \& X \\
        X \& X
        \arrow["{\ptb_V^X(\delta^{l,e})}"', from=1-1, to=2-1]
        \arrow["{e_{XX} = \id}", from=1-2, to=2-2]
        \arrow["{\le_X}", "\shortmid"{marking}, no head, from=1-1, to=1-2]
        \arrow["{\le_X}"', "\shortmid"{marking}, no head, from=2-1, to=2-2]
    \end{tikzcd}\]

    Taking $\delta^{l,e} = \id$ (the identity of the monoid), we observe that since
    $\ptb_V^X$ is a homomorphism of monoids, we have $\ptb_V^X(\id) = \id$.
    Now it is clear that the above square commutes.

    \vspace{3ex}

    We also need to show that $R$ satisfies the four push-pull properties.
    We show one; the others are similar. Let $\delta_X \in P_V^X$.
    We need to define $\push(\delta_X) \in P_V^X$ such that the following square commutes:

    % https://q.uiver.app/#q=WzAsNCxbMCwwLCJYIl0sWzAsMSwiWCJdLFsxLDAsIlgiXSxbMSwxLCJYIl0sWzAsMiwiXFxsZV9YIiwwLHsic3R5bGUiOnsiYm9keSI6eyJuYW1lIjoiYmFycmVkIn0sImhlYWQiOnsibmFtZSI6Im5vbmUifX19XSxbMSwzLCJcXGxlX1giLDIseyJzdHlsZSI6eyJib2R5Ijp7Im5hbWUiOiJiYXJyZWQifSwiaGVhZCI6eyJuYW1lIjoibm9uZSJ9fX1dLFswLDEsIlxccHRiX1ZeWChcXGRlbHRhX1gpIiwyXSxbMiwzLCJcXHB0Yl9WXlgoXFxwdXNoKFxcZGVsdGFfWCkpIiwwLHsic3R5bGUiOnsiYm9keSI6eyJuYW1lIjoiZGFzaGVkIn19fV1d
    \[\begin{tikzcd}[ampersand replacement=\&]
	X \& X \\
	X \& X
	\arrow["{\le_X}", "\shortmid"{marking}, no head, from=1-1, to=1-2]
	\arrow["{\le_X}"', "\shortmid"{marking}, no head, from=2-1, to=2-2]
	\arrow["{\ptb_V^X(\delta_X)}"', from=1-1, to=2-1]
	\arrow["{\ptb_V^X(\push(\delta_X))}", dashed, from=1-2, to=2-2]
    \end{tikzcd}\]

    Let $\push(\delta_X) = \delta_X$. We need to show that for $x \le_X x'$,
    we have $\ptb_V^X(\delta_X)(x) \le \ptb_V^X(\delta_X)(x')$, which holds because
    $\ptb_V^X(\delta_X)$ is monotone with respect to $\le_X$.


    \item \emph{Horizontal composition}:
    
    Let $R : X \nrightarrow Y$ and $S : Y \nrightarrow Z$.
    We define
    $e_{XZ} = e_{YZ} \circ e_{XY}$ and $p_{XZ} = \ext{p_{XY}}{} \circ p_{YZ}$.
    We define the distinguished perturbations in the representability rules as follows:

    \begin{align*}
        \delta_{R\circ S}^{l,e} &= \pull_R(\delta_S^{l,e}) \cdot \delta_R^{l,e} \\
        \delta_{R\circ S}^{r,e} &= \delta_S^{r,e} \cdot \push_S(\delta_R^{r,e}) \\
        \delta_{R\circ S}^{l,p} &= \delta_R^{l,p} \cdot \pull_{LR}(\delta_S^{l,p}) \\
        \delta_{R\circ S}^{r,p} &= \push_{LS}(\delta_R^{r,p}) \cdot \delta_S^{r,p}
    \end{align*}
    where $\cdot$ denotes composition in the appropriate monoid of perturbations.

    We can then show that the four quasi-representability rules are valid with these definitions.
    \textbf{TODO show one or two of the cases}

    \vspace{3ex}

    We also need to show that the push-pull rules hold of the composition $R \circ S$.
    This follows from the fact that they hold for both $R$ and $S$.
    Specifically, we define

    \begin{align*}
        \push_{R \circ S}(\delta^X) &= \push_S(\push_R(\delta^X)) \\
        \pull_{R \circ S}(\delta^Z) &= \pull_R(\pull_S(\delta^Z)) \\
        \push_{L(R \circ S)}(\delta^X) &= \push_{LS}(\push_{LR}(\delta^X)) \\
        \pull_{L(R \circ S)}(\delta^Z) &= \pull_{LR}(\pull_{LS}(\delta^Z))
    \end{align*}

    Then we can verify that the relevant push-pull inequalities hold using the above definitions.
    \textbf{TODO maybe show one of the cases}

    \item \emph{Identity two-cells}:
    The horizontal identity two-cells have the form

    % https://q.uiver.app/#q=WzAsNCxbMCwwLCJYIl0sWzAsMSwiWSJdLFsxLDAsIlgiXSxbMSwxLCJZIl0sWzAsMSwiZiIsMl0sWzIsMywiZiJdLFswLDIsIlxcbGVfWCJdLFsxLDMsIlxcbGVfWSIsMl0sWzQsNSwiXFxsdGR5biIsMSx7InNob3J0ZW4iOnsic291cmNlIjoyMCwidGFyZ2V0IjoyMH0sInN0eWxlIjp7ImJvZHkiOnsibmFtZSI6Im5vbmUifSwiaGVhZCI6eyJuYW1lIjoibm9uZSJ9fX1dXQ==
    \[\begin{tikzcd}[ampersand replacement=\&]
        X \& X \\
        Y \& Y
        \arrow[""{name=0, anchor=center, inner sep=0}, "f"', from=1-1, to=2-1]
        \arrow[""{name=1, anchor=center, inner sep=0}, "f", from=1-2, to=2-2]
        \arrow["{\le_X}", from=1-1, to=1-2]
        \arrow["{\le_Y}"', from=2-1, to=2-2]
        \arrow["\ltdyn"{description}, draw=none, from=0, to=1]
    \end{tikzcd}\]

    This square commutes because $f$ is monotone with respect to the ordering relation.

    The vertical two-cells have the form

    % https://q.uiver.app/#q=WzAsNCxbMCwwLCJYIl0sWzAsMSwiWCJdLFsxLDAsIlkiXSxbMSwxLCJZIl0sWzAsMiwiUiIsMCx7InN0eWxlIjp7ImJvZHkiOnsibmFtZSI6ImJhcnJlZCJ9LCJoZWFkIjp7Im5hbWUiOiJub25lIn19fV0sWzEsMywiUiIsMCx7InN0eWxlIjp7ImJvZHkiOnsibmFtZSI6ImJhcnJlZCJ9LCJoZWFkIjp7Im5hbWUiOiJub25lIn19fV0sWzAsMSwiXFxpZF9YIiwyXSxbMiwzLCJcXGlkX1kiXSxbNiw3LCJcXGx0ZHluIiwxLHsic2hvcnRlbiI6eyJzb3VyY2UiOjIwLCJ0YXJnZXQiOjIwfSwic3R5bGUiOnsiYm9keSI6eyJuYW1lIjoibm9uZSJ9LCJoZWFkIjp7Im5hbWUiOiJub25lIn19fV1d
    \[\begin{tikzcd}[ampersand replacement=\&]
        X \& Y \\
        X \& Y
        \arrow["R", "\shortmid"{marking}, no head, from=1-1, to=1-2]
        \arrow["R", "\shortmid"{marking}, no head, from=2-1, to=2-2]
        \arrow[""{name=0, anchor=center, inner sep=0}, "{\id_X}"', from=1-1, to=2-1]
        \arrow[""{name=1, anchor=center, inner sep=0}, "{\id_Y}", from=1-2, to=2-2]
        \arrow["\ltdyn"{description}, draw=none, from=0, to=1],
    \end{tikzcd}\]

    which commutes trivially.

    \item \emph{Composition of two-cells}:
    Two-cells compose vertically and horizontally, which follows from the definition
    definition of a two-cell in this category.

    \textbf{TODO: elaborate}
\end{itemize}

\vspace{3ex}

\textbf{Kleisli internal hom: TODO}

% The arrow => takes a value type and an algebra and constructs an algebra
%\section{A Simple Denotational Semantics for the Terms of GTLC}\label{sec:gtlc-terms}

In this section, we introduce the term syntax for the gradually-typed
lambda calculus (GTLC) and give a set-theoretic denotational semantics
using tools from SGDT. This serves two purposes: First, it is a simple setting
in which to employ the tools of SGDT.
Second, constructing this semantic model establishes the validity of the
beta and eta principles for the gradually-typed lambda calculus.

In Section \ref{sec:gtlc-precision}, we will discuss how to extend the denotational
semantics to accommodate the type and term precision orderings.


\subsection{Syntax}\label{sec:term-syntax}

Our syntax is based on fine-grained call by value, and as such it has
separate value and producer terms and typing judgments for each.

% Given a term $M$ of type $A$, the term $\bind{x}{M}{N}$ should be thought of as
% running $M$ to a value $V$ and then continuing as $N$, with $V$ in place of $x$.


\begin{align*}
  &\text{Types } A := \nat \alt \,\dyn \alt (A \ra A') \\
  &\text{Contexts } \Gamma := \cdot \alt (\Gamma, x : A) \\
  &\text{Values } V :=  \zro \alt \suc\, V \alt \lda{x}{M} \alt \up{A}{B} V \\ 
  &\text{Producers } M, N := \err_B \alt \matchnat {V} {M} {n} {M'} \\ 
  &\quad\quad \alt \ret {V} \alt \bind{x}{M}{N} \alt V_f\, V_x \alt \dn{A}{B} M 
\end{align*}


The value typing judgment is written $\vhasty{\Gamma}{V}{A}$ and 
the producer typing judgment is written $\phasty{\Gamma}{M}{A}$.

The typing rules are as expected, with a cast between $A$ to $B$ allowed only when $A \ltdyn B$.
The precise rules for $A \ltdyn B$ will be given below.
Notice that the upcast of a value is a value, since it always succeeds, while the downcast
of a value is a producer, since it may fail.

\begin{mathpar}
    % Var
    \inferrule*{ }{\vhasty {\cdot, \Gamma, x : A, \Gamma'} x A}

    % Err
    \inferrule*{ }{\phasty {\cdot, \Gamma} {\err_A} A} 
  
    % Zero and suc
    \inferrule*{ }{\vhasty \Gamma \zro \nat}
  
    \inferrule*{\vhasty \Gamma V \nat} {\vhasty \Gamma {\suc\, V} \nat}

    % Match-nat
    \inferrule*
    {\vhasty \Gamma V \nat \and 
     \phasty \Delta M A \and \phasty {\Gamma, n : \nat} {M'} A}
    {\phasty \Gamma {\matchnat {V} {M} {n} {M'}} A}
  
    % Lambda
    \inferrule* 
    {\phasty {\Gamma, x : A} M {A'}} 
    {\vhasty \Gamma {\lda x M} {A \ra A'}}
  
    % App
    \inferrule*
    {\vhasty \Gamma {V_f} {A \ra A'} \and \vhasty \Gamma {V_x} A}
    {\phasty {\Gamma} {V_f \, V_x} {A'}}

    % Ret
    \inferrule*
    {\vhasty \Gamma V A}
    {\phasty {\Gamma} {\ret\, V} {A}}

    % Bind
    \inferrule*
    {\phasty \Gamma M {A} \and \phasty{\Gamma, x : A}{N}{B} } % Need x : A in context
    {\phasty {\Gamma} {\bind{x}{M}{N}} {B}}

    % Upcast
    \inferrule*
    {A \ltdyn A' \and \vhasty \Gamma V A}
    {\vhasty \Gamma {\up A {A'} V} {A'} }

    \inferrule* % TODO is this correct?
    {A \ltdyn A' \and \phasty {\Gamma} {M} {A'}}
    {\phasty {\Gamma} {\dn A {A'} M} {A}}

\end{mathpar}


In the equational theory, we have $\beta$ and $\eta$ laws for function type,
as well a $\beta$ and $\eta$ law for bind.

\begin{mathpar}
  % Function Beta and Eta
  \inferrule*
  {\phasty {\Gamma, x : A} M {B} \and \vhasty \Gamma V A}
  {(\lda x M)\, V = M[V/x]}

  \inferrule*
  {\vhasty \Gamma V {A \ra A}}
  {\Gamma \vdash V = \lda x {V\, x}}

  % Ret Beta and Eta
  \inferrule*
  {}
  {(\bind{x}{\ret\, V}{N}) = N[V/x]}

  \inferrule*
  {\phasty {\Gamma} {M} {B}}
  {\bind{x}{M}{\ret x} = M}

  % Match-nat Beta
  \inferrule*
  {\phasty \Delta M A \and \phasty {\Gamma, n : \nat} {M'} A}
  {\matchnat{\zro}{M}{n}{M'} = M}

  \inferrule*
  {\vhasty \Gamma V \nat \and 
   \phasty \Gamma M B \and \phasty {\Gamma, n : \nat} {M'} B}
  {\matchnat{\suc\, V}{M}{n}{M'} = M'}

  % Match-nat Eta
  % This doesn't build in substitution
  \inferrule*
  {\hasty {\Gamma , x : \nat} M A}
  {M = \matchnat{x} {M[\zro / x]} {n} {M[(\suc\, n) / x]}}

\end{mathpar}

\subsubsection{Type Precision}\label{sec:type-precision}

The type precision rules specify what it means for a type $A$ to be more precise than $A'$.
We have reflexivity rules for $\dyn$ and $\nat$, as well as rules that $\nat$ is more precise than $\dyn$
and $\dyntodyn$ is more precise than $\dyn$.
We also have a congruence rule for function types stating that given $A_i \ltdyn A'_i$ and $A_o \ltdyn A'_o$, we can prove
$A_i \ra A_o \ltdyn A'_i \ra A'_o$. Note that precision is covariant in both the domain and codomain.
Finally, we can lift a relation on value types $A \ltdyn A'$ to a relation $\Ret A \ltdyn \Ret A'$ on
computation types.

\begin{mathpar}
  \inferrule*[right = \dyn]
    { }{\dyn \ltdyn\, \dyn}

  \inferrule*[right = \nat]
    { }{\nat \ltdyn \nat}

  \inferrule*[right = $\ra$]
    {A_i \ltdyn A'_i \and A_o \ltdyn A'_o }
    {(A_i \ra A_o) \ltdyn (A'_i \ra A'_o)}

  \inferrule*[right = $\textsf{Inj}_\nat$]
    { }{\nat \ltdyn\, \dyn}

  \inferrule*[right = $\textsf{Inj}_{\ra}$]
    { }
    {(\dyntodyn) \ltdyn\, \dyn}

  \inferrule*[right = $\injarr{}$]
    {(R \ra S) \ltdyn\, (\dyntodyn)}
    {(R \ra S) \ltdyn\, \dyn}

  
\end{mathpar}

We can prove that transitivity of type precision is admissible, i.e.,
if $A \ltdyn B$ and $B \ltdyn C$, then $A \ltdyn C$.

% Type precision derivations
Note that as a consequence of this presentation of the type precision rules, we
have that if $A \ltdyn A'$, there is a unique precision derivation that witnesses this.
As in previous work, we go a step farther and make these derivations first-class objects,
known as \emph{type precision derivations}.
Specifically, for every $A \ltdyn A'$, we have a derivation $c : A \ltdyn A'$ that is constructed
using the rules above. For instance, there is a derivation $\dyn : \dyn \ltdyn \dyn$, and a derivation
$\nat : \nat \ltdyn \nat$, and if $c_i : A_i \ltdyn A_i$ and $c_o : A_o \ltdyn A'_o$, then
there is a derivation $c_i \ra c_o : (A_i \ra A_o) \ltdyn (A'_i \ra A'_o)$. Likewise for
the remaining rules. The benefit to making these derivations explicit in the syntax is that we
can perform induction over them.
Note also that for any type $A$, we use $A$ to denote the reflexivity derivation that $A \ltdyn A$,
i.e., $A : A \ltdyn A$.
Finally, observe that for type precision derivations $c : A \ltdyn A'$ and $c' : A' \ltdyn A''$, we
can define their composition $c \relcomp c' : A \ltdyn A''$.
This notion will be used below in the statement of transitivity of the term precision relation.


\begin{comment}
\subsection{Removing Casts as Primitives}

% We now observe that all casts, except those between $\nat$ and $\dyn$
% and between $\dyntodyn$ and $\dyn$, are admissible, in the sense that
% we can start from $\extlcm$, remove casts except the aforementioned ones,
% and in the resulting language we will be able to derive the other casts.

We now observe that all casts, except those between $\nat$ and $\dyn$
and between $\dyntodyn$ and $\dyn$, are admissible.
That is, consider a new language ($\extlcprime$) in which
instead of having arbitrary casts, we have injections from $\nat$ and
$\dyntodyn$ into $\dyn$, and a case inspection on $\dyn$.
We claim that in $\extlcprime$, all of the casts present in $\extlc$ are derivable.
It will suffice to verify that casts for function type are derivable.
This holds because function casts are constructed inductively from the casts
of their domain and codomain. The base case is one of the casts involving $\nat$
or $\dyntodyn$ which are present in $\extlcprime$ as injections and case inspections.


The resulting calculus $\extlcprime$ now lacks arbitrary casts as a primitive notion:

%%%%%%%%%%%%%%%%%%%%%%%%%%%%%%%%%%%%%%%%%%%%%%
% TODO update

\begin{align*}
  &\text{Types } A := \nat \alt \dyn \alt (A \ra A') \\
  &\text{Contexts } \Gamma := \cdot \alt (\Gamma, x : A) \\
  &\text{Values } V :=  \zro \alt \suc\, V \alt \lda{x}{M} \alt \injnat V \alt \injarr V \\ 
  &\text{Producers } M, N := \err_B \alt \ret {V} \alt \bind{x}{M}{N}
    \alt V_f\, V_x \alt
    \\ & \quad\quad \casenat{V}{M_{no}}{n}{M_{yes}} 
    \alt \casearr{V}{M_{no}}{f}{M_{yes}}
\end{align*}


% New rules
Figure \ref{fig:extlc-minus-minus-typing} shows the new typing rules,
and Figure \ref{fig:extlc-minus-minus-eqns} shows the equational rules
for case-nat (the rules for case-arrow are analogous).

\begin{figure}
  \begin{mathpar}
      % inj-nat
      \inferrule*
      {\hasty \Gamma M \nat}
      {\hasty \Gamma {\injnat M} \dyn}

      % inj-arr 
      \inferrule*
      {\hasty \Gamma M (\dyntodyn)}
      {\hasty \Gamma {\injarr M} \dyn}

      % Case dyn
      \inferrule*
      {\hasty{\Delta|_V}{V}{\dyn} \and
        \hasty{\Delta , x : \nat }{M_{nat}}{B} \and 
        \hasty{\Delta , x : (\dyntodyn) }{M_{fun}}{B}
      }
      {\hasty {\Delta} {\casedyn{V}{n}{M_{nat}}{f}{M_{fun}}} {B}}
  \end{mathpar}
  \caption{New typing rules for $\extlcmm$.}
  \label{fig:extlc-minus-minus-typing}
\end{figure}


\begin{figure}
  \begin{mathpar}
     % Case-dyn Beta
     \inferrule*
     {\hasty \Gamma V \nat}
     {\casedyn {\injnat {V}} {n} {M_{nat}} {f} {M_{fun}} = M_{nat}[V/n]}

     \inferrule*
     {\hasty \Gamma V {\dyntodyn} }
     {\casedyn {\injarr {V}} {n} {M_{nat}} {f} {M_{fun}} = M_{fun}[V/f]}

     % Case-dyn Eta
     \inferrule*
     {}
     {\Gamma , x :\, \dyn \vdash M = \casedyn{x}{n}{M[(\injnat{n}) / x]}{f}{M[(\injarr{f}) / x]} }


  \end{mathpar}
  \caption{New equational rules for $\extlcprime$ (rules for case-arrow are analogous
           and hence are omitted).}
  \label{fig:extlc-minus-minus-eqns}
\end{figure}

\end{comment}


% \section{Term Semantics}\label{sec:term-semantics}

\subsection{Semantic Constructions}\label{sec:domain-theory}

In this section, we discuss the fundamental objects of the model into which we will embed
the terms of the gradually-typed lambda calculus.
It is important to remember that the constructions in this section are entirely
independent of the syntax described in the previous section; the notions defined 
here exist in their own right as purely mathematical constructs.
In Section \ref{sec:term-interpretation}, we will link the syntax and semantics
via a semantic interpretation function.


\subsection{Modeling the Dynamic Type}

When modeling the dynamic type $\dyn$, we need a semantic object $D$ that satisfies the
isomorphism

\[ D \cong \Nat + (D \to (D + 1)). \]

where the $D + 1$ represents the fact that in the function case, the function may return an error.
Unfortunately, this equation does not have inductive or coinductive solutions. The usual way of
dealing with such equations is via domain theory, by which we can obtain an exact solution.
However, the heavy machinery of domain theory can be difficult for language designers to learn
and apply in the mechanized setting.
Instead, we will leverage the tools of guarded type theory, considering instead the following
similar-looking equation:

\[ D \cong \Nat + \later (D \to (D + 1)). \]

Since the negative occurrence of $D$ is guarded under a later, this equation has a (guarded) solution.
Specifically, we consider the following function $f$ of type
$\later \type \to \type$:

\[ \lambda (D' : \later \type) . \Nat + \later_t (D'_t \to (D'_t + 1)). \]

(Recall that the tick $t : \tick$ is evidence that time has passed, and since
$D'$ has type $\later \type$, i.e. $\tick \to \type$, then $D'_t$ has type $\type$.)

Then we define 

\[ D = \fix f. \]

% TODO explain better
As it turns out, this definition is not quite correct, as it doesn't provide a way to
model functions that are potentially non-terminating.
Another way to think about this is that by using guarded recursion to solve the
equation for the dynamic type, the solution to the equation involves a notion of
``time" or ``steps".
So, in addition to returning a value or erroring, programs may now take one or
more observable steps of computation, and this possibility must be reflected in
in the equation for the dynamic type.

% Therefore, the semantics of terms will need to allow for terms that potentially
% do not terminate. This is accomplished using an extension of the guarded lift monad,
% which we describe in the next section.
More specifically, in the equation for the semantics of $\dyn$, the result of the
function should be a computation that may return a value, error, \emph{or} take an observable step.
We model such computations using an extension of the so-called guarded lift monad
\cite{mogelberg-paviotti2016} which we describe in the next section.
We will then use this to give the correct definition of the semantics of the dynamic type.

\subsubsection{The Lift + Error Monad}\label{sec:lift-monad}

% TODO ensure the previous section flows into this one
When thinking about how to model gradually-typed programs where we track the steps they take,
we should consider their possible behaviors. On the one hand, we have \emph{failure}: a program may fail
at run-time because of a type error. In addition to this, a program may take one or more steps of computation.
If a program steps forever, then it never returns a value,
so we can think of the idea of stepping as a way of intensionally modelling \emph{partiality}.

With this in mind, we can describe a semantic object that models these behaviors: a monad
for embedding computations that has cases for failure and ``stepping''.
Previous work has studied such a construct in the setting of the latter, called the lift
monad \cite{mogelberg-paviotti2016}; here, we augment it with the additional effect of failure.

For a type $A$, we define the \emph{lift monad with failure} $\li A$, which we will just call
the \emph{lift monad}, as the following datatype:

\begin{align*}
  \li A &:= \\
  &\eta \colon A \to \li A \\
  &\mho \colon \li A \\
  &\theta \colon \later (\li A) \to \li A
\end{align*}

Unless otherwise mentioned, all constructs involving $\later$ or $\fix$
are understood to be with respect to a fixed clock $k$. So for the above, we really have for each
clock $k$ a type $\li^k A$ with respect to that clock.

Formally, the lift monad $\li A$ is defined as the solution to the guarded recursive type equation

\[ \li A \cong A + 1 + \later \li A. \]

An element of $\li A$ should be viewed as a computation that can either (1) return a value (via $\eta$),
(2) raise an error and stop (via $\mho$), or (3) take a step (via $\theta$).
%
Notice there is a computation $\fix \theta$ of type $\li A$. This represents a computation
that runs forever and never returns a value.

Since we claimed that $\li A$ is a monad, we need to define the monadic operations
and show that they respect the monadic laws. The return is just $\eta$, and extend
is defined via guarded recursion by cases on the input.
% It is instructive to give at least one example of a use of guarded recursion, so
% we show below how to define extend:
% TODO
%
%
Verifying that the monadic laws hold uses \lob-induction and is straightforward.

%\subsubsection{Model-Theoretic Description}
%We can describe the lift monad in the topos of trees model as follows.

\subsubsection{Revisiting the Dynamic Type}
Now we can state the correct definition of the semantics for the dynamic type.
The set $D$ is defined to be the solution to the guarded equation

\[ D \cong \Nat + \later (D \to \textcolor{red}{\li} D). \]


\subsection{Interpretation}\label{sec:term-interpretation}

We can now give a semantics to the \emph{terms} of the gradual lambda calculus we defined
above. The full definition is given in Figure \ref{fig:term-semantics}.
%
Much of the semantics is similar to a normal call-by-value denotational semantics:
We will interpret types as sets, and terms as functions.
Contexts $\Gamma = x_1 \colon A_1, \dots, x_n \colon A_n$
will be interpreted as the product $\sem{A_1} \times \cdots \times \sem{A_n}$.


The semantics of the dynamic type $\dyn$ is the set $\Dyn$ introduced in Section
\ref{sec:predomains}.
%
The interpretation of a value $\vhasty {\Gamma} V A$ will be a function from
$\sem{\Gamma}$ to $\sem{A}$. Likewise, a term $\phasty {\Gamma} M {{A}}$ will be interpreted
as a function from $\sem{\Gamma}$ to $\li \sem{A}$.

Recall that $\Dyn$ is isomorphic to $\Nat\, + \later (\Dyn \to \li \Dyn)$.
Thus, the semantics of $\injnat{\cdot}$ is simply $\inl$ and the semantics
of $\injarr{\cdot}$ is simply $\inr \circ \nxt$.
The semantics of case inspection on dyn performs a case analysis on the sum.

The interpretation of $\lda{x}{M}$ works as follows. Recall by the typing rule for
lambda that $\phasty {\Gamma, x : A_i} M {{A_o}}$, so the interpretation of $M$
has type $(\sem{\Gamma} \times \sem{A_i})$ to $\li \sem{A_o}$.
The interpretation of lambda is thus a function (in the ambient type theory) that takes
a value $a$ representing the argument and applies it (along with $\gamma$) as argument to
the interpretation of $M$.
%
The interpretation of bind makes use the monadic extend function on $\li A$.
%
The interpretation of case-nat and case-arrow is simply a case inspection on the
interpretation of the scrutinee, which has type $\Dyn$.


\vspace{2ex}


\begin{figure*}
  \noindent Types:
  \begin{align*}
    \sem{\nat} &= \Nat \\
    \sem{\dyn} &= \Dyn \\
    \sem{A \ra A'} &= \sem{A} \To \li \sem{A'} \\
  \end{align*}

  % Contexts:

  % TODO check these, especially the semantics of bind, case-nat, and case-arr
  % with respect to their context argument
  \noindent Values and terms:
  \begin{align*}
    \sem{\zro}         &= \lambda \gamma . 0 \\
    \sem{\suc\, V}     &= \lambda \gamma . (\sem{V}\, \gamma) + 1 \\
    \sem{x \in \Gamma} &= \lambda \gamma . \gamma(x) \\
    \sem{\lda{x}{M}}   &= \lambda \gamma . \lambda a . \sem{M}\, (*,\, (\gamma , a))  \\
    \sem{\injnat{V_n}} &= \lambda \gamma . \inl\, (\sem{V_n}\, \gamma) \\
    \sem{\injarr{V_f}} &= \lambda \gamma . \inr\, (\sem{V_f}\, \gamma) \\[2ex]
    % \sem{\nxt\, V}     &= \lambda \gamma . \nxt (\sem{V}\, \gamma) \\
    % \sem{\theta}       &= \lambda \gamma . \theta \\
  %
    \sem{\err_B}         &= \lambda \delta . \mho \\
    \sem{\ret\, V}       &= \lambda \gamma . \eta\, \sem{V} \\
    \sem{\bind{x}{M}{N}} &= \lambda \delta . \ext {(\lambda x . \sem{N}\, (\delta, x))} {\sem{M}\, \delta} \\
    \sem{V_f\, V_x}      &= \lambda \gamma . {({(\sem{V_f}\, \gamma)} \, {(\sem{V_x}\, \gamma)})} \\
    \sem{\casedyn{V}{n}{M_{nat}}{\tilde{f}}{M_{fun}}} &=
      \lambda \delta . \text{case $(\sem{V}\, \delta)$ of} \\ 
      &\quad\quad\quad\quad \alt \inl(n) \to \sem{M_{nat}}(n) \\
      &\quad\quad\quad\quad \alt \inr(\tilde{f}) \to \sem{M_{fun}}(\tilde{f})
  \end{align*}

  \caption{Term semantics for the gradually-typed lambda calculus.}
  \label{fig:term-semantics}
\end{figure*}
\section{A Guarded Model of Graduality}\label{sec:concrete-relational-model}

In this section, we describe in detail our model of gradual typing. To manage
the complexity, we carry out the construction in two phases, beginning with a
relational model consisting of partially-ordered sets with a notion of
bisimilarity and then adding requirements for a structured set of syntactic
perturbations on each type as well as the notion of quasi-representability for
relations that models the cast rules. We then discuss the construction of the
dynamic type and the refined denotation of term precision that relates terms
that can be synchronized to be in lock-step. We conclude by demonstrating the
adequacy of our model with respect to the graduality property.

\subsection{Phase One: Predomains and Error Domains}

In the previous section, we introduced the notion of weak bisimilarity as a
relation on the guarded lift monad. Recall that weak bisimilarity on $\li A$ is
parameterized by a relation on $A$. To define a compositional model for all
types, we equip the posets and error domains with a reflexive, symmetric
bisimilarity relation. This leads to our first new definition: 

\begin{definition}
A \textbf{predomain} $A$ consists of a set $A$ along with two relations:
\begin{itemize}
    \item A partial order $\le_A$.
    \item A reflexive, symmetric ``bisimilarity'' relation $\bisim_A$.
\end{itemize}
\end{definition}

We will generally write $A$ for both a predomain and its underlying set; if we
need to emphasize the difference we will write $|A|$ for the underlying set of
$A$.

Given a predomain $A$, we can form the predomain $\later A$. The underlying set
is $\later |A|$ and the relation is defined in the obvious way, i.e., $\tilde{x}
\le_{\later A} \tilde{x'}$ iff $\later_t(\tilde{x}_t \le_A \tilde{x'}_t)$.
Likewise for bisimilarity.
%
We also give a predomain structure to the natural numbers $\mathbb{N}$, where
both the ordering and the bisimilarity relation are equality.
%
Morphisms of predomains are functions between the underlying sets that preserve
the ordering and the bisimilarity relation. More formally:
%
\begin{definition}
Let $A$ and $A'$ be predomains. A \textbf{morphism of predomains} $f : A \to A'$
is a function between the underlying sets such that for all $x, x'$, if $x \le_A
x'$, then $f(x) \le f(x')$, and if $x \bisim_A x'$, then $f(x) \bisim_{A'}
f(x')$.
\end{definition}

We also equip error domains with a bisimilarity relation, and we impose a
condition relating the $\theta$ morphism to the bisimilarity relation. More
concretely:
%
\begin{definition}
An \textbf{error domain} $B$ consists of a predomain $B$ along with the following data:
\begin{itemize}
    \item A distinguished ``error'' element $\mho_B \in B$
    \item A morphism of predomains $\theta_B \colon \later B \to B$
    \item For all $x : B$ we have $\theta_B(\nxt\, x) \bisim_B x$
\end{itemize}
\end{definition}
%
For an error domain $B$, we define the predomain morphism $\delta_B := \theta_B
\circ \nxt$.
%
A \textbf{morphism of error domains} is simply a morphism of the underlying
predomains that preserves the error element and the $\theta$ map.

%
The definition of relations between predomains is the same as that between
posets introduced in the previous section, as the bisimilarity relations of the
predomains are not involved in the definition of monotone relation.

Next we define an analogous notion of \emph{relation} between error domains $B$
and $B'$. An error domain relation $d : B \rel B'$ is a relation on the
underlying posets that preserves error and respects $\theta$. More formally:
%
\begin{definition}
  Let $B$ and $B'$ be error domains. An \textbf{error domain relation} between
  $B$ and $B'$ is a monotone relation $d$ on the underlying partially-ordered sets such that:
  \begin{enumerate}
     \item (Respects error): For all $y \in B'$, we have $(\mho_B, y) \in d$.
     \item (Preserves $\theta$): For all $\tilde{x}$ in $\laterhs B$ and
     $\tilde{y} \in \laterhs B'$, if $\laterhs_t ((\tilde{x}_t, \tilde{y}_t) \in
     d)$, then $(\theta_B (\tilde{x}), \theta_{B'} (\tilde{y})) \in d$. 
  \end{enumerate}
\end{definition}
%
As with predomain relations, we can compose error domain relations. However,
because of a technical restriction involving the interaction between
propositional truncation and the later modality, we cannot define the
composition of error domain relations as the composition of their underlying
predomain relations. Instead, we define composition of error domain relations
$d$ on $B_1$ and $B_2$ and $d'$ on $B_2$ and $B_3$ inductively to be the least
relation containing $d$ and $d'$ that is downward closed under $\le_{B_1}$,
upward-closed under $\le_{B_3}$, respects error, and preserves $\theta$.
Specifically, it is defined inductively by the following rules:
% %
% \begin{mathpar}
%     \inferrule*[right = Comp]
%     {b_1 \mathbin{d} b_2 \and b_2 \mathbin{d'} b_3}
%     {b_1 \mathbin{d \relcomp d'} b_3}

%     \inferrule*[right = DnClosed]
%     {b_1' \le_{B_1} b_1 \and b_1 \mathbin{d \relcomp d'} b_3}
%     {b_1' \mathbin{d \relcomp d'} b_3}

%     \inferrule*[right = UpClosed]
%     {b_1 \mathbin{d \relcomp d'} b_3 \and b_3 \le_{B_3} b_3'}
%     {b_1 \mathbin{d \relcomp d'} b_3'}

%     \inferrule*[right = PresErr]
%     { }
%     {\mho_{B_1} \mathbin{d \relcomp d'} b_3}

%     \inferrule*[right = PresTheta]
%     {\later_t( \tilde{b_1} \mathbin{d \relcomp d'} \tilde{b_3} ) }
%     {\theta_{B_1}(\tilde{b_1}) \mathbin{d \relcomp d'} \theta_{B_3}(\tilde{b_3}) }
% \end{mathpar}
% %

The definitions of squares for predomains are also unchanged from the previous
section, except that the morphisms on the left and right are now morphisms of
predomains. We likewise define a notion of \textbf{error domain square}: Let
$d_i : B_i \rel B_i'$ and $d_o : B_o \rel B_o'$ be error domain relations, and
let $\phi : B_i \multimap B_o$ and $\phi' : B_i' \multimap B_o'$ be morphisms of
error domains. We say that there is an error domain square $\phi \ltsq{d_i}{d_o}
\phi'$ exactly when there is a predomain square involving the underlying
predomain morphisms and relations.


\subsubsection{Functors}\label{sec:free-error-domain}

The functors $U$, $\li$, $\times$, and $\arr$ introduced in the previous section
extend to predomains, error domains, and their relations. The functor $U$ simply
takes the underlying predomain relation of an error domain relation. Given a
relation $c_1 : A_1 \rel A_1'$ and $c_2 : A_2 \rel A_2'$, the relation $c_1
\times c_2 : (A_1 \times A_2) \rel (A_1' \times A_2')$ is defined by $((x, y),
(x', y')) \in (c_1 \times c_2)$ if and only if $(x, x') \in c_1$ and $(y, y')
\in c_2$. Given a relation $c : A \rel A'$ and an error domain relation $d : B
\rel B'$, the relation $c \arr d : (A \arr B) \rel (A' \arr B')$ is the error
domain relation defined by $(f, g) \in (c \arr d)$ if and only if for all $x :
A$ and $x' : A'$ with $(x , x') \in c$ we have $(f(x), g(x')) \in Ud$. The fact
that this relation preserves error and respects $\theta$ follows from the
corresponding properties for $d$.

% Free error domain
The functor $\li$ takes a predomain $A$ to the free error domain on $A$. We first
define the predomain $U\li A$. The underlying set of $U\li A$ is defined to be $\li
|A|$ (the guarded lift monad applied to $|A|$). The ordering relation is the
lock-step error ordering introduced in the previous section (taking $A = A'$ in
the definition), and the bisimilarity relation is the weak bisimilarity relation
on $\li |A|$ also defined in the previous section. With these relations, the
constructors $\eta$ and $\theta$ of the guarded lift monad are in fact morphisms
of predomains, i.e., they are monotone and preserve bisimilarity.
%
We observe that this predomain structure extends to an error domain structure by
noting that the required error element is given by the constructor $\mho$ and
the required $\theta$ map is the constructor $\theta$. Lastly, it can be shown
that the delay morphism $\delta = \theta \circ \nxt : UB \to UB$ satisfies $x \bisim
\delta\, x$ for all $x$ as is required by the definition of error domain.

The monadic bind operation $-^\dagger$ takes a predomain morphism $f : A \to UB$
to an error domain morphism $\li A \multimap B$; with this we define the action of $\li$ on
morphisms: Given $f : A \to A'$ we define $\li f = (\eta \circ f)^\dagger : \li A \to \li A'$.

Finally, we define the action of $\li$ on a predomain relation $c$ to be the
heterogeneous lock-step error ordering between $U\li A$ and $U\li A'$ and note
that it by definition satisfies the additional requirements of being an error
domain relation between the error domains $\li A$ and $\li A'$.



\subsection{Phase Two: Perturbations and Quasi-Representable Relations}

We now introduce the additional definitions needed to complete the construction
of our model. Recall from Section \ref{sec:lock-step-and-weak-bisim} that in order
to establish the DnL rule for the downcast of $\iarr$, we needed to adjust the
rule by inserting a ``delay'' on the right-hand side. A similar adjustment is
needed for the DnR rule for $\iarr$. Moreover, the need to insert delays impacts
the semantics of the cast rules for \emph{all} relations, because of the
functorial nature of casts. That is, the upcast at a function type $c_i \ra c_o$
involves a downcast in the domain and an upcast in the codomain. This has two
consequences: first, the squares corresponding to the rules UpL and UpR may also
require the insertion of a delay. Second, we need to be able to insert
``higher-order'' delays in a way that follows the structures of the casts.

To accomplish this, we equip every predomain $A$ with a monoid $M_A$ of
\emph{syntactic perturbations}, as well as a means of \emph{interpreting} these
syntactic perturbations as \emph{semantic} perturbations on $A$, i.e., as
endomorphisms that are bisimilar to the identity.
% Moreover, we want the
% resulting endomorphisms to be extensionally equivalent to the identity function,
% a condition we express by saying that they are weakly bisimilar to the identity
% morphism on $A$. 
More formally, for every $A$ we require a homomorphism of monoids $i_A : M_A \to
\morbisimid{A} := \{ f : A \to A \mid f \bisim \id_A \}$. We likewise equip
every error domain $B$ with a monoid $M_B$ and interpretation homomorphism $i_B
: M_B \to \morbisimid{B}$. This grouping of data deserves its own definition: We
call the triple $(A, M_A, iA)$ a \textbf{value object} as it will be the
denotation of value types in our final model. Likewise, the triple $(B, M_B,
i_B)$ will be called a \textbf{computation object}.

We must impose an additional condition on predomain relations $c$ and error
domain relations $d$ that specify how they interact with these perturbations. In
particular, we need to be able to ``push'' and ``pull'' perturbations along
relations $c$ and $d$. The intuition for this requirement comes from the
construction of the square corresponding to the original UpL rule from the
square for the simplified version as was shown in Section
\ref{sec:towards-relational-model}. Recall that we horizontally composed the
simplified UpL square for $c$ with the identity square for $c'$ on the right.
Now that the square for UpL involves a perturbation on the right instead of the
identity, this construction will not work unless we can ``push'' the
perturbation on $A_2$ to one on $A_3$. That is, given a relation $c : A \rel A'$
and a syntactic perturbation $m_A \in M_A$, we need to be able to turn it into a
syntactic perturbation $m_{A'} \in M_{A'}$ such that the resulting semantic
perturbations obtained by applying the respective homomorphisms $i_A$ and
$i_{A'}$ form a square. Dually, we will need to ``pull'' a perturbation from
right to left. This notion is made formal by the following definition:
%
\begin{definition}
    Let $(A, M_A, i_A)$ and $(A', M_{A'}, i_{A'})$ be value objects, and let $c
    : A \rel A'$ be a predomain relation. A \textbf{push-pull structure} for $c$
    (with respect to $M_A$ and $M_{A'}$) consists of monoid homomorphisms $\push
    : M_A \to M_A'$ and $\pull : M_A' \to M_A$ such that for all $m_A \in M_A$
    there is a square $i_A(m_A) \ltsq{c}{c} i_{A'}(\push\, m_A)$ and for all
    $m_{A'} \in M_{A'}$ there is a square $i_A(\pull\, m_{A'}) \ltsq{c}{c}
    i_{A'}(m_{A'})$.
\end{definition}
%
We make the analogous definition for computation objects.

With the notion of syntactic perturbation defined, we can now make formal the
final aspect of the model: the notion of quasi-represntability of relations.
%
\begin{definition}[quasi-left-representable relations]
Let $(A, M_A, i_A)$ and $(A', M_{A'}, i_{A'})$ be value objects, and let $c : A
\rel A'$ be a predomain relation. We say that $c$ is
\emph{quasi-left-representable} by a predomain morphism $e : A \to A'$ if there
are perturbations $\delle_c \in M_A$ and $\delre_c \in M_{A'}$ such that the
following two squares exist: (1) $\upl$: $e \ltsq{c}{\le_{A'}} i_{A'}(\delre_c)$, and
(2) $\upr$: $i_A(\delle_c) \ltsq{\le_{A}}{c} e$.
\end{definition}
%
Observe that this weakens the previous notion of representability, since under
that definition the perturbations were required to be the identity.
%
We make the analogous definition of quasi-left-representability for error domain
relations $d$.

Likewise, we define quasi-right-representability as follows:
%
\begin{definition}
Let $(B, M_B, i_B)$ and $(B', M_{B'}, i_{B'})$ be computation objects, and let
$d : B \rel B'$ be an error domain relation. We say that $d$ is
\emph{quasi-right-representable} by an error domain morphism $p : B' \multimap B$ if
there are perturbations $\dellp_d \in M_B$ and $\delrp_d \in M_{B'}$ such that
the following two squares exist: 
(1) $\dnl$: $p \ltsq{\le_{B'}}{d} i_{B'}(\delrp_d)$ and
(2) $\dnr$: $i_B(\dellp_d) \ltsq{d}{\le_{B}} p$.
\end{definition}
%
We make the analogous definition of quasi-right-representability for predomain
relations $c$.

Given $(A, M_A, i_A)$ and $(A', M_{A'}, i_{A'})$, it makes sense to refer to the
quasi-representability of a predomain relation $c : A \rel A'$. However,
quasi-representable relations do not in general compose: in order to construct
the needed squares for the composition of two relations, we require that the
relations have push-pull structures.

We can now give the final definition of value and computation relations for our
model:
%
\begin{definition}[value relations]
A \textbf{value relation} between value objects $(A, M_A, i_A)$ and $(A',
M_{A'}, i_{A'})$ consists of a predomain relation $c : A \rel A'$ that has a
push-pull structure, is quasi-left-representable by a morphism $e_c : A \to A'$,
and is such that $\li c$ is quasi-right-representable by an error domain morphism
$p_c : \li A' \to \li A$.
\end{definition}

\begin{definition}[computation relations]
A \textbf{computation relation} between computation objects $(B, M_B, i_B)$ and
$(B', M_{B'}, i_{B'})$ consists of an error domain relation $d$ that has a
push-pull structure, is quasi-right-representable by a morphism $p_d : B' \to
B$, and is such that $Ud$ is quasi-left-representable by a morphism $e_d : UB
\to UB'$
\end{definition}

The reason for requiring $\li c$ to be quasi-right-representable in the
definition of value relations is so that we can define the action of the functor
$\li$ on relations, taking a value relation to a computation relation. Likewise,
we require $Ud$ to be quasi-left-representable in the definition of computation
relations so that we can define the action of $U$ on relations.

It remains to specify the actions of the functors on value and computation
objects and relations.
% Because we have augmented the definitions of object and relation for our model,
% we need to specify the actions of the functors $F$, $U$, $\times$, and $\arr$ on
% these objects and relations. 
% For instance, the functor $F$ now acts on \emph{value objects} $(A, M_A, i_A)$,
% which means we need to specify the monoid $M_{FA}$ and interpretation $i_{FA}$
% corresponding to the error domain $FA$.

\subsubsection{Constructions involving Perturbations}\label{sec:perturbation-constructions}

We first discuss the action of $\li$ on objects. Given a value object $(A, M_A,
i_A)$ we define the syntactic perturbations $M_{\li A}$ to be $\mathbb{N} \oplus
M_A$, where $\oplus$ denotes the free product of monoids (the coproduct in the
category of monoids). The intuition behind needing to include $\mathbb{N}$ comes
from the example of the downcast for $\iarr$. Recall that we adjusted the
squares corresponding to the DnL and DnR rules by adding $(\delta \circ
\eta)^\dagger$ on the side opposite the downcast. Thus, we need a syntactic
perturbation in $M_{\li A}$ that will be interpreted as $(\delta \circ
\eta)^\dagger$. We accomplish this by taking the coproduct with $\mathbb{N}$.
%
Then to interpret the perturbations as endomorphisms, by the universal property
of the coproduct of monoids it suffices to define two homomorphisms, one from
$\mathbb{N} \to \morbisimid{\li A}$ and one from $M_A \to \morbisimid{\li A}$.
The first homomorphism sends the generator $1 \in \mathbb{N}$ to the error
domain endomorpism $(\delta \circ \eta)^\dagger$, and the second sends $m_A \in
M_A$ to $\li(i_A(m_A))$ and we observe in both cases that the resulting
endomorphisms are bisimilar to the identity since the action of $\li$ on
morphisms preserves bisimilarity.

Now we discuss the action of $U$ on objects. Given a computation object $(B,
M_B, i_B)$, we define $M_{UB}$ to be $\mathbb{N} \oplus M_B$. The reason for
requiring $\mathbb{N}$ is related to the Kleisli arrow functor: to establish
quasi-representability of $U(c \arr d)$ given quasi-representability of $c$ and
$d$, we will need to turn a perturbation on $\li A$ into a perturbation on $U(A
\arr B)$. Since the perturbations for $\li A$ involve $\mathbb{N}$, so must the
perturbations for $UB$. The interpretation $i_{UB}$ of the perturbations on $UB$
works in the same manner as that of $\li A$.

The action of $\times$ on objects is as follows. Given $(A_1, M_{A_1}, i_{A_1})$
and $(A_2, M_{A_2}, i_{A_2})$, we define the monoid $M_{A_1 \times A_2} =
M_{A_1} \oplus M_{A_2}$. To construct the homomorphism $i_{A_1 \times A_2}$, we
appeal to the universal property and give homomorphisms $M_{A_1} \to
\morbisimid{A_1 \times A_2}$ and $M_{A_2} \to \morbisimid{A_1 \times A_2}$. The
former applies $i_{A_1}$ in the first component and the identity in the second
component, while the latter does the reverse with $i_{A_2}$.

The action of $\arr$ on objects is as follows. Given $(A, M_A, iA)$ and $(B,
M_B, i_B)$ we define the monoid $M_{A \arr B} = M_A^{op} \oplus M_B$. Then to
construct the interpretation as endomorphisms it suffices to define a
homomorphism $M_A^{op} \to \morbisimid{A \arr B}$ and $M_B \to \morbisimid{A
\arr B}$. The former takes an element $m_A$ and a predomain morphism $f : A \to
UB$ and returns the composition $f \circ i_A(m_A)$, while the latter is defined
similarly by post-composition with $i_B$.

\subsubsection{Constructions involving Value and Computation Relations}


\eric{What, if anything, should go here? Should we discuss composition of
value/computation relations? Actions of the functors on relations?}



\subsection{The Dynamic Type}
We now describe the denotation of the dynamic type. Analogously to how we
defined the dynamic type in Section \ref{sec:concrete-term-model}, the predomain
representing the dynamic type is defined using a combination of least fixpoint
and guarded recursion as the solution to the equation of \emph{predomains}
%
\[ D \cong \mathbb{N}\, + (D \times D)\, + \laterhs U(D \arr \li D). \]
%
As before, we name the constructors $\inat$, $\itimes$, and $\iarr$
respectively, and the upcasts $e_{\nat}$, $e_\times$, and $e_\to$ are the same
as the constructors except for $e_{\iarr}$ which includes a $\nxt$.

%
% We define $e_\mathbb{N} : \mathbb{N} \to D$ to be the $\text{nat}$ constructor,
% $e_\times : D \times D \to D$ to be $\text{times}$, and $e_\to : U(D \arr F D)$
% to be the morphism $\nxt$ followed by $\text{fun}$.

The ordering $\le_D$ is given by:
%
\begin{align*}
    \inat(n) \le \inat(n') 
        &\iff n = n' \\
    \itimes (d_1, d_2) \le \itimes (d_1', d_2')
        &\iff d_1 \le_D d_2 \text{ and } d_1' \le_D d_2'\\
    \iarr(\tilde{f}) \le \iarr(\tilde{f'}) 
        &\iff \later_t(\tilde{f}_t \le \tilde{f'}_t),
\end{align*}
%
and the bisimilarity relation is defined analogously.
%
We define three relations involving $D$ as follows. We define $\inat :
\mathbb{N} \rel D$ by $(n, d) \in \inat$ iff $e_\nat(n) \le_D d$. We similarly
define $\itimes : D \times D \rel D$ by $((d_1, d_2), d) \in \itimes$ iff
$e_\times(d_1, d_2) \le_D d$, and we define $\text{inj}_\to : U(D \arr \li D)
\rel D$ by $(f, d) \in \iarr$ iff $e_\to(f) \le_D d$.

Now we define the perturbations for $D$. Recall that to each predomain $A$ we
associate a monoid $M_A$ of perturbations and a homomorphism into the monoid of
endomorphisms bisimilar to the identity, and likewise for error domains. It may
seem as though we need to define the perturbations for $D$ by guarded recursion,
but in fact we define them via least-fixpoint in the category of monoids:
%
\( M_D \cong (M_{D \times D}) \oplus M_{U(D \to \li D)}. \)
%
We now explain how to interpret these perturbations as endomorphisms.

\eric{Check this}

We define $i_D : M_D \to \{ f : D \to D \mid f \bisim \id \}$ by induction as follows:
%
\begin{align*}
  i_D(m_\times) &= \lambda d. \text{case $d$ of }
    \{ \itimes (d_1, d_2) \To \itimes (i_{D \times D}(m_\times)(d_1, d_2)) \alt x \To x \} \\
  i_D(m_\to) &= \lambda d. \text{case $d$ of }
    \{ \iarr (\tilde{f}) \To \iarr (\lambda t. i_{U(D \to \li D)}(m_\to)(\tilde{f}_t)) \alt x \To x \} \\
\end{align*}
%
In the case of a perturbation on $D \times D$, we use the interpretation for the
perturbations of $D \times D$ as defined earlier in this section, which in turn
will use $i_D$ inductively. Similarly, for a perturbation on $U(D \to \li D)$ we
use the interpretation for perturbations on $U(D \to \li D)$.

% \begin{align*}
%  i_D(m_{\text{times}}, m_{\text{fun}}) &= \lambda d.\text{case $d$ of}  \\
%  &\alt \tnat(m) \mapsto \tnat(m) \\
%     &\alt \ttimes(d_1, d_2) \mapsto {\ttimes(i_{D \times D}(p_\text{times})(d_1, d_2))} \\
%     &\alt \tfun(\tilde{f}) \mapsto {\tfun(\lambda t. i_{U(D \to \li D)}(\tilde{f}_t))}
% \end{align*}
 
One can verify that this defines a homomorphism from $M_D \to \{ f : D \to D : f
\bisim \id \}$. We claim that the three relations $\inat$, $\itimes$, and
$\iarr$ satisfy the push-pull property. As an illustrative case, we establish
the push-pull property for the relation $\iarr$. We define $\pull_{\iarr} : M_D
\to M_{U(D \arr \li D)}$ by giving a homomorphism from $M_{D \times D} \to
M_{U(D \arr \li D)}$ and from $M_{U(D \to \li D)} \to M_{U(D \arr \li D)}$. The
former is the trivial homomorphism sending everything to the identity element,
while the latter is the identity homomorphism.
%
Conversely, we define $\push_{\iarr} : M_{U(D \arr \li D)} \to M_D$ as the
inclusion into the coproduct.
%
Showing that the relevant squares exist involving push and pull is
straightforward.

We next claim that the relations $\inat$, $\itimes$, and $\iarr$ are
quasi-left-representable, and that their lifts are quasi-right-representable.
Indeed, it is straightforward to see that they are quasi-left-representable
where all of the perturbations are taken to be the identity elements of the
respective monoids.
%
For quasi-right-representability, the only nontrivial case is $\li(\iarr)$.
Recall the defintion of the downcast for $\iarr$ given in Section
\ref{sec:term-interpretation}.

We know that in order for the DnR and DnL squares for $\li \iarr$ to exist, we
must insert a delay on the side opposite the downcast. In terms of syntactic
perturbations, this means that we take $\dellp$ and $\delrp$ in the definition
of quasi-right-representability to both be $\inl\, 1$, where $\inl$ is the
injection into the coproduct of monoids. We recall that the definition of
syntactic perturbations for $\li$ involves a coproduct with $\mathbb{N}$, and
that the interpretation homomorphism maps the generator $1$ to the endomorphism
of error domains $(\delta \circ \eta)^\dagger$.
%
With this choice for the perturbations, it is straightforward to show that the
DnL and DnR squares exist using the definition of the downcast and the
definition of the relation $\iarr$. 

It is also straightforward to establish the retraction property for each of
these three relations. In the case of $\iarr$, we have that the property holds
up to a delay.

\subsection{Interpreting Term Precision: Extensional Squares}

As introduced in the previous section, to obtain a model of term precision that
is oblivious to the stepping behavior of terms, we weaken our notion of square
to allow for the morphisms on either side to be synchronized to be in lock-step.
More formally, let $c_i : A_i \rel A_i'$ and $c_o : A_o \rel A_o'$.
%
% We define an extensional square between morphisms $f : A_i \to A_o$ and $g :
% A_i' \to A_o'$ to consist of:
%
% \begin{itemize}
%   \item A morphism $f' : A_i \to A_o$ with $f \bisim f'$.
%   \item A morphism $g' : A_i' \to A_o'$ with $g \bisim g'$.
%   \item A square $f' \ltsq{c_i}{c_o} g'$.
% \end{itemize}
%
We say that $f \ltsqbisim{c^i}{c_o} g$ if there exist $f'$ and $g'$ such that
%
\[ f \bisim f' \ltsq{c_i}{c_o} g' \bisim g. \]
%
We call a square $f \ltsqbisim{c^i}{c_o} g$ an \emph{extensional square}, and we
will call a square $f \ltsq{c^i}{c_o} g$ an \emph{intensional square}. If the
relations $c_i$ and $c_o$ are obvious from the context, we will simply write $f
\ltbisim g$. We make the analogous construction for the computation squares.
%
We now verify the existence of the squares for the cast rules. For UpL, the
relevant extensional square is obtained as follows:
%
% https://q.uiver.app/#q=WzAsNixbMCwwLCJBXzEiXSxbMSwwLCJBXzIiXSxbMiwwLCJBXzMiXSxbMCwxLCJBXzIiXSxbMSwxLCJBXzIiXSxbMiwxLCJBXzMiXSxbMCwzLCJlX2MiLDIseyJjdXJ2ZSI6MX1dLFsxLDQsIlxcZGVscmVfYyIsMl0sWzIsNSwiXFxwdXNoX3tjJ30oXFxkZWxyZV9jKSIsMix7ImN1cnZlIjoxfV0sWzAsMSwiYyIsMCx7InN0eWxlIjp7ImJvZHkiOnsibmFtZSI6ImJhcnJlZCJ9LCJoZWFkIjp7Im5hbWUiOiJub25lIn19fV0sWzEsMiwiYyciLDAseyJzdHlsZSI6eyJib2R5Ijp7Im5hbWUiOiJiYXJyZWQifSwiaGVhZCI6eyJuYW1lIjoibm9uZSJ9fX1dLFszLDQsIlxcbGVfe0FfMn0iLDIseyJzdHlsZSI6eyJib2R5Ijp7Im5hbWUiOiJiYXJyZWQifSwiaGVhZCI6eyJuYW1lIjoibm9uZSJ9fX1dLFs0LDUsImMnIiwyLHsic3R5bGUiOnsiYm9keSI6eyJuYW1lIjoiYmFycmVkIn0sImhlYWQiOnsibmFtZSI6Im5vbmUifX19XSxbMiw1LCJcXGlkIiwwLHsiY3VydmUiOi0xfV0sWzAsMywiZV9jIiwwLHsiY3VydmUiOi0xfV0sWzgsMTMsIlxcYmlzaW0iLDEseyJzaG9ydGVuIjp7InNvdXJjZSI6MjAsInRhcmdldCI6MjB9LCJzdHlsZSI6eyJib2R5Ijp7Im5hbWUiOiJub25lIn0sImhlYWQiOnsibmFtZSI6Im5vbmUifX19XSxbNiwxNCwiXFxiaXNpbSIsMSx7InNob3J0ZW4iOnsic291cmNlIjoyMCwidGFyZ2V0IjoyMH0sInN0eWxlIjp7ImJvZHkiOnsibmFtZSI6Im5vbmUifSwiaGVhZCI6eyJuYW1lIjoibm9uZSJ9fX1dXQ==
\[\begin{tikzcd}[ampersand replacement=\&,column sep=large]
	{A_1} \& {A_2} \& {A_3} \\
	{A_2} \& {A_2} \& {A_3}
	\arrow["c", "\shortmid"{marking}, no head, from=1-1, to=1-2]
	\arrow[""{name=0, anchor=center, inner sep=0}, "{e_c}"', curve={height=6pt}, from=1-1, to=2-1]
	\arrow[""{name=1, anchor=center, inner sep=0}, "{e_c}", curve={height=-6pt}, from=1-1, to=2-1]
	\arrow["{c'}", "\shortmid"{marking}, no head, from=1-2, to=1-3]
	\arrow["{\delre_c}"', from=1-2, to=2-2]
	\arrow[""{name=2, anchor=center, inner sep=0}, "{\push_{c'}(\delre_c)}"', curve={height=6pt}, from=1-3, to=2-3]
	\arrow[""{name=3, anchor=center, inner sep=0}, "\id", curve={height=-6pt}, from=1-3, to=2-3]
	\arrow["{\le_{A_2}}"', "\shortmid"{marking}, no head, from=2-1, to=2-2]
	\arrow["{c'}"', "\shortmid"{marking}, no head, from=2-2, to=2-3]
	\arrow["\bisim"{description}, draw=none, from=0, to=1]
	\arrow["\bisim"{description}, draw=none, from=2, to=3]
\end{tikzcd}\]
%
% Note that the horizontal composition of the squares is defined, as they are
% intensional squares.
We have used the push-pull structure on $c'$ to turn the perturbation on $A_2$
into a perturbation on $A_3$. Lastly we note that the bottom edge of the square
is equal to $c'$ by the fact that $c'$ is downward closed. The squares for UpR,
DnL, and DnR are obtained similarly.

% Now that we have defined an intensional model with an interpretation for the
% dynamic type, we can apply the abstract constructions introduced in Section
% \ref{sec:extensional-model-construction}. Doing so, we obtain an extensional
% model of gradual typing, where the squares are given by the ``bisimilarity
% closure'' of the intensional error ordering.

\eric{TODO: Does this belong here?}

Recall the equivalence rules for type precision given in Section \ref{sec:GTLC},
for example the rule that $(c_i \ra c_o)(c_i' \ra c_o')\equiv (c_ic_i' \ra
c_oc_o')$ To validate this rule in the model, we construct two squares $\id
\ltsqbisim{(c_i \ra c_o)(c_i' \ra c_o')}{(c_ic_i' \ra c_oc_o')} \id$ and $\id
\ltsqbisim{(c_ic_i' \ra c_oc_o')}{(c_i \ra c_o)(c_i' \ra c_o')} \id$. Since the
definition of extensional square allows us to specify intermediate morphisms
that are synchronized in the lock-step error ordering, it will be sufficient to
construct two intensional squares involving perturbations on both sides. This
motivates the definition of \emph{quasi-order-equivalence}.

\begin{definition}[quasi-order-equivalence]\label{def:quasi-order-equivalent}
  Let $c, c' : A \rel A'$. We say that $c$ and $c'$ are
  \textbf{quasi-order-equivalent}, written $c \qordeq c'$, if there exist
  perturbations $\delta^l_1, \delta^l_2 \in \pv_A$ and $\delta^r_1, \delta^r_2
  \in \pv_{A'}$ such that there is a square $\delta^l_1 \ltsq{c}{c'} \delta^r_1$
  and a square $\delta^l_2 \ltsq{c'}{c} \delta^r_2$.
\end{definition}
We make the analogous definition for error domain relations $d, d' : B \rel B'$.

Then we have the following result that shows that the functors
$U,\li,\times,\to$ are \emph{quasi-functorial} on relations:

\begin{lemma}
\begin{itemize}
  \item $U(d \comp d') \qordeq U(d) \comp U(d')$
  \item $F(c \comp c') \qordeq F(c) \comp F(c')$
  \item $(c \comp c') \to (d \comp d') \qordeq (c \to d) \comp (c' \to d')$
  \item $(c_1 \comp c_1') \times (c_2 \comp c_2') \qordeq (c_1 \times c_2) \comp (c_1'\times c_2')$
\end{itemize}
\end{lemma}
\begin{proof} See appendix \eric{TODO: make sure this is in the appendix}
\end{proof}


\subsection{Relational Adequacy}\label{sec:adequacy}

In this section, we prove the adequacy of the interpretation of term precision.
First we establish some notation. Fix a morphism $f : 1 \to \li \mathbb{N} \cong
\li \mathbb{N}$. We write $f \da n$ to mean that there exists $m$ such that $f =
\delta^m(\eta n)$ and $f \da \mho$ to mean that there exists $m$ such that $f =
\delta^m(\mho)$.

Recall that $\ltbisim$ denotes the relation on value morphisms defined as the
closure of the intensional error-ordering on morphisms under weak bisimilarity
on bboth sides. Specialized to the setting of morphisms from $1 \to U \li
\mathbb{N}$, we have $f \ltbisim g$ iff there exist $f'$ and $g'$ with
%
\[ f \bisim_{\li \mathbb{N}} f' \le_{\li \mathbb{N}} g' \bisim_{\li \mathbb{N}} g. \]
%

We first note that in this ordering, the semantics of error is not equivalent to
the semantics of the diverging term. The main result we would like to show is as
follows: If $f \ltbisim g : \li \mathbb{N}$, then:
\begin{itemize}
  \item If $f \da n$ then $g \da n$.
  \item If $g \da \mho$ then $f \da \mho$.
  \item If $g \da n$ then $f \da n$ or $f \da \mho$.
\end{itemize}

%
Unfortunately, this is actually not provable! Roughly speaking, the issue is
that this is a ``global'' result, and it is not possible to prove such results
in the guarded setting. In particular, if we tried to prove the above result in
the guarded setting we would run into a problem where we would have a natural
number ``stuck'' under a $\later$, with no way to get out the underlying number.
%
Thus, to prove our adequacy result, we need to pass to leave the guarded
setting. We have seen in Section \ref{sec:big-step-term-semantics} that clock
quantification makes this possible.

Recall that the definitions of our model, e.g., the free error domain, the
lock-step ordering, and weak bisimilarity, have all been parameterized by a
clock $k$. We saw in previously that there is an isomorphism between the
globalization of the free error domain and Capretta's coinductive delay monad:
$\li^{gl} \mathbb{N} \cong \delay(\mathbb{N} + {\mho})$.

Now consider the predomain $\mathbb{N}$ with ordering and bisimilarity both the
equality relation. We can define a ``global'' version of the lock-step error
ordering and the weak bisimilarity relation on elements of the $\li^{gl}
\mathbb{N}$; the former is defined by
%
\( x \ltls^{gl}_X y := \forall k. x[k] \ltls y[k], \)
%
and the latter is defined by
%
\( x \bisim^{gl}_X y := \forall k. x[k] \bisim y[k]. \)
%
On the other hand, we can define coinductively a ``lock-step error ordering''
relation on $\delay(\mathbb{N} + {\mho})$:
%
\begin{mathpar}
  \inferrule*[]
  { }
  {\tnow (\inr\, 1) \ledelay d}

  \inferrule*[]
  {x_1 \le_X x_2}
  {\tnow (\inl\, x_1) \ledelay \tnow (\inl\, x_2)}

  \inferrule*[]
  {d_1 \ledelay d_2}
  {\tlater\, d_1 \ledelay \tlater\, d_2}
\end{mathpar}
%
And we similarly define by coinduction a ``weak bisimilarity'' relation on
$\delay(\mathbb{N} + {\mho})$. This uses a relation $d \Da x_?$ between
$\delay(\mathbb{N} + {\mho})$ and $\mathbb{N} + {\mho}$ that is defined as $d
\Da n_? := \Sigma_{i \in \mathbb{N}} d = \tlater^i(\tnow\, n_?)$. Then weak
bisimilarity for the delay monad is defined coinductively by the rules
%
\begin{mathpar}
  \inferrule*[]
  {n_? \bisim_{\mathbb{N} + {\mho}} m_?}
  {\tnow\, n_? \bisimdelay \tnow\, m_? }

  \inferrule*[leftskip=1.5em]
  {d_1 \Da n_? \and n_? \bisim_{\mathbb{N} + {\mho}} m_?}
  {\tlater\, d_1 \bisimdelay \tnow\, m_? }

  \inferrule*[leftskip=1.5em]
  {d_2 \Da m_? \and n_? \bisim_{\mathbb{N} + {\mho}} m_?}
  {\tnow\, n_? \bisimdelay \tlater\, d_2}

  \inferrule*[leftskip=1.5em]
  {d_1 \bisimdelay d_2}
  {\tlater\, d_1 \bisimdelay \tlater\, d_2 }
  %
  % \inferrule*[]
  % {d_1 \Da x_? \and d_2 \Da y_? \and x_? \bisim_{X + 1} y_?}
  % {d_1 \bisimdelay d_2}
  %
  % \inferrule*[]
  % {d_1 \bisimdelay d_2}
  % {\tlater d_1 \bisimdelay \tlater d_2 }
\end{mathpar}
%
Note the similarity of these definitions to the corresponding guarded
definitions. By adapting Theorem 4.3 of \cite{kristensen-mogelberg-vezzosi2022}
to the setting of inductively-defined relations, we can show that both the
global lock-step error ordering and the global weak bisimilarity admit
coinductive definitions. In particular, modulo the above isomorphism between
$\li^{gl} X$ and $\delay(\mathbb{N} + {\mho})$, the global version of the
lock-step error ordering is equivalent to the lock-step error ordering on
$\delay(\mathbb{N} + {\mho})$, and likewise, the global version of the weak
bisimilarity relation is equivalent to the weak bisimilarity relation on
$\delay(\mathbb{N} + {\mho})$.

This implies that the global version of the term precision semantics for
$\li^{gl} \mathbb{N}$ agrees with the corresponding notion for
$\delay(\mathbb{N} + {\mho})$. Then adequacy follows by proving the
corresponding result for $\delay(\mathbb{N} + {\mho})$ which in turn follows
from the definitions of the relations.


% We have been writing the type as $\li X$, but it is perhaps more accurate to write it as $\li^k X$ to
% emphasize that the construction is parameterized by a clock $k$.

% Need : nat is clock irrelevant, as well as the inputs and outputs of effects
% Axioms about forcing clock
% Adapt prior argument to get that the defining of the global bisim
% and global lock-step error ordering are coinductive


\section{Discussion}\label{sec:discussion}

\subsection{Related Work}

We give an overview of current approaches to proving graduality and discuss
their limitations where appropriate.

%%\subsubsection{From Static to Gradual}

The methods of Abstracting Gradual Typing \cite{garcia-clark-tanter2016} and the
formal tools of the Gradualizer \cite{cimini-siek2016} provide a way to derive
from a statically-typed language a language that satisfies graduality by
construction. The main downside to these approaches lies in their inflexibility:
since the process in entirely mechanical, the language designer must adhere to
the predefined framework.  Many gradually typed languages do not fit into either
framework, e.g., Typed Racket \cite{tobin-hochstadt06, tobin-hochstadt08}, and
the semantics produced is not always the desired one.
%
Furthermore, while these frameworks do prove graduality of the resulting
languages, they do not show the correctness of the equational theory, which is
equally important to sound gradual typing.


% \subsubsection{Graduality via Embedding-Projection Pairs}

A line of work by New, Licata and Ahmed
\cite{new-ahmed2018,new-licata18,new-licata-ahmed2019} develops an axiomatic
account of gradual typing and proves graduality by interpreting type precision
$c : A \ltdyn A'$ as an \emph{embedding-projection pair}, that is, a pure
embedding function $e : A \to A'$ and a possibly erroring projection $p : \li A'
\multimap \li A$ such that $p \circ e = \id$ and $e \circ p \ltdyn \id$. At
first glance, our model of precision looks different: it is a \emph{relation}
$\sem{c} : A \rel A'$ with a quasi-representability structure that gives us
something close to $e$ and $p$ in an embedding-projection pair. The relationship
between these models is that \emph{if} the relation were \emph{truly}
representable we would have that $e$ and $p$ form a \emph{Galois connection} $e
\circ p \ltdyn \id$ and $\id \ltdyn p \circ e$. However, we have dropped the
stronger property of retraction from our analysis in this work. With true
representability, the relation $c$ is \emph{uniquely determined} by the
embedding $e$, but since we only have quasi-representability, we need to keep
the relation around explicitly. So the extra complexity of managing explicit
relations is a cost of the intensional reasoning that guarded type theory
introduces. 
%
The line of work by New, Licata, and Ahmed involved both denotational methods
and operational logical relations approaches.
% The New-Licata-Ahmed axiomatic approach to gradual typing can be
% applied to both operational and denotational semantics.


% Later, New, Licata and Ahmed \cite{new-licata-ahmed2019} extended this axiomatic
% treatment from call-by-name to call-by-value as well by giving an axiomatic
% theory of type and term precision in call-by-push-value. 

% \subsubsection{Denotational and Operational Approaches}


%\subsubsection{Double-Categorical Semantics}

On the denotational side, New and Licata \cite{new-licata18} showed that the
graduality proof could be modeled using semantics in certain kinds of double
categories, which are categories internal to the category of categories. A
double category extends a category with a second notion of morphism, often a
notion of ``relation'' to be paired with the notion of functional morphism, as
well as a notion of functional morphisms preserving relations. In gradual
typing, the notion of relation models type precision and the squares model the
term precision relation.
%
% This approach was influenced by the semantics of parametricity using reflexive
% graph categories \cite{ma-reynolds,dunphythesis,reynoldsprogramme}: reflexive
% graph categories are essentially double categories without a notion of
% relational composition. In addition to capturing the notions of type and term
% precision, the double categorical approach allows for a \emph{universal
% property} for casts: upcasts are the \emph{universal} way to turn a relation
% arrow into a function in a forward direction and downcasts are the dual
% universal arrow.
%
The double-categorical framework serves as a foundation on which we base our
model construction in this paper. However, our model is not actually a double
category: the lack of transitivity of bisimilarity means that we cannot compose
extensional squares horizontally.

With the notion of abstract categorical model for gradual typing in hand,
denotational semantics is then the work of constructing concrete models that
exhibit the categorical construction. New and Licata \cite{new-licata18} present
such a model using categories of $\omega$-CPOs, and this model was extended by
Lennon-Bertrand, Maillard, Tabareau and Tanter \cite{gradualizing-cic} to prove
graduality of a gradual dependently typed calculus $\textrm{CastCIC}^{\mathcal
G}$. This domain-theoretic approach meets our criterion of being a semantic
framework for proving graduality, but suffers from the limitations of classical
domain theory: the inability to model viciously self-referential structures such
as higher-order extensible state and similar features such as runtime-extensible
dynamic types.

% Since these features are quite common in dynamically typed languages, we have
% developed in this work a new denotational framework that can be extended to
% model these type system features.

%\subsubsection{Step-Indexed Logical Relations Models}
% The standard alternative to domain theory that scales to essentially
% arbitrary self-referential definitions is \emph{step-indexing} or its
% synthetic form of \emph{guarded recursion}. 

On the operational side, a series of works
\cite{new-ahmed2018, new-licata-ahmed2019, new-jamner-ahmed19}
developed step-indexed logical relations models of gradually typed languages.
Unlike classical domain theory, such step-indexed techniques can scale to
essentially arbitrary self-referential definitions, which means they can model
higher-order store and runtime-extensible dynamic types
\cite{appelmcallester01,ahmed06,neis09,new-jamner-ahmed19}. However, as is
common with operational approaches, their proof developments are highly
repetitive and technical, with each development formulating a logical relation
from first-principles and proving many of the same tedious lemmas without
reusable mathematical abstractions.
%
%
% Additionally, the logical relations constructed to prove graduality in prior
% work \cite{new-ahmed2018,new-licata-ahmed2019,new-giovannini-licata-2022} suffer
% from technical complications of requiring two separate expression relations, one
% that counts steps on the left and the other on the right, and there is no
% analogue of this in our approach. However, using two expression relations allows
% some but not all transitive reasoning of term precision to be recovered. In the
% future we aim to explore if this approach is feasible in guarded semantics.
%
This is addressed somewhat by Siek and Chen \cite{siek-chen2021}, who give a
proof in Agda of graduality for an operational semantics using a \emph{guarded
logic of propositions} shallowly embedded in Agda. The guarded logic simplifies
the treatment of step-indexed logical relations, but the approach is still
fundamentally operational, and so the main lemmas of the work are still tied to
the particular operational syntactic calculus being modeled.
% A further advantage of the denotational approach is that it easily validates
% equational reasoning, not just graduality, and it is completely independent of
% any particular syntax of gradual typing.

Our work combines the benefits of both the denotational and step-indexed
approaches. We take as a starting point the denotational approach of New and
Licata, but we work with guarded type theory (a synthetic form of step-indexing)
rather than classical domain theory. As we have seen, in the guarded setting the compositional
theory of New-Licata breaks down. 
% when accounting for steps, the casts are no
% longer embedding-projection pairs, and we cannot carry out compositional
% reasoning on terms when those terms differ by an arbitrary unknown number of
% steps. 
Our work provides a novel refinement to their theory that allows for some
compositional reasoning to be recovered even in the guarded setting.

% \subsubsection{Gradual Dependent Types}
Eremondi \cite{Eremondi_2023} uses guarded type theory to
define a syntactic model for a gradually-typed source
language with dependent types. By working in guarded type theory, they are
able to give an exact semantics to non-terminating computations in a language
which is logically consistent, allowing for metatheoretic results about the
source language to be proven.
%
Similarly to our approach, they define a guarded lift monad to model potentially-
nonterminating computations and use guarded recursion to model the dynamic type.
However, they do not give a denotational semantics to term precision and it is unclear
how to prove graduality in this setting.
Their work includes a formalization of the syntactic model in Guarded Cubical Agda.

\subsection{Comparison of Denotational and Operational Approaches}
\eric{Some of this could be folded into the previous section, but this was one of the
big-picture questions from one of the reviewers.}

Denotational methods have several benefits compared to operational approaches,
and this carries over to the setting of gradual typing. First, denotational
methods are compositional and reusable, and allow for the use of existing
mathematical constructs and theorems, e.g., partial orderings, monads, etc,
while operational methods tend to require a significant amount of boilerplate
work to be done from scratch in each new development.
%
As a specific example of the compositional nature of our approach, the treatment
of the cast rules in our work is more compositional than in previous work using
operational semantics. Recall from Section \ref{sec:towards-relational-model}
that the cast rules needed for the proof of graduality build in composition of
type precision derivations. Rather than proving these from scratch in the model,
we are able to take as primitive simpler versions of the cast rules whose
validity in the model is easier to establish. Then from these simpler rules, we
derive the original ones using compositional reasoning.

A further advantage of the denotational approach is that it is completely
independent of any particular syntax of gradual typing. This allows for reuse
across multiple languages and makes it particularly straightforward to
accommodate additions to a language: adding support for a new type amounts to
defining a new object and extending the dynamic type accordingly. It also may be
possible to model alternative cast semantics by changing the definition of the
dynamic type and some type casts.
%
In contrast, operational methods are not as readily extensible, generally
requiring adding cases to the logical relations and the inductive proofs. In a
mechanized metatheory, this manifests as a ``copy-paste'' reusability rather than
the true code reuse one obtains with a denotational semantics.
%
An additional benefit to the denotational approach is that it is trivial to
establish the validity of the $\beta$ and $\eta$ principles, because they are
equalities in the semantics, whereas they require tedious manual proof in the
operational approach.

Finally, while we advocate for a denotational approach, the overall structure of
our semantics could be applied to an operational logical relations proof. For
example, there could be a step-indexed logical relation for strong error
ordering and one for weak bisimilarity, corresponding to the fact that objects
in our denotational model have an error ordering and a bisimilarity relation.
Then the precision rules for casts could be similarly validated using
perturbations as we did in this work.


\subsection{Mechanization}\label{sec:mechanization}
% Discuss Guarded Cubical Agda and mechanization efforts
In parallel with developing the theory discussed in this paper, we have
developed a partial formalization of our results in Guarded Cubical Agda
\cite{veltri-vezzosi2020}.
%
We formalized the major components of the definition of the concrete model
described in Section \ref{sec:concrete-relational-model}: predomains, error
domains, morphisms, relations, squares, using the guarded
features to define the free error domain, the lock-step error ordering, and weak
bisimilarity. We also formalized the no-go theorem from Section
\ref{sec:towards-relational-model}.

Building on these definitions, we formalized the notion of semantic
perturbations and push-pull structures as well as quasi-representable relations,
culminating in the definition of value and computation types and relations as
introduced in Section \ref{sec:concrete-relational-model}.
%
% culminating in the definition of value and computation types as predomains
% (resp. error domains) equipped with a monoid of syntactic perturbations with an
% interpretation homomorphism into the semantic perturbations. 
%
We constructed the predomain for the dynamic type via mixed induction and
guarded recursion, and defined its monoid of perturbations. We also defined the
relations $\inat$, $\itimes$, and $\iarr$, and proved that they are
quasi-representable.

% We have not yet formalized the constructions involving value and computation
% relations: showing that these relations compose, and defining the actions of the
% functors $\li$, $U$, $\times$, and $\arr$ on the relations. We have also not yet
% formalized the syntax-to-semantics translation.

% \max{need to say what's not formalized}

% We plan to formalize the construction of perturbations and 
% quasi-representable relations, but we have yet to decide
% whether to follow the approach we take in this work and define
% the abstract notion of intensional model and formalize the constructions in that setting,
% and then apply those abstract constructions to the concrete model.
% Alternatively, it may be better from a mechanization standpoint
% to carry out those abstract constructions explicitly in the concrete model,
% i.e., our representation of objects in the concrete model of predomains
% would include a field for the perturbations and our notion of relations
% would include fields for the push-pull property and quasi-representability.
% We leave this investigation to future work.

Lastly, we have formalized the big-step term semantics discussed in Section
\ref{sec:big-step-term-semantics} and the adequacy of the relational model
discussed in Section \ref{sec:adequacy}. This required us to add axioms about
clock quantification as well as axioms asserting the \emph{clock-irrelevance} of
booleans and natural numbers since as of this writing these axioms are not
built-in to Guarded Cubical Agda. These axioms are discussed in prior work on
guarded type theory \cite{atkey-mcbride2013, kristensen-mogelberg-vezzosi2022}.
The mechanization of adequacy also required us to formalize some essential
lemmas involving clocks and clock-irrelevance; we are considering later
refactoring these as part of a ``standard library'' for Guarded Cubical
Agda.
% \max{say these axioms are taken from prior work and cite that}

The remaining formalization work is the following:
\begin{enumerate}
    \item Showing that the functors $\times$ and $\arr$ preserve
    quasi-representability, which requires tedious reasoning about the Kleisli actions of
    $\times$ and $\arr$. These results are proved in the Appendix.
    \item Verifying the rules in the model corresponding to the equations for
    type precision derivations (see the bottom of Figure \ref{fig:gtlc-syntax}).
    These are also included in the Appendix.
    \item Formalizing the syntax-to-semantics translation.
\end{enumerate} 

\begin{comment}
% prove graduality in the syntax of 
% GTLC, which involves the construction of the abstract model described in 
% \ref{sec:concrete-model} and the extensional model with external dynamic 
% type. We also plan to formalize the adequacy result in \ref{sec:appendix-adequacy}.

% step-2 
Then we plan to construct the step-2 intensional model. Besides all the 
data in step-1, we need to include perturbations, functors $\times$, $\arr$, $U$, and $F$ that preserve 
perturbations and push/pull properties for all morphisms on value and 
computation types. Notice that for any object $A$ which has value type, 
we will take not only the monoid of perturbations $P^V_A$ and the monoid 
homomorphism $\ptbv_A : \pv_A \to \vf(A,A)$ on itself, but also $P^C_{F A}
$ and $\ptbe_{F A} : \pe_{F A} \to \ef(F A,F A)$ on $F A$, which have 
computation types. Similarly, for any computation object $B$, we will 
construct the perturbations on $U B$ besides the monoid $P^C_B$ and 
monoid homomorphism $\ptbe_B : \pe_B \to \ef(B,B)$. Also, for functors 
that preserves perturbations, we need to include the ones in the context 
of Kleisli category. For this part, we need to define the perturbation on 
not only the objects itself, but also the global lift and delay of objects, 
which requires us to provide each piece of supporting constructor. This step 
and futher steps towards to the model construction are still 
work-in-progress, but once it's finished, we will provide a complete 
framework which takes formalization on an explicit type and obtains an 
extensional model.

% step-3
In the step-3 intensional model, we will enhance it with 
quasi-representability. For any value relation $c : A \rel A'$, we need 
to show that there exists a left-representation structure for $c$ and a 
right-representation structure for $F\ c$. Correspondingly, for any 
computation relation $d : B \rel B'$, we will show there exists a 
right-representation structure for $d$ and a left-representation 
structure for $U\ d$. As we define the quasi-representability for value 
and computation relation, we will construct the quasi-representability on 
the function and product of the relation, which makes it necessary to 
have the dual version of quasi-representability.

% step-4 construct a concrete dynamic type and apply it to the abstract model
After defining the abstract model and its interface, we will model GTLC 
by providing explicit construction triples of dynamic type at each step, 
which includes defining Dyn as a predomain, its pure and Kleisli 
perturbation monoids, push/pull property for pure and Kleisli 
perturbation, as well as quasi-representability. The 
quasi-representability involves explicit rules which show that Nat is 
more precise than Dyn (Inj-Nat) and Dyn $\to$ Dyn is more precise than 
Dyn (Inj-Arr). Currently, we have formalized the concrete construction of 
Dyn in Cubical Agda and it was more challenging than expected because we 
define Dyn using the technique of guarded recursion and fixed point, which 
means that every time we analyze the case inside of Dyn, we need to unfold 
it and add corresponding proof. 

% adequacy
Besides the abstract model and its concrete construction on dynamic type, 
we will also formalize the adequacy result in \ref{sec:appendix-adequacy}, 
which involves clock quantification of the lift monad, the weak bisim 
relation, and the lock-step error ordering. In order to prove adequacy, 
we will first prove that the global lift of X is isomorphic to Delay(1 + X)
whether X is clock-irrelevant or not. Then, we aim to prove the equivalence 
between the global lock-step error ordering and the error ordering observed 
in Delay(1 + X) and equivalence between the global weak bisimilarity 
relation and the weak bisimilarity relation on Delay(1 + X). We have 
finished some prerequisite proofs on clock quantification and postulated 
some theorems on clock globalization.
\end{comment}



% \subsection{Synthetic Ordering}
% \max{cut this subsection if we need space}
% A key to managing the complexity of our concrete construction is in
% using a \emph{synthetic} approach to step-indexing rather than working
% analytically with presheaves. This has helped immensely in our ongoing
% mechanization in cubical Agda as it sidesteps the need to formalize
% these constructions internally. 
% %
% However, there are other aspects of the model, the bisimilarity and
% the monotonicity, which are treated analytically and are similarly
% tedious.
% %
% It may be possible to utilize further synthetic techniques to reduce
% this burden as well, and have all type intrinsically carry a notion of
% bisimilarity and ordering relation, and all constructions to
% automatically preserve them.
% %
% A synthetic approach to ordering is common in (non-guarded) synthetic
% domain theory and has also been used for synthetic reasoning for cost
% models \cite{fiore_1997,GrodinNSH24}.

\subsection{Future Work}

% \max{Expand on this, e.g. what some of the challenges might be and what would be reusable}
% In the future, we plan to apply our approach to give a denotational
% semantics for languages that feature higher-order state or
% runtime-extensible dynamic typing
% \cite{DBLP:journals/corr/abs-2210-02169} as well as richer type
% disciplines such as gradual dependent types and effect systems.

Our immediate next step is to apply our approach to give a denotational
semantics to gradually-typed languages with advanced type systems including
higher-order state or runtime-extensible dynamic typing
\cite{DBLP:journals/corr/abs-2210-02169}, as well as richer type disciplines
such as gradual dependent types and effect systems. This will involve adapting
prior operational models based on step-indexing (e.g.,
\cite{new-giovannini-licata-2022}) to the denotational setting and using guarded
type theory to obtain solutions to guarded domain equations as we did in this
work for the denotation of the dynamic type. The generality of our techniques
should allow for many of the constructions to be reused across different
languages. For instance, the free error domain construction can be easily
extended to model effects besides error and stepping
\cite{guarded-interaction-trees, probabilistic-fpc-guarded}.
We also aim to complete our Agda formalization and evolve it into a
reusable framework for mechanized denotational semantics of gradually-typed
languages.

This work has focused on the ``Natural'' semantics of casts , which
validate the full call-by-value $\eta$ laws for types, but other cast
semantics have been proposed such as eager
(\cite{herman-tomb-flanagan-2010}) and transient (\cite{transient})
cast semantics which trade off certain $\eta$ equalities for other
benefits such as early error detection or reduced runtime
overhead\cite{deepandshallowtypes,newlicataahmed19}. It would be a
good test of the generality of our semantic framework to see if it can
model these alternative semantics by adjusting the semantics of types
and casts, and providing proofs of weakened $\eta$ principles.

Additionally, while our focus in this work has been on reasoning up to
weak bisimilarity, the presence of explicit step counting in the model
could be viewed as a form of \emph{cost semantics}, where a runtime
cost is incurred from inspecting the dynamic type. This could possibly
be used to verify that cast optimizations such as space efficient
implementations \cite{herman-tomb-flanagan-2010} are not only
extensionally correct, but have a lower abstract cost.

% \max{Expand on this, e.g. what some of the challenges might be and what would be reusable}
% In the future, we plan to apply our approach to give a denotational semantics
% to gradually-typed languages that feature higher-order state or runtime-extensible dynamic
% typing \cite{DBLP:journals/corr/abs-2210-02169} as well as richer type
% disciplines such as gradual dependent types and effect systems. This will
% involve adapting prior operational models based on step-indexing (e.g.,
% \cite{new-giovannini-licata-2022}) to the denotational setting and using guarded
% type theory to obtain solutions to guarded domain equations as we did in this
% work for the denotation of the dynamic type. The generality of our techniques
% should allow for many of the constructions to be reused across different
% languages. For instance, the free error domain construction can be easily
% extended with additional cases to model effects besides error and stepping.
% Additionally, we aim to complete our Agda formalization and evolve it into a
% reusable framework for mechanized denotational semantics of gradually-typed
% languages.

%% to gradually-typed
%% languages with algebraic effects, building on prior work on gradual typing for effect handlers
%% \cite{greff}. In particular, that work proves graduality via a complicated step-indexed logical relation,
%% and we hope to prove their results by building a denotational model for GrEff.
%% This would serve as a step towards applying our techniques to prove graduality for languages
%% with other advanced features.

%% The extensional model we construct differs from the usual notion of extensional
%% model considered in prior work on gradual typing in that it lacks horizontal composition of squares.
%% We would like to clarify the relationship between our notion of model and prior extensional models,
%% with the aim of determining whether our approach could allow for the construction of such a model.


\bibliographystyle{ACM-Reference-Format}
\begin{DIFnomarkup}
\bibliography{references}
\end{DIFnomarkup}

\newpage
\appendix
\section{Call-by-push-value}

In CBPV models, all the type constructors are interpreted as functors:
\begin{enumerate}
\item $\to : \op\calV \times \calE \to \calE$
\item $\times : \calV \times \calV \to \calV$
\item $F : \calV \to \calE$
\item $U : \calE \to \calV$
\end{enumerate}
That is, they all have functorial actions on \emph{pure} morphisms of
value types and \emph{linear} morphisms of computation types.
%
We use these functorial actions extensively in the construction of
casts and their corresponding perturbations. But when defining
downcasts of value types and upcasts of computation types, we
additionally need a second functorial action of these categories:
functoriality in \emph{impure} morphisms of value types and
\emph{non-linear} morphisms of computation types. These notions of
morphism are given by the \emph{Kleisli} categories $\calVk$ and
$\calEk$ which have value types and computation types as objects but
morphisms are defined as
\[ \calVk(A,A') = \calE(F A, FA')\]
\[ \calEk(B,B') = \calV(U B, U B')\]
with composition given by composition in $\calE/\calV$.  That is we
need to define a second functorial action, that agrees with the above
on objects for these Kleisli categories:
\begin{enumerate}
\item $\tok : \op\calVk \square \calEk \to \calEk$
\item $\timesk : \calVk \square \calVk \to \calVk$
\item $\Fk : \calVk \to \calEk$
\item $\Uk : \calEk \to \calVk$
\end{enumerate}
Note that rather than the product of categories we use the ``funny
tensor product'' $\square$. This is because the action on
impure/non-linear morphisms for $\tok/\timesk$ do not satisfy ``joint
functoriality'' but instead only ``separate functoriality'', meaning
we give rather than an action on morphisms in both categories
simultaneously instead an action on each argument categories morphisms
with the object in the other category fixed. The existence of these
functorial actions for $\tok$ and $\timesk$ is reliant on the
\emph{strength} of the adjunction. We describe them using the internal
language of CBPV in order to more easily verify their
existence/functoriality:
\begin{enumerate}
\item For $\tok$ we define for $\phi : \calE(F A,F A')$ and $B \in \calE$ the morphism $\phi \tok B : \calV(U(A' \to B),U(A\to B))$ as
  \[ t:U(A'\to B) \vdash \phi \tok B = \{ \lambda x. x' \leftarrow \phi\,[\ret x]; ! t x'\} : U(A \to B) \]
  and for $A \in \calV$ and $f : \calV(UB,UB')$ we define $A \tok f : \calV(U(A \to B),U(A\to B'))$ as
  \[ t : U(A \to B) \vdash A \tok f = \{ \lambda x. !f[\{ ! t x \}]\} \]
\item For $\timesk$ we define for $\phi : \calE(F A_1,FA_2)$ and $A' \in \calV$ the morphism $\phi \timesk A_2$ as
  \[ \bullet : F(A_1\times A_2) \vdash \phi \timesk A_2 = (x_1,x_2) \leftarrow \bullet; x_1' \leftarrow \phi[\ret x_1]; \ret (x_1',x_2) : F(A_1'\times A_2)\]
  and $A_1 \timesk \phi$ is defined symmetrically.
\item For $\Uk$ we need to define for $f : \calV(UB,UB')$ a morphism $\Uk f : \calE(FUB,FUB')$. This is simply given by the functorial action of $F$: $\Uk f = F(f)$
\item Similarly $\Fk \phi = U\phi$
\end{enumerate}

Functoriality in each argument is easily established, meaning for
example for the function type is functorial in each argument:
\begin{enumerate}
\item $(\phi \circ \phi') \tok B = (\phi' \tok B) \circ (\phi \tok B)$
\item $\id \tok B = \id$
\item $A \tok (f \circ f') = (A \tok f) \circ (A \tok f)$
\item $A \tok \id = \id$
\end{enumerate}

Finally, note that all of these constructions lift to squares in a
double CBPV model since the squares themselves form a CBPV model and
the projection functions preserve CBPV structure. For instance, given a square
$\alpha : \phi \ltdyn_{F c_o}^{F c_i} \phi'$ and a horizontal morphism $d : B \rel B'$ of appropriate type, we get a square
\[ \alpha \tok d : \phi \tok B \ltdyn_{U(c_o \to d)}^{U(c_i \to d)} \phi' \tok B' \]

\section{Details of the Construction of an Extensional Model}

In Section \ref{sec:extensional-model-construction}, we outline the construction
of an extensional model of gradual typing starting from a step-1 intensional model.
In this section, we provide the details for each of the constructions mentioned there.

\begin{lemma}\label{lem:step-1-model-to-step-2-model}
Let $\mathcal M$ be a \hyperref[def:step-1-model]{step-1 intensional model} with dyn.

Then we can construct a \hyperref[def:step-2-model]{step-2 intensional model} with dyn.
\end{lemma}
\begin{proof}
    % Write 
    % %
    % \[ \mathcal M = (\vf, \vsq, \ef, \esq, \Ff, \Fsq, \Uf, \Usq, \arrf, \arrsq). \] 
    % %

    Define a step-2 model $\mathcal M'$ as follows:
    \begin{itemize}
      \item Value objects are tuples consisting of:
      \begin{itemize}
        \item A value object $A$ in $\vf$ 
        \item A monoid of ``pure'' perturbations $P_A$ 
        \item A homomorphism of monoids $\ptb_A : P_A \to \{ f \in \vf(A, A) \mid f \bisim \id_A \}$
        \item A monoid of ``impure'' perturbations $P^K_A$ that contains a distinguished element $\delta^*$
        \item A homomorphism of monoids $\ptbk_A : P^K_A \to \{ \phi \in \ef(FA, FA) \mid \phi \bisim \id_{FA} \}$
        such that $\ptbk_A(\delta^*) = \delta_A^*$
      \end{itemize}  

      \item Computation objects are tuples consisting of:
      \begin{itemize}
        \item A computation object $B$ in $\ef$
        \item A monoid of ``pure'' perturbations $P_B$
        \item A homomorphism of monoids $\ptb_B : P_B \to \{ \phi \in \ef(B, B) \mid \phi \bisim \id_B \}$
        \item A monoid of ``impure'' perturbations $P^K_B$
        \item A homomorphism of monoids $\ptbk_B : P^K_B \to \{ g \in \vf(UB, UB) \mid g \bisim \id_{UB} \}$.
      \end{itemize}

      \item Morphisms are given by morphisms of the underlying objects in $\vf$ and $\ef$, respectively
      %, i.e.,
      % \[ \vf'((A, P_A, \ptb_A, P^K_A, \ptbk_A), (A', P_{A'}, \ptb_{A'}, P^K_{A'}, \ptbk_{A'})) = \vf(A, A') \]
      %
      % and likewise for computations.
   
    \end{itemize}

    Before introducing the relations, we make a definition.

    \begin{definition}[push-pull structure]
      Let $c : A \rel A'$ be a value relation of $\mathcal M$. A \emph{value push-pull structure} $\piv_c$ for $c$ consists of:
      \begin{itemize}
        \item A function $\push : P_A \to P_{A'}$ 
              such that for all $\delta^l \in P_A$ we have $\delta^l \ltdyn_c^c \push(\delta^l)$.
        \item A function $\push^K : P^K_A \to P^K_{A'}$ 
              such that for all $\delta^K_l \in P^K_A$ we have $\delta^K_l \ltdyn_{Fc}^{Fc} \push(\delta^K_l)$.
        \item A function $\pull : P_{A'} \to P_A$
              such that for all $\delta_r \in P_{A'}$ we have $\pull(\delta^r) \ltdyn_{c}^c \delta^r$.
        \item A function $\pull^K : P^K_{A'} \to P^K_A$
              such that for all $\delta^K_r \in P^K_{A'}$ we have $\pull(\delta^K_r) \ltdyn_{Fc}^{Fc} \delta^K_r$.
      \end{itemize}

      For $d : B \rel B'$ a computation relation, we define a \emph{computation push-pull structure} $\pie_d$ for $d$
      in an analogous manner.
    \end{definition}


    Now we continue with the description of the construction:
    \begin{itemize}

      \item The objects of $\vsq'$ (i.e., the value relations) are pairs consisting of:
      \begin{itemize}
        \item A value relation $c \in \vsq$
        \item A push-pull structure $\piv_c$ for $c$
      \end{itemize}

      The objects of $\esq'$ are defined analogously.
            
      \item The morphisms of $\vsq'$ and $\esq'$ are given by the morphisms of $\vsq$ and $\esq$.
      
      % Functors \times, +, F, U, arrow
     
      % \item We define $F$ on objects by $F (A, \pv_A, \ptbv_A) = (FA, (1 + \pv_A), h_F)$
      % where $1$ is the trivial monoid, $+$ is the coproduct in the category of monoids, and $h_F$ is the homomorphism defined as follows:

      % \item We define $U$ on objects by $U (B, \pe_B, \ptbe_B) = (UB, \pe_B, h_U)$
      % where $h_U(p_B) = U(\ptbe_B(p_B))$.
      
      % \item We define $(A, \pv_A, \ptbv_A) \arr (B, \pe_B, \ptbe_B) = (A \arr B, \pv_A \times \pe_B, h_\arr)$
      % where $\times$ is the product in the category of monoids, and $h_\arr$ is defined by 
      % $h_\arr(p_A, p_B) = \ptbv_A(p_A) \arr \ptbe_B(p_B)$.
    \end{itemize}
\end{proof}

%%%%%%%%%%%%%%%%%%%%%%%%%%%%%%%%%%%%%%%%%%%%%%%%%%%%%%%%%%%%%%%%%%%%%%%%%%%%%%

\begin{lemma}\label{lem:step-2-model-to-step-3-model}
  Let $\mathcal M$ be a \hyperref[def:step-2-model]{step-2 intensional model}.

  Then we can construct a \hyperref[def:step-3-model]{step-3 intensional model}.
\end{lemma}
\begin{proof}
  Write 
  %
  \[ \mathcal M = (\vf, \vsq, \ef, \esq, \Ff, \Fsq, \Uf, \Usq, \arrf, \arrsq). \] 
  %

  We begin with a definition.

  \begin{definition}[representation structure]
  Let $c : A \rel A'$ be a value relation. A \emph{left-representation structure} $\rho^L_c$ for $c$ consists of
  a value morphism $e_c \in \vf(A, A')$ such that $c$ is quasi-left-representable by $e_c$ (see Definition \ref{def:quasi-left-representable}).
  
  Likewise, let $d : B \rel B'$. A \emph{right-representation structure} $\rho^R_d$ for $d$ consists of
  a computation morphism $p_d \in \ef(B', B)$ such that $d$ is quasi-right-representable by $p_d$ (see Definition \ref{def:quasi-right-representable}).
  \end{definition}
  
  (Notice that the direction of the morphism is opposite in the definition of right-representation structure.)

  We define a step-3 model $\mathcal M'$ as follows:
  \begin{itemize}
    \item The objects of $\mathcal M'$ are defined to be the same as the objects of $\mathcal M$.
    \item The value and computation morphisms in $\mathcal M'$ are the same as those of $\mathcal M$.
    \item A value relation is defined to be a tuple $(c, \rho^L_c, \rho^R_{Fc})$ where:
    \begin{itemize}
      \item $c$ is a value relation in $\mathcal M$, and 
      \item $\rho^L_c$ a left-representation structure for $c$, and 
      \item $\rho^R_{Fc}$ a right-representation structure for $Fc$.
    \end{itemize}
    \item Likewise, a computation relation is defined to be a tuple $(d, \rho^R_d, \rho^L_{Ud})$ with
    \begin{itemize}
      \item $d$ a computation relation in $\mathcal M$
      \item $\rho^R_d$ a right-representation structure for $d$
      \item $\rho^L_{Ud}$ a left-representation structure for $Ud$
    \end{itemize}
    \item Morphisms of value relations (i.e., the value squares) are defined by simply
    ignoring the representation structures. That is, a morphism of value relations
    $\alpha \in \vsq'((c, \rho^L_c, \rho^R_{Fc}), (c' \rho^L_{c'}, \rho^R_{Fc'}))$ is simply a morphism of value
    relations in $\vsq(c, c')$. Likewise for computations.
  \end{itemize}
\end{proof}

% Now we define the functors $F$, $U$, $\times$, and $\arr$.

% On objects, the behavior is the same as the respective functors in $\mathcal M$.

% For relations, we define 
% $\Fsq' (c, \rho^L_c, \rho^R_{Fc}) = (\Fsq c, \rho^R_{Fc}, UF(\rho^L_c))$ and
% $\Usq' (d, \rho^R_d, \rho^L_{Ud}) = (\Usq d, \rho^L_{Ud}, FU(\rho^R_d))$.

% We define $(c, \rho^L_c) \arr (d, \rho^R_d) = (c \arr d, \rho^R_{c \arr d})$.



% We now verify that the construction meets the requirements of a step-3 model.
% First, we check that composition of value relations (resp. computation relations)
% is well-defined.


%%%%%%%%%%%%%%%%%%%%%%%%%%%%%%%%%%%%%%%%%%%%%%%%%%%%%%%%%%%%%%%%%%%%%%%%%%%%%%


\begin{lemma}\label{lem:step-4-model-to-extensional-model}
  Let $\mathcal M$ be a \hyperref[def:step-4-model]{step-4 intensional model}.
  Then we can define an extensional model.
\end{lemma}
\begin{proof}
  
  
  % More formally, we define an extensional model $\mathcal M_e$ as follows.
  % \begin{itemize}
  %   \item 
  % \end{itemize}
\end{proof}



\section{Adequacy}\label{sec:appendix-adequacy}

In this section, we show an adequacy result for the extensional model of GTT we obtained by
applying the abstract construction introduced in Section
\ref{sec:extensional-model-construction} to the concrete model

First we establish some notation. Fix a morphism $f : 1 \to \li \Nat \cong \li \Nat$.
We write that $f \da n$ to mean that there exists $m$ such that $f = \delta^m(\eta n)$
and $f \da \mho$ to mean that there exists $m$ such that $f = \delta^m(\mho)$.

Recall that $\ltls$ denotes the relation on value morphisms defined as the bisimilarity-closure
of the intensional error-ordering on morphisms.
More concretely, we have $f \ltls g$ iff there exists $f'$ and $g'$ with

\[ f \bisim f' \le g' \bisim g. \]

The result we would like to show is as follows:
\begin{lemma}
If $f \ltls g : \li \Nat$, then:
\begin{itemize}
  \item If $f \da n$ then $g \da n$.
  \item If $g \da \mho$ then $f \da \mho$.
  \item If $g \da n$ then $f \da n$.
\end{itemize}
\end{lemma}

Unfortunately, this result is actually not provable!
Roughly speaking, the issue is that this is a ``global'' result, and it is not possible
to prove such results inside of the guarded setting. 
In particular, if we tried to prove a result such as the above in the guarded setting,
we would run into a problem where we would have a natural number ``stuck'' under a $\later$
with no way to get at the underlying number.

Thus, to prove our adequacy result, we need to leave the guarded setting and pass back
to the normal set-theoretic world.
As mentioned in the Technical Background section (Section \ref{sec:sgdt}), we can do this
using \emph{clock quantification}.

Recall that all of the constructions we have made in SGDT take place in the context of a clock $k$.
All of our uses of the later modality and guarded recursion happen with respect to this clock.
For example, consider the definition of the lift monad by guarded recursion in Section \ref{TODO}.
% We define the lift monad $\li^k X$ as the guarded fixpoint of $\lambda \tilde{T}. X + 1 + \later^k_t (\tilde{T}_t)$.
We can view this definition as being parameterized by a clock $k$: $\li^k : \type \to \type$.
Then for $X$ satisfying a certain technical requirement, we can define the ``global lift'' monad as $\li^{gl} X = \forall k. \li^k X$.


It can be shown that the global lift monad is isomorphic to the so-called Delay monad of Capretta \cite{TODO}.


% We have been writing the type as $\li X$, but it is perhaps more accurate to write it as $\li^k X$ to
% emphasize that the construction is parameterized by a clock $k$.



\end{document}

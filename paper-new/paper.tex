\documentclass[sigconf,anonymous,review,screen,9pt]{acmart}
\let\Bbbk\relax

\usepackage{quiver}
\usepackage{mathpartir}
% \usepackage{tikz-cd}
\usepackage{enumitem}
\usepackage{wrapfig}
\usepackage{fancyvrb}
\usepackage{comment}
\usepackage{array}


%% Rights management information.  This information is sent to you
%% when you complete the rights form.  These commands have SAMPLE
%% values in them; it is your responsibility as an author to replace
%% the commands and values with those provided to you when you
%% complete the rights form.
\setcopyright{acmcopyright}
\copyrightyear{2018}
\acmYear{2018}
\acmDOI{10.1145/1122445.1122456}

%% These commands are for a PROCEEDINGS abstract or paper.
\acmConference[Woodstock '18]{Woodstock '18: ACM Symposium on Neural
  Gaze Detection}{June 03--05, 2018}{Woodstock, NY}
\acmBooktitle{Woodstock '18: ACM Symposium on Neural Gaze Detection,
  June 03--05, 2018, Woodstock, NY}
\acmPrice{15.00}
\acmISBN{978-1-4503-XXXX-X/18/06}

\newcommand{\To}{\Rightarrow}
\newcommand{\inl}{\mathsf{inl}}
\newcommand{\inr}{\mathsf{inr}}
\newcommand{\alt}{\mathrel{\bf \,\mid\,}}

\newcommand{\ob}{\text{Ob}}


\newcommand{\extlc}{\text{Ext-}\lambda}
\newcommand{\extlcm}{\text{Ext-}\lambda^{-\text{trans}}}
\newcommand{\extlcmm}{\text{Ext-}\lambda^{-\text{trans}-\text{cast}}}
\newcommand{\extlcprime}{\text{Ext-}\lambda'}
\newcommand{\intlc}{\text{Int-}\lambda}
\newcommand{\intlcbisim}{\text{Int$_\approx$-}\lambda}
\newcommand{\erase}[1]{\lfloor {#1} \rfloor}


\newcommand{\uarrowl}{\mathrel{\rotatebox[origin=c]{-30}{$\leftarrowtail$}}}
\newcommand{\uarrowr}{\mathrel{\rotatebox[origin=c]{60}{$\leftarrowtail$}}}
\newcommand{\darrowl}{\mathrel{\rotatebox[origin=c]{30}{$\twoheadleftarrow$}}}
\newcommand{\darrowr}{\mathrel{\rotatebox[origin=c]{120}{$\twoheadleftarrow$}}}
\newcommand{\vuarrow}{\mathrel{\rotatebox[origin=c]{-90}{$\leftarrowtail$}}}
\newcommand{\vdarrow}{\mathrel{\rotatebox[origin=c]{90}{$\twoheadleftarrow$}}}

% Types, terms, and precision 

\newcommand{\dyn}{{?}}
\newcommand{\nat}{\text{Nat}}
\newcommand{\bool}{\text{Bool}}
\newcommand{\ra}{\rightharpoonup}
\newcommand{\Ret}[1]{\mathsf{Ret\,}{#1}}
\newcommand{\hole}[1]{\bullet \colon {#1}}
\newcommand{\dyntodyn}{\dyn \ra\, \dyn}


\newcommand{\up}[2]{\langle{#2}\uarrowl{#1}\rangle}
\newcommand{\dn}[2]{\langle{#1}\darrowl{#2}\rangle}

\newcommand{\upc}[1]{\text{up}\,{#1}\,}
\newcommand{\dnc}[1]{\text{dn}\,{#1}\,}


% \newcommand{\ret}{\mathsf{ret}}
\newcommand{\err}{\mho}
\newcommand{\zro}{\textsf{zro}}
\newcommand{\suc}{\textsf{suc}}
\newcommand{\lda}[2]{\lambda {#1} . {#2}}

\newcommand{\injarr}[1]{\textsf{Inj}_\ra ({#1})}
\newcommand{\injnat}[1]{\textsf{Inj}_\text{nat} ({#1})}
\newcommand{\casenat}[4]{\text{Case}_\text{nat} ({#1}) \{ \text{no} \to {#2} \alt \text{nat}({#3}) \to {#4} \}}
\newcommand{\casearr}[4]{\text{Case}_\ra ({#1}) \{ \text{no} \to {#2} \alt \text{fun}({#3}) \to {#4} \}}
\newcommand{\casedyn}[5]{\text{Case}_\Dyn ({#1}) \{ \text{nat}({#2}) \to {#3} \alt \text{fun}({#4}) \to {#5} \}}
%\newcommand{\bind}[3]{\text{var } {#1} = {#2} \text{ in } {#3}}
\newcommand{\matchnat}[4]{\text{match } {#1} \text{ with } \{ \textsf {zero} \Rightarrow {#2} \alt \textsf{ suc } {#3} \Rightarrow {#4} \}}

\newcommand{\refl}{\text{refl}}

\newcommand{\Lift}{\text{Lift}}

%\newcommand{\rel}{\circ\hspace{-4px}-\hspace{-4px}\bullet}
\newcommand{\rel}{\mathrel{\circ\mkern-6mu-\mkern-6mu\bullet}}
% \newcommand{\wand}{\mathrel{-\mkern-6mu*}}
\newcommand{\wand}{\mathrel{\multimap}}
\newcommand{\ltdyn}{\sqsubseteq}
\newcommand{\gtdyn}{\sqsupseteq}
\newcommand{\equidyn}{\mathrel{\gtdyn\ltdyn}}
\newcommand{\gamlt}{\Gamma^\ltdyn}
\newcommand{\deltalt}{\Delta^\ltdyn}
\newcommand{\relcomp}{\odot}

\newcommand{\hasty}[3]{{#1} \vdash {#2} \colon {#3}}
\newcommand{\vhasty}[3]{{#1} \vdash^v {#2} \colon {#3}}
\newcommand{\phasty}[3]{{#1} \vdash^c {#2} \colon {#3}}
\newcommand{\etmprec}[4]{{#1} \vdash {#2} \ltdyn_e {#3} \colon {#4}}
\newcommand{\itmprec}[4]{{#1} \vdash {#2} \ltdyn_i {#3} \colon {#4}}
\newcommand{\etmequidyn}[4]{{#1} \vdash {#2} \equidyn_e {#3} \colon {#4}}
\newcommand{\itmequidyn}[4]{{#1} \vdash {#2} \equidyn_i {#3} \colon {#4}}
\newcommand{\synbisim}{\approx_{\text{syn}}}

\newcommand{\Dwn}{\Downarrow}
\newcommand{\qte}[1]{\text{quote}({#1})}

\newcommand{\elab}[1]{\text{Elab}({#1})}

% Perturbations
\newcommand{\pertp}{\text{Pert}^\text{P}}
\newcommand{\perte}{\text{Pert}^\text{E}}
\newcommand{\pertdyn}[2]{\text{pert-dyn}({#1}, {#2})}
\newcommand{\delaypert}[1]{\text{delay-pert}({#1})}

\newcommand{\pertc}{\text{Pert}_{\text{C}}}
\newcommand{\pertv}{\text{Pert}_{\text{V}}}


% SGDT and Intensional Stuff

\newcommand{\later}{{\vartriangleright}}
\newcommand{\laterhs}{{\later}}
\newcommand{\type}{\texttt{Type}}
\newcommand{\lob}{\text{L\"{o}b}}
\newcommand{\tick}{\mathsf{tick}}
\newcommand{\nxt}{\mathsf{next}}
\newcommand{\fix}{\mathsf{fix}}
\newcommand{\kpa}{\kappa}

% Model-related stuff
\newcommand{\calV}{\mathcal{V}}
\newcommand{\calE}{\mathcal{E}}
\newcommand{\calVk}{\mathcal{V}_k}
\newcommand{\calEk}{\mathcal{E}_k}
\newcommand{\tok}{\mathrel{\mathop{\to}\limits^{\textrm{\footnotesize k}}}}
\newcommand{\timesk}{\mathrel{\mathop{\times}\limits^{\textrm{k}}}}
\newcommand{\Fk}{\mathrel{\mathop{F}\limits^{\textrm{k}}}}
\newcommand{\Uk}{\mathrel{\mathop{U}\limits^{\textrm{k}}}}

\newcommand{\op}[1]{{#1}^{\textrm{op}}}
\newcommand{\calC}{\mathcal{C}}
\newcommand{\Set}{\mathsf{Set}}
\newcommand{\ErrDom}{\mathsf{ErrDom}}
\newcommand{\Yo}{\mathsf{Yo}}
\newcommand{\Hom}{\mathsf{Hom}}
\newcommand{\calS}{\mathcal{S}}
\newcommand{\gfix}{\texttt{gfix}}
\newcommand{\qfix}{\texttt{qfix}}
\newcommand{\calU}{\mathcal{U}}
\newcommand{\laterhat}{\widehat{\later}}
\newcommand{\El}{\mathsf{El}}
\newcommand{\Clock}{\mathsf{Clock}}

\newcommand{\Machine}[1]{\mathsf{Machine}\, {#1}}


% Predomains and EP pairs
\newcommand{\Nat}{\mathsf{Nat}}
\newcommand{\Dyn}{\mathsf{Dyn}}
\newcommand{\ty}[1]{\langle {#1} \rangle}
\newcommand{\li}{L_\mho}
\newcommand{\liclk}[1]{L_\mho [{#1}]}

\newcommand{\ext}[2]{\text{ext}\,{#1}\,{#2}}
\newcommand{\map}[2]{\text{map}\,{#1}\,{#2}}

\newcommand{\ltls}{\ltdyn}
\newcommand{\bisim}{\approx}
\newcommand{\semlt}{\le}
\newcommand{\semltbad}{\lesssim}

%\newcommand{\injarr}{\textsf{Inj}_\to}
%\newcommand{\injnat}{\textsf{Inj}_\mathbb{N}}

\newcommand{\id}{\mathsf{id}}
\newcommand{\ep}{\leadsto}

\newcommand{\emb}[2]{\mathsf{emb}_{#1}({#2})}
\newcommand{\proj}[2]{\mathsf{proj}_{#1}({#2})}

\newcommand{\monto}{\to_m}


% Notation for wait functions
\newcommand{\wre}{w_r^e}
\newcommand{\wle}{w_l^e}
\newcommand{\wrp}{w_r^p}
\newcommand{\wlp}{w_l^p}

\newcommand{\sem}[1]{\llbracket {#1} \rrbracket}
\newcommand{\semgl}[1]{{\sem{#1}}^\text{gl}}


% Denotational model
\newcommand{\errdom}{\textsf{ErrDom}}

\newcommand{\ptb}{\text{ptb}}
\newcommand{\push}{\text{push}}
\newcommand{\pull}{\text{pull}}

\newcommand{\pv}{P^{\mathcal{V}}}
\newcommand{\pe}{P^{\mathcal{E}}}
\newcommand{\ptbv}{\text{ptb}^\mathcal{V}}
\newcommand{\ptbe}{\text{ptb}^\mathcal{E}}

\newcommand{\Ppure}{P}
\newcommand{\Pk}{P^K}
\newcommand{\ptbk}{\text{ptb}^K}


\newcommand{\upf}{\text{up}}
\newcommand{\dnf}{\text{dn}
}
\newcommand{\arr}{\to}
\newcommand{\comp}{}

\newcommand{\ltsq}[2]{\mathrel{\ltdyn^{#1}_{#2}}}
\newcommand{\ltsqbisim}[2]{\mathrel{{\widetilde{\ltdyn}}^{#1}_{#2}}}
\newcommand{\ltbisim}{\mathrel{\widetilde{\ltdyn}}}
\newcommand{\vf}{\mathcal{V}_f}
\newcommand{\vsq}{\mathcal{V}_{sq}}
\newcommand{\ef}{\mathcal{E}_f}
\newcommand{\esq}{\mathcal{E}_{sq}}

\newcommand{\morbisimid}[1]{\text{Endo}_\bisim({#1})}


\newcommand{\sv}{s_{\mathcal{V}}}
\newcommand{\tv}{t_{\mathcal{V}}}
\newcommand{\rv}{r_{\mathcal{V}}}
\newcommand{\se}{s_{\mathcal{E}}}
\newcommand{\te}{t_{\mathcal{E}}}
\newcommand{\re}{r_{\mathcal{E}}}

\newcommand{\Ff}{F_f}
\newcommand{\Fsq}{F_{sq}}
\newcommand{\Uf}{U_f}
\newcommand{\Usq}{U_{sq}}
\newcommand{\arrf}{\arr_f}
\newcommand{\arrsq}{\arr_{sq}}

\newcommand{\vsim}{\mathcal{V}_{\bisim}}
\newcommand{\esim}{\mathcal{E}_{\bisim}}
\newcommand{\svsim}{s_{\mathcal{V}}^\bisim}
\newcommand{\tvsim}{t_{\mathcal{V}}^\bisim}
\newcommand{\rvsim}{r_{\mathcal{V}}^\bisim}
\newcommand{\sesim}{s_{\mathcal{E}}^\bisim}
\newcommand{\tesim}{t_{\mathcal{E}}^\bisim}
\newcommand{\resim}{r_{\mathcal{E}}^\bisim}



\newcommand{\ve}{\mathcal{V}_e}
\newcommand{\ee}{\mathcal{E}_e}

\newcommand{\vr}{\mathcal{V}_r}
\newcommand{\er}{\mathcal{E}_r}

\newcommand{\relatedin}[3]{{#1}\, {#3}\, {#2}}
\newcommand{\binrel}[1]{\mathbin{#1}}


\newcommand{\da}{\downarrow}

\newcommand{\ret}{\text{ret}}
\newcommand{\bind}[3]{{#1} <- {#2}\, ; \, {#3}}

\newcommand{\piv}{\Pi^\mathcal{V}}
\newcommand{\pie}{\Pi^\mathcal{E}}

\newcommand{\upl}{\textsc{UpL}}
\newcommand{\upr}{\textsc{UpR}}
\newcommand{\dnl}{\textsc{DnL}}
\newcommand{\dnr}{\textsc{DnR}}

\newcommand{\delre}{\delta^{r,e}}
\newcommand{\delle}{\delta^{l,e}}
\newcommand{\delrp}{\delta^{r,p}}
\newcommand{\dellp}{\delta^{l,p}}

\newcommand{\qordeq}{\bisim}

\newcommand{\inat}{\text{Inj}_\mathbb{N}}
\newcommand{\itimes}{\text{Inj}_\times}
\newcommand{\iarr}{\text{Inj}_\to}

\newcommand{\tnat}{\mathsf{nat}}
\newcommand{\ttimes}{\mathsf{times}}
\newcommand{\tfun}{\mathsf{fun}}

\newcommand{\cased}[7]{
    \begin{align*}
    \text{Case}_D ({#1}) &\text{ of } \{ \\
        &\alt \text{nat}({#2}) \to {#3}  \\
        &\alt \text{times}({#4}) \to {#5}  \\
        &\alt \text{fun}({#6}) \to {#7} \}
    \end{align*}
}

\newcommand{\inone}{\mathsf{in}_1}
\newcommand{\intwo}{\mathsf{in}_2}
\newcommand{\inthree}{\mathsf{in}_3}
\newcommand{\infour}{\mathsf{in}_4}
\newcommand{\infive}{\mathsf{in}_5}


\newcommand{\delay}{\text{Delay}}
\newcommand{\ledelay}{\le^{\text{Del}}}
\newcommand{\bisimdelay}{\bisim^{\text{Del}}}
\newcommand{\tnow}{\mathsf{now}}
\newcommand{\tlater}{\mathsf{later}}
\newcommand{\Da}{\Downarrow}



\begin{document}

\title{Denotational Semantics of Gradual Typing using Synthetic Guarded Domain Theory}
\author{Eric Giovannini}
\affiliation{
  \department{Electrical Engineering and Computer Science}
  \institution{University of Michigan}
  \country{USA}
}
\email{ericgio@umich.edu}

\author{Tingting Ding}
\affiliation{
  \department{Electrical Engineering and Computer Science}
  \institution{University of Michigan}
  \country{USA}
}
\email{tingtind@umich.edu}

\author{Max S. New}
\affiliation{
  \department{Electrical Engineering and Computer Science}
  \institution{University of Michigan}
  \country{USA}
}
\email{maxsnew@umich.edu}

\begin{abstract}
  Gradually typed programming languages, which allow for soundly
  mixing static and dynamically typed programming styles, present a
  strong challenge for metatheorists. Even the simplest sound
  gradually typed languages feature at least recursion and errors,
  with realistic languages featuring furthermore runtime allocation of
  memory locations and dynamic type tags. Further, the desired
  metatheoretic properties of gradually typed languages have become
  increasingly sophisticated: validity of type based equational
  reasoning as well as the relational property known as the gradual
  guarantee or graduality. Many recent works have tackled verifying
  these properties, but the resulting mathematical developments are
  highly repetitive and tedious, with few reusable theorems persisting
  across different developments.

  In this work, we present a new denotational account of gradual
  typing semantics developed using guarded domain theory. Guarded
  domain theory combines the expressive power of step-indexed logical
  relations for modeling recursive features with the modularity and
  reusability of denotational semantics. Further, recent extensions to
  cubical Agda mean that synthetic guarded domain theory is readily
  mechanized in a proof assistant. We demonstrate the feasibility of
  this approach with a model of gradually typed lambda calculus and
  prove the validity of beta-eta equality and the graduality theorem
  for the denotational model. This model should provide the basis for
  a reusable mathematical theory of gradually typed program semantics.
%%     We develop a denotational semantics for a simple gradually typed language
%%     that is adequate and proves the graduality theorem.
%%     %
%%     The denotational semantics is constructed using \emph{synthetic
%%     guarded domain theory} working in a type theory with a later
%%     modality and clock quantification.
%%     %
%%     This provides a remarkably simple presentation of the semantics,
%%     where gradual types are interpreted as ordinary types in our ambient
%%     type theory equipped with an ordinary preorder structure to model
%%     the error ordering.
%%     %
%%     This avoids the complexities of classical domain-theoretic models
%%     (New and Licata) or logical relations models using explicit
%%     step-indexing (New and Ahmed).
%%     %
%%     In particular, we avoid a major technical complexity of New and
%%     Ahmed that requires two logical relations to prove the graduality
%%     theorem.
  
%%     By working synthetically we can treat the domains in which gradual
%%     types are interpreted as if they were ordinary sets. This allows us
%%     to give a ``na\"ive'' presentation of gradual typing where each
%%     gradual type is modeled as a well-behaved subset of the universal
%%     domain used to model the dynamic type, and type precision is modeled
%%     as simply a subset relation.
%%     %
\end{abstract}

\maketitle

% Outline

% 1. Intro: What do we want out of gradually typed languages and why
%    is it hard to prove? Explanation: Gradual Typing inherently
%    involves recursive types, multiple effects, relational properties.
%    We argue that the increasing complexity of the metatheory of
%    gradual typing makes it a good candidate for 

% 2. Extensional Dream Semantics: double categorical

% 3. The Problem with Step-indexing

% 4. A Compromise

% 5. Formalization in Guarded Cubical Agda

\newif\ifdraft
\drafttrue
\renewcommand{\max}[1]{\ifdraft{\color{blue}[{\bf Max}: #1]}\fi}
\newcommand{\eric}[1]{\ifdraft{\color{orange}[{\bf Eric}: #1]}\fi}
\newcommand{\tingting}[1]{\ifdraft{\color{red}[{\bf Tingting}: #1]}\fi}

\section{Introduction}
  
% gradual typing, graduality
\subsection{Gradual Typing and Graduality}
In programming language design, there is a tension between \emph{static} typing
and \emph{dynamic} typing disciplines. With static typing, the code is
type-checked at compile time, while in dynamic typing, the type checking is
deferred to run-time. Both approaches have benefits and excel in different
scenarios, with static typing offering compile-time assurance of a program's
type safety and type-based reasoning principles that justify program
optimizations, and dynamic typing allowing for rapid prototyping of a codebase
without committing to fixed type signatures.
%
Most languages choose between static or dynamic typing and as a result,
programmers that initially write their code in a dynamically typed language need
to rewrite some or all of their codebase in a static language if they would like
to receive the benefits of static typing once their codebase has matured.

\emph{Gradually typed languages} \cite{siek-taha06, tobin-hochstadt06} seek to
resolve this tension by allowing for both static and dynamic typing disciplines
to be used in the same codebase. These languages support smooth interoperability
between statically-typed and dynamically-typed styles, allowing the programmer to
begin with fully dynamically-typed code and \emph{gradually} migrate portions of the
codebase to a statically typed style without needing to rewrite the project in a
completely different language.

%Gradually-typed languages should satisfy two intuitive properties.
% The following two properties have been identified as useful for gradually typed languages.

% \eric{This paragraph could be deleted}
%% In order for this to work as expected, gradually-typed languages should allow for
%% different parts of the codebase to be in different places along the spectrum from
%% dynamic to static, and allow for those different parts to interact with one another.
%% Moreover, gradually-typed languages should support the smooth migration from
%% dynamic typing to static typing, in that the programmer can initially leave off the
%% typing annotations and provide them later without altering the meaning of the program.
%% % Sound gradual typing
%% Furthermore, the parts of the program that are written in a dynamic
%% style should soundly interoperate with the parts that are written in a
%% static style.  That is, the interaction between the static and dynamic
%% components of the codebase should preserve, to the extent possible,
%% the guarantees made by the static types.  In particular, while
%% statically-typed code can error at runtime in a gradually-typed
%% language, such an error can always be traced back to a
%% dynamically-typed term that violated the typing contract imposed by
%% statically typed code. Further, static type assertions are sound in
%% the static portion, and should enable type-based reasoning and
%% optimization.

% Moreover, gradually-typed languages should allow for
% different parts of the codebase to be in different places along the spectrum from
% dynamic to static, and allow for those different parts to interact with one another.
% In a \emph{sound} gradually-typed language,
% this interaction should respect the guarantees made by the static types.

% Graduality property
One of the fundamental theorems for gradually typed languages is
\emph{graduality}, also known as the \emph{dynamic gradual guarantee},
originally defined by Siek, Vitousek, Cimini, and Boyland
\cite{siek_et_al:LIPIcs:2015:5031, new-ahmed2018}.
%
Informally, graduality says that migrating code from dynamic to
static typing should only allow for the introduction of static or
dynamic type errors, and not otherwise change the behavior of the
program.
%
This is a way to capture programmer intuition that increasing type
precision corresponds to a generalized form of runtime assertions in
that there are no observable behavioral changes up to the point of the
first dynamic type error\footnote{once a dynamic type error is raised,
in languages where the type error can be caught, program behavior may
then further diverge, but this is typically not modeled in gradual
calculi.}.
%
Fundamentally, this property comes down to the behavior of
\emph{runtime type casts}, which implement these generalized runtime
assertions.

Additionally, gradually typed languages should offer some of the
benefits of static typing. While standard type soundness, that
well-typed programs are free from runtime errors, is not compatible
with runtime type errors, it is possible instead to prove that
gradually typed languages validate \emph{type-based reasoning}. For
instance, while dynamically typed $\lambda$ calculi only satisfy
$\beta$ equality for their type formers, statically typed $\lambda$
calculi additionally satisfy type-dependent $\eta$ properties that
ensure that functions are determined by their behavior under
application and that pattern matching on data types is safe and
exhaustive. A gradually typed calculus that validates these
type-dependent $\eta$ laws then provides some of the type-based
reasoning that dynamic languages lack.

% moved the remaining paragraphs to a new section later in the intro

\subsection{Denotational Semantics in Guarded Domain Theory}

Our goal in this work is to provide an \emph{expressive},
\emph{reusable}, \emph{compositional} semantic framework for defining
such well-behaved semantics of gradually typed programs.
%
Our approach to achieving this goal is to provide a compositional
\emph{denotational semantics}, mapping types to a kind of semantic
domain, terms to functions and relations such as term precision to
proofs of semantic relations between the denoted functions.
%
Since the denotational constructions are all syntax-independent, the
constructions we provide may be reused for similar languages. Since it
is compositional, components can be mixed and matched depending on
what source language features are present.

Providing a semantics for gradual typing is inherently complicated in
that it involves: (1) recursion and recursive types through the
presence of dynamic types, (2) effects in the form of divergence and
errors (3) relational models in capturing the graduality
property. Recursion and recursive types must be handled using some
flavor of domain theory. Effects can be modeled using monads in the
style of Moggi, or adjunctions in the style of
Levy\cite{moggi,levy}. Relational properties and their verification
lead naturally to the use of reflexive graph categories or double
categories\cite{reflgraphcats,doublecats}.

The only prior denotational semantics for gradual typing was given by
New and Licata and is based on a classical Scott-style \emph{domain
theory} \cite{new-licata18}. The fundamental idea is to equip
$\omega$-CPOs with an additional ``error ordering'' $\ltdyn$ which
models the graduality ordering, and for casts to arise from
\emph{embedding-projection pairs}. Then the graduality property
follows as long as all language constructs can be interpreted using
constructions that are monotone with respect to the error ordering.
%
This framework has the benefit of being compositional, and was
expressive enough to be extended to model dependently typed gradual
typing \cite{gradualizing-cic}.
%
However, an approach based on classical domain theory has fundamental
limitations: domain theory is incapable of modeling certain perversely
recursive features of programming languages such as dynamic type tag
generation and higher-order references, which are commonplace in
real-world gradually typed systems as well as gradual calculi.
%
Our long-term goal is to develop a denotational approach that can
scale up to these advanced features, and so we must abandon classical
domain theory as the foundation in order to make progress. In this
work, we provide first steps towards this goal by adapting work on a
simple gradually typed lambda calculus to \emph{guarded domain
theory}. Our goal is for this model to scale to dynamic type tag
generation and higher-order references in future work.

\emph{Guarded} domain theory is the main denotational alternative to
classical domain theory that can successfully model these advanced
features. While classical domain theory is based on modeling types as
ordered sets with certain joins, guarded domain theory is based on an
entirely different foundations, sometimes (ultra)metric spaces but
more commonly as ``step-indexed sets'', i.e., objects in the
\emph{topos of trees}
\cite{birkedal-mogelberg-schwinghammer-stovring2011}.  Such an object
consists of a family $\{X_n\}_{n \in \mathbb{N}}$ of sets along with
restriction functions $r_n : X_{n+1} \to X_n$ for all $n$.  (in
category theoretic terminology, these are presheaves on the poset of
natural numbers.)  We think of a $\mathbb{N}$-indexed set as an
infinite sequence of increasingly precise approximations to the true
type being modeled.
%
Key to guarded domain theory is that there is an operator
$\triangleright$ on step-indexed sets called ``later''. In terms of
sequences of approximations, the later operator delays the
approximation by one step. Then the crucial axiom of guarded domain
theory is that any guarded domain equation $X \cong F(\triangleright
X)$ has a unique solution. This allows guarded domain theory to model
essentially \emph{any} recursive concept, with the caveat that the
recursion is \emph{guarded} by a use of the later operator.

% The theorems we want are about the whole sequence of approximations
% whereas working with presheaves is about each finite approximation
% The statement of graduality is in terms of global elements
\eric{This next paragraph could be moved to the section on adequacy, but since
the example is in the analytic setting it also makes sense to keep it here. I'm
not sure which option makes more sense.} While the definitions in guarded domain
theory are constructed as sequences of approximations, results about semantics
of programs, e.g., graduality, are actually statements concerning entire
\emph{sequences} of approximations. As a simple example, consider the set of
programs that may take a computational step or return unit.\footnote{This
example is adapted from \cite{mogelberg-paviotti2016}.} In the topos of trees,
we can model the set of these programs as the indexed family of sets $\{X_n\}_{n
\in \mathbb{N}}$ where for each $n \ge 1$ we define $X_n = \{0,1,\dots,n-1,
\bot\}$. Here, $i$ denotes a program that steps $i$ times and then returns, and
$\bot$ denotes a program that fails to terminate in $n-1$ steps or fewer. For
any fixed $n$, the set $X_n$ fails to distinguish a program that takes $n$ steps
and then terminates from a program that never terminates. On the other hand, if
we define the denotation of a program to be a \emph{sequence} of elements $x_i
\in X_i$ for all $i \in \mathbb{N}$, then the denotation of the diverging

%% \max{what is the point you are trying to convey in this paragraph?}
%% Because the solutions we obtain working in guarded domain theory are constructed
%% as a sequence of increasingly better approximations, when we want to establish a
%% property that holds in the limit, we must reason about all finite
%% approximations. In the setting of the topos of trees, this manifests as the need
%% to consider \emph{global elements} of the presheaves, i.e., a family of elements
%% $x_i \in X_i$ compatible with the restriction maps. For example, consider the
%% set of programs that may take a computational step or return unit.\footnote{This example is adapted from \cite{mogelberg-paviotti2016}.} In the
%% topos of trees, we can model this set as the indexed family of sets $\{X_n\}_{n \in \mathbb{N}}$
%% where for each $n \ge 1$ we define $X_n = \{0,1,\dots,n-1, \bot\}$. Here, $i$ denotes a program that steps
%% $i$ times and then returns, and $\bot$ denotes a program that fails to terminate
%% in $n-1$ steps or fewer. For any fixed $n$, the set $X_n$ fails to distinguish a
%% program that takes $n$ steps and then terminates from a program that never
%% terminates. On the other hand, if we instead consider the denotation of a
%% program to be a global element of the presheaf $\{X_n\}_{n \in \mathbb{N}}$ -- a family of elements
%% $x_i \in X_i$ for all $i \in \mathbb{N}$ -- then the denotation of the diverging
%% >>>>>>> Stashed changes
program will be distinct from that of a program that terminates, regardless of
the number of steps it takes.

% Because the solutions we obtain working in guarded domain theory are constructed
% as a sequence of increasingly better approximations, when we want to establish a
% property that holds in the limit, we must reason about all finite
% approximations. In the setting of the topos of trees, this manifests as the need
% to consider \emph{global elements} of the presheaves, i.e., a family of elements
% $x_i \in X_i$ compatible with the restriction maps. 

% \max{this next paragraph is a bit too mysterious}
% This has important ramifications for defining an adequate denotational semantics
% in guarded domain theory, as we seek to do in this paper: we do not want our
% denotational semantics to conflate a diverging program with a program that fails
% to terminate in sufficiently few steps. Thus, establishing adequacy of a
% semantics in guarded domain theory will require a means of globalizing,
% a point we will return to in Section \ref{sec:big-step-term-semantics}.


\subsection{Synthetic Guarded Domain Theory}

While guarded domain theory can be presented analytically using
ultrametric spaces or the topos of trees, in practice it is
considerably simpler to work \emph{synthetically} by working in a
non-standard foundational system such as \emph{guarded type
theory}. In guarded type theory later is taken as a primitive
operation on types, and we take as an axiom that guarded domain
equations have a (necessarily unique) solution. The benefit of this
synthetic approach is that when working in the non-standard
foundation, we don't need to model an object language type as a
step-indexed set, but instead simply as a set, and object-language
terms can be modeled in the Kleisli category of a simple monad defined
using guarded recursion. Not only does this make on-paper reasoning
about guarded domain theory easier, it also enables a simpler avenue
to verification in a proof assistant. Whereas formalizing analytic
guarded domain theory would require a significant theory of presheaves
and making sure that all constructions are functors on categories of
presheaves, formalizing synthetic guarded domain theory can be done by
directly adding the later modality and the guarded fixed point
property axiomatically.
%
% \max{TODO: more here, specifically this is where we should talk about adequacy I think}
%
In this paper, we work informally in a guarded type theory which is described in
more detail in Section \ref{sec:guarded-type-theory}.
%
% As in the setting of analytic guarded domain theory, in the synthetic setting it
% is also possible to construct global solutions. The specific approach we use in
% this work is that of \emph{clocks} and \emph{clock-quantification}
% \cite{atkey-mcbride2013}.

% \max{shouldn't we say we use clock quantification?}
% In the setting of synthetic guarded domain theory, there is an analogue of
% constructing global solutions. To construct a global solution we must use an
% additional construct, e.g., the $\square$ modality
% \cite{10.1007/978-3-319-08918-8_8}, whose semantics in the topos of trees model
% is to compute the global elements of a presheaf. A related approach involves
% objects known as \emph{clocks}, whereby \emph{clock-quantification}
% \cite{atkey-mcbride2013} provides a means to obtain a global solution.


\subsection{Adapting the New-Licata Model to the Guarded Setting}

% Working in a step-counting model, but graduality is independent of steps
% Casts take stpes, so we need to reason up to weak bisimilarity, but it is not transitive
% New-Licata freely uses transitivity, but we can't adapt their proof methodology because it uses transitivity pervasively
% Thus the old proofs from New-Licata that use transitivity don't apply.
% Work with a combination of weak bisimilarity and step-counting reasoning
% Need a tiny bit of weak bisimilarity which are perturbations, explicit synchronizations that we manipulate syntactically

Since guarded domain theory only provides solutions to guarded domain equations,
there is no systematic way to convert a classical domain-theoretic semantics to
a guarded one.  Classical domain theory has limitations in what it can model,
but it provides \emph{exact} solutions to domain equations when it applies. When
adapting the New-Licata approach to guarded domain theory, the presence of later
in the semantics makes it \emph{intensional}: unfolding the dynamic type
requires an observable computational step. For example, a function that pattern
matches on an element of the dynamic type and then returns it unchanged is
\emph{not} equal to the identity function, because it ``costs'' a step to
perform the pattern match.

% - The essential role of transitivity
% - We can't have transitivity + extensional reasoning
% - Solution: split the ordering into two, and introduce perturbations
%   to recover some amount of transitive reasoning

The intensional nature of guarded domain theory leads to the main departure of
our semantics from the New-Licata approach. Because casts involve computational
steps, the graduality property must be insensitive to the steps taken by terms.
This means that the model must allow for reasoning \emph{up to weak
bisimilarity}, where two terms are weakly bisimilar if they differ only in their
number of computational steps. However, the weak bisimilarity relation is not
transitive in the guarded setting, which follows from a no-go theorem we
establish (Theorem \ref{thm:no-go}). The New-Licata proofs freely use
transitivity, and we argue in Section \ref{sec:towards-relational-model} that
some amount of transitive reasoning is nessecary for defining a
syntax-independent model of gradual typing. As a result, the lack of
transitivity of weak bisimilarity presents a challenge in adapting the
New-Licata model to the guarded setting. Our solution is to work with a
combination of weak bisimilarity and step-sensitive reasoning, where the latter
notion \emph{is} transitive. To deal with the fact that casts take computational
steps, we introduce the novel concept of \emph{syntactic perturbations}, which
are explicit synchronizations that we manipulate syntactically. The combination
of lock-step reasoning with perturbations is transitive and is able to account
for the step-insensitive nature of the graduality property.

% These can then be used to explicitly \emph{synchronize} elements to ensure
% that they are in lock-step.
% %
% In the end, we have come to a resolution of the issue: the lock-step error
% ordering where we employ syntactic perturbations to synchronize elements is
% still transitive, while simultaneously being able to relate terms that involve
% an explicit, known pattern of computational steps.


\subsection{Adequacy}

Our approach to proving graduality by constructing a denotational model in
guarded type theory consists of two main steps. First, there is the construction
of the model itself. This gives us an interpretation of the terms as well as the
axioms that model the graduality property. However, since this model is defined
in guarded domain theory, the denotations of terms are constructed as sequences of
approximations and our graduality result is actually a statement about each
finite approximation. The actual graduality property should be a statement about
entire sequences of approximations.
%
What we need is to show that the guarded model induces a sensible set-theoretic
semantics in ordinary mathematics. We cannot hope to derive such a semantics for
\emph{all} types (e.g., the dynamic type), but for the subset of closed terms of
base type, we should be able to extract a well-behaved semantics for which the
graduality property holds.

More concretely, consider a gradually typed language whose only effects are
gradual type errors and divergence (errors arise from failing casts; divergence
arises because we can use the dynamic type to encode untyped lambda calculus;
see Section \ref{sec:GTLC}). If we fix a result type of natural numbers, a
``big-step'' semantics is a partial function from closed programs to either natural
numbers or errors:
%
\[ -\Downarrow : \{M \,|\, \cdot \vdash M : \nat \} \rightharpoonup \mathbb{N}
\cup \{\mho\} \] 
%
where $\mho$ is notation for a runtime type error. We write $M \Downarrow n$ and
$M\Downarrow \mho$ to mean this semantics is defined as a number or error, and
$M\Uparrow$ to mean the semantics is undefined, representing divergence.
%
A well-behaved semantics should then satisfy several properties. First, it
should be \emph{adequate}: natural number constants should evaluate to
themselves: $n \Downarrow n$. Second, it should validate type based reasoning.
To formalize type based reasoning, languages typically have an equational theory
$M \equiv N$ specifying when two terms should be considered equivalent.
Then we want to verify that the big step semantics respects this equational
theory: if closed programs $M \equiv N$ are equivalent in the equational theory
then they have the same semantics, $M \Downarrow n \iff N \Downarrow n$ and
$M\Uparrow \iff N \Uparrow$ and $M \Downarrow \mho \iff N \Downarrow \mho$.

To define the big-step semantics, we apply the technique of
\emph{clock-quantification} \cite{atkey-mcbride2013}, which is the synthetic
analogue of computing global elements in analytic guarded domain theory. This
gives us an interpretation of closed programs of base type as partial functions.
We describe the process in more detail in Section
\ref{sec:big-step-term-semantics}. The resulting term semantics is adequate, and
furthemore it validates the equational theory, since equivalent terms in the
equational theory denote equal terms in the big-step semantics.

Lastly, the semantics should be adequate for the graduality property. Graduality
is typically axiomatized by giving an \emph{inequational} theory called term
precision, where $M \ltdyn N$ roughly means that $M$ and $N$ have the same type
erasure and $M$ has at each point in the program a more precise/static type than
$N$. Then adequacy says that if $M$ and $N$ are whole programs returning type
$\nat$ and $M \ltdyn N$, then the denotations of $M$ and $N$ as big-step terms
should be related in the expected way: either $M$ errors, or $M$ and $N$ have
the same extensional behavior. That is, either $M\Downarrow \mho$ or $M
\Downarrow n $ and $N \Downarrow n$ or $M \Uparrow $ and $N
\Uparrow$\footnote{we use a slightly more complex definition of this relation in
our technical development below that is classically equivalent but
constructively weaker}.
%
To prove that the semantics is adequate for graduality, we again apply clock
quantification, this time to the relation that denotes the term precision
ordering. We show that a term precision ordering $M \ltdyn N$ implies the
corresponding ordering on the partial functions denoted by $M$ and $N$. The
details of the proof are given in Section \ref{sec:adequacy}.


% For example, when establishing the adequacy of a semantics defined using
% guarded type theory, we must reason about all finite approximations.

% In this paper, we develop an adequate denotational semantics that satisfies
% graduality and soundness of the equational theory of cast calculi using
% synthetic guarded domain theory.  


\subsection{Contributions and Outline}

The main contribution of this work is a compositional denotational semantics in
guarded type theory for a simple gradually typed language that validates
$\beta\eta$ equality and satisfies a graduality theorem.
%
% In particular, the notion of syntactic perturbations stands out as a key
% technical contribution and is the most significant and novel feature of our
% model that allowed us to successfully adapt the denotational approach of New and
% Licata to the guarded setting.
% In particular, the notion of syntactic perturbations stands out as a key feature
% of our model, and is our most significant technical contribution allowing us to
% successfully adapt the denotational approach of New and Licata to the guarded
% setting.
The goal we have kept in mind while designing our model is to handle
gradually-typed languages with advanced features that classical domain theory in
incapable of modeling. We do not address these features in this work but leave
them instead to immediate future work, as handling even the simplest
gradually-typed language already poses interesting challenges.
%
Within our semantics, the notion of syntactic perturbation is our most
significant technical contribution, allowing us to successfully adapt the
denotational approach of New and Licata to the guarded setting.
%
Most of our work has further been verified in Guarded Cubical Agda
\cite{veltri-vezzosi2020}, demonstrating that the semantics is readily
mechanizable. We provide an overview of the mechanization effort, and what
remains to be formalized, in Section \ref{sec:mechanization}.
%


% Syntactic perturbations 
%
% The notion of \emph{syntactic perturbations} allow us to encode the steps
% taken by a term in a form that we can manipulate. This allows us to recover
% enough extensional reasoning to model the graduality property compositionally.

% Syntactic perturbations give us a type-directed way of imposing delays on terms.

\begin{comment}
\begin{enumerate}
\item First, we give a simple concrete term semantics where we show
  how to model the dynamic type as a solution to a guarded domain equation.
\item Next, we identify where prior work on classical domain theoretic
  semantics of gradual typing breaks down when using guarded semantics
  of recursive types.
\item We develop a key new concept of \emph{syntactic perturbations},
  which allow us to recover enough extensional reasoning to model the
  graduality property compositionally.
\item We combine this insight together with an abstract categorical
  model of gradual typing using reflexive graph categories and
  call-by-push-value to give a compositional construction of our
  denotational model.
\item We prove that the resulting denotational model provides a
  well-behaved semantics as defined above by proving \emph{adequacy},
  respect for an equational theory and the graduality property.
\end{enumerate}
\end{comment}

The paper is laid out as follows:
% \max{I think there's too much detail in this outline, especially parts 4 and 5. Can we instead expand those into the Intro?}
\begin{enumerate}
\item In Section \ref{sec:GTLC} we fix our input language, a fairly
  typical gradually typed cast calculus.
\item In Section \ref{sec:concrete-term-model} we develop a
  denotational semantics in synthetic guarded domain theory for the
  \emph{terms} of the gradual lambda calculus.  The model is adeqauate
  and validates the equational theory, but it does not satisfy
  graduality. We use this to introduce some of our main technical
  tools: modeling recursive types in guarded type theory and modeling
  effects using call-by-push-value.
\item In Section \ref{sec:towards-relational-model} we show where the New-Licata
  classical domain theoretic approach fails to adapt cleanly to the guarded
  setting and explore the difficulties of proving graduality in an intensional
  model. We prove the no-go theorem about extensional, transitive relations,
  introduce the lock-step error ordering and weak bisimilarity relation, and
  motivate the need for perturbations.
\item In Section \ref{sec:concrete-relational-model} we describe the
  construction of the model in detail, and discuss the
  translation of the syntax and axioms of the gradually typed cast calculus into the model.
  Lastly, we prove that the model is adequate for the graduality property.
  % The resulting model validates the axioms for type and term precision specified
  % in Section \ref{sec:GTLC}. The final step is to prove that the model is
  % \emph{adequate} for the graduality property: a closed term precision $M \ltdyn
  % N : \nat$ has the expected semantics, i.e., that $M$ errors or $M$ and $N$
  % have the same extensional behavior. We do so by extending the globalization
  % techniques used in defining the big-step term model to account for the
  % interpretation of term precision.
\item In Section \ref{sec:discussion} we discuss prior work on proving
  graduality, the partial mechanization of our results in Agda, and
  future directions for denotational semantics of gradual typing.
\end{enumerate}


\section{Background on Synthetic Guarded Domain Theory}\label{sec:technical-background}

% Modeling the dynamic type as a recursive sum type?
% Observational equivalence and approximation?

% synthetic guarded domain theory, denotational semantics therein

%% \subsection{Difficulties in Prior Semantics}
%%   % Difficulties in prior semantics

%%   In this work, we compare our approach to proving graduality to the approach
%%   introduced by New and Ahmed \cite{new-ahmed2018} which constructs a step-indexed
%%   logical relations model and shows that this model is sound with respect to their
%%   notion of contextual error approximation.

%%   Because the dynamic type is modeled as a non-well-founded
%%   recursive type, their logical relation needs to be paramterized by natural numbers
%%   to restore well-foundedness. This technique is known as a \emph{step-indexed logical relation}.
%%   Intuitively, step-indexing makes the steps taken by terms observable.
%%   %
%%   Reasoning about step-indexed logical relations
%%   can be tedious and error-prone, and there are some very subtle aspects that must
%%   be taken into account in the proofs.
%%   %
%%   In particular, the prior approach of New and Ahmed requires two separate logical
%%   relations for terms, one in which the steps of the left-hand term are counted,
%%   and another in which the steps of the right-hand term are counted.
%%   Then two terms $M$ and $N$ are related in the ``combined'' logical relation if they are
%%   related in both of the one-sided logical relations. Having two separate logical relations
%%   complicates the statement of the lemmas used to prove graduality, because any statement that
%%   involves a term stepping needs to take into account whether we are counting steps on the left
%%   or the right. Some of the differences can be abstracted over, but difficulties arise for properties %/results
%%   as fundamental and seemingly straightforward as transitivity.

%%   Specifically, for transitivity, we would like to say that if $M$ is related to $N$ at
%%   index $i$ and $N$ is related to $P$ at index $i$, then $M$ is related to $P$ at $i$.
%%   But this does not actually hold: we require that one of the two pairs of terms
%%   be related ``at infinity'', i.e., that they are related at $i$ for all $i \in \mathbb{N}$.
%%   Which pair is required to satisfy this depends on which logical relation we are considering,
%%   (i.e., is it counting steps on the left or on the right),
%%   and so any argument that uses transitivity needs to consider two cases, one
%%   where $M$ and $N$ must be shown to be related for all $i$, and another where $N$ and $P$ must
%%   be related for all $i$. % This may not even be possible to show in some scenarios!

%%   \begin{comment}
%%   As another example, the axioms that specify the behavior of casts do not hold in the
%%   step-indexed setting. Consider, for example, the ``lower bound'' rule for downcasting:

%%   \begin{mathpar}
%%     \inferrule*
%%     {\gamlt \vdash M \ltdyn N : dyn}
%%     {\gamlt \vdash \dn{\dyn \ra \dyn}{\dyn} M \ltdyn N}
%%   \end{mathpar}

%%   In the language of the step-indexed logical relation used in prior work, this would
%%   take the form
  
%%   \begin{mathpar}
%%     \inferrule*
%%     {(M, N) \in \mathcal{E}^{\sim}_{i}\sem{\dyn}}
%%     {(\dn{\dyn \ra \dyn}{\dyn} M, N) \in \mathcal{E}^{\sim}_{i}\sem{\injarr{}}}
%%   \end{mathpar}

%%   where $\sim$ stands for $\mathrel{\preceq}$ or $\mathrel{\succeq}$, i.e.,
%%   counting steps on the left or right respectively.
%%   %
%%   To show this, we would use the fact that the left-hand side steps and
%%   apply an anti-reduction lemma, showing that the term to which the LHS steps
%%   is related to the RHS where our fuel is now
  
%%   The left-hand side steps to a case inspection on $M$,
%%   where we unfold the recursive $\dyn$ type into a sum type and see whether the result
%%   is a function type.

%%   One way around these difficulties is to demand that the rules only hold
%%   ``extensionally'', i.e., we quantify universally over the step-index and
%%   reason about the ``global'' behavior of terms for all possible step indices.
%%   This is the approach taken in prior work.
%% \end{comment}

%%   % These complications introduced by step-indexing lead one to wonder whether there is a
%%   % way of proving graduality without relying on tedious arguments involving natural numbers.
%%   % An alternative approach, which we investigate in this paper, is provided by
%%   % \emph{synthetic guarded domain theory}, as discussed below.
%%   % Synthetic guarded domain theory allows the resulting logical relation to look almost
%%   % identical to a typical, non-step-indexed logical relation.

%% \subsection{Synthetic Guarded Domain Theory}\label{sec:sgdt}
One way to avoid the tedious reasoning associated with step-indexing is to work
axiomatically inside of a logical system that can reason about non-well-founded recursive
constructions while abstracting away the specific details of step-indexing required
if we were working analytically.
The system that proves useful for this purpose is called \emph{synthetic guarded
domain theory}, or SGDT for short. We provide a brief overview here, but more
details can be found in \cite{birkedal-mogelberg-schwinghammer-stovring2011}.

SGDT offers a synthetic approach to domain theory that allows for guarded recursion
to be expressed syntactically via a type constructor $\later : \type \to \type$ 
(pronounced ``later''). The use of a modality to express guarded recursion
was introduced by Nakano \cite{Nakano2000}.
%
Given a type $A$, the type $\later A$ represents an element of type $A$
that is available one time step later. There is an operator $\nxt : A \to\, \later A$
that ``delays'' an element available now to make it available later.
We will use a tilde to denote a term of type $\later A$, e.g., $\tilde{M}$.

% TODO later is an applicative functor, but not a monad

There is a \emph{guarded fixpoint} operator
%
\[
  \fix : \forall T, (\later T \to T) \to T.
\]
%
That is, to construct a term of type $T$, it suffices to assume that we have access to
such a term ``later'' and use that to help us build a term ``now''.
This operator satisfies the axiom that $\fix f = f (\nxt (\fix f))$.
In particular, this axiom applies to propositions $P : \texttt{Prop}$; proving
a statement in this manner is known as $\lob$-induction.

The operators $\later$, $\nxt$, and $\fix$ described above can be indexed by objects
called \emph{clocks}. A clock serves as a reference relative to which steps are counted.
For instance, given a clock $k$ and type $T$, the type $\later^k T$ represents a value of type
$T$ one unit of time in the future according to clock $k$.
If we only ever had one clock, then we would not need to bother defining this notion.
However, the notion of \emph{clock quantification} is crucial for encoding coinductive types using guarded
recursion, an idea first introduced by Atkey and McBride \cite{atkey-mcbride2013}.

Most of the developments in this paper will take place in the context of a single clock $k$,
but later on, we will need to make use of clock quantification.

% Clocked Cubical Type Theory
\subsection{Ticked Cubical Type Theory}
% TODO motivation for Clocked Cubical Type Theory, e.g., delayed substitutions?

Ticked Cubical Type Theory \cite{mogelberg-veltri2019} is an extension of
Cubical Type Theory \cite{CohenCoquandHuberMortberg2017} that has clocks as well
as an additional sort called \emph{ticks}. Ticks were originally introduced in
\cite{bahr-grathwohl-bugge-mogelberg2017}. Given a clock $k$, a tick $t :
\tick\, k$ serves as evidence that one unit of time has passed according to the
clock $k$. In Ticked Cubical Type Theory, the type $\later^k A$ is the type of
dependent functions from ticks of the clock $k$ to $A$. The type $A$ is allowed
to depend on $t$, in which case we write $\later^k_t A$ to emphasize the
dependence.

% TODO next as a function that ignores its input tick argument?

% TODO include a figure with some of the rules for ticks

The rules for tick abstraction and application are similar to those of ordinary
$\Pi$ types. A context now consists of ordinary variables $x : A$ as well as
tick variables $t : \tick\, k$. The presence of the tick variable $t$ in context
$\Gamma, (t : \tick\, k), \Gamma'$ intuitively means that the values of the
variables in $\Gamma$ arrive ``first'', then one time step occurs on $k$, and
then the values of the variables in $\Gamma'$ arrive.

The abstraction rule for ticks states that if in context $\Gamma, t : \tick\, k$
the term $M$ has type $A$, then in context $\Gamma$ the term $\lambda t.M$ has
type $\later^k_t A$. % TODO dependent version?
%
Conversely, if we have a term $M$ of type $\later^k A$, and we have available in
the context a tick $t' : \tick\, k$, then we can apply the tick to $M$ to get a
term $M[t'] : A[t'/t]$. However, there is an important restriction on when we
are allowed to apply ticks. In order to apply $M$ to tick $t$, $M$ must be
well-typed in the prefix of the context occurring before the tick $t$. That is,
all variables mentioned in $M$ must be available before $t$. This ensures that
we cannot, for example, define a term of type $\later \laterhs A \to\, \laterhs
A$ via repeated tick application.
% TODO restriction on tick application
%
For the sake of brevity, we will also write tick application as $M_t$.

% TODO mention Agda implementation of Clocked Cubical Type Theory?
%The statements in this paper have been formalized in a variant of Agda called
%Guarded Cubical Agda \cite{veltri-vezzosi2020}, an implementation of Clocked Cubical Type Theory.


% TODO axioms (clock irrelevance, tick irrelevance)?


\section{Syntactic Theory of Gradually Typed Lambda Calculus}\label{sec:GTLC}

Here we give an overview of a fairly standard cast calculus for
gradual typing along with its (in-)equational theory that capture our
desired notion of type-based reasoning and graduality. The main
departure from prior work is our explicit treatment of type precision
derivations and an equational theory of those derivations.

We give the basic syntax and select typing rules in
Figure~\ref{fig:gtlc-syntax}. We include a dynamic type, a type of
numbers, the call-by-value function type $A \ra A'$ and products.
%
We include a syntax for \emph{type precision} derivations $c : A
\ltdyn A'$; the typing is given in Figure~\ref{fig:typrec}.
%
Any type precision derivation $c : A \ltdyn A'$ induces a pair of
casts, the upcast $\upc c : A \ra A'$ and the downcast $\dnc c : A' \ra
A$.
%
The syntactic intuition is that $c$ is a proof that $A$ is ``less
dynamic'' than $A'$. Semantically, this gives us coercions back and
forth where the upcast is (to a first-order) a pure function whereas
the downcast can fail.
%
These casts are inserted automatically in an elaboration from a
surface language. In this work, we are focused on semantic aspects and
so elide these standard details.
%
The syntax of precision derivations includes reflexivity $r(A)$ and
transitivity $cc'$ as well as monotonicity $c \ra c'$ and $c \times
c'$ that are \emph{covariant} in all arguments and finally generators
$\inat,\iarr,\itimes$ that correspond to the type tags of our dynamic
type.
%
We additionally impose an equational theory $c \equiv c'$ on the
derivations that implies that the corresponding casts are weakly
bisimilar in the semantics.
%
We impose category axioms for the reflexivity and
transitivity and functoriality for the monotonicity rules.
%
We note the following two admissible principles: any two derivations
$c,c' : A \ltdyn A'$ of the same fact are equivalent $c \equiv c'$ and
for any $A$, there is a derivation $\textrm{dyn}(A): A \ltdyn\dyn$. That is, $\dyn$ is the ``most dynamic'' type.

\begin{figure}
  \begin{mathpar}
    \begin{array}{rcl}
    \text{Types } A &::=& \nat \alt \,\dyn \alt A \ra A' \alt A \times A'\\
    \text{Type Precision } c &::=& r(A) \alt c c' \alt \iarr \alt \inat \alt \itimes \alt c \ra c' \alt c \times c'\\
    \text{Values } V &::=& x \alt \upc c V \alt \zro \alt \suc\, V \alt \lda{x}{M} \alt (V,V') \\ 
    \text{Terms } M,N &::=& \err\alt \upc c M \alt \dnc c M \alt \zro \alt \suc\, M \alt \lda{x}{M} \\ 
     &&\alt M\, N \alt (M,N) \alt \textrm{let } (x,y) = M \textrm{ in } N\\
    \text{Contexts } \Gamma &::= &\cdot \alt \Gamma, x : A \\
    \text{Ctx Precision } \Delta &::=& \cdot\alt \Delta,x:c
  \end{array}

  \inferrule
  {\Gamma \vdash M : A \and c : A \ltdyn A'}
  {\Gamma \vdash \upc c M : A'}

  \inferrule
  {\Gamma \vdash N : A' \and c : A \ltdyn A'}
  {\Gamma \vdash \dnc c N : A}

  \inferrule{}{\Gamma \vdash \mho : A}
  \end{mathpar}
  \caption{GTLC Cast Calculus Syntax}
  \label{fig:gtlc-syntax}
\end{figure}

\begin{figure}
  \begin{mathpar}
    \inferrule{}{r(A) : A \ltdyn A}\and
    \inferrule{c : A \ltdyn A' \and c' : A' \ltdyn A''}{cc' : A \ltdyn A''}\and
    \inferrule{}{\iarr \colon \dyn \ra \dyn \ltdyn \dyn}\and
    \inferrule{}{\inat \colon \nat \ltdyn \dyn}\and
    \inferrule{}{\itimes \colon \dyn \times \dyn \ltdyn \dyn}\and
    \inferrule{c_i : A_i \ltdyn A_i' \and c_o : A_o \ltdyn A_o'}{c_i \ra c_o : (A_i \ra A_o) \ltdyn (A_i' \ra A_o')}\and
    \inferrule{c_1 : A_1 \ltdyn A_1' \and c_2 : A_2 \ltdyn A_2'}{c_1 \times c_2 : (A_1 \times A_2) \ltdyn (A_1' \times A_2')}\and
     r(A)c \equiv c\and
     c \equiv cr(A')\and
     c(c'c'') \equiv (cc')c''\and
     r(A_i \ra A_o) \equiv r(A_i) \ra r(A_o)\and
     r(A_1\times A_2) \equiv r(A_1) \times r(A_2)\and
     (c_i \ra c_o)(c_i' \ra c_o')\equiv (c_ic_i' \ra c_oc_o') \and
     (c_1\times c_2)(c_1'\times c_2')\equiv (c_1c_1' \times c_2c_2')
  \end{mathpar}
  \caption{Type Precision Derivations and equational theory}
  \label{fig:typrec}
\end{figure}

Next, we consider the axiomatic (in)equational reasoning principles
for terms: $\beta\eta$ equality and term precision in
Figure~\ref{fig:term-prec}.
%
We include standard CBV $\beta\eta$ rules for function and product
types, as well as equations stating that casts are given functorially.
%
Next, we have \emph{term} precision, an extension of
type precision to terms.
%
The form of the term precision rule is $\Delta \vdash M \ltdyn M' : c$
where $\Delta$ is a context where variables are assigned to type
precision derivations.
%
The judgment is only well formed when every use of $x : c$ for $c : A
\ltdyn A'$ is used with type $A$ in $M$ and $A'$ in $M'$ and similarly
the output types match $c$.
%
We elide the congruence rules for every type constructor, e.g., that
$M \ltdyn M'$ and $N \ltdyn N'$ that $M\,N \ltdyn M'\,N'$.
%
With such congruence rules, reflexivity $M \ltdyn M$ is
admissible. Transitivity, on the other hand, is intentionally not
taken as a primitive rule, matching the original formulation of the
dynamic gradual guarantee \cite{siek_et_al:LIPIcs:2015:5031}.
%
We include a rule that says that equivalent type precision derivations
$c \equiv c'$ are equivalent for the purposes of term precision.
%
% Removed retraction
%
% The next rule is the \emph{retraction} principle, which states that a
% downcast after an upcast is equivalent to doing nothing at all, since
% intuitively the upcasted value should already satisfy the type. Here
% $\equidyn$ means we require each is $\ltdyn$ the other, with
% reflexivity precision derivations.
%
Finally, we include 4 rules for reasoning about casts. Intuitively
these say that the upcast is a kind of \emph{least upper bound} and
dually that the downcast is a \emph{greatest lower bound}.

As a higher-order gradually typed language we inherently have to deal
with two effects: errors and divergence. Errors arise from failing
casts, e.g. casting a number to dynamic to a function
$\dnc{\iarr}\upc{\inat} x$. Divergence arises because our dynamic type
allows us to encode untyped lambda calculus, and so we can encode the
$\Omega$ term with the help of casts $\Omega = (\lambda
x:\dyn. (\dnc{\iarr} x)x)(\upc{\iarr}(\lambda x:\dyn. (\dnc{\iarr}
x)x))$.

\begin{figure}
  \begin{mathpar}
  (\lambda x. M)(V) = M[V/x] \and (V : A \ra A') = \lambda x. V\,x\\

   \textrm{let } (x,y) = (V,V') \textrm{ in } N = N[V/x,V'/y] \and
   M[V:A\times A'/p] = \textrm{let } (x,y) = V \textrm{ in } M[(x,y)/p]

  \upc{(r(A))}M = M \and
  \upc{c'}\upc{c}M = \upc{cc'}M \and
  \dnc{(r(A))}M = M \and
  \dnc{c}\dnc{c'}M = \dnc{cc'}M

  \inferrule
  {\Delta\vdash M \ltdyn M' : c \and c \equiv c'}
  {\Delta\vdash M \ltdyn M' : c'}

  \inferrule
  {}
  {\Delta \vdash \mho \ltdyn M : c}

  % Removed retraction
  % \inferrule
  % {}
  % {\dnc {c} \upc {c} M \equidyn M}

  \inferrule*[Right=UpL]
  {M \ltdyn M' : cc_r}
  {\upc {c} M \ltdyn M' : c_r}

  \inferrule*[Right=UpR]
  {M \ltdyn M' : c_l}
  {M \ltdyn \upc {c} M' : c_lc}

  \inferrule*[Right=DnL]
  {M \ltdyn M' : c_r}
  {\dnc {c} M \ltdyn M' : cc_r}

  \inferrule*[Right=DnR]
  {M \ltdyn M' : c_lc}
  {M \ltdyn \dnc {c} M' : c_l}
  \end{mathpar}
  \caption{Equality and Term Precision Rules (Selected)}
  \label{fig:term-prec}
\end{figure}

Our goal in the remainder of this work is to develop compositional
denotational semantics of types, terms, type and term precision from
which we can easily extract a big step semantics that satisfies
graduality and respects the equational theory of the calculus.

%% Here we describe the syntax and typing for the gradually-typed lambda calculus.
%% We also give the rules for syntactic type and term precision.
%% % We define four separate calculi: the normal gradually-typed lambda calculus, which we
%% % call the extensional or \emph{step-insensitive} lambda calculus ($\extlc$),
%% % as well as an \emph{intensional} lambda calculus
%% % ($\intlc$) whose syntax makes explicit the steps taken by a program.

%% Before diving into the details, let us give a brief overview of what we will define.
%% We begin with a gradually-typed lambda calculus $(\extlc)$, which is similar to
%% the normal call-by-value gradually-typed lambda calculus, but differs in that it
%% is actually a fragment of call-by-push-value specialized such that there are no
%% non-trivial computation types. We do this for convenience, as either way
%% we would need a distinction between values and effectful terms; the framework of
%% of call-by-push-value gives us a convenient language to define what we need.

%% We then show that composition of type precision derivations is admissible, as is
%% heterogeneous transitivity for term precision, so it will suffice to consider a new
%% language ($\extlcm$) in which we don't have composition of type precision derivations
%% or heterogeneous transitivity of term precision.

%% We then observe that all casts, except those between $\nat$ and $\dyn$
%% and between $\dyn \ra \dyn$ and $\dyn$, are admissible.
%% % (we can define the cast of a function type functorially using the casts for its domain and codomain).
%% This means it will be sufficient to consider a new language ($\extlcmm$) in which
%% instead of having arbitrary casts, we have injections from $\nat$ and
%% $\dyn \ra \dyn$ into $\dyn$, and case inspections from $\dyn$ to $\nat$ and
%% $\dyn$ to $\dyn \ra \dyn$.

%% From here, we define a \emph{step-sensitive} (also called \emph{intensional}) GSTLC,
%% so-named because it makes the intensional stepping behavior of programs explicit in the syntax.
%% This is accomplished by adding a syntactic ``later'' type and a
%% syntactic $\theta$ that maps terms of type later $A$ to terms of type $A$.
%% Finally, we define a \emph{quotiented} version of the step-sensitive language where
%% we add a rule that equates terms that are the same up to their stepping behavior.

%% % ---------------------------------------------------------------------------------------
%% % ---------------------------------------------------------------------------------------

%% \subsection{Syntax}

%% The language is based on Call-By-Push-Value \cite{levy01:phd}, and as such it has two kinds of types:
%% \emph{value types}, representing pure values, and \emph{computation types}, representing
%% potentially effectful computations.
%% In the language, all computation types have the form $\Ret A$ for some value type $A$.
%% Given a value $V$ of type $A$, the term $\ret V$ views $V$ as a term of computation type $\Ret A$.
%% Given a term $M$ of computation type $B$, the term $\bind{x}{M}{N}$ should be thought of as
%% running $M$ to a value $V$ and then continuing as $N$, with $V$ in place of $x$.


%% We also have value contexts and computation contexts, where the latter can be viewed
%% as a pair consisting of (1) a stoup $\Sigma$, which is either empty or a hole of type $B$,
%% and (2) a (potentially empty) value context $\Gamma$.

%% \begin{align*} % TODO is hole a term?
%%   &\text{Value Types } A := \nat \alt \,\dyn \alt (A \ra A') \\
%%   &\text{Computation Types } B := \Ret A \\
%%   &\text{Value Contexts } \Gamma := \cdot \alt (\Gamma, x : A) \\
%%   &\text{Computation Contexts } \Delta := \cdot \alt \hole B \alt \Delta , x : A \\
%%   &\text{Values } V :=  \zro \alt \suc\, V \alt \lda{x}{M} \alt \up{A}{B} V \\ 
%%   &\text{Terms } M, N := \err_B \alt \matchnat {V} {M} {n} {M'} \\ 
%%   &\quad\quad \alt \ret {V} \alt \bind{x}{M}{N} \alt V_f\, V_x \alt \dn{A}{B} M 
%% \end{align*}

%% The value typing judgment is written $\hasty{\Gamma}{V}{A}$ and 
%% the computation typing judgment is written $\hasty{\Delta}{M}{B}$.

%% \begin{comment}
%% We define substitution for value contexts by the following rules:

%% \begin{mathpar}
%%   \inferrule*
%%   { \gamma : \Gamma' \to \Gamma \and 
%%     \hasty{\Gamma'}{V}{A}}
%%   { (\gamma , V/x ) \colon \Gamma' \to \Gamma , x : A }

%%   \inferrule*
%%   {}
%%   {\cdot \colon \cdot \to \cdot}
%% \end{mathpar}

%% We define substitution for computation contexts by the following rules:

%% \begin{mathpar}
%%     \inferrule*
%%     { \delta : \Delta' \to \Delta \and 
%%       \hasty{\Delta'|_V}{V}{A}}
%%     { (\delta , V/x ) \colon \Delta' \to \Delta , x : A }

%%     \inferrule*
%%     {}
%%     {\cdot \colon \cdot \to \cdot}

%%     \inferrule*
%%     {\hasty{\Delta'}{M}{B}}
%%     {M \colon \Delta' \to \hole{B}}
%% \end{mathpar}
%% \end{comment}

%% The typing rules are as expected, with a cast between $A$ to $B$ allowed only when $A \ltdyn B$.
%% Notice that the upcast of a value is a value, since it always succeeds, while the downcast
%% of a value is a computation, since it may fail.

%% \begin{mathpar}
%%     % Var
%%     \inferrule*{ }{\hasty {\cdot, \Gamma, x : A, \Gamma'} x A}

%%     % Err
%%     \inferrule*{ }{\hasty {\cdot, \Gamma} {\err_B} B} 
  
%%     % Zero and suc
%%     \inferrule*{ }{\hasty \Gamma \zro \nat}
  
%%     \inferrule*{\hasty \Gamma V \nat} {\hasty \Gamma {\suc\, V} \nat}

%%     % Match-nat
%%     \inferrule*
%%     {\hasty \Gamma V \nat \and 
%%      \hasty \Delta M B \and \hasty {\Delta, n : \nat} {M'} B}
%%     {\hasty \Delta {\matchnat {V} {M} {n} {M'}} B}
  
%%     % Lambda
%%     \inferrule* 
%%     {\hasty {\cdot, \Gamma, x : A} M {\Ret A'}} 
%%     {\hasty \Gamma {\lda x M} {A \ra A'}}
  
%%     % App
%%     \inferrule*
%%     {\hasty \Gamma {V_f} {A \ra A'} \and \hasty \Gamma {V_x} A}
%%     {\hasty {\cdot , \Gamma} {V_f \, V_x} {\Ret A'}}

%%     % Ret
%%     \inferrule*
%%     {\hasty \Gamma V A}
%%     {\hasty {\cdot , \Gamma} {\ret\, V} {\Ret A}}
%%     % TODO should this involve a Delta?

%%     % Bind
%%     \inferrule*
%%     {\hasty \Delta M {\Ret A} \and \hasty{\cdot , \Delta|_V , x : A}{N}{B} } % Need x : A in context
%%     {\hasty {\Delta} {\bind{x}{M}{N}} {B}}

%%     % Upcast
%%     \inferrule*
%%     {A \ltdyn A' \and \hasty \Gamma V A}
%%     {\hasty \Gamma {\up A {A'} V} {A'} }

%%     % Downcast
%%     % \inferrule*
%%     % {A \ltdyn A' \and \hasty {\Gamma} V {A'}}
%%     % {\hasty {\cdot, \Gamma} {\dn A {A'} V} {\Ret A}}

%%     \inferrule* % TODO is this correct?
%%     {B \ltdyn B' \and \hasty {\Delta} {M} {B'}}
%%     {\hasty {\Delta} {\dn B {B'} M} {B}}

%% \end{mathpar}


%% In the equational theory, we have $\beta$ and $\eta$ laws for function type,
%% as well a $\beta$ and $\eta$ law for $\Ret A$.

%% % TODO do we need to add a substitution rule here?
%% \begin{mathpar}
%%   % Function Beta and Eta
%%   \inferrule*
%%   {\hasty {\cdot, \Gamma, x : A} M {\Ret A'} \and \hasty \Gamma V A}
%%   {(\lda x M)\, V = M[V/x]}

%%   \inferrule*
%%   {\hasty \Gamma V {A \ra A}}
%%   {\Gamma \vdash V = \lda x {V\, x}}

%%   % Ret Beta and Eta
%%   \inferrule*
%%   {}
%%   {(\bind{x}{\ret\, V}{N}) = N[V/x]}

%%   \inferrule*
%%   {\hasty {\hole{\Ret A} , \Gamma} {M} {B}}
%%   {\hole{\Ret A}, \Gamma \vdash M = (\bind{x}{\bullet}{M[\ret\, x]})}

%%   % Match-nat Beta
%%   \inferrule*
%%   {\hasty \Delta M B \and \hasty {\Delta, n : \nat} {M'} B}
%%   {\matchnat{\zro}{M}{n}{M'} = M}

%%   \inferrule*
%%   {\hasty \Gamma V \nat \and 
%%    \hasty \Delta M B \and \hasty {\Delta, n : \nat} {M'} B}
%%   {\matchnat{\suc\, V}{M}{n}{M'} = M'}

%%   % Match-nat Eta
%%   % This doesn't build in substitution
%%   \inferrule*
%%   {\hasty {\Delta , x : \nat} M A}
%%   {M = \matchnat{x} {M[\zro / x]} {n} {M[(\suc\, n) / x]}}



%% \end{mathpar}

%% % ---------------------------------------------------------------------------------------
%% % ---------------------------------------------------------------------------------------

%% \subsection{Type Precision}

%% The type precision rules specify what it means for a type $A$ to be more precise than $A'$.
%% We have reflexivity rules for $\dyn$ and $\nat$, as well as rules that $\nat$ is more precise than $\dyn$
%% and $\dyn \ra \dyn$ is more precise than $\dyn$.
%% We also have a transitivity rule for composition of type precision,
%% and also a rule for function types stating that given $A_i \ltdyn A'_i$ and $A_o \ltdyn A'_o$, we can prove
%% $A_i \ra A_o \ltdyn A'_i \ra A'_o$.
%% Finally, we can lift a relation on value types $A \ltdyn A'$ to a relation $\Ret A \ltdyn \Ret A'$ on
%% computation types.

%% \begin{mathpar}
%%   \inferrule*[right = \dyn]
%%     { }{\dyn \ltdyn\, \dyn}

%%   \inferrule*[right = \nat]
%%     { }{\nat \ltdyn \nat}

%%   \inferrule*[right = $\ra$]
%%     {A_i \ltdyn A'_i \and A_o \ltdyn A'_o }
%%     {(A_i \ra A_o) \ltdyn (A'_i \ra A'_o)}

%%   \inferrule*[right = $\textsf{Inj}_\nat$]
%%     { }{\nat \ltdyn\, \dyn}

%%   \inferrule*[right=$\textsf{Inj}_{\ra}$]
%%     { }
%%     {(\dyn \ra \dyn) \ltdyn\, \dyn}

%%   \inferrule*[right=ValTrans]
%%     {A \ltdyn A' \and A' \ltdyn A''}
%%     {A \ltdyn A''}

%%   \inferrule*[right=CompTrans]
%%     {B \ltdyn B' \and B' \ltdyn B''}
%%     {B \ltdyn B''}

%%   \inferrule*[right=$\Ret{}$]
%%     {A \ltdyn A'}
%%     {\Ret {A} \ltdyn \Ret {A'}}

%%     % TODO are there other rules needed for computation types?

  
%% \end{mathpar}

%% % Type precision derivations
%% Note that as a consequence of this presentation of the type precision rules, we
%% have that if $A \ltdyn A'$, there is a unique precision derivation that witnesses this.
%% As in previous work, we go a step farther and make these derivations first-class objects,
%% known as \emph{type precision derivations}.
%% Specifically, for every $A \ltdyn A'$, we have a derivation $c : A \ltdyn A'$ that is constructed
%% using the rules above. For instance, there is a derivation $\dyn : \dyn \ltdyn \dyn$, and a derivation
%% $\nat : \nat \ltdyn \nat$, and if $c_i : A_i \ltdyn A_i$ and $c_o : A_o \ltdyn A'_o$, then
%% there is a derivation $c_i \ra c_o : (A_i \ra A_o) \ltdyn (A'_i \ra A'_o)$. Likewise for
%% the remaining rules. The benefit to making these derivations explicit in the syntax is that we
%% can perform induction over them.
%% Note also that for any type $A$, we use $A$ to denote the reflexivity derivation that $A \ltdyn A$,
%% i.e., $A : A \ltdyn A$.
%% Finally, observe that for type precision derivations $c : A \ltdyn A'$ and $c' : A' \ltdyn A''$, we
%% can define (via the rule ValComp) their composition $c \relcomp c' : A \ltdyn A''$.
%% The same holds for computation type precision derivations.
%% This notion will be used below in the statement of transitivity of the term precision relation.

%% % ---------------------------------------------------------------------------------------
%% % ---------------------------------------------------------------------------------------

%% \subsection{Term Precision}

%% We allow for a \emph{heterogeneous} term precision judgment on terms values $V$ of type
%% $A$ and $V'$ of type $A'$ provided that $A \ltdyn A'$ holds. Likewise, for computation
%% types $B \ltdyn B'$, if $M$ has type $B$ and $M'$ has type $B'$, we can form the judgment
%% that $M \ltdyn M'$.

%% % Type precision contexts
%% % TODO should we include the formal definitions of value and computation type precision contexts?
%% In order to deal with open terms, we will need the notion of a type precision \emph{context}, which we denote
%% $\gamlt$. This is similar to a normal context but instead of mapping variables to types,
%% it maps variables $x$ to related types $A \ltdyn A'$, where $x$ has type $A$ in the left-hand term
%% and $B$ in the right-hand term. We may also write $x : d$ where $d : A \ltdyn A'$ to indicate this.
%% Similarly, we have computation type precision contexts $\Delta^\ltdyn$. Similar to ``normal'' computation
%% type precision contexts $\Delta$, these consist of (1) a stoup $\Sigma$ which is either empty or
%% has a hole $\hole{d}$ for some computation type precision derivation $d$, and (2) a value type precision context
%% $\Gamma^\ltdyn$.

%% % An equivalent way of thinking of type precision contexts is as a pair of ``normal" typing
%% % contexts $\Gamma, \Gamma'$ with the same domain such that $\Gamma(x) \ltdyn \Gamma'(x)$ for
%% % each $x$ in the domain.
%% % We will write $\gamlt : \Gamma \ltdyn \Gamma'$ when we want to emphasize the pair of contexts.
%% % Conversely, if we are given $\gamlt$, we write $\gamlt_l$ and $\gamlt_r$ for the normal typing contexts on each side.

%% An equivalent way of thinking of a type precision context $\gamlt$ is as a
%% pair of ``normal" typing contexts, $\gamlt_l$ and $\gamlt_r$, with the same
%% domain and such that $\gamlt_l(x) \ltdyn \gamlt_r(x)$ for each $x$ in the domain.
%% We will write $\gamlt : \gamlt_l \ltdyn \gamlt_r$ when we want to emphasize the pair of contexts.

%% As with type precision derivations, we write $\Gamma$ to mean the ``reflexivity" type precision context
%% $\Gamma : \Gamma \ltdyn \Gamma$.
%% Concretely, this consists of reflexivity type precision derivations $\Gamma(x) \ltdyn \Gamma(x)$ for
%% each $x$ in the domain of $\Gamma$.
%% Similarly, we also have reflexivity for computation type precision contexts.
%% %
%% Furthermore, we write $\gamlt_1 \relcomp \gamlt_2$ to denote the ``composition'' of $\gamlt_1$ and $\gamlt_2$
%% --- that is, the precision context whose value at $x$ is the type precision derivation
%% $\gamlt_1(x) \relcomp \gamlt_2(x)$. This of course assumes that each of the type precision
%% derivations is composable, i.e., that the RHS of $\gamlt_1(x)$ is the same as the left-hand side of $\gamlt_2(x)$.
%% We define the same for computation type precision contexts $\deltalt_1$ and $\deltalt_2$,
%% provided that both the computation type precision contexts have the same ``shape'', which is defined as
%% (1) either the stoup is empty in both, or the stoup has a hole in both, say $\hole{d}$ and $\hole{d'}$
%% where $d$ and $d'$ are composable, and (2) their value type precision contexts are composable as described above.

%% The rules for term precision come in two forms. We first have the \emph{congruence} rules,
%% one for each term constructor. These assert that the term constructors respect term precision.
%% The congruence rules are as follows:

%% \begin{mathpar}

%%   \inferrule*[right = Var]
%%     { c : A \ltdyn B \and \gamlt(x) = (A, B) } 
%%     { \etmprec {\gamlt} x x c }

%%   \inferrule*[right = Zro]
%%     { } {\etmprec \gamlt \zro \zro \nat }

%%   \inferrule*[right = Suc]
%%     { \etmprec \gamlt V {V'} \nat } {\etmprec \gamlt {\suc\, V} {\suc\, V'} \nat}

%%   \inferrule*[right = MatchNat]
%%   {\etmprec \gamlt V {V'} \nat \and 
%%     \etmprec \deltalt M {M'} d \and \etmprec {\deltalt, n : \nat} {N} {N'} d}
%%   {\etmprec \deltalt {\matchnat {V} {M} {n} {N}} {\matchnat {V'} {M'} {n} {N'}} d}

%%   \inferrule*[right = Lambda]
%%     { c_i : A_i \ltdyn A'_i \and 
%%       c_o : A_o \ltdyn A'_o \and 
%%       \etmprec {\cdot , \gamlt , x : c_i} {M} {M'} {\Ret c_o} } 
%%     { \etmprec \gamlt {\lda x M} {\lda x {M'}} {(c_i \ra c_o)} }

%%   \inferrule*[right = App]
%%     { c_i : A_i \ltdyn A'_i \and
%%       c_o : A_o \ltdyn A'_o \\\\
%%       \etmprec \gamlt {V_f} {V_f'} {(c_i \ra c_o)} \and
%%       \etmprec \gamlt {V_x} {V_x'} {c_i}
%%     } 
%%     { \etmprec {\cdot , \gamlt} {V_f\, V_x} {V_f'\, V_x'} {\Ret {c_o}}}

%%   \inferrule*[right = Ret]
%%     {\etmprec {\gamlt} V {V'} c}
%%     {\etmprec {\cdot , \gamlt} {\ret\, V} {\ret\, V'} {\Ret c}}

%%   \inferrule*[right = Bind]
%%     {\etmprec {\deltalt} {M} {M'} {\Ret c} \and 
%%      \etmprec {\cdot , \deltalt|_V , x : c} {N} {N'} {d} }
%%     {\etmprec {\deltalt} {\bind {x} {M} {N}} {\bind {x} {M'} {N'}} {d}}
%% \end{mathpar}

%% We then have additional equational axioms, including transitivity, $\beta$ and $\eta$ laws, and
%% rules characterizing upcasts as least upper bounds, and downcasts as greatest lower bounds.

%% We write $M \equidyn N$ to mean that both $M \ltdyn N$ and $N \ltdyn M$.

%% % TODO adapt these for value/computation distinction
%% % TODO substitution rules for values and terms?
%% \begin{mathpar}
%%   \inferrule*[right = $\err$]
%%     { \hasty {\deltalt_l} M B }
%%     {\etmprec {\Delta} {\err_B} M B}

%%   \inferrule*[right = Transitivity]
%%     { d : B \ltdyn B' \and d' : B' \ltdyn B'' \\\\
%%      \etmprec {\deltalt_1} {M} {M'} {d} \and
%%      \etmprec {\deltalt_2} {M'} {M''} {d'} } 
%%     {\etmprec {\deltalt_1 \relcomp \deltalt_2} {M} {M''} {d \relcomp d'} }


%%   \inferrule*[right = $\beta$-fun]
%%     { \hasty {\cdot, \Gamma, x : A_i} M {\Ret A_o} \and
%%       \hasty {\Gamma} V {A_i} } 
%%     { \etmequidyn {\cdot, \Gamma} {(\lda x M)\, V} {M[V/x]} {\Ret A_o} }

%%   \inferrule*[right = $\eta$-fun]
%%     { \hasty {\Gamma} {V} {A_i \ra A_o} } 
%%     { \etmequidyn \Gamma {\lda x (V\, x)} V {A_i \ra A_o} }

%%   % Match-nat beta and eta



%%   \inferrule*[right = $\beta$-ret]
%%     {}
%%     {\bind{x}{\ret\, V}{N} \equidyn N[V/x]}

%%   \inferrule*[right = $\eta$-ret]
%%     {\hasty {\hole{\Ret A} , \Gamma} {M} {B}}
%%     {\hole{\Ret A}, \Gamma \vdash M \equidyn \bind{x}{\bullet}{M[\ret\, x]}}
    

%%   % Could specify \gamlt : \Gamma \ltdyn \Gamma'
%%   % and then we wouldn't need to say l and r

%%   \inferrule*[right = UpR]
%%     { d : A \ltdyn A' \and 
%%       \hasty {\Delta} {M} {A} } 
%%     { \etmprec {\Delta} {M} {\up {A} {A'} M} {d}  }

%%   \inferrule*[right = UpL]
%%     { d : A \ltdyn A' \and
%%       \etmprec {\deltalt} {M} {N} {d} } 
%%     { \etmprec {\deltalt} {\up {A} {A'} M} {N} {A'} }

%%   \inferrule*[right = DnL]
%%     { d : B \ltdyn B' \and 
%%       \hasty {\Delta} {M} {B'} } 
%%     { \etmprec {\Delta} {\dn {B} {B'} M} {M} {d} }

%%   \inferrule*[right = DnR]
%%     { d : B \ltdyn B' \and
%%       \etmprec {\deltalt} {M} {N} {d} } 
%%     { \etmprec {\deltalt} {M} {\dn {B} {B'} N} {B} }
%% \end{mathpar}

%% % TODO explain the least upper bound/greatest lower bound rules
%% The rules UpR, UpL, DnL, and DnR were introduced in \cite{new-licata18} as a means
%% of cleanly axiomatizing the intended behavior of casts in a way that
%% doesn't depend on the specific constructs of the language.
%% Intuitively, rule UpR says that the upcast of $M$ is an upper bound for $M$
%% in that $M$ may error more, and UpL says that the upcast is the \emph{least}
%% such upper bound, in that it errors more than any other upper bound for $M$.
%% Conversely, DnL says that the downcast of $M$ is a lower bound, and DnR says
%% that it is the \emph{greatest} lower bound.
%% % These rules provide a clean axiomatization of the behavior of casts that doesn't
%% % depend on the specific constructs of the language.

%% % ---------------------------------------------------------------------------------------
%% % ---------------------------------------------------------------------------------------
%% \subsection{Removing Transitivity as a Primitive}

%% The first observation we make is that transitivity of type precision, and heterogeneous
%% transitivity of term precision, are admissible. That is, consider a related language which
%% is the same as $\extlc$ except that we have removed the composition rule for type precision and
%% the heterogeneous transitivity rule for type precision. Denote this language by $\extlcm$.
%% We claim that in this new language, the rules we removed are derivable from the remaining rules.

%% To see this, suppose $\gamlt : \Gamma \ltdyn \Gamma'$ and $d : A \ltdyn A'$, and that
%%  $\etmprec {\gamlt} {V} {V'} {d}$, as shown in the diagram below:

%% % https://q.uiver.app/?q=WzAsNCxbMCwwLCJcXEdhbW1hIl0sWzAsMSwiXFxHYW1tYSciXSxbMSwwLCJBIl0sWzEsMSwiQSciXSxbMCwxLCJcXGx0ZHluIiwzLHsic3R5bGUiOnsiYm9keSI6eyJuYW1lIjoibm9uZSJ9LCJoZWFkIjp7Im5hbWUiOiJub25lIn19fV0sWzIsMywiXFxsdGR5biIsMyx7InN0eWxlIjp7ImJvZHkiOnsibmFtZSI6Im5vbmUifSwiaGVhZCI6eyJuYW1lIjoibm9uZSJ9fX1dLFswLDIsIlYiXSxbMSwzLCJWJyJdLFs2LDcsIlxcbHRkeW4iLDMseyJzaG9ydGVuIjp7InNvdXJjZSI6MjAsInRhcmdldCI6MjB9LCJzdHlsZSI6eyJib2R5Ijp7Im5hbWUiOiJub25lIn0sImhlYWQiOnsibmFtZSI6Im5vbmUifX19XV0=
%% \[\begin{tikzcd}[ampersand replacement=\&]
%% 	\Gamma \& A \\
%% 	{\Gamma'} \& {A'}
%% 	\arrow["\ltdyn"{marking}, draw=none, from=1-1, to=2-1]
%% 	\arrow["\ltdyn"{marking}, draw=none, from=1-2, to=2-2]
%% 	\arrow[""{name=0, anchor=center, inner sep=0}, "V", from=1-1, to=1-2]
%% 	\arrow[""{name=1, anchor=center, inner sep=0}, "{V'}", from=2-1, to=2-2]
%% 	\arrow["\ltdyn"{marking}, draw=none, from=0, to=1]
%% \end{tikzcd}\]

%% Now note that this is equivalent, by the cast rule UpL, to
%% $\etmprec {\Gamma'} {\up{A}{A'} V} {V'} {A'}$,
%% where as noted above, $\Gamma'$ refers to the context $\Gamma'$ viewed as a reflexivity
%% precision context and likewise the $A'$ at the end refers to the reflexivity derivation $A' \ltdyn A'$.

%% % https://q.uiver.app/?q=WzAsMixbMCwwLCJcXEdhbW1hJyJdLFsxLDAsIkEnIl0sWzAsMSwiXFx1cCB7QX0ge0EnfSBWIiwwLHsiY3VydmUiOi0yfV0sWzAsMSwiViciLDIseyJjdXJ2ZSI6Mn1dLFsyLDMsIlxcbHRkeW4iLDMseyJzaG9ydGVuIjp7InNvdXJjZSI6MjAsInRhcmdldCI6MjB9LCJzdHlsZSI6eyJib2R5Ijp7Im5hbWUiOiJub25lIn0sImhlYWQiOnsibmFtZSI6Im5vbmUifX19XV0=
%% \[\begin{tikzcd}[ampersand replacement=\&]
%% 	{\Gamma'} \& {A'}
%% 	\arrow[""{name=0, anchor=center, inner sep=0}, "{\up {A} {A'} V}", curve={height=-12pt}, from=1-1, to=1-2]
%% 	\arrow[""{name=1, anchor=center, inner sep=0}, "{V'}"', curve={height=12pt}, from=1-1, to=1-2]
%% 	\arrow["\ltdyn"{marking}, draw=none, from=0, to=1]
%% \end{tikzcd}\]

%% Now consider the situation shown below:

%% % https://q.uiver.app/?q=WzAsNixbMCwwLCJcXEdhbW1hIl0sWzAsMSwiXFxHYW1tYSciXSxbMCwyLCJcXEdhbW1hJyciXSxbMiwwLCJBIl0sWzIsMSwiQSciXSxbMiwyLCJBJyciXSxbMiw1LCJWJyciXSxbMSw0LCJWJyJdLFswLDMsIlYiXSxbMyw0LCJcXGx0ZHluIiwzLHsic3R5bGUiOnsiYm9keSI6eyJuYW1lIjoibm9uZSJ9LCJoZWFkIjp7Im5hbWUiOiJub25lIn19fV0sWzQsNSwiXFxsdGR5biIsMyx7InN0eWxlIjp7ImJvZHkiOnsibmFtZSI6Im5vbmUifSwiaGVhZCI6eyJuYW1lIjoibm9uZSJ9fX1dLFswLDEsIlxcbHRkeW4iLDMseyJzdHlsZSI6eyJib2R5Ijp7Im5hbWUiOiJub25lIn0sImhlYWQiOnsibmFtZSI6Im5vbmUifX19XSxbMSwyLCIiLDEseyJzdHlsZSI6eyJib2R5Ijp7Im5hbWUiOiJub25lIn0sImhlYWQiOnsibmFtZSI6Im5vbmUifX19XSxbMSwyLCJcXGx0ZHluIiwzLHsic3R5bGUiOnsiYm9keSI6eyJuYW1lIjoibm9uZSJ9LCJoZWFkIjp7Im5hbWUiOiJub25lIn19fV0sWzgsNywiXFxsdGR5biIsMyx7InNob3J0ZW4iOnsic291cmNlIjoyMCwidGFyZ2V0IjoyMH0sInN0eWxlIjp7ImJvZHkiOnsibmFtZSI6Im5vbmUifSwiaGVhZCI6eyJuYW1lIjoibm9uZSJ9fX1dLFs3LDYsIlxcbHRkeW4iLDMseyJzaG9ydGVuIjp7InNvdXJjZSI6MjAsInRhcmdldCI6MjB9LCJzdHlsZSI6eyJib2R5Ijp7Im5hbWUiOiJub25lIn0sImhlYWQiOnsibmFtZSI6Im5vbmUifX19XV0=
%% \[\begin{tikzcd}[ampersand replacement=\&]
%% 	\Gamma \&\& A \\
%% 	{\Gamma'} \&\& {A'} \\
%% 	{\Gamma''} \&\& {A''}
%% 	\arrow[""{name=0, anchor=center, inner sep=0}, "{V''}", from=3-1, to=3-3]
%% 	\arrow[""{name=1, anchor=center, inner sep=0}, "{V'}", from=2-1, to=2-3]
%% 	\arrow[""{name=2, anchor=center, inner sep=0}, "V", from=1-1, to=1-3]
%% 	\arrow["\ltdyn"{marking}, draw=none, from=1-3, to=2-3]
%% 	\arrow["\ltdyn"{marking}, draw=none, from=2-3, to=3-3]
%% 	\arrow["\ltdyn"{marking}, draw=none, from=1-1, to=2-1]
%% 	\arrow[draw=none, from=2-1, to=3-1]
%% 	\arrow["\ltdyn"{marking}, draw=none, from=2-1, to=3-1]
%% 	\arrow["\ltdyn"{marking}, draw=none, from=2, to=1]
%% 	\arrow["\ltdyn"{marking}, draw=none, from=1, to=0]
%% \end{tikzcd}\]


%% Using the above observation, we have that the above is equivalent to

%% % https://q.uiver.app/?q=WzAsNCxbMCwwLCJcXEdhbW1hJyJdLFswLDEsIlxcR2FtbWEnJyJdLFsyLDAsIkEnIl0sWzIsMSwiQScnIl0sWzAsMiwiXFx1cCB7QX0ge0EnfSBWIiwwLHsiY3VydmUiOi0yfV0sWzAsMiwiViciLDIseyJjdXJ2ZSI6Mn1dLFsxLDMsIlYnJyIsMix7ImN1cnZlIjoyfV0sWzAsMSwiXFxsdGR5biIsMyx7InN0eWxlIjp7ImJvZHkiOnsibmFtZSI6Im5vbmUifSwiaGVhZCI6eyJuYW1lIjoibm9uZSJ9fX1dLFsyLDMsIlxcbHRkeW4iLDMseyJzdHlsZSI6eyJib2R5Ijp7Im5hbWUiOiJub25lIn0sImhlYWQiOnsibmFtZSI6Im5vbmUifX19XSxbNCw1LCJcXGx0ZHluIiwzLHsic2hvcnRlbiI6eyJzb3VyY2UiOjIwLCJ0YXJnZXQiOjIwfSwic3R5bGUiOnsiYm9keSI6eyJuYW1lIjoibm9uZSJ9LCJoZWFkIjp7Im5hbWUiOiJub25lIn19fV0sWzUsNiwiXFxsdGR5biIsMyx7InNob3J0ZW4iOnsic291cmNlIjoyMCwidGFyZ2V0IjoyMH0sInN0eWxlIjp7ImJvZHkiOnsibmFtZSI6Im5vbmUifSwiaGVhZCI6eyJuYW1lIjoibm9uZSJ9fX1dXQ==
%% \[\begin{tikzcd}[ampersand replacement=\&]
%% 	{\Gamma'} \&\& {A'} \\
%% 	{\Gamma''} \&\& {A''}
%% 	\arrow[""{name=0, anchor=center, inner sep=0}, "{\up {A} {A'} V}", curve={height=-12pt}, from=1-1, to=1-3]
%% 	\arrow[""{name=1, anchor=center, inner sep=0}, "{V'}"', curve={height=12pt}, from=1-1, to=1-3]
%% 	\arrow[""{name=2, anchor=center, inner sep=0}, "{V''}"', curve={height=12pt}, from=2-1, to=2-3]
%% 	\arrow["\ltdyn"{marking}, draw=none, from=1-1, to=2-1]
%% 	\arrow["\ltdyn"{marking}, draw=none, from=1-3, to=2-3]
%% 	\arrow["\ltdyn"{marking}, draw=none, from=0, to=1]
%% 	\arrow["\ltdyn"{marking}, draw=none, from=1, to=2]
%% \end{tikzcd}\]

%% % TODO finish the explanation
  

%% % ---------------------------------------------------------------------------------------
%% % ---------------------------------------------------------------------------------------

%% \subsection{Removing Casts as Primitives}

%% % We now observe that all casts, except those between $\nat$ and $\dyn$
%% % and between $\dyn \ra \dyn$ and $\dyn$, are admissible, in the sense that
%% % we can start from $\extlcm$, remove casts except the aforementioned ones,
%% % and in the resulting language we will be able to derive the other casts.

%% We now observe that all casts, except those between $\nat$ and $\dyn$
%% and between $\dyn \ra \dyn$ and $\dyn$, are admissible.
%% That is, consider a new language ($\extlcmm$) in which
%% instead of having arbitrary casts, we have injections from $\nat$ and
%% $\dyn \ra \dyn$ into $\dyn$, and case inspections from $\dyn$ to $\nat$ and
%% $\dyn$ to $\dyn \ra \dyn$. We claim that in $\extlcmm$, all of the casts
%% present in $\extlcm$ are derivable.
%% It will suffice to verify that casts for function type are derivable.
%% This holds because function casts are constructed inductively from the casts
%% of their domain and codomain. The base case is one of the casts involving $\nat$
%% or $\dyn \ra \dyn$ which are present in $\extlcmm$ as injections and case inspections.


%% The resulting calculus $\extlcmm$ now lacks transitivity of type precision,
%% heterogeneous transitivity of term precision, and arbitrary casts as primitive
%% notions.

%% \begin{align*}
%%   &\text{Value Types } A := \nat \alt \dyn \alt (A \ra A') \\
%%   &\text{Computation Types } B := \Ret A \\
%%   &\text{Value Contexts } \Gamma := \cdot \alt (\Gamma, x : A) \\
%%   &\text{Computation Contexts } \Delta := \cdot \alt \hole B \alt \Delta , x : A \\
%%   &\text{Values } V :=  \zro \alt \suc\, V \alt \lda{x}{M} \alt \injnat V \alt \injarr V \\ 
%%   &\text{Terms } M, N := \err_B \alt \ret {V} \alt \bind{x}{M}{N}
%%     \alt V_f\, V_x \alt
%%     \\ & \quad\quad \casenat{V}{M_{no}}{n}{M_{yes}} 
%%     \alt \casearr{V}{M_{no}}{f}{M_{yes}}
%% \end{align*}

%% In this setting, rather than type precision, it makes more sense to
%% speak of arbitrary \emph{monotone relations} on types, which we denote by $A \rel A'$.
%% We have relations on value types, as well as on computation types. We also have
%% value relation contexts and computation relation contexts, analogous to the value type
%% precision contexts and computation type precision contexts from before.

%% \begin{align*}
%%   &\text{Value Relations } R := \nat \alt \dyn \alt (R \ra R) \alt\, \dyn\, R(V_1, V_2)\\
%%   &\text{Computation Relations } S := \li R \\
%%   &\text{Value Relation Contexts } \Gamma^{\rel} := \cdot \alt \Gamma^{\rel} , A^{\rel} (x_l : A_l , x_r : A_r)\\
%%   &\text{Computation Relation Contexts } \Delta^{\rel} := \cdot \alt \hole{B^{\rel}} \alt 
%%     \Delta^{\rel} , A^{\rel} (x_l : A_l , x_r : A_r)   \\
%% \end{align*}

%% % TODO rules for relations
%% The forms for relations are as follows:

%% \begin{align*}
%%   A^{\rel}      &\colon A_l      \rel A_r \\
%%   \Gamma^{\rel} &\colon \Gamma_l \rel \Gamma_r \\
%%   B^{\rel}      &\colon B_l      \rel B_r \\
%%   \Delta^{\rel} &\colon \Delta_l \rel \Delta_r
%% \end{align*}



%% Figure \ref{fig:relation-rules} shows the rules for relations. We show only those for value types;
%% the corresponding computation type relation rules are analogous.
%% The rules for relations are as follows. First, we require relations to be reflexive.
%% We also require that they are \emph{profunctorial}, in the sense that a relation between
%% $A$ and $A'$ is closed under the ``homogeneous'' relations on both sides.
%% We also require that they satisfy a substitution principle.

%% \begin{figure}
%%   \begin{mathpar}
%%     \inferrule*[right = Reflexivity]
%%     {\hasty \Gamma V A}
%%     {\refl(\Gamma) \vdash \refl(A)(V, V)}

%%     \inferrule*[right = Profunctoriality]
%%     { \refl(\Gamma^{\rel}_l) \vdash  \refl(A^{\rel}_l) (V_l' , V_l) \\\\ 
%%         \Gamma^{\rel}    \vdash    A^{\rel}    (V_l  , V_r) \\\\
%%       \refl(\Gamma^{\rel}_r) \vdash  \refl(A^{\rel}_r) (V_r  , V_r')
%%     }
%%     {\Gamma^{\rel} \vdash A^{\rel} (V_l', V_r')}

%%     \inferrule*[right = Subst]
%%     { \Gamma'^{\rel} \vdash \Gamma^{\rel} (\gamma_l, \gamma_r) \\\\
%%       \Gamma^{\rel} A^{\rel} (V_l, V_r)
%%     }
%%     {\Gamma'^{\rel} \vdash A^{\rel} (V_l[\gamma_l] , V_r[\gamma_r]) }

%%     % \inferrule*[right = TermSubst]
%%     % { \Delta'^{\rel} \vdash \Delta^{\rel} (\delta_l, \delta_r) \\\\
%%     %   \Delta^{\rel} B^{\rel} (M_l, M_r)
%%     % }
%%     % {\Delta'^{\rel} \vdash B^{\rel} (M_l[\delta_l] , M_r[\delta_r]) }

%%   \end{mathpar}
%%   \caption{Rules for value type relations. The rules for computation type relations are analogous.}
%%   \label{fig:relation-rules}
%% \end{figure}

%% We also have a rule for the restriction of a relation along a function,
%% and we have a rule characterizing relation at function type. The latter states that
%% if under the assumption that $x$ is related to $x'$ by $A^{\rel}$, we can show that $M$
%% is related to $M'$ by $\li A'^{\rel}$, then we have that $\lda{x}{M}$ is related to
%% $\lda{x'}{M'}$ by $A^{\rel} \ra A'^{\rel}$.

%% \begin{mathpar}
%%   \mprset{fraction={===}}

%%   % \inferrule*[]
%%   % { A^{\rel}  (x_l, x_r) \vdash A^{\rel} (V_l, V_r) }
%%   % { A'^{\rel} (x_l, x_r) \vdash A^{\rel} (V_l, V_r)(x_l, x_r) }

%%   \inferrule*[right = Restriction]
%%   { \Gamma^{\rel} \vdash A^{\rel} (V_l (V_l'), V_r (V_r')) }
%%   { \Gamma^{\rel} \vdash (A^{\rel} (V_l, V_r)) (V_l', V_r') }

%%   \inferrule*[right = $\text{Rel}_\ra$]
%%   { A^{\rel} (x, x') \vdash (\li A'^{\rel})(M , M') }
%%   {  \vdash (A^{\rel} \ra A'^{\rel}) (\lda{x}{M}) , (\lda{x'}{M'})}

%% \end{mathpar}



%% % New rules
%% Figure \ref{fig:extlc-minus-minus-typing} shows the new typing rules,
%% and Figure \ref{fig:extlc-minus-minus-eqns} shows the equational rules
%% for case-nat (the rules for case-arrow are analogous).

%% \begin{figure}
%%   \begin{mathpar}
%%       % inj-nat
%%       \inferrule*
%%       {\hasty \Gamma M \nat}
%%       {\hasty \Gamma {\injnat M} \dyn}

%%       % inj-arr 
%%       \inferrule*
%%       {\hasty \Gamma M (\dyn \ra \dyn)}
%%       {\hasty \Gamma {\injarr M} \dyn}

%%       % Case nat
%%       \inferrule*
%%       {\hasty{\Delta|_V}{V}{\dyn} \and 
%%         \hasty{\Delta , x : \nat }{M_{yes}}{B} \and 
%%         \hasty{\Delta}{M_{no}}{B}}
%%       {\hasty {\Delta} {\casenat{V}{M_{no}}{n}{M_{yes}}} {B}}
    
%%       % Case arr
%%       \inferrule*
%%       {\hasty{\Delta|_V}{V}{\dyn} \and 
%%         \hasty{\Delta , x : (\dyn \ra \dyn) }{M_{yes}}{B} \and 
%%         \hasty{\Delta}{M_{no}}{B}}
%%       {\hasty {\Delta} {\casearr{V}{M_{no}}{f}{M_{yes}}} {B}}
%%   \end{mathpar}
%%   \caption{New typing rules for $\extlcmm$.}
%%   \label{fig:extlc-minus-minus-typing}
%% \end{figure}


%% \begin{figure}
%%   \begin{mathpar}
%%      % Case-nat Beta
%%      \inferrule*
%%      {\hasty \Gamma V \nat}
%%      {\casenat {\injnat {V}} {M_{no}} {n} {M_{yes}} = M_{yes}[V/n]}

%%      \inferrule*
%%      {\hasty \Gamma V {\dyn \ra \dyn} }
%%      {\casenat {\injarr {V}} {M_{no}} {n} {M_{yes}} = M_{no}}

%%      % Case-nat Eta
%%      \inferrule*
%%      {}
%%      {\Gamma , x :\, \dyn \vdash M = \casenat{x}{M}{n}{M[(\injnat{n}) / x]} }


%%      % Case-arr Beta


%%      % Case-arr Eta


%%   \end{mathpar}
%%   \caption{New equational rules for $\extlcmm$ (rules for case-arrow are analogous
%%            and hence are omitted).}
%%   \label{fig:extlc-minus-minus-eqns}
%% \end{figure}



%% % TODO : Updated term precision rules



%% \subsection{The Step-Sensitive Lambda Calculus}\label{sec:step-sensitive-lc}

%% % \textbf{TODO: Subject to change!}

%% Rather than give a semantics to $\extlcmm$ directly, we first introduce another intermediary
%% language, a \emph{step-sensitive} (also called \emph{intensional}) calculus.
%% As mentioned, this language makes the intensional stepping behavior of programs
%% explicit in the syntax. We do this by adding a syntactic ``later'' type and a
%% syntactic $\theta$ that takes terms of type later $A$ to terms of type $A$.

%% % In the step-sensitive syntax, we add a type constructor for later, as well as a
%% % syntactic $\theta$ term and a syntactic $\nxt$ term.
%% We add rules for these new constructs, and also modify the rules for inj-arr and
%% case-arrow, since now the function is not $\Dyn \ra \Dyn$ but rather $\later (\Dyn \ra \Dyn)$.
%% We also add congruence relations for $\later$ and $\nxt$.

%% % TODO show changes

%% \noindent Modified syntax:
%% \begin{align*}
%%   &\text{Value Types } A := \nat \alt \dyn \alt (A \ra A') \alt {\color{red} \later A} \\
%%   &\text{Values } V :=  \zro \alt \suc\, V \alt \lda{x}{M} \alt \injnat V \alt \injarr V 
%%     \alt {\color{red} \nxt\, V} \alt {\color{red} \mathbf{\theta}}
%% \end{align*}

%% \noindent Additional typing rules:
%% \begin{mathpar}
%%   \inferrule
%%   {\hasty \Gamma V A}
%%   {\hasty \Gamma {\nxt\, V} {\later A}}

%%   \inferrule
%%   {}
%%   {\hasty \Gamma \theta {\later A \ra A}}

%%   % \theta(\nxt x) = \theta(y); \texttt{ret}\, x
%% \end{mathpar}

%% \noindent Modified typing rules:
%% \begin{mathpar}

%%   % inj-arr 
%%   \inferrule*
%%   {\hasty \Gamma M {\color{red} \later (\dyn \ra \dyn)}}
%%   {\hasty \Gamma {\injarr M} \dyn}

%%   % Case arr
%%   % TODO if the extensional version is incorrect and needs to change, make
%%   % sure to change this one accordingly
%%   \inferrule*
%%   {\hasty{\Delta|_V}{V}{\dyn} \and 
%%     \hasty{\Delta , x \colon {\color{red} \later (\dyn \ra \dyn)} }{M_{yes}}{B} \and 
%%     \hasty{\Delta}{M_{no}}{B}}
%%   {\hasty {\Delta} {\casearr{V}{M_{no}}{\tilde{f}}{M_{yes}}} {B}}  
%% \end{mathpar}

%% \noindent Additional relations:
%% \begin{mathpar}
%%   \inferrule*[]
%%   {A^{\rel} : A_l \rel A_r}
%%   {\later A^{\rel} : \later A_l \rel \later A_r}

%%   \inferrule*[]
%%   {A^{\rel} (V_l, V_r)}
%%   {\later A^{\rel} (\nxt\, V_l, \nxt\, V_r)}

%% \end{mathpar}

%% % TODO what about the relation for theta? Or is that automatic since it's a function symbol?

%% % TODO beta rule for theta

%% We define the term $\delta$ to be the function $\lda {x} {\theta\, (\nxt\, x)}$.

%% % We define an erasure function from step-sensitive syntax to step-insensitive syntax
%% % by induction on the step-sensitive types and terms.
%% % The basic idea is that the syntactic type $\later A$ erases to $A$,
%% % and $\nxt$ and $\theta$ erase to the identity.



%% \subsection{Quotienting by Syntactic Bisimilarity}

%% We now define a quotiented variant of the above step-sensitive calculus,
%% which we denote by $\intlcbisim$.
%% In this syntax, we add a rule saying, roughly speaking, that 
%% $\theta \circ \nxt$ is the identity. This causes terms that differ only in
%% their intensional behavior to become equal.
%% Note that a priori, this is not the same language as the step-insensitive
%% calculus on which we based the insensitive calculus.

%% Formally, the equational theory for the quotiented syntax is the same as
%% that of the original step-sensitive language, with the addition of the following
%% rule:

%% % TODO is this correct?
%% \begin{mathpar}
%%   \inferrule*
%%   { }
%%   { \theta\, (\nxt\, x) = \bind{y}{(\theta\, V')}{\ret\, x}  }
%% \end{mathpar}

%% This states that the application of $\theta$ to $\nxt\, x$ is equivalent to
%% the computation that applies $\theta$ to $V'$ to obtain a variable $y$, and
%% then simply returns $x$.




\section{Idealized Double Categorical Models of Graduality}
\label{sec:cbpv}

In order to organize our construction of denotational models we first
develop sufficient \emph{abstract} categorical semantics of gradually
typed languages. We start by modeling the type and term structure of
gradual typing and then extend this to type and term precision.
%
Gradually typed languages inherently involve computational effects of
errors and non-termination and typically in practice many others such
as mutable state and I/O.
%
To model this cleanly categorically, we follow New, Licata and Ahmed's
GTT calculus and base our models off of Levy's Call-by-push-value
(CBPV) calculus which is a standard model of effectful programming
\cite{levy99}.
%
There are several notions of model of CBPV from the literature with
varying requirements of which connectives are present
\cite{levy99,cfm2016,eec}, we will use a variant which models precisely
the connectives we require and no more
($1,\times,F,U,\to$)\footnote{It is essential in this case that we do
not require a cartesian closed category of values as there is no way
to implement casts for an exponential in general.}.

\begin{enumerate}
\item A cartesian category $\mathcal V$ and a category $\mathcal E$.
\item An action of $\mathcal V^{op}$ (with the $\mathcal V$ cartesian
  product as monoidal structure) on $\mathcal E$. We write this with
  an arrow $A \arr B$.
  This means we have natural isomorphisms
  $\alpha : {A_1 \times A_2} \arr B \cong A_2 \arr (A_1 \arr B)$ and $i : 1 \arr B \cong B$ satisfying pentagon and triangle identities\cite{action}.
\item $F \dashv U$ where $U : \mathcal E \to \mathcal V$ such that $U$ ``preserves
  powering'' in that every $U(A \arr B)$ is an exponential of $UB$ by $A$
  and that $U\alpha$ and $Ui$ are mapped to the canonical isomorphisms
  for exponentials.
\end{enumerate}

\begin{example}
  Given a strong monad $T$ on a bicartesian closed category $\mathcal
  V$, we can extend this to a CBPV model by defining $\mathcal E$ to
  be the category $\mathcal V^T$ of algebras of the monad, defining $A
  \to B$ as the powering of algebras, $F$ as the free algebra and $U$
  as the underlying object functor.
\end{example}

To additionally model the error terms, we add a requirement that there
is a natural transformation $\mho : 1 \to U$. The naturality
requirement encodes that strict morphisms (e.g., the denotations of
evaluation contexts) preserve errors.

We can then model CBV terms and types in a straightforward adaptation
of Levy's interpretation of CBV in CBPV. We interpret types $A$ as
objects $A \in \mathcal V$ and CBV terms $\Gamma \vdash M : A$ as
morphism of any of the equivalent forms $\mathcal E(F(\times\Gamma),
F(A)) \cong \mathcal V(\times\Gamma \vdash UF(A)) \cong \mathcal
E(F(1), \Gamma \to F(A))$. The most interesting type translation is
the CBV function type: $A \ra A' = U(A \to F A')$.
%
Such a model validates all type-based equational reasoning, i.e.,
$\beta\eta$ equality, and models the introduction and elimination
rules for CBV.
%
Thus a CBPV model is sufficient to interpret the CBV term language. We
will require additional structure to interpret the precision and type
casts.

\subsection{Double Categorical Semantics of Graduality}

New and Licata modeled the graduality and type casts for call-by-name
gradual typing using \emph{double categories}, which are defined to be
categories internal to the category of categories. That is, a double
category $\mathcal C$ consists of a category $\mathcal C_o$ of
``objects and function morphisms'' and a category $\mathcal C_{sq}$ of
``relation morphisms and squares'' with functors (reflexive relation)
$r : C_o \to C_{sq}$ and (source and target) $s,t : C_{sq} \to C_o$
satisfying $sr = tr = \id$ as well as a composition operation $c :
C_{sq} \times_{s,t} C_{sq} \to C_{sq}$ respecting source and
target. This models an abstract notion of functions and relations. For
notation, we write function morphisms as $f : A \to B$ and relation
morphisms as $c : A \rel B$ where $c \in C_{sq}$ and $s(c) = A$ and
$t(c) = B$. Finally a morphism $\alpha$ from $c$ to $d$ with
$s(\alpha) = f$ and $s(\beta) = g$ is visualized as
% https://q.uiver.app/#q=WzAsNCxbMCwwLCJBIl0sWzEsMCwiQiJdLFswLDEsIkEnIl0sWzEsMSwiQiciXSxbMCwyLCJmIiwyXSxbMSwzLCJnIl0sWzAsMSwiYyIsMCx7InN0eWxlIjp7ImJvZHkiOnsibmFtZSI6ImJhcnJlZCJ9LCJoZWFkIjp7Im5hbWUiOiJub25lIn19fV0sWzIsMywiZCIsMix7InN0eWxlIjp7ImJvZHkiOnsibmFtZSI6ImJhcnJlZCJ9LCJoZWFkIjp7Im5hbWUiOiJub25lIn19fV1d
\[\begin{tikzcd}[ampersand replacement=\&]
	A \& B \\
	{A'} \& {B'}
	\arrow["f"', from=1-1, to=2-1]
	\arrow["g", from=1-2, to=2-2]
	\arrow["c", "\shortmid"{marking}, no head, from=1-1, to=1-2]
	\arrow["d"', "\shortmid"{marking}, no head, from=2-1, to=2-2]
\end{tikzcd}\]
And is thought of as an abstraction of the notion of relatedness of
functions: functions take related inputs to related outputs. The
composition operations and functoriality give us a notion of
composition of relations as well as functions and vertical and
horizontal composition of squares. In this work we will be chiefly
interested in \emph{locally thin} double categories, that is, double
categories where there is at most one square for any $f,c,g,d$. In
this case we use the notation $f \leq_{c,d} g$ to mean that a square
like the above exists.

New, Licata and Ahmed \cite{new-licata-ahmed2019} extended the axiomatic
syntax to call-by-push-value but did not analyze the structure
categorically. We fill in this missing analysis now: a model of the
congruence rules of their system can be given by a locally thin
``double CBPV model'', which we define as a category internal to the
category of CBPV models and \emph{strict} homomorphisms of CBPV
models\footnote{it may be possible to also define this as a notion of
CBPV model internal to some structured $2$-category of categories, but
the authors are not aware of any such definition of an internal CBPV
model}. A strict homomorphism of CBPV models from $(\mathcal
V,\mathcal E,\ldots)$ to $(\mathcal V', \mathcal E',\ldots)$ consists
of functors $G_v : \mathcal V \to \mathcal V'$ and $G_e : \mathcal E
\to \mathcal E'$ that strictly preserve all CBPV constructions, see
the appendix for a more detailed definition. We call this a strict morphism in contrast to a \emph{lax} morphism, which only preserves CBPV constructions up to transformation.
Some of the data of a double CBPV model can be visualized as follows:
% https://q.uiver.app/#q=WzAsNCxbMCwwLCJcXHZzcSJdLFsyLDAsIlxcZXNxIl0sWzAsMiwiXFx2ZiJdLFsyLDIsIlxcZWYiXSxbMiwzLCJcXEZmIiwwLHsiY3VydmUiOi0yfV0sWzMsMiwiXFxVZiIsMCx7ImN1cnZlIjotMn1dLFswLDEsIlxcRnNxIiwwLHsiY3VydmUiOi0yfV0sWzEsMCwiXFxVc3EiLDAseyJjdXJ2ZSI6LTJ9XSxbMiwwLCJcXHJ2Il0sWzAsMiwiXFxzdiIsMCx7ImN1cnZlIjotMn1dLFswLDIsIlxcdHYiLDIseyJjdXJ2ZSI6Mn1dLFsxLDMsIlxcc2UiLDAseyJjdXJ2ZSI6LTJ9XSxbMSwzLCJcXHRlIiwyLHsiY3VydmUiOjJ9XSxbMywxLCJcXHJlIl0sWzQsNSwiXFxib3QiLDEseyJzaG9ydGVuIjp7InNvdXJjZSI6MjAsInRhcmdldCI6MjB9LCJzdHlsZSI6eyJib2R5Ijp7Im5hbWUiOiJub25lIn0sImhlYWQiOnsibmFtZSI6Im5vbmUifX19XSxbNiw3LCJcXGJvdCIsMSx7InNob3J0ZW4iOnsic291cmNlIjoyMCwidGFyZ2V0IjoyMH0sInN0eWxlIjp7ImJvZHkiOnsibmFtZSI6Im5vbmUifSwiaGVhZCI6eyJuYW1lIjoibm9uZSJ9fX1dXQ==
\[\begin{tikzcd}[ampersand replacement=\&]
	\vsq \&\& \esq \\
	\\
	\vf \&\& \ef
	\arrow[""{name=0, anchor=center, inner sep=0}, "\Ff", curve={height=-12pt}, from=3-1, to=3-3]
	\arrow[""{name=1, anchor=center, inner sep=0}, "\Uf", curve={height=-12pt}, from=3-3, to=3-1]
	\arrow[""{name=2, anchor=center, inner sep=0}, "\Fsq", curve={height=-12pt}, from=1-1, to=1-3]
	\arrow[""{name=3, anchor=center, inner sep=0}, "\Usq", curve={height=-12pt}, from=1-3, to=1-1]
	\arrow["\rv", from=3-1, to=1-1]
	\arrow["\sv", curve={height=-12pt}, from=1-1, to=3-1]
	\arrow["\tv"', curve={height=12pt}, from=1-1, to=3-1]
	\arrow["\se", curve={height=-12pt}, from=1-3, to=3-3]
	\arrow["\te"', curve={height=12pt}, from=1-3, to=3-3]
	\arrow["\re", from=3-3, to=1-3]
	\arrow["\bot"{description}, draw=none, from=0, to=1]
	\arrow["\bot"{description}, draw=none, from=2, to=3]
\end{tikzcd}\]
Type precision $A \ltdyn A'$ is interpreted as a relation morphism
$c_A : A \rel A'$ in $\mathcal V_{sq}$, and term precision $\Gamma
\ltdyn \Gamma' \vdash M \ltdyn M' : A \ltdyn A'$ is interpreted as a
square $M \ltdyn_{c_\Gamma,UF c_A} M'$. The fact that $t,r$ and the
composition are all given by strict CBPV homomorphisms says that all
the type constructors lift to precision (monotonicity of type
constructors) as well as all term constructors (congruence). Further,
$r$ and composition being strict homomorphisms implies that all type
constructors strictly preserve the identity relation (identity
extension) and composition.

Next, to model type casts, their model further requires that every
value relation $c : A \rel A'$ is \emph{left representable} by a
function $u_c : A \to A'$ and every computation relation $d : B \rel
B'$ is \emph{right representable} by a function $d_c : B' \to B$. In a locally thin double category, these are defined as follows:
\begin{definition}
  $c : A \rel B$ is left representable by $f : A \to B$ if $f \ltsq{c}{r(B)} \id$ and $\id \ltsq{r(A)}{c} f $.

  Dually, $c : A \rel B$ is right representable by $g : B \to A$ if
  $\id \ltsq{c}{r(A)} g$ and $g \ltsq{r(B)}{c} \id$.
\end{definition}
These rules are sufficient to model the UpL/UpR/DnL/DnR rules for
casts. Additionally, since representable morphisms compose and so the
compositionality of casts comes for free. However, the retraction
property must be added as an additional axiom to the model.  To model
the error being a least element we add the requirement that $\mho
\circ ! \ltsq{r(A)}{r(UB)} f$ holds for all $f : \mathcal V(A,B)$.
Finally, the dynamic type can be modeled as an arbitrary value type $D$ with
arbitrary relations $\nat \rel D$ and $D \ra D \rel D$ and $D \times D
\rel D$ (or whatever basic type cases are required).

\begin{example}
  (Adapted from \cite{new-licata18}): Define a double CBPV model where
  $\mathcal V$ is the category of predomain preorders: sets with an
  $\omega$-CPO structure $\leq$ as well as a poset structure
  $\ltdyn$. Functional morphisms are given by $\leq$-continuous and
  $\ltdyn$-monotone functions. Then define $\mathcal E$ to be the
  category of pointed domain preorders which are domain preorders with
  least elements $\bot$ for $\leq$ and $\mho$ for $\ltdyn$ such that
  $\bot$ is $\ltdyn$-maximal, and morphisms are as before but preserve
  $\bot$ and $\mho$. This can be extended to a CBPV model with
  forgetful functor $U : \mathcal E \to \mathcal V$. $D$ can be
  defined by solving a domain equation.

  This can be extended to a double CBPV model by defining a value
  relation $A \rel A'$ to be a $\ltdyn$-\emph{embedding}: a morphism
  $e : A \to A'$ that is injective and such that $F e : F A \to F A'$
  has a right adjoint (with respect to $\ltdyn$) and a square $f
  \ltsq{e}{e'} f' = f \circ e \ltdyn e' \circ f$. Similarly
  computation relations $B \rel B'$ are defined to be
  \emph{projections}: morphisms $p : B' \to B$ that are surjective and
  $Up$ has a left adjoint, with squares defined similarly. A suitable
  dynamic type can be constructed by solving a domain equation $D
  \cong \nat + U(D \to FD) + (D \times D)$.
\end{example}

\subsection{Weakening the Double Category Semantics}

While the double categorical semantics can be satisfied with classical
domain theoretic models, there are obstructions to developing a
semantics based on \emph{guarded} recursion. While guarded type theory
makes construction of arbitrary guarded recursive definitions
possible, it comes at a significant cost to reasoning: unfolding of
recursive definitions is explicit. In a denotational semantics, this
means that non-well-founded recursion must perform a kind of
\emph{observable step}. This is a difficulty in proving graduality,
which is a property that is oblivious to the number of steps that a
program takes. Therefore to prove graduality we must impose a kind of
weak bisimilarity relation on our programs to reason about
step-independent relational properties. However, here we arrive at the
fundamental issue with double categorical semantics using
representable relations:

\begin{enumerate}
\item When reasoning up to weak bisimilarity, transitive reasoning is
  not possible, and so horizontal pasting of squares is not a valid
  reasoning principle.
\item When reasoning with observable computation steps, certain casts
  take observable steps, and can no longer be modeled using
  representable morphisms.
\end{enumerate}

Our solution to this dilemma comes in two parts. Since graduality
ignores computational steps, the syntactic theory of graduality must
be modeled up to weak bisimilarity, where transitive reasoning is not
valid. For this purpose we develop a notion of \emph{extensional}
model which weakens from double categories to a reflexive graph
categories, dropping the operation of horizontal pasting of squares,
but still maintaining a form of representability of
relations.

However, transitive reasoning is essential for compositional semantic
constructions, and so we need to work with an intermediate notion of
\emph{intensional} model which is based on double categories, but
where representability has to be weakened to
``quasi-representability'', a kind of representability \emph{up to}
observable steps. To reason up to observable steps without using weak
bisimilarity, we develop a notion we call \emph{perturbations},
certain terms that are bisimilar to the identity but can be
manipulated explicitly in constructions. We then show that the
compositional construction of casts from domain theoretic models can
be adapted to this guarded setting by incorporating some explicit
manipulation of perturbations. Finally, we show that taking an
``extensional collapse'' by bisimilarity provides a model of the
extensional theory, which can then be used to model the global
graduality property.


% \section{Denotational Semantics}

First, we define a denotational semantics of types and terms of the
cast calculus by giving a standard monadic denotational semantics in
the cartesian closed category of preorders and monotone functions,
extended to model the primitives of gradual typing: the dynamic type,
errors and type casts. The most interesting part of this semantics is
the construction of the monad and the dynamic type.



\section{Domain-Theoretic Constructions}\label{sec:domain-theory}

In this section, we discuss the fundamental objects of the model into which we will embed
the intensional lambda calculus and inequational theory. It is important to remember that
the constructions in this section are entirely independent of the syntax described in the
previous section; the notions defined here exist in their own right as purely mathematical
constructs. In the next section, we will link the syntax and semantics via an interpretation
function.

\subsection{The Lift Monad}

When thinking about how to model intensional gradually-typed programs, we must consider
their possible behaviors. On the one hand, we have \emph{failure}: a program may fail
at run-time because of a type error. In addition to this, a program may ``think'',
i.e., take a step of computation. If a program thinks forever, then it never returns a value,
so we can think of the idea of thinking as a way of intensionally modelling \emph{partiality}.

With this in mind, we can describe a semantic object that models these behaviors: a monad
for embedding computations that has cases for failure and ``thinking''.
Previous work has studied such a construct in the setting of the latter, called the lift
monad \cite{mogelberg-paviotti2016}; here, we augment it with the additional effect of failure.

For a type $A$, we define the \emph{lift monad with failure} $\li A$, which we will just call
the \emph{lift monad}, as the following datatype:

\begin{align*}
  \li A &:= \\
  &\eta \colon A \to \li A \\
  &\mho \colon \li A \\
  &\theta \colon \later (\li A) \to \li A
\end{align*}

Unless otherwise mentioned, all constructs involving $\later$ or $\fix$
are understood to be with repsect to a fixed clock $k$. So for the above, we really have for each
clock $k$ a type $\li^k A$ with respect to that clock.

Formally, the lift monad $\li A$ is defined as the solution to the guarded recursive type equation

\[ \li A \cong A + 1 + \later \li A. \]

An element of $\li A$ should be viewed as a computation that can either (1) return a value (via $\eta$),
(2) raise an error and stop (via $\mho$), or (3) think for a step (via $\theta$).
%
Notice there is a computation $\fix \theta$ of type $\li A$. This represents a computation
that thinks forever and never returns a value.

Since we claimed that $\li A$ is a monad, we need to define the monadic operations
and show that they repect the monadic laws. The return is just $\eta$, and extend
is defined via by guarded recursion by cases on the input.
% It is instructive to give at least one example of a use of guarded recursion, so
% we show below how to define extend:
% TODO
%
%
Verifying that the monadic laws hold requires \lob-induction and is straightforward.

The lift monad has the following universal property. Let $f$ be a function from $A$ to $B$,
where $B$ is a $\later$-algebra, i.e., there is $\theta_B \colon \later B \to B$.
Further suppose that $B$ is also an ``error-algebra'', that is, an algebra of the
constant functor $1 \colon \text{Type} \to \text{Type}$ mapping all types to Unit.
This latter statement amounts to saying that there is a map $\text{Unit} \to B$, so $B$ has a
distinguished ``error element" $\mho_B \colon B$.

Then there is a unique homomorphism of algebras $f' \colon \li A \to B$ such that
$f' \circ \eta = f$. The function $f'(l)$ is defined via guarded fixpoint by cases on $l$. 
In the $\mho$ case, we simply return $\mho_B$.
In the $\theta(\tilde{l})$ case, we will return

\[\theta_B (\lambda t . (f'_t \, \tilde{l}_t)). \]

Recalling that $f'$ is a guaded fixpoint, it is available ``later'' and by
applying the tick we get a function we can apply ``now''; for the argument,
we apply the tick to $\tilde{l}$ to get a term of type $\li A$.


%\subsubsection{Model-Theoretic Description}
%We can describe the lift monad in the topos of trees model as follows.


\subsection{Predomains}

The next important construction is that of a \emph{predomain}. A predomain is intended to
model the notion of error ordering that we want terms to have. Thus, we define a predomain $A$
as a partially-ordered set, which consists of a type which we denote $\ty{A}$ and a reflexive,
transitive, and antisymmetric relation $\le_P$ on $A$.

For each type we want to represent, we define a predomain for the corresponding semantic
type. For instance, we define a predomain for natural numbers, a predomain for the
dynamic type, a predomain for functions, and a predomain for the lift monad. We
describe each of these below.

We define monotone functions between predomain as expected. Given predomains
$A$ and $B$, we write $f \colon A \monto B$ to indicate that $f$ is a monotone
function from $A$ to $B$, i.e, for all $a_1 \le_A a_2$, we have $f(a_1) \le_B f(a_2)$.

\begin{itemize}
  \item There is a predomain Nat for natural numbers, where the ordering is equality.
  
  \item There is a predomain Dyn to represent the dynamic type. The underlying type
  for this predomain is defined by guarded fixpoint to be such that
  $\ty{\Dyn} \cong \mathbb{N}\, +\, \later (\ty{\Dyn} \monto \ty{\Dyn})$.
  This definition is valid because the occurrences of Dyn are guarded by the $\later$.
  The ordering is defined via guarded recursion by cases on the argument, using the
  ordering on $\mathbb{N}$ and the ordering on monotone functions described below.

  \item For a predomain $A$, there is a predomain $\li A$ for the ``lift'' of $A$
  using the lift monad. We use the same notation for $\li A$ when $A$ is a type
  and $A$ is a predomain, since the context should make clear which one we are referring to.
  The underling type of $\li A$ is simply $\li \ty{A}$, i.e., the lift of the underlying
  type of $A$.
  The ordering on $\li A$ is the ``step-sensitive error-ordering'' which we describe in
  \ref{subsec:lock-step}.

  \item For predomains $A_i$ and $A_o$, we form the predomain of monotone functions
  from $A_i$ to $A_o$, which we denote by $A_i \To A_o$.
  The ordering is such that $f$ is below $g$ if for all $a$ in $\ty{A_i}$, we have
  $f(a)$ is below $g(a)$ in the ordering for $A_o$. 
\end{itemize}



\subsection{Step-Sensitive Error Ordering}\label{subsec:lock-step}

As mentioned, the ordering on the lift of a predomain $A$ is called the
\emph{step-sensitive error-ordering} (also called ``lock-step error ordering''),
the idea being that two computations $l$ and $l'$ are related if they are in
lock-step with regard to their intensional behavior, up to $l$ erroring.
Formally, we define this ordering as follows:

\begin{itemize}
  \item 	$\eta\, x \ltls \eta\, y$ if $x \le_A y$.
  \item 	$\mho \ltls l$ for all $l$ 
  \item   $\theta\, \tilde{r} \ltls \theta\, \tilde{r'}$ if
          $\later_t (\tilde{r}_t \ltls \tilde{r'}_t)$
\end{itemize}

We also define a heterogeneous version of this ordering between the lifts of two
different predomains $A$ and $B$, parameterized by a relation $R$ between $A$ and $B$.

\subsection{Step-Insensitive Relation}

We define another ordering on $\li A$, called the ``step-insensitive ordering'' or
``weak bisimilarity'', written $l \bisim l'$.
Intuitively, we say $l \bisim l'$ if they are equivalent ``up to delays''.
We introduce the notation $x \sim_A y$ to mean $x \le_A y$ and $y \le_A x$.

The weak bisimilarity relation is defined by guarded fixpoint as follows:

\begin{align*}
  &\mho \bisim \mho \\
%
  &\eta\, x \bisim \eta\, y \text{ if } 
    x \sim_A y \\
%		
  &\theta\, \tilde{x} \bisim \theta\, \tilde{y} \text{ if } 
    \later_t (\tilde{x}_t \bisim \tilde{y}_t) \\
%	
  &\theta\, \tilde{x} \bisim \mho \text{ if } 
    \theta\, \tilde{x} = \delta^n(\mho) \text { for some $n$ } \\
%	
  &\theta\, \tilde{x} \bisim \eta\, y \text{ if }
    (\theta\, \tilde{x} = \delta^n(\eta\, x))
  \text { for some $n$ and $x : \ty{A}$ such that $x \sim_A y$ } \\
%
  &\mho \bisim \theta\, \tilde{y} \text { if } 
    \theta\, \tilde{y} = \delta^n(\mho) \text { for some $n$ } \\
%	
  &\eta\, x \bisim \theta\, \tilde{y} \text { if }
    (\theta\, \tilde{y} = \delta^n (\eta\, y))
  \text { for some $n$ and $y : \ty{A}$ such that $x \sim_A y$ }
\end{align*}

\subsection{Error Domains}


\subsection{Globalization}

Recall that in the above definitions, any occurrences of $\later$ were with
repsect to a fixed clock $k$. Intuitively, this corresponds to a step-indexed set.
It will be necessary to consider the ``globalization'' of these definitions,
i.e., the ``global'' behavior of the type over all potential time steps.
This is accomplished in the type theory by \emph{clock quantification} \cite{atkey-mcbride2013},
whereby given a type $X$ parameterized by a clock $k$, we consider the type
$\forall k. X[k]$. This corresponds to leaving the step-indexed world and passing to
the usual semantics in the category of sets.


\section{Semantics}\label{sec:semantics}


\subsection{Relational Semantics}

\subsubsection{Term Precision via the Step-Sensitive Error Ordering}
% Homogeneous vs heterogeneous term precision

% \subsection{Logical Relations Semantics}


%% \section{Extending the Semantics to Precision}\label{sec:gtlc-precision}

In this section, we extend the set-theoretic semantics for terms given in
the previous section to a semantics for the type and term precision relations
of the gradually-typed lambda calculus. We first introduce the type and term precision
relations, then show how to give them a semantics using SGDT.

% TODO mention intensional syntax


\subsection{Term Precision for GTLC}\label{sec:gtlc-term-precision-axioms}

% ---------------------------------------------------------------------------------------
% ---------------------------------------------------------------------------------------

%\subsubsection{Term Precision}\label{sec:term-precision}

We allow for a \emph{heterogeneous} term precision judgment on values $V$ of type
$A$ and $V'$ of type $A'$ provided that $A \ltdyn A'$ holds. Likewise, for producers,
if $M$ has type $A$ and $M'$ has type $A'$, we can form the judgment that $M \ltdyn M'$.
We use the same notation for the precision relation on both values and producers.

% Type precision contexts
In order to deal with open terms, we will need the notion of a type precision \emph{context}, which we denote
$\gamlt$. This is similar to a normal context but instead of mapping variables to types,
it maps variables $x$ to related types $A \ltdyn A'$, where $x$ has type $A$ in the left-hand term
and $A'$ in the right-hand term. We may also write $x : d$ where $d : A \ltdyn A'$ to indicate this.

% An equivalent way of thinking of type precision contexts is as a pair of ``normal" typing
% contexts $\Gamma, \Gamma'$ with the same domain such that $\Gamma(x) \ltdyn \Gamma'(x)$ for
% each $x$ in the domain.
% We will write $\gamlt : \Gamma \ltdyn \Gamma'$ when we want to emphasize the pair of contexts.
% Conversely, if we are given $\gamlt$, we write $\gamlt_l$ and $\gamlt_r$ for the normal typing contexts on each side.

An equivalent way of thinking of a type precision context $\gamlt$ is as a
pair of ``normal" typing contexts, $\gamlt_l$ and $\gamlt_r$, with the same
domain and such that $\gamlt_l(x) \ltdyn \gamlt_r(x)$ for each $x$ in the domain.
We will write $\gamlt : \gamlt_l \ltdyn \gamlt_r$ when we want to emphasize the pair of contexts.

As with type precision derivations, we write $\Gamma$ to mean the ``reflexivity" type precision context
$\Gamma : \Gamma \ltdyn \Gamma$.
Concretely, this consists of reflexivity type precision derivations $\Gamma(x) \ltdyn \Gamma(x)$ for
each $x$ in the domain of $\Gamma$.

Furthermore, we write $\gamlt_1 \relcomp \gamlt_2$ to denote the ``composition'' of $\gamlt_1$ and $\gamlt_2$
--- that is, the precision context whose value at $x$ is the type precision derivation
$\gamlt_1(x) \relcomp \gamlt_2(x)$. This of course assumes that each of the type precision
derivations is composable, i.e., that the RHS of $\gamlt_1(x)$ is the same as the left-hand side of $\gamlt_2(x)$.

% We define the same for computation type precision contexts $\deltalt_1$ and $\deltalt_2$,
% provided that both the computation type precision contexts have the same ``shape'', which is defined as
% (1) either the stoup is empty in both, or the stoup has a hole in both, say $\hole{d}$ and $\hole{d'}$
% where $d$ and $d'$ are composable, and (2) their value type precision contexts are composable as described above.

The rules for term precision come in two forms. We first have the \emph{congruence} rules,
one for each term constructor. These assert that the term constructors respect term precision.
The congruence rules are as follows:

\begin{mathpar}

  \inferrule*[right = Var]
    { c : A \ltdyn B \and \gamlt(x) = (A, B) } 
    { \etmprec {\gamlt} x x c }

  \inferrule*[right = Zro]
    { } {\etmprec \gamlt \zro \zro \nat }

  \inferrule*[right = Suc]
    { \etmprec \gamlt V {V'} \nat } {\etmprec \gamlt {\suc\, V} {\suc\, V'} \nat}

  \inferrule*[right = MatchNat]
  {\etmprec \gamlt V {V'} \nat \and 
    \etmprec \deltalt M {M'} d \and \etmprec {\deltalt, n : \nat} {N} {N'} d}
  {\etmprec \deltalt {\matchnat {V} {M} {n} {N}} {\matchnat {V'} {M'} {n} {N'}} d}

  \inferrule*[right = Lambda]
    { c_i : A_i \ltdyn A'_i \and 
      c_o : A_o \ltdyn A'_o \and 
      \etmprec {\gamlt, x : c_i} {M} {M'} {c_o} } 
    { \etmprec \gamlt {\lda x M} {\lda x {M'}} {(c_i \ra c_o)} }

  \inferrule*[right = App]
    { c_i : A_i \ltdyn A'_i \and
      c_o : A_o \ltdyn A'_o \\\\
      \etmprec \gamlt {V_f} {V_f'} {(c_i \ra c_o)} \and
      \etmprec \gamlt {V_x} {V_x'} {c_i}
    } 
    { \etmprec {\gamlt} {V_f\, V_x} {V_f'\, V_x'} {{c_o}}}

  \inferrule*[right = Ret]
    {\etmprec {\gamlt} V {V'} c}
    {\etmprec {\gamlt} {\ret\, V} {\ret\, V'} {c}}

  \inferrule*[right = Bind]
    {\etmprec {\gamlt} {M} {M'} {c} \and 
     \etmprec {\gamlt, x : c} {N} {N'} {d} }
    {\etmprec {\gamlt} {\bind {x} {M} {N}} {\bind {x} {M'} {N'}} {d}}
\end{mathpar}

We then have additional equational axioms, including $\beta$ and $\eta$ laws, and
rules characterizing upcasts as least upper bounds, and downcasts as greatest lower bounds.
For the sake of familiarity, we formulate the cast rules using arbitrary casts; later we
will state the analogous versions for the version of the calculus without arbitrary casts.

We write $M \equidyn N$ to mean that both $M \ltdyn N$ and $N \ltdyn M$.

\begin{mathpar}
  \inferrule*[right = $\err$]
    {\phasty {\Gamma} M B }
    {\etmprec {\Gamma} {\err_B} M B}

  \inferrule*[right = $\beta$-fun]
    { \phasty {\Gamma, x : A_i} M {A_o} \and
      \vhasty {\Gamma} V {A_i} } 
    { \etmequidyn {\Gamma} {(\lda x M)\, V} {M[V/x]} {A_o} }

  \inferrule*[right = $\eta$-fun]
    { \vhasty {\Gamma} {V} {A_i \ra A_o} } 
    { \etmequidyn \Gamma {\lda x (V\, x)} V {A_i \ra A_o} }

  \inferrule*[right = UpR]
    { c : A \ltdyn B \and d : B \ltdyn C \and 
      \etmprec {\gamlt} {M} {N} {c} } 
    { \etmprec {\gamlt} {M} {\up {B} {C} N} {c \circ d}  }

  \inferrule*[right = UpL]
    { c : A \ltdyn B \and d : B \ltdyn C \and
      \etmprec {\gamlt} {M} {N} {c \circ d} } 
    { \etmprec {\gamlt} {\up {A} {B} M} {N} {d} }

  \inferrule*[right = DnL]
    { c : A \ltdyn B \and d : B \ltdyn C \and
      \etmprec {\gamlt} {M} {N} {d} } 
    { \etmprec {\gamlt} {\dn {A} {B} M} {N} {c \circ d} }

  \inferrule*[right = DnR]
    { c : A \ltdyn B \and d : B \ltdyn C \and
      \etmprec {\gamlt} {M} {N} {c \circ d} } 
    { \etmprec {\gamlt} {M} {\dn {B} {C} N} {c} }
\end{mathpar}

% TODO explain the least upper bound/greatest lower bound rules
The rules UpR, UpL, DnL, and DnR were introduced in \cite{new-licata18} as a means
of cleanly axiomatizing the intended behavior of casts in a way that
doesn't depend on the specific constructs of the language.
Intuitively, rule UpR says that the upcast of $M$ is an upper bound for $M$
in that $M$ may error more, and UpL says that the upcast is the \emph{least}
such upper bound, in that it errors more than any other upper bound for $M$.
Conversely, DnL says that the downcast of $M$ is a lower bound, and DnR says
that it is the \emph{greatest} lower bound.
% These rules provide a clean axiomatization of the behavior of casts that doesn't
% depend on the specific constructs of the language.



\subsection{Semantics for Precision}

As a first attempt at giving a semantics to the ordering, we could try to model types as
sets equipped with an ordering that models term precision. Since term precision is reflexive
and transitive, and since we identify terms that are equi-precise, we choose to model types
as partially-ordered sets. We model the term precision ordering $M \ltdyn N : A \ltdyn B$ as an
ordering relation between the posets denoted by $A$ and $B$.

However, it turns out that modeling term precision by a relation defined by guarded fixpoint
is not as straightforward as one might hope.
A first attempt might be to define an ordering $\semltbad$ between $\li X$ and $\li Y$
that allows for computations that may take different numbers of steps to be related.
The relation is parameterized by a relation $\le$ between $X$ and $Y$, and is defined
by guarded fixpoint as follows:
% simultaneously captures the notions of error approximation and equivalence up to stepping behavior:

\begin{align*}
  &\eta\, x \semltbad \eta\, y \text{ if } 
    x \semlt y \\
%		
  &\mho \semltbad l \\
%
  &\theta\, \tilde{l} \semltbad \theta\, \tilde{l'} \text{ if } 
    \later_t (\tilde{l}_t \semltbad \tilde{l'}_t) \\
%	
  &\theta\, \tilde{l} \semltbad \mho \text{ if } 
    \theta\, \tilde{l} = \delta^n(\mho) \text { for some $n$ } \\
%	
  &\theta\, \tilde{l} \semltbad \eta\, y \text{ if }
    (\theta\, \tilde{l} = \delta^n(\eta\, x))
  \text { for some $n$ and $x : \ty{X}$ such that $x \le y$ } \\
%
  &\mho \semltbad \theta\, \tilde{l'} \text { if } 
    \theta\, \tilde{l'} = \delta^n(\mho) \text { for some $n$ } \\
%	
  &\eta\, x \semltbad \theta\, \tilde{l'} \text { if }
    (\theta\, \tilde{l'} = \delta^n (\eta\, y))
  \text { for some $n$ and $y : \ty{Y}$ such that $x \le y$ }
\end{align*}

Two computations that immediately return $(\eta)$ are related if the underlying
values are related in the underlying ordering. 
%
The computation that errors $(\mho)$ is below everything else.
%
If both sides step (i.e., both sides are $\theta$),
then we allow one time step to pass and compare the resulting terms.
(This is where use the relation defined ``later''.)
%
Lastly, if one side steps and the other returns a value, the side that steps should
terminate with a value in some finite number of steps $n$, and that value should
be related to the value returned by the other side.
Likewise, if one side steps and the other errors, then the side that steps
should terminate with error.

The problem with this definition is that the resulting relation is \emph{provably} not
transitive: it can be shown (in Clocked Cubical Type Theory) that if $R$ is a
relation on $\li X$ satisfying three specific properties, one of which is
transitivity, then that relation is trivial.
(The other two properties are that the relation is a congruence with respect to $\theta$,
and that the relation is closed under delays $\delta = \theta \circ \nxt$ on either side.)
Since the above relation \emph{does} satisfy the other two properties, we conclude
that it must not be transitive.

%But having a non-transitive relation to model term precision presents a problem
%for...

We are therefore led to wonder whether we can formulate a version of the relation
that \emph{is} transitive.
It turns out that we can, by sacrificing another of the three properties from
the above lemma. Namely, we give up on closure under delays. Doing so, we end up
with a \emph{lock-step} error ordering, where, roughly speaking, in order for
computations to be related, they must have the same stepping behavior.
%
We then formulate a separate relation, \emph{weak bisimilarity}, that relates computations
that are extensionally equal and may only differ in their stepping behavior.

% As a result, we instead separate the semantics of term precision into two relations:
% an intensional, step-sensitive \emph{error ordering} and a \emph{bisimilarity relation}.


\subsubsection{Double Posets}\label{sec:predomains}

As discussed above, there are two relations that we would like to define
in the semantics: a step-sensitive error ordering, and weak bisimilarity of computations.
%
The semantic objects that interpret our types should therefore be equipped with
two relations. We call these objects ``double posets''.
A double poset $A$ is a set with two relations: an partial order $\semlt_A$ on $A$, and
a reflexive, symmetric relation $\bisim_A$ on $A$.
We write the underling set of $A$ as $\ty{A}$.

We define morphisms of double posets as functions that preserve both
the ordering and the bisimilarity relation. Given double posets
$A$ and $B$, we write $f \colon A \monto B$ to indicate that $f$ is a morphism
from $A$ to $B$, i.e, the following hold:
(1) for all $a_1 \semlt_A a_2$, we have $f(a_1) \semlt_{B} f(a_2)$, and
(2) for all $a_1 \bisim_A a_2$, we have $f(a_1) \bisim_{B} f(a_2)$.


%%%%% RESUME HERE

We define an ordering on morphisms of double posets as
$f \le g$ if for all $a$ in $\ty{A_i}$, we have $f(a) \le_{A_o} g(a)$,
and similarly bisimilarity extends to morphisms via
$f \bisim g$ if for all $a$ in $\ty{A_i}$, we have $f(a) \bisim_{A_o} g(a)$.

For each type we want to represent, we define a double poset for the corresponding semantic
type. For instance, we define a double poset for natural numbers, for the
dynamic type, for functions, and for the lift monad. We
describe each of these below.

\begin{itemize}
  \item There is a double poset $\Nat$ for natural numbers, where the ordering and the
  bisimilarity relations are both equality.
  
  % TODO explain that there is a theta operator for posets?
  \item There is a double poset $\Dyn$ to represent the dynamic type. The underlying type
  for this double poset is defined by guarded fixpoint to be such that
  $\ty{\Dyn} \cong \mathbb{N}\, +\, \later (\ty{\Dyn} \monto \li \ty{\Dyn})$.
  This definition is valid because the occurrences of Dyn are guarded by the $\later$.
  The ordering is defined via guarded recursion by cases on the argument, using the
  ordering on $\mathbb{N}$ and the ordering on monotone functions described above.

  \item For a double poset $A$, there is a double poset $\li A$ for the ``lift'' of $A$
  using the lift monad. We use the same notation for $\li A$ when $A$ is a type
  and $A$ is a double poset, since the context should make clear which one we are referring to.
  The underling type of $\li A$ is simply $\li \ty{A}$, i.e., the lift of the underlying
  type of $A$.
  The ordering on $\li A$ is the ``lock-step error-ordering'' which we describe in
  \ref{subsec:lock-step}. The bismilarity relation is the ``weak bisimilarity''
  described in Section \ref{}

  \item For double posets $A_i$ and $A_o$, we form the double poset of monotone functions
  from $A_i$ to $A_o$, which we denote by $A_i \To A_o$.

  \item Given a double poset $A$, we can form the double poset $\later A$ whose underlying
  type is $\later \ty{A}$. We define $\tilde{x} \le_{\later A} \tilde{y}$ to be
  $\later_t (\tilde{x}_t \le_A \tilde{y}_t)$.
\end{itemize}

\subsubsection{Step-Sensitive Error Ordering}\label{subsec:lock-step}

As mentioned, the ordering on the lift of a double poset $A$ is called the
\emph{step-sensitive error-ordering} (also called ``lock-step error ordering''),
the idea being that two computations $l$ and $l'$ are related if they are in
lock-step with regard to their intensional behavior, up to $l$ erroring.
Formally, we define this ordering as follows:

\begin{itemize}
  \item 	$\eta\, x \ltls \eta\, y$ if $x \le_A y$.
  \item 	$\mho \ltls l$ for all $l$ 
  \item   $\theta\, \tilde{r} \ltls \theta\, \tilde{r'}$ if
          $\later_t (\tilde{r}_t \ltls \tilde{r'}_t)$
\end{itemize}

We also define a heterogeneous version of this ordering between the lifts of two
different double posets $A$ and $B$, parameterized by a relation $R$ between $A$ and $B$.

\subsubsection{Weak Bisimilarity Relation}

For a double poset $A$, we define a relation on $\li A$, called ``weak bisimilarity",
written $l \bisim l'$. Intuitively, we say $l \bisim l'$ if they are equivalent
``up to delays''.
% We introduce the notation $x \sim_A y$ to mean $x \le_A y$ and $y \le_A x$.
% TODO if A is a poset, then we can just say that x = y
%
The weak bisimilarity relation is defined by guarded fixpoint as follows:

\begin{align*}
  &\mho \bisim \mho \\
%
  &\eta\, x \bisim \eta\, y \text{ if } 
    x \bisim_A y \\
%		
  &\theta\, \tilde{x} \bisim \theta\, \tilde{y} \text{ if } 
    \later_t (\tilde{x}_t \bisim \tilde{y}_t) \\
%	
  &\theta\, \tilde{x} \bisim \mho \text{ if } 
    \theta\, \tilde{x} = \delta^n(\mho) \text { for some $n$ } \\
%	
  &\theta\, \tilde{x} \bisim \eta\, y \text{ if }
    (\theta\, \tilde{x} = \delta^n(\eta\, x))
  \text { for some $n$ and $x : \ty{A}$ such that $x \sim_A y$ } \\
%
  &\mho \bisim \theta\, \tilde{y} \text { if } 
    \theta\, \tilde{y} = \delta^n(\mho) \text { for some $n$ } \\
%	
  &\eta\, x \bisim \theta\, \tilde{y} \text { if }
    (\theta\, \tilde{y} = \delta^n (\eta\, y))
  \text { for some $n$ and $y : \ty{A}$ such that $x \sim_A y$ }
\end{align*}

When both sides are $\eta$, then we ensure that the underlying values are bisimilar
in the underlying bisimilarity relation on $A$.
When one side is a $\theta$ and the other is $\eta x$ (i.e., one side steps),
we stipulate that the $\theta$-term runs to $\eta y$ where $x$ is related to $y$.
Similarly when one side is $\theta$ and the other $\mho$.
If both sides step, then we allow one time step to pass and compare the resulting terms.
In this way, the definition captures the intuition of terms being equivalent up to
delays.

It can be shown (by \lob-induction) that the step-sensitive relation is symmetric.
However, it can also be shown that this relation is \emph{not} transitive:
The argument is the same as that used to show that the step-insensitive error
ordering $\semltbad$ described above is not transitive. Namely, we show that
if it were transitive, then it would have to be trivial in that $l \bisim l'$ for all $l, l'$.
that if this relation were transitive, then in fact it would be trivial in that
%This issue will be resolved when we consider the relation's \emph{globalization}.

\subsection{The Cast Rules}

Unfortunately, the four cast rules defined above do not hold in
the intensional setting where we are tracking the steps taken by terms.
The source of the problem is that the downcast from the dynamic type to
a function involves a delay, i.e., a $\theta$.
So in order to keep the other term in lock-step, we need to insert a ``delay"
that is extensionally equivalent to the identity function.
More concretely, consider a simplified version of the DnL rule shown below:

\begin{mathpar}
  \inferrule*{M \ltdyn_i N : B}
             {\dnc{c}{M} \ltdyn_i N : c}
\end{mathpar}

If $c$ is inj-arr, then when we downcast $M$ from $dyn$ to $\dyntodyn$,
semantically this will involve a $\theta$ because the value of type $dyn$
in the semantics will contain a \emph{later} function $\tilde{f}$.
Thus, in order for the right-hand side to be related to the downcast,
we need to insert a delay on the right.
%
The need for delays affects the cast rules involving upcasts as well, because
the upcast for functions involves a downcast on the domain:

\[ \up{A_i \ra A_o}{B_i \ra B_o}{M} \equiv \lambda (x : B_i). \up{A_o}{B_o}(M\, (\dn {A_i}{B_i} x)). \]

Thus, the correct versions of the cast rules involve delays on the side that was not casted.


% Delays for function types and for inj-arr(c)


\subsubsection{Perturbations}

We can describe precisely how the delays are inserted for any type precision
derivation $c$.

To do so, we first define simultaneously an inductive type of \emph{perturbations}
for embeddings $\perte$ and for projections $\pertp$ by the following rules:

\begin{mathpar}

\inferrule{}{\id : \perte A}

\inferrule{}{\id : \pertp A}

\inferrule
  {\delta_c : \pertp A \and \delta_d : \perte B}
  {\delta_c \ra \delta_d : \perte (A \ra B)}

\inferrule
  {\delta_c : \perte A \and \delta_d : \pertp B}
  {\delta_c \ra \delta_d : \pertp (A \ra B)}

\inferrule
  {\delta_\nat : \perte \nat \and \delta_f : \perte (\dyntodyn)}
  {\pertdyn{\delta_\nat}{\delta_f} : \perte \dyn}

\inferrule
  {\delta_\nat : \pertp \nat \and \delta_f : \pertp (\dyntodyn)}
  {\pertdyn{\delta_\nat}{\delta_f} : \pertp \dyn}

\end{mathpar}

The structure of embedding perturbations is designed to follow the structure
of the corresponding embeddings, and likewise for the projection perturbations.
Thus, in the function case, an embedding perturbation consists of a \emph{projection}
perturbation for the domain and an \emph{embedding} perturbation for the codomain.
The opposite holds for the projection perturbation for functions.

Another way in which the two kinds of perturbations differ is that there is an additional
projection perturbation for delaying $\delaypert{\delta}$.
This corresponds to the actual delay term $\delta = \theta \circ \nxt$ in the semantics,
and it is the generator/source of all non-trivial perturbations.

Given a perturbation $\delta$, we can turn it into a term, which we also write as
$\delta$ unless there is opportunity for confusion.



%% \section{Graduality}\label{sec:graduality}
Now we move to the proof of graduality.
The main theorem we would like to prove is the following:

% TODO use the etmprec notation
\begin{theorem}[Graduality]
  If $\cdot \vdash M \ltdyn N : \nat$, then
  \begin{enumerate}
    \item $M \Downarrow$ iff $N \Downarrow$
    \item If $M \Downarrow v_?$ and $N \Downarrow v'_?$ then either $v_? = \mho$, or $v_? = v'_?$.
  \end{enumerate}
\end{theorem}

\subsection{Proof Overview}

We split the proof into three main steps. The first concerns the translation
of the surface syntax $\extlc$ to the intensional syntax $\intlc$.

\begin{lemma}[]
  If $\etmprec{\gamlt}{M}{N}{c}$, then there exist $M'$ and $N'$ such that

  \[ i(M) \synbisim M' \ltdyn_i N' \synbisim i(N). \]
\end{lemma}

Next we show that the notions of intensional syntactic term precision and
syntactic bisimilarity imply their corresponding semantic counterparts:

\begin{lemma}[]
  If $\itmprec{\gamlt}{M}{N}{c}$, then $\sem{M} \semlt {\sem{N}}$
\end{lemma}

\begin{lemma}
  If $M \synbisim N$, then $\sem{M} \bisim \sem{N}$.
\end{lemma}

Finally, we need an adequacy result stating that the semantic notions of error
ordering and bisimilarity for natural numbers imply that the values of the
lift monad that they relate have the expected behavior.

\begin{lemma}
  Let $l, l' : \li \Nat$.
  If $l \semlt l'$, then either (1) $l = \delta^n(\mho)$ for some $n$, or (2) 
  $l = l' = \delta^n(\eta m)$ for some $n$ and $m$, or (3) $l$ and $l'$ fail to terminate.
\end{lemma}

\begin{lemma}
  If $l \bisim l'$, then either (1) $l = \delta^n(\mho)$ and $l' = \delta^{n'}(\mho)$ for some $n$ and $n'$,
  (2) $l = \delta^n(\eta m)$ and $l' = \delta^{n'}(\eta m)$, or $l$ and $l'$ fail to terminate.
\end{lemma}

\section{Revised Categorical Models of Graduality}\label{sec:abstract-models}

Next, we develop our appropriate weakened notions of models of
graduality, which we divide into \emph{extensional} models, where
ordering is up to weak bisimilarity, and \emph{intensional} models,
where ordering relates terms that considers computational steps to be
observable. We develop the notion of intensional model of gradual
typing in stages and show how to develop one from a base model of
effectful functions and relations.

\subsection{Extensional Models of Gradual Typing}

Since we lack transitivity of ordering when reasoning in guarded type
theory, our weakened notion of extensional model is based on reflexive
graph categories rather than double categories. This means we lose the
reasoning principle of horizontal pasting of squares. We will still
require a notion of \emph{composition} of relations, to model the
transitivity of type precision. We note that without horizontal
pasting of squares, the notion of left/right representability of
squares is not sufficient to interpret the cast rules of gradual
typing. Instead we generalize the notion of representability to match
the syntactic rules in Section~\ref{sec:GTLC}.

Let $c : A \rel A'$ and $f : A \to A'$ in a reflexive graph category
with composition of relations. We say that $c$ is \emph{universally
left-representable by} $f$ if for any $c_l : A_l \rel A$ and $c_r : A'
\rel A_r$ we have $f \ltsq{cc_r}{c_r} \id$ and $\id \ltsq{c_l}{c_lc}
f$. Dually, let $d : B \rel B'$ and $g : B' \to B$.  We say that $d$
is \emph{universally right-representable by} $g$ if for any $d_l : B_l
\rel B$ and $d_r : B' \rel B_r$ we have $\id \ltsq{d_ld}{d_l} \phi$
and $\phi \ltsq{d_r}{dd_r}\id$. In a reflexive graph category these
are stronger than the definitions of left/right representability since
we can pick $c_l,c_r,d_l,d_r$ to be the reflexive relations. In a
double category these are equivalent, but the equivalence uses
horizontal pasting.

Additionally, while in the presence of horizontal pasting
compositionality of representable morphisms is automatic, without this
principle we must require it explicitly. Finally, we explicitly add in
a requirement that the type constructors are functorial in the 

In summary, an extensional model consists of:
\begin{enumerate}
  \item A locally thin reflexive graph internal to CBPV models.
  \item Composition of value and computation relations that form a category with the reflexive relations as identity. Call these categories $\mathcal V_r,\mathcal E_r$
  \item Identity-on-objects functors $\upf : \mathcal V_r \to \mathcal V_f$ and $\dnf : \mathcal E_r^{op} \to \mathcal E_f$ such that every $\upf c$ universally left-represents $c$ and every $\dnf d$ universally represents $d$.
  \item The CBPV connectives $U,F,\times,\to$ are all \emph{covariant} functorial on relations up to equivalence: $U(dd') \equidyn U(d)U(d')$ etc.\footnote{the reflexive graph structure already requires that these functors preserve identity relations}
    %% \begin{itemize}
    %% \item $U(dd') \equidyn U(d)U(d')$
    %% \item $F(cc') \equidyn F(c)F(c')$
    %% \item $(cc') \to (dd') \equidyn (c \to d)(c' \to d')$
    %% \item $(c_1c_1') \times (c_2c_2') \equidyn (c_1 \times c_2)(c_1'\times c_2')$
    %% \end{itemize}
    where $c \equidyn c'$ means $\id \ltsq{c}{c'}\id$ and $\id \ltsq{c'}{c} \id$.
  \item A natural transformation $\mho : 1 \Rightarrow U$ such that
    $\mho \circ ! \ltsq{r(A)}{r(UB)} f$ for any $f : A \to UB$
  \item Distinguished value type $\nat$ with morphisms $z : \mathcal
    V(1,\nat)$ and $s : \mathcal V(\nat,\nat)$.
  \item Distinguished value types $D$ with distinguished relations
    $\iarr{}: U(D \to F D) \rel D$ and $\inat : \nat \rel D$ and $\itimes : D \times D \rel D$
    each satisfying the retraction property $\dnc {\injarr{}}F(\upc{\injarr{}}) \equidyn \id$.
\end{enumerate}

This mathematical structure is sufficient to interpret the theory of
gradual typing in Section~\ref{sec:GTLC}.
\begin{definition}
  Given any extensional gradual typing model, we interpret
  \begin{enumerate}
  \item Each type $A$ as a value type, interpreting the base types and
    $\times$ as their semantic analogues and $\sem {A \ra A'}$ as $U(\sem{A} \to
    F(\sem{A'}))$.
  \item Type precision derivations $c : A \ltdyn A'$ are
    interpreted as relation morphisms $\sem{c} : \sem{A} \rel \sem{A'}$ in the obvious
    way. Every equivalence axiom $c \equiv c'$ implies that $\sem{c} \equidyn \sem{c'}$.
  \item Every term $\Gamma \vdash M : A$ is interpreted as a morphism
    $\sem{M} : \mathcal V_f (\times\sem{\Gamma},UF\sem{A})$. Upcasts are interpreted as
    $UF(u_{\sem{c}})$ and downcasts as $Ud_{F\sem{c}}$.
  \item If $\Delta \vdash M \ltdyn M' : c$ then $\sem{M}
    \ltsq{\times\sem{\Delta}}{\sem{c}} \sem{M'}$ holds.
  \end{enumerate}
\end{definition}

%% %% A model $\mathcal{M}$ of extensional gradual typing consists of the
%% %% following:
%% \begin{itemize}
%% \item A reflexive graph internal to the category of CBPV models and
%%   strict morphisms that is locally thin: there is at most one morphism
%%   in $\mathcal V_{sq}$ for any given domain, codomain, source and
%%   target, and as well for $\mathcal E$.
%% \item 
%% \item A natural transformation $\mho : 1 \Rightarrow U$ satisfying $\mho \circ ! \ltsq{r(A)}{r(B)} f$ for every $f$.
%% \item Such that all relation value morphisms are \emph{universally}
%%   left representable and all relation computation morphisms are
%%   universally right representable.
%% \end{itemize}

%% satisfying certain additional conditions that will be described below.

%% % % https://q.uiver.app/#q=WzAsMixbMCwxLCJcXG1hdGhjYWx7TX1fe3NxfSJdLFswLDAsIlxcbWF0aGNhbHtNfV9mIl0sWzEsMCwiciJdLFswLDEsInMiLDAseyJjdXJ2ZSI6LTJ9XSxbMCwxLCJ0IiwyLHsiY3VydmUiOjJ9XV0=
%% % \[\begin{tikzcd}[ampersand replacement=\&]
%% % 	{\mathcal{M}_f} \\
%% % 	{\mathcal{M}_{sq}}
%% % 	\arrow["r", from=1-1, to=2-1]
%% % 	\arrow["s", curve={height=-12pt}, from=2-1, to=1-1]
%% % 	\arrow["t"', curve={height=12pt}, from=2-1, to=1-1]
%% % \end{tikzcd}\]

%% Spelling this out in light of the above definitions, we see that this is
%% equivalent to the following in the category $\textbf{Cat}$:

%% % https://q.uiver.app/#q=WzAsNCxbMCwyLCJcXHZzcSJdLFsyLDIsIlxcZXNxIl0sWzAsMCwiXFx2ZiJdLFsyLDAsIlxcZWYiXSxbMiwzLCJcXEZmIiwwLHsiY3VydmUiOi0yfV0sWzMsMiwiXFxVZiIsMCx7ImN1cnZlIjotMn1dLFswLDEsIlxcRnNxIiwwLHsiY3VydmUiOi0yfV0sWzEsMCwiXFxVc3EiLDAseyJjdXJ2ZSI6LTJ9XSxbMiwwLCJcXHJ2Il0sWzAsMiwiXFxzdiIsMCx7ImN1cnZlIjotMn1dLFsyLDAsIlxcdHYiLDAseyJjdXJ2ZSI6LTJ9XSxbMSwzLCJcXHNlIiwwLHsiY3VydmUiOi0yfV0sWzMsMSwiXFx0ZSIsMCx7ImN1cnZlIjotMn1dLFszLDEsIlxccmUiXSxbNCw1LCJcXGJvdCIsMSx7InNob3J0ZW4iOnsic291cmNlIjoyMCwidGFyZ2V0IjoyMH0sInN0eWxlIjp7ImJvZHkiOnsibmFtZSI6Im5vbmUifSwiaGVhZCI6eyJuYW1lIjoibm9uZSJ9fX1dLFs2LDcsIlxcYm90IiwxLHsic2hvcnRlbiI6eyJzb3VyY2UiOjIwLCJ0YXJnZXQiOjIwfSwic3R5bGUiOnsiYm9keSI6eyJuYW1lIjoibm9uZSJ9LCJoZWFkIjp7Im5hbWUiOiJub25lIn19fV1d
%% \[\begin{tikzcd}[ampersand replacement=\&]
%% 	\vf \&\& \ef \\
%% 	\\
%% 	\vsq \&\& \esq
%% 	\arrow[""{name=0, anchor=center, inner sep=0}, "\Ff", curve={height=-12pt}, from=1-1, to=1-3]
%% 	\arrow[""{name=1, anchor=center, inner sep=0}, "\Uf", curve={height=-12pt}, from=1-3, to=1-1]
%% 	\arrow[""{name=2, anchor=center, inner sep=0}, "\Fsq", curve={height=-12pt}, from=3-1, to=3-3]
%% 	\arrow[""{name=3, anchor=center, inner sep=0}, "\Usq", curve={height=-12pt}, from=3-3, to=3-1]
%% 	\arrow["\rv", from=1-1, to=3-1]
%% 	\arrow["\sv", curve={height=-12pt}, from=3-1, to=1-1]
%% 	\arrow["\tv", curve={height=-12pt}, from=1-1, to=3-1]
%% 	\arrow["\se", curve={height=-12pt}, from=3-3, to=1-3]
%% 	\arrow["\te", curve={height=-12pt}, from=1-3, to=3-3]
%% 	\arrow["\re", from=1-3, to=3-3]
%% 	\arrow["\bot"{description}, draw=none, from=0, to=1]
%% 	\arrow["\bot"{description}, draw=none, from=2, to=3]
%% \end{tikzcd}\]


% The above definition can be interpreted in any compact closed equipment
% (if someone were to figure out a definition for a compact closed
% equipment, that is,\ldots). Then we can get a model of a form of GTT
% by taking a CBPV object in the equipment of \emph{reflexive graph
% categories}. Since reflexive graphs form a topos we can get at this by
% interpreting the above definition \emph{internally} to the topos of
% reflexive graphs. Essentially what this means is that everything above
% has a ``vertex'' component and an ``edge'' component, so we get a
% cartesian category $\mathcal V_f$ which we think of as the value types
% and pure functions but we also get a cartesian category $\mathcal V_{sq}$
% which we think of as the ``value edges'' and ``squares''.

%% That is, for the values, we have

%% \begin{enumerate}
%%   \item A cartesian category $\mathcal V_f$.
%%   The objects of $\mathcal V_f$ will be called \emph{value types}.
%%   The morphisms of $\mathcal V_f$ will be called \emph{(pure) functions}.
  
%%   \item A cartesian category $\mathcal V_{sq}$. 
%%   The objects of $\mathcal V_{sq}$ will be called \emph{value edges} or
%%   \emph{value relations}, and the morphisms are \emph{commuting squares}.

%%   \item Functors $\sv, \tv : \mathcal V_{sq} \to \mathcal V_f$ and 
%%   $\rv : \mathcal V_f \to \mathcal V_{sq}$.
%% \end{enumerate}

%% Likewise, we have the analogous definitions for computations.

%% We write $c : A \rel A'$ to mean that $c \in \ob(\vsq)$ such that 
%% $\sv(c) = A$ and $\tv(c) = A'$.
%% %
%% Likewise, let $c_i : A_i \rel A_i'$ and $c_o : A_o \rel A_o'$,
%% and let $f \in \vf(A_i, A_o)$ and $f' \in \vf(A_i', A_o')$.
%% The notation $\beta : f \ltdyn_{c_o}^{c_i} f'$ is defined to mean

%% \begin{enumerate}
%%   \item $\beta \in \vsq(c_i, c_o)$
%%   \item $\sv(\beta) = f$
%%   \item $\tv(\beta) = f'$
%% \end{enumerate}

%% (Recall that $\sv$ and $\tv$ are functors, so in addition to acting on
%% the objects of $\vsq$ they also act on morphisms.)

%% Picorially, this is depicted as a commuting square:

%% % https://q.uiver.app/#q=WzAsNCxbMCwwLCJBX2kiXSxbMSwwLCJBX2knIl0sWzAsMSwiQV9vIl0sWzEsMSwiQV9vJyJdLFswLDIsImYiLDJdLFsxLDMsImYnIl0sWzAsMSwiY19pIiwwLHsic3R5bGUiOnsiYm9keSI6eyJuYW1lIjoiYmFycmVkIn0sImhlYWQiOnsibmFtZSI6Im5vbmUifX19XSxbMiwzLCJjX28iLDIseyJzdHlsZSI6eyJib2R5Ijp7Im5hbWUiOiJiYXJyZWQifSwiaGVhZCI6eyJuYW1lIjoibm9uZSJ9fX1dLFs0LDUsIlxcYWxwaGEiLDEseyJzaG9ydGVuIjp7InNvdXJjZSI6MjAsInRhcmdldCI6MjB9LCJzdHlsZSI6eyJib2R5Ijp7Im5hbWUiOiJub25lIn0sImhlYWQiOnsibmFtZSI6Im5vbmUifX19XV0=
%% \[\begin{tikzcd}[ampersand replacement=\&]
%% 	{A_i} \& {A_i'} \\
%% 	{A_o} \& {A_o'}
%% 	\arrow[""{name=0, anchor=center, inner sep=0}, "f"', from=1-1, to=2-1]
%% 	\arrow[""{name=1, anchor=center, inner sep=0}, "{f'}", from=1-2, to=2-2]
%% 	\arrow["{c_i}", "\shortmid"{marking}, no head, from=1-1, to=1-2]
%% 	\arrow["{c_o}"', "\shortmid"{marking}, no head, from=2-1, to=2-2]
%% 	\arrow["\alpha"{description}, draw=none, from=0, to=1]
%% \end{tikzcd}\]

%% % Note: When the identity of the square $\beta$ is not important, we may omit it
%% % and write $f \ltdyn_{c_o}^{c_i} f'$. In this case the meaning is that there exists
%% % a square $\beta : f \ltdyn_{c_o}^{c_i} f'$.

%% Composition of squares $\beta : f \ltdyn_{c_2}^{c_1} g$ and $\beta' : f' \ltdyn_{c_3}^{c_2} g'$
%% corresponds to ``stacking'' the square for $\beta'$ below the square for $\beta$.
%% Fuctoriality of $s$ and $t$ ensure that the left and right sides of the resulting square are as expected,
%% i.e., we get $\beta' \circ \beta : f' \circ f \ltdyn_{c_3}^{c_1} g' \circ g$.

%% % Fuctoriality of $s$ and $t$ ensure that we can ``vertically" compose 
%% % $\beta : f \ltdyn_{c_2}^{c_1} g$ and $\beta' : f' \ltdyn_{c_3}^{c_2} g'$
%% % to obtain $\beta' \circ \beta : f' \circ f \ltdyn_{c_3}^{c_1} g' \circ g$.
%% % Pictorially, this is represented by ``stacking'' the square for
%% % $\beta'$ below the square for $\beta$.

%% All of the above holds in an analogous manner for the computations.


%% We will work in ``locally thin'' models where there is at most one
%% square with a given boundary.
%% That is, if $\beta, \beta' : f \ltdyn_{c_o}^{c_i} f'$, then $\beta = \beta'$.
%% Thus, we may unambiguously omit the identity of the square, i.e., we may write
%% $f \ltdyn_{c_o}^{c_i} f'$.

%% We also have a ``horizontal" composition operation on value edges and on computation edges.
%% That is, let 
%% %
%% \[ X = \{ (c, c') \in \ob(\vsq) \times \ob(\vsq) \mid \tv(c) = \sv(c') \}. \]
%% %
%% There is an operation $\comp : X \to \ob(\vsq)$ such that $\sv(c \comp c') = \sv(c)$
%% and $\tv(c \comp c') = \tv(c')$. Likewise for computations.
%% %
%% Importantly, we emphasize that this composition is \emph{not} a functor: we only
%% require that it act on \emph{objects} of $\vsq$ (i.e. edges) and not on the morphisms
%% (i.e. the squares). Intuivitely, this has to do with the fact that in the extensional
%% setting, the semantic term precision function is \emph{not} transitive.

%% With this definition, is easily shown that there is a category $\ve$ of value relations,
%% whose objects are the objects of $\vf$, and such that $\ve(A, A')$ is the set of objects
%% $c$ of $\vsq$ whose source is $A$ and whose target is $A'$.
%% Composition of morphisms is defined using the above operation $\comp$.
%% Similarly, we have a category $\ee$ of computation relations.


%% We also require that the category $\ve$ of relations is thin ``up to an
%% identity square'', i.e., for any $c, c' \in \ve(A, A')$ we have that the
%% following square commutes:

%% \[\begin{tikzcd}[ampersand replacement=\&]
%%   A \& {A'} \\
%%   A \& {A'}
%%   \arrow[from=1-1, to=2-1, Rightarrow, no head]
%%   \arrow[from=1-2, to=2-2, Rightarrow, no head]
%%   \arrow["c", "\shortmid"{marking}, no head, from=1-1, to=1-2]
%%   \arrow["c'"', "\shortmid"{marking}, no head, from=2-1, to=2-2]
%% \end{tikzcd}\]

% In addition to the ordinary universal properties above, when working
% with reflexive graph models we also have access to new notions of
% universal property that relate the ``function'' morphisms to the
% ``edges''.

%% Then we formulate the relationship between relation morphisms and
%% function morphisms as follows:
%% \begin{enumerate}
%% \item There is an identity-on-objects functor $\upf : \mathcal V_e \to
%%   \mathcal V_f$ such that every $c$ is left-representable by $\upf(c)$.
%% \item There is an identity-on-objects functor $\dnf : \mathcal
%%   E_e^{op} \to \mathcal \mathcal E_f$ such that every $d$ is
%%   right-representable by $\dnf(d)$.
%% \end{enumerate}

%% \textbf{TODO: do we still need this?}
%% We also want something like
%% \[ F_c : \mathcal V_u^{op} \to \mathcal E_d \]
%% \[ U_c : \mathcal E_d^{op} \to \mathcal V_u \]
%% which ensures that if $R$ is a value edge equivalent to $A(u,-)$ then
%% \[ F(R) = F(A(u,-)) = (F A)(-,F u) \]

%%%%%%%%%%%%%%%%%%%%%%%%%%%%%%%%%%%%%%%%%%%%%%%%%%%%%%%%%%%%%%%%%%%%%%%%%%%%%%%%%%%%
%%%%%%%%%%%%%%%%%%%%%%%%%%%%%%%%%%%%%%%%%%%%%%%%%%%%%%%%%%%%%%%%%%%%%%%%%%%%%%%%%%%%

\subsection{Intensional Models}\label{sec:abstract-intensional-models}

An intensional model of gradual typing is defined similarly to an extensional model,
with a few key differences that will be discussed below.
%
The starting point is similar to that of the extensional model: an intensional model
will be given by a diagram in the category of CBPV objects, satisfying
additional properties.
%
This time, however, since we are working intensionally, the semantic denotation of
term precision \emph{is} transitive, so we \emph{do} have a horizontal composition
operation on squares. Compare this to the extensional case, where we could only
compose \emph{relations} horizontally, not squares.
What this means is that we can define a functor for composition of value relations
and squares, and a functor for composition of computation relations and squares.

We can specify this compactly as a category internal to the category of CBPV models
and lax morphisms, where we require that the reflexivity, source, and target morphisms
are strict.
In particular, as in the extensional case, there is a CBPV model of ``objects''
$\mathcal M_f$ and a CBPV model of ``arrows'' $\mathcal M_{sq}$. 
There are CBPV morphisms
$r : \mathcal M_f \to \mathcal M_{sq}$ and $s, t : \mathcal M_{sq} \to \mathcal M_f$,
just as before.
%
But now, we also have a CBPV morphism $m$ from the pullback 
$\mathcal M_{sq} \times_{s = t} \mathcal M_{sq}$ to $M_{sq}$, i.e., ``composition of arrows".
In particular, this consists of a functor 
$m_{\mathcal{V}} : \vsq \times_{\sv = \tv} \vsq \to \vsq$ for composition of value
relations/squares, and a functor $m_{\mathcal{E}} : \esq \times_{\se = \te} \esq \to \esq$
for composition of computation relations/squares.
Furthermore, $s \circ m = s \circ \pi_1$ and $t \circ m = t \circ \pi_2$.

As in the extensional model, we also require the existence of a natural transformation
$\mho : 1 \Rightarrow U$ such that $\mho \circ ! \ltsq{r(A)}{r(UB)} f$ for any $f : A \to UB$,
and a distinguished value type $\nat$ with morphisms $z : \mathcal V(1,\nat)$ and $s : \mathcal V(\nat,\nat)$.
% Lastly, we require the existence of a morphism $\mho_B \in \vf(\Gamma, UB)$ for all $B$
% satisfying the same ordering and commutativity properties required in the extensional model.

% If we spell this all out explicitly, we end up with a definition similar to the
% one for the extensional case, but now with the addition of a functor $m_{\mathcal{V}}$ for
% composition of value relations/squares and a functor $m_{\mathcal{E}}$ for
% composition of computation relations/squares.

For the sake of ease of reference, we recap the definition of a step-0 model:
%
\begin{definition}
  A \emph{step-0} model of intensional gradual typing consists of:
  \begin{itemize}
    \item A category internal to the category of CBPV models and lax morphisms,
    where we require that the morphisms $r$, $s$, and $t$ are strict.
    \item A natural transformation $\mho : 1 \Rightarrow U$ such that $\mho \circ ! \ltsq{r(A)}{r(UB)} f$ for any $f : A \to UB$.
    \item A value type $\nat$ with morphisms $z : \mathcal V(1,\nat)$ and $s : \mathcal V(\nat,\nat)$.
  \end{itemize}

\end{definition}

% In particular, as before, we have cartesian categories $\mathcal V_f$,
% $\mathcal V_e$, and $\mathcal V_{sq}$, in addition to $\mathcal E_f$,
% $\mathcal E_e$, and $\mathcal E_{sq}$.
% But we now have horizontal composition of squares as well.

%%%%%%%%%%%%%%%%%%%%%%%%%%%%%%%%%%%%%%%%%%%%%%%%%%%%%%%%%%%%%%%%%%%%%%%%%%%%%%%%%%%%

\subsubsection{Bisimilarity}\label{sec:abstract-model-bisimilarity}

Working intensionally means we need to take into consideration the steps
taken by terms. One consequence of this is that we need a way to specify
that two morphisms are the same ``up to delay'', i.e., they differ only in that
one may wait more than the other.

In particular, for any pair of objects $A$ and $A'$, in $\vf$,
we require that there is a reflexive, symmetric relation $\bisim_{A,A'}$ on the
hom-set $\vf(A, A')$, called the \emph{weak bisimilarity} relation.
Similarly for the computation category: there is a reflexive, symmetric relation
$\bisim_{B,B'}$ defined on each hom-set $\ef(B, B')$.
%
Additionally, the weak bisimilarity relation should respect composition:
if $f \bisim_{A,A'} f'$ and $g \bisim_{A',A''} g'$, then
$g \circ f \bisim_{A,A''} g' \circ f'$, and likewise for computations.

We can specify all of this abstractly via categories $\vsim$ and $\esim$ along with
functors $\rvsim : \vf \to \vsim$ and $\svsim, \tvsim : \vsim \to \vf$,
and likewise for computations.
Since bisimilarity of morphisms $f$ and $f'$ requires that they share source and target,
we require that $\svsim$ and $\tvsim$ agree on objects and likewise for $\sesim$ and $\tesim$.
Thus, the objects of $\vsim$ are identified with $\ob(\vf)$.
The morphisms of $\vsim$ are ``bisimilarity proofs'', analogous to the commuting squares of $\vsq$.

There is also a ``symmetry'' endofunctor $\text{sym}_{\mathcal{V}}^\bisim : \vsim \to \vsim$
such that $\svsim \circ \text{sym}_{\mathcal{V}}^\bisim = \tvsim$
and $\tvsim \circ \text{sym}_{\mathcal{V}}^\bisim = \svsim$,
and $\text{sym}_{\mathcal{V}}^\bisim \circ \text{sym}_{\mathcal{V}}^\bisim$ is the identity.
Likewise there is a symmetry endofunctor $\text{sym}_{\mathcal{E}}^\bisim : \esim \to \esim$.

In this setting, we write $\refl_A : A \bisim A$ to mean that $\refl_A \in \ob(\vsim)$,
such that $\svsim(\refl_A) = A = \tvsim(\refl_A)$.
Let $f, f' \in \vf(A_i, A_o)$.
The judgment $\gamma : f \bisim_{A_i, A_o} f'$ is defined to mean:

\begin{enumerate}
  \item $\gamma \in \vsim(\refl_{A_i}, \refl_{A_o})$
  \item $\svsim(\gamma) = f$
  \item $\tvsim(\gamma) = f'$
\end{enumerate}

% Spelling this all out concretely, for any pair...


% Lastly, we require that for any value object $A$, the hom-set $\ef(FA, FA)$ contains a
% distinguished morphism $\delta_{FA}^*$, such that $\delta_A^* \bisim_{FA, FA} \id_{FA}$.
% Moreover, we require that these morphisms are related in that for any
% $c : A \rel A'$, we have a square $\delta_{FA}^* \ltdyn_{Fc}^{Fc} \delta_{FA'}^*$.

Lastly, we require that for any computation object $B$, the hom-set $\vf(UB, UB)$ contains a
distinguished morphism $\delta_{UB}^*$, such that $\delta_{UB}^* \bisim_{UB, UB} \id_{UB}$.
Moreover, we require that these morphisms are related in that for any
$d : B \rel B'$, we have a square $\delta_{UB}^* \ltdyn_{Ud}^{Ud} \delta_{UB'}^*$.
We also require that these morphisms commute with computation morphisms, in the sense
that for any $\phi \in \ef(B, B')$ we have $U\phi \circ \delta_{UB_1}^* = \delta_{UB_2} \circ U\phi$.
%
Given the existence of the morphisms $\delta_{UB}^*$, we can define a computation morphism
$\delta_{FA}^* \in \ef(FA, FA)$ for all $A$ by composing the unit $\eta_A \in \vf(A, UFA)$ with
the morphism $\delta_{UFA}^* \in \vf(UFA, UFA)$, and then by the adjunction
we get a computation morphism $\delta_{FA}^ \in \ef(FA, FA)$. Moreover, we get a square
$\delta_{FA}^* \ltdyn_{Fc}^{Fc} \delta_{FA'}^*$ for all $c : A \rel A$.


\begin{definition}\label{def:step-1-model}
A \emph{step-1 intensional model} consists of all the data of a step-0 intensional model along
with:
\begin{itemize}
  \item The categories and functors for bisimiarity described above.
  \item The existence of a distinguished value morphism $\delta_{UB}^* \bisim \id_{UB}$ for each $B$.
  \item A square $\delta_{UB}^* \ltdyn_{Ud}^{Ud} \delta_{UB'}^*$ for all $d : B \rel B'$.
  \item The commutativity condition $U\phi \circ \delta_{UB_1}^* = \delta_{UB_2} \circ U\phi$ for any $\phi \in \ef(B, B')$.
\end{itemize}
% We also require the existence of a distinguised computation morphism $\delta_{FA}^* \bisim \id_{FA}$ for each $A$,
% and a square $\delta_{FA}^* \ltdyn_{Fc}^{Fc} \delta_{FA'}^*$ for all $c : A \rel A'$.
\end{definition}

%%%%%%%%%%%%%%%%%%%%%%%%%%%%%%%%%%%%%%%%%%%%%%%%%%%%%%%%%%%%%%%%%%%%%%%%%%%%%%%%%%%%

\subsubsection{Perturbations}\label{sec:abstract-model-perturbations}

A second consequence of working intensionally is that the squares in the representable
properties must now involve a notion of ``delay" or ``perturbation'' in order to
keep the function morphisms on each side in lock-step. Intuitively, the perturbations
have no effect other than to cause the function to which they are applied to ``wait''
in a specific manner.
We formalize this notion by requiring that for each object $A$ in $\vf$,
there is a monoid of \emph{value perturbations} $P_A$ and a homomorphism of monoids
$\ptb_A : P_A \to \{ f \in \vf(A,A) \mid f \bisim \id \}$.
Similarly, for each $B : \ef$ there is a monoid $\pe_B$ of
\emph{computation perturbations} and a homomorphism of monoids 
$\ptb_B : P_B \to \{ g \in \ef(B,B) \mid g \bisim \id \}$.

% If $\delta \in P^V_A$, we will sometimes omit the homomorphism $\ptbv_A$ and simply write
% $\delta$ to refer to the morphism $\ptbv_A(\delta) \in \vf(A,A)$, and likewise
% for computation perturbations. The context will make clear whether we are referring
% to an element of the perturbation monoid or the corresponding morphism.

Note that we require that all perturbations be weakly bisimilar to the identity morphism,
capturing the notion that they have no effect other than to delay.
% We observe that
% the set of endomorphisms $f$ such that $f$ is weakly bisimilar to the identity
% forms a monoid under composition.

We will slightly abuse notation and refer to an endomorphism $f \in \vf(A, A)$ as being ``in''
the monoid of perturbations, by which we actually mean there is an element $p \in P_A$
that is mapped to $f$ under the homomorphism.

We require that $\delta_{UB}^* \in P_{UB}$ for all $B$, where $\delta_{UB}^*$ is the distinguished
morphism that is required to be present in every hom-set $\vf(UB, UB)$ per the definition
of a step-1 model.

The perturbations must be preserved by $\times$ $\timesk$, $\arr$, $\tok$, $U$, $\Uk$, $F$, and $\Fk$.

Perturbations must also satisfy a property that we call the ``push-pull'' property,
which is formulated as follows. Let $c : A \rel A'$.
Given a perturbation $\delta \in P_A$, there is a corresponding perturbation
$\push_c(\delta) \in P_{A'}$. % making the following square commute:
%
Likewise, given $\delta' \in P_{A'}$ there is a perturbation $\pull_c(\delta') \in P_A$.
% making the following square commute:

Moreover, push-pull states that the following squares must commute:

\begin{center}
  \begin{tabular}{ m{9em} m{9em} } 
    \begin{tikzcd}[ampersand replacement=\&]
      A \& {A'} \\
      A \& {A'}
      \arrow["\delta"', from=1-1, to=2-1]
      \arrow["{\push_c(\delta)}", from=1-2, to=2-2]
      \arrow["c", "\shortmid"{marking}, no head, from=1-1, to=1-2]
      \arrow["c"', "\shortmid"{marking}, no head, from=2-1, to=2-2]
    \end{tikzcd}
    &
    \begin{tikzcd}[ampersand replacement=\&]
      A \& {A'} \\
      A \& {A'}
      \arrow["{\pull_c(\delta')}"', from=1-1, to=2-1]
      \arrow["{\delta'}", from=1-2, to=2-2]
      \arrow["c", "\shortmid"{marking}, no head, from=1-1, to=1-2]
      \arrow["c"', "\shortmid"{marking}, no head, from=2-1, to=2-2]
    \end{tikzcd}
  \end{tabular}
\end{center}

The analogous property should also hold for computation relations and perturbations.

This is summarized below:

\begin{definition}\label{def:step-2-model}
  A \emph{step-2} model of intensional gradual typing consists of all the data of a step-1 model plus:
  \begin{enumerate}
    \item For each value type $A$, there is a monoid $P_A$ and a homomorphism of monoids
    $\ptb_A : P_A \to \{ f \in \vf(A,A) \mid f \bisim \id \}$.
    \item For each computation type $B$, there is a monoid $\pe_B$ and a homomorphism of monoids
    $\ptb_B : P_B \to \{ g \in \ef(B,B) \mid g \bisim \id \}$.
    \item For all $B$, the distinguished endomorphism $\delta_{UB}^*$ is in $P_{UB}$.
    % \item $\pv_A$ and a monoid homomorphism 
    %   \[ \ptbv_A : \pv_A \to \{ f \in \vf(A,A) \mid f \bisim \id \} \]
    % \item $\pe_B$ and a monoid homomorphism
    %   \[ \ptbe_B : \pe_B \to \{ g \in \ef(B,B) \mid g \bisim \id \} \]
    \item The functors $\times, \timesk$, $\arr, \tok$, $U, \Uk$, $F, \Fk$ preserve perturbations.
    \item The push-pull property holds for all $c : A \rel A'$ and all $d : B \rel B'$.
  \end{enumerate}
\end{definition}

%%%%%%%%%%%%%%%%%%%%%%%%%%%%%%%%%%%%%%%%%%%%%%%%%%%%%%%%%%%%%%%%%%%%%%%%%%%%%%%%%%%%

\subsubsection{Behavior of Casts}


% We similarly have thin subcategories $\mathcal V_u$ and $\mathcal E_d$ of
% upcasts and downcasts. The relation between function morphisms and edges
% is as follows.

As is the case in the extensional model, there is a relationship between
vertical (i.e., function) morphisms and horizontal (i.e., relation) morphisms,
but as mentioned above, now there
are perturbations involved in order to keep both sides ``in lock-step".
We begin with a step-2 intensional model as defined in the previous section,
and provide the additional conditions that axiomatize the behavior of casts.
The precise definitions are as follows.

First, let $\mathcal M$ be any double category with a notion of perturbations,
i.e., for any object $X$ in $\mathcal M$ there is a monoid $P_X$ with a monoid
homomorphism into the endomorphisms on $X$.

\begin{definition}\label{def:quasi-left-representable}
  Let $R$ be a horizontal morphism in $\mathcal M$ between objects $X$ and $Y$.
  We say that $R$ is quasi-left-representable by a vertical morphism $f$ in
  $\mathcal M(X, Y)$ if there are perturbations $\delle_R \in P_X$ and
  $\delre_R \in P_Y$ such that the following squares commute:

  \begin{center}
    \begin{tabular}{ m{7em} m{7em} } 
      % UpL
      \begin{tikzcd}[ampersand replacement=\&]
        X \& {Y} \\
        {Y} \& {Y}
        \arrow["f"', from=1-1, to=2-1]
        \arrow["\delta_R^{r,e}", from=1-2, to=2-2]
        \arrow["R", "\shortmid"{marking}, no head, from=1-1, to=1-2]
        \arrow[from=2-1, to=2-2, Rightarrow, no head]
      \end{tikzcd}
      &
      % UpR
      \begin{tikzcd}[ampersand replacement=\&]
        X \& {X} \\
        {X} \& {Y}
        \arrow["\delta_R^{l,e}"', from=1-1, to=2-1]
        \arrow["f", from=1-2, to=2-2]
        \arrow[from=1-1, to=1-2, Rightarrow, no head]
        \arrow["R"', "\shortmid"{marking}, no head, from=2-1, to=2-2]
      \end{tikzcd}
    \end{tabular}
  \end{center}

  We call the first square $\upl$ and the second square $\upr$. 

\end{definition}

\begin{definition}\label{def:quasi-right-representable}
  Let $R$ be a horizontal morphism between $X$ and $Y$. We say that $R$ is
  \emph{quasi-right-representable by} $f \in \mathcal M(Y, X)$
  if there exist perturbations $\dellp_R \in P_X$ and
  $\delrp_R \in P_Y$ such that the following squares commute:
  
  \begin{center}
    \begin{tabular}{ m{7em} m{7em} } 
      % DnR
      \begin{tikzcd}[ampersand replacement=\&]
        {X} \& {Y} \\
        {X} \& {X}
        \arrow["\delta_R^{l,p}"', from=1-1, to=2-1]
        \arrow["f", from=1-2, to=2-2]
        \arrow["R", "\shortmid"{marking}, no head, from=1-1, to=1-2]
        \arrow[from=2-1, to=2-2, Rightarrow, no head]
      \end{tikzcd}
      &
      % DnL
      \begin{tikzcd}[ampersand replacement=\&]
        {Y} \& {Y} \\
        {X} \& {Y}
        \arrow["f"', from=1-1, to=2-1]
        \arrow["\delta_R^{r,p}", from=1-2, to=2-2]
        \arrow[from=1-1, to=1-2, Rightarrow, no head]
        \arrow["R"', "\shortmid"{marking}, no head, from=2-1, to=2-2]
      \end{tikzcd}
    \end{tabular}
  \end{center}
  
  We call the first square $\dnr$ and the second square $\dnl$.
  
  \end{definition}


% TODO: Give these squares names
% \begin{definition}\label{def:quasi-left-representable}
% Let $c : A \rel A'$ be a value relation. We say that $c$ is \emph{quasi-left-representable by}
% $f \in \vf(A, A')$ if there are perturbations $\delta_c^{l,e} \in \pv_A$ and
% $\delta_c^{r,e} \in \pv_{A'}$ such that the following squares commute:

% \begin{center}
%   \begin{tabular}{ m{7em} m{7em} } 
%     % UpL
%     \begin{tikzcd}[ampersand replacement=\&]
%       A \& {A'} \\
%       {A'} \& {A'}
%       \arrow["f"', from=1-1, to=2-1]
%       \arrow["\delta_c^{r,e}", from=1-2, to=2-2]
%       \arrow["c", "\shortmid"{marking}, no head, from=1-1, to=1-2]
%       \arrow[from=2-1, to=2-2, Rightarrow, no head]
%     \end{tikzcd}
%     &
%     % UpR
%     \begin{tikzcd}[ampersand replacement=\&]
%       A \& {A} \\
%       {A} \& {A'}
%       \arrow["\delta_c^{l,e}"', from=1-1, to=2-1]
%       \arrow["f", from=1-2, to=2-2]
%       \arrow[from=1-1, to=1-2, Rightarrow, no head]
%       \arrow["c"', "\shortmid"{marking}, no head, from=2-1, to=2-2]
%     \end{tikzcd}
%   \end{tabular}
% \end{center}

% We call the first square $\upl$ and the second square $\upr$.

% \end{definition}

% \begin{definition}\label{def:quasi-right-representable}
% Let $d : B \rel B'$ be a computation relation. We say that $d$ is
% \emph{quasi-right-representable by} $f \in \ef(B', B)$
% if there exist perturbations $\delta_d^{l,p} \in \pe_B$ and
% $\delta_d^{r,p} \in \pe_{B'}$ such that the following squares commute:

% \begin{center}
%   \begin{tabular}{ m{7em} m{7em} } 
%     % DnR
%     \begin{tikzcd}[ampersand replacement=\&]
%       {B} \& {B'} \\
%       {B} \& {B}
%       \arrow["\delta_d^{l,p}"', from=1-1, to=2-1]
%       \arrow["g", from=1-2, to=2-2]
%       \arrow["R", "\shortmid"{marking}, no head, from=1-1, to=1-2]
%       \arrow[from=2-1, to=2-2, Rightarrow, no head]
%     \end{tikzcd}
%     &
%     % DnL
%     \begin{tikzcd}[ampersand replacement=\&]
%       {B'} \& {B'} \\
%       {B} \& {B'}
%       \arrow["g"', from=1-1, to=2-1]
%       \arrow["\delta_d^{r,p}", from=1-2, to=2-2]
%       \arrow[from=1-1, to=1-2, Rightarrow, no head]
%       \arrow["R"', "\shortmid"{marking}, no head, from=2-1, to=2-2]
%     \end{tikzcd}
%   \end{tabular}
% \end{center}

% We call the first square $\dnr$ and the second square $\dnl$.

% \end{definition}

With these definitions, we return to the more specific setting of a step-2
intensional model $\mathcal M$ and specify the new requirements for relations.
We require that there are functors $\upf : \ve \to \vf$ and $\dnf : \ee^{op} \to \ef$
Every value edge $c : A \rel A'$ must be quasi-left-representable by $\upf(c)$,
and every computation edge $d : B \rel B'$ is quasi-right-representable by $\dnf(d)$.

Besides the perturbations, one other difference between the extensional
and intensional versions of the representability axioms is that in the
extensional setting, the rules build in the notion of composition, whereas
their intensional counterparts do not.
In the extensional setting, we do not have horizontal composition of squares, which
is required to derive the versions of the rules that build in composition
from the versions that do not.
In the intensional setting, we do have horizontal composition of squares,
so we can take the simpler versions as primitive and derive the ones
involving composition.

Lastly, we require that the model satisfy a weak version of functoriality for 
the CBPV connectives $U,F,\times,\to$. 
First, we will need a definition:

\begin{definition}[quasi-order-equivalence]\label{def:quasi-order-equivalent}
  Let $c, c' : A \rel A'$. We say that $c$ and $c'$ are \emph{quasi-order-equivalent},
  written $c \qordeq c'$, if there exist perturbations $\delta^l_1, \delta^l_2 \in \pv_A$ and 
  $\delta^r_1, \delta^r_2 \in \pv_{A'}$ such that the following two squares exist:

  \begin{center}
    \begin{tabular}{ m{9em} m{9em} } 
      \begin{tikzcd}[ampersand replacement=\&]
        A \& {A'} \\
        A \& {A'}
        \arrow["\delta^l_1"', from=1-1, to=2-1]
        \arrow["\delta^r_1", from=1-2, to=2-2]
        \arrow["c", "\shortmid"{marking}, no head, from=1-1, to=1-2]
        \arrow["c'"', "\shortmid"{marking}, no head, from=2-1, to=2-2]
      \end{tikzcd}
      &
      \begin{tikzcd}[ampersand replacement=\&]
        A \& {A'} \\
        A \& {A'}
        \arrow["\delta^l_2"', from=1-1, to=2-1]
        \arrow["\delta^r_2", from=1-2, to=2-2]
        \arrow["c'", "\shortmid"{marking}, no head, from=1-1, to=1-2]
        \arrow["c"', "\shortmid"{marking}, no head, from=2-1, to=2-2]
      \end{tikzcd}
    \end{tabular}
  \end{center}

  We make the analogous definition for computation relations $d, d' : B \rel B'$.
\end{definition}


We require that the CBPV connectives $U,F,\times,\to$ are \emph{quasi-functorial} on relations,
which we specify as follows:
\begin{itemize}
  \item $U(d \comp d') \qordeq U(d)U(d')$
  \item $F(c \comp c'') \qordeq F(c)F(c')$
  \item $(cc') \to (dd') \qordeq (c \to d)(c' \to d')$
  \item $(c_1c_1') \times (c_2c_2') \qordeq (c_1 \times c_2)(c_1'\times c_2')$
\end{itemize}


We summarize the requirements of a step-3 model below:

\begin{definition}\label{def:step-3-model}
  A \emph{step-3 intensional model} consists
  of all the data of a step-2 intensional model, such that additionally:
  \begin{enumerate}
    \item There are functors $\upf : \ve \to \vf$ and $\dnf : \ee^{op} \to \ef$ % TODO: image is thin?
    \item Every value edge $c : A \rel A'$ is quasi-left-representable by $\upf(c)$ and
    every computation edge $d : B \rel B'$ is quasi-right-representable by $\dnf(d)$.
    \item The CBPV connectives $U,F,\times,\to$ are quasi-functorial on relations.
  \end{enumerate}
\end{definition}

% Want: U d \comp U d' = U(d \comp d')
%       F c \comp F c' = F(c \comp c')
% Add requirement: Either the model is functorial with respect to up/downcasts or with repsect to relations
% 

%%%%%%%%%%%%%%%%%%%%%%%%%%%%%%%%%%%%%%%%%%%%%%%%%%%%%%%%%%%%%%%%%%%%%%%%%%%%%%%%%%%%

\subsubsection{The Dynamic Type}

Now we can discuss what it means for an intensional model to model the dynamic type.
% This applies to any of the above abstract model definitions, i.e., steps 0-3.

\begin{definition}\label{def:step-4-model}
  % A \emph{step-$i$ intensional model with dyn} is a step-$i$ model $\mathcal M$ such that:
  A step-4 intensional model is a step-3 intensional model $\mathcal M$ such that:
  %a distinguished value object $D \in \ob(\vf)$ such that:
  %
  \begin{enumerate}
    \item There is a distinguished value object $D \in \ob(\vf)$.
    \item There are distinguished value relations 
    $\iarr{}: U(D \to F D) \rel D$ and $\inat : \nat \rel D$ and $\itimes : D \times D \rel D$
    each satisfying the retraction property up to bisimilarity.
    %  $\dnc {\injarr{}}F(\upc{\injarr{}}) \equidyn \id$.
   
    %\item For each value type $A$, there is a value relation $\text{inj}_A : A \rel D$.
    
    %\item \eric{Do we need this?} If $c : A \rel A'$, then $\text{inj}_{A} = c \comp \text{inj}_{A'}$.
  \end{enumerate}
\end{definition}




% (By definition of a step-3 model, this relation satisfies the push-pull property and is
% quasi-left-representable.)

% (By definition of a step-3 model, this means there is also a monoid $\pv_D$ of
% perturbations and a homomorphism $\ptbv_D$.)



%%%%%%%%%%%%%%%%%%%%%%%%%%%%%%%%%%%%%%%%%%%%%%%%%%%%%%%%%%%%%%%%%%%%%%%%%%%%%%%%%%%%
%%%%%%%%%%%%%%%%%%%%%%%%%%%%%%%%%%%%%%%%%%%%%%%%%%%%%%%%%%%%%%%%%%%%%%%%%%%%%%%%%%%%

\subsection{Constructing an Extensional Model}\label{sec:extensional-model-construction}

In the previous section, we have given the definition of an intensional model
of gradual typing as a series of steps with each definition building on the previous one.
%
Here, we discuss how to construct an extensional model from an intensional model with dyn.
We do so in several phases, beginning with a step-1 intensional model with dyn
and ending with an extensional model.
Moreover, this construction is \emph{modular}, in that each phase of the
construction does not depend on the details of the previous ones.
% However, this process cannot proceed in isolation: some phases require
% additional inputs. We will make clear what data must be supplied to each phase.

\subsubsection{Adding Perturbations}\label{sec:constructing-perturbations}

Suppose we have a \hyperref[def:step-1-model]{step-1 intensional model} $\mathcal{M}$.
Recall that a step-1 intensional model consists of a step-0 model (i.e., a
category internal to the category of CBPV models), along with the necessary
categories and functors for bisimilarity as discussed in Section
\ref{sec:abstract-model-bisimilarity}.
Further, recall that a \hyperref[def:step-2-model]{step-2 model} has everything
a step-1 model has, with the addition of perturbation monoids $\pv_A$ for all
$A$ and $\pe_B$ for all $B$.
Moreover, the push-pull property must hold for all
value relations $c$ and all computation relations $d$.

We claim that from a step-1 model, we can construct a step-2 model. 
The value objects of the model are defined to be triples $(A, P_A, \ptb_A)$ where $A$ is a
value object in $\mathcal{M}$, $P_A$ is a monoid and $\ptb_A$ is a homomorphism of monoids
from $P_A$ to the endomorphisms on $A$ that are bisimilar to the identity.
Likewise, computation objects are triples $(B, P_B, \ptb_B)$.
The morphisms are the same as the morphisms of $\mathcal{M}$.
A value relation between $(A, P_A, \ptb_A)$ and $(A', P_{A'}, \ptb_{A'})$ is given by
a pair of a relation in $c$ and a \emph{push-pull structure} $\Pi_c$ specifying that
that $c$ satisfies the push-pull property. Computation relations are defined analogously.
The squares are the same as those of $\mathcal{M}$.
We define the action of the functor $F$ on objects as $F(A, P_A, \ptb_A) = 
(FA, \mathbb{N} \times P_A, \ptb_{FA})$ where $\ptb_{FA}(n, a) = (\delta_{FA}^*)^n \circ F(\ptb_A(a))$
(i.e., we use the distinguished delay morphism $\delta_{FA}^*$).
The action of $U$ on objects is defined similarly, where the perturbations
are defined to be $\mathbb{N} \times P_B$.

For the full details of the construction, see Lemma \ref{lem:step-1-model-to-step-2-model} in the Appendix.

%%%%%%%%%%%%%%%%%%%%%%%%%%%%%%%%%%%%%%%%%%%%%%%%%%%%%%%%%%%%%%%%%%%%%%%%%%%%%%%%%%%%

\subsubsection{Adding Quasi-Representability}

Now suppose we have a step-2 intensional model $\mathcal{M}$.
We claim that we can construct a \hyperref[def:step-3-model]{step-3 intensional model} $\mathcal{M'}$.
In the construction, the objects and morphisms are the same as those of $\mathcal M$,
and a value relation between $A$ and $A'$ consists of a triple $(c, \rho^L_c, \rho^R_{Fc})$
where $c : A \rel A'$ is a relation in $\mathcal{M}$,
$\rho^L_c$ is a quasi-\emph{left}-representation for $c$ 
(i.e., an embedding $e_c$, perturbations $\delre_c \in P_{A'}$ and $\delle_c \in P_A$, 
and the two relevant squares), and similarly $\rho^R_{Fc}$ is
a quasi-\emph{right}-representation for $Fc$.
Computation relations are triples $(d, \rho^R_d, \rho^L_{Ud})$ where $d : B \rel B'$,
$\rho^R_d$ is a quasi-right-representation for $d$ and $\rho^L{Ud}$ is a quasi-left-representation for $Ud$.
The value and computation squares are the same as those of $\mathcal{M}$.
We then define composition of relations and the action of the functors $F$, $U$, $\times$, and $\arr$.

For the full details of the construction, see Lemma \ref{lem:step-2-model-to-step-3-model} in the Appendix.


%%%%%%%%%%%%%%%%%%%%%%%%%%%%%%%%%%%%%%%%%%%%%%%%%%%%%%%%%%%%%%%%%%%%%%%%%%%%%%%%%%%%

\subsubsection{Constructing an Extensional Model}\label{sec:extensional-model-definition}

Finally, suppose $\mathcal M$ is a \hyperref[def:step-4-model]{step-4 intensional model}
(i.e., a step-3 model with an interpretation of the dynamic type).
We now describe how to build an extensional model.

The idea is to define an extensional model whose squares are the ``bisimilarity-closure''
of the squares of the provided intensional model $\mathcal M$.
  
The categories $\vf$, $\ef$ are the same as those of $\mathcal M$.
Additionally, the objects of $\vsq$ and $\esq$, i.e., the value and
computation relations, are the same.
The difference arises in the \emph{morphisms} of $\vsq$ and $\esq$,
i.e., the commuting squares. In particular, a morphism
$\alpha_e \in \vsq'(c_i, c_o)$ with source $f$ and target $g$ is given by:
\begin{itemize}
  \item a morphism $f' \in \vf(A_i, A_o)$ with $f \bisim f'$.
  \item a morphism $g' \in \vf(A_i', A_o')$ with $g \bisim g'$.
  \item a square $\alpha_i \in \vsq(c_i, c_o)$ with source $f'$ and target $g'$.
\end{itemize}

Using our existing notation, we say that $f \ltls_{c_o}^{c_i} g$ if there exist $f'$ and $g'$ such that

\[ f \bisim_{A_i,A_o} f' \ltdyn_{c_o}^{c_i} g' \bisim_{A_i',A_o'} g. \]

We make the analogous construction for the computation squares.

The proof that this indeed defines an extensional model is given in the Appendix 
(see Section \ref{sec:extensional-construction-appendix}).

% Next, we check that the requirements of an extensional model are satisfied.
% In particular, we need to verify the representability properties.
% We define functors $\upf$ and $\dnf$






%\section{A Denotational Model for Intensional Gradual Typing}

In this section we model intensional gradual typing in a suitable double category.

We construct a (thin) double category $\mathsf{IGTT}$ as follows.

% First, we fix a set $D$ with a partial order $\le_D$ and bisimilarity relation $\bisim_D$.
% This is intended to model the dynamic type. Now we define the double category:

\begin{itemize}
  \item \textbf{Objects}: An object consists of the following data:
    \begin{itemize}
        \item A double poset $X$, i.e., a set $X$ equipped with a partial order $\le_X$
        and a reflexive, symmetric ``bisimilarity'' relation $\bisim_X$.
        \item Two commutative monoids of perturbations $P_V$ and $P_C$ with homomorphisms
        \begin{align*}
        \ptb_V &: P_V \to \{ f : X \to_m X \mid f \bisim \id \} \\
        \ptb_C &: P_C \to \{ f : X \to_m \li X \mid f \bisim \eta \}
        \end{align*}
        (where composition in the latter monoid of functions is given by Kleisli composition).

    \end{itemize}

  \item \textbf{Vertical arrows}: An vertical arrow from $(X, P_V^X, P_C^X)$ to $(Y, P_V^Y, P_C^Y)$ is
  a function $f : X \to Y$ that is \emph{monotone} (preserves ordering) and preserves the bisimilarity relation.
  % that preserves ordering and bisimilarity.
  
  \item \textbf{Horizontal arrows}: A horizontal arrow from $(X, P_V^X, P_C^X)$ to $(Y, P_V^Y, P_C^Y)$
  consists of:
  \begin{itemize}
    \item A relation $R : X \nrightarrow Y$ that is antitone with respect to $\le_X$ and
    monotone with respect to $\le_Y$.
    \item An embedding $e_{XY} : X \to_m Y$ preserving ordering and bisimilarity.
    \item A projection $p_{XY} : Y \to_m \li X$ preserving ordering and bisimilarity.
  \end{itemize}
  
  such that (1) $R$ is \emph{quasi-representable} by $e_{XY}$ and $p_{XY}$, and 
  (2) $R$ satisfies the \emph{push-pull} property.
  
  \vspace{3ex}
  The former means that there are distinguished elements $\delta^{l,e} \in P_V^X$, $\delta^{r,e} \in P_V^Y$, 
  $\delta^{l,p} \in P_C^X$ and $\delta^{r,p} \in P_C^Y$ such that the following squares commute:

  \begin{center}
    \begin{tabular}{ c | c } 
        \hline
        \hspace{3em} 
        % UpL
        % https://q.uiver.app/#q=WzAsNCxbMCwwLCJYIl0sWzAsMSwiWSJdLFsxLDAsIlkiXSxbMSwxLCJZIl0sWzAsMiwiUiIsMCx7InN0eWxlIjp7ImJvZHkiOnsibmFtZSI6ImJhcnJlZCJ9LCJoZWFkIjp7Im5hbWUiOiJub25lIn19fV0sWzEsMywiXFxsZV9ZIiwyLHsic3R5bGUiOnsiYm9keSI6eyJuYW1lIjoiYmFycmVkIn0sImhlYWQiOnsibmFtZSI6Im5vbmUifX19XSxbMCwxLCJlX3tYWX0iLDJdLFsyLDMsIlxccHRiX1ZeWShcXGRlbHRhXntyLGV9KSJdLFs2LDcsIlxcbHRkeW4iLDEseyJzaG9ydGVuIjp7InNvdXJjZSI6MjAsInRhcmdldCI6MjB9LCJzdHlsZSI6eyJib2R5Ijp7Im5hbWUiOiJub25lIn0sImhlYWQiOnsibmFtZSI6Im5vbmUifX19XV0=
        \begin{tikzcd}[ampersand replacement=\&]
            X \& Y \\
            Y \& Y
            \arrow["R", "\shortmid"{marking}, no head, from=1-1, to=1-2]
            \arrow["{\le_Y}"', "\shortmid"{marking}, no head, from=2-1, to=2-2]
            \arrow[""{name=0, anchor=center, inner sep=0}, "{e_{XY}}"', from=1-1, to=2-1]
            \arrow[""{name=1, anchor=center, inner sep=0}, "{\ptb_V^Y(\delta^{r,e})}", from=1-2, to=2-2]
            \arrow["\ltdyn"{description}, draw=none, from=0, to=1]
        \end{tikzcd} & 
        % UpR
        % https://q.uiver.app/#q=WzAsNCxbMCwwLCJYIl0sWzAsMSwiWCJdLFsxLDAsIlgiXSxbMSwxLCJZIl0sWzAsMSwiXFxwdGJfVl5YKFxcZGVsdGFee2wsZX0pIiwyXSxbMiwzLCJlX3tYWX0iXSxbMSwzLCJSIiwyLHsic3R5bGUiOnsiYm9keSI6eyJuYW1lIjoiYmFycmVkIn0sImhlYWQiOnsibmFtZSI6Im5vbmUifX19XSxbMCwyLCJcXGxlX1giLDAseyJzdHlsZSI6eyJib2R5Ijp7Im5hbWUiOiJiYXJyZWQifSwiaGVhZCI6eyJuYW1lIjoibm9uZSJ9fX1dLFs0LDUsIlxcbHRkeW4iLDEseyJzaG9ydGVuIjp7InNvdXJjZSI6MjAsInRhcmdldCI6MjB9LCJzdHlsZSI6eyJib2R5Ijp7Im5hbWUiOiJub25lIn0sImhlYWQiOnsibmFtZSI6Im5vbmUifX19XV0=
        \begin{tikzcd}[ampersand replacement=\&]
            X \& X \\
            X \& Y
            \arrow[""{name=0, anchor=center, inner sep=0}, "{\ptb_V^X(\delta^{l,e})}"', from=1-1, to=2-1]
            \arrow[""{name=1, anchor=center, inner sep=0}, "{e_{XY}}", from=1-2, to=2-2]
            \arrow["R"', "\shortmid"{marking}, no head, from=2-1, to=2-2]
            \arrow["{\le_X}", "\shortmid"{marking}, no head, from=1-1, to=1-2]
            \arrow["\ltdyn"{description}, draw=none, from=0, to=1]
        \end{tikzcd} \\
        \hline
        % DnR
        % https://q.uiver.app/#q=WzAsNCxbMCwwLCJYIl0sWzEsMCwiWSJdLFswLDEsIkxYIl0sWzEsMSwiTFgiXSxbMCwyLCJcXHB0Yl9DXlgoXFxkZWx0YV57bCxwfSkiLDJdLFsxLDMsInBfe1hZfSJdLFswLDEsIlIiLDAseyJzdHlsZSI6eyJib2R5Ijp7Im5hbWUiOiJiYXJyZWQifSwiaGVhZCI6eyJuYW1lIjoibm9uZSJ9fX1dLFsyLDMsIlxcbGVfe0xYfSIsMix7InN0eWxlIjp7ImJvZHkiOnsibmFtZSI6ImJhcnJlZCJ9LCJoZWFkIjp7Im5hbWUiOiJub25lIn19fV0sWzQsNSwiXFxsdGR5biIsMSx7InNob3J0ZW4iOnsic291cmNlIjoyMCwidGFyZ2V0IjoyMH0sInN0eWxlIjp7ImJvZHkiOnsibmFtZSI6Im5vbmUifSwiaGVhZCI6eyJuYW1lIjoibm9uZSJ9fX1dXQ==
        \begin{tikzcd}[ampersand replacement=\&]
            X \& Y \\
            LX \& LX
            \arrow[""{name=0, anchor=center, inner sep=0}, "{\ptb_C^X(\delta^{l,p})}"', from=1-1, to=2-1]
            \arrow[""{name=1, anchor=center, inner sep=0}, "{p_{XY}}", from=1-2, to=2-2]
            \arrow["R", "\shortmid"{marking}, no head, from=1-1, to=1-2]
            \arrow["{\le_{LX}}"', "\shortmid"{marking}, no head, from=2-1, to=2-2]
            \arrow["\ltdyn"{description}, draw=none, from=0, to=1]
        \end{tikzcd} & 
            \hspace{3em} 
            % DnL
            % https://q.uiver.app/#q=WzAsNCxbMCwwLCJZIl0sWzEsMCwiWSJdLFswLDEsIkxYIl0sWzEsMSwiTFkiXSxbMCwxLCJcXGxlX1kiLDAseyJzdHlsZSI6eyJib2R5Ijp7Im5hbWUiOiJiYXJyZWQifSwiaGVhZCI6eyJuYW1lIjoibm9uZSJ9fX1dLFswLDIsInBfe1hZfSIsMl0sWzEsMywiXFxwdGJfQ15ZKFxcZGVsdGFee3IscH0pIl0sWzIsMywiTFIiLDIseyJzdHlsZSI6eyJib2R5Ijp7Im5hbWUiOiJiYXJyZWQifSwiaGVhZCI6eyJuYW1lIjoibm9uZSJ9fX1dLFs1LDYsIlxcbHRkeW4iLDEseyJzaG9ydGVuIjp7InNvdXJjZSI6MjAsInRhcmdldCI6MjB9LCJzdHlsZSI6eyJib2R5Ijp7Im5hbWUiOiJub25lIn0sImhlYWQiOnsibmFtZSI6Im5vbmUifX19XV0=
            \begin{tikzcd}[ampersand replacement=\&]
                Y \& Y \\
                LX \& LY
                \arrow["{\le_Y}", "\shortmid"{marking}, no head, from=1-1, to=1-2]
                \arrow[""{name=0, anchor=center, inner sep=0}, "{p_{XY}}"', from=1-1, to=2-1]
                \arrow[""{name=1, anchor=center, inner sep=0}, "{\ptb_C^Y(\delta^{r,p})}", from=1-2, to=2-2]
                \arrow["LR"', "\shortmid"{marking}, no head, from=2-1, to=2-2]
                \arrow["\ltdyn"{description}, draw=none, from=0, to=1]
            \end{tikzcd} \\ 
        \hline
    \end{tabular}
    \end{center}
    (Here, $LR$ is the lock step lifting of the relation $R$.)

    \vspace{3ex}

    The push-pull property is defined as follows:
    \begin{itemize}
        \item Given any perturbation $\delta_X \in P_V^X$, we can \emph{push} it forward along $R$ to a
        perturbation $\push(\delta_X) \in P_V^Y$, such that $\ptb_V^X(\delta_X) \le \ptb_V^Y(\push(\delta_X))$.

        \item Conversely, given any perturbation $\delta_Y \in P_V^Y$, we can \emph{pull} it back along $R$
        to a perturbation $\pull(\delta_Y) \in P_V^X$, such that $\ptb_V^X(\pull(\delta_Y)) \le \ptb_V^Y(\delta_Y)$.

        \item Likewise, we can push any perturbation $\delta_X \in P_C^X$ along $LR$
        to get a perturbation $\push(\delta_X) \in P_C^Y$ such that
        $\ptb_C^X(\delta_X) \le \ptb_C^Y(\push(\delta_X))$.

        \item And similarly, we can pull a perturbation in $P_C^Y$ along $LR$ to a perturbation in $P_C^X$
        satisfying the analogous property.
    \end{itemize}

    \textbf{TODO: push and pull might need to be monoid homomorphisms}

  \item \textbf{Two-cells}: Let $f : W \to X$ and $g : Y \to Z$ and let $R : W \nrightarrow Y$ and 
  $S : X \nrightarrow Z$. We define $f \le g$ to mean for all $(w, y) \in R$, we have
  $(f(w), g(y)) \in S$. This is depicted in the square below:

  % https://q.uiver.app/#q=WzAsNCxbMCwwLCJXIl0sWzAsMSwiWCJdLFsxLDAsIlkiXSxbMSwxLCJaIl0sWzAsMiwiUiIsMCx7InN0eWxlIjp7ImJvZHkiOnsibmFtZSI6ImJhcnJlZCJ9LCJoZWFkIjp7Im5hbWUiOiJub25lIn19fV0sWzEsMywiUyIsMix7InN0eWxlIjp7ImJvZHkiOnsibmFtZSI6ImJhcnJlZCJ9LCJoZWFkIjp7Im5hbWUiOiJub25lIn19fV0sWzAsMSwiZiIsMl0sWzIsMywiZyJdLFs0LDUsIlxcc3FzdWJzZXRlcSIsMSx7InNob3J0ZW4iOnsic291cmNlIjoyMCwidGFyZ2V0IjoyMH0sInN0eWxlIjp7ImJvZHkiOnsibmFtZSI6Im5vbmUifSwiaGVhZCI6eyJuYW1lIjoibm9uZSJ9fX1dXQ==
\[\begin{tikzcd}[ampersand replacement=\&]
	W \& Y \\
	X \& Z
	\arrow[""{name=0, anchor=center, inner sep=0}, "R", "\shortmid"{marking}, no head, from=1-1, to=1-2]
	\arrow[""{name=1, anchor=center, inner sep=0}, "S"', "\shortmid"{marking}, no head, from=2-1, to=2-2]
	\arrow["f"', from=1-1, to=2-1]
	\arrow["g", from=1-2, to=2-2]
	\arrow["\sqsubseteq"{description}, draw=none, from=0, to=1]
\end{tikzcd}\]
  
\end{itemize}

The category satisfies the following additional properties:
\begin{itemize}
    \item \emph{Existence of Dyn}: There is an object $D$ with the property that for any
    object $X$, there is a horizontal arrow $X \nrightarrow D$.
    The underlying double poset is defined by guarded recursion as the solution to
    \[ D \cong \mathbb{N}\, + \later \hspace{-0.5ex} (D \to_m \li D). \]

    \textbf{TODO: define the perturbations for Dyn and show there is a horizontal arrow $X \nrightarrow D$ for all $X$.}
    
    \item \emph{Thinness}: There is at most one two-cell for any given square.

    % \item \emph{Push-Pull}: Let $X$ and $Y$ be objects, and let $R : X \nrightarrow Y$.
    % \begin{itemize}
    %     \item Given any perturbation $\delta_X \in P_V^X$, we can \emph{push} it forward along $R$ to a
    %     perturbation $\push(\delta_X) \in P_V^Y$, such that $\ptb_V^X(\delta_X) \le \ptb_V^Y(\push(\delta_X))$.

    %     \item Conversely, given any perturbation $\delta_Y \in P_V^Y$, we can \emph{pull} it back along $R$
    %     to a perturbation $\pull(\delta_Y) \in P_V^X$, such that $\ptb_V^X(\pull(\delta_Y)) \le \ptb_V^Y(\delta_Y)$.

    %     \item Likewise, we can push any perturbation $\delta_X \in P_C^X$ along $L\, R$
    %     (the lock-step lifting of the relation $R$) to get a perturbation $\push(\delta_X) \in P_C^Y$ such that
    %     $\ptb_C^X(\delta_X) \le \ptb_C^Y(\push(\delta_X))$.

    %     \item And similarly we can pull a perturbation in $P_C^Y$ to a perturbation in $P_C^X$
    %     satisfying the analogous property.
    % \end{itemize}
\end{itemize}

% Composability of embedding and projections

We need to verify that this forms a thin double category. 
\begin{itemize}
    \item \emph{Horizontal identity morphism}: 
    Let $X$ be an object. We take $R$ to be $\le_X$ (the ordering relation on $X$),
    which is trivially antitone and monotone with respect to itself.
    We let $e_{XX} = \id$ and $p_{XX} = \eta$. These clearly preserve the
    ordering and bisimilarity.

    \vspace{3ex}
    
    We first need to show that $R$ is quasi-representable.
    We prove the UpR rule; the others are similar.
    We need to specify a distinguished element $\delta^{l,e} \in P_V^X$ such that
    the following square commutes:
    
    % https://q.uiver.app/#q=WzAsNCxbMCwwLCJYIl0sWzAsMSwiWCJdLFsxLDAsIlgiXSxbMSwxLCJYIl0sWzAsMSwiXFxwdGJfVl5YKFxcZGVsdGFee2wsZX0pIiwyXSxbMiwzLCJlX3tYWH0gPSBcXGlkIl0sWzAsMiwiXFxsZV9YIiwwLHsic3R5bGUiOnsiYm9keSI6eyJuYW1lIjoiYmFycmVkIn0sImhlYWQiOnsibmFtZSI6Im5vbmUifX19XSxbMSwzLCJcXGxlX1giLDIseyJzdHlsZSI6eyJib2R5Ijp7Im5hbWUiOiJiYXJyZWQifSwiaGVhZCI6eyJuYW1lIjoibm9uZSJ9fX1dXQ==
    \[\begin{tikzcd}[ampersand replacement=\&]
        X \& X \\
        X \& X
        \arrow["{\ptb_V^X(\delta^{l,e})}"', from=1-1, to=2-1]
        \arrow["{e_{XX} = \id}", from=1-2, to=2-2]
        \arrow["{\le_X}", "\shortmid"{marking}, no head, from=1-1, to=1-2]
        \arrow["{\le_X}"', "\shortmid"{marking}, no head, from=2-1, to=2-2]
    \end{tikzcd}\]

    Taking $\delta^{l,e} = \id$ (the identity of the monoid), we observe that since
    $\ptb_V^X$ is a homomorphism of monoids, we have $\ptb_V^X(\id) = \id$.
    Now it is clear that the above square commutes.

    \vspace{3ex}

    We also need to show that $R$ satisfies the four push-pull properties.
    We show one; the others are similar. Let $\delta_X \in P_V^X$.
    We need to define $\push(\delta_X) \in P_V^X$ such that the following square commutes:

    % https://q.uiver.app/#q=WzAsNCxbMCwwLCJYIl0sWzAsMSwiWCJdLFsxLDAsIlgiXSxbMSwxLCJYIl0sWzAsMiwiXFxsZV9YIiwwLHsic3R5bGUiOnsiYm9keSI6eyJuYW1lIjoiYmFycmVkIn0sImhlYWQiOnsibmFtZSI6Im5vbmUifX19XSxbMSwzLCJcXGxlX1giLDIseyJzdHlsZSI6eyJib2R5Ijp7Im5hbWUiOiJiYXJyZWQifSwiaGVhZCI6eyJuYW1lIjoibm9uZSJ9fX1dLFswLDEsIlxccHRiX1ZeWChcXGRlbHRhX1gpIiwyXSxbMiwzLCJcXHB0Yl9WXlgoXFxwdXNoKFxcZGVsdGFfWCkpIiwwLHsic3R5bGUiOnsiYm9keSI6eyJuYW1lIjoiZGFzaGVkIn19fV1d
    \[\begin{tikzcd}[ampersand replacement=\&]
	X \& X \\
	X \& X
	\arrow["{\le_X}", "\shortmid"{marking}, no head, from=1-1, to=1-2]
	\arrow["{\le_X}"', "\shortmid"{marking}, no head, from=2-1, to=2-2]
	\arrow["{\ptb_V^X(\delta_X)}"', from=1-1, to=2-1]
	\arrow["{\ptb_V^X(\push(\delta_X))}", dashed, from=1-2, to=2-2]
    \end{tikzcd}\]

    Let $\push(\delta_X) = \delta_X$. We need to show that for $x \le_X x'$,
    we have $\ptb_V^X(\delta_X)(x) \le \ptb_V^X(\delta_X)(x')$, which holds because
    $\ptb_V^X(\delta_X)$ is monotone with respect to $\le_X$.


    \item \emph{Horizontal composition}:
    
    Let $R : X \nrightarrow Y$ and $S : Y \nrightarrow Z$.
    We define
    $e_{XZ} = e_{YZ} \circ e_{XY}$ and $p_{XZ} = \ext{p_{XY}}{} \circ p_{YZ}$.
    We define the distinguished perturbations in the representability rules as follows:

    \begin{align*}
        \delta_{R\circ S}^{l,e} &= \pull_R(\delta_S^{l,e}) \cdot \delta_R^{l,e} \\
        \delta_{R\circ S}^{r,e} &= \delta_S^{r,e} \cdot \push_S(\delta_R^{r,e}) \\
        \delta_{R\circ S}^{l,p} &= \delta_R^{l,p} \cdot \pull_{LR}(\delta_S^{l,p}) \\
        \delta_{R\circ S}^{r,p} &= \push_{LS}(\delta_R^{r,p}) \cdot \delta_S^{r,p}
    \end{align*}
    where $\cdot$ denotes composition in the appropriate monoid of perturbations.

    We can then show that the four quasi-representability rules are valid with these definitions.
    \textbf{TODO show one or two of the cases}

    \vspace{3ex}

    We also need to show that the push-pull rules hold of the composition $R \circ S$.
    This follows from the fact that they hold for both $R$ and $S$.
    Specifically, we define

    \begin{align*}
        \push_{R \circ S}(\delta^X) &= \push_S(\push_R(\delta^X)) \\
        \pull_{R \circ S}(\delta^Z) &= \pull_R(\pull_S(\delta^Z)) \\
        \push_{L(R \circ S)}(\delta^X) &= \push_{LS}(\push_{LR}(\delta^X)) \\
        \pull_{L(R \circ S)}(\delta^Z) &= \pull_{LR}(\pull_{LS}(\delta^Z))
    \end{align*}

    Then we can verify that the relevant push-pull inequalities hold using the above definitions.
    \textbf{TODO maybe show one of the cases}

    \item \emph{Identity two-cells}:
    The horizontal identity two-cells have the form

    % https://q.uiver.app/#q=WzAsNCxbMCwwLCJYIl0sWzAsMSwiWSJdLFsxLDAsIlgiXSxbMSwxLCJZIl0sWzAsMSwiZiIsMl0sWzIsMywiZiJdLFswLDIsIlxcbGVfWCJdLFsxLDMsIlxcbGVfWSIsMl0sWzQsNSwiXFxsdGR5biIsMSx7InNob3J0ZW4iOnsic291cmNlIjoyMCwidGFyZ2V0IjoyMH0sInN0eWxlIjp7ImJvZHkiOnsibmFtZSI6Im5vbmUifSwiaGVhZCI6eyJuYW1lIjoibm9uZSJ9fX1dXQ==
    \[\begin{tikzcd}[ampersand replacement=\&]
        X \& X \\
        Y \& Y
        \arrow[""{name=0, anchor=center, inner sep=0}, "f"', from=1-1, to=2-1]
        \arrow[""{name=1, anchor=center, inner sep=0}, "f", from=1-2, to=2-2]
        \arrow["{\le_X}", from=1-1, to=1-2]
        \arrow["{\le_Y}"', from=2-1, to=2-2]
        \arrow["\ltdyn"{description}, draw=none, from=0, to=1]
    \end{tikzcd}\]

    This square commutes because $f$ is monotone with respect to the ordering relation.

    The vertical two-cells have the form

    % https://q.uiver.app/#q=WzAsNCxbMCwwLCJYIl0sWzAsMSwiWCJdLFsxLDAsIlkiXSxbMSwxLCJZIl0sWzAsMiwiUiIsMCx7InN0eWxlIjp7ImJvZHkiOnsibmFtZSI6ImJhcnJlZCJ9LCJoZWFkIjp7Im5hbWUiOiJub25lIn19fV0sWzEsMywiUiIsMCx7InN0eWxlIjp7ImJvZHkiOnsibmFtZSI6ImJhcnJlZCJ9LCJoZWFkIjp7Im5hbWUiOiJub25lIn19fV0sWzAsMSwiXFxpZF9YIiwyXSxbMiwzLCJcXGlkX1kiXSxbNiw3LCJcXGx0ZHluIiwxLHsic2hvcnRlbiI6eyJzb3VyY2UiOjIwLCJ0YXJnZXQiOjIwfSwic3R5bGUiOnsiYm9keSI6eyJuYW1lIjoibm9uZSJ9LCJoZWFkIjp7Im5hbWUiOiJub25lIn19fV1d
    \[\begin{tikzcd}[ampersand replacement=\&]
        X \& Y \\
        X \& Y
        \arrow["R", "\shortmid"{marking}, no head, from=1-1, to=1-2]
        \arrow["R", "\shortmid"{marking}, no head, from=2-1, to=2-2]
        \arrow[""{name=0, anchor=center, inner sep=0}, "{\id_X}"', from=1-1, to=2-1]
        \arrow[""{name=1, anchor=center, inner sep=0}, "{\id_Y}", from=1-2, to=2-2]
        \arrow["\ltdyn"{description}, draw=none, from=0, to=1],
    \end{tikzcd}\]

    which commutes trivially.

    \item \emph{Composition of two-cells}:
    Two-cells compose vertically and horizontally, which follows from the definition
    definition of a two-cell in this category.

    \textbf{TODO: elaborate}
\end{itemize}

\vspace{3ex}

\textbf{Kleisli internal hom: TODO}

% The arrow => takes a value type and an algebra and constructs an algebra
\section{A Simple Denotational Semantics for the Terms of GTLC}\label{sec:gtlc-terms}

In this section, we introduce the term syntax for the gradually-typed
lambda calculus (GTLC) and give a set-theoretic denotational semantics
using tools from SGDT.

In the following section, we will extend the denotational semantics to accommodate
the type and term precision orderings.


\subsection{Syntax}\label{sec:term-syntax}

Our syntax is based on fine-grained call by value, and as such it has
separate value and producer terms and typing judgments for each.

% Given a term $M$ of type $A$, the term $\bind{x}{M}{N}$ should be thought of as
% running $M$ to a value $V$ and then continuing as $N$, with $V$ in place of $x$.


\begin{align*}
  &\text{Types } A := \nat \alt \,\dyn \alt (A \ra A') \\
  &\text{Contexts } \Gamma := \cdot \alt (\Gamma, x : A) \\
  &\text{Values } V :=  \zro \alt \suc\, V \alt \lda{x}{M} \alt \up{A}{B} V \\ 
  &\text{Producers } M, N := \err_B \alt \matchnat {V} {M} {n} {M'} \\ 
  &\quad\quad \alt \ret {V} \alt \bind{x}{M}{N} \alt V_f\, V_x \alt \dn{A}{B} M 
\end{align*}


The value typing judgment is written $\vhasty{\Gamma}{V}{A}$ and 
the producer typing judgment is written $\phasty{\Gamma}{M}{A}$.

The typing rules are as expected, with a cast between $A$ to $B$ allowed only when $A \ltdyn B$.
The precise rules for $A \ltdyn B$ will be given below.
Notice that the upcast of a value is a value, since it always succeeds, while the downcast
of a value is a producer, since it may fail.

\begin{mathpar}
    % Var
    \inferrule*{ }{\vhasty {\cdot, \Gamma, x : A, \Gamma'} x A}

    % Err
    \inferrule*{ }{\phasty {\cdot, \Gamma} {\err_A} A} 
  
    % Zero and suc
    \inferrule*{ }{\vhasty \Gamma \zro \nat}
  
    \inferrule*{\vhasty \Gamma V \nat} {\vhasty \Gamma {\suc\, V} \nat}

    % Match-nat
    \inferrule*
    {\vhasty \Gamma V \nat \and 
     \phasty \Delta M A \and \phasty {\Gamma, n : \nat} {M'} A}
    {\phasty \Gamma {\matchnat {V} {M} {n} {M'}} A}
  
    % Lambda
    \inferrule* 
    {\phasty {\Gamma, x : A} M {A'}} 
    {\vhasty \Gamma {\lda x M} {A \ra A'}}
  
    % App
    \inferrule*
    {\vhasty \Gamma {V_f} {A \ra A'} \and \vhasty \Gamma {V_x} A}
    {\phasty {\Gamma} {V_f \, V_x} {A'}}

    % Ret
    \inferrule*
    {\vhasty \Gamma V A}
    {\phasty {\Gamma} {\ret\, V} {A}}

    % Bind
    \inferrule*
    {\phasty \Gamma M {A} \and \phasty{\Gamma, x : A}{N}{B} } % Need x : A in context
    {\phasty {\Gamma} {\bind{x}{M}{N}} {B}}

    % Upcast
    \inferrule*
    {A \ltdyn A' \and \vhasty \Gamma V A}
    {\vhasty \Gamma {\up A {A'} V} {A'} }

    \inferrule* % TODO is this correct?
    {A \ltdyn A' \and \phasty {\Gamma} {M} {A'}}
    {\phasty {\Gamma} {\dn A {A'} M} {A}}

\end{mathpar}


In the equational theory, we have $\beta$ and $\eta$ laws for function type,
as well a $\beta$ and $\eta$ law for bind.

\begin{mathpar}
  % Function Beta and Eta
  \inferrule*
  {\phasty {\Gamma, x : A} M {B} \and \vhasty \Gamma V A}
  {(\lda x M)\, V = M[V/x]}

  \inferrule*
  {\vhasty \Gamma V {A \ra A}}
  {\Gamma \vdash V = \lda x {V\, x}}

  % Ret Beta and Eta
  \inferrule*
  {}
  {(\bind{x}{\ret\, V}{N}) = N[V/x]}

  \inferrule*
  {\phasty {\Gamma} {M} {B}}
  {\bind{x}{M}{\ret x} = M}

  % Match-nat Beta
  \inferrule*
  {\phasty \Delta M A \and \phasty {\Gamma, n : \nat} {M'} A}
  {\matchnat{\zro}{M}{n}{M'} = M}

  \inferrule*
  {\vhasty \Gamma V \nat \and 
   \phasty \Gamma M B \and \phasty {\Gamma, n : \nat} {M'} B}
  {\matchnat{\suc\, V}{M}{n}{M'} = M'}

  % Match-nat Eta
  % This doesn't build in substitution
  \inferrule*
  {\hasty {\Gamma , x : \nat} M A}
  {M = \matchnat{x} {M[\zro / x]} {n} {M[(\suc\, n) / x]}}

\end{mathpar}

\subsubsection{Type Precision}\label{sec:type-precision}

The type precision rules specify what it means for a type $A$ to be more precise than $A'$.
We have reflexivity rules for $\dyn$ and $\nat$, as well as rules that $\nat$ is more precise than $\dyn$
and $\dyntodyn$ is more precise than $\dyn$.
We also have a congruence rule for function types stating that given $A_i \ltdyn A'_i$ and $A_o \ltdyn A'_o$, we can prove
$A_i \ra A_o \ltdyn A'_i \ra A'_o$. Note that precision is covariant in both the domain and codomain.
Finally, we can lift a relation on value types $A \ltdyn A'$ to a relation $\Ret A \ltdyn \Ret A'$ on
computation types.

\begin{mathpar}
  \inferrule*[right = \dyn]
    { }{\dyn \ltdyn\, \dyn}

  \inferrule*[right = \nat]
    { }{\nat \ltdyn \nat}

  \inferrule*[right = $\ra$]
    {A_i \ltdyn A'_i \and A_o \ltdyn A'_o }
    {(A_i \ra A_o) \ltdyn (A'_i \ra A'_o)}

  \inferrule*[right = $\textsf{Inj}_\nat$]
    { }{\nat \ltdyn\, \dyn}

  \inferrule*[right = $\textsf{Inj}_{\ra}$]
    { }
    {(\dyntodyn) \ltdyn\, \dyn}

  \inferrule*[right = $\injarr{}$]
    {(R \ra S) \ltdyn\, (\dyntodyn)}
    {(R \ra S) \ltdyn\, \dyn}

  
\end{mathpar}

We can prove that transitivity of type precision is admissible, i.e.,
if $A \ltdyn B$ and $B \ltdyn C$, then $A \ltdyn C$.

% Type precision derivations
Note that as a consequence of this presentation of the type precision rules, we
have that if $A \ltdyn A'$, there is a unique precision derivation that witnesses this.
As in previous work, we go a step farther and make these derivations first-class objects,
known as \emph{type precision derivations}.
Specifically, for every $A \ltdyn A'$, we have a derivation $c : A \ltdyn A'$ that is constructed
using the rules above. For instance, there is a derivation $\dyn : \dyn \ltdyn \dyn$, and a derivation
$\nat : \nat \ltdyn \nat$, and if $c_i : A_i \ltdyn A_i$ and $c_o : A_o \ltdyn A'_o$, then
there is a derivation $c_i \ra c_o : (A_i \ra A_o) \ltdyn (A'_i \ra A'_o)$. Likewise for
the remaining rules. The benefit to making these derivations explicit in the syntax is that we
can perform induction over them.
Note also that for any type $A$, we use $A$ to denote the reflexivity derivation that $A \ltdyn A$,
i.e., $A : A \ltdyn A$.
Finally, observe that for type precision derivations $c : A \ltdyn A'$ and $c' : A' \ltdyn A''$, we
can define their composition $c \relcomp c' : A \ltdyn A''$.
This notion will be used below in the statement of transitivity of the term precision relation.

\subsection{Removing Casts as Primitives}

% We now observe that all casts, except those between $\nat$ and $\dyn$
% and between $\dyntodyn$ and $\dyn$, are admissible, in the sense that
% we can start from $\extlcm$, remove casts except the aforementioned ones,
% and in the resulting language we will be able to derive the other casts.

We now observe that all casts, except those between $\nat$ and $\dyn$
and between $\dyntodyn$ and $\dyn$, are admissible.
That is, consider a new language ($\extlcprime$) in which
instead of having arbitrary casts, we have injections from $\nat$ and
$\dyntodyn$ into $\dyn$, and a case inspection on $\dyn$.
We claim that in $\extlcprime$, all of the casts present in $\extlc$ are derivable.
It will suffice to verify that casts for function type are derivable.
This holds because function casts are constructed inductively from the casts
of their domain and codomain. The base case is one of the casts involving $\nat$
or $\dyntodyn$ which are present in $\extlcprime$ as injections and case inspections.


The resulting calculus $\extlcprime$ now lacks arbitrary casts as a primitive notion:

%%%%%%%%%%%%%%%%%%%%%%%%%%%%%%%%%%%%%%%%%%%%%%
% TODO update

\begin{align*}
  &\text{Types } A := \nat \alt \dyn \alt (A \ra A') \\
  &\text{Contexts } \Gamma := \cdot \alt (\Gamma, x : A) \\
  &\text{Values } V :=  \zro \alt \suc\, V \alt \lda{x}{M} \alt \injnat V \alt \injarr V \\ 
  &\text{Producers } M, N := \err_B \alt \ret {V} \alt \bind{x}{M}{N}
    \alt V_f\, V_x \alt
    \\ & \quad\quad \casenat{V}{M_{no}}{n}{M_{yes}} 
    \alt \casearr{V}{M_{no}}{f}{M_{yes}}
\end{align*}


% New rules
Figure \ref{fig:extlc-minus-minus-typing} shows the new typing rules,
and Figure \ref{fig:extlc-minus-minus-eqns} shows the equational rules
for case-nat (the rules for case-arrow are analogous).

\begin{figure}
  \begin{mathpar}
      % inj-nat
      \inferrule*
      {\hasty \Gamma M \nat}
      {\hasty \Gamma {\injnat M} \dyn}

      % inj-arr 
      \inferrule*
      {\hasty \Gamma M (\dyntodyn)}
      {\hasty \Gamma {\injarr M} \dyn}

      % Case dyn
      \inferrule*
      {\hasty{\Delta|_V}{V}{\dyn} \and
        \hasty{\Delta , x : \nat }{M_{nat}}{B} \and 
        \hasty{\Delta , x : (\dyntodyn) }{M_{fun}}{B}
      }
      {\hasty {\Delta} {\casedyn{V}{n}{M_{nat}}{f}{M_{fun}}} {B}}
  \end{mathpar}
  \caption{New typing rules for $\extlcmm$.}
  \label{fig:extlc-minus-minus-typing}
\end{figure}


\begin{figure}
  \begin{mathpar}
     % Case-dyn Beta
     \inferrule*
     {\hasty \Gamma V \nat}
     {\casedyn {\injnat {V}} {n} {M_{nat}} {f} {M_{fun}} = M_{nat}[V/n]}

     \inferrule*
     {\hasty \Gamma V {\dyntodyn} }
     {\casedyn {\injarr {V}} {n} {M_{nat}} {f} {M_{fun}} = M_{fun}[V/f]}

     % Case-dyn Eta
     \inferrule*
     {}
     {\Gamma , x :\, \dyn \vdash M = \casedyn{x}{n}{M[(\injnat{n}) / x]}{f}{M[(\injarr{f}) / x]} }


  \end{mathpar}
  \caption{New equational rules for $\extlcprime$ (rules for case-arrow are analogous
           and hence are omitted).}
  \label{fig:extlc-minus-minus-eqns}
\end{figure}




\section{Term Semantics}\label{sec:term-semantics}

\subsection{Domain-Theoretic Constructions}\label{sec:domain-theory}

In this section, we discuss the fundamental objects of the model into which we will embed
the gradual lambda calculus.
It is important to remember that the constructions in this section are entirely
independent of the syntax described in the previous section; the notions defined 
here exist in their own right as purely mathematical constructs.
In Section \ref{sec:term-interpretation}, we will link the syntax and semantics
via a semantic interpretation function.


\subsection{Modeling the Dynamic Type}

When modeling the dynamic type $\dyn$, we need a semantic object $D$ that satisfies the
isomorphism

\[ D \cong \Nat + (D \to (D + 1)). \]

where the $D + 1$ represents the fact that in the function case, the function may return an error.
Unfortunately, this equation does not have inductive or coinductive solutions. The usual way of
dealing with such equations is via domain theory, by which we can obtain an exact solution.
However, the heavy machinery of domain theory can be difficult for language designers to learn
and apply in the mechanized setting.
Instead, we will leverage the tools of guarded type theory, considering instead the following
similar-looking equation:

\[ D \cong \Nat + \later (D \to (D + 1)). \]

Since the negative occurrence of $D$ is guarded under a later, this equation has a (guarded) solution.
Specifically, we consider the following function $f$ of type
$\later \type \to \type$:

\[ \lambda (D' : \later \type) . \Nat + \later_t (D'_t \to (D'_t + 1)). \]

(Recall that the tick $t : \tick$ is evidence that time has passed, and since
$D'$ has type $\later \type$, i.e. $\tick \to \type$, then $D'_t$ has type $\type$.)

Then we define 

\[ D = \fix f. \]

% TODO explain better
As it turns out, this definition is not quite correct; let us try to understand why.
The price we pay for such a simple solution to the above equation is that we have
introduced a notion of ``time'', i.e., we must now consider the stepping behavior of terms.
% Therefore, the semantics of terms will need to allow for terms that potentially
% do not terminate. This is accomplished using an extension of the guarded lift monad,
% which we describe in the next section.
Thus, in the equation for the semantics of $\dyn$, the result of the function should be
a computation that may return a value, error, \emph{or} take an observable step.
We model such computations using an extension of the so-called guarded lift monad
\cite{mogelberg-paviotti2016} which we describe in the next section.
We will then discuss the correct definition of the semantics of the dynamic type.

\subsubsection{The Lift + Error Monad}

% TODO ensure the previous section flows into this one
When thinking about how to model gradually-typed programs where we track the steps they take,
we should consider their possible behaviors. On the one hand, we have \emph{failure}: a program may fail
at run-time because of a type error. In addition to this, a program may ``think'',
i.e., take a step of computation. If a program thinks forever, then it never returns a value,
so we can think of the idea of thinking as a way of intensionally modelling \emph{partiality}.

With this in mind, we can describe a semantic object that models these behaviors: a monad
for embedding computations that has cases for failure and ``thinking''.
Previous work has studied such a construct in the setting of the latter, called the lift
monad \cite{mogelberg-paviotti2016}; here, we augment it with the additional effect of failure.

For a type $A$, we define the \emph{lift monad with failure} $\li A$, which we will just call
the \emph{lift monad}, as the following datatype:

\begin{align*}
  \li A &:= \\
  &\eta \colon A \to \li A \\
  &\mho \colon \li A \\
  &\theta \colon \later (\li A) \to \li A
\end{align*}

Unless otherwise mentioned, all constructs involving $\later$ or $\fix$
are understood to be with respect to a fixed clock $k$. So for the above, we really have for each
clock $k$ a type $\li^k A$ with respect to that clock.

Formally, the lift monad $\li A$ is defined as the solution to the guarded recursive type equation

\[ \li A \cong A + 1 + \later \li A. \]

An element of $\li A$ should be viewed as a computation that can either (1) return a value (via $\eta$),
(2) raise an error and stop (via $\mho$), or (3) think for a step (via $\theta$).
%
Notice there is a computation $\fix \theta$ of type $\li A$. This represents a computation
that runs forever and never returns a value.

Since we claimed that $\li A$ is a monad, we need to define the monadic operations
and show that they respect the monadic laws. The return is just $\eta$, and extend
is defined via guarded recursion by cases on the input.
% It is instructive to give at least one example of a use of guarded recursion, so
% we show below how to define extend:
% TODO
%
%
Verifying that the monadic laws hold uses \lob-induction and is straightforward.

%\subsubsection{Model-Theoretic Description}
%We can describe the lift monad in the topos of trees model as follows.

\subsubsection{Revisiting the Dynamic Type}
Now we can state the correct definition of the semantics for the dynamic type.
The set $D$ is defined to be the solution to the guarded equation

\[ D \cong \Nat + \later (D \to \textcolor{red}{\li} D). \]


\subsection{Interpretation}\label{sec:term-interpretation}

We can now give a semantics to the gradual lambda calculus we defined
in Section \ref{sec:step-sensitive-lc}.
%
Much of the semantics is similar to a normal call-by-value denotational semantics;
we highlight the differences.
We will interpret types as sets, and terms as functions.
Contexts $\Gamma = x_1 \colon A_1, \dots, x_n \colon A_n$
will be interpreted as the product $\sem{A_1} \times \cdots \times \sem{A_n}$.


The semantics of the dynamic type $\dyn$ is the set $\Dyn$ introduced in Section
\ref{sec:predomains}.
%
The interpretation of a value $\vhasty {\Gamma} V A$ will be a monotone function from
$\sem{\Gamma}$ to $\sem{A}$. Likewise, a term $\phasty {\Gamma} M {{A}}$ will be interpreted
as a monotone function from $\sem{\Gamma}$ to $\li \sem{A}$.

Recall that $\Dyn$ is isomorphic to $\Nat\, + \later (\Dyn \to \li \Dyn)$.
Thus, the semantics of $\injnat{\cdot}$ is simply $\inl$ and the semantics
of $\injarr{\cdot}$ is simply $\inr \circ \nxt$.
The semantics of case inspection on dyn performs a case analysis on the sum.

The interpretation of $\lda{x}{M}$ works as follows. Recall by the typing rule for
lambda that $\phasty {\Gamma, x : A_i} M {{A_o}}$, so the interpretation of $M$
has type $(\sem{\Gamma} \times \sem{A_i})$ to $\li \sem{A_o}$.
The interpretation of lambda is thus a function (in the ambient type theory) that takes
a value $a$ representing the argument and applies it (along with $\gamma$) as argument to
the interpretation of $M$.
%
The interpretation of bind and of application both make use the monadic extend function on $\li A$.
%
The interpretation of case-nat and case-arrow is simply a case inspection on the
interpretation of the scrutinee, which has type $\Dyn$.


\vspace{2ex}


\noindent Types:
\begin{align*}
  \sem{\nat} &= \Nat \\
  \sem{\dyn} &= \Dyn \\
  \sem{A \ra A'} &= \sem{A} \To \li \sem{A'} \\
\end{align*}

% Contexts:

% TODO check these, especially the semantics of bind, case-nat, and case-arr
% with respect to their context argument
\noindent Values and terms:
\begin{align*}
  \sem{\zro}         &= \lambda \gamma . 0 \\
  \sem{\suc\, V}     &= \lambda \gamma . (\sem{V}\, \gamma) + 1 \\
  \sem{x \in \Gamma} &= \lambda \gamma . \gamma(x) \\
  \sem{\lda{x}{M}}   &= \lambda \gamma . \lambda a . \sem{M}\, (*,\, (\gamma , a))  \\
  \sem{\injnat{V_n}} &= \lambda \gamma . \inl\, (\sem{V_n}\, \gamma) \\
  \sem{\injarr{V_f}} &= \lambda \gamma . \inr\, (\sem{V_f}\, \gamma) \\[2ex]
  \sem{\nxt\, V}     &= \lambda \gamma . \nxt (\sem{V}\, \gamma) \\
  \sem{\theta}       &= \lambda \gamma . \theta \\
%
  \sem{\err_B}         &= \lambda \delta . \mho \\
  \sem{\ret\, V}       &= \lambda \gamma . \eta\, \sem{V} \\
  \sem{\bind{x}{M}{N}} &= \lambda \delta . \ext {(\lambda x . \sem{N}\, (\delta, x))} {\sem{M}\, \delta} \\
  \sem{V_f\, V_x}      &= \lambda \gamma . {({(\sem{V_f}\, \gamma)} \, {(\sem{V_x}\, \gamma)})} \\
  \sem{\casedyn{V}{n}{M_{nat}}{\tilde{f}}{M_{fun}}} &=
    \lambda \delta . \text{case $(\sem{V}\, \delta)$ of} \\ 
    &\quad\quad\quad\quad \alt \inl(n) \to \sem{M_{nat}}(n) \\
    &\quad\quad\quad\quad \alt \inr(\tilde{f}) \to \sem{M_{fun}}(\tilde{f})
\end{align*}

\section{Constructing a Concrete Model}\label{sec:concrete-model}

In this section, we build a concrete extensional model of gradual typing.
We begin by defining a step-1 intensional model, and then
apply the abstract constructions outlined in the previous section
to obtain an extensional model.

We begin with some definitions:

\begin{definition}
A \textbf{predomain} $A$ consists of a set $A$ along with two relations:
\begin{itemize}
    \item A partial order $\le_A$.
    \item A reflexive, symmetric ``bisimilarity'' relation $\bisim_A$.
\end{itemize}
\end{definition}

% TODO: later of a predomain
Given a predomain $A$, we can form the predomain $\later A$.
The underlying set is $\later |A|$ and the relation is defined in the obvious way,
i.e., $\tilde{x} \le_{\later A} \tilde{x'}$ iff $\later_t(\tilde{x}_t \le_A \tilde{x'}_t)$.
Likewise for bisimilarity.

We also give a predomain structure to the natural numbers $\mathbb{N}$, where both the
ordering and the bisimiarity relation are equality.

Morphisms of predomains are functions between the underlying sets that preserve the ordering
and the bisimilarity relation. More formally:
%
\begin{definition}
Let $A$ and $A'$ be predomains.
A morphism $f : A \to A'$ is a function between the underlying sets such that for all $x, x'$,
if $x \le_A x'$, then $f(x) \le f(x')$, and if $x \bisim_A x'$, then $f(x) \bisim_{A'} f(x')$.
\end{definition}

\begin{definition}
An \textbf{error domain} $B$ consists of a predomain $B$ along with the following data:
\begin{itemize}
    \item A distinguished ``error" element $\mho_B \in B$
    \item A morphism of predomains $\theta_B \colon \later B \to B$
\end{itemize}
\end{definition}


Morphisms of error domains are morphisms of the underlying predomains that preserve the
algebraic structure. More formally:
%
\begin{definition}
Let $B$ and $B'$ be error domains.
A morphism $\phi : B \wand B'$ is a morphism between the underlying predomains such that:
\begin{enumerate}
    \item $\phi(\mho_B) = \mho_{B'}$
    \item $\phi(\theta_B(\tilde{x})) = \theta_{B'}(\lambda t. \phi(\tilde{x}_t))$
\end{enumerate}
\end{definition}


We define a (monotone) relation on predomains $A$ and $A'$ to be a relation on the
underlying sets that is downward-closed under $\le_A$ and upward-closed under $\le_{A'}$.
More formally:

\begin{definition}
Let $A$ and $A'$ be predomains. A \emph{predomain relation} between $A$ and $A'$
is a relation $R$ between the underlying sets such that:
\begin{enumerate}
    \item (Downward closure): For all $x_1, x_2 \in A$ and $y \in A'$,
    if $x_1 \le_A x_2$ and $x_2 \mathbin{R} y$, then $x_1 \mathbin{R} y$.
    \item (Upward closure): For all $x \in A$ and $y_1, y_2 \in A'$,
    if $x \mathbin{R} y_1$ and $y_1 \le_{A'} y_2$, then $x \mathbin{R} y_2$.
\end{enumerate}
\end{definition}

Composition of relations on predomains is the usual relational composition.

Similarly, we define a (monotone) relation on error domains to be a relation on the
underlying predomains that respects error and preserves $\theta$.
%
\begin{definition}
    Let $B$ and $B'$ be error domains. An \emph{error domain relation} between
    $B$ and $B'$ is a relation $R$ between the underlying predomains such that
    \begin{enumerate}
       \item (Respects error): For all $y \in B'$, we have $\mho_B \mathbin{R} y$.
       \item (Preserves $\theta$): For all $\tilde{x}$ in $\later B$ and $\tilde{y} \in \later B'$,
       if 
       \[ \later_t( \tilde{x}_t \mathbin{R} \tilde{y}_t ), \]
       then
       \[ \theta_B(\tilde{x}) \mathbin{R} \theta_{B'}(\tilde{y}). \]
    \end{enumerate}
\end{definition}


We define composition of error domain relations $R$ on $B_1$ and $B_2$ and $S$
on $B_2$ and $B_3$ to be the least relation containing $R$ and $S$ that respects
error and preserves $\theta$.
Specifically, it is defined inductively by the following rules:

\begin{mathpar}
    \inferrule*[right = Comp]
    {b_1 \mathbin{R} b_2 \and b_2 \mathbin{S} b_3}
    {b_1 \mathbin{R \relcomp S} b_3}

    \inferrule*[right = PresErr]
    { }
    {\mho_{B_1} \mathbin{R \relcomp S} b_3}

    \inferrule*[right = PresTheta]
    {\later_t( \tilde{b_1} \mathbin{R \relcomp S} \tilde{b_3} ) }
    {\theta_{B_1}(\tilde{b_1}) \mathbin{R \relcomp S} \theta_{B_3}(\tilde{b_3}) }
\end{mathpar}

% We note that this composition has the following universal property.


We now describe the ``commuting squares".
Suppose we are given predomains $A_i, A_o, A_i'$, and $A_o'$,
relations $R_i$ and $R_o$, and morphisms $f, f'$ as shown below.

% https://q.uiver.app/#q=WzAsNCxbMCwwLCJBX2kiXSxbMSwwLCJBX2knIl0sWzAsMSwiQV9vIl0sWzEsMSwiQV9vJyJdLFswLDIsImYiLDJdLFsxLDMsImYnIl0sWzAsMSwiUl9pIiwwLHsic3R5bGUiOnsiYm9keSI6eyJuYW1lIjoiYmFycmVkIn0sImhlYWQiOnsibmFtZSI6Im5vbmUifX19XSxbMiwzLCJSX28iLDIseyJzdHlsZSI6eyJib2R5Ijp7Im5hbWUiOiJiYXJyZWQifSwiaGVhZCI6eyJuYW1lIjoibm9uZSJ9fX1dLFs0LDUsIlxcbHRkeW4iLDEseyJzaG9ydGVuIjp7InNvdXJjZSI6MjAsInRhcmdldCI6MjB9LCJzdHlsZSI6eyJib2R5Ijp7Im5hbWUiOiJub25lIn0sImhlYWQiOnsibmFtZSI6Im5vbmUifX19XV0=
\[\begin{tikzcd}[ampersand replacement=\&]
	{A_i} \& {A_i'} \\
	{A_o} \& {A_o'}
	\arrow[""{name=0, anchor=center, inner sep=0}, "f"', from=1-1, to=2-1]
	\arrow[""{name=1, anchor=center, inner sep=0}, "{f'}", from=1-2, to=2-2]
	\arrow["{R_i}", "\shortmid"{marking}, no head, from=1-1, to=1-2]
	\arrow["{R_o}"', "\shortmid"{marking}, no head, from=2-1, to=2-2]
	\arrow["\ltdyn"{description}, draw=none, from=0, to=1]
\end{tikzcd}\]

We say that the above square commutes, written $f \le f'$, if for all
$x \in A_i$ and $x' \in A_i'$ with $x \mathbin{R_i} x'$, we have
$f(x) \mathbin{R_o} f'(x')$.

We make the analogous definition for error domains.



\subsection{Guarded Lift Monad}\label{sec:guarded-lift-monad}

% Lift monad
The guarded error-lift monad $\li$ takes a predomain $A$ to the error domain $\li A$.
It is defined as follows:

\begin{align*}
  \li A &:= \\
  &\eta \colon A \to \li A \\
  &\mho \colon \li A \\
  &\theta \colon \later (\li A) \to \li A
\end{align*}

Formally, the lift monad $\li A$ is defined as the solution to the guarded recursive type equation

\[ \li A \cong A + 1 + \later \li A. \]

This captures the intuition that a program may either return a value,
fail at run-time, or take one or more observable steps of computation.
Previous work has studied such a similar construct, called the guarded lift
monad \cite{mogelberg-paviotti2016}; our version here our version augments it with the notion of error.

Since we claimed that $\li A$ is a monad, we need to define the monadic operations
and show that they respect the monadic laws. The return is just $\eta$, and the monadic extend
is defined via guarded recursion by cases on the input.
Verifying that the monadic laws hold uses \lob-induction and is straightforward.

% \eric{Check}
% We can also show that $\li A$ is the free error- and -later algebra on $A$, in that
% for any morphism of predomains $f : A \to UB$, there is a unique morphism of error domains
% $f^* : \li A \to \li A$ extending $f$.

There is a functor $U$ from error domains to predomains that on objects simply returns the
underlying predomain, and on morphisms returns the underlying morphism of predomains.
%
% TODO check this

It is easily verified that $\li A$ is the free error- and later-algebra on the predomain $A$,
so we have that $\li$ is left-adjoint to $U$.

% i.e., error domain morphisms from $\li A$ to $B$ are in one-to-one correspondence with
% predomain morphisms from $A$ to $UB$.

% We define $\delta : U(\li A) \To U(\li A)$ by $\delta(x) = \theta(\nxt x)$.
% We define $\hat{\delta} : \li A \arr \li A$ by $\hat{\delta} = \ext{(\delta \circ \eta)}{}$.
% Note that by definition of $\text{ext}$, we have that $\hat{\delta}$ is a morphism of error domains.

%\subsubsection{Lock-Step Error Ordering}\label{sec:lock-step}

The partial order $\le_{\li A}$ is the lock-step error ordering defined by guarded recursion as follows:
%
\begin{itemize}
    \item 	$\eta\, x \le_{\li A} \eta\, y$ if $x \le_A y$.
    \item 	$\mho \le_{\li A} l$ for all $l$ 
    \item   $\theta\, \tilde{r} \le_{\li A} \theta\, \tilde{r'}$ if
            $\later_t (\tilde{r}_t \le_{\li A} \tilde{r'}_t)$
  \end{itemize}
%
The idea is that two computations $l$ and $l'$ are related if they are in
lock-step with regard to their intensional behavior, up to $l$ erroring.

Given a relation $R : A \rel A'$, we define in an analogous manner a heterogeneous
version of the lock-step error ordering between $\li R : \li A \rel \li A'$.
% We define the action of $\li$ on a relation $R$ between $A$ and $A'$ to be the
% ``heterogeneous" version of the lock-step error ordering.

% TODO action of \li on commuting squares

%\subsubsection{Weak Bisimilarity}\label{sec:weak-bisimilarity}

For a predomain $A$, we define a relation on $\li A$, called ``weak bisimilarity",
written $l \bisim l'$. Intuitively, we say $l \bisim l'$ if they are equivalent ``up to delay''.
The weak bisimilarity relation is defined by guarded recursion as follows:
%
\begin{align*}
  &\mho \bisim \mho \\
%
  &\eta\, x \bisim \eta\, y \text{ if } 
    x \bisim_A y \\
%		
  &\theta\, \tilde{x} \bisim \theta\, \tilde{y} \text{ if } 
    \later_t (\tilde{x}_t \bisim \tilde{y}_t) \\
%	
  &\theta\, \tilde{x} \bisim \mho \text{ if } 
    \theta\, \tilde{x} = \delta^n(\mho) \text { for some $n$ } \\
%	
  &\theta\, \tilde{x} \bisim \eta\, y \text{ if }
    (\theta\, \tilde{x} = \delta^n(\eta\, x))
  \text { for some $n$ and $x : \ty{A}$ such that $x \bisim_A y$ } \\
%
  &\mho \bisim \theta\, \tilde{y} \text { if } 
    \theta\, \tilde{y} = \delta^n(\mho) \text { for some $n$ } \\
%	
  &\eta\, x \bisim \theta\, \tilde{y} \text { if }
    (\theta\, \tilde{y} = \delta^n (\eta\, y))
  \text { for some $n$ and $y : \ty{A}$ such that $x \bisim_A y$ }
\end{align*}
%
When both sides are $\eta$, then we ensure that the underlying values are related
by the bisimilarity relation on $A$.
When one side is a $\theta$ and the other is $\eta x$ (i.e., one side steps),
we stipulate that the $\theta$-term runs to $\eta y$ where $x$ is bisimilar to $y$.
Similarly when one side is $\theta$ and the other $\mho$.
If both sides step, then we allow one time step to pass and compare the resulting terms.
In this way, the definition captures the intuition of terms being equivalent up to
delays.

It can be shown (by \lob-induction) that the step-sensitive relation is symmetric.
However, it can also be shown that this relation is \emph{not} transitive:
The argument is the same as that used to show that the step-insensitive error
ordering $\semltbad$ described above is not transitive. Namely, we show that
if it were transitive, then it would have to be trivial in that $l \bisim l'$ for all $l, l'$.
that if this relation were transitive, then it would relate all values of type $\li A$.


% internal hom for predomains and error domains
Given predomains $A$ and $A'$, we can form the predomain of
predomain morphisms from $A$ to $A'$, denoted $A \To A'$.
\begin{itemize}
    % Should we give the definition involving x and x'?
    \item The ordering is defined by $f \le_{A \To A'} f'$ iff for all
    $x \in A$, we have $f(x) \le_{A'} f'(x)$.
    \item The bisimilarity relation is defined by $f \bisim_{A \To A'} f'$ iff
    for all $x, x' \in A$ with $x \bisim_{A} x'$, we have $f(x) \bisim_{A'} f'(x')$. 
\end{itemize}

Given $f : A_1' \to A_1$ and $g : A_2 \to A_2'$ we define the predomain morphism
$f \To g : (A_1 \To A_2) \to (A_1' \To A_2')$ by $\lambda h. \lambda x'. g(h(f(x')))$.

% TODO: include this?
% The monadic extension operation $\ext{\cdot}{} : (A_1 \To U (\li A_2)) \To (\li A_1 \To U(\li A_2))$
% is a morphism of predomains from $A_1 \To U(\li A_2)$ to $U(\li A_1) \To U(\li A_2)$, i.e.,
% it preserves the ordering and bisimilarity relations.


% Given a predomain $A$ and error domain $B$, we define $A \arr B := A \To UB$.
We note that $A \To UB$ carries a natural error domain structure
(in the below, the lambda is a meta-theoretic notation):
\begin{itemize}
    \item The error is given by $\lambda x . \mho_B$
    \item The $\theta$ operation is defined by
      \[ \theta_{A \To UB}(\tilde{f}) = \lambda x . \theta_B(\lambda t . \tilde{f}_t(x)). \]
\end{itemize}

Given a predomain $A$ and error domain $B$, we define
$A \arr B$ to be the error domain such that $U(A \arr B) = A \To UB$,
and whose error and $\theta$ operations are as defined above.
We can define the functorial action of $\arr$ on morphisms
$f \arr \phi$ in the obvious way.

It is easily verified that $A \arr B$ is an exponential of $UB$ by $A$
in the category of predomains and their morphisms.

Lastly, given a relation of predomains $R$ between $A$ and $A'$, and a relation
of error domains $S$ between $B$ and $B'$, we define the relation $R \arr S$
between $A \arr B$ and $A' \arr B'$ in the obvious way, i.e., $f \in A \arr B$
is related to $g \in A' \arr B'$ iff for all $x \in A$ and $x' \in A'$ with
$x \mathrel{R} x'$, we have $f(x) \mathrel{S} g(x')$.
%
One can verify that this relation is indeed a relation of error domains
in that it respects error and preserves $\theta$.

With all of the above data, we can form a step-1 intensional model of gradual typing
(See Definition \ref{def:step-1-model}).

\subsection{The Dynamic Type}

The predomain representing the dynamic type will be defined using guarded recursion
as the solution to the equation

\[ D \cong \mathbb{N}\, + (D \times D)\, + \laterhs U(D \arr FD). \]

% Note that the operators in the above equation are all combinators for predomains, so
% this also defines the ordering and the bisimilarity relation for $D$.

For the sake of clarity, we name the ``constructors" $\text{nat}$, $\text{times}$,
and $\text{fun}$, respectively.

We define $e_\mathbb{N} : \mathbb{N} \to D$ to be the injection into the first
component of the sum, and $e_\times : D \times D \to D$ to be the injection into
the second component of the sum, and $e_\to : U(D \arr F D)$ to be the morphism
$\nxt$ followed by the injection into the third component of the sum.

Explicitly, the ordering on $D$ is given by:

\begin{align*}
    \tnat(n) \le \tnat(n') 
        &\iff n = n' \\
    \ttimes (d_1, d_2) \le \ttimes (d_1', d_2')
        &\iff d_1 \le d_2 \text{ and } d_1' \le d_2'\\
    \tfun(\tilde{f}) \le \tfun(\tilde{f'}) 
        &\iff \later_t(\tilde{f}_t \le \tilde{f'}_t)
\end{align*}

We define a relation $\inat : \mathbb{N} \rel D$ by
$(n, d) \in \inat$ iff $e_\mathbb{N} \le_D d$.
We similarly define $\itimes : D \times D \rel D$ by
$((d_1, d_2), d) \in \itimes$ iff $e_\times(d_1, d_2) \le_D d$,
and we define $\text{inj}_\to : U(D \arr F D) \rel D$ by
$(f, d) \in \iarr$ iff $e_\to(f) \le_D d$.

Now we define the perturbations for $D$.
Recall from our construction of a model with perturbations
(Section \ref{sec:constructing-perturbations}) that for each value
type $A$ we associate a monoid $P_A$ of perturbations
and a homomorphism into the monoid of endomorphisms bisimilar to the identity,
and likewise for computation types.

We define the perturbations for $D$ via least-fixpoint in the category of monoids as

\[ P_D \cong (P_{D \times D}) \times P_{U(D \to FD)}. \]

Unfolding these definitions explicitly, this is

\[ P_D \cong (P_D \times P_D) \times (\mathbb{N} \times P_D^{op} \times \mathbb{N} \times P_D). \]

We now explain how to interpret these perturbations as endomorphisms.
We define $\ptb_D : P_D \to \{ f : D \to D \mid f \bisim \id \}$ below,
% via the universal property of the coproduct of monoids, giving a case for each of the generators.
% In the below, note the use of the functorial action of $\arr$ on morphisms.

%
% \[ P_D \cong (P_D \times P_D) \times (P_D \times (\mathbb{N} \times P_D)), \]
%

$\ptb_D((p_{\text{times}}), p_{\text{fun}}) = \lambda d.\text{case $d$ of}$
\begin{align*}
    &\alt \tnat(m) \mapsto \tnat(m) \\
    &\alt \ttimes(d_1, d_2) \mapsto {\ttimes(\ptb_{D \times D}(p_\text{times})(d_1, d_2))} \\
    &\alt \tfun(\tilde{f}) \mapsto {\tfun(\lambda t. \ptb_{U(D \to FD)}()(\tilde{f}_t))} \\
\end{align*}

    % \item $\ptb_D(1)$
    % \item $\ptb_D(\delta^K_D)$ is defined similarly to the previous but has
    % \[ \id \arr i^K(\delta^K_D) \] instead.
    % \item $\ptb_D(\delta^K_D) = \lambda d.\text{case $d$ of}$
    %   \begin{align*} 
    %     &\alt \ttimes(d_1, d_2) \mapsto {\ttimes(i^K(\delta^K_D)(d_1), d_2)} \\
    %     &\alt d' \to d'
    %   \end{align*}
    % \item $\ptb_D(\delta^K_D)$ is defined similarly to the previous but has 
    % \[ (d_1, \ptb_D(\delta^K_D)(d_2)) \] instead.

One can verify that this forms a homomorphism from $P_D \to \{ f : D \to D : f \bisim \id \}$.

We claim that the three relations $\inat$, $\itimes$, and $\iarr$
%and their lifted versions 
satisfy the push-pull property.
As an illustrative case, we establish the push-pull property for the relation $\iarr$.

We define $\pull_{\iarr} : P_D \to P_{U(D \arr FD)}$ by

\[ \pull_{\iarr}(p_{\text{times}}, p_{\text{fun}}) = p_{\text{fun}}, \]
% (recall that $P_{U(D \arr FD)} = \mathbb{N} \times P_D^{op} \times \mathbb{N} \times P_D$).

i.e., we simply forget the other perturbation.

We define $\push_{\iarr} : P_{U(D \arr FD)} \to P_D$ by

\[ \push{\iarr}(p_{\text{fun}}) = (\id, p_{\text{fun}}), \]

Showing that the relevant squares commute is straightforward.

% To do so, consider an arbitrary perturbation $p$ on $D$. 
% Let $(f, d) \in \iarr$. This means that $d$ must be of the form $\tfun{\tilde{f}}$. 
% When we interpret the perturbation on $D$, obtaining and endomorphism to which we then apply $d$,
% we will be perform the action that was performed by the perturbation on the other side and thus
% we will be done by our assumption that $f$ is related to $d$.


We next claim that the relations $\inat$, $\itimes$, and $\iarr$ are quasi-left-representable,
and that their lifts are quasi-right-representable.
Indeed, since the relations are functional, it is easy to see that they are quasi-left-representable
where the perturbations are taken to be the identity.

For quasi-right-representability, the most interesting case is $\li(\iarr)$.
Defining the projection $p_{\iarr} : FD \to FU(D \to FD)$ is equivalent to defining
$p' : D \to UFU(D \to FD)$. We define

$p' = \lambda d.\text{case $d$ of}$
\begin{align*}
    &\alt \tnat(m) \mapsto \mho \\
    &\alt \ttimes(d_1, d_2) \mapsto \mho \\
    &\alt \tfun(\tilde{f}) \mapsto \theta (\lambda t. \eta(\tilde{f}_t)).
\end{align*}

We define $\dellp_D = \delrp_D = \theta \circ \nxt$.
Then it is easy to show that the squares for $\dnl$ and $\dnr$ commute.
% Then for $\dnr$, we need to show that if $(f, d) \in \iarr$ then

% TODO retraction property
It is also straightforward to establish the retraction property for
each of these three relations. In the case of $\iarr$, we will
have that the property holds up-to a delay $\theta \circ \nxt$.


\begin{comment}
Before defining the perturbations, recall from our concrete construction
(Section \ref{TODO}) that to each value type $A$ we associate a monoid $P_A$
of pure perturbations and a monoid $P^K_{A}$ of Kleisli perturbations.

We define simultaneously the pure and Kleisli perturbation monoids for $D$,
denoted $P_D$ and $P^K_D$ respectively.
We define $P_D$ to be the monoid such that

\[ P_D \cong ((P^K_D)^{op} \times P_D) \oplus (P_D \times P_D), \]

and we define $P^K_D$ to be the monoid such that

\[ P^K_D \cong \mathbb{N} \oplus ((P_D)^{op} \oplus P^K_D) \oplus (P^K_D \oplus P^K_D), \]

where we use $\mathbb{N}$ because it is the free monoid on one generator.

Now we explain how to interpret these monoids as submonoids of endomorphisms.
We define $i : P_D \to \{ f : D \to D \mid f \bisim \id \}$
and $i^K : P^K_D \to \{ \phi : FD \to FD \mid \phi \bisim \id \}$ mutually
via the universal property of the coproduct of monoids, giving a case for each of the generators.
In the below, note the use of the functorial action of $\arr$ on morphisms.

\begin{itemize}
    \item $i(\delta^K_D, \delta_D) = \lambda d.\text{case $d$ of}$
    \begin{align*}
        &\alt \tfun(\tilde{f}) \mapsto 
          \tfun(\lambda t. (i^K(\delta^K_D) \to i(\delta_D))(\tilde{f}_t)) \\
        &\alt d' \to d'
    \end{align*}

    \item $i(\delta_D^1, \delta_D^2) = \lambda d.\text{case $d$ of}$ 
    \begin{align*}
        &\alt \ttimes(d_1, d_2) \mapsto
           \ttimes(i(\delta_D^1)(d_1), i(\delta_D^2)(d_2)) \\
        &\alt d' \to d'
    \end{align*}
\end{itemize}

We define $i^K(\delta^D)$ by
%
%  P^K_D \cong \mathbb{N} \oplus ((P_D)^{op} \oplus P^K_D) \oplus (P^K_D \oplus P^K_D)
%
\begin{itemize}
    \item $i^K(1) = \hat{\delta}$
    \item $i^K(\delta_D) = F(\lambda d.\text{case $d$ of}$
      \begin{align*} 
        &\alt \tfun(\tilde{f}) \mapsto {\tfun(\lambda t. (i(\delta_D) \arr \id)(\tilde{f}_t))} \\
        &\alt d' \to d')
      \end{align*}
    \item $i^K(\delta^K_D)$ is defined similarly to the previous but has
    \[ \id \arr i^K(\delta^K_D) \] instead.
    \item $i^K(\delta^K_D) = F(\lambda d.\text{case $d$ of}$
      \begin{align*} 
        &\alt \ttimes(d_1, d_2) \mapsto {\ttimes(i^K(\delta^K_D)(d_1), d_2)} \\
        &\alt d' \to d')
      \end{align*}
    \item $i^K(\delta^K_D)$ is defined similarly to the previous but has 
    \[ (d_1, i^K(\delta^K_D)(d_2)) \] instead.
\end{itemize}


% perturbations and quasi-representability
We claim that the three relations $\inat$, $\itimes$, and $\iarr$ and their lifted
versions satisfy the push-pull property.
As an illustrative case, we establish the pull property for the $\li \iarr$.
We define $\pull : P^K_D \to P^K_{U(D \arr FD)}$ by cases as

\begin{align*}
 \pull(1) = \hat{\delta}
\end{align*}

We need to check that the relevant squares commute for all $\delta \in P^K_D$.
It suffices by the universal property of the coproduct of monoids to ensure that
this holds for the generators.

We claim that the relations $\inat$, $\itimes$, and $\iarr$ are quasi-left-representable,
and that their lifts are quasi-right-representable.
Indeed, since the relations are functional, it is easy to see that they are left-representable
where the perturbations are simply the identity.

The most interesting case is the right-quasi-representability of $\li(\iarr)$.
We define the projection as follows:

\begin{align*}
\end{align*}

And for the perturbations, we take




% The perturbation monoid for $D$ is defined inductively as the free monoid $M$ equipped
% with an operation $\To : M \times M \to M$, i.e., we take the set defined inductively by

% \begin{mathpar}
%     \inferrule*[]
%     {}
%     {e \in M}

%     \inferrule*[]
%     {p \in M \and p' \in M}
%     {p \odot p' \in M}

%     \inferrule*[]
%     {p \in M \and p' \in M}
%     {p \To p' \in M}
% \end{mathpar}

% and quotient by the necessary monoid equations.

\end{comment}


% \subsection{Obtaining an Extensional Model}

Now that we have defined an intensional model with an interpretation for the dynamic type, we can apply
the abstract constructions introduced in Section \ref{sec:extensional-model-construction}.
Doing so, we obtain an extensional model of gradual typing, where the squares are given by the
``bisimilarity closure'' of the intensional error ordering.

\subsection{Adequacy}\label{sec:adequacy}

In this section, we prove an adequacy result for the concrete extensional model of GTT we obtained in the previous section.
applying the abstract constructions introduced in Section
\ref{sec:extensional-model-construction} to the concrete model built in the previous section.
%\ref{sec:concrete-model}.

First we establish some notation. Fix a morphism $f : 1 \to \li \mathbb{N} \cong \li \mathbb{N}$.
We write that $f \da n$ to mean that there exists $m$ such that $f = \delta^m(\eta n)$
and $f \da \mho$ to mean that there exists $m$ such that $f = \delta^m(\mho)$.

Recall that $\ltls$ denotes the relation on value morphisms defined as the bisimilarity-closure
of the intensional error-ordering on morphisms.
That is, we have $f \ltls g$ iff there exists $f'$ and $g'$ with
%
\[ f \bisim f' \le g' \bisim g. \]
%
The result we would like to show is as follows:
\begin{lemma}
If $f \ltls g : \li \mathbb{N}$, then:
\begin{itemize}
  \item If $f \da n$ then $g \da n$.
  \item If $g \da \mho$ then $f \da \mho$.
  \item If $g \da n$ then $f \da n$.
\end{itemize}
\end{lemma}
%
Unfortunately, this is actually not provable!
Roughly speaking, the issue is that this is a ``global'' result, and it is not possible
to prove such results inside of the guarded setting. 
In particular, if we tried to prove the above result in the guarded
setting, we would run into a problem where we would have a natural number
``stuck'' under a $\later$, with no way to get out the underlying number.

Thus, to prove our adequacy result, we need to leave the guarded setting and pass back
to the more familiar, set-theoretic world with no internal notion of step-indexing.
As mentioned in the Technical Background section (Section \ref{sec:sgdt}), we can do this
using \emph{clock quantification}.

Recall that all of the constructions we have made in SGDT take place in the context of a clock $k$.
All of our uses of the later modality and guarded recursion have taken place with respect to this clock.
For example, recall the definition of the lift monad by guarded recursion.
% We define the lift monad $\li^k X$ as the guarded fixpoint of $\lambda \tilde{T}. X + 1 + \later^k_t (\tilde{T}_t)$.
We can view this definition as being parameterized by a clock $k$: $\li^k : \type \to \type$.
Then for $X$ satisfying a certain technical requirement known as \emph{clock-irrelevance},
\footnote{A type $X$ is clock-irrelevant if there is an isomorphism $\forall k.X \cong X$.}
we can define the ``global lift'' monad as $\li^{gl} X := \forall k. \li^k X$.

It can be shown that there is an isomorphism between the global lift monad and the
delay monad of Capretta \cite{lmcs:2265}.
Recall that, given a type $X$, the delay monad $\text{Delay}(X)$ is defined as the coinductive
type generated by 
$\tnow : X \to \delay(X)$ and $\tlater : \delay(X) \to \delay(X)$.

% solution to the equation

% \[ \text{Delay}(X) \cong X + \text{Delay}(X). \]

It can be shown that for a clock-irrelevant type $X$, $\li^{gl} X$ is a final
coalgebra of the functor $F(Y) = X + 1 + Y$ (For example, this follows from Theorem 4.3 in
\cite{kristensen-mogelberg-vezzosi2022}.) 
\footnote{The proof relies on the existence of an operation  
$\mathsf{force} : \forall k. \later^k A \to \forall k. X$ that
allows us to eliminate the later operator under a clock quantifier.
This must be added as an axiom in guarded type theory.}
Since $\delay(X + 1)$ is also a final coalgebra
of this functor, then we have $\li^{gl} X \cong \delay(X + 1)$.

Given a predomain $X$ on a clock-irrelevant type, we can define a
``global'' version of the lock-step error ordering and the
weak bisimilarity relation on elements of the global lift; the former is defined by
%
\[ x \le^{gl}_X y := \forall k. x[k] \le y[k], \]
%
and the latter is defined by
%
\[ x \bisim^{gl}_X y := \forall k. x[k] \bisim y[k]. \]
%
On the other hand, we can define coinductively a ``lock-step error ordering"
relation on $\delay(X + 1)$:
%
\begin{mathpar}
  \inferrule*[]
  { }
  {\tnow (\inr\, 1) \ledelay d}

  \inferrule*[]
  {x_1 \le_X x_2}
  {\tnow (\inl\, x_1) \ledelay \tnow (\inl\, x_2)}

  \inferrule*[]
  {d_1 \ledelay d_2}
  {\tlater\, d_1 \ledelay \tlater\, d_2}
\end{mathpar}
%
And we similarly define by coinduction a ``weak bisimilarity'' relation on $\delay(X + 1)$, which uses
a relation $d \Da x_?$ between $\delay(X+1)$ and $X+1$ that is defined as 
$d \Da x_? := \Sigma_{m \in \mathbb{N}} x = \tlater^m(\tnow\, x_?)$.
Then weak bisimilarity is defined by the rules
%
\begin{mathpar}
  \inferrule*[]
  {x_? \bisim_{X + 1} y_?}
  {\tnow\, x_? \bisimdelay \tnow\, y_? }

  \inferrule*[]
  {d_1 \Da x_? \and x_? \bisim_{X + 1} y'_?}
  {\tlater\, d_1 \bisimdelay \tnow\, y'_? }

  \inferrule*[]
  {d_2 \Da y_? \and x_? \bisim_{X + 1} y'_?}
  {\tnow\, x_? \bisimdelay \tlater\, d_2}

  \inferrule*[]
  {d_1 \bisimdelay d_2}
  {\tlater\, d_1 \bisimdelay \tlater\, d_2 }

  % \inferrule*[]
  % {d_1 \Da x_? \and d_2 \Da y_? \and x_? \bisim_{X + 1} y_?}
  % {d_1 \bisimdelay d_2}

  % \inferrule*[]
  % {d_1 \bisimdelay d_2}
  % {\tlater d_1 \bisimdelay \tlater d_2 }

\end{mathpar}
%
Note the similarity of these definitions to the corresponding guarded definitions.
By adapting the aforementioned theorem to the setting of inductively-defined relations,
we can show that both the global lock-step error ordering and the global weak bisimilarity
admit coinductive definitions. In particular, modulo the above isomorphism
between $\li^{gl} X$ and $\delay(X+1)$, the global version of the lock-step
error ordering is equivalent to the lock-step error ordering on $\delay(X + 1)$,
and likewise, the global version of the weak bisimilarity relation is equivalent to the
weak bisimilarity relation on $\delay(X + 1)$.

This implies that the global version of the extensional term precision semantics for
$\li^{gl} X$ agrees with the corresponding notion for $\delay(X + 1)$.
Then adequacy follows by proving the corresponding
result for $\delay(X + 1)$ which in turn follows from the definitions of the relations.


% We have been writing the type as $\li X$, but it is perhaps more accurate to write it as $\li^k X$ to
% emphasize that the construction is parameterized by a clock $k$.

% Need : nat is clock irrelevant, as well as the inputs and outputs of effects
% Axioms about forcing clock
% Adapt prior argument to get that the defining of the global bisim
% and global lock-step error ordering are coinductive

\section{Discussion}

\subsection{Related Work}
% Discuss Joey Eremondi's thesis on gradual dependent types

% Discuss Jeremy Siek's work on graduality in Agda

\subsection{Mechanization}
% Discuss Guarded Cubical Agda and mechanization efforts
We will use Guarded Cubical Agda to prove graduality in the syntax of 
GTLC, which involves the construction of the abstract model described in 
\ref{sec:concrete-model} and the extensional model with external dynamic 
type. We also plan to formalize the adequacy result in \ref{sec:appendix-adequacy}.

% step-1
Currently, we have defined the step-1 intensional model in Cubical 
Agda. For value types, we constructed predomain which holds ordering and 
bisimilarity, and morphisms between predmains. For computation types, we 
constructed error domain which contains its underlying predomain, 
elements of $\mho$ and $\theta$ and morphisms between error domains. We 
also defined two functors $U$ and $F$ which support conversions between 
predomains and error domains.

% step-2 
Then we plan to construct the step-2 intensional model. Besides all the 
data in step-1, we need to include perturbations, functors $\times$, $\arr$, $U$, and $F$ that preserve 
perturbations and push/pull properties for all morphisms on value and 
computation types. Notice that for any object $A$ which has value type, 
we will take not only the monoid of perturbations $P^V_A$ and the monoid 
homomorphism $\ptbv_A : \pv_A \to \vf(A,A)$ on itself, but also $P^C_{F A}
$ and $\ptbe_{F A} : \pe_{F A} \to \ef(F A,F A)$ on $F A$, which have 
computation types. Similarly, for any computation object $B$, we will 
construct the perturbations on $U B$ besides the monoid $P^C_B$ and 
monoid homomorphism $\ptbe_B : \pe_B \to \ef(B,B)$. Also, for functors 
that preserves perturbations, we need to include the ones in the context 
of Kleisli category. For this part, we need to define the perturbation on 
not only the objects itself, but also the global lift and delay of objects, 
which requires us to provide each piece of supporting constructor. This step 
and futher steps towards to the model construction are still 
work-in-progress, but once it's finished, we will provide a complete 
framework which takes formalization on an explicit type and obtains an 
extensional model.

% step-3
In the step-3 intensional model, we will enhance it with 
quasi-representability. For any value relation $c : A \rel A'$, we need 
to show that there exists a left-representation structure for $c$ and a 
right-representation structure for $F\ c$. Correspondingly, for any 
computation relation $d : B \rel B'$, we will show there exists a 
right-representation structure for $d$ and a left-representation 
structure for $U\ d$. As we define the quasi-representability for value 
and computation relation, we will construct the quasi-representability on 
the function and product of the relation, which makes it necessary to 
have the dual version of quasi-representability.

% step-4 construct a concrete dynamic type and apply it to the abstract model
After defining the abstract model and its interface, we will model GTLC 
by providing explicit construction triples of dynamic type at each step, 
which includes defining Dyn as a predomain, its pure and Kleisli 
perturbation monoids, push/pull property for pure and Kleisli 
perturbation, as well as quasi-representability. The 
quasi-representability involves explicit rules which show that Nat is 
more precise than Dyn (Inj-Nat) and Dyn $\to$ Dyn is more precise than 
Dyn (Inj-Arr). Currently, we have formalized the concrete construction of 
Dyn in Cubical Agda and it was more challenging than expected because we 
define Dyn using the technique of guarded recursion and fixed point, which 
means that every time we analyze the case inside of Dyn, we need to unfold 
it and add corresponding proof. 

% adequacy
Besides the abstract model and its concrete construction on dynamic type, 
we will also formalize the adequacy result in \ref{sec:appendix-adequacy}, 
which involves clock quantification of the lift monad, the weak bisim 
relation, and the lock-step error ordering. In order to prove adequacy, 
we will first prove that the global lift of X is isomorphic to Delay(1 + X)
whether X is clock-irrelevant or not. Then, we aim to prove the equivalence 
between the global lock-step error ordering and the error ordering observed 
in Delay(1 + X) and equivalence between the global weak bisimilarity 
relation and the weak bisimilarity relation on Delay(1 + X). We have 
finished some prerequisite proofs on clock quantification and postulated 
some theorems on clock globalization.

\subsection{Benefits and Drawbacks of the Synthetic Approach}

\subsection{Synthetic Ordering}

A key to managing the complexity of our concrete construction is in
using a \emph{synthetic} approach to step-indexing rather than working
analytically with presheaves. This has helped immensely in our ongoing
mechanization in cubical Agda as it sidesteps the need to formalize
these constructions internally. 
%
However, there are other aspects of the model, the bisimilarity and
the monotonicity, which are treated analytically and are .
%
It may be possible to utilize further synthetic techniques to reduce
this burden as well, and have all type intrinsically carry a notion of
bisimilarity and ordering relation, and all constructions to
automatically preserve them.
%
A synthetic approach to ordering is common in (non-guarded) synthetic
domain theory and has also been used for synthetic reasoning for cost
models \cite{synthetic-domain-theory,decalf}.

\subsection{Future Work}


% Cite GrEff paper



\bibliographystyle{ACM-Reference-Format}
\bibliography{references}

\appendix
\section{Call-by-push-value}

In CBPV models, all the type constructors are interpreted as functors:
\begin{enumerate}
\item $\to : \op\calV \times \calE \to \calE$
\item $\times : \calV \times \calV \to \calV$
\item $F : \calV \to \calE$
\item $U : \calE \to \calV$
\end{enumerate}
That is, they all have functorial actions on \emph{pure} morphisms of
value types and \emph{linear} morphisms of computation types.
%
We use these functorial actions extensively in the construction of
casts and their corresponding perturbations. But when defining
downcasts of value types and upcasts of computation types, we
additionally need a second functorial action of these categories:
functoriality in \emph{impure} morphisms of value types and
\emph{non-linear} morphisms of computation types. These notions of
morphism are given by the \emph{Kleisli} categories $\calVk$ and
$\calEk$ which have value types and computation types as objects but
morphisms are defined as
\[ \calVk(A,A') = \calE(F A, FA')\]
\[ \calEk(B,B') = \calV(U B, U B')\]
with composition given by composition in $\calE/\calV$.  That is we
need to define a second functorial action, that agrees with the above
on objects for these Kleisli categories:
\begin{enumerate}
\item $\tok : \op\calVk \square \calEk \to \calEk$
\item $\timesk : \calVk \square \calVk \to \calVk$
\item $\Fk : \calVk \to \calEk$
\item $\Uk : \calEk \to \calVk$
\end{enumerate}
Note that rather than the product of categories we use the ``funny
tensor product'' $\square$. This is because the action on
impure/non-linear morphisms for $\tok/\timesk$ do not satisfy ``joint
functoriality'' but instead only ``separate functoriality'', meaning
we give rather than an action on morphisms in both categories
simultaneously instead an action on each argument categories morphisms
with the object in the other category fixed. The existence of these
functorial actions for $\tok$ and $\timesk$ is reliant on the
\emph{strength} of the adjunction. We describe them using the internal
language of CBPV in order to more easily verify their
existence/functoriality:
\begin{enumerate}
\item For $\tok$ we define for $\phi : \calE(F A,F A')$ and $B \in \calE$ the morphism $\phi \tok B : \calV(U(A' \to B),U(A\to B))$ as
  \[ t:U(A'\to B) \vdash \phi \tok B = \{ \lambda x. x' \leftarrow \phi\,[\ret x]; ! t x'\} : U(A \to B) \]
  and for $A \in \calV$ and $f : \calV(UB,UB')$ we define $A \tok f : \calV(U(A \to B),U(A\to B'))$ as
  \[ t : U(A \to B) \vdash A \tok f = \{ \lambda x. !f[\{ ! t x \}]\} \]
\item For $\timesk$ we define for $\phi : \calE(F A_1,FA_2)$ and $A' \in \calV$ the morphism $\phi \timesk A_2$ as
  \[ \bullet : F(A_1\times A_2) \vdash \phi \timesk A_2 = (x_1,x_2) \leftarrow \bullet; x_1' \leftarrow \phi[\ret x_1]; \ret (x_1',x_2) : F(A_1'\times A_2)\]
  and $A_1 \timesk \phi$ is defined symmetrically.
\item For $\Uk$ we need to define for $f : \calV(UB,UB')$ a morphism $\Uk f : \calE(FUB,FUB')$. This is simply given by the functorial action of $F$: $\Uk f = F(f)$
\item Similarly $\Fk \phi = U\phi$
\end{enumerate}

Functoriality in each argument is easily established, meaning for
example for the function type is functorial in each argument:
\begin{enumerate}
\item $(\phi \circ \phi') \tok B = (\phi' \tok B) \circ (\phi \tok B)$
\item $\id \tok B = \id$
\item $A \tok (f \circ f') = (A \tok f) \circ (A \tok f)$
\item $A \tok \id = \id$
\end{enumerate}

Finally, note that all of these constructions lift to squares in a
double CBPV model since the squares themselves form a CBPV model and
the projection functions preserve CBPV structure. For instance, given a square
$\alpha : \phi \ltdyn_{F c_o}^{F c_i} \phi'$ and a horizontal morphism $d : B \rel B'$ of appropriate type, we get a square
\[ \alpha \tok d : \phi \tok B \ltdyn_{U(c_o \to d)}^{U(c_i \to d)} \phi' \tok B' \]

\section{Details of the Construction of an Extensional Model}

In Section \ref{sec:extensional-model-construction}, we outline the construction
of an extensional model of gradual typing starting from a step-1 intensional model.
In this section, we provide the details for each of the constructions mentioned there.

\begin{lemma}\label{lem:step-1-model-to-step-2-model}
Let $\mathcal M$ be a \hyperref[def:step-1-model]{step-1 intensional model} with dyn.

Then we can construct a \hyperref[def:step-2-model]{step-2 intensional model} with dyn.
\end{lemma}
\begin{proof}
    % Write 
    % %
    % \[ \mathcal M = (\vf, \vsq, \ef, \esq, \Ff, \Fsq, \Uf, \Usq, \arrf, \arrsq). \] 
    % %

    Define a step-2 model $\mathcal M'$ as follows:
    \begin{itemize}
      \item Value objects are tuples consisting of:
      \begin{itemize}
        \item A value object $A$ in $\vf$ 
        \item A monoid of ``pure'' perturbations $P_A$ 
        \item A homomorphism of monoids $\ptb_A : P_A \to \{ f \in \vf(A, A) \mid f \bisim \id_A \}$
        \item A monoid of ``impure'' perturbations $P^K_A$ that contains a distinguished element $\delta^*$
        \item A homomorphism of monoids $\ptbk_A : P^K_A \to \{ \phi \in \ef(FA, FA) \mid \phi \bisim \id_{FA} \}$
        such that $\ptbk_A(\delta^*) = \delta_A^*$
      \end{itemize}  

      \item Computation objects are tuples consisting of:
      \begin{itemize}
        \item A computation object $B$ in $\ef$
        \item A monoid of ``pure'' perturbations $P_B$
        \item A homomorphism of monoids $\ptb_B : P_B \to \{ \phi \in \ef(B, B) \mid \phi \bisim \id_B \}$
        \item A monoid of ``impure'' perturbations $P^K_B$
        \item A homomorphism of monoids $\ptbk_B : P^K_B \to \{ g \in \vf(UB, UB) \mid g \bisim \id_{UB} \}$.
      \end{itemize}

      \item Morphisms are given by morphisms of the underlying objects in $\vf$ and $\ef$, respectively
      %, i.e.,
      % \[ \vf'((A, P_A, \ptb_A, P^K_A, \ptbk_A), (A', P_{A'}, \ptb_{A'}, P^K_{A'}, \ptbk_{A'})) = \vf(A, A') \]
      %
      % and likewise for computations.
   
    \end{itemize}

    Before introducing the relations, we make a definition.

    \begin{definition}[push-pull structure]
      Let $c : A \rel A'$ be a value relation of $\mathcal M$. A \emph{value push-pull structure} $\piv_c$ for $c$ consists of:
      \begin{itemize}
        \item A function $\push : P_A \to P_{A'}$ 
              such that for all $\delta^l \in P_A$ we have $\delta^l \ltdyn_c^c \push(\delta^l)$.
        \item A function $\push^K : P^K_A \to P^K_{A'}$ 
              such that for all $\delta^K_l \in P^K_A$ we have $\delta^K_l \ltdyn_{Fc}^{Fc} \push(\delta^K_l)$.
        \item A function $\pull : P_{A'} \to P_A$
              such that for all $\delta_r \in P_{A'}$ we have $\pull(\delta^r) \ltdyn_{c}^c \delta^r$.
        \item A function $\pull^K : P^K_{A'} \to P^K_A$
              such that for all $\delta^K_r \in P^K_{A'}$ we have $\pull(\delta^K_r) \ltdyn_{Fc}^{Fc} \delta^K_r$.
      \end{itemize}

      For $d : B \rel B'$ a computation relation, we define a \emph{computation push-pull structure} $\pie_d$ for $d$
      in an analogous manner.
    \end{definition}


    Now we continue with the description of the construction:
    \begin{itemize}

      \item The objects of $\vsq'$ (i.e., the value relations) are pairs consisting of:
      \begin{itemize}
        \item A value relation $c \in \vsq$
        \item A push-pull structure $\piv_c$ for $c$
      \end{itemize}

      The objects of $\esq'$ are defined analogously.
            
      \item The morphisms of $\vsq'$ and $\esq'$ are given by the morphisms of $\vsq$ and $\esq$.
      
      % Functors \times, +, F, U, arrow
     
      % \item We define $F$ on objects by $F (A, \pv_A, \ptbv_A) = (FA, (1 + \pv_A), h_F)$
      % where $1$ is the trivial monoid, $+$ is the coproduct in the category of monoids, and $h_F$ is the homomorphism defined as follows:

      % \item We define $U$ on objects by $U (B, \pe_B, \ptbe_B) = (UB, \pe_B, h_U)$
      % where $h_U(p_B) = U(\ptbe_B(p_B))$.
      
      % \item We define $(A, \pv_A, \ptbv_A) \arr (B, \pe_B, \ptbe_B) = (A \arr B, \pv_A \times \pe_B, h_\arr)$
      % where $\times$ is the product in the category of monoids, and $h_\arr$ is defined by 
      % $h_\arr(p_A, p_B) = \ptbv_A(p_A) \arr \ptbe_B(p_B)$.
    \end{itemize}
\end{proof}

%%%%%%%%%%%%%%%%%%%%%%%%%%%%%%%%%%%%%%%%%%%%%%%%%%%%%%%%%%%%%%%%%%%%%%%%%%%%%%

\begin{lemma}\label{lem:step-2-model-to-step-3-model}
  Let $\mathcal M$ be a \hyperref[def:step-2-model]{step-2 intensional model}.

  Then we can construct a \hyperref[def:step-3-model]{step-3 intensional model}.
\end{lemma}
\begin{proof}
  Write 
  %
  \[ \mathcal M = (\vf, \vsq, \ef, \esq, \Ff, \Fsq, \Uf, \Usq, \arrf, \arrsq). \] 
  %

  We begin with a definition.

  \begin{definition}[representation structure]
  Let $c : A \rel A'$ be a value relation. A \emph{left-representation structure} $\rho^L_c$ for $c$ consists of
  a value morphism $e_c \in \vf(A, A')$ such that $c$ is quasi-left-representable by $e_c$ (see Definition \ref{def:quasi-left-representable}).
  
  Likewise, let $d : B \rel B'$. A \emph{right-representation structure} $\rho^R_d$ for $d$ consists of
  a computation morphism $p_d \in \ef(B', B)$ such that $d$ is quasi-right-representable by $p_d$ (see Definition \ref{def:quasi-right-representable}).
  \end{definition}
  
  (Notice that the direction of the morphism is opposite in the definition of right-representation structure.)

  We define a step-3 model $\mathcal M'$ as follows:
  \begin{itemize}
    \item The objects of $\mathcal M'$ are defined to be the same as the objects of $\mathcal M$.
    \item The value and computation morphisms in $\mathcal M'$ are the same as those of $\mathcal M$.
    \item A value relation is defined to be a tuple $(c, \rho^L_c, \rho^R_{Fc})$ where:
    \begin{itemize}
      \item $c$ is a value relation in $\mathcal M$, and 
      \item $\rho^L_c$ a left-representation structure for $c$, and 
      \item $\rho^R_{Fc}$ a right-representation structure for $Fc$.
    \end{itemize}
    \item Likewise, a computation relation is defined to be a tuple $(d, \rho^R_d, \rho^L_{Ud})$ with
    \begin{itemize}
      \item $d$ a computation relation in $\mathcal M$
      \item $\rho^R_d$ a right-representation structure for $d$
      \item $\rho^L_{Ud}$ a left-representation structure for $Ud$
    \end{itemize}
    \item Morphisms of value relations (i.e., the value squares) are defined by simply
    ignoring the representation structures. That is, a morphism of value relations
    $\alpha \in \vsq'((c, \rho^L_c, \rho^R_{Fc}), (c' \rho^L_{c'}, \rho^R_{Fc'}))$ is simply a morphism of value
    relations in $\vsq(c, c')$. Likewise for computations.
  \end{itemize}
\end{proof}

% Now we define the functors $F$, $U$, $\times$, and $\arr$.

% On objects, the behavior is the same as the respective functors in $\mathcal M$.

% For relations, we define 
% $\Fsq' (c, \rho^L_c, \rho^R_{Fc}) = (\Fsq c, \rho^R_{Fc}, UF(\rho^L_c))$ and
% $\Usq' (d, \rho^R_d, \rho^L_{Ud}) = (\Usq d, \rho^L_{Ud}, FU(\rho^R_d))$.

% We define $(c, \rho^L_c) \arr (d, \rho^R_d) = (c \arr d, \rho^R_{c \arr d})$.



% We now verify that the construction meets the requirements of a step-3 model.
% First, we check that composition of value relations (resp. computation relations)
% is well-defined.


%%%%%%%%%%%%%%%%%%%%%%%%%%%%%%%%%%%%%%%%%%%%%%%%%%%%%%%%%%%%%%%%%%%%%%%%%%%%%%


\begin{lemma}\label{lem:step-4-model-to-extensional-model}
  Let $\mathcal M$ be a \hyperref[def:step-4-model]{step-4 intensional model}.
  Then we can define an extensional model.
\end{lemma}
\begin{proof}
  
  
  % More formally, we define an extensional model $\mathcal M_e$ as follows.
  % \begin{itemize}
  %   \item 
  % \end{itemize}
\end{proof}



\section{Adequacy}\label{sec:appendix-adequacy}

In this section, we show an adequacy result for the extensional model of GTT we obtained by
applying the abstract construction introduced in Section
\ref{sec:extensional-model-construction} to the concrete model

First we establish some notation. Fix a morphism $f : 1 \to \li \Nat \cong \li \Nat$.
We write that $f \da n$ to mean that there exists $m$ such that $f = \delta^m(\eta n)$
and $f \da \mho$ to mean that there exists $m$ such that $f = \delta^m(\mho)$.

Recall that $\ltls$ denotes the relation on value morphisms defined as the bisimilarity-closure
of the intensional error-ordering on morphisms.
More concretely, we have $f \ltls g$ iff there exists $f'$ and $g'$ with

\[ f \bisim f' \le g' \bisim g. \]

The result we would like to show is as follows:
\begin{lemma}
If $f \ltls g : \li \Nat$, then:
\begin{itemize}
  \item If $f \da n$ then $g \da n$.
  \item If $g \da \mho$ then $f \da \mho$.
  \item If $g \da n$ then $f \da n$.
\end{itemize}
\end{lemma}

Unfortunately, this result is actually not provable!
Roughly speaking, the issue is that this is a ``global'' result, and it is not possible
to prove such results inside of the guarded setting. 
In particular, if we tried to prove a result such as the above in the guarded setting,
we would run into a problem where we would have a natural number ``stuck'' under a $\later$
with no way to get at the underlying number.

Thus, to prove our adequacy result, we need to leave the guarded setting and pass back
to the normal set-theoretic world.
As mentioned in the Technical Background section (Section \ref{sec:sgdt}), we can do this
using \emph{clock quantification}.

Recall that all of the constructions we have made in SGDT take place in the context of a clock $k$.
All of our uses of the later modality and guarded recursion happen with respect to this clock.
For example, consider the definition of the lift monad by guarded recursion in Section \ref{TODO}.
% We define the lift monad $\li^k X$ as the guarded fixpoint of $\lambda \tilde{T}. X + 1 + \later^k_t (\tilde{T}_t)$.
We can view this definition as being parameterized by a clock $k$: $\li^k : \type \to \type$.
Then for $X$ satisfying a certain technical requirement, we can define the ``global lift'' monad as $\li^{gl} X = \forall k. \li^k X$.


It can be shown that the global lift monad is isomorphic to the so-called Delay monad of Capretta \cite{TODO}.


% We have been writing the type as $\li X$, but it is perhaps more accurate to write it as $\li^k X$ to
% emphasize that the construction is parameterized by a clock $k$.




% \section{Discussion}\label{sec:discussion}

% \subsection{Synthetic Ordering}

% While the use of synthetic guarded domain theory allows us to very
% conveniently work with non-well-founded recursive constructions while
% abstracting away the precise details of step-indexing, we do work with
% the error ordering in a mostly analytic fashion in that gradual types
% are interpreted as sets equipped with an ordering relation, and all
% terms must be proven to be monotone.
% %
% It is possible that a combination of synthetic guarded domain theory
% with \emph{directed} type theory would allow for an a synthetic
% treatment of the error ordering as well.


\end{document}

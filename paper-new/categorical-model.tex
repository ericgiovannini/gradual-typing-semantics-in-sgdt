\section{Abstract Categorical Models of Graduality}\label{sec:abstract-models}

First, what is a categorical model of call-by-push-value? We will use
the following notion as our basic structure\footnote{There are models
that are closer to syntax e.g., which distinguish between value types
and contexts but this suffices for all of the models we consider}:
\begin{enumerate}
\item A cartesian category $\mathcal V$
\item A category $\mathcal E$
\item An action of $\mathcal V^{op}$ (with the $\mathcal V$ cartesian
  product as monoidal structure) on $\mathcal E$. We write this with
  an arrow $A \arr B$.

  This means we have 
  \[ \alpha : {A_1 \times A_2} \arr B \cong A_2 \arr (A_1 \arr B) \]
  and
  \[ i : 1 \arr B \cong B \]
  satisfying coherence isomorphisms.
\item A functor $U : \mathcal E \to \mathcal V$ that ``preserves
  powering'' in that every $U(A \arr B)$ is an exponential of $UB$ by $A$
  and that $U\alpha$ and $Ui$ are mapped to the canonical isomorphisms
  for exponentials.
\item A left adjoint $F \dashv U$
\item Distributive finite coproducts in $\mathcal V$
\end{enumerate}

\begin{example}
  Given a strong monad $T$ on a bicartesian closed category $\mathcal
  V$, we can extend this to a CBPV model by defining $\mathcal E$ to
  be the category $\mathcal V^T$ of algebras
\end{example}

% morphisms of CBPV models
Given call-by-push-value models
$\mathcal M_1 = (\mathcal V_1, \mathcal E_1, \arr_1, U_1, F_1)$ and
$\mathcal M_2 = (\mathcal V_2, \mathcal E_2, \arr_2, U_2, F_2)$,
A morphism $G$ from $M_1$ to $M_2$ is given by a pair of functors
$G_{\mathcal{V}}: V_1 \to V_2$ and $G_{\mathcal{E}} : E_1 \to E_2$ such that

\begin{enumerate}
  \item $G_{\mathcal{E}} \circ F_1 = F_2 \circ G_{\mathcal{V}}$
  \item $G_{\mathcal{V}} \circ U_1 = U_2 \circ G_{\mathcal{E}}$
  \item $G_{\mathcal{E}}(A \arr_1 B) = G_{\mathcal{V}}(A) \arr_2 G_{\mathcal{E}}(B)$
\end{enumerate}

% TODO what about the product and coproducts in V?
% Do we need that G_V(A \times_1 A') = G_V(A) \times_2 G_V(A')
% and likewise for coproducts?

With these definitions, we can define a category whose objects are CBPV
models, and whose arrows are CBPV morphisms.
This category will serve as the setting in which we formulate the definition
of a model of extensional gradual typing.


\subsection{Extensional Models of Gradual Typing}

An model $\mathcal{M}$ of extensional gradual typing consists of:
\begin{itemize}
  \item CBPV models $\mathcal M_f$ and $\mathcal M_{sq}$
  \item CBPV morphisms $r : \mathcal M_f \to \mathcal M_{sq}$ and 
  $s, t : \mathcal M_{sq} \to \mathcal M_f$
\end{itemize}

satisfying certain additional conditions that will be described below.

% % https://q.uiver.app/#q=WzAsMixbMCwxLCJcXG1hdGhjYWx7TX1fe3NxfSJdLFswLDAsIlxcbWF0aGNhbHtNfV9mIl0sWzEsMCwiciJdLFswLDEsInMiLDAseyJjdXJ2ZSI6LTJ9XSxbMCwxLCJ0IiwyLHsiY3VydmUiOjJ9XV0=
% \[\begin{tikzcd}[ampersand replacement=\&]
% 	{\mathcal{M}_f} \\
% 	{\mathcal{M}_{sq}}
% 	\arrow["r", from=1-1, to=2-1]
% 	\arrow["s", curve={height=-12pt}, from=2-1, to=1-1]
% 	\arrow["t"', curve={height=12pt}, from=2-1, to=1-1]
% \end{tikzcd}\]

Spelling this out in light of the above definitions, we see that this is
equivalent to the following in the category $\textbf{Cat}$:

% https://q.uiver.app/#q=WzAsNCxbMCwyLCJcXHZzcSJdLFsyLDIsIlxcZXNxIl0sWzAsMCwiXFx2ZiJdLFsyLDAsIlxcZWYiXSxbMiwzLCJcXEZmIiwwLHsiY3VydmUiOi0yfV0sWzMsMiwiXFxVZiIsMCx7ImN1cnZlIjotMn1dLFswLDEsIlxcRnNxIiwwLHsiY3VydmUiOi0yfV0sWzEsMCwiXFxVc3EiLDAseyJjdXJ2ZSI6LTJ9XSxbMiwwLCJcXHJ2Il0sWzAsMiwiXFxzdiIsMCx7ImN1cnZlIjotMn1dLFsyLDAsIlxcdHYiLDAseyJjdXJ2ZSI6LTJ9XSxbMSwzLCJcXHNlIiwwLHsiY3VydmUiOi0yfV0sWzMsMSwiXFx0ZSIsMCx7ImN1cnZlIjotMn1dLFszLDEsIlxccmUiXSxbNCw1LCJcXGJvdCIsMSx7InNob3J0ZW4iOnsic291cmNlIjoyMCwidGFyZ2V0IjoyMH0sInN0eWxlIjp7ImJvZHkiOnsibmFtZSI6Im5vbmUifSwiaGVhZCI6eyJuYW1lIjoibm9uZSJ9fX1dLFs2LDcsIlxcYm90IiwxLHsic2hvcnRlbiI6eyJzb3VyY2UiOjIwLCJ0YXJnZXQiOjIwfSwic3R5bGUiOnsiYm9keSI6eyJuYW1lIjoibm9uZSJ9LCJoZWFkIjp7Im5hbWUiOiJub25lIn19fV1d
\[\begin{tikzcd}[ampersand replacement=\&]
	\vf \&\& \ef \\
	\\
	\vsq \&\& \esq
	\arrow[""{name=0, anchor=center, inner sep=0}, "\Ff", curve={height=-12pt}, from=1-1, to=1-3]
	\arrow[""{name=1, anchor=center, inner sep=0}, "\Uf", curve={height=-12pt}, from=1-3, to=1-1]
	\arrow[""{name=2, anchor=center, inner sep=0}, "\Fsq", curve={height=-12pt}, from=3-1, to=3-3]
	\arrow[""{name=3, anchor=center, inner sep=0}, "\Usq", curve={height=-12pt}, from=3-3, to=3-1]
	\arrow["\rv", from=1-1, to=3-1]
	\arrow["\sv", curve={height=-12pt}, from=3-1, to=1-1]
	\arrow["\tv", curve={height=-12pt}, from=1-1, to=3-1]
	\arrow["\se", curve={height=-12pt}, from=3-3, to=1-3]
	\arrow["\te", curve={height=-12pt}, from=1-3, to=3-3]
	\arrow["\re", from=1-3, to=3-3]
	\arrow["\bot"{description}, draw=none, from=0, to=1]
	\arrow["\bot"{description}, draw=none, from=2, to=3]
\end{tikzcd}\]


% The above definition can be interpreted in any compact closed equipment
% (if someone were to figure out a definition for a compact closed
% equipment, that is,\ldots). Then we can get a model of a form of GTT
% by taking a CBPV object in the equipment of \emph{reflexive graph
% categories}. Since reflexive graphs form a topos we can get at this by
% interpreting the above definition \emph{internally} to the topos of
% reflexive graphs. Essentially what this means is that everything above
% has a ``vertex'' component and an ``edge'' component, so we get a
% cartesian category $\mathcal V_f$ which we think of as the value types
% and pure functions but we also get a cartesian category $\mathcal V_{sq}$
% which we think of as the ``value edges'' and ``squares''.

That is, for the values, we have

\begin{enumerate}
  \item A cartesian category $\mathcal V_f$.
  The objects of $\mathcal V_f$ will be called \emph{value types}.
  The morphisms of $\mathcal V_f$ will be called \emph{(pure) functions}.
  
  \item A cartesian category $\mathcal V_{sq}$. 
  The objects of $\mathcal V_{sq}$ will be called \emph{value edges} or
  \emph{value relations}, and the morphisms are \emph{commuting squares}.

  \item Functors $\sv, \tv : \mathcal V_{sq} \to \mathcal V_f$ and 
  $\rv : \mathcal V_f \to \mathcal V_{sq}$.
\end{enumerate}

Likewise, we have the analogous definitions for computations.

We write $c : A \rel A'$ to mean that $c \in \ob(\vsq)$ such that 
$\sv(c) = A$ and $\tv(c) = A'$.
%
Likewise, let $c_i : A_i \rel A_i'$ and $c_o : A_o \rel A_o'$,
and let $f \in \vf(A_i, A_o)$ and $f' \in \vf(A_i', A_o')$.
The notation $\beta : f \ltdyn_{c_o}^{c_i} f'$ is defined to mean

\begin{enumerate}
  \item $\beta \in \vsq(c_i, c_o)$
  \item $\sv(\beta) = f$
  \item $\tv(\beta) = f'$
\end{enumerate}

(Recall that $\sv$ and $\tv$ are functors, so in addition to acting on
the objects of $\vsq$ they also act on morphisms.)

Picorially, this is depicted as a commuting square:

% https://q.uiver.app/#q=WzAsNCxbMCwwLCJBX2kiXSxbMSwwLCJBX2knIl0sWzAsMSwiQV9vIl0sWzEsMSwiQV9vJyJdLFswLDIsImYiLDJdLFsxLDMsImYnIl0sWzAsMSwiY19pIiwwLHsic3R5bGUiOnsiYm9keSI6eyJuYW1lIjoiYmFycmVkIn0sImhlYWQiOnsibmFtZSI6Im5vbmUifX19XSxbMiwzLCJjX28iLDIseyJzdHlsZSI6eyJib2R5Ijp7Im5hbWUiOiJiYXJyZWQifSwiaGVhZCI6eyJuYW1lIjoibm9uZSJ9fX1dLFs0LDUsIlxcYWxwaGEiLDEseyJzaG9ydGVuIjp7InNvdXJjZSI6MjAsInRhcmdldCI6MjB9LCJzdHlsZSI6eyJib2R5Ijp7Im5hbWUiOiJub25lIn0sImhlYWQiOnsibmFtZSI6Im5vbmUifX19XV0=
\[\begin{tikzcd}[ampersand replacement=\&]
	{A_i} \& {A_i'} \\
	{A_o} \& {A_o'}
	\arrow[""{name=0, anchor=center, inner sep=0}, "f"', from=1-1, to=2-1]
	\arrow[""{name=1, anchor=center, inner sep=0}, "{f'}", from=1-2, to=2-2]
	\arrow["{c_i}", "\shortmid"{marking}, no head, from=1-1, to=1-2]
	\arrow["{c_o}"', "\shortmid"{marking}, no head, from=2-1, to=2-2]
	\arrow["\alpha"{description}, draw=none, from=0, to=1]
\end{tikzcd}\]

% Note: When the identity of the square $\beta$ is not important, we may omit it
% and write $f \ltdyn_{c_o}^{c_i} f'$. In this case the meaning is that there exists
% a square $\beta : f \ltdyn_{c_o}^{c_i} f'$.

Composition of squares $\beta : f \ltdyn_{c_2}^{c_1} g$ and $\beta' : f' \ltdyn_{c_3}^{c_2} g'$
corresponds to ``stacking'' the square for $\beta'$ below the square for $\beta$.
Fuctoriality of $s$ and $t$ ensure that the left and right sides of the resulting square are as expected,
i.e., we get $\beta' \circ \beta : f' \circ f \ltdyn_{c_3}^{c_1} g' \circ g$.

% Fuctoriality of $s$ and $t$ ensure that we can ``vertically" compose 
% $\beta : f \ltdyn_{c_2}^{c_1} g$ and $\beta' : f' \ltdyn_{c_3}^{c_2} g'$
% to obtain $\beta' \circ \beta : f' \circ f \ltdyn_{c_3}^{c_1} g' \circ g$.
% Pictorially, this is represented by ``stacking'' the square for
% $\beta'$ below the square for $\beta$.

All of the above holds in an analogous manner for the computations.


We will work in ``locally thin'' models where there is at most one
square with a given boundary.
That is, if $\beta, \beta' : f \ltdyn_{c_o}^{c_i} f'$, then $\beta = \beta'$.
Thus, we may unambiguously omit the identity of the square, i.e., we may write
$f \ltdyn_{c_o}^{c_i} f'$.

We also have a ``horizontal" composition operation on value edges and on computation edges.
That is, let 
%
\[ X = \{ (c, c') \in \ob(\vsq) \times \ob(\vsq) \mid \tv(c) = \sv(c') \}. \]
%
There is an operation $\comp : X \to \ob(\vsq)$ such that $\sv(c \comp c') = \sv(c)$
and $\tv(c \comp c') = \tv(c')$. Likewise for computations.
%
Importantly, we emphasize that this composition is \emph{not} a functor: we only
require that it act on \emph{objects} of $\vsq$ (i.e. edges) and not on the morphisms
(i.e. the squares). Intuivitely, this has to do with the fact that in the extensional
setting, the semantic term precision function is \emph{not} transitive.

With this definition, is easily shown that there is a category $\ve$ of value relations,
whose objects are the objects of $\vf$, and such that $\ve(A, A')$ is the set of objects
$c$ of $\vsq$ whose source is $A$ and whose target is $A'$.
Composition of morphisms is defined using the above operation $\comp$.
Similarly, we have a category $\ee$ of computation relations.


We also require that the category $\ve$ of relations is thin ``up to an
identity square'', i.e., for any $c, c' \in \ve(A, A')$ we have that the
following square commutes:

\[\begin{tikzcd}[ampersand replacement=\&]
  A \& {A'} \\
  A \& {A'}
  \arrow[from=1-1, to=2-1, Rightarrow, no head]
  \arrow[from=1-2, to=2-2, Rightarrow, no head]
  \arrow["c", "\shortmid"{marking}, no head, from=1-1, to=1-2]
  \arrow["c'"', "\shortmid"{marking}, no head, from=2-1, to=2-2]
\end{tikzcd}\]

% In addition to the ordinary universal properties above, when working
% with reflexive graph models we also have access to new notions of
% universal property that relate the ``function'' morphisms to the
% ``edges''.

Finally, we specify the abstract behavior of casts.
In brief, given $c : A \rel A'$, an upcast will be a morphism $\upf\, c$ in $\vf(A, A')$
such that $\upf\, c$ ``represents $c$" (we will define this shortly). % such that $c$ is \emph{left-representable} by $\up c$.
In reality, it is slightly more complicated, since we need to build composition of edges
into the definition, but the idea is similar.

In particular, let $c : A \rel A'$ and $d : A' \rel A''$,
and let $f \in \mathcal V_f(A, A')$ and $g \in \mathcal V_f(A', A'')$.
We say that the pair $(c,d)$ is \emph{left-representable by} $(f, g)$
if the following two squares commute:

\begin{center}
  \begin{tabular}{ m{14em} m{14em} } 
    % UpL
    \begin{tikzcd}[ampersand replacement=\&]
      A \& {A'} \& {A''} \\
      {A'} \&\& {A''}
      \arrow["f"', from=1-1, to=2-1]
      %
      \arrow[from=1-3, to=2-3, Rightarrow, no head]
      %
      \arrow["c", "\shortmid"{marking}, no head, from=1-1, to=1-2]
      \arrow["d", "\shortmid"{marking}, no head, from=1-2, to=1-3]
      %
      \arrow["d"', "\shortmid"{marking}, no head, from=2-1, to=2-3]
    \end{tikzcd}
    &
    % UpR
    \begin{tikzcd}[ampersand replacement=\&]
      A \&\& {A'} \\
      {A} \& {A'} \& {A''}
      \arrow[from=1-1, to=2-1, Rightarrow, no head]
      %
      \arrow["g", from=1-3, to=2-3]
      %
      \arrow["c", "\shortmid"{marking}, no head, from=1-1, to=1-3]
      %
      \arrow["c"', "\shortmid"{marking}, no head, from=2-1, to=2-2]
      \arrow["d"', "\shortmid"{marking}, no head, from=2-2, to=2-3]
    \end{tikzcd}
  \end{tabular}
\end{center}

Dually, let $c \in \mathcal E_e(B, B')$ and $d \in \mathcal E_e(B', B'')$,
and let $f \in \mathcal E_f(B'', B')$ and $g \in \mathcal E_f(B', B)$.
We say the pair $(c,d)$ is \emph{right-representable by} $(f,g)$ if the
following two squares commute:

\begin{center}
  \begin{tabular}{ m{14em} m{14em} } 
    % DnR
    \begin{tikzcd}[ampersand replacement=\&]
      {B} \& {B'} \& {B''} \\
      {B} \&\& {B'}
      \arrow[from=1-1, to=2-1, Rightarrow, no head]
      %
      \arrow["f", from=1-3, to=2-3]
      %
      \arrow["c", "\shortmid"{marking}, no head, from=1-1, to=1-2]
      \arrow["d", "\shortmid"{marking}, no head, from=1-2, to=1-3]
      %
      \arrow["c"', from=2-1, to=2-3, no head]
    \end{tikzcd}
    &
    % DnL
    \begin{tikzcd}[ampersand replacement=\&]
      {B'} \&\& {B''} \\
      {B} \& {B'} \& {B''}
      \arrow["g"', from=1-1, to=2-1]
      %
      \arrow[from=1-3, to=2-3, Rightarrow, no head]
      %
      \arrow["d", from=1-1, to=1-3, no head]
      %
      \arrow["c"', "\shortmid"{marking}, no head, from=2-1, to=2-2]
      \arrow["d"', "\shortmid"{marking}, no head, from=2-2, to=2-3]
    \end{tikzcd}
  \end{tabular}
\end{center}

Then we formulate the relationship between value relation morphisms and
function morphisms as follows:
\begin{enumerate}
  \item There is a functor $\upf : \mathcal V_e \to \mathcal V_f$ that is the
  identity on objects, such that $\upf_*(\mathcal V_e)$, the essential image of
  $\mathcal V_e$ under $\upf$, is thin.
  The objects of $\upf_*(\mathcal V_e)$ are the objects of $\mathcal V_f$, and
  the hom-set
  $\upf_*(\mathcal V_e)(A, A') = \{ f \in \mathcal V_f(A,A') \mid \exists c \in \mathcal V_e(A,A'). \upf(c) = f \}$.
  
  \item Every pair of morphisms $(c,d) \in \mathcal V_e(A, A') \times \mathcal V_e(A', A'')$ is
  left-representable by $(\upf(c), \upf(d))$.
\end{enumerate}

And likewise for computations:
\begin{enumerate}
  \item There is a functor $\dnf : \mathcal E_e^{op} \to \mathcal V_f$ that is
  the identity on objects, such that the essential image of $\mathcal E_e$ under
  $\dnf$ is thin.

  \item Every pair of morphisms $(c,d) \in \mathcal E_e(B, B') \times \mathcal E_e(B',B'')$
  is right-representable by $(\dnf(d), \dnf(c))$.
\end{enumerate}



\textbf{TODO: do we still need this?}
We also want something like
\[ F_c : \mathcal V_u^{op} \to \mathcal E_d \]
\[ U_c : \mathcal E_d^{op} \to \mathcal V_u \]
which ensures that if $R$ is a value edge equivalent to $A(u,-)$ then
\[ F(R) = F(A(u,-)) = (F A)(-,F u) \]

In summary, an extensional model consists of:
\begin{enumerate}
  \item CBPV models $\mathcal M_f$ and $\mathcal M_{sq}$
  \item CBPV morphisms $r : \mathcal M_f \to \mathcal M_{sq}$ and $s, t : \mathcal M_{sq} \to \mathcal M_f$
  \item Thinness: there is at most one square with a given boundary
  \item A ``horizontal composition" operation on value relations and on computation relations
  (from which we can define the categories $\ve$ and $\ee$ of value types/relations and computation types/relations, repsectively).
  \item The categories $\ve$ and $\ee$ are thin up to an identity square
  \item A functor $\upf : \mathcal V_e \to \mathcal V_f$ that is the identity on objects,
  such that $\upf_*(\mathcal V_e)$, the essential image of $\mathcal V_e$ under $\upf$, is thin.
  \item Every pair of morphisms $(c,d) \in \mathcal V_e(A, A') \times \mathcal V_e(A', A'')$ is
  left-representable by $(\upf(c), \upf(d))$.
  \item A functor $\dnf : \mathcal E_e^{op} \to \mathcal V_f$ that is
  the identity on objects, such that the essential image of $\mathcal E_e$ under $\dnf$ is thin.
  \item Every pair of morphisms $(c,d) \in \mathcal E_e(B, B') \times \mathcal E_e(B',B'')$
  is right-representable by $(\dnf(d), \dnf(c))$.
\end{enumerate}



%%%%%%%%%%%%%%%%%%%%%%%%%%%%%%%%%%%%%%%%%%%%%%%%%%%%%%%%%%%%%%%%%%%%%%%%%%%%%%%%%%%%
%%%%%%%%%%%%%%%%%%%%%%%%%%%%%%%%%%%%%%%%%%%%%%%%%%%%%%%%%%%%%%%%%%%%%%%%%%%%%%%%%%%%

\subsection{Intensional Models}\label{sec:abstract-intensional-models}

An intensional model of gradual typing is defined similarly to an extensional model,
with a few key differences that will be discussed below.
%
The starting point is similar to that of the extensional model: an intensional model
will be given by a diagram in the category of CBPV objects, satisfying
additional properties.
%
This time, however, since we are working intensionally, the semantic denotation of
term precision \emph{is} transitive, so we \emph{do} have a horizontal composition
operation on squares. Compare this to the extensional case, where we could only
compose \emph{relations} horizontally, not squares.
What this means is that we can define a functor for composition of value relations
and squares, and a functor for composition of computation relations and squares.

We can specify this elegantly as a category internal to the category of CBPV models.
In particular, as in the extensional case, there is a CBPV model of ``objects''
$\mathcal M_f$ and a CBPV model of ``arrows'' $\mathcal M_{sq}$. 
There are CBPV morphisms
$r : \mathcal M_f \to \mathcal M_{sq}$ and $s, t : \mathcal M_{sq} \to \mathcal M_f$,
just as before.
%
But now, we also have a CBPV morphism $m$ from the pullback 
$\mathcal M_{sq} \times_{s = t} \mathcal M_{sq}$ to $M_{sq}$, i.e., ``composition of arrows".
In particular, this consists of a functor 
$m_{\mathcal{V}} : \vsq \times_{\sv = \tv} \vsq \to \vsq$ for composition of value
relations/squares, and a functor $m_{\mathcal{E}} : \esq \times_{\se = \te} \esq \to \esq$
for composition of computation relations/squares.
Furthermore, $s \circ m = s \circ \pi_1$ and $t \circ m = t \circ \pi_2$.

% If we spell this all out explicitly, we end up with a definition similar to the
% one for the extensional case, but now with the addition of a functor $m_{\mathcal{V}}$ for
% composition of value relations/squares and a functor $m_{\mathcal{E}}$ for
% composition of computation relations/squares.

For the sake of ease of reference, we recap the definition of a step-0 model:

\begin{definition}
  A \emph{step-0} model of intensional gradual typing consists of a category internal to the
  category of CBPV models.
\end{definition}

% In particular, as before, we have cartesian categories $\mathcal V_f$,
% $\mathcal V_e$, and $\mathcal V_{sq}$, in addition to $\mathcal E_f$,
% $\mathcal E_e$, and $\mathcal E_{sq}$.
% But we now have horizontal composition of squares as well.


\subsubsection{Bisimilarity}\label{sec:abstract-model-bisimilarity}

Working intensionally means we need to take into consideration the steps
taken by terms. One consequence of this is that we need a way to specify
that two morphisms are the same ``up to delay'', i.e., they differ only in that
one may wait more than the other.

In particular, for any pair of objects $A$ and $A'$, in $\vf$,
we require that there is a reflexive, symmetric relation $\bisim_{A,A'}$ on the
hom-set $\vf(A, A')$, called the \emph{weak bisimilarity} relation.
Similarly for the computation category: there is a reflexive, symmetric relation
$\bisim_{B,B'}$ defined on each hom-set $\ef(B, B')$.
%
Additionally, the weak bisimilarity relation should respect composition:
if $f \bisim_{A,A'} f'$ and $g \bisim_{A',A''} g'$, then
$g \circ f \bisim_{A,A''} g' \circ f'$, and likewise for computations.

We can specify all of this abstractly via categories $\vsim$ and $\esim$ along with
functors $\rvsim : \vf \to \vsim$ and $\svsim, \tvsim : \vsim \to \vf$,
and likewise for computations.
Since bisimilarity of morphisms $f$ and $f'$ requires that they share source and target,
we require that $\svsim$ and $\tvsim$ agree on objects and likewise for $\sesim$ and $\tesim$.
Thus, the objects of $\vsim$ are identified with $\ob(\vf)$.
The morphisms of $\vsim$ are ``bisimilarity proofs'', analogous to the commuting squares of $\vsq$.

There is also a ``symmetry'' endofunctor $\text{sym}_{\mathcal{V}}^\bisim : \vsim \to \vsim$
such that $\svsim \circ \text{sym}_{\mathcal{V}}^\bisim = \tvsim$
and $\tvsim \circ \text{sym}_{\mathcal{V}}^\bisim = \svsim$,
and $\text{sym}_{\mathcal{V}}^\bisim \circ \text{sym}_{\mathcal{V}}^\bisim$ is the identity.
Likewise there is a symmetry endofunctor $\text{sym}_{\mathcal{E}}^\bisim : \esim \to \esim$.

% TODO explain what this means in terms of bisimilarity "squares"?
In this setting, we write $\refl_A : A \bisim A$ to mean that $\refl_A \in \ob(\vsim)$,
such that $\svsim(\refl_A) = A = \tvsim(\refl_A)$.
Let $f, f' \in \vf(A_i, A_o)$.
The judgment $\gamma : f \bisim_{A_i, A_o} f'$ is defined to mean:

\begin{enumerate}
  \item $\gamma \in \vsim(\refl_{A_i}, \refl_{A_o})$
  \item $\svsim(\gamma) = f$
  \item $\tvsim(\gamma) = f'$
\end{enumerate}

Lastly, we require that for any value object $A$, the hom-set $\ef(FA, FA)$ contains a
distinguished morphism $\delta_A$, such that $\delta_A \bisim_{FA, FA} \id_{FA}$.


% Spelling this all out concretely, for any pair...

\begin{definition}\label{def:step-1-model}
A \emph{step-1 intensional model} consists of all the data of a step-0 intensional model along
with the additional categories and functors for bisimiarity described above, as  well as
the existence of a distinguised computation morphism $\delta_A \bisim \id_{FA}$ for each $A$.
\end{definition}

\subsubsection{Perturbations}\label{sec:abstract-model-perturbations}

A second consequence of working intensionally is that the squares in the representable
properties must now involve a notion of ``delay" or ``perturbation'' in order to
keep the function morphisms on each side in lock-step. Intuitively, the perturbations
have no effect other than to cause the function to which they are applied to ``wait''
in a specific manner.
We formalize this notion by requiring that for each object $A$ in $\vf$,
there is a monoid of \emph{perturbations} $P^V_A$ and a monoid homomorphism
$\ptbv_A : \pv_A \to \vf(A,A)$.
Similarly, for each $B : \ef$ there is a monoid $P^C_B$ and a
homomorphism $\ptbe_B : \pe_B \to \ef(B,B)$.
If $\delta \in P^V_A$, we will sometimes omit the homomorphism $\ptbv_A$ and simply write
$\delta$ to refer to the morphism $\ptbv_A(\delta) \in \vf(A,A)$, and likewise
for computation perturbations. The context will make clear whether we are referring
to an element of the perturbation monoid or the corresponding morphism.

We require that $\delta_A \in \pe_{FA}$ for all $A$, where $\delta_A$ is the distinguished
morphism that is required to be present in every hom-set $\ef(FA, FA)$ per the definition
of a step-1 model.

The perturbations must be preserved by $\times$, $\arr$, $U$, and $F$.

For reasons that will be made clear in the next section, perturbations must also satisfy a property that we call the ``push-pull'' property,
which is formulated as follows. Let $c : A \rel A'$.
Given a perturbation $\delta \in \pv_A$, there is a corresponding perturbation
$\push_c(\delta) \in \pv_{A'}$. % making the following square commute:
%
Likewise, given $\delta' \in \pv_{A'}$ there is a perturbation $\pull_c(\delta') \in \pv_A$.
% making the following square commute:

Moreover, push-pull states that the following squares must commute:

\begin{center}
  \begin{tabular}{ m{9em} m{9em} } 
    \begin{tikzcd}[ampersand replacement=\&]
      A \& {A'} \\
      A \& {A'}
      \arrow["\delta"', from=1-1, to=2-1]
      \arrow["{\push_c(\delta)}", from=1-2, to=2-2]
      \arrow["c", "\shortmid"{marking}, no head, from=1-1, to=1-2]
      \arrow["c"', "\shortmid"{marking}, no head, from=2-1, to=2-2]
    \end{tikzcd}
    &
    \begin{tikzcd}[ampersand replacement=\&]
      A \& {A'} \\
      A \& {A'}
      \arrow["{\pull_c(\delta')}"', from=1-1, to=2-1]
      \arrow["{\delta'}", from=1-2, to=2-2]
      \arrow["c", "\shortmid"{marking}, no head, from=1-1, to=1-2]
      \arrow["c"', "\shortmid"{marking}, no head, from=2-1, to=2-2]
    \end{tikzcd}
  \end{tabular}
\end{center}

The analogous property should also hold for computation relations and perturbations.

Lastly, we require that all perturbations be weakly bisimilar to the identity morphism,
capturing the notion that they have no effect other than to delay. We observe that
the set of endomorphisms $f$ such that $f$ is weakly bisimilar to the identity
forms a monoid under composition.

This is summarized below:

\begin{definition}\label{def:step-2-model}
  A \emph{step-2} model of intensional gradual typing consists of all the data of a step-1 model plus:
  \begin{enumerate}
    \item $\pv_A$ and a monoid homomorphism 
      \[ \ptbv_A : \pv_A \to \{ f \in \vf(A,A) \mid f \bisim \id \} \]
    \item $\pe_B$ and a monoid homomorphism 
      \[ \ptbe_B : \pe_B \to \{ g \in \ef(B,B) \mid g \bisim \id \} \]
    \item The functors $\times$, $\arr$, $U$, and $F$ preserve perturbations.
    \item The push-pull property holds for all $c : A \rel A'$ and all $d : B \rel B'$.
  \end{enumerate}
\end{definition}

\subsubsection{Behavior of Casts}


% We similarly have thin subcategories $\mathcal V_u$ and $\mathcal E_d$ of
% upcasts and downcasts. The relation between function morphisms and edges
% is as follows.

As is the case in the extensional model, there is a relationship between
vertical (i.e., function) morphisms and horizontal (i.e., relation) morphisms,
but as mentioned above, now there
are perturbations involved in order to keep both sides ``in lock-step".
The precise definitions are as follows.

\begin{definition}\label{def:quasi-left-representable}
Let $c : A \rel A'$ be a value relation. We say that $c$ is \emph{quasi-left-representable by}
$f \in \vf(A, A')$ if there are perturbations $\delta_c^{l,e} \in \pv_A$ and
$\delta_c^{r,e} \in \pv_{A'}$ such that the following squares commute:

\begin{center}
  \begin{tabular}{ m{7em} m{7em} } 
    % UpL
    \begin{tikzcd}[ampersand replacement=\&]
      A \& {A'} \\
      {A'} \& {A'}
      \arrow["f"', from=1-1, to=2-1]
      \arrow["\delta_c^{r,e}", from=1-2, to=2-2]
      \arrow["c", "\shortmid"{marking}, no head, from=1-1, to=1-2]
      \arrow[from=2-1, to=2-2, Rightarrow, no head]
    \end{tikzcd}
    &
    % UpR
    \begin{tikzcd}[ampersand replacement=\&]
      A \& {A} \\
      {A} \& {A'}
      \arrow["\delta_c^{l,e}"', from=1-1, to=2-1]
      \arrow["f", from=1-2, to=2-2]
      \arrow[from=1-1, to=1-2, Rightarrow, no head]
      \arrow["c"', "\shortmid"{marking}, no head, from=2-1, to=2-2]
    \end{tikzcd}
  \end{tabular}
\end{center}
\end{definition}

\begin{definition}\label{def:quasi-right-representable}
Let $d : B \rel B'$ be a computation relation. We say that $d$ is
\emph{quasi-right-representable by} $f \in \ef(B', B)$
if there exist perturbations $\delta_d^{l,p} \in \pe_B$ and
$\delta_d^{r,p} \in \pe_{B'}$ such that the following squares commute:

\begin{center}
  \begin{tabular}{ m{7em} m{7em} } 
    % DnR
    \begin{tikzcd}[ampersand replacement=\&]
      {B} \& {B'} \\
      {B} \& {B}
      \arrow["\delta_d^{l,p}"', from=1-1, to=2-1]
      \arrow["g", from=1-2, to=2-2]
      \arrow["R", "\shortmid"{marking}, no head, from=1-1, to=1-2]
      \arrow[from=2-1, to=2-2, Rightarrow, no head]
    \end{tikzcd}
    &
    % DnL
    \begin{tikzcd}[ampersand replacement=\&]
      {B'} \& {B'} \\
      {B} \& {B'}
      \arrow["g"', from=1-1, to=2-1]
      \arrow["\delta_d^{r,p}", from=1-2, to=2-2]
      \arrow[from=1-1, to=1-2, Rightarrow, no head]
      \arrow["R"', "\shortmid"{marking}, no head, from=2-1, to=2-2]
    \end{tikzcd}
  \end{tabular}
\end{center}
\end{definition}

Besides the perturbations, one other difference between the extensional
and intensional versions of the representability axioms is that in the
extensional setting, the rules build in the notion of composition, whereas
their intensional counterparts do not.
In the extensional setting, we do not have horizontal composition of squares, which
is required to derive the versions of the rules that build in composition
from the versions that do not.
In the intensional setting, we do have horizontal composition of squares,
so we can take the simpler versions as primitive and derive the ones
involving composition (though we must be careful about the perturbations
when doing so!).

\begin{definition}\label{def:step-3-model}
  A \emph{step-3 intensional model} consists
  of all the data of a step-2 intensional model, with the following additional data:
  \begin{enumerate}
    \item Functors $\upf : \ve \to \vf$ and $\dnf : \ee^{op} \to \ef$ % TODO: image is thin?
    \item Every value edge $c : A \rel A'$ is quasi-left-representable by $\upf(c)$ and
    every computation edge $d : B \rel B'$ is quasi-right-representable by $\dnf(d)$.
  \end{enumerate}
\end{definition}


\subsubsection{The Dynamic Type}

Now we can discuss what it means for an intensional model to model the dynamic type.
This applies to any of the above abstract model definitions, i.e., steps 0-3.

\begin{definition}\label{def:step-4-model}
  A \emph{step-$i$ intensional model with dyn} is a step-$i$ model $\mathcal M$ such that:
  %a distinguished value object $D \in \ob(\vf)$ such that:
  %
  \begin{enumerate}
    \item There is a distinguished value object $D \in \ob(\vf)$.
   
    \item For each value type $A$, there is a value relation $\text{inj}_A : A \rel D$.
    
    \item If $c : A \rel A'$, then $\text{inj}_{A} = c \comp \text{inj}_{A'}$.
  \end{enumerate}
\end{definition}

% (By definition of a step-3 model, this relation satisfies the push-pull property and is
% quasi-left-representable.)

% (By definition of a step-3 model, this means there is also a monoid $\pv_D$ of
% perturbations and a homomorphism $\ptbv_D$.)



%%%%%%%%%%%%%%%%%%%%%%%%%%%%%%%%%%%%%%%%%%%%%%%%%%%%%%%%%%%%%%%%%%%%%%%%%%%%%%%%%%%%
%%%%%%%%%%%%%%%%%%%%%%%%%%%%%%%%%%%%%%%%%%%%%%%%%%%%%%%%%%%%%%%%%%%%%%%%%%%%%%%%%%%%

\subsection{Constructing an Extensional Model}\label{sec:extensional-model-construction}

In the previous section, we have given the definition of an intensional model
of gradual typing as a series of steps with each definition building on the previous one.
%
Here, we discuss how to construct an extensional model from an intensional model with dyn.
We do so in several phases, beginning with a step-1 intensional model with dyn
and ending with an extensional model.
Moreover, this construction is \emph{modular}, in that each phase of the
construction does not depend on the details of the previous ones.
% However, this process cannot proceed in isolation: some phases require
% additional inputs. We will make clear what data must be supplied to each phase.

\subsubsection{Adding Perturbations}

Suppose we have a \hyperref[def:step-1-model]{step-1 intensional model} $\mathcal{M}$ with dyn.
Recall that a step-1 intensional model consists of a step-0 model (i.e., a
category internal to the category of CBPV models), along with the necessary
categories and functors for bisimilarity as discussed in Section
\ref{sec:abstract-model-bisimilarity}.
Further, recall that a \hyperref[def:step-2-model]{step-2 model} has everything
a step-1 model has, with the addition of perturbation monoids $\pv_A$ for all
$A$ and $\pe_B$ for all $B$.
These perturbations must be interpretable via homomorphisms
$\ptb_A$ and $\ptb_B$ as endomorphisms bisimilar to the identity,
and moreover, the push-pull property must hold for all
value relations $c$ and all computation relations $d$.

% TODO do we really need all of this data?
What ``external" data do we need to define such a model?
Suppose we are given, for each value type $A$, a monoid $\pv_A$ with homomorphism
$\ptb_A$ into the endomorphisms on $A$ bisimilar to the identity, and likewise, for
each computation type $B$, a monoid $\pv_B$ and homomorphism $\ptb_B$.
Furthermore, suppose we are given, for each value type $A$, a distinguished endomorphism
$\delta_A \in \ef(FA, FA)$ such that $\delta_A \bisim \id_{FA}$.
Finally, suppose that each value relation $c : A \rel A'$ satisfies the push-pull property
and likewise for each computation relation $d : B \rel B'$.

Then we claim that from the above data, we can construct a step-2 intensional model.
For the proof, see Lemma \ref{lem:step-1-model-to-step-2-model} in the Appendix.

%%%%%%%%%%%%%%%%%%%%%%%%%%%%%%%%%%%%%%%%%%%%%%%%%%%%%%%%%%%%%%%%%%%%%%%%%%%%%%%%%%%%

\subsubsection{Adding Quasi-Representability}

Now suppose we have a step-2 intensional model $\mathcal{M}$.
We will describe how to construct a \hyperref[def:step-3-model]{step-3 intensional model} $\mathcal{M'}$.

We now describe the construction of a step-3 model $\mathcal M'$.




%%%%%%%%%%%%%%%%%%%%%%%%%%%%%%%%%%%%%%%%%%%%%%%%%%%%%%%%%%%%%%%%%%%%%%%%%%%%%%%%%%%%

\subsubsection{Defining an Extensional Model}

Finally, suppose $\mathcal M$ is a step-3 intensional model.
We now describe how to build an extensional model.

The idea is to define an extensional model whose squares are the ``bisimilarity-closure''
of the squares of the provided intensional model $\mathcal M$.
  
The categories $\vf$, $\ef$ are the same as those of $\mathcal M$.
Additionally, the objects of $\vsq$ and $\esq$, i.e., the value and
computation relations, are the same.
The difference arises in the \emph{morphisms} of $\vsq$ and $\esq$,
i.e., the commuting squares. In particular, a morphism
$\alpha_e \in \vsq'(c_i, c_o)$ with source $f$ and target $g$ is given by:
\begin{itemize}
  \item a morphism $f' \in \vf(A_i, A_o)$ with $f \bisim f'$.
  \item a morphism $g' \in \vf(A_i', A_o')$ with $g \bisim g'$.
  \item a square $\alpha_i \in \vsq(c_i, c_o)$ with source $f'$ and target $g'$.
\end{itemize}

Using our existing notation, we say that $f \le_{c_o}^{c_i} g$ if there exist $f'$ and $g'$ such that

\[ f \bisim_{A_i,A_o} f' \ltdyn_{c_o}^{c_i} g' \bisim_{A_i',A_o'} g. \]

We make the analogous construction for the computation squares.

Next, we check that the requirements of an extensional model are satisfied.
In particular, we need to verify the representability properties.
We define functors $\upf$ and $\dnf$





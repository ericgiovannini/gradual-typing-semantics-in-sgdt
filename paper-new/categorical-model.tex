\section{Abstract Categorical Models of Graduality}

First, what is a categorical model of call-by-push-value? We will use
the following notion as our basic structure\footnote{There are models
that are closer to syntax e.g., which distinguish between value types
and contexts but this suffices for all of the models we consider}:
\begin{enumerate}
\item A cartesian category $\mathcal V$
\item A category $\mathcal E$
\item An action of $\mathcal V^{op}$ (with the $\mathcal V$ cartesian
  product as monoidal structure) on $\mathcal E$. We write this as
  exponentiation $B^A$.

  This means we have 
  \[ \alpha : B^{A_1 \times A_2} \cong (B^{A_1})^{A^2} \]
  and
  \[ i : B^1 \cong B \]
  satisfying coherence isomorphisms
\item A functor $U : \mathcal E \to \mathcal V$ that ``preserves
  powering'' in that every $U(B^A)$ is an exponential of $UB$ by $A$
  and that $U\alpha$ and $Ui$ are mapped to the canonical isomorphisms
  for exponentials.
\item A left adjoint $F \dashv U$
\item Distributive finite coproducts in $\mathcal V$
\end{enumerate}

\begin{example}
  Given a strong monad $T$ on a bicartesian closed category $\mathcal
  V$, we can extend this to a CBPV model by defining $\mathcal E$ to
  be the category $\mathcal V^T$ of algebras
\end{example}

The above definition can interpreted in any compact closed equipment
(if someone were to figure out a definition for a compact closed
equipment, that is,\ldots). Then we can get a model of a form of GTT
by taking a CBPV object in the equipment of \emph{reflexive graph
categories}. Since reflexive graphs form a topos we can get at this by
interpreting the above definition \emph{internally} to the topos of
reflexive graphs. Essentially what this means is that everything above
has a ``vertex'' component and an ``edge'' component, so we get a
cartesian category $\mathcal V_f$ which we think of as the value types
and pure functions but we also get a cartesian category $\mathcal V_{sq}$
which we think of as the ``value edges'' and ``squares''.

More formally, for the values, we have

\begin{enumerate}
  \item Cartesian categories $\mathcal V_f$ and $\mathcal V_e$ with the same set of objects,
  namely, the value types.
  The morphisms of $\mathcal V_f$ will be called \emph{(pure) functions}, and the
  morphisms of $\mathcal V_e$ will be called \emph{edges} or \emph{relations}.

  \item A cartesian category $\mathcal V_{sq}$ whose objects are the morphisms of $\mathcal V_e$
  and whose morphisms are \emph{commuting squares}.

  \item Functors $s, t : \mathcal V_{sq} \to \mathcal V_f$ and $i : \mathcal V_f \to \mathcal V_{sq}$.
\end{enumerate}

Likewise, we have an analogous definition for computations.

We will work in ``locally thin'' models where there is at most one
square with a given boundary.

In addition to the ordinary universal properties above, when working
with reflexive graph models we also have access to new notions of
universal property that relate the ``function'' morphisms to the
``edges''.

In particular, let $c \in \mathcal V_e(A, A')$ and $d \in \mathcal V_e(A',A'')$,
and let $f \in \mathcal V_f(A, A')$ and $g \in \mathcal V_f(A', A'')$.
We say that the pair $(c,d)$ is \emph{left-representable by} $(f, g)$
if the following two squares commute:

\begin{center}
  \begin{tabular}{ m{14em} m{14em} } 
    % UpL
    \begin{tikzcd}[ampersand replacement=\&]
      A \& {A'} \& {A''} \\
      {A'} \&\& {A''}
      \arrow["f"', from=1-1, to=2-1]
      %
      \arrow[from=1-3, to=2-3, Rightarrow, no head]
      %
      \arrow["c", "\shortmid"{marking}, no head, from=1-1, to=1-2]
      \arrow["d", "\shortmid"{marking}, no head, from=1-2, to=1-3]
      %
      \arrow["d"', "\shortmid"{marking}, no head, from=2-1, to=2-3]
    \end{tikzcd}
    &
    % UpR
    \begin{tikzcd}[ampersand replacement=\&]
      A \&\& {A'} \\
      {A} \& {A'} \& {A''}
      \arrow[from=1-1, to=2-1, Rightarrow, no head]
      %
      \arrow["g", from=1-3, to=2-3]
      %
      \arrow["c", "\shortmid"{marking}, no head, from=1-1, to=1-3]
      %
      \arrow["c"', "\shortmid"{marking}, no head, from=2-1, to=2-2]
      \arrow["d"', "\shortmid"{marking}, no head, from=2-2, to=2-3]
    \end{tikzcd}
  \end{tabular}
\end{center}

Dually, let $c \in \mathcal E_e(B, B')$ and $d \in \mathcal E_e(B', B'')$,
and let $f \in \mathcal E_f(B'', B')$ and $g \in \mathcal E_f(B', B)$.
We say the pair $(c,d)$ is \emph{right-representable by} $(f,g)$ if the
following two squares commute:

\begin{center}
  \begin{tabular}{ m{14em} m{14em} } 
    % DnR
    \begin{tikzcd}[ampersand replacement=\&]
      {B} \& {B'} \& {B''} \\
      {B} \&\& {B'}
      \arrow[from=1-1, to=2-1, Rightarrow, no head]
      %
      \arrow["f", from=1-3, to=2-3]
      %
      \arrow["c", "\shortmid"{marking}, no head, from=1-1, to=1-2]
      \arrow["d", "\shortmid"{marking}, no head, from=1-2, to=1-3]
      %
      \arrow["c"', from=2-1, to=2-3, no head]
    \end{tikzcd}
    &
    % DnL
    \begin{tikzcd}[ampersand replacement=\&]
      {B'} \&\& {B''} \\
      {B} \& {B'} \& {B''}
      \arrow["g"', from=1-1, to=2-1]
      %
      \arrow[from=1-3, to=2-3, Rightarrow, no head]
      %
      \arrow["d", from=1-1, to=1-3, no head]
      %
      \arrow["c"', "\shortmid"{marking}, no head, from=2-1, to=2-2]
      \arrow["d"', "\shortmid"{marking}, no head, from=2-2, to=2-3]
    \end{tikzcd}
  \end{tabular}
\end{center}

Then we formulate the relationship between value edges and functions as follows:
\begin{enumerate}
  \item There is a functor $\upf : \mathcal V_e \to \mathcal V_f$ that is the
  identity on objects, such that $\upf_*(\mathcal V_e)$, the essential image of
  $\mathcal V_e$ under $\upf$, is thin.
  The objects of $\upf_*(\mathcal V_e)$ are the objects of $\mathcal V_f$, and
  the hom-set
  $\upf_*(\mathcal V_e)(A, A') = \{ f \in \mathcal V_f(A,A') \mid \exists R \in \mathcal V_e(A,A'). \upf(R) = f \}$.
  
  \item Every pair of morphisms $(c,d) \in \mathcal V_e(A, A') \times \mathcal V_e(A', A'')$ is
  left-representable by $(\upf(c), \upf(d))$.
\end{enumerate}

And likewise for computations:
\begin{enumerate}
  \item There is a functor $\dnf : \mathcal E_e^{op} \to \mathcal V_f$ that is
  the identity on objects, such that the essential image of $\mathcal E_e$ under
  $\dnf$ is thin.

  \item Every pair of morphisms $(c,d) \in \mathcal E_e(B, B') \times \mathcal E_e(B',B'')$
  is right-representable by $(\dnf(d), \dnf(c))$.
\end{enumerate}

We also require that the category $\mathcal V_e$ of relations is thin ``up to an
identity square'', i.e., for any $R, R' \in \mathcal V_e(A, A')$ we have that the
following square commutes:

\[\begin{tikzcd}[ampersand replacement=\&]
  A \& {A'} \\
  A \& {A'}
  \arrow[from=1-1, to=2-1, Rightarrow, no head]
  \arrow[from=1-2, to=2-2, Rightarrow, no head]
  \arrow["R", "\shortmid"{marking}, no head, from=1-1, to=1-2]
  \arrow["R'"', "\shortmid"{marking}, no head, from=2-1, to=2-2]
\end{tikzcd}\]


\textbf{TODO: do we still need this?}
We also want something like
\[ F_c : \mathcal V_u^{op} \to \mathcal E_d \]
\[ U_c : \mathcal E_d^{op} \to \mathcal V_u \]
which ensures that if $R$ is a value edge equivalent to $A(u,-)$ then
\[ F(R) = F(A(u,-)) = (F A)(-,F u) \]


\subsection{Intensional Model}

An intensional model of gradual typing is defined similarly to an extensional model,
but unlike in the extensional model, here we work internally to the category of
categories, giving us a double category.

In particular, as before, we have cartesian categories $\mathcal V_f$,
$\mathcal V_e$, and $\mathcal V_{sq}$, in addition to $\mathcal E_f$,
$\mathcal E_e$, and $\mathcal E_{sq}$.
But we now have horizontal composition of squares as well.

Working intensionally means we need to take into consideration the steps
taken by terms. One consequence of this is that the squares in the representable
properties must now involve a notion of ``delay" or ``perturbation'' in order to
keep the function morphisms on each side in lock-step. Intuitively, the perturbations
have no effect other than to cause the function to which they are applied to ``wait''
in a specific manner.
We formalize this notion by requiring that for each object $A : \mathcal V_v$,
there is a monoid of \emph{perturbations} $P^V_A$  and a monoid homomorphism
$\ptb^V_A : P^V_A \to \mathcal V_v(A,A)$.
Similarly, for each $B : \mathcal E_v$ there is a monoid $P^C_B$ and a
homomorphism $\ptb^C_B : P^C_B \to \mathcal E_v(B,B)$.
If $\delta \in P^V_A$, we will sometimes omit the homomorphism $\ptb^V_A$ and simply write
$\delta$ to refer to the morphism $\ptb^V_A(\delta) \in \mathcal V_v(A,A)$, and likewise
for computation perturbations. The context will make clear whether we are referring
to an element of the perturbation monoid or the corresponding morphism.

The perturbations must satisfy a property that we call the ``push-pull'' property,
which is formulated as follows. Let $c \in \mathcal V_e(A, A')$.
Given a perturbation $\delta \in P^V_A$, there is a corresponding perturbation
$\push_c(\delta) \in P^V_{A'}$. % making the following square commute:
%
Likewise, given $\delta' \in P^V_{A'}$ there is a perturbation $\pull_c(\delta') \in P^V_A$.
% making the following square commute:

Moreover, the following squares commute:

\begin{center}
  \begin{tabular}{ m{9em} m{9em} } 
    \begin{tikzcd}[ampersand replacement=\&]
      A \& {A'} \\
      A \& {A'}
      \arrow["\delta"', from=1-1, to=2-1]
      \arrow["{\push_c(\delta)}", from=1-2, to=2-2]
      \arrow["c", "\shortmid"{marking}, no head, from=1-1, to=1-2]
      \arrow["c"', "\shortmid"{marking}, no head, from=2-1, to=2-2]
    \end{tikzcd}
    &
    \begin{tikzcd}[ampersand replacement=\&]
      A \& {A'} \\
      A \& {A'}
      \arrow["{\pull_c(\delta')}"', from=1-1, to=2-1]
      \arrow["{\delta'}", from=1-2, to=2-2]
      \arrow["c", "\shortmid"{marking}, no head, from=1-1, to=1-2]
      \arrow["c"', "\shortmid"{marking}, no head, from=2-1, to=2-2]
    \end{tikzcd}
  \end{tabular}
\end{center}

The analogous property should also hold for computation relations and perturbations.

A second consequence of working intensionally is that we need a way to specify
that two morphisms are the same ``up to delay'', i.e., they differ only in that
one may wait more than the other.
%
To this end, for any pair of objects $A$ and $A'$, in $\mathcal V_f$, we require that
there is a reflexive, symmetric relation $\bisim_{A,A'}$ on the hom-set $\mathcal V_f(A, A')$,
called the \emph{weak bisimilarity} relation.
Similarly for the computation category: there is a relation $\bisim_{B,B'}$
defined on each hom-set $\mathcal E_v(B, B')$.

We require that all perturbations be weakly bisimilar to the identity function,
capturing the notion that they have no effect other than to delay.

The weak bisimilarity relation should respect composition:
if $f \bisim_{A,A'} f'$ and $g \bisim_{A',A''} g'$, then
$g \circ f \bisim_{A,A''} g' \circ f'$, and likewise for computations.

% We similarly have thin subcategories $\mathcal V_u$ and $\mathcal E_d$ of
% upcasts and downcasts. The relation between function morphisms and edges
% is as follows.

As is the case in the extensional model, there is a relationship between
function morphisms and relation morphisms, but as mentioned above, now there
are perturbations involved in order to keep both morphisms in lock-step.
The precise definitions are as follows.

Let $c \in \mathcal V_e(A, A')$. We say that $c$ is \emph{quasi-left-representable by}
$f \in \mathcal V_f(A, A')$ if there exist perturbations $\delta_c^{l,e} \in P^V_A$ and
$\delta_c^{r,e} \in P^V_{A'}$ such that the following squares commute:

\begin{center}
  \begin{tabular}{ m{7em} m{7em} } 
    % UpL
    \begin{tikzcd}[ampersand replacement=\&]
      A \& {A'} \\
      {A'} \& {A'}
      \arrow["f"', from=1-1, to=2-1]
      \arrow["\delta_c^{r,e}", from=1-2, to=2-2]
      \arrow["c", "\shortmid"{marking}, no head, from=1-1, to=1-2]
      \arrow[from=2-1, to=2-2, Rightarrow, no head]
    \end{tikzcd}
    &
    % UpR
    \begin{tikzcd}[ampersand replacement=\&]
      A \& {A} \\
      {A} \& {A'}
      \arrow["\delta_c^{l,e}"', from=1-1, to=2-1]
      \arrow["f", from=1-2, to=2-2]
      \arrow[from=1-1, to=1-2, Rightarrow, no head]
      \arrow["c"', "\shortmid"{marking}, no head, from=2-1, to=2-2]
    \end{tikzcd}
  \end{tabular}
\end{center}

Similarly, let $d \in \mathcal E_e(B, B')$. We say that $d$ is
\emph{quasi-right-representable by} $f \in \mathcal E_f(B', B)$
if there exist perturbations $\delta_d^{l,p} \in P^C_B$ and
$\delta_d^{r,p} \in P^C{B'}$ such that the following squares commute:

\begin{center}
  \begin{tabular}{ m{7em} m{7em} } 
    % DnR
    \begin{tikzcd}[ampersand replacement=\&]
      {B} \& {B'} \\
      {B} \& {B}
      \arrow["\delta_d^{l,p}"', from=1-1, to=2-1]
      \arrow["g", from=1-2, to=2-2]
      \arrow["R", "\shortmid"{marking}, no head, from=1-1, to=1-2]
      \arrow[from=2-1, to=2-2, Rightarrow, no head]
    \end{tikzcd}
    &
    % DnL
    \begin{tikzcd}[ampersand replacement=\&]
      {B'} \& {B'} \\
      {B} \& {B'}
      \arrow["g"', from=1-1, to=2-1]
      \arrow["\delta_d^{r,p}", from=1-2, to=2-2]
      \arrow[from=1-1, to=1-2, Rightarrow, no head]
      \arrow["R"', "\shortmid"{marking}, no head, from=2-1, to=2-2]
    \end{tikzcd}
  \end{tabular}
\end{center}


Besides the perturbations, one other difference between the extensional
and intensional versions of the representability axioms is that in the
extensional setting, the rules build in the notion of composition, whereas
their intensional counterparts do not.
In the extensional setting, we do not have horizontal composition of squares, which
is required to derive the versions of the rules that build in composition
from the versions that do not.
In the intensional setting, we do have horizontal composition of squares,
so we can take the simpler versions as primitive and derive the ones
involving composition (though we must be careful about the perturbations
when doing so!).

\subsection{Constructing an Extensional Model from an Intensional Model}



\section{Revised Categorical Models of Graduality}\label{sec:abstract-models}

Next, we develop our appropriate weakened notions of models of
graduality, which we divide into \emph{extensional} models, where
ordering is up to weak bisimilarity, and \emph{intensional} models,
where ordering relates terms that considers computational steps to be
observable. We develop the notion of intensional model of gradual
typing in stages and show how to develop one from a base model of
effectful functions and relations.

\subsection{Extensional Models of Gradual Typing}

Since we lack transitivity of ordering when reasoning in guarded type
theory, our weakened notion of extensional model is based on reflexive
graph categories rather than double categories. This means we lose the
reasoning principle of horizontal pasting of squares. We will still
require a notion of \emph{composition} of relations, to model the
transitivity of type precision. We note that without horizontal
pasting of squares, the notion of left/right representability of
squares is not sufficient to interpret the cast rules of gradual
typing. Instead we generalize the notion of representability to match
the syntactic rules in Section~\ref{sec:GTLC}.

Let $c : A \rel A'$ and $f : A \to A'$ in a reflexive graph category
with composition of relations. We say that $c$ is \emph{universally
left-representable by} $f$ if for any $c_l : A_l \rel A$ and $c_r : A'
\rel A_r$ we have $f \ltsq{cc_r}{c_r} \id$ and $\id \ltsq{c_l}{c_lc}
f$. Dually, let $d : B \rel B'$ and $\phi : B' \to B$.  We say that $d$
is \emph{universally right-representable by} $\phi$ if for any $d_l : B_l
\rel B$ and $d_r : B' \rel B_r$ we have $\id \ltsq{d_ld}{d_l} \phi$
and $\phi \ltsq{d_r}{dd_r}\id$. In a reflexive graph category these
are stronger than the definitions of left/right representability since
we can pick $c_l,c_r,d_l,d_r$ to be the reflexive relations. In a
double category these are equivalent, but the equivalence uses
horizontal pasting.

Additionally, while in the presence of horizontal pasting
compositionality of representable morphisms is automatic, without this
principle we must require it explicitly. Finally, we explicitly add in
a requirement that the type constructors are functorial on relations
up to order equivalence.

In summary, an extensional model consists of:
\begin{enumerate}
  \item A locally thin reflexive graph internal to CBPV models.
  \item Composition of value and computation relations that form a category with the reflexive relations as identity. Call these categories $\mathcal V_r,\mathcal E_r$
  \item Identity-on-objects functors $\upf : \mathcal V_r \to \mathcal V_f$ and $\dnf : \mathcal E_r^{op} \to \mathcal E_f$ such that every $\upf\, c$ universally left-represents $c$ and every $\dnf\, d$ universally right-represents $d$.
  \item The CBPV connectives $U,F,\times,\to$ are all \emph{covariant} functorial on relations up to equivalence: $U(dd') \equidyn U(d)U(d')$ etc.\footnote{the reflexive graph structure already requires that these functors preserve identity relations}
    %% \begin{itemize}
    %% \item $U(dd') \equidyn U(d)U(d')$
    %% \item $F(cc') \equidyn F(c)F(c')$
    %% \item $(cc') \to (dd') \equidyn (c \to d)(c' \to d')$
    %% \item $(c_1c_1') \times (c_2c_2') \equidyn (c_1 \times c_2)(c_1'\times c_2')$
    %% \end{itemize}
    where $c \equidyn c'$ means $\id \ltsq{c}{c'}\id$ and $\id \ltsq{c'}{c} \id$.
  \item A natural transformation $\mho : 1 \Rightarrow U$ such that
    $\mho \circ {!} \ltsq{r(A)}{r(UB)} f$ for any $f : A \to UB$
  \item Distinguished value type $\nat$ with morphisms $z : \mathcal
    V(1,\nat)$ and $s : \mathcal V(\nat,\nat)$.
  \item Distinguished value types $D$ with distinguished relations
    $\iarr{}: U(D \to F D) \rel D$ and $\inat : \nat \rel D$ and $\itimes : D \times D \rel D$
    each satisfying the retraction property $\dnc {\iarr{}}F(\upc{\iarr{}}) \equidyn \id$.
\end{enumerate}

This mathematical structure is sufficient to interpret the theory of
gradual typing in Section~\ref{sec:GTLC}.
\begin{definition}
  Given any extensional gradual typing model, we interpret
  \begin{enumerate}
  \item Each type $A$ as a value type, interpreting the base types and
    $\times$ as their semantic analogues and $\sem {A \ra A'}$ as $U(\sem{A} \to
    F(\sem{A'}))$.
  \item Type precision derivations $c : A \ltdyn A'$ are
    interpreted as relation morphisms $\sem{c} : \sem{A} \rel \sem{A'}$ in the obvious
    way. Every equivalence axiom $c \equiv c'$ implies that $\sem{c} \equidyn \sem{c'}$.
  \item Every term $\Gamma \vdash M : A$ is interpreted as a morphism
    $\sem{M} : \mathcal V_f (\times\sem{\Gamma},UF\sem{A})$. Upcasts are interpreted as
    $UF(\upf\, {\sem{c}})$ and downcasts as $U (\dnf\, {(F\sem{c})})$.
  \item If $\Delta \vdash M \ltdyn M' : c$ then $\sem{M}
    \ltsq{\times\sem{\Delta}}{\sem{c}} \sem{M'}$ holds.
  \end{enumerate}
\end{definition}

%% %% A model $\mathcal{M}$ of extensional gradual typing consists of the
%% %% following:
%% \begin{itemize}
%% \item A reflexive graph internal to the category of CBPV models and
%%   strict morphisms that is locally thin: there is at most one morphism
%%   in $\mathcal V_{sq}$ for any given domain, codomain, source and
%%   target, and as well for $\mathcal E$.
%% \item 
%% \item A natural transformation $\mho : 1 \Rightarrow U$ satisfying $\mho \circ ! \ltsq{r(A)}{r(B)} f$ for every $f$.
%% \item Such that all relation value morphisms are \emph{universally}
%%   left representable and all relation computation morphisms are
%%   universally right representable.
%% \end{itemize}

%% satisfying certain additional conditions that will be described below.

%% % % https://q.uiver.app/#q=WzAsMixbMCwxLCJcXG1hdGhjYWx7TX1fe3NxfSJdLFswLDAsIlxcbWF0aGNhbHtNfV9mIl0sWzEsMCwiciJdLFswLDEsInMiLDAseyJjdXJ2ZSI6LTJ9XSxbMCwxLCJ0IiwyLHsiY3VydmUiOjJ9XV0=
%% % \[\begin{tikzcd}[ampersand replacement=\&]
%% % 	{\mathcal{M}_f} \\
%% % 	{\mathcal{M}_{sq}}
%% % 	\arrow["r", from=1-1, to=2-1]
%% % 	\arrow["s", curve={height=-12pt}, from=2-1, to=1-1]
%% % 	\arrow["t"', curve={height=12pt}, from=2-1, to=1-1]
%% % \end{tikzcd}\]

%% Spelling this out in light of the above definitions, we see that this is
%% equivalent to the following in the category $\textbf{Cat}$:

%% % https://q.uiver.app/#q=WzAsNCxbMCwyLCJcXHZzcSJdLFsyLDIsIlxcZXNxIl0sWzAsMCwiXFx2ZiJdLFsyLDAsIlxcZWYiXSxbMiwzLCJcXEZmIiwwLHsiY3VydmUiOi0yfV0sWzMsMiwiXFxVZiIsMCx7ImN1cnZlIjotMn1dLFswLDEsIlxcRnNxIiwwLHsiY3VydmUiOi0yfV0sWzEsMCwiXFxVc3EiLDAseyJjdXJ2ZSI6LTJ9XSxbMiwwLCJcXHJ2Il0sWzAsMiwiXFxzdiIsMCx7ImN1cnZlIjotMn1dLFsyLDAsIlxcdHYiLDAseyJjdXJ2ZSI6LTJ9XSxbMSwzLCJcXHNlIiwwLHsiY3VydmUiOi0yfV0sWzMsMSwiXFx0ZSIsMCx7ImN1cnZlIjotMn1dLFszLDEsIlxccmUiXSxbNCw1LCJcXGJvdCIsMSx7InNob3J0ZW4iOnsic291cmNlIjoyMCwidGFyZ2V0IjoyMH0sInN0eWxlIjp7ImJvZHkiOnsibmFtZSI6Im5vbmUifSwiaGVhZCI6eyJuYW1lIjoibm9uZSJ9fX1dLFs2LDcsIlxcYm90IiwxLHsic2hvcnRlbiI6eyJzb3VyY2UiOjIwLCJ0YXJnZXQiOjIwfSwic3R5bGUiOnsiYm9keSI6eyJuYW1lIjoibm9uZSJ9LCJoZWFkIjp7Im5hbWUiOiJub25lIn19fV1d
%% \[\begin{tikzcd}[ampersand replacement=\&]
%% 	\vf \&\& \ef \\
%% 	\\
%% 	\vsq \&\& \esq
%% 	\arrow[""{name=0, anchor=center, inner sep=0}, "\Ff", curve={height=-12pt}, from=1-1, to=1-3]
%% 	\arrow[""{name=1, anchor=center, inner sep=0}, "\Uf", curve={height=-12pt}, from=1-3, to=1-1]
%% 	\arrow[""{name=2, anchor=center, inner sep=0}, "\Fsq", curve={height=-12pt}, from=3-1, to=3-3]
%% 	\arrow[""{name=3, anchor=center, inner sep=0}, "\Usq", curve={height=-12pt}, from=3-3, to=3-1]
%% 	\arrow["\rv", from=1-1, to=3-1]
%% 	\arrow["\sv", curve={height=-12pt}, from=3-1, to=1-1]
%% 	\arrow["\tv", curve={height=-12pt}, from=1-1, to=3-1]
%% 	\arrow["\se", curve={height=-12pt}, from=3-3, to=1-3]
%% 	\arrow["\te", curve={height=-12pt}, from=1-3, to=3-3]
%% 	\arrow["\re", from=1-3, to=3-3]
%% 	\arrow["\bot"{description}, draw=none, from=0, to=1]
%% 	\arrow["\bot"{description}, draw=none, from=2, to=3]
%% \end{tikzcd}\]


% The above definition can be interpreted in any compact closed equipment
% (if someone were to figure out a definition for a compact closed
% equipment, that is,\ldots). Then we can get a model of a form of GTT
% by taking a CBPV object in the equipment of \emph{reflexive graph
% categories}. Since reflexive graphs form a topos we can get at this by
% interpreting the above definition \emph{internally} to the topos of
% reflexive graphs. Essentially what this means is that everything above
% has a ``vertex'' component and an ``edge'' component, so we get a
% cartesian category $\mathcal V_f$ which we think of as the value types
% and pure functions but we also get a cartesian category $\mathcal V_{sq}$
% which we think of as the ``value edges'' and ``squares''.

%% That is, for the values, we have

%% \begin{enumerate}
%%   \item A cartesian category $\mathcal V_f$.
%%   The objects of $\mathcal V_f$ will be called \emph{value types}.
%%   The morphisms of $\mathcal V_f$ will be called \emph{(pure) functions}.
  
%%   \item A cartesian category $\mathcal V_{sq}$. 
%%   The objects of $\mathcal V_{sq}$ will be called \emph{value edges} or
%%   \emph{value relations}, and the morphisms are \emph{commuting squares}.

%%   \item Functors $\sv, \tv : \mathcal V_{sq} \to \mathcal V_f$ and 
%%   $\rv : \mathcal V_f \to \mathcal V_{sq}$.
%% \end{enumerate}

%% Likewise, we have the analogous definitions for computations.

%% We write $c : A \rel A'$ to mean that $c \in \ob(\vsq)$ such that 
%% $\sv(c) = A$ and $\tv(c) = A'$.
%% %
%% Likewise, let $c_i : A_i \rel A_i'$ and $c_o : A_o \rel A_o'$,
%% and let $f \in \vf(A_i, A_o)$ and $f' \in \vf(A_i', A_o')$.
%% The notation $\beta : f \ltdyn_{c_o}^{c_i} f'$ is defined to mean

%% \begin{enumerate}
%%   \item $\beta \in \vsq(c_i, c_o)$
%%   \item $\sv(\beta) = f$
%%   \item $\tv(\beta) = f'$
%% \end{enumerate}

%% (Recall that $\sv$ and $\tv$ are functors, so in addition to acting on
%% the objects of $\vsq$ they also act on morphisms.)

%% Picorially, this is depicted as a commuting square:

%% % https://q.uiver.app/#q=WzAsNCxbMCwwLCJBX2kiXSxbMSwwLCJBX2knIl0sWzAsMSwiQV9vIl0sWzEsMSwiQV9vJyJdLFswLDIsImYiLDJdLFsxLDMsImYnIl0sWzAsMSwiY19pIiwwLHsic3R5bGUiOnsiYm9keSI6eyJuYW1lIjoiYmFycmVkIn0sImhlYWQiOnsibmFtZSI6Im5vbmUifX19XSxbMiwzLCJjX28iLDIseyJzdHlsZSI6eyJib2R5Ijp7Im5hbWUiOiJiYXJyZWQifSwiaGVhZCI6eyJuYW1lIjoibm9uZSJ9fX1dLFs0LDUsIlxcYWxwaGEiLDEseyJzaG9ydGVuIjp7InNvdXJjZSI6MjAsInRhcmdldCI6MjB9LCJzdHlsZSI6eyJib2R5Ijp7Im5hbWUiOiJub25lIn0sImhlYWQiOnsibmFtZSI6Im5vbmUifX19XV0=
%% \[\begin{tikzcd}[ampersand replacement=\&]
%% 	{A_i} \& {A_i'} \\
%% 	{A_o} \& {A_o'}
%% 	\arrow[""{name=0, anchor=center, inner sep=0}, "f"', from=1-1, to=2-1]
%% 	\arrow[""{name=1, anchor=center, inner sep=0}, "{f'}", from=1-2, to=2-2]
%% 	\arrow["{c_i}", "\shortmid"{marking}, no head, from=1-1, to=1-2]
%% 	\arrow["{c_o}"', "\shortmid"{marking}, no head, from=2-1, to=2-2]
%% 	\arrow["\alpha"{description}, draw=none, from=0, to=1]
%% \end{tikzcd}\]

%% % Note: When the identity of the square $\beta$ is not important, we may omit it
%% % and write $f \ltdyn_{c_o}^{c_i} f'$. In this case the meaning is that there exists
%% % a square $\beta : f \ltdyn_{c_o}^{c_i} f'$.

%% Composition of squares $\beta : f \ltdyn_{c_2}^{c_1} g$ and $\beta' : f' \ltdyn_{c_3}^{c_2} g'$
%% corresponds to ``stacking'' the square for $\beta'$ below the square for $\beta$.
%% Fuctoriality of $s$ and $t$ ensure that the left and right sides of the resulting square are as expected,
%% i.e., we get $\beta' \circ \beta : f' \circ f \ltdyn_{c_3}^{c_1} g' \circ g$.

%% % Fuctoriality of $s$ and $t$ ensure that we can ``vertically" compose 
%% % $\beta : f \ltdyn_{c_2}^{c_1} g$ and $\beta' : f' \ltdyn_{c_3}^{c_2} g'$
%% % to obtain $\beta' \circ \beta : f' \circ f \ltdyn_{c_3}^{c_1} g' \circ g$.
%% % Pictorially, this is represented by ``stacking'' the square for
%% % $\beta'$ below the square for $\beta$.

%% All of the above holds in an analogous manner for the computations.


%% We will work in ``locally thin'' models where there is at most one
%% square with a given boundary.
%% That is, if $\beta, \beta' : f \ltdyn_{c_o}^{c_i} f'$, then $\beta = \beta'$.
%% Thus, we may unambiguously omit the identity of the square, i.e., we may write
%% $f \ltdyn_{c_o}^{c_i} f'$.

%% We also have a ``horizontal" composition operation on value edges and on computation edges.
%% That is, let 
%% %
%% \[ X = \{ (c, c') \in \ob(\vsq) \times \ob(\vsq) \mid \tv(c) = \sv(c') \}. \]
%% %
%% There is an operation $\comp : X \to \ob(\vsq)$ such that $\sv(c \comp c') = \sv(c)$
%% and $\tv(c \comp c') = \tv(c')$. Likewise for computations.
%% %
%% Importantly, we emphasize that this composition is \emph{not} a functor: we only
%% require that it act on \emph{objects} of $\vsq$ (i.e. edges) and not on the morphisms
%% (i.e. the squares). Intuivitely, this has to do with the fact that in the extensional
%% setting, the semantic term precision function is \emph{not} transitive.

%% With this definition, is easily shown that there is a category $\ve$ of value relations,
%% whose objects are the objects of $\vf$, and such that $\ve(A, A')$ is the set of objects
%% $c$ of $\vsq$ whose source is $A$ and whose target is $A'$.
%% Composition of morphisms is defined using the above operation $\comp$.
%% Similarly, we have a category $\ee$ of computation relations.


%% We also require that the category $\ve$ of relations is thin ``up to an
%% identity square'', i.e., for any $c, c' \in \ve(A, A')$ we have that the
%% following square commutes:

%% \[\begin{tikzcd}[ampersand replacement=\&]
%%   A \& {A'} \\
%%   A \& {A'}
%%   \arrow[from=1-1, to=2-1, Rightarrow, no head]
%%   \arrow[from=1-2, to=2-2, Rightarrow, no head]
%%   \arrow["c", "\shortmid"{marking}, no head, from=1-1, to=1-2]
%%   \arrow["c'"', "\shortmid"{marking}, no head, from=2-1, to=2-2]
%% \end{tikzcd}\]

% In addition to the ordinary universal properties above, when working
% with reflexive graph models we also have access to new notions of
% universal property that relate the ``function'' morphisms to the
% ``edges''.

%% Then we formulate the relationship between relation morphisms and
%% function morphisms as follows:
%% \begin{enumerate}
%% \item There is an identity-on-objects functor $\upf : \mathcal V_e \to
%%   \mathcal V_f$ such that every $c$ is left-representable by $\upf(c)$.
%% \item There is an identity-on-objects functor $\dnf : \mathcal
%%   E_e^{op} \to \mathcal \mathcal E_f$ such that every $d$ is
%%   right-representable by $\dnf(d)$.
%% \end{enumerate}

%% \textbf{TODO: do we still need this?}
%% We also want something like
%% \[ F_c : \mathcal V_u^{op} \to \mathcal E_d \]
%% \[ U_c : \mathcal E_d^{op} \to \mathcal V_u \]
%% which ensures that if $R$ is a value edge equivalent to $A(u,-)$ then
%% \[ F(R) = F(A(u,-)) = (F A)(-,F u) \]

%%%%%%%%%%%%%%%%%%%%%%%%%%%%%%%%%%%%%%%%%%%%%%%%%%%%%%%%%%%%%%%%%%%%%%%%%%%%%%%%%%%%
%%%%%%%%%%%%%%%%%%%%%%%%%%%%%%%%%%%%%%%%%%%%%%%%%%%%%%%%%%%%%%%%%%%%%%%%%%%%%%%%%%%%

\subsection{Intensional Models}\label{sec:abstract-intensional-models}

An intensional model of gradual typing is defined similarly to an extensional model,
with a few key differences that will be discussed below.
To aid in understanding the definition of an intensional model, we break the
definition into five steps, with each one building on the previous.
The last step will be the definition we have in mind when we refer to an intensional model.

% This also facilitates the construction of an intensional model by breaking it down
% into a few modular constructions between each subsequent phase.

%
The starting point is similar to that of the extensional model.
This time, however, since we are working intensionally, the semantic denotation of
term precision \emph{is} transitive, so we \emph{do} have a horizontal composition
operation on squares. Compare this to the extensional case, where we could only
compose \emph{relations} horizontally, not squares.
% What this means is that we can define a functor for composition of value relations
% and squares, and a functor for composition of computation relations and squares.
We can specify this compactly as a category internal to the category of CBPV models
and lax morphisms, where we require that the reflexivity, source, and target morphisms
are strict.
In particular, as in the extensional case, there is a CBPV model of ``objects''
$\mathcal M_f$ and a CBPV model of ``arrows'' $\mathcal M_{sq}$. 
There are strict CBPV morphisms
$r : \mathcal M_f \to \mathcal M_{sq}$ and $s, t : \mathcal M_{sq} \to \mathcal M_f$,
just as before.
%
But now, we also have a CBPV morphism $m$ from the pullback 
$\mathcal M_{sq} \times_{s = t} \mathcal M_{sq}$ to $\mathcal M_{sq}$, i.e., ``composition of arrows".
In particular, this consists of a functor 
$m_{\mathcal{V}} : \vsq \times_{\sv = \tv} \vsq \to \vsq$ for composition of value
relations/squares, and a functor $m_{\mathcal{E}} : \esq \times_{\se = \te} \esq \to \esq$
for composition of computation relations/squares.
Furthermore, $s \circ m = s \circ \pi_1$ and $t \circ m = t \circ \pi_2$.

As in the extensional model, we also require the existence of a natural transformation
$\mho : 1 \Rightarrow U$ such that $\mho \circ {!} \ltsq{r(A)}{r(UB)} f$ for any $f : A \to UB$,
and a distinguished value type $\nat$ with morphisms $z : \mathcal V(1,\nat)$ and $s : \mathcal V(\nat,\nat)$.
% Lastly, we require the existence of a morphism $\mho_B \in \vf(\Gamma, UB)$ for all $B$
% satisfying the same ordering and commutativity properties required in the extensional model.

% If we spell this all out explicitly, we end up with a definition similar to the
% one for the extensional case, but now with the addition of a functor $m_{\mathcal{V}}$ for
% composition of value relations/squares and a functor $m_{\mathcal{E}}$ for
% composition of computation relations/squares.

We call a model satisfying these properties a \emph{step-0 intensional model}.

% For the sake of ease of reference, we recap the definition of a step-0 model:
% %
% \begin{definition}
%   A \emph{step-0} model of intensional gradual typing consists of:
%   \begin{itemize}
%     \item A category internal to the category of CBPV models and lax morphisms,
%     where we require that the morphisms $r$, $s$, and $t$ are strict.
%     \item A natural transformation $\mho : 1 \Rightarrow U$ such that $\mho \circ ! \ltsq{r(A)}{r(UB)} f$ for any $f : A \to UB$.
%     \item A value type $\nat$ with morphisms $z : \mathcal V(1,\nat)$ and $s : \mathcal V(\nat,\nat)$.
%   \end{itemize}

% \end{definition}

% In particular, as before, we have cartesian categories $\mathcal V_f$,
% $\mathcal V_e$, and $\mathcal V_{sq}$, in addition to $\mathcal E_f$,
% $\mathcal E_e$, and $\mathcal E_{sq}$.
% But we now have horizontal composition of squares as well.

%%%%%%%%%%%%%%%%%%%%%%%%%%%%%%%%%%%%%%%%%%%%%%%%%%%%%%%%%%%%%%%%%%%%%%%%%%%%%%%%%%%%

\subsubsection{Bisimilarity}\label{sec:abstract-model-bisimilarity}

Working intensionally means we need to take into consideration the steps
taken by terms. One consequence of this is that we need a way to specify
that two morphisms are the same ``up to delay'', i.e., they differ only in that
one may wait more than the other.

In particular, for any pair of objects $A$ and $A'$, in $\vf$,
we require that there is a reflexive, symmetric relation $\bisim_{A,A'}$ on the
hom-set $\vf(A, A')$, called the \emph{weak bisimilarity} relation.
Similarly for the computation category: there is a reflexive, symmetric relation
$\bisim_{B,B'}$ defined on each hom-set $\ef(B, B')$.
%
Additionally, the weak bisimilarity relation should respect composition:
if $f \bisim_{A,A'} f'$ and $g \bisim_{A',A''} g'$, then
$g \circ f \bisim_{A,A''} g' \circ f'$, and likewise for computations.

% We can specify all of this abstractly via categories $\vsim$ and $\esim$ along with
% functors $\rvsim : \vf \to \vsim$ and $\svsim, \tvsim : \vsim \to \vf$,
% and likewise for computations.
% Since bisimilarity of morphisms $f$ and $f'$ requires that they share source and target,
% we require that $\svsim$ and $\tvsim$ agree on objects and likewise for $\sesim$ and $\tesim$.
% Thus, the objects of $\vsim$ are identified with $\ob(\vf)$.
% The morphisms of $\vsim$ are ``bisimilarity proofs'', analogous to the commuting squares of $\vsq$.

% There is also a ``symmetry'' endofunctor $\text{sym}_{\mathcal{V}}^\bisim : \vsim \to \vsim$
% such that $\svsim \circ \text{sym}_{\mathcal{V}}^\bisim = \tvsim$
% and $\tvsim \circ \text{sym}_{\mathcal{V}}^\bisim = \svsim$,
% and $\text{sym}_{\mathcal{V}}^\bisim \circ \text{sym}_{\mathcal{V}}^\bisim$ is the identity.
% Likewise there is a symmetry endofunctor $\text{sym}_{\mathcal{E}}^\bisim : \esim \to \esim$.

% In this setting, we write $\refl_A : A \bisim A$ to mean that $\refl_A \in \ob(\vsim)$,
% such that $\svsim(\refl_A) = A = \tvsim(\refl_A)$.
% Let $f, f' \in \vf(A_i, A_o)$.
% The judgment $\gamma : f \bisim_{A_i, A_o} f'$ is defined to mean:

% \begin{enumerate}
%   \item $\gamma \in \vsim(\refl_{A_i}, \refl_{A_o})$
%   \item $\svsim(\gamma) = f$
%   \item $\tvsim(\gamma) = f'$
% \end{enumerate}

% Spelling this all out concretely, for any pair...


% Lastly, we require that for any value object $A$, the hom-set $\ef(FA, FA)$ contains a
% distinguished morphism $\delta_{FA}^*$, such that $\delta_A^* \bisim_{FA, FA} \id_{FA}$.
% Moreover, we require that these morphisms are related in that for any
% $c : A \rel A'$, we have a square $\delta_{FA}^* \ltdyn_{Fc}^{Fc} \delta_{FA'}^*$.

Lastly, we require that for any computation object $B$, the hom-set $\vf(UB, UB)$ contains a
distinguished morphism $\delta_{UB}^*$, such that $\delta_{UB}^* \bisim_{UB, UB} \id_{UB}$.
Moreover, we require that these morphisms are related in that for any
$d : B \rel B'$, we have a square $\delta_{UB}^* \ltdyn_{Ud}^{Ud} \delta_{UB'}^*$.
We also require that these morphisms commute with computation morphisms, in the sense
that for any $\phi \in \ef(B, B')$ we have $U\phi \circ \delta_{UB_1}^* = \delta_{UB_2} \circ U\phi$.
%
Given the existence of the morphisms $\delta_{UB}^*$, we can define a computation morphism
$\delta_{FA}^* \in \ef(FA, FA)$ for all $A$ by composing the unit $\eta_A \in \vf(A, UFA)$ with
the morphism $\delta_{UFA}^* \in \vf(UFA, UFA)$, and then by the adjunction
we get a computation morphism $\delta_{FA}^ \in \ef(FA, FA)$. Moreover, we get a square
$\delta_{FA}^* \ltdyn_{Fc}^{Fc} \delta_{FA'}^*$ for all $c : A \rel A$.


\begin{definition}\label{def:step-1-model}
A \emph{step-1 intensional model} consists of all the data of a step-0 intensional model along
with:
\begin{itemize}
  \item The existence of a reflexive, symmetric relation $\bisim$ on the hom-sets $\vf(A,A')$ and $\ef(B,B')$
  such that $\bisim$ preserves composition.
  \item The existence of a distinguished value morphism $\delta_{UB}^* \bisim \id_{UB}$ for each $B$.
  \item A square $\delta_{UB}^* \ltdyn_{Ud}^{Ud} \delta_{UB'}^*$ for all $d : B \rel B'$.
  \item The commutativity condition $U\phi \circ \delta_{UB_1}^* = \delta_{UB_2} \circ U\phi$ for any $\phi \in \ef(B, B')$.
\end{itemize}
% We also require the existence of a distinguised computation morphism $\delta_{FA}^* \bisim \id_{FA}$ for each $A$,
% and a square $\delta_{FA}^* \ltdyn_{Fc}^{Fc} \delta_{FA'}^*$ for all $c : A \rel A'$.
\end{definition}

%%%%%%%%%%%%%%%%%%%%%%%%%%%%%%%%%%%%%%%%%%%%%%%%%%%%%%%%%%%%%%%%%%%%%%%%%%%%%%%%%%%%

\subsubsection{Perturbations}\label{sec:abstract-model-perturbations}

A second consequence of working intensionally is that the squares in the representable
properties must now involve a notion of ``delay" or ``perturbation'' in order to
keep the function morphisms on each side in lock-step. Intuitively, the perturbations
have no effect other than to cause the function to which they are applied to ``wait''
in a specific manner.
We formalize this notion by requiring that for each object $A$ in $\vf$,
there is a monoid of \emph{value perturbations} $P_A$ and a homomorphism of monoids
$\ptb_A : P_A \to \{ f \in \vf(A,A) \mid f \bisim \id \}$.
Similarly, for each $B : \ef$ there is a monoid $\pe_B$ of
\emph{computation perturbations} and a homomorphism of monoids 
$\ptb_B : P_B \to \{ g \in \ef(B,B) \mid g \bisim \id \}$.

% If $\delta \in P^V_A$, we will sometimes omit the homomorphism $\ptbv_A$ and simply write
% $\delta$ to refer to the morphism $\ptbv_A(\delta) \in \vf(A,A)$, and likewise
% for computation perturbations. The context will make clear whether we are referring
% to an element of the perturbation monoid or the corresponding morphism.

Note that we require that all perturbations be weakly bisimilar to the identity morphism,
capturing the notion that they have no effect other than to delay.
% We observe that
% the set of endomorphisms $f$ such that $f$ is weakly bisimilar to the identity
% forms a monoid under composition.

We will slightly abuse notation and refer to an endomorphism $f \in \vf(A, A)$ as being ``in''
the monoid of perturbations, by which we actually mean there is an element $p \in P_A$
that is mapped to $f$ under the homomorphism.

We require that $\delta_{UB}^* \in P_{UB}$ for all $B$, where $\delta_{UB}^*$ is the distinguished
morphism that is required to be present in every hom-set $\vf(UB, UB)$ per the definition
of a step-1 model.

The perturbations must be preserved by $\times$ $\timesk$, $\arr$, $\tok$, $U$, $\Uk$, $F$, and $\Fk$.
%
Perturbations must also satisfy a property that we call the ``push-pull'' property,
which is formulated as follows. Let $c : A \rel A'$.
Given a perturbation $\delta \in P_A$, there is a corresponding perturbation
$\push_c(\delta) \in P_{A'}$. % making the following square commute:
%
Likewise, given $\delta' \in P_{A'}$ there is a perturbation $\pull_c(\delta') \in P_A$.
% making the following square commute:
%
Moreover, push-pull states that there are squares 
$\delta \ltsq{c}{c} \push_c(\delta)$ and
$\pull_c(\delta') \ltsq{c}{c} \delta'$.

% \begin{center}
%   \begin{tabular}{ m{9em} m{9em} } 
%     \begin{tikzcd}[ampersand replacement=\&]
%       A \& {A'} \\
%       A \& {A'}
%       \arrow["\delta"', from=1-1, to=2-1]
%       \arrow["{\push_c(\delta)}", from=1-2, to=2-2]
%       \arrow["c", "\shortmid"{marking}, no head, from=1-1, to=1-2]
%       \arrow["c"', "\shortmid"{marking}, no head, from=2-1, to=2-2]
%     \end{tikzcd}
%     &
%     \begin{tikzcd}[ampersand replacement=\&]
%       A \& {A'} \\
%       A \& {A'}
%       \arrow["{\pull_c(\delta')}"', from=1-1, to=2-1]
%       \arrow["{\delta'}", from=1-2, to=2-2]
%       \arrow["c", "\shortmid"{marking}, no head, from=1-1, to=1-2]
%       \arrow["c"', "\shortmid"{marking}, no head, from=2-1, to=2-2]
%     \end{tikzcd}
%   \end{tabular}
% \end{center}

The analogous property should also hold for computation relations and perturbations.
The requirements are summarized below:
%
\begin{definition}\label{def:step-2-model}
  A \emph{step-2} model of intensional gradual typing consists of all the data of a step-1 model plus:
  \begin{enumerate}
    \item For each value type $A$, there is a monoid $\pv_A$ and a homomorphism of monoids
    $\ptb_A : \pv_A \to \{ f \in \vf(A,A) \mid f \bisim \id \}$.
    \item For each computation type $B$, there is a monoid $\pe_B$ and a homomorphism of monoids
    $\ptb_B : \pe_B \to \{ g \in \ef(B,B) \mid g \bisim \id \}$.
    \item For all $B$, the distinguished endomorphism $\delta_{UB}^*$ is in $\pv_{UB}$.
    % \item $\pv_A$ and a monoid homomorphism 
    %   \[ \ptbv_A : \pv_A \to \{ f \in \vf(A,A) \mid f \bisim \id \} \]
    % \item $\pe_B$ and a monoid homomorphism
    %   \[ \ptbe_B : \pe_B \to \{ g \in \ef(B,B) \mid g \bisim \id \} \]
    \item The functors $\times, \timesk$, $\arr, \tok$, $U, \Uk$, $F, \Fk$ preserve perturbations.
    \item The push-pull property holds for all $c : A \rel A'$ and all $d : B \rel B'$.
  \end{enumerate}
\end{definition}

%%%%%%%%%%%%%%%%%%%%%%%%%%%%%%%%%%%%%%%%%%%%%%%%%%%%%%%%%%%%%%%%%%%%%%%%%%%%%%%%%%%%

\subsubsection{Behavior of Casts}


% We similarly have thin subcategories $\mathcal V_u$ and $\mathcal E_d$ of
% upcasts and downcasts. The relation between function morphisms and edges
% is as follows.

As is the case in the extensional model, there is a relationship between
vertical (i.e., function) morphisms and horizontal (i.e., relation) morphisms,
but as mentioned above, now there
are perturbations involved in order to keep both sides ``in lock-step".
We begin with a step-2 intensional model as defined in the previous section,
and provide the additional conditions that axiomatize the behavior of casts.
The precise definitions are as follows.

First, let $\mathcal M$ be any double category with a notion of perturbations,
i.e., for any object $X$ in $\mathcal M$ there is a monoid $P_X$ with a monoid
homomorphism into the endomorphisms on $X$.
%
\begin{definition}\label{def:quasi-left-representable}
  Let $R$ be a horizontal morphism in $\mathcal M$ between objects $X$ and $Y$.
  We say that $R$ is \emph{quasi-left-representable by} a vertical morphism $f$ in
  $\mathcal M(X, Y)$ if there are perturbations $\delle_R \in P_X$ and
  $\delre_R \in P_Y$ such that there is a square
  $\upl : f \ltsq{R}{r(Y)} \delre_R$ and a square 
  $\upr : \delle_R \ltsq{r(X)}{R} f$.
%
%   \begin{center}
%     \begin{tabular}{ m{7em} m{7em} } 
%       % UpL
%       \begin{tikzcd}[ampersand replacement=\&]
%         X \& {Y} \\
%         {Y} \& {Y}
%         \arrow["f"', from=1-1, to=2-1]
%         \arrow["\delta_R^{r,e}", from=1-2, to=2-2]
%         \arrow["R", "\shortmid"{marking}, no head, from=1-1, to=1-2]
%         \arrow[from=2-1, to=2-2, Rightarrow, no head]
%       \end{tikzcd}
%       &
%       % UpR
%       \begin{tikzcd}[ampersand replacement=\&]
%         X \& {X} \\
%         {X} \& {Y}
%         \arrow["\delta_R^{l,e}"', from=1-1, to=2-1]
%         \arrow["f", from=1-2, to=2-2]
%         \arrow[from=1-1, to=1-2, Rightarrow, no head]
%         \arrow["R"', "\shortmid"{marking}, no head, from=2-1, to=2-2]
%       \end{tikzcd}
%     \end{tabular}
%   \end{center}
% %
%   We call the first square $\upl$ and the second square $\upr$. 
%
\end{definition}
%
\begin{definition}\label{def:quasi-right-representable}
  Let $R$ be a horizontal morphism between $X$ and $Y$. We say that $R$ is
  \emph{quasi-right-representable by} $f \in \mathcal M(Y, X)$
  if there exist perturbations $\dellp_R \in P_X$ and
  $\delrp_R \in P_Y$ such that we have a square 
  $\dnr : \dellp_R \ltsq{R}{r(X)} f$
  and a square $\dnl : f \ltsq{r(Y)}{R} \delrp_R$.
  %
  % \begin{center}
  %   \begin{tabular}{ m{7em} m{7em} } 
  %     % DnR
  %     \begin{tikzcd}[ampersand replacement=\&]
  %       {X} \& {Y} \\
  %       {X} \& {X}
  %       \arrow["\delta_R^{l,p}"', from=1-1, to=2-1]
  %       \arrow["f", from=1-2, to=2-2]
  %       \arrow["R", "\shortmid"{marking}, no head, from=1-1, to=1-2]
  %       \arrow[from=2-1, to=2-2, Rightarrow, no head]
  %     \end{tikzcd}
  %     &
  %     % DnL
  %     \begin{tikzcd}[ampersand replacement=\&]
  %       {Y} \& {Y} \\
  %       {X} \& {Y}
  %       \arrow["f"', from=1-1, to=2-1]
  %       \arrow["\delta_R^{r,p}", from=1-2, to=2-2]
  %       \arrow[from=1-1, to=1-2, Rightarrow, no head]
  %       \arrow["R"', "\shortmid"{marking}, no head, from=2-1, to=2-2]
  %     \end{tikzcd}
  %   \end{tabular}
  % \end{center}
  % %
  % We call the first square $\dnr$ and the second square $\dnl$.
  %
  \end{definition}

% TODO: Give these squares names
% \begin{definition}\label{def:quasi-left-representable}
% Let $c : A \rel A'$ be a value relation. We say that $c$ is \emph{quasi-left-representable by}
% $f \in \vf(A, A')$ if there are perturbations $\delta_c^{l,e} \in \pv_A$ and
% $\delta_c^{r,e} \in \pv_{A'}$ such that the following squares commute:

% \begin{center}
%   \begin{tabular}{ m{7em} m{7em} } 
%     % UpL
%     \begin{tikzcd}[ampersand replacement=\&]
%       A \& {A'} \\
%       {A'} \& {A'}
%       \arrow["f"', from=1-1, to=2-1]
%       \arrow["\delta_c^{r,e}", from=1-2, to=2-2]
%       \arrow["c", "\shortmid"{marking}, no head, from=1-1, to=1-2]
%       \arrow[from=2-1, to=2-2, Rightarrow, no head]
%     \end{tikzcd}
%     &
%     % UpR
%     \begin{tikzcd}[ampersand replacement=\&]
%       A \& {A} \\
%       {A} \& {A'}
%       \arrow["\delta_c^{l,e}"', from=1-1, to=2-1]
%       \arrow["f", from=1-2, to=2-2]
%       \arrow[from=1-1, to=1-2, Rightarrow, no head]
%       \arrow["c"', "\shortmid"{marking}, no head, from=2-1, to=2-2]
%     \end{tikzcd}
%   \end{tabular}
% \end{center}

% We call the first square $\upl$ and the second square $\upr$.

% \end{definition}

% \begin{definition}\label{def:quasi-right-representable}
% Let $d : B \rel B'$ be a computation relation. We say that $d$ is
% \emph{quasi-right-representable by} $f \in \ef(B', B)$
% if there exist perturbations $\delta_d^{l,p} \in \pe_B$ and
% $\delta_d^{r,p} \in \pe_{B'}$ such that the following squares commute:

% \begin{center}
%   \begin{tabular}{ m{7em} m{7em} } 
%     % DnR
%     \begin{tikzcd}[ampersand replacement=\&]
%       {B} \& {B'} \\
%       {B} \& {B}
%       \arrow["\delta_d^{l,p}"', from=1-1, to=2-1]
%       \arrow["g", from=1-2, to=2-2]
%       \arrow["R", "\shortmid"{marking}, no head, from=1-1, to=1-2]
%       \arrow[from=2-1, to=2-2, Rightarrow, no head]
%     \end{tikzcd}
%     &
%     % DnL
%     \begin{tikzcd}[ampersand replacement=\&]
%       {B'} \& {B'} \\
%       {B} \& {B'}
%       \arrow["g"', from=1-1, to=2-1]
%       \arrow["\delta_d^{r,p}", from=1-2, to=2-2]
%       \arrow[from=1-1, to=1-2, Rightarrow, no head]
%       \arrow["R"', "\shortmid"{marking}, no head, from=2-1, to=2-2]
%     \end{tikzcd}
%   \end{tabular}
% \end{center}

% We call the first square $\dnr$ and the second square $\dnl$.

% \end{definition}

With these definitions, we return to the more specific setting of a step-2
intensional model $\mathcal M$ and specify the new requirements for relations.
We require that there are functors $\upf : \ve \to \vf$ and $\dnf : \ee^{op} \to \ef$
Every value edge $c : A \rel A'$ must be quasi-left-representable by $\upf(c)$,
and every computation edge $d : B \rel B'$ is quasi-right-representable by $\dnf(d)$.

Besides the perturbations, one other difference between the extensional
and intensional versions of the representability axioms is that in the
extensional setting, the rules build in the notion of composition, whereas
their intensional counterparts do not.
In the extensional setting, we do not have horizontal composition of squares, which
is required to derive the versions of the rules that build in composition
from the versions that do not.
In the intensional setting, we do have horizontal composition of squares,
so we can take the simpler versions as primitive and derive the ones
involving composition.

Lastly, we require that the model satisfy a weak version of functoriality for 
the CBPV connectives $U,F,\times,\to$. 
First, we will need a definition:
%
\begin{definition}[quasi-order-equivalence]\label{def:quasi-order-equivalent}
  Let $c, c' : A \rel A'$. We say that $c$ and $c'$ are \emph{quasi-order-equivalent},
  written $c \qordeq c'$, if there exist perturbations $\delta^l_1, \delta^l_2 \in \pv_A$ and 
  $\delta^r_1, \delta^r_2 \in \pv_{A'}$ such that there is a square
  $\delta^l_1 \ltsq{c}{c'} \delta^r_1$ and a square $\delta^l_2 \ltsq{c'}{c} \delta^r_2$.
%
  % \begin{center}
  %   \begin{tabular}{ m{9em} m{9em} } 
  %     \begin{tikzcd}[ampersand replacement=\&]
  %       A \& {A'} \\
  %       A \& {A'}
  %       \arrow["\delta^l_1"', from=1-1, to=2-1]
  %       \arrow["\delta^r_1", from=1-2, to=2-2]
  %       \arrow["c", "\shortmid"{marking}, no head, from=1-1, to=1-2]
  %       \arrow["c'"', "\shortmid"{marking}, no head, from=2-1, to=2-2]
  %     \end{tikzcd}
  %     &
  %     \begin{tikzcd}[ampersand replacement=\&]
  %       A \& {A'} \\
  %       A \& {A'}
  %       \arrow["\delta^l_2"', from=1-1, to=2-1]
  %       \arrow["\delta^r_2", from=1-2, to=2-2]
  %       \arrow["c'", "\shortmid"{marking}, no head, from=1-1, to=1-2]
  %       \arrow["c"', "\shortmid"{marking}, no head, from=2-1, to=2-2]
  %     \end{tikzcd}
  %   \end{tabular}
  % \end{center}
%
  We make the analogous definition for computation relations $d, d' : B \rel B'$.
\end{definition}
%
We require that the CBPV connectives $U,F,\times,\to$ are \emph{quasi-functorial} on relations,
which we specify as follows, with $c \comp c'$ denoting the relational composition of $c$ and $c'$:
\begin{itemize}
  \item $U(d \comp d') \qordeq U(d) \comp U(d')$
  \item $F(c \comp c') \qordeq F(c) \comp F(c')$
  \item $(c \comp c') \to (d \comp d') \qordeq (c \to d) \comp (c' \to d')$
  \item $(c_1 \comp c_1') \times (c_2 \comp c_2') \qordeq (c_1 \times c_2) \comp (c_1'\times c_2')$
\end{itemize}

We summarize the requirements of a step-3 model below:
%
\begin{definition}\label{def:step-3-model}
  A \emph{step-3 intensional model} consists
  of all the data of a step-2 intensional model, such that additionally:
  \begin{enumerate}
    \item There are functors $\upf : \vr \to \vf$ and $\dnf : \er^{op} \to \ef$ % TODO: image is thin?
    \item Every value edge $c : A \rel A'$ is quasi-left-representable by $\upf(c)$ and
    every computation edge $d : B \rel B'$ is quasi-right-representable by $\dnf(d)$.
    \item The CBPV connectives $U,F,\times,\to$ are quasi-functorial on relations.
  \end{enumerate}
\end{definition}

% Want: U d \comp U d' = U(d \comp d')
%       F c \comp F c' = F(c \comp c')
% Add requirement: Either the model is functorial with respect to up/downcasts or with repsect to relations
% 

%%%%%%%%%%%%%%%%%%%%%%%%%%%%%%%%%%%%%%%%%%%%%%%%%%%%%%%%%%%%%%%%%%%%%%%%%%%%%%%%%%%%

\subsubsection{The Dynamic Type}

Now we can discuss what it means for an intensional model to model the dynamic type.
% This applies to any of the above abstract model definitions, i.e., steps 0-3.

\begin{definition}\label{def:step-4-model}
  % A \emph{step-$i$ intensional model with dyn} is a step-$i$ model $\mathcal M$ such that:
  A step-4 intensional model is a step-3 intensional model $\mathcal M$ such that:
  %a distinguished value object $D \in \ob(\vf)$ such that:
  %
  \begin{enumerate}
    \item There is a distinguished value object $D \in \ob(\vf)$.
    \item There are distinguished value relations 
    $\iarr{}: U(D \to F D) \rel D$ and $\inat : \nat \rel D$ and $\itimes : D \times D \rel D$
    each satisfying the retraction property up to bisimilarity.
    %  $\dnc {\injarr{}}F(\upc{\injarr{}}) \equidyn \id$.
   
    %\item For each value type $A$, there is a value relation $\text{inj}_A : A \rel D$.
    
    %\item \eric{Do we need this?} If $c : A \rel A'$, then $\text{inj}_{A} = c \comp \text{inj}_{A'}$.
  \end{enumerate}
\end{definition}




% (By definition of a step-3 model, this relation satisfies the push-pull property and is
% quasi-left-representable.)

% (By definition of a step-3 model, this means there is also a monoid $\pv_D$ of
% perturbations and a homomorphism $\ptbv_D$.)



%%%%%%%%%%%%%%%%%%%%%%%%%%%%%%%%%%%%%%%%%%%%%%%%%%%%%%%%%%%%%%%%%%%%%%%%%%%%%%%%%%%%
%%%%%%%%%%%%%%%%%%%%%%%%%%%%%%%%%%%%%%%%%%%%%%%%%%%%%%%%%%%%%%%%%%%%%%%%%%%%%%%%%%%%

\subsection{Constructing an Extensional Model}\label{sec:extensional-model-construction}

In the previous section, we have given the definition of an intensional model
of gradual typing as a series of steps with each definition building on the previous one.
%
Here, we discuss how to construct an extensional model from a step-4 intensional model.
We do so in several phases, beginning with a step-1 intensional model with dyn
and ending with an extensional model.
Moreover, this construction is \emph{modular}, in that each phase of the
construction does not depend on the details of the previous ones.
% However, this process cannot proceed in isolation: some phases require
% additional inputs. We will make clear what data must be supplied to each phase.

\subsubsection{Adding Perturbations}\label{sec:constructing-perturbations}

Suppose we have a \hyperref[def:step-1-model]{step-1 intensional model} $\mathcal{M}$.
Recall that a step-1 intensional model consists of a step-0 model (i.e., a
category internal to the category of CBPV models), along with the necessary
categories and functors for bisimilarity as discussed in Section
\ref{sec:abstract-model-bisimilarity}.
Further, recall that a \hyperref[def:step-2-model]{step-2 model} has everything
a step-1 model has, with the addition of perturbation monoids $\pv_A$ for all
$A$ and $\pe_B$ for all $B$.
Moreover, the push-pull property must hold for all
value relations $c$ and all computation relations $d$.

We claim that from a step-1 model, we can construct a step-2 model. 
The value objects of the model are defined to be triples $(A, P_A, \ptb_A)$ where $A$ is a
value object in $\mathcal{M}$, $P_A$ is a monoid and $\ptb_A$ is a homomorphism of monoids
from $P_A$ to the endomorphisms on $A$ that are bisimilar to the identity.
Likewise, computation objects are triples $(B, P_B, \ptb_B)$.
The morphisms are the same as the morphisms of $\mathcal{M}$.
A value relation between $(A, P_A, \ptb_A)$ and $(A', P_{A'}, \ptb_{A'})$ is given by
a pair of a relation in $c$ and a \emph{push-pull structure} $\Pi_c$ specifying that
that $c$ satisfies the push-pull property. Computation relations are defined analogously.
The squares are the same as those of $\mathcal{M}$.
We define the action of the functor $F$ on objects as $F(A, P_A, \ptb_A) = 
(FA, \mathbb{N} \times P_A, \ptb_{FA})$ where $\ptb_{FA}(n, a) = (\delta_{FA}^*)^n \circ F(\ptb_A(a))$
(i.e., we use the distinguished delay morphism $\delta_{FA}^*$).
The action of $U$ on objects is defined similarly, where the perturbations
are defined to be $\mathbb{N} \times P_B$.

For the full details of the construction, see Lemma \ref{lem:step-1-model-to-step-2-model} in the Appendix.

%%%%%%%%%%%%%%%%%%%%%%%%%%%%%%%%%%%%%%%%%%%%%%%%%%%%%%%%%%%%%%%%%%%%%%%%%%%%%%%%%%%%

\subsubsection{Adding Quasi-Representability}

Now suppose we have a step-2 intensional model $\mathcal{M}$.
We claim that we can construct a \hyperref[def:step-3-model]{step-3 intensional model} $\mathcal{M'}$.
In the construction, the objects and morphisms are the same as those of $\mathcal M$,
and a value relation between $A$ and $A'$ consists of a triple $(c, \rho^L_c, \rho^R_{Fc})$
where $c : A \rel A'$ is a relation in $\mathcal{M}$,
$\rho^L_c$ is a quasi-\emph{left}-representation for $c$ 
(i.e., an embedding $e_c$, perturbations $\delre_c \in P_{A'}$ and $\delle_c \in P_A$, 
and the two relevant squares), and similarly $\rho^R_{Fc}$ is
a quasi-\emph{right}-representation for $Fc$.
Computation relations are triples $(d, \rho^R_d, \rho^L_{Ud})$ where $d : B \rel B'$,
$\rho^R_d$ is a quasi-right-representation for $d$ and $\rho^L_{Ud}$ is a quasi-left-representation for $Ud$.
The value and computation squares are the same as those of $\mathcal{M}$.
We then define composition of relations and the action of the functors $F$, $U$, $\times$, and $\arr$.

For the full details of the construction, see Lemma \ref{lem:step-2-model-to-step-3-model} in the Appendix.


%%%%%%%%%%%%%%%%%%%%%%%%%%%%%%%%%%%%%%%%%%%%%%%%%%%%%%%%%%%%%%%%%%%%%%%%%%%%%%%%%%%%

\subsubsection{Constructing an Extensional Model}\label{sec:extensional-model-definition}

Finally, suppose $\mathcal M$ is a \hyperref[def:step-4-model]{step-4 intensional model}
(i.e., a step-3 model with an interpretation of the dynamic type).
We now describe how to build an extensional model.

The idea is to define an extensional model whose squares are the ``bisimilarity-closure''
of the squares of the provided intensional model $\mathcal M$.
  
The categories $\vf$, $\ef$ are the same as those of $\mathcal M$.
Additionally, the objects of $\vsq$ and $\esq$, i.e., the value and
computation relations, are the same.
The difference arises in the \emph{morphisms} of $\vsq$ and $\esq$,
i.e., the commuting squares. In particular, a morphism
$\alpha_e \in \vsq'(c_i, c_o)$ with source $f$ and target $g$ is given by:
\begin{itemize}
  \item a morphism $f' \in \vf(A_i, A_o)$ with $f \bisim f'$.
  \item a morphism $g' \in \vf(A_i', A_o')$ with $g \bisim g'$.
  \item a square $\alpha_i \in \vsq(c_i, c_o)$ with source $f'$ and target $g'$.
\end{itemize}

Using our existing notation, we say that $f \ltls_{c_o}^{c_i} g$ if there exist $f'$ and $g'$ such that

\[ f \bisim_{A_i,A_o} f' \ltdyn_{c_o}^{c_i} g' \bisim_{A_i',A_o'} g. \]

We make the analogous construction for the computation squares.

The proof that this indeed defines an extensional model is given in the Appendix 
(see Section \ref{sec:extensional-construction-appendix}).

% Next, we check that the requirements of an extensional model are satisfied.
% In particular, we need to verify the representability properties.
% We define functors $\upf$ and $\dnf$





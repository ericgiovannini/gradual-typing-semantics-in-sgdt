\section{Background on Synthetic Guarded Domain Theory}\label{sec:technical-background}

% Modeling the dynamic type as a recursive sum type?
% Observational equivalence and approximation?

% synthetic guarded domain theory, denotational semantics therein

%% \subsection{Difficulties in Prior Semantics}
%%   % Difficulties in prior semantics

%%   In this work, we compare our approach to proving graduality to the approach
%%   introduced by New and Ahmed \cite{new-ahmed2018} which constructs a step-indexed
%%   logical relations model and shows that this model is sound with respect to their
%%   notion of contextual error approximation.

%%   Because the dynamic type is modeled as a non-well-founded
%%   recursive type, their logical relation needs to be paramterized by natural numbers
%%   to restore well-foundedness. This technique is known as a \emph{step-indexed logical relation}.
%%   Intuitively, step-indexing makes the steps taken by terms observable.
%%   %
%%   Reasoning about step-indexed logical relations
%%   can be tedious and error-prone, and there are some very subtle aspects that must
%%   be taken into account in the proofs.
%%   %
%%   In particular, the prior approach of New and Ahmed requires two separate logical
%%   relations for terms, one in which the steps of the left-hand term are counted,
%%   and another in which the steps of the right-hand term are counted.
%%   Then two terms $M$ and $N$ are related in the ``combined'' logical relation if they are
%%   related in both of the one-sided logical relations. Having two separate logical relations
%%   complicates the statement of the lemmas used to prove graduality, because any statement that
%%   involves a term stepping needs to take into account whether we are counting steps on the left
%%   or the right. Some of the differences can be abstracted over, but difficulties arise for properties %/results
%%   as fundamental and seemingly straightforward as transitivity.

%%   Specifically, for transitivity, we would like to say that if $M$ is related to $N$ at
%%   index $i$ and $N$ is related to $P$ at index $i$, then $M$ is related to $P$ at $i$.
%%   But this does not actually hold: we require that one of the two pairs of terms
%%   be related ``at infinity'', i.e., that they are related at $i$ for all $i \in \mathbb{N}$.
%%   Which pair is required to satisfy this depends on which logical relation we are considering,
%%   (i.e., is it counting steps on the left or on the right),
%%   and so any argument that uses transitivity needs to consider two cases, one
%%   where $M$ and $N$ must be shown to be related for all $i$, and another where $N$ and $P$ must
%%   be related for all $i$. % This may not even be possible to show in some scenarios!

%%   \begin{comment}
%%   As another example, the axioms that specify the behavior of casts do not hold in the
%%   step-indexed setting. Consider, for example, the ``lower bound'' rule for downcasting:

%%   \begin{mathpar}
%%     \inferrule*
%%     {\gamlt \vdash M \ltdyn N : dyn}
%%     {\gamlt \vdash \dn{\dyn \ra \dyn}{\dyn} M \ltdyn N}
%%   \end{mathpar}

%%   In the language of the step-indexed logical relation used in prior work, this would
%%   take the form
  
%%   \begin{mathpar}
%%     \inferrule*
%%     {(M, N) \in \mathcal{E}^{\sim}_{i}\sem{\dyn}}
%%     {(\dn{\dyn \ra \dyn}{\dyn} M, N) \in \mathcal{E}^{\sim}_{i}\sem{\injarr{}}}
%%   \end{mathpar}

%%   where $\sim$ stands for $\mathrel{\preceq}$ or $\mathrel{\succeq}$, i.e.,
%%   counting steps on the left or right respectively.
%%   %
%%   To show this, we would use the fact that the left-hand side steps and
%%   apply an anti-reduction lemma, showing that the term to which the LHS steps
%%   is related to the RHS where our fuel is now
  
%%   The left-hand side steps to a case inspection on $M$,
%%   where we unfold the recursive $\dyn$ type into a sum type and see whether the result
%%   is a function type.

%%   One way around these difficulties is to demand that the rules only hold
%%   ``extensionally'', i.e., we quantify universally over the step-index and
%%   reason about the ``global'' behavior of terms for all possible step indices.
%%   This is the approach taken in prior work.
%% \end{comment}

%%   % These complications introduced by step-indexing lead one to wonder whether there is a
%%   % way of proving graduality without relying on tedious arguments involving natural numbers.
%%   % An alternative approach, which we investigate in this paper, is provided by
%%   % \emph{synthetic guarded domain theory}, as discussed below.
%%   % Synthetic guarded domain theory allows the resulting logical relation to look almost
%%   % identical to a typical, non-step-indexed logical relation.

%% \subsection{Synthetic Guarded Domain Theory}\label{sec:sgdt}
One way to avoid the tedious reasoning associated with step-indexing is to work
axiomatically inside of a logical system that can reason about non-well-founded recursive
constructions while abstracting away the specific details of step-indexing required
if we were working analytically.
The system that proves useful for this purpose is called \emph{synthetic guarded
domain theory}, or SGDT for short. We provide a brief overview here, but more
details can be found in \cite{birkedal-mogelberg-schwinghammer-stovring2011}.

SGDT offers a synthetic approach to domain theory that allows for guarded recursion
to be expressed syntactically via a type constructor $\later \colon \type \to \type$ 
(pronounced ``later''). The use of a modality to express guarded recursion
was introduced by Nakano \cite{Nakano2000}.
%
Given a type $A$, the type $\later A$ represents an element of type $A$
that is available one time step later. There is an operator $\nxt : A \to\, \later A$
that ``delays'' an element available now to make it available later.
We will use a tilde to denote a term of type $\later A$, e.g., $\tilde{M}$.

% TODO later is an applicative functor, but not a monad

There is a \emph{guarded fixpoint} operator
%
\[
  \fix : \forall T, (\later T \to T) \to T.
\]
%
That is, to construct a term of type $T$, it suffices to assume that we have access to
such a term ``later'' and use that to help us build a term ``now''.
This operator satisfies the axiom that $\fix f = f (\nxt (\fix f))$.
In particular, this axiom applies to propositions $P : \texttt{Prop}$; proving
a statement in this manner is known as $\lob$-induction.

The operators $\later$, $\nxt$, and $\fix$ described above can be indexed by objects
called \emph{clocks}. A clock serves as a reference relative to which steps are counted.
For instance, given a clock $k$ and type $T$, the type $\later^k T$ represents a value of type
$T$ one unit of time in the future according to clock $k$.
If we only ever had one clock, then we would not need to bother defining this notion.
However, the notion of \emph{clock quantification} is crucial for encoding coinductive types using guarded
recursion, an idea first introduced by Atkey and McBride \cite{atkey-mcbride2013}.

Most of the developments in this paper will take place in the context of a single clock $k$,
but later on, we will need to make use of clock quantification.

% Clocked Cubical Type Theory
\subsection{Ticked Cubical Type Theory}
% TODO motivation for Clocked Cubical Type Theory, e.g., delayed substitutions?

Ticked Cubical Type Theory \cite{mogelberg-veltri2019} is an extension of
Cubical Type Theory \cite{CohenCoquandHuberMortberg2017} that has clocks as well
as an additional sort called \emph{ticks}. Ticks were originally introduced in
\cite{bahr-grathwohl-bugge-mogelberg2017}. Given a clock $k$, a tick $t :
\tick\, k$ serves as evidence that one unit of time has passed according to the
clock $k$. In Ticked Cubical Type Theory, the type $\later^k A$ is the type of
dependent functions from ticks of the clock $k$ to $A$. The type $A$ is allowed
to depend on $t$, in which case we write $\later^k_t A$ to emphasize the
dependence.

% TODO next as a function that ignores its input tick argument?

% TODO include a figure with some of the rules for ticks

The rules for tick abstraction and application are similar to those of ordinary
$\Pi$ types. A context now consists of ordinary variables $x : A$ as well as
tick variables $t : \tick\, k$. The presence of the tick variable $t$ in context
$\Gamma, (t : \tick\, k), \Gamma'$ intuitively means that the values of the
variables in $\Gamma$ arrive ``first'', then one time step occurs on $k$, and
then the values of the variables in $\Gamma'$ arrive.

The abstraction rule for ticks states that if in context $\Gamma, t : \tick\, k$
the term $M$ has type $A$, then in context $\Gamma$ the term $\lambda t.M$ has
type $\later^k_t A$. % TODO dependent version?
%
Conversely, if we have a term $M$ of type $\later^k A$, and we have available in
the context a tick $t' : \tick\, k$, then we can apply the tick to $M$ to get a
term $M[t'] : A[t'/t]$. However, there is an important restriction on when we
are allowed to apply ticks. In order to apply $M$ to tick $t$, $M$ must be
well-typed in the prefix of the context occurring before the tick $t$. That is,
all variables mentioned in $M$ must be available before $t$. This ensures that
we cannot, for example, define a term of type $\later \laterhs A \to\, \laterhs
A$ via repeated tick application.
% TODO restriction on tick application
%
For the sake of brevity, we will also write tick application as $M_t$.

% TODO mention Agda implementation of Clocked Cubical Type Theory?
%The statements in this paper have been formalized in a variant of Agda called
%Guarded Cubical Agda \cite{veltri-vezzosi2020}, an implementation of Clocked Cubical Type Theory.


% TODO axioms (clock irrelevance, tick irrelevance)?

\section{Technical Background}\label{sec:technical-background}

\subsection{Gradual Typing}

% Cast calculi
In a gradually-typed language, the mixing of static and dynamic code is seamless, in that
the dynamically typed parts are checked at runtime. This type checking occurs at the elimination
forms of the language (e.g., pattern matching, field reference, etc.).
Gradual languages are generally elaborated to a \emph{cast calculus}, in which the dynamic
type checking is made explicit through the insertion of \emph{type casts}.

% Up and down casts
In a cast calculus, there is a relation $\ltdyn$ on types such that $A \ltdyn B$ means that
$A$ is a \emph{more precise} type than $B$.
There a dynamic type $\dyn$ with the property that $A \ltdyn\, \dyn$ for all $A$.
%
If $A \ltdyn B$, a term $M$ of type $A$ may be \emph{up}casted to $B$, written $\up A B M$,
and a term $N$ of type $B$ may be \emph{down}casted to $A$, written $\dn A B N$.
Upcasts always succeed, while downcasts may fail at runtime.
%
% Syntactic term precision
We also have a notion of \emph{syntactic term precision}.
If $A \ltdyn B$, and $M$ and $N$ are terms of type $A$ and $B$ respectively, we write
$M \ltdyn N : A \ltdyn B$ to mean that $M$ is more precise than $N$, i.e., $M$ and $N$
behave the same except that $M$ may error more.

% Modeling the dynamic type as a recursive sum type?
% Observational equivalence and approximation?

% synthetic guarded domain theory, denotational semantics therein

\subsection{Difficulties in Prior Semantics}
  % Difficulties in prior semantics

  In this work, we compare our approach to proving graduality to the approach
  introduced by New and Ahmed \cite{new-ahmed2018} which constructs a step-indexed
  logical relations model and shows that this model is sound with respect to their
  notion of contextual error approximation.

  Because the dynamic type is modeled as a non-well-founded
  recursive type, their logical relation needs to be paramterized by natural numbers
  to restore well-foundedness. This technique is known as a \emph{step-indexed logical relation}.
  Reasoning about step-indexed logical relations
  can be tedious and error-prone, and there are some very subtle aspects that must
  be taken into account in the proofs. Figure \ref{TODO} shows an example of a step-indexed logical
  relation for the gradually-typed lambda calculus.

  In particular, the prior approach of New and Ahmed requires two separate logical
  relations for terms, one in which the steps of the left-hand term are counted,
  and another in which the steps of the right-hand term are counted.
  Then two terms $M$ and $N$ are related in the ``combined'' logical relation if they are
  related in both of the one-sided logical relations. Having two separate logical relations
  complicates the statement of the lemmas used to prove graduality, becasue any statement that
  involves a term stepping needs to take into account whether we are counting steps on the left
  or the right. Some of the differences can be abstracted over, but difficulties arise for properties %/results
  as fundamental and seemingly straightforward as transitivty.

  Specifically, for transitivity, we would like to say that if $M$ is related to $N$ at
  index $i$ and $N$ is related to $P$ at index $i$, then $M$ is related to $P$ at $i$.
  But this does not actually hold: we requrie that one of the two pairs of terms
  be related ``at infinity'', i.e., that they are related at $i$ for all $i \in \mathbb{N}$.
  Which pair is required to satisfy this depends on which logical relation we are considering,
  (i.e., is it counting steps on the left or on the right),
  and so any argument that uses transitivity needs to consider two cases, one
  where $M$ and $N$ must be shown to be related for all $i$, and another where $N$ and $P$ must
  be related for all $i$. % This may not even be possible to show in some scenarios!

  % These complications introduced by step-indexing lead one to wonder whether there is a
  % way of proving graduality without relying on tedious arguments involving natural numbers.
  % An alternative approach, which we investigate in this paper, is provided by
  % \emph{synthetic guarded domain theory}, as discussed below.
  % Synthetic guarded domain theory allows the resulting logical relation to look almost
  % identical to a typical, non-step-indexed logical relation.

\subsection{Synthetic Guarded Domain Theory}
One way to avoid the tedious reasoning associated with step-indexing is to work
axiomatically inside of a logical system that can reason about non-well-founded recursive
constructions while abstracting away the specific details of step-indexing required
if we were working analytically.
The system that proves useful for this purpose is called \emph{synthetic guarded
domain theory}, or SGDT for short. We provide a brief overview here, but more
details can be found in \cite{birkedal-mogelberg-schwinghammer-stovring2011}.

SGDT offers a synthetic approach to domain theory that allows for guarded recursion
to be expressed syntactically via a type constructor $\later : \type \to \type$ 
(pronounced ``later''). The use of a modality to express guarded recursion
was introduced by Nakano \cite{Nakano2000}.
%
Given a type $A$, the type $\later A$ represents an element of type $A$
that is available one time step later. There is an operator $\nxt : A \to\, \later A$
that ``delays'' an element available now to make it available later.
We will use a tilde to denote a term of type $\later A$, e.g., $\tilde{M}$.

% TODO later is an applicative functor, but not a monad

There is a \emph{guarded fixpoint} operator

\[
  \fix : \forall T, (\later T \to T) \to T.
\]

That is, to construct a term of type $T$, it suffices to assume that we have access to
such a term ``later'' and use that to help us build a term ``now''.
This operator satisfies the axiom that $\fix f = f (\nxt (\fix f))$.
In particular, this axiom applies to propositions $P : \texttt{Prop}$; proving
a statement in this manner is known as $\lob$-induction.

The operators $\later$, $\next$, and $\fix$ described above can be indexed by objects
called \emph{clocks}. A clock serves as a reference relative to which steps are counted.
For instance, given a clock $k$ and type $T$, the type $\later^k T$ represents a value of type
$T$ one unit of time in the future according to clock $k$.
If we only ever had one clock, then we would not need to bother defining this notion.
However, the notion of \emph{clock quantification} is crucial for encoding coinductive types using guarded
recursion, an idea first introduced by Atkey and McBride \cite{atkey-mcbride2013}.


% Clocked Cubical Type Theory
\subsubsection{Ticked Cubical Type Theory}
% TODO motivation for Clocked Cubical Type Theory, e.g., delayed substitutions?

In Ticked Cubical Type Theory \cite{TODO}, there is an additional sort
called \emph{ticks}. Given a clock $k$, a tick $t : \tick k$ serves
as evidence that one unit of time has passed according to the clock $k$.
The type $\later A$ is represented as a function from ticks of a clock $k$ to $A$.
The type $A$ is allowed to depend on $t$, in which case we write $\later^k_t A$
to emphasize the dependence.

% TODO next as a function that ignores its input tick argument?

The rules for tick abstraction and application are similar to those of dependent
$\Pi$ types. 
In particular, if we have a term $M$ of type $\later^k A$, and we
have available in the context a tick $t' : \tick k$, then we can apply the tick to $M$ to get
a term $M[t'] : A[t'/t]$. We will also write tick application as $M_t$.
Conversely, if in a context $\Gamma, t : \tick k$ we have that $M$ has type $A$,
then in context $\Gamma$ we have $\lambda t.M$ has type $\later A$. % TODO dependent version?

The statements in this paper have been formalized in a variant of Agda called
Guarded Cubical Agda \cite{TODO}, an implementation of Clocked Cubical Type Theory.


% TODO axioms (clock irrelevance, tick irrelevance)?


\subsubsection{The Topos of Trees Model}

The topos of trees model provides a useful intuition for reasoning in SGDT 
\cite{birkedal-mogelberg-schwinghammer-stovring2011}.
This section presupposes knowledge of category theory and can be safely skipped without
disrupting the continuity.

The topos of trees $\calS$ is the presheaf category $\Set^{\omega^o}$. 
%
We assume a universe $\calU$ of types, with encodings for operations such
as sum types (written as $\widehat{+}$). There is also an operator 
$\laterhat \colon \later \calU \to \calU$ such that 
$\El(\laterhat(\nxt A)) =\,\, \later \El(A)$, where $\El$ is the type corresponding
to the code $A$.

An object $X$ is a family $\{X_i\}$ of sets indexed by natural numbers,
along with restriction maps $r^X_i \colon X_{i+1} \to X_i$ (see Figure \ref{fig:topos-of-trees-object}).
These should be thought of as ``sets changing over time", where $X_i$ is the view of the set if we have $i+1$
time steps to reason about it.
We can also think of an ongoing computation, with $X_i$ representing the potential results of the computation
after it has run for $i+1$ steps.

\begin{figure}
  % https://q.uiver.app/?q=WzAsNSxbMiwwLCJYXzIiXSxbMCwwLCJYXzAiXSxbMSwwLCJYXzEiXSxbMywwLCJYXzMiXSxbNCwwLCJcXGRvdHMiXSxbMiwxLCJyXlhfMCIsMl0sWzAsMiwicl5YXzEiLDJdLFszLDAsInJeWF8yIiwyXSxbNCwzLCJyXlhfMyIsMl1d
\[\begin{tikzcd}[ampersand replacement=\&]
	{X_0} \& {X_1} \& {X_2} \& {X_3} \& \dots
	\arrow["{r^X_0}"', from=1-2, to=1-1]
	\arrow["{r^X_1}"', from=1-3, to=1-2]
	\arrow["{r^X_2}"', from=1-4, to=1-3]
	\arrow["{r^X_3}"', from=1-5, to=1-4]
\end{tikzcd}\]
  \caption{An example of an object in the topos of trees.}
  \label{fig:topos-of-trees-object}
\end{figure}

A morphism from $\{X_i\}$ to $\{Y_i\}$ is a family of functions $f_i \colon X_i \to Y_i$
that commute with the restriction maps in the obvious way, that is,
$f_i \circ r^X_i = r^Y_i \circ f_{i+1}$ (see Figure \ref{fig:topos-of-trees-morphism}).

\begin{figure}
% https://q.uiver.app/?q=WzAsMTAsWzIsMCwiWF8yIl0sWzAsMCwiWF8wIl0sWzEsMCwiWF8xIl0sWzMsMCwiWF8zIl0sWzQsMCwiXFxkb3RzIl0sWzAsMSwiWV8wIl0sWzEsMSwiWV8xIl0sWzIsMSwiWV8yIl0sWzMsMSwiWV8zIl0sWzQsMSwiXFxkb3RzIl0sWzIsMSwicl5YXzAiLDJdLFswLDIsInJeWF8xIiwyXSxbMywwLCJyXlhfMiIsMl0sWzQsMywicl5YXzMiLDJdLFs2LDUsInJeWV8wIiwyXSxbNyw2LCJyXllfMSIsMl0sWzgsNywicl5ZXzIiLDJdLFs5LDgsInJeWV8zIiwyXSxbMSw1LCJmXzAiLDJdLFsyLDYsImZfMSIsMl0sWzAsNywiZl8yIiwyXSxbMyw4LCJmXzMiLDJdXQ==
\[\begin{tikzcd}[ampersand replacement=\&]
	{X_0} \& {X_1} \& {X_2} \& {X_3} \& \dots \\
	{Y_0} \& {Y_1} \& {Y_2} \& {Y_3} \& \dots
	\arrow["{r^X_0}"', from=1-2, to=1-1]
	\arrow["{r^X_1}"', from=1-3, to=1-2]
	\arrow["{r^X_2}"', from=1-4, to=1-3]
	\arrow["{r^X_3}"', from=1-5, to=1-4]
	\arrow["{r^Y_0}"', from=2-2, to=2-1]
	\arrow["{r^Y_1}"', from=2-3, to=2-2]
	\arrow["{r^Y_2}"', from=2-4, to=2-3]
	\arrow["{r^Y_3}"', from=2-5, to=2-4]
	\arrow["{f_0}"', from=1-1, to=2-1]
	\arrow["{f_1}"', from=1-2, to=2-2]
	\arrow["{f_2}"', from=1-3, to=2-3]
	\arrow["{f_3}"', from=1-4, to=2-4]
\end{tikzcd}\]
  \caption{An example of a morphism in the topos of trees.}
  \label{fig:topos-of-trees-morphism}
\end{figure}


The type operator $\later$ is defined on an object $X$ by
$(\later X)_0 = 1$ and $(\later X)_{i+1} = X_i$.
The restriction maps are given by $r^\later_0 =\, !$, where $!$ is the
unique map into $1$, and $r^\later_{i+1} = r^X_i$.
The morphism $\nxt^X \colon X \to \later X$ is defined pointwise by
$\nxt^X_0 =\, !$, and $\nxt^X_{i+1} = r^X_i$. It is easily checked that
this satisfies the commutativity conditions required of a morphism in $\calS$.
%
Given a morphism $f \colon \later X \to X$, i.e., a family of functions
$f_i \colon (\later X)_i \to X_i$ that commute with the restrictions in the appropriate way,
we define $\fix(f) \colon 1 \to X$ pointwise
by $\fix(f)_i = f_{i} \circ \dots \circ f_0$.
This can be visualized as a diagram in the category of sets as shown in
Figure \ref{fig:topos-of-trees-fix}.
% Observe that the fixpoint is a \emph{global element} in the topos of trees.
% Global elements allow us to view the entire computation on a global level.


% https://q.uiver.app/?q=WzAsOCxbMSwwLCJYXzAiXSxbMiwwLCJYXzEiXSxbMywwLCJYXzIiXSxbMCwwLCIxIl0sWzAsMSwiWF8wIl0sWzEsMSwiWF8xIl0sWzIsMSwiWF8yIl0sWzMsMSwiWF8zIl0sWzMsNCwiZl8wIl0sWzAsNSwiZl8xIl0sWzEsNiwiZl8yIl0sWzIsNywiZl8zIl0sWzAsMywiISIsMl0sWzEsMCwicl5YXzAiLDJdLFsyLDEsInJeWF8xIiwyXSxbNSw0LCJyXlhfMCJdLFs2LDUsInJeWF8xIl0sWzcsNiwicl5YXzIiXV0=
% \[\begin{tikzcd}[ampersand replacement=\&]
% 	1 \& {X_0} \& {X_1} \& {X_2} \\
% 	{X_0} \& {X_1} \& {X_2} \& {X_3}
% 	\arrow["{f_0}", from=1-1, to=2-1]
% 	\arrow["{f_1}", from=1-2, to=2-2]
% 	\arrow["{f_2}", from=1-3, to=2-3]
% 	\arrow["{f_3}", from=1-4, to=2-4]
% 	\arrow["{!}"', from=1-2, to=1-1]
% 	\arrow["{r^X_0}"', from=1-3, to=1-2]
% 	\arrow["{r^X_1}"', from=1-4, to=1-3]
% 	\arrow["{r^X_0}", from=2-2, to=2-1]
% 	\arrow["{r^X_1}", from=2-3, to=2-2]
% 	\arrow["{r^X_2}", from=2-4, to=2-3]
% \end{tikzcd}\]

% \begin{figure}
%   % https://q.uiver.app/?q=WzAsMTksWzEsMiwiWF8wIl0sWzIsMywiWF8xIl0sWzMsMSwiMSJdLFswLDEsIjEiXSxbMCwyLCJYXzAiXSxbMSwzLCJYXzEiXSxbMSwxLCIxIl0sWzIsMSwiMSJdLFsyLDIsIlhfMCJdLFsyLDQsIlhfMiJdLFszLDIsIlhfMCJdLFszLDMsIlhfMSJdLFszLDQsIlhfMiJdLFszLDUsIlhfMyJdLFs0LDIsIlxcY2RvdHMiXSxbMCwwLCJcXGZpeChmKV8wIl0sWzEsMCwiXFxmaXgoZilfMSJdLFsyLDAsIlxcZml4KGYpXzIiXSxbMywwLCJcXGZpeChmKV8zIl0sWzMsNCwiZl8wIl0sWzAsNSwiZl8xIl0sWzYsMCwiZl8wIl0sWzcsOCwiZl8wIl0sWzgsMSwiZl8xIl0sWzEsOSwiZl8yIl0sWzIsMTAsImZfMCJdLFsxMCwxMSwiZl8xIl0sWzExLDEyLCJmXzIiXSxbMTIsMTMsImZfMyJdXQ==
%   \[\begin{tikzcd}[ampersand replacement=\&]
%     {\fix(f)_0} \& {\fix(f)_1} \& {\fix(f)_2} \& {\fix(f)_3} \\
%     1 \& 1 \& 1 \& 1 \\
%     {X_0} \& {X_0} \& {X_0} \& {X_0} \& \cdots \\
%     \& {X_1} \& {X_1} \& {X_1} \\
%     \&\& {X_2} \& {X_2} \\
%     \&\&\& {X_3}
%     \arrow["{f_0}", from=2-1, to=3-1]
%     \arrow["{f_1}", from=3-2, to=4-2]
%     \arrow["{f_0}", from=2-2, to=3-2]
%     \arrow["{f_0}", from=2-3, to=3-3]
%     \arrow["{f_1}", from=3-3, to=4-3]
%     \arrow["{f_2}", from=4-3, to=5-3]
%     \arrow["{f_0}", from=2-4, to=3-4]
%     \arrow["{f_1}", from=3-4, to=4-4]
%     \arrow["{f_2}", from=4-4, to=5-4]
%     \arrow["{f_3}", from=5-4, to=6-4]
%   \end{tikzcd}\]
%   \caption{The first few approximations to the guarded fixpoint of $f$.}
%   \label{fig:topos-of-trees-fix-approx}
% \end{figure}


\begin{figure}
  % https://q.uiver.app/?q=WzAsNixbMywwLCIxIl0sWzAsMiwiWF8wIl0sWzIsMiwiWF8xIl0sWzQsMiwiWF8yIl0sWzYsMiwiWF8zIl0sWzgsMiwiXFxkb3RzIl0sWzIsMSwicl5YXzAiXSxbNCwzLCJyXlhfMiJdLFswLDEsIlxcZml4KGYpXzAiLDFdLFswLDIsIlxcZml4KGYpXzEiLDFdLFswLDMsIlxcZml4KGYpXzIiLDFdLFswLDQsIlxcZml4KGYpXzMiLDFdLFswLDUsIlxcY2RvdHMiLDMseyJzdHlsZSI6eyJib2R5Ijp7Im5hbWUiOiJub25lIn0sImhlYWQiOnsibmFtZSI6Im5vbmUifX19XSxbMywyLCJyXlhfMSJdLFs1LDQsInJeWF8zIl1d
  \[\begin{tikzcd}[ampersand replacement=\&,column sep=2.25em]
    \&\&\& 1 \\
    \\
    {X_0} \&\& {X_1} \&\& {X_2} \&\& {X_3} \&\& \dots
    \arrow["{r^X_0}", from=3-3, to=3-1]
    \arrow["{r^X_2}", from=3-7, to=3-5]
    \arrow["{\fix(f)_0}"{description}, from=1-4, to=3-1]
    \arrow["{\fix(f)_1}"{description}, from=1-4, to=3-3]
    \arrow["{\fix(f)_2}"{description}, from=1-4, to=3-5]
    \arrow["{\fix(f)_3}"{description}, from=1-4, to=3-7]
    \arrow["\cdots"{marking}, draw=none, from=1-4, to=3-9]
    \arrow["{r^X_1}", from=3-5, to=3-3]
    \arrow["{r^X_3}", from=3-9, to=3-7]
  \end{tikzcd}\]
  \caption{The guarded fixpoint of $f$.}
  \label{fig:topos-of-trees-fix}
\end{figure}

% TODO global elements?
\section{Towards a Full Relational Model}\label{sec:towards-relational-model}

% In the previous section, we gave a denotational semantics to the terms of the
% gradually-typed lambda calculus using the tools of SGDT. We defined a big-step
% semantics for closed terms of type $\nat$, as functions from $\mathbb{N} \to
% ((\mathbb{N} + 1) + 1)$.

Now that we have seen how to model the terms without regards to
graduality, we next turn to how to enhance this model to provide a
compositional semantics that additionally satisfies graduality. By a
compositional semantics, we mean that we want to provide a
compositional interpretation of \emph{type} and \emph{term precision}
such that we can extract the graduality relation from our model of
term precision.

\subsection{A first attempt: Modeling Term Precision with Posets}

In prior work, New and Licata model the term precision relation by
equipping $\omega$-CPOs with a second ordering $\ltdyn$ they call the
``error ordering'' such that a term precision relation $M \ltdyn M' :
r(A)$ is modeled by $\sem{M} \ltdyn \sem{M'}$ in the semantics. Then
the error ordering on the free error domain $\li A$ is based on the
graduality property: essentially either $\sem{M}$ errors or both terms
diverge or both terms terminate with values related by the ordering
relation on $A$.
%
The obvious first try at a guarded semantics that models graduality
would be to similarly enhance our value types and computation types to
additionally carry a poset structure. For this value types can simply
be posets. For computation types, we want our error domains to be
posets as well. To consider what additional properties we want the
error domains to have, let's consider the particular case of the free
error domain $\li A$.

The ordering on $\li A$ should model the graduality property, but how
to interpret this is not so straightforward. The graduality property
as stated mentions divergence, but we only work indirectly with
diverging computations in the form of the abstract steps/think
maps. The closest direct encoding of the graduality property is an
ordering we call the \emph{step-insensitive error ordering}, defined
by guarded recursion in
Figure~\ref{fig:step-insensitive-error-ordering}.
%
Firstly, to model graduality, $\mho$ is the least element. Next, two
returned values are related just when they are in the ordering $\leq$
on $A$.
%
Two thinking elements are related just when they are related later.
%
The interesting cases are then those where one side is thinking and
the other has completed to either an error or a value.
%
Since the graduality property is \emph{extensional}, i.e., oblivious
to the number of steps taken, it is sensible to say that if one side
has terminated and the other side is thinking, that we must require
the thinking side to eventually terminate with a related behavior,
which is the content of the final $4$ cases of the definition.

\begin{figure*}
    \begin{align*}
        &\mho \semltbad l \text{ iff } \top \\
        %
        &\eta\, x \semltbad \eta\, y \text{ if } 
            x \semlt y \\
        %		
        &\theta\, \tilde{l} \semltbad \theta\, \tilde{l'} \text{ iff } 
            \later_t (\tilde{l}_t \semltbad \tilde{l'}_t) \\
        %	
        &\theta\, \tilde{l} \semltbad \mho \text{ iff } \exists n. \theta\, \tilde{l} = \delta^n(\mho) \\
        %	
        &\theta\, \tilde{l} \semltbad \eta\, y \text{ iff } \exists n. \exists x \leq y.
            (\theta\, \tilde{l} = \delta^n(\eta\, x)) \\
        %
        &\theta\, \tilde{l} \semltbad \mho \text { iff } 
            \exists n. \theta\, \tilde{l} = \delta^n(\mho) \\
        %	
        &\eta\, x \semltbad \theta\, \tilde{l'} \text { iff }
            \exists n. \exists y \geq x. (\theta\, \tilde{l'} = \delta^n (\eta\, y))
    \end{align*}
    \caption{The step-insensitive error ordering.}
    \label{fig:step-insensitive-error-ordering}
\end{figure*}

This definition of $\semltbad$ is an entirely reasonable attempt to
model graduality: for instance it is true that if $l \semltbad l'$ for
$\li \mathbb{N}$ where $\mathbb{N}$ is the flat poset whose ordering
is equality, then the bigstep semantics we derive for $l, l'$ do in
fact satisfy the intended graduality relation.
%
However, there is a major issue with this definition: it's not a poset
at all: it is not anti-symmetric, but more importantly it is
\emph{not} transitive!
%
To see why, we observe the following undesirable property of any
relation that is transitive and ``step-insensitive'' on one side:
\begin{theorem}\label{thm:no-go}
  Let $R$ be a binary relation on the free error domain $U(\li A)$. If
  $R$ satisfies the following properties
  \begin{enumerate}
  \item Transitivity
  \item $\theta$-congruence: If $\later_t (\tilde{x}_t \binrel{R} \tilde{y}_t)$, then $\theta(\tilde{x}) \binrel{R} \theta(\tilde{y})$.
  \item Right step-insensitivity: If $x \binrel{R} y$ then $x \binrel{R} \delta y$.
  \end{enumerate}
  Then for any $l : U(\li A)$, we have $l R \Omega$. If $R$ is left
  step-insensitive instead then $\Omega \binrel{R} x$.
\end{theorem}
\begin{proof}
  By L\"ob induction, in the appendix.
\end{proof}
\begin{corollary}
  Let $R$ be a binary relation on $U(\li A)$. If $R$ satisfies
  transitivity, $\theta$-congruence and left and right
  step-insensitivity, then $R$ is the total relation: $\forall x, y. x
  \binrel{R} y$.
\end{corollary}
\begin{proof}
  $x \binrel{R} \Omega$ and $\Omega \binrel{R} y$ by the previous
  theorem. Then by transitivity $x \binrel R y$.
\end{proof}

This in turn implies $\semltbad$ is not transitive, since it is a
$\theta$-congruence and step-insensitive on both sides, but not
generally trivial: e.g., for the flat poset on $\mathbb N$, $\eta 0 \semltbad
\eta 1$ is false.
%
This shows definitively that we cannot provide a compositional
semantic graduality proof based on types as posets if we use this
step-insensitive ordering on $\li A$.

\subsection{Why transitivity is essential}

At this point we might try to weaken from our initial attempt at
providing a poset semantics to a semantics where types are equipped
with merely a reflexive relation, and continue using the
step-insensitive ordering. However some level of transitivity is
absolutely essential to providing compositional reasoning, and is used
pervasively in the New-Licata development.
%
The reason that it is essential has to do with the details of the
graduality proof, so first we expand on how we prove graduality
compositionally. First of all, we need to extend our compositional
semantics of types and terms to also a compositional semantics of
\emph{type precision derivations} and \emph{term precision}.
%
Sticking to a poset-based semantics, if our value types are posets, we then want
our precision derivations $c : A \ltdyn A'$ to be \emph{poset relations} on $A,
A'$, that is relations between the underlying sets of $A, A'$ that are
\emph{downward-closed} in $A$ and \emph{upward-closed} in $A'$.  We denote such
a relation between $A$ and $A'$ by $c : A \rel A'$. The obvious example of a
poset relation is the poset ordering itself, $\leq_A : A \rel A$.

% \max{TODO: Eric pick
%   up from here. Go into the squares for casts and how representability
%   gives us a compositional approach to proving them}

\eric{Picked up here; Max, please read this.}

To model term precision, we use the notion of \emph{square}, which specifies a
relation between two monotone functions. This has been used to model term
precision in previous work on double-categorical models for gradual typing
\cite{new-licata18}. The definition of square is as follows. Let $A_i, A_o,
A_i', A_o'$ be partially-ordered sets. Let $c_i : A_i \rel A_i'$ and $c_o : A_o
\rel A_o'$ be poset relations, and let $f : A_i \to A_o$ and $g : A_i' \to A_o'$
be monotone functions. We say that $f \ltsq{c_i}{c_o} g$ if for all $x : A_i$
and $y : A_i'$ with $(x, y) \in c_i$, we have $(f(x), g(y)) \in c_o$. We
visualize this situation as follows:
%
% https://q.uiver.app/#q=WzAsNCxbMCwwLCJBX2kiXSxbMCwxLCJBX28iXSxbMSwwLCJBX2knIl0sWzEsMSwiQV9vJyJdLFsyLDMsImciXSxbMCwxLCJmIiwyXSxbMCwyLCJjX2kiLDAseyJzdHlsZSI6eyJib2R5Ijp7Im5hbWUiOiJiYXJyZWQifSwiaGVhZCI6eyJuYW1lIjoibm9uZSJ9fX1dLFsxLDMsImNfbyIsMix7InN0eWxlIjp7ImJvZHkiOnsibmFtZSI6ImJhcnJlZCJ9LCJoZWFkIjp7Im5hbWUiOiJub25lIn19fV1d
\[\begin{tikzcd}[ampersand replacement=\&]
  {A_i} \& {A_i'} \\
  {A_o} \& {A_o'}
  \arrow["{c_i}", "\shortmid"{marking}, no head, from=1-1, to=1-2]
  \arrow["f"', from=1-1, to=2-1]
  \arrow["g", from=1-2, to=2-2]
  \arrow["{c_o}"', "\shortmid"{marking}, no head, from=2-1, to=2-2]
\end{tikzcd}\]
%

\eric{Not sure if this belongs here...}
Squares are used to model the term precision ordering $\Gamma \ltdyn \Gamma'
\vdash M \ltdyn N : A \ltdyn A'$ as an ordering relation between the monotone
functions denoted by $M$ and $N$:

% https://q.uiver.app/#q=WzAsNCxbMCwwLCJcXEdhbW1hIl0sWzAsMSwiVVxcbGkgQSJdLFsxLDAsIlxcR2FtbWEnIl0sWzEsMSwiVVxcbGkgQSciXSxbMiwzLCJcXHNlbXtOfSJdLFswLDEsIlxcc2Vte019IiwyXSxbMCwyLCJcXEdhbW1hXntcXGx0ZHlufSIsMCx7InN0eWxlIjp7ImJvZHkiOnsibmFtZSI6ImJhcnJlZCJ9LCJoZWFkIjp7Im5hbWUiOiJub25lIn19fV0sWzEsMywiVVxcbGkgYyIsMix7InN0eWxlIjp7ImJvZHkiOnsibmFtZSI6ImJhcnJlZCJ9LCJoZWFkIjp7Im5hbWUiOiJub25lIn19fV1d
\[\begin{tikzcd}[ampersand replacement=\&]
	\Gamma \& {\Gamma'} \\
	{U\li A} \& {U\li A'}
	\arrow["{\Gamma^{\ltdyn}}", "\shortmid"{marking}, no head, from=1-1, to=1-2]
	\arrow["{\sem{M}}"', from=1-1, to=2-1]
	\arrow["{\sem{N}}", from=1-2, to=2-2]
	\arrow["{U\li c}"', "\shortmid"{marking}, no head, from=2-1, to=2-2]
\end{tikzcd}\]

where $\Gamma = \{A_1, A_2, \dots\}$ and $\Gamma' = \{A_1' , A_2', \dots\}$ and for
each $j$, $c_j : A_j \ltdyn A_j'$, and the relation $\Gamma^{\ltdyn}$ is simply
the product of the $c_i$.

We note that for any $c$, there is a ``vertical identity'' square $\id
\ltsq{c}{c} \id$, as can be seen by unfolding the definition. Similarly, for any
montone function $f : A_i \to A_o$, there is a ``horizontal identity'' square $f
\ltsq{r(A_i)}{r(A_o)}$ by the definition of monotone function.
We likewise make the analogous definition of square for error domains.

Having defined squares, we now discuss the interpretation of the cast rules UpL
and UpR \footnote{The cast rules DnL and DnR involve error domain relations and
squares, which for the sake of this discussion are not relevant.}. These rules
specify a relationship between the semantics of type precision and the
corresponding casts (see Section \ref{sec:GTLC}). For example, consider the UpL
rule. It states that if $c : A_1 \ltdyn A_2$ and $c' : A_2 \ltdyn A_3$ and $x :
A_1$ is related to $y : A_3$ via the composition $c \comp c'$, then the upcast
$\upc{c} x$ is related to $y$ via $c'$. This is equivalent to the existence of
the following square:
%
% https://q.uiver.app/#q=WzAsNSxbMCwwLCJBXzEiXSxbMSwwLCJBXzIiXSxbMiwwLCJBXzMiXSxbMCwxLCJBXzIiXSxbMiwxLCJBXzMiXSxbMyw0LCJjJyIsMix7InN0eWxlIjp7ImJvZHkiOnsibmFtZSI6ImJhcnJlZCJ9LCJoZWFkIjp7Im5hbWUiOiJub25lIn19fV0sWzAsMywiXFx1cGN7Y30iLDJdLFsyLDQsIlxcaWQiXSxbMCwxLCJjIiwwLHsic3R5bGUiOnsiYm9keSI6eyJuYW1lIjoiYmFycmVkIn0sImhlYWQiOnsibmFtZSI6Im5vbmUifX19XSxbMSwyLCJjJyIsMCx7InN0eWxlIjp7ImJvZHkiOnsibmFtZSI6ImJhcnJlZCJ9LCJoZWFkIjp7Im5hbWUiOiJub25lIn19fV1d
\[\begin{tikzcd}[ampersand replacement=\&]
  {A_1} \& {A_2} \& {A_3} \\
  {A_2} \&\& {A_3}
  \arrow["c", "\shortmid"{marking}, no head, from=1-1, to=1-2]
  \arrow["{\upc{c}}"', from=1-1, to=2-1]
  \arrow["{c'}", "\shortmid"{marking}, no head, from=1-2, to=1-3]
  \arrow["\id", from=1-3, to=2-3]
  \arrow["{c'}"', "\shortmid"{marking}, no head, from=2-1, to=2-3]
\end{tikzcd}\]
%
Observe that although this is really a fact about the relationship between $c$
and $\upc{c}$, we must universally quantify over all relations $c'$. This means
we cannot require the existence of this square as part of the definition of a
monotone relation, as it is self-referential. This would seem to imply that we
cannot give a syntax-independent semantic model for gradual typing. However, the
key observation is that through compositionality, we can \emph{derive} the above
squares from simpler ones that do not involve composition of relations. These
simpler squares have been used in prior work for giving a compositional
semantics to gradual typing \cite{new-licata18}. Below is the square
corresponding to UpL:
%
% https://q.uiver.app/#q=WzAsNCxbMCwwLCJBXzEiXSxbMCwxLCJBXzIiXSxbMSwwLCJBXzIiXSxbMSwxLCJBXzIiXSxbMCwyLCJjIiwwLHsic3R5bGUiOnsiYm9keSI6eyJuYW1lIjoiYmFycmVkIn0sImhlYWQiOnsibmFtZSI6Im5vbmUifX19XSxbMSwzLCJyKEFfMikiLDIseyJzdHlsZSI6eyJib2R5Ijp7Im5hbWUiOiJiYXJyZWQifSwiaGVhZCI6eyJuYW1lIjoibm9uZSJ9fX1dLFswLDEsIlxcdXBje2N9IiwyXSxbMiwzLCJcXGlkIl1d
\[\begin{tikzcd}[ampersand replacement=\&]
  {A_1} \& {A_2} \\
  {A_2} \& {A_2}
  \arrow["c", "\shortmid"{marking}, no head, from=1-1, to=1-2]
  \arrow["{\upc{c}}"', from=1-1, to=2-1]
  \arrow["\id", from=1-2, to=2-2]
  \arrow["{r(A_2)}"', "\shortmid"{marking}, no head, from=2-1, to=2-2]
\end{tikzcd}\]
%
This square, and the corresponding one related to UpR, together express the
\emph{representability} of $c$ by the upcast morphism $\upc{c}$, which
intuitively says that $c$ is determined by the behavior of $\upc{c}$. We can use
this simpler square to derive the original UpL square by horizontally composing
with the identity square for $c'$:
%
% https://q.uiver.app/#q=WzAsNixbMCwwLCJBXzEiXSxbMSwwLCJBXzIiXSxbMiwwLCJBXzMiXSxbMCwxLCJBXzIiXSxbMiwxLCJBXzMiXSxbMSwxLCJBXzIiXSxbMCwxLCJjIiwwLHsic3R5bGUiOnsiYm9keSI6eyJuYW1lIjoiYmFycmVkIn0sImhlYWQiOnsibmFtZSI6Im5vbmUifX19XSxbMSwyLCJjJyIsMCx7InN0eWxlIjp7ImJvZHkiOnsibmFtZSI6ImJhcnJlZCJ9LCJoZWFkIjp7Im5hbWUiOiJub25lIn19fV0sWzMsNSwicihBXzIpIiwyLHsic3R5bGUiOnsiYm9keSI6eyJuYW1lIjoiYmFycmVkIn0sImhlYWQiOnsibmFtZSI6Im5vbmUifX19XSxbNSw0LCJjJyIsMix7InN0eWxlIjp7ImJvZHkiOnsibmFtZSI6ImJhcnJlZCJ9LCJoZWFkIjp7Im5hbWUiOiJub25lIn19fV0sWzAsMywiXFx1cGN7Y30iLDJdLFsyLDQsIlxcaWQiXSxbMSw1LCJcXGlkIiwyXV0=
\[\begin{tikzcd}[ampersand replacement=\&]
  {A_1} \& {A_2} \& {A_3} \\
  {A_2} \& {A_2} \& {A_3}
  \arrow["c", "\shortmid"{marking}, no head, from=1-1, to=1-2]
  \arrow["{\upc{c}}"', from=1-1, to=2-1]
  \arrow["{c'}", "\shortmid"{marking}, no head, from=1-2, to=1-3]
  \arrow["\id"', from=1-2, to=2-2]
  \arrow["\id", from=1-3, to=2-3]
  \arrow["{r(A_2)}"', "\shortmid"{marking}, no head, from=2-1, to=2-2]
  \arrow["{c'}"', "\shortmid"{marking}, no head, from=2-2, to=2-3]
\end{tikzcd}\]
%
The existence of these simpler squares for a relation can be stated without
reference to other relations, and hence we can require that these squares exist
in the definition of relation without running into the circularity issue we
faced with the original rules. This allows us to formulate a notion of model
that is independent of the particular syntax of any cast calculus.

We now see why compositionality of squares, and hence transitivity of the
relations on the objects that model types, is an essential property that we
cannot sacrifice.

\subsection{Resolution: Splitting Error Ordering and Bisimilarity}\label{sec:lock-step-and-weak-bisim}

Given that compositional reasoning is essential, we seek to formulate an
ordering relation on $\li A$ that is transitive. The theorem above
tells us, however, that any such relation must either not be a congruence with
respect to $\theta$, or must not be left- and right-step-insensitive.
\footnote{Technically, we only need to give up step insensitivity on \emph{one}
side, as the above theorem requires $R$ to be both left- and
right-step-insensitive. We could define two separate ``step-semi-sensitive''
relations on $\li A$, each closed under delays on the left and
right respectively. The denotation of term precision would then involve the
conjunction of those relations. We do not pursue this approach in this paper,
because (1) it is not clear if the step-semi-sensitive relations would be
transitive, and (2) we would need to reason about the two ordering relations
separately for every pair of terms.}
%
We cannot forego the property of being a $\theta$-congruence, as without this we
would not be able to prove properties of the relation using \lob-induction,
e.g., that the monadic extension of a monotone function $f$ is a \emph{monotone}
function with respect to the ordering.
%and moreover this would mean that $\theta$ would not be a monotone function.

Thus, we choose to sacrifice left and right step-insensitivity. That is, we will
define an ordering relation that requires terms to be in ``lock-step''. In order
for two computations to be related in this ordering, they must have the same
stepping behavior (or the left-hand side must be an error). To deal with terms
that take differing numbers of steps, we then define a separate relation called
\emph{weak bisimilarity} that relates terms that are extensionally equal and may
differ by a finite number of delays. Then, the semantics of the error ordering
for the guarded lift monad will involve a combination of these two relations:
a ``closure'' of the lock-step error ordering under weak bisimilarity on both
sides.
% Although the combined relation will not be transitive (for the same reason that
% $\semltbad$ is not transitive), this...
This decomposition has the advantage that we can recover some transitive
reasoning and push much of the reasoning about stepping to the margins of the
development.

More formally, we define the \emph{lock-step error-ordering}, with the idea
being that two computations $l$ and $l'$ are related if they are in lock-step
with regard to their intensional behavior, up to $l$ erroring. Figure
\ref{fig:lock-step-error-ordering} gives the definition.

\begin{figure}
    \fbox{$l_1 \ltls l_2$}
    \begin{align*}
        &\eta\, x \ltls \eta\, y \text{ if } 
            x \mathbin{\le_A} y \\
        %		
        &\mho \ltls l' \\
        %
        &\theta\, \tilde{l} \ltls \theta\, \tilde{l'} \text{ if } 
            \later_t (\tilde{l}_t \ltls \tilde{l'}_t)
    \end{align*}
    \caption{The lock-step error ordering.}
    \label{fig:lock-step-error-ordering}
\end{figure}

When both sides are $\eta$, then we check that the returned values are related
in $\le_A$. The error term $\mho$ is the least element. Lastly, if both sides step
(i.e., are a $\theta$) then we compare the resulting computations one time step
later.
%
It is straightforward to prove using \lob-induction that this relation is
reflexive, transitive and anti-symmetric given that the underyling relation $R$
has those properties. The lock-step ordering is therefore the partial ordering
we will associate with $\li A$.
%
We similarly define a heterogeneous version of this ordering between the lifts
of two different types $A$ and $A'$, parameterized by a poset relation $c : A \rel A'$.

The second relation we define on the guarded lift monad is \emph{weak bisimilarity}.
% For a type $X$, we define a relation on $\li X$, called ``weak bisimilarity",
% written $l \bisim l'$. 
Intuitively, we say $l \bisim l'$ if they are the same ``up to delays''.
%
Weak bisimilarity on $\li A$ is parameterized by an underlying relation $R$ on $A$
and is defined by guarded fixpoint as shown in Figure \ref{fig:weak-bisimilarity}.

\begin{figure}
    \fbox{$l_1 \bisim l_2$}
    \begin{align*}
        &\mho \bisim \mho \\
      %
        &\eta\, x \bisim \eta\, y \text{ if } 
          x \mathbin{R} y \\
      %		
        &\theta\, \tilde{x} \bisim \theta\, \tilde{y} \text{ if } 
          \later_t (\tilde{x}_t \bisim \tilde{y}_t) \\
      %	
        &\theta\, \tilde{x} \bisim \mho \text{ if } 
          \theta\, \tilde{x} = \delta^n(\mho) \text { for some $n$ } \\
      %	
        &\theta\, \tilde{x} \bisim \eta\, y \text{ if }
          (\theta\, \tilde{x} = \delta^n(\eta\, x))
        \text { for some $n$ and $x : X$ such that $x \mathbin{R} y$ } \\
      %
        &\mho \bisim \theta\, \tilde{y} \text { if } 
          \theta\, \tilde{y} = \delta^n(\mho) \text { for some $n$ } \\
      %	
        &\eta\, x \bisim \theta\, \tilde{y} \text { if }
          (\theta\, \tilde{y} = \delta^n (\eta\, y))
        \text { for some $n$ and $y : X$ such that $x \mathbin{R} y$ }
      \end{align*}
    \caption{The weak bisimilarity relation.}
    \label{fig:weak-bisimilarity}
\end{figure}

When both sides are $\eta$, then we ensure that the underlying values are
bisimilar in the underlying bisimilarity relation on $A$. When one side is a
$\theta$ and the other is $\eta x$ (i.e., one side steps), we stipulate that the
$\theta$-term runs to $\eta y$ where $x$ is related to $y$. When one side is
$\theta$ and the other $\mho$, we require that the $\theta$-term runs to $\mho$.
If both sides step, then we allow one time step to pass and compare the
resulting terms. This definition captures the intuition of two terms being
related up to delays.

It can be shown (by \lob-induction) that weak bisimilarity is reflexive and
symmetric. We can also show that this relation is \emph{not} transitive. The
argument is the same as that used to show that the step-insensitive error
ordering $\semltbad$ described above is not transitive. In particular, we
observe that weak bisimilarity is a congruence with respect to $\theta$, and we
can show by $\lob$-induction that it is closed under delay on both sides. Thus,
we may appeal to the no-go theorem as before.
% We appeal to the no-go theorem given above

Now we can define the denotation of term precision for closed terms of type
$\nat$ as follows:
%
\[ l_1 \leq^{\bisim} l_2 := 
  \Sigma_{l_1', l_2' \in \li \mathbb{N}}. l_1 \bisim l_1' \ltls l_2' \bisim l_2. \]

Importantly, we note that as with the original ordering $\semltbad$ defined at
the beginning of this section, the relation just defined is not transitive
(again as a result of the above no-go theorem). It may therefore seem that we
have not solved the original issue we faced. In a sense, this is true. The model
of gradual typing that we end up with will not support \emph{extensional}
compositional reasoning. However, by decomposing the denotation of term
precision in the above manner, we can employ \emph{intensional} compositional
reasoning by working with the lock-step error ordering. Thus, the tentative plan
going forward will be to carry out the proofs compositionally using the
lock-step ordering. We will then apply the closure under weak bisimilarity in
the denotation of term precision given above to handle the aspects involving
stepping.

% ...to carry out the proofs using the lock-step ordering and handle the stepping
% behavior separately via weak bisimilarity. In the end we combine everything
% together using the above denotation for term precision


\subsection{Adjusting the Cast Rules}\label{sec:adjusting-cast-rules}

\eric{We may want to introduce error domain relations and squares at this point,
since the squares for DnL and DnR involve lift.}

We have seen how we can recover transitive reasoning by working with the
lock-step error ordering. However, it turns out that we cannot completely
isolate the reasoning about steps from the reasoning about the ordering.
% Unfortunately, by defining the partial order on the guarded lift monad to be
% the lock-step error ordering, we introduce a new issue.
In particular, the cast rules required for proving graduality no longer hold if
interpreted using the lock-step ordering. The source of the problem is that the
downcast from the dynamic type to a function takes a step, i.e., involves a
$\theta$, and the lock-step ordering requires both sides to be in sync. More
specifically, consider the DnL rule for $\iarr \colon \dyntodyn \ltdyn\, \dyn$
shown below:
%
\begin{mathpar}
  \inferrule*{M \ltdyn N : \iarr}
             {\dnc{\iarr}{M} \ltdyn N : c}
\end{mathpar}
%
This rule corresponds to the following square:

% https://q.uiver.app/#q=WzAsNCxbMCwwLCJcXGxpIEQiXSxbMCwxLCJcXGxpIFUoRCBcXGFyciBcXGxpIEQpIl0sWzEsMCwiXFxsaSBEIl0sWzEsMSwiXFxsaSBEIl0sWzAsMiwiXFxsZXFfe1xcbGkgRH0iLDAseyJzdHlsZSI6eyJib2R5Ijp7Im5hbWUiOiJiYXJyZWQifSwiaGVhZCI6eyJuYW1lIjoibm9uZSJ9fX1dLFsxLDMsIlxcbGlcXGlhcnIiLDIseyJzdHlsZSI6eyJib2R5Ijp7Im5hbWUiOiJiYXJyZWQifSwiaGVhZCI6eyJuYW1lIjoibm9uZSJ9fX1dLFswLDEsIlxcZG5je1xcaWFycn0iLDJdLFsyLDMsIlxcaWQiXV0=
\[\begin{tikzcd}[ampersand replacement=\&]
	{\li D} \& {\li D} \\
	{\li U(D \arr \li D)} \& {\li D}
	\arrow["{\leq_{\li D}}", "\shortmid"{marking}, no head, from=1-1, to=1-2]
	\arrow["{\dnc{\iarr}}"', from=1-1, to=2-1]
	\arrow["\id", from=1-2, to=2-2]
	\arrow["\li\iarr"', "\shortmid"{marking}, no head, from=2-1, to=2-2]
\end{tikzcd}\]
%
where $\li \iarr$ is the heterogeneous version of the lock-step error ordering
discussed above applied to the relation $\iarr$. We note that the precise
definition of $\iarr$ as a poset relation is not important for this discussion.
%
We now recall the definition of the semantics of a downcast given in Section
\ref{sec:term-interpretation}. The downcast on the left will insert a $\theta$
in the case where the value of type $D$ is a later-function $\tilde{f}$.
%
Thus, in order for the downcast to be related to the RHS in the lock-step error
ordering on $\li (D \to \li D)$, the RHS must be of the form $\theta(\dots)$,
and moreover, after one time step, the argument of $\theta$ on the LHS must be
related to the argument of $\theta$ on the RHS. As it stands now, this need not
be the case, e.g., if the RHS is of the form $\eta(\dots)$. Thus, we conclude
that the DnL rule does not in general hold under the lock-step error ordering.
However, observe that we can remedy this particular situation by inserting a
``delay'' on the right, i.e., wrapping the RHS in $\delta^* = (\delta \circ
\eta)^\dagger$ where $\delta = \theta \circ \nxt$. Notice that this delay
function is weakly bisimilar to the identity function in that $\delta^*\, x
\bisim x$ for all $x$. \footnote{This follows from the fact that $\delta$ is
weakly bisimilar to the identity ($\delta\, x \bisim x$ for all $x$) and that
$-^\dagger$ preserves weak bisimilarity.} Thus, it will disappear when we apply
the bisimilarity-closure construction described above. That is, this delay makes
no difference in the extensional setting, but its presence is crucial in the
intensional setting.


\subsection{Extending the Model to Higher-Order Types}

% At this point, we could carry out the full construction of a concrete relational
% model with these additional features. 

The above techniques give a semantic interpretation of term precision for closed
terms of base type. We now consider how to extend these constructions to
potentially-open terms of \emph{all} types. In particular, in order to model
higher-order data types (i.e., functions) we need to equip \emph{every} semantic
object with not only a partial ordering relation but also a ``bisimilarity''
relation that is reflexive and symmetric. We must similarly equip every object
with a structured set of delays that can be inserted and manipulated to ensure
the appropriate cast rules hold at all higher-order types. For example, the
upcast for a derivation $c_i \ra c_o$ involves the downcast corresponding to
$c_i$ and the upcast corresponding to $c_o$. We must be able to insert delays in
a functorial manner to mimic the structure of the casts. In the next section, we
will make the necessary definitions and describe the relevant constructions.

% For the sake of reusability and modularity, rather than carry out this
% construction in the concrete setting developed here, we will instead return to
% the abstract setting and adjust our definition of model to account for these
% requirements. We will break the construction into smaller steps and isolate the
% pieces that require the techniques of SGDT from those that do not. Then with
% this framework at our disposal, we will return to the construction of a concrete
% model in Section \ref{sec:concrete-model}.
% , taking as a starting point the definitions introduced in the current section.

% Thus in the next section we define revised notions of a model of
% gradual typing based on the lessons learned in the present section. 







% The lack of transitivity presents a major barrier towards giving a compositional
% semantics to gradual typing: if our relations are not transitive, then we cannot
% compose squares horizontally. But we have argued above that horizontal
% composition is essential for modelling the axioms of gradual typing in a
% syntax-independent manner.

% In particular, to model the UpL/UpR/DnL/DnR rules
% for casts in the presence of transitivity, it is sufficient to establish the
% validity of simpler rules that do not ``build in'' composition. We can then use
% transitivity to derive the original versions of the rules. On the other hand,
% without transitivity we must instead validate the cast rules in the model ``from
% scratch''. In fact, we cannot even define the semantic interpretation of type
% precision in a syntax-independent manner. The issue is that in the definition of
% relation, the requirement that it be representable now involves quantifying over
% all other relations, which is circular.
 
% So, although this approach would suffice for proving graduality, it lacks the
% compositionality that we seek in a reusable framework for the semantics of
% gradual typing.

%Because it is convenient to make use of transitive reasoning in proving graduality, ...


% Doing so, we define a \emph{lock-step} error ordering, where roughly speaking,
% in order for computations to be related, they must have the same stepping
% behavior. We then formulate a separate relation, \emph{weak bisimilarity}, that
% relates computations that are extensionally equal and may differ only in their
% stepping behavior. % up to a finite number of \delta's
% Finally, the semantics of term precision will involve a combination of these two
% relations, a sort of closure of the lock-step ordering under weak bisimilarity
% on both sides. This decomposition has the advantage that we can recover some
% transitive reasoning and push the parts involving stepping to the margins of the
% development.



\begin{comment}
\subsection{Modeling Type and Term Precision}
\max{TODO: figure out where this stuff goes}
% We will model type precision $c : A \ltdyn A'$ as a relation between the sets
% $\sem{A}$ and $\sem{A'}$. 

To model term precision, we begin by equipping the denotation of every type with
an ordering relation. Since term precision is reflexive and transitive, and
since we identify terms that are equi-precise, we model value types as
partially-ordered sets and values $\Gamma \vdash V : A$ as \emph{monotone}
functions. Analogously, every error domain is now equipped with a partial
ordering for which the error element is the bottom element, and the map
$\theta_B : \laterhs B \to B$ is now required to be monotone. 
%
Morphisms of error domains are morphisms of the underlying partially-ordered
sets that preserve the error element and $\theta$ map, as was the case in the
previous section.

We model a type precision relation $c : A \ltdyn A'$ as a \emph{monotone
relation}, i.e., a relation $c$ that is upward-closed under the relation on $A'$
and downward-closed under the relation on $A$. We denote such a relation between
$A$ and $A'$ by $c : A \rel A'$. The relation on the poset $A$ is denoted
$r(A)$.
% This extends to products $c_1 \times c_2$ in the obvious way.
Composition of relations on predomains is the usual relational composition
(which is truncated to be propositional).



\subsubsection{Modelling Term Precision}\label{sec:modeling-term-precision}




It remains to define $Fc$, i.e., the action of $F$ on relations.
%
% However, lifting a relation between $A$ and $A'$ to a relation between $\li A$
% and $\li A'$ proves problematic if we allow computations that take different
% numbers of steps to be related. To illustrate the issue, let us define an
% ordering $\semltbad$ between $\li A$ and $\li A'$; we call this the
% \emph{step-insensitive error ordering}. 
%
We note that the graduality property and the axioms of the inequational theory
are independent of the intensional stepping behavior of terms, so our ultimate
notion that interprets term precision will need to be oblivious to stepping as
well.
%
To that end, let us define a \emph{step-insensitive error ordering} $\semltbad$
between $\li A$ and $\li A'$; The ordering is parameterized by an ordering
relation $\le$ between $A$ and $A'$. The definition is by guarded recursion and
is shown in Figure \ref{fig:step-insensitive-error-ordering}. Recall that
$\delta : \li A \to \li A$ is defined by $\delta = \theta_A \circ \nxt$.

Two computations that immediately return $(\eta)$ are related if the returned
values are related in the underlying ordering. The computation that errors
$(\mho)$ is the least term in the ordering. If both sides step (i.e., both sides
are $\theta$), then we allow one time step to pass and compare the resulting
terms (this is where use the relation defined ``later'' and is why we employ
guarded recursion to define the relation).
%
Lastly, if one side steps and the other immediately returns a value, then in
order for these terms to be related, the side that steps must terminate with a
value in some finite number of steps $n$, and that value must be related to the
value returned by the other side. Likewise, if the LHS steps and the RHS
immediately errors, then in order to be related, the LHS must eventually
terminate with error.

\end{comment}
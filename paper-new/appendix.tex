\section{Gradual Typing Syntax}\label{sec:appendix-gtlc-syntax}

\newcommand{\dynof}[1]{\textrm{dyn}(#1)}
\begin{theorem}
  For every $A$, there is a derivation $\dynof A : A \ltdyn D$
\end{theorem}
\begin{proof}
  By induction on $A$:
  \begin{enumerate}
  \item $A = \dyn$, then $r(\dyn) : \dyn \ltdyn \dyn$
  \item $A = \nat$, then $\inat : \nat \ltdyn \dyn$
  \item $A = A_1 \ra A_2$ then $(\dynof {A_1} \ra \dynof {A_2})\iarr : A_1 \ra A_2 \ltdyn \dyn$ 
  \item $A = A_1 \times A_2$ then $(\dynof {A_1} \times \dynof{A_2})\itimes : A_1 \times A_2 \ltdyn \dyn$ 
  \end{enumerate}
\end{proof}

\begin{theorem}
  \label{thm:thin}
  For any two derivations $c,c' : A \ltdyn A'$ of the same precision
  $c \equiv c'$
\end{theorem}
\begin{proof}
  \begin{enumerate}
  \item We show this by showing that derivations have a canonical
    form.

    The following presentation of precision derivations has unique derivations
    \begin{mathpar}
      \inferrule{}{\textrm{refl}(D) : D \ltdyn D}\and
      \inferrule{}{\textsf{Inj}_{\text{nat}} : \nat \ltdyn D}\and
      \inferrule{}{\textrm{refl}(\nat) : \nat \ltdyn \nat}\and
      \inferrule{c : A_i \ra A_o \ltdyn D\ra D}{c(\textsf{Inj}_{\text{arr}}) : A_i \ra A_o \ltdyn \nat}\and
      \inferrule{c : A_i \ltdyn A_i' \and d : A_o \ltdyn A_o'}{c \ra d : A_i \ra A_o \ltdyn A_i'\ra A_o'}
      %% \inferrule{A_1 \times A_2 \ltdyn D\times D}{A_1 \times A_2 \ltdyn \nat}
      %% \inferrule{A_1 \ltdyn A_1' \and A_2 \ltdyn A_2'}{A_1 \times A_2 \ltdyn A_1'\times A_2'}
    \end{mathpar}
    Since it satisfies reflexivity, cut-elimination and congruence, it
    is a model of the original theory. Since it is a sub-theory of the
    original theory, it is equivalent.
  \end{enumerate}
\end{proof}

As mentioned in Section~\ref{sec:GTLC}, the cast calculus
separates gradual type casts into downcasts and upcasts, whereas many
gradual calculi are built out of a single notion of cast
\[ \inferrule{\Gamma \vdash M : A}{\Gamma \vdash (M :: A') : A' }\]
The term precision rules for this style of calculus are as follows\cite{siek_et_al:LIPIcs:2015:5031}:
\begin{mathpar}
  \inferrule*[Right=CastRight]
  {\Delta \vdash M \ltdyn N_1 : c_1 \and
    c_{1} : A \ltdyn A_1\and
    c_2 : A \ltdyn A_2
  }
  {\Delta \vdash M \ltdyn (N :: A_2) : c_2}

  \inferrule*[Right=CastLeft]
  {\Delta \vdash M_1 \ltdyn N : c_1 \and
    c_1 : A_1 \ltdyn A\and
    c_2 : A_2 \ltdyn A
  }
  {\Delta \vdash (M_1 :: A_2) \ltdyn N : c_2}
\end{mathpar}
New-Ahmed and New-Licata showed that such if the cast $M_1 :: A_2$ for $M_1 : A_1$ is interpreted as
\[ \dnc {(\dynof {A_2})}{\upc {(\dynof{A_1})} M} \]
then the CastRight and CastLeft rules are derivable from the UpL/UpR/DnL/DnR rules
and the \emph{retraction} property of casts: that a downcast after an upcast
is equal to the identity.
%
In the intensional setting, this retraction property needs to be
weakened: a downcast after an upcast is merely \emph{weakly bisimilar}
to the identity.
%
But this weak bisimilarity is no longer strong enough to show the validity
of CastRight and CastLeft, due to the lack of transitivity of weak
bisimilarity.

Fortunately there is an alternative argument that makes CastRight and
CastLeft valid if we change slightly the interpretation of casts. To
show this alternative translation we need to first introduce the
\emph{least upper bound} of types:
\begin{lemma}
  The preordered set of types with type precision as ordering has all
  binary least upper bounds defined as follows:
  \begin{align*}
    (A_1 \times A_2) \sqcup (A_1' \times A_2') &= (A_1 \sqcup A_1') \times (A_2 \sqcup A_2')\\
    (A_1 \ra A_2) \sqcup (A_1' \ra A_2') &= (A_1 \ra A_1') \times (A_2 \ra A_2')\\
    \nat \sqcup \nat &= \nat\\
    A \sqcup A' &= \dyn \text{ otherwise }
  \end{align*}
\end{lemma}

Then for $M_1 : A_1$ we translate $M_1 :: A_2$ to
\[ \dnc {c^\sqcup_2}\upc{c^\sqcup_1} M \]
where $c^\sqcup_1 : A_1 \ltdyn A_1\sqcup A_2$ and $c^\sqcup_2 : A_2 \ltdyn
A_1\sqcup A_2$. In the presence of the strong retraction property, New
and Ahmed showed that this is equal to the prior interpretation. With
a weak retraction property, it is still weakly bisimilar to the prior
interpretation, so for the extensional properties we care about
ultimately this semantics is equivalent. However our compositional
arguments do care about this difference, and so we use instead this
translation that avoids the need for any retraction:
\begin{theorem}
  If $M_1 :: A_2$ is interpreted as $\dnc {c^\sqcup_2}\upc{c^\sqcup_1} M$,
  then CastRight and CastLeft are admissible.
\end{theorem}
\begin{proof}
  We show CastLeft, as this is the one that required retraction in
  prior work. CastRight follows similarly.

  Assume $M_1 \ltdyn N : c_1$, we seek to prove that
  \[ \dnc {c^\sqcup_2}\upc{c^\sqcup_1} M \ltdyn N : c_2 \]
  %
  Since $A \sqcup A_2$ is the
  least upper bound, we have there must be a derivation $c^{\sqcup} :
  A_1\sqcup A_2 \ltdyn A$.
  %
  Further by Theorem~\ref{thm:thin}, we have
  $c^\sqcup_2c^{\sqcup} \equiv c_2$, so it suffices to show
  \[M_1 \ltdyn \dnc {c^\sqcup_2}\upc{c^\sqcup} N : c^\sqcup_2c^{\sqcup} \]
  %
  Then applying DnL and UpL it is sufficient to show
  \[ M_1 \ltdyn N : c^\sqcup_1c^\sqcup \]
  But again by Theorem~\ref{thm:thin} we have $c^\sqcup_1c^\sqcup
  \equiv c_1$ so this follows by our assumption that $M_1 \ltdyn N
  :c_1$.
\end{proof}

\section{Kleisli Actions}\label{sec:kleisli-actions}

We now define the Kleisli actions of the type constructors. These are the following
\begin{enumerate}
\item $- \tok B : \op\PreDom_k \to \ErrDom_k$
\item $A \tok - : \ErrDom_k \to \ErrDom_k$
\item $- \timesk A_2 : \PreDom_k \to \PreDom_k$
\item $A_1 \timesk - : \PreDom_k \to \PreDom_k$
\end{enumerate}
While these can be defined in any CBPV model, for simplicity we just
give their definition explicitly for predomains and error domains:
\begin{definition}{~}
  Given $\phi : \li A_i \multimap \li A_o$, we define $\phi \tok B : U(A_o \to B) \to U(A_i \to B)$ as
  \[ (\phi \tok B)(f)(x) = f^\dagger(\phi(\eta(x))) \]

  This is functorial in that $\id \tok B = \id$ and $(\phi \circ
  \phi') \tok B = (\phi' \tok B) \circ (\phi \tok B)$.

  Further, this preserves squares in that if $\phi \ltsq{\li c_i}{\li c_o} \phi'$ then $\phi \tok B \ltsq{U(c_o \to d)}{U(c_i \to d)} \phi' \tok B'$.
\end{definition}
\begin{proof}
  {~}
  \begin{itemize}
  \item For identity
    \[ (\id \tok B)(f)(x) = f^\dagger(\eta(x)) = f(x) \]

  \item 
    For composition, expanding definitions it suffices to show
    \[ f^\dagger(\phi(\phi'(\eta(x)))) = ((\phi \tok B)(f))^\dagger(\phi'(\eta(x))) \]
    Which follows if we can show
    \[ ((\phi \tok B)(f))^\dagger = f^\dagger \circ \phi \]
    By the freeness of $\li -$ it suffices to show it for inputs of the form $\eta y$:
    \[ ((\phi \tok B)(f))^\dagger(\eta y) = ((\phi \tok B)(f))(y) = f^\dagger(\phi(\eta y))\]
  \item For squares, we assume $f \binrel{U(c_o\to d)} f'$ and $x
    \binrel {c_i} x'$ and we need to show $f^\dagger(\phi(x)) \binrel
    d (f')^\dagger(\phi'(x))$. This follows easily since $f^\dagger
    \ltsq{\li c_o}{d} f^{\prime\dagger}$.
  \end{itemize}
\end{proof}

\begin{definition}
  Given $f : UB \to UB'$, define $A \tok f : U(A \to B) \to U(A \to B')$ as
    \[ (A \tok f)(g)(x) = f(g(x)) \]

  This is functorial in that $A \tok \id = \id$ and $A \tok (f \circ
  f') = (A \tok f) \circ (A \tok f')$.

  Further this preserves squares in that if $f \ltsq{Ud_i}{Ud_o} f'$
  then $A \tok f \ltsq{U(c \to d_i)}{U(c \to d_o)} A' \tok f'$
\end{definition}
\begin{proof}
  \begin{itemize}
  \item For identity
    \[ (A \tok \id)(g)(x) = g(x) \]
  \item 
    for composition
    \[  (A \tok (f \circ f'))(g)(x) = f(f'(g(x))) = (A\tok f)((A \tok f')(g))(x) \]
  \item 
   for squares, given $g \binrel{U(c \to d_i)} g'$ and $x \binrel{c}
   x'$ we have $f(g(x)) \binrel{d_o} f'(g'(x'))$.
  \end{itemize}
\end{proof}

\begin{definition}[Kleisli Actions of $\times$]
  Given $\phi_1 : \li A_1 \multimap \li A_1'$ we define $\phi_1 \timesk A_2 : \li (A_1 \times A_2) \multimap (A_1' \times A_2)$ as the unique extension of the following to a homomorphism:
  \[ (\phi_1 \timesk A_2)(\eta (x_1,x_2)) = (\lambda x_1'. \eta(x_1',x_2))^\dagger(\phi_1(\eta(x_1))) \]
  and similarly define
  \[ (A_1 \timesk \phi_2)(\eta (x_1,x_2)) = (\lambda x_2'. \eta(x_1,x_2'))^\dagger(\phi_2(\eta(x_2)))\]
\end{definition}
\begin{proof}
  We show cases only for $-\timesk A_2$ as the other is symmetric.
  \begin{itemize}
  \item For identity it is sufficient to consider inputs of the form $\eta(x_1,x_2)$, then indeed
    \[ (\id \timesk A_2)(\eta(x_1,x_2)) = (\lambda x_1'. \eta(x_1',x_2))^\dagger(\eta(x_1)) = \eta(x_1,x_2) \]
  \item For composition, again for pure inputs this reduces to
    \[ (\lambda x_1'. \eta(x_1',x_2))^\dagger(\phi_1'(\phi_1(\eta(x_1))))
    = (\phi_1'\timesk A_2)((\lambda x_1'. \eta(x_1',x_2))^\dagger(\phi_1(\eta(x_1)))) \]
    So it suffices to show
    \[ (\lambda x_1'. \eta(x_1',x_2))^\dagger \circ \phi_1' = (\phi_1'\timesk A_2) \circ (\lambda x_1'. \eta(x_1',x_2))^\dagger \]
    Since both sides are homomorphisms out of the free error domain by freeness it suffices to show they are equal for inputs of the form $\eta(x_1')$:
    \[  (\lambda x_1'. \eta(x_1',x_2))^\dagger(\phi_1'(\eta(x_1')))
    = (\phi_1'\timesk A_2)(\eta(x_1',x_2))
    = (\phi_1'\timesk A_2)((\lambda x_1'. \eta(x_1',x_2))^\dagger(\eta(x_1')))
    \]

  \item Finally, for squares, we need to show if $\phi_l \ltsq{\li
    c_1}{\li c_1'} \phi_r$ then $\phi_l \timesk A_{2l} \ltsq{\li (c_1
    \times c_2)}{\li (c_1' \times c_2)} \phi_r \timesk A_{2r}$.

    By the universal property of $\li-$ on relations, it suffices to
    assume $x_{1l} \binrel{c_1} x_{1r}$ and $x_{2l} \binrel{c_2} x_{2r}$ and prove
    \[ (\phi_l \timesk A_{2l})(\eta(x_{1l},x_{2l})) \binrel{(\li (c_1' \times c_2))} (\phi_r \timesk A_{2r})(\eta(x_{1r},x_{2r})) \]
    expanding definitions this is
    \[ (\lambda x_{1l}'. \eta(x_{1l}',x_{2l}))^\dagger\phi_l(\eta(x_{1l}))
    \binrel{(\li (c_1' \times c_2))}
    (\lambda x_{1r}'. \eta(x_{1r}',x_{2r}))^\dagger\phi_r(\eta(x_{1r}))
    \]
    By our assumptions we have $\phi_l(\eta(x_{1l}) \binrel {\li c_1'} \phi_r(\eta(x_{1r})$ so since $-^\dagger$ preserves squares it suffices to show
    \[ (\lambda x_{1l}'. \eta(x_{1l}',x_{2l})) \ltsq{\li c_1}{\li (c_1' \times c_2)} (\lambda x_{1r}'. \eta(x_{1r}',x_{2r})) \]
    To show this assume $x_{1l}' \binrel{c_1'} x_{1r}'$. Then we have
    \[ \eta(x_{1l}',x_{2l}) \binrel{\li (c_1' \times c_2)} \eta(x_{1r'},x_{2l})  \]
    holds.
  \end{itemize}
\end{proof}

\section{Omitted Proofs Section \ref{sec:towards-relational-model}}

\begin{theorem}
  Let $R$ be a binary relation on the free error domain $U(\li A)$. If
  $R$ satisfies the following properties
  \begin{enumerate}
  \item Transitivity
  \item $\theta$-congruence: If $\later_t (\tilde{x}_t \binrel{R} \tilde{y}_t)$, then $\theta(\tilde{x}) \binrel{R} \theta(\tilde{y})$.
  \item Right step-insensitivity: If $x \binrel{R} y$ then $x \binrel{R} \delta y$.
  \end{enumerate}
  Then for any $l : U(\li A)$, we have $l \binrel{R} \Omega$. If $R$ is left
  step-insensitive instead then $\Omega \binrel{R} x$.
\end{theorem}
\begin{proof}
  By \lob-induction: we assume that $\laterhs (l \binrel{R} \Omega)$, and we
  show $(l \binrel{R} \Omega)$. We have $l \binrel{R} \delta l$ by right
  step-insensitivity.
  %
  By transitivity, it will therefore suffice to show that $\delta l \binrel{R}
  \Omega$.
  %
  Then since $\Omega = \delta \Omega$, it suffices to show $\delta l \binrel{R}
  \delta\Omega$.
  %
  By $\theta$-congruence, it suffices to show that
  $\later_t [(\nxt\, l)_t \binrel{R} (\nxt\, \Omega)_t]$, i.e., $\later (l \binrel{R} \Omega)$,
  which is our induction hypothesis.

  The proof when $R$ is left step-insensitive is symmetric.
\end{proof}

\section{Details of the Relational Model Construction}

In Section \ref{sec:concrete-relational-model} we described our model of gradual
typing. We omitted several technical proofs involving the well-definedness of
composition of relations and the actions of functors. We provide those proofs in
this section.

% \subsection{Free Composition of Error Domain Relations}
% \begin{definition}\label{def:free-comp-ed-rel}
%   Let $d : B_1 \rel B_2$ and $d' : B_2 \rel B_3$ be error domain relations. We
%   define the composition $dd'$ inductively by the following rules:
%   \begin{mathpar}
%       \inferrule*[right = Comp]
%       {b_1 \mathbin{d} b_2 \and b_2 \mathbin{d'} b_3}
%       {b_1 \mathbin{d \relcomp d'} b_3}

%       \inferrule*[right = DnClosed]
%       {b_1' \le_{B_1} b_1 \and b_1 \mathbin{d \relcomp d'} b_3}
%       {b_1' \mathbin{d \relcomp d'} b_3}

%       \inferrule*[right = UpClosed]
%       {b_1 \mathbin{d \relcomp d'} b_3 \and b_3 \le_{B_3} b_3'}
%       {b_1 \mathbin{d \relcomp d'} b_3'}

%       \inferrule*[right = PresErr]
%       { }
%       {\mho_{B_1} \mathbin{d \relcomp d'} b_3}

%       \inferrule*[right = PresTheta]
%       {\later_t( \tilde{b_1} \mathbin{d \relcomp d'} \tilde{b_3} ) }
%       {\theta_{B_1}(\tilde{b_1}) \mathbin{d \relcomp d'} \theta_{B_3}(\tilde{b_3}) }
%   \end{mathpar}

%   This has the universal property that $\phi \ltsq{d \relcomp d'}{d''}
%   \phi'$ if and only if for any $b \binrel{d} b'$ and $b' \binrel{d'}
%   b''$ that $\phi(b) \binrel{d''} \phi'(b'')$.
% \end{definition}

\subsection{The Dynamic Type}
The ordering $\le_D$ on the predomain $D$ is defined by
\begin{align*}
  \inat(n) \le \inat(n') 
      &\iff n = n' \\
  \itimes (d_1, d_2) \le \itimes (d_1', d_2')
      &\iff d_1 \le_D d_2 \text{ and } d_1' \le_D d_2'\\
  \iarr(\tilde{f}) \le \iarr(\tilde{f'}) 
      &\iff \later_t(\tilde{f}_t \le \tilde{f'}_t),
\end{align*}

and the bisimilarity relation is analogous.


% \subsection{Functorial Actions on Syntactic Perturbations}

% We provide details on the definition of the homomorphisms specifying the
% interpretation of perturbations corresponding for the functors $\li$, $U$,
% $\times$, and $\arr$.

% \begin{itemize}
%   \item To define the homomorphism $i_{\li A}$, by the universal property
%   of the coproduct of monoids it suffices to define two homomorphisms, one from
%   $\mathbb{N} \to \morbisimid{\li A}$ and one from $M_A \to \morbisimid{\li A}$.
%   The first homomorphism sends the generator $1 \in \mathbb{N}$ to the error
%   domain endomorpism $(\delta \circ \eta)^\dagger$, and the second sends $m_A \in
%   M_A$ to $\li(i_A(m_A))$ and we observe in both cases that the resulting
%   endomorphisms are bisimilar to the identity since the action of $\li$ on
%   morphisms preserves bisimilarity.

%   \item To define the homomorphism $i_{A_1 \times A_2}$, we
%   appeal to the universal property and give homomorphisms $M_{A_1} \to
%   \morbisimid{A_1 \times A_2}$ and $M_{A_2} \to \morbisimid{A_1 \times A_2}$. The
%   former applies $i_{A_1}$ in the first component and the identity in the second
%   component, while the latter does the reverse with $i_{A_2}$.

%   \item To define the homomorphism $i_{A \arr B}$ it suffices to define a
%   homomorphism $M_A^{op} \to \morbisimid{A \arr B}$ and $M_B \to \morbisimid{A
%   \arr B}$. The former takes an element $m_A$ and a predomain morphism $f : A \to
%   UB$ and returns the composition $f \circ i_A(m_A)$, while the latter is defined
%   similarly by post-composition with $i_B$.

% \end{itemize}

\subsection{Lemmas about Perturbations}

The goal of this section is to prove the following lemmas:

%%%%%%%%%%%%%%%
% Composition %
%%%%%%%%%%%%%%%
\begin{lemma}\label{lem:push-pull-comp}
  Let $(A_1, M_{A_1}, i_{A_1})$, $(A_2, M_{A_2}, i_{A_2})$, and $(A_3, M_{A_3},
  i_{A_3})$ be value objects, and let $c : A_1 \rel A_2$ and $c' : A_2 \rel A_3$
  be predomain relations.

  Given a push-pull structure $\Pi_c$ for $c$ and $\Pi_{c'}$ for $c'$, we can
  define a push-pull structure $\Pi_{c \comp c'}$ for $c \comp c'$.

  Likewise, we can define a push-pull structure for the composition of error
  domain relations.
\end{lemma}
\begin{proof}
  We define $\piv_{c \comp c'}$ as follows: We first define
  %
  $\push_{c \comp c'} = \push_{c'} \circ \push_{c}$ and 
  $\pull_{c \comp c'} = \pull_{c} \circ \pull_{c'}$.
  %
  We then observe that the required squares exist for both push and pull. In
  particular, for push we have that $i_{A_1}(m_{A_1}) \ltdyn_c^c
  i_{A_2}(\push_c(m_{A_1}))$ using the push property for $c$, and then using the
  push property for $c'$ we have $i_{A_2}(\push_c(m_{A_1})) \ltdyn_{c'}^{c'}
  i_{A_3}(\push_{c'}(\push_c(m_{A_1})))$. We can then compose these squares
  horizontally to obtain the desired square. The pull property follows
  similarly.

  The push-pull structure for the composition of computation relations is
  defined analogously.
\end{proof}


%%%%%%%%%%%
% U and F %
%%%%%%%%%%%
\begin{lemma}\label{lem:push-pull-U-F} 
  
  Let $A$ and $A'$ be value objects and let $c : A \rel A'$ be a relation on the
  underlying predomains. Given a push-pull structure $\Pi_c$ for $c$ we can
  define a push-pull structure $\Pi_{\li c}$ for $\li c$ with respect to $\li A$
  and $\li A'$.

  Likewise, let $B$ and $B'$ be computation objects and let $d : B \rel B'$ be
  an error domain relation. Given a push-pull structure for $d$ with respect to
  $B$ and $B'$, we can define a push-pull structure for $Ud$ with respect to
  $UB$ and $UB'$.
\end{lemma}
\begin{proof}
  
  We define $\push_{\li c} : \mathbb{N} \oplus M_{A} \to \mathbb{N} \oplus
  M_{A'}$ by the universal property of the coproduct of monoids. In particular,
  it suffices to define a homomorphism of monoids $\mathbb{N} \to \mathbb{N}
  \oplus M_{A'}$ and $M_{A} \to \mathbb{N} \oplus M_{A'}$.

  The former is simply $\inl$ and the latter is $\inr \circ \push_c$.
  
  Then to establish the push property, it suffices by the universal property of
  the coproduct, and the fact that $\mathbb{N}$ is the free monoid on one
  generator, to show that the following two squares exist:

% https://q.uiver.app/#q=WzAsNCxbMCwwLCJcXGxpIEEiXSxbMSwwLCJcXGxpIEEnIl0sWzAsMSwiXFxsaSBBIl0sWzEsMSwiXFxsaSBBJyJdLFswLDEsIlxcbGkgYyIsMCx7InN0eWxlIjp7ImJvZHkiOnsibmFtZSI6ImJhcnJlZCJ9LCJoZWFkIjp7Im5hbWUiOiJub25lIn19fV0sWzIsMywiXFxsaSBjIiwyLHsic3R5bGUiOnsiYm9keSI6eyJuYW1lIjoiYmFycmVkIn0sImhlYWQiOnsibmFtZSI6Im5vbmUifX19XSxbMCwyLCJcXGRlbHRhXipfQSIsMl0sWzEsMywiXFxkZWx0YV4qX3tBJ30iXV0=
\[\begin{tikzcd}[ampersand replacement=\&]
	{\li A} \& {\li A'} \\
	{\li A} \& {\li A'}
	\arrow["{\li c}", "\shortmid"{marking}, no head, from=1-1, to=1-2]
	\arrow["{\delta^*_A}"', from=1-1, to=2-1]
	\arrow["{\delta^*_{A'}}", from=1-2, to=2-2]
	\arrow["{\li c}"', "\shortmid"{marking}, no head, from=2-1, to=2-2]
\end{tikzcd}\]

% https://q.uiver.app/#q=WzAsNCxbMCwwLCJcXGxpIEEiXSxbMCwxLCJcXGxpIEEiXSxbMSwwLCJcXGxpIEEnIl0sWzEsMSwiXFxsaSBBJyJdLFswLDIsIlxcbGkgYyIsMCx7InN0eWxlIjp7ImJvZHkiOnsibmFtZSI6ImJhcnJlZCJ9LCJoZWFkIjp7Im5hbWUiOiJub25lIn19fV0sWzEsMywiXFxsaSBjIiwyLHsic3R5bGUiOnsiYm9keSI6eyJuYW1lIjoiYmFycmVkIn0sImhlYWQiOnsibmFtZSI6Im5vbmUifX19XSxbMCwxLCJcXGxpKGlfQShtX0EpKSIsMl0sWzIsMywiXFxsaSAoaV97QSd9KFxccHVzaF9jKG1fQSkpKSJdXQ==
\[\begin{tikzcd}[ampersand replacement=\&]
	{\li A} \& {\li A'} \\
	{\li A} \& {\li A'}
	\arrow["{\li c}", "\shortmid"{marking}, no head, from=1-1, to=1-2]
	\arrow["{\li(i_A(m_A))}"', from=1-1, to=2-1]
	\arrow["{\li (i_{A'}(\push_c(m_A)))}", from=1-2, to=2-2]
	\arrow["{\li c}"', "\shortmid"{marking}, no head, from=2-1, to=2-2]
\end{tikzcd}\]

where $\delta* = (\delta \circ \eta)^\dagger$

The first square exists by the fact that the monadic extension operation
$-^\dagger$ is monotone in its argument, and the fact that there is a
square $\delta_A \circ \eta_A \ltsq{}{} \delta_{A'} \circ \eta_{A'}$.

The second square exists by the push property for $c$ and the fact that $\li$
acts on squares.

The proof for the pull property is dual, and the construction of a push-pull
structure for $Ud$ is analogous.
\end{proof}


%%%%%%%%%%%%
% Products %
%%%%%%%%%%%%
\begin{lemma}\label{lem:push-pull-times} Let $A_1$, $A_1'$, $A_2$, and $A_2'$ be
  value objects. Let $c_1 : A_1 \rel A_1'$ and $c_2 : A_2 \rel A_2'$ be
  relations on the underlying predomains. Given push-pull structures $\Pi_{c_1}$
  for $c_1$ and $\Pi_{c_2}$ for $c_2$ we can define a push-pull structure
  $\Pi_{c_1 \times c_2}$ for $c_1 \times c_2$.
\end{lemma}
\begin{proof}
Recall that the monoid of syntactic perturbations for $A_1 \times A_2$ is
$M_{A_1} \oplus M_{A_2}$ and the monoid for $A_1' \times A_2'$ is $M_{A_1'}
\oplus M_{A_2'}$. Thus to define the push homomorphism for $c_1 \times c_2$, it
suffices by the universal property of the coproduct of monoids to define a
homomorphism $M_{A_1} \to M_{A_1'} \oplus M_{A_2'}$ and $M_{A_2} \to M_{A_1'}
\oplus M_{A_2'}$. The former is $\inl \circ \push_{c_1}$, and the latter is
$\inr \circ \push_{c_2}$. The pull homomorphism is defined similarly.

For the push property, we need to show that there is a square 
$i_{A_1 \times A_2}(m_1) \ltsq{}{} i_{A_1' \times A_2'}(\push_{c_1 \times c_2}(m_1))$.
%
By the universal property of the coproduct, it suffices to consider two cases: a
perturbation $m_1 \in M_{A_1}$ and a perturbation $m_2 \in M_{A_2}$. In the
first case, using the definition of the interpretation homomorphism for the
product, the square becomes $(i_{A_1}(m_1) \times \id_{A_2}) \ltsq{}{}
(i_{A_1'}(\push_{c_1}(m_1)) \times \id_{A_2'})$.
%
Then by the action of $\times$ on squares, it suffices to show that there are
squares $i_{A_1}(m_1) \ltsq{}{} i_{A_1'}(\push_{c_1}(m_1))$ and $\id_{A_2}
\ltsq{}{} \id_{A_2'}$. The former is the push property for $c_1$, and the latter
is an identity square.

The second case, i.e., $m_2 \in M_{A_2}$, is symmetric, and the proof of the
pull property is dual.
\end{proof}

%%%%%%%%%
% Arrow %
%%%%%%%%%
\begin{lemma}\label{lem:push-pull-arrow} Let $A$ and $A'$ be value objects and
  $B$ and $B'$ be computation objects. Let $c : A \rel A'$ be a relation on the
  underlying predomains and $d : B \rel B'$ a relation on the underlying error
  domains. Given push-pull structures $\Pi_c$ for $c$ and $\Pi_d$ for $d$, we
  can define a push-pull structure $\Pi_{c \arr d}$ for $c \arr d$.
\end{lemma}
\begin{proof}
  Similar to the product case above.
\end{proof}


We next define the actions of the Kleisli arrow and product functors on
syntactic perturbations.

% ``Kleisli arrow'' action on syntactic perturbations that takes a
% perturbation in $M_{UB}$ to a perturbation in $M_{U(A \to B)}$, as follows:

\begin{definition}\label{def:kleisli-arrow-perturbations} Let $A$ be a value
  object and $B$ be a computation object. We define a homomorphism of monoids
  $id \tok - \colon M_{UB} \to M_{U(A \arr B)}$ via the universal property of the
  coproduct of monoids by sending $\mathbb{N}$ to the first injection in $M_{U(A
  \arr B)} = \mathbb{N} \oplus M_A^{op} \oplus M_B$, and sending $M_B$ to the
  third injection.

  Likewise, we define $- \tok \id \colon M_{\li A}^{op} \to M_{U(A \arr B)}$ as
  follows. First note that $M_{\li A}^{op} = \mathbb{N} \oplus M_A^{op}$. Then
  we construct the homomorphism via the universal property, sending $\mathbb{N}$
  to the first injection $M_A^{op}$ to the second injection.
\end{definition}

We can similarly construct a Kleisli product operation on perturbations:

\begin{definition}\label{def:kleisli-product-perturbations}
  Let $A_1$ and $A_2$ be value objects. We define a homomorphism of monoids 
  $- \timesk \id \colon M_{\li A_1} \to M_{\li (A_1 \times A_2)}$ using the universal property,
  sending $\mathbb{N}$ to the first injection and $M_{A_1}$ to the second injection.

  Likewise, we define $\id \timesk - \colon M_{\li A_2} \to M_{\li (A_1 \times A_2)}$
  by sending $\mathbb{N}$ to the first injection and $M_{A_2}$ to the third injection.

  %  Given a perturbation $(n, a_1, a_2) \in $ we define $(n, a_1, a_2)
  % \timesk \id = (n, a_1, \id_{P_{A_2}})$, and likewise we define $\id \timesk (n,
  % a_1, a_2) = (n, \id_{P_{A_1}}, a_2)$.
\end{definition}

It is then easy to verify that the Kleisli arrow action on perturbation is well-defined 
in that for any $m \in M_{UB}$, we have
%
\[ \ptb_{U(A \arr B)}(\id \tok m) = \id \tok i_{UB}(m), \]
%
where the LHS is the Kleisli action on perturbation and the RHS is the Kleisli
action on morphisms. 
%
The proof uses the fact that the morphism $\delta$ commutes with all error
domain morphisms, which is a consequence of the definition of error domain
morphism.
%
The other verification involving the Kleisli arrow action is similar, as are the
two Kleisli product actions.

% %
% \[ (\delta_{U(A \arr B)}^*)^n \circ U(\id_{A} \arr \i_B(b)) = 
%     \id \tok (\delta_{UB}^*)^n \circ U(\ptb_B(b)) \]
% %
% This can be checked by unfolding the definition of $\tok$; the proof uses the
% fact that $\delta_{UB}^*$ commutes with computation morphisms.
% %

% Given a perturbation $q = (n, a) \in P_{FA}$ we define $(n, a) \tok \id = (n, a,
% \id_{P_B}) \in P_{U(A \to B)}$. We need to show that
% %
% \[ \ptb_{U(A \arr B)}((n, a) \tok \id) = \ptb_{FA}(n, a) \tok \id. \]
% %
% By similar reasoning to the above, this holds because $\delta_{FA}^*$ commutes
% with computation morphisms.



% We have $\ptb_{U(A \arr B)}(\id \tok (n, b)) = \ptb_{U(A \arr B)}(n, \id, b)$, 
% which is equal to $\delta_{U(A \arr B)}^n (\id \to \ptb_B(b))$


% %%%%%%%%%%%%%%%%%%%%%%
% % Model Construction %
% %%%%%%%%%%%%%%%%%%%%%%
% We now proceed with the construction of the model:

%     Define a step-2 model $\mathcal M'$ as follows:
%     \begin{itemize}
%       \item Value objects are pairs consisting of:
%       \begin{itemize}
%         \item A value object $A$ in $\vf$.
%         \item A monoid $P_A$ of perturbations
%         and a monoid homomorphism $\ptb_A : P_A \to \{ f \in \vf(A, A) \mid f \bisim \id_A \}$.
       
%       \end{itemize}  

%       \item Computation objects are pairs consisting of:
%       \begin{itemize}
%         \item A computation object $B$ in $\ef$.
%         \item A monoid $P_B$ of perturbations and a monoid homomorphism
%         $\ptb_B : P_B \to \{ \phi \in \ef(B, B) \mid \phi \bisim \id_B \}$.
%       \end{itemize}

%       \item Morphisms are given by morphisms of the underlying objects in $\vf$ and $\ef$, respectively.
%       %, i.e.,
%       % \[ \vf'((A, P_A, \ptb_A, P^K_A, \ptbk_A), (A', P_{A'}, \ptb_{A'}, P^K_{A'}, \ptbk_{A'})) = \vf(A, A') \]
%       %
%       % and likewise for computations.

%       \item Given objects $(A, P_A, \ptb_A)$ and $(A', P_{A'}, \ptb_{A'})$ a value relation
%       is a pair consisting of
%       \begin{itemize}
%         \item A value relation $c \in \vsq$.
%         \item A push-pull structure $\piv_c$ for $c$ with respect to $P_A$ and $P_{A'}$.
%         %\item A push-pull structure $\pie_{Fc}$ for $Fc$ with respect to $P^K_A$ and $P^K_{A'}$.
%       \end{itemize}

%       \item Similarly, a computation relation between $(B, P_B, \ptb_B)$ and $(B', P_{B'}, \ptb_{B'})$
%       consists of
%       \begin{itemize}
%         \item A computation relation $d \in \esq$.
%         \item A push-pull structure $\pie_d$ for $d$ with respect to $P_B$ and $P_{B'}$.
%         %\item A push-pull structure $\piv_{Ud}$ for $Ud$ with respect to $P^K_B$ and $P^K_{B'}$.
%       \end{itemize}
            
%       \item The squares are the same as the squares of the original model $\mathcal M$
%       %morphisms of $\vsq'$ and $\esq'$ are given by the morphisms of $\vsq$ and $\esq$.
      
%       \item We define composition of relations $(c, \piv_c)$ and $(c', \piv_{c'})$
%       as $(c \comp c', \piv_{c \comp c'})$ where $\piv_{c \comp c'}$ is as in Lemma
%       \ref{lem:push-pull-comp}, and likewise for computation relations.

%     \end{itemize}
      
%       % Functors \times, +, F, U, arrow
%       Now we define the actions of the functors:
%       \begin{itemize}

%       \item We define $\times$ on objects by
      
%       \[ (A_1, P_{A_1}, \ptb_{A_1}) \times (A_2, P_{A_2}, \ptb_{A_2}) =
%         (A_1 \times A_2, P_{A_1} \times P_{A_2}, \ptb_{A_1 \times A_2}) \]

%       where $\ptb_{A_1 \times A_2}(p_1, p_2) = \ptb_{A_1}(p_1) \times \ptb_{A_2}(p_2)$.

%       Using Lemma \ref{lem:push-pull-times}, we define $\times$ on relations by

%       \[ (c_1, \piv_{c_1}), (c_2 \piv_{c_2}) = 
%         (c_1 \times c_2, \piv_{c_1 \times c_2}). \]

%       \item We define $F$ on objects by 
      
%       \[ F(A, P_A, \ptb_A) = (FA, \mathbb{N} \times P_A, \ptb_{FA}), \]

%       where we define $\ptb_{FA}(n, a) = (\delta_{FA}^*)^n \circ F(\ptb_A(a))$.

%       Using Lemma \ref{lem:push-pull-U-F}, we define $F$ on relations by

%       \[ F(c, \piv_c) = (Fc, \pie_{Fc}). \]

%       \item We define $U$ on objects by
       
%       \[ U(B, P_B, \ptb_B) = (UB, \mathbb{N} \times P_B, \ptb_{UB}), \]

%       where we define $\ptb_{UB}(n, b) = (\delta_{UB}^*)^n \circ U(\ptb_B(b))$.

%       Using Lemma \ref{lem:push-pull-U-F}, we define $U$ on relations by

%       \[ U(d, \pie_d) = (Ud, \piv_{Ud}).  \]

%       We define $\arr$ on objects by

%       \[ (A, P_A, \ptb_A) \arr (B, P_B, \ptb_B) = 
%         (A \arr B, P_A^{op} \times P_B, \ptb_{A \arr B}), \]

%       where we define $\ptb_{A \arr B}(a, b) = \ptb_A(a) \arr \ptb_B(b)$.

%       Using Lemma \ref{lem:push-pull-arrow}, we define $\arr$ on relations by

%       \[ (c, \piv_c) \arr (d, \pie_d) =
%         (c \arr d, \pie_{c \arr d}). \]

%     \end{itemize}


    


%%%%%%%%%%%%%%%%%%%%%%%%%%%%%%%%%%%%%%%%%%%%%%%%%%%%%%%%%%%%%%%%%%%%%%%%%%%%%%

\subsection{Lemmas involving Quasi-Representable Relations}

In this section we prove lemmas about quasi-representability needed for showing
that our notions of value and computation relation are compositional, and for
defining the action of the functors $U$, $\li$, $\arr$, and $\times$ on value
and computation relations.

Recall the notion of quasi-equivalence of relations as defined in Definition
\ref{def:quasi-equivalent}. The following lemma will be useful in showing
that two relations are quasi-equivalent.

%%%%%%%%%%%%%%%%%%%%%%%%%%%%%%%%%%%%%%%%%%%%%%%%%%%%%%%%%%%%%%
% quasi-representable by same morphism --> quasi-equivalent
%%%%%%%%%%%%%%%%%%%%%%%%%%%%%%%%%%%%%%%%%%%%%%%%%%%%%%%%%%%%%%
\begin{lemma}\label{lem:left-rep-by-same-morphism} Let $A$ and $A'$ be value
  objects, and let $c, c' : A \rel A'$ be relations between the underlying
  predomains. If $c$ and $c'$ are both quasi-left-representable by the same
  predomain morphism $f : A \to A'$, then $c \qordeq c'$. If $c$ and $c'$ are
  both quasi-right-representable by the same predomain morphism $g : A' \to A$
  then $c \qordeq c'$.

  Dually, let $B$ and $B'$ be computation objects, and let $d, d' : B \rel B'$
  be relations between the underlying error domains. If $d$ and $d'$ are both
  quasi-right-representable by the same error domain morphism $\phi : B'
  \multimap B$, then $d \qordeq d'$. If $d$ and $d'$ are both
  quasi-left-representable by the same $\phi : B \multimap B'$
\end{lemma}
\begin{proof}
  We show the result for $c$ and $c'$ that are quasi-left-representable by the same $f$;
  the other proofs are analogous. 

  By $\upr$ for $c'$, there exists a perturbation $\delle_{c'}$ and a square
  $\upr_{c'} : \delle_{c'} \ltdyn_{c'}^{r(A)} f$.

  By $\upl$ for $c$, there exists a perturbation $\delre_c$ and a square 
  $\upl_c : f \ltdyn_{r(A')}^{c} \delre_c$.
  
  Composing these horizontally we get the following square:

  % https://q.uiver.app/#q=WzAsNixbMCwwLCJBIl0sWzEsMCwiQSJdLFsyLDAsIkEnIl0sWzAsMSwiQSJdLFsxLDEsIkEnIl0sWzIsMSwiQSciXSxbMCwxLCJyKEEpIiwwLHsic3R5bGUiOnsiYm9keSI6eyJuYW1lIjoiYmFycmVkIn0sImhlYWQiOnsibmFtZSI6Im5vbmUifX19XSxbMSwyLCJjIiwwLHsic3R5bGUiOnsiYm9keSI6eyJuYW1lIjoiYmFycmVkIn0sImhlYWQiOnsibmFtZSI6Im5vbmUifX19XSxbMyw0LCJjJyIsMix7InN0eWxlIjp7ImJvZHkiOnsibmFtZSI6ImJhcnJlZCJ9LCJoZWFkIjp7Im5hbWUiOiJub25lIn19fV0sWzQsNSwicihBJykiLDIseyJzdHlsZSI6eyJib2R5Ijp7Im5hbWUiOiJiYXJyZWQifSwiaGVhZCI6eyJuYW1lIjoibm9uZSJ9fX1dLFswLDMsImlfQShcXGRlbGxlX3tjJ30pIiwyXSxbMSw0LCJmIiwyXSxbMiw1LCJpX3tBJ30oXFxkZWxyZV9jKSJdXQ==
  \[\begin{tikzcd}[ampersand replacement=\&]
    A \& A \& {A'} \\
    A \& {A'} \& {A'}
    \arrow["{r(A)}", "\shortmid"{marking}, no head, from=1-1, to=1-2]
    \arrow["{i_A(\delle_{c'})}"', from=1-1, to=2-1]
    \arrow["c", "\shortmid"{marking}, no head, from=1-2, to=1-3]
    \arrow["f"', from=1-2, to=2-2]
    \arrow["{i_{A'}(\delre_c)}", from=1-3, to=2-3]
    \arrow["{c'}"', "\shortmid"{marking}, no head, from=2-1, to=2-2]
    \arrow["{r(A')}"', "\shortmid"{marking}, no head, from=2-2, to=2-3]
  \end{tikzcd}\]

  Then since $r(A) \comp c = c$ and $c' \comp r(A') = c'$, we obtain the desired square.
  %
  The other square (i.e., with $c'$ on top) is constructed in an analogous
  manner.

\end{proof}


Next we show that the functors $\li$ and $U$ preserve quasi-representability.

%%%%%%%%%%%%%
% U and F
%%%%%%%%%%%%%
\begin{lemma}\label{lem:representation-U-F}
  Let $A$ and $A'$ be value objects and let $c : A \rel A'$ be a predomain relation.
  If $c$ is quasi-left-representable, then $\li c$ is quasi-left-representable.
  Likewise, if $c$ is quasi-right-representable, then $\li c$ is quasi-right-representable.

  %Then we can define a left-representation structure $\rho^L_{UF(c)}$ for $UF(c)$.

  Similarly, let $B$ and $B'$ be computation objects and let $d : B \rel B'$ be an error domain relation.
  If $d$ is quasi-right-representable, then $Ud$ is quasi-right-representable.

  % Then we can define a right-representation structure $\rho^R_{FU(d)}$ for $FU(d)$.
\end{lemma}
\begin{proof}
  To show that $\li c$ is quasi-left-representable, we define
  \begin{itemize}
    \item $e_{\li c} = \li e_c$
    \item $\delre_{\li c} = \inr (\delre_c)$ 
    (recalling that the monoid $M_{\li A'} = \mathbb{N} \oplus M_{A'}$)
    \item $\delle_{\li c} = \inr (\delle_c)$
  \end{itemize}
  We obtain the two squares $\upl$ and $\upr$ using the definition of the
  interpretation of syntactic perturbations for $\li$ and the functorial action
  of $\li$ on the corresponding squares for $c$.

  The proof for $Ud$ is analogous.
\end{proof}



%%%%%%%%%%%%
% Identity %
%%%%%%%%%%%%

Next we show that the reflexive relation is quasi-representable.

\begin{lemma}\label{lem:reflexive-rel-quasi-rep}
  Let $A$ be a value object. Then $r(A)$ is quasi-left-representable and quasi-right-representable.
  Likewise, $r(B)$ is quasi-left- and quasi-right-representable for any computation object $B$.
\end{lemma}
\begin{proof}
  To show $r(A)$ is quasi-left-representable, we take the embedding to be the
  identity morphism and the perturbations $\delle = \delre = \id_{M_{A}}$, i.e.,
  the identtiy of the monoid. Then because $i_A$ is a homomorphism of monoids,
  it sends the identity element to the identity morphism, so the $\upl$ and
  $\upr$ squares are just identity squares.

  The same argument shows that $r(A)$ is quasi-right-representable and that
  $r(B)$ is quasi-left- and quasi-right-representable.
\end{proof}


%%%%%%%%%%%%%%%
% Composition %
%%%%%%%%%%%%%%%

We begin by showing that the composition of quasi-representable relations with
push-pull structures is quasi-representable.

\begin{lemma}\label{lem:representation-comp}
  Let $A$, $A'$, and $A''$ be value objects, and $B$, $B'$, and
  $B''$ be computation objects. Let $c : A \rel A'$ and $c' : A' \rel A''$ be
  predomain relations with push-pull structures, and let $d : B \rel B'$ and $d'
  : B' \rel B''$ be error domain relations with push-pull structures.
  \begin{enumerate}
    \item If $c$ and $c'$ are quasi-left- (resp. right)-representable, then $c
    \comp c'$ is quasi-left- (resp. right)-representable.

    \item If $d$ and $d'$ are quasi-left- (resp. right)-representable, then
    $d \comp d'$ is quasi-left (resp. right)-representable.
  \end{enumerate}
\end{lemma}
\begin{proof}
    % 1.
    \item To show $c \comp c'$ is quasi-left-representable we define
    \begin{itemize}
      \item $e_{c \comp c'} = e_{c'} \circ e_c$
      \item $\delre_{c \comp c'} = \delre_{c'} \cdot \push_{c'}(\delre_c)$ where
      $\cdot$ denotes multiplication in the monoid $M_A$
      \item $\delle_{c \comp c'} = \pull_c(\delle_{c'}) \cdot \delle_c$
      \item $\upl$ is the following square: % https://q.uiver.app/#q=WzAsMTAsWzAsMCwiQSJdLFswLDEsIkEnIl0sWzAsMiwiQSciXSxbMiwwLCJBJyciXSxbMiwxLCJBJyciXSxbMiwyLCJBJyciXSxbMCwzLCJBJyciXSxbMiwzLCJBJyciXSxbMSwxLCJBJyJdLFsxLDAsIkEnIl0sWzAsMSwiZV9jIiwyXSxbMSwyLCJcXGlkIiwyXSxbMiw2LCJlX3tjJ30iLDJdLFs0LDUsIlxcaWQiXSxbMSw4LCJyKEEnKSIsMCx7InN0eWxlIjp7ImJvZHkiOnsibmFtZSI6ImJhcnJlZCJ9LCJoZWFkIjp7Im5hbWUiOiJub25lIn19fV0sWzgsNCwiYyciLDAseyJzdHlsZSI6eyJib2R5Ijp7Im5hbWUiOiJiYXJyZWQifSwiaGVhZCI6eyJuYW1lIjoibm9uZSJ9fX1dLFszLDQsIlxccHVzaF97Yyd9KFxcZGVscmVfYykiXSxbMCw5LCJjIiwwLHsic3R5bGUiOnsiYm9keSI6eyJuYW1lIjoiYmFycmVkIn0sImhlYWQiOnsibmFtZSI6Im5vbmUifX19XSxbOSwzLCJjJyIsMCx7InN0eWxlIjp7ImJvZHkiOnsibmFtZSI6ImJhcnJlZCJ9LCJoZWFkIjp7Im5hbWUiOiJub25lIn19fV0sWzksOCwiXFxkZWxyZV9jIl0sWzUsNywiXFxkZWxyZV97Yyd9ICJdLFsyLDUsImMnIiwwLHsic3R5bGUiOnsiYm9keSI6eyJuYW1lIjoiYmFycmVkIn0sImhlYWQiOnsibmFtZSI6Im5vbmUifX19XSxbNiw3LCJyKEEnJykiLDIseyJzdHlsZSI6eyJib2R5Ijp7Im5hbWUiOiJiYXJyZWQifSwiaGVhZCI6eyJuYW1lIjoibm9uZSJ9fX1dLFsxMCwxOSwiXFx1cGxfYyIsMSx7InNob3J0ZW4iOnsic291cmNlIjoyMCwidGFyZ2V0IjoyMH0sInN0eWxlIjp7ImJvZHkiOnsibmFtZSI6Im5vbmUifSwiaGVhZCI6eyJuYW1lIjoibm9uZSJ9fX1dLFsxMSwxMywiKCoqKSIsMSx7InNob3J0ZW4iOnsic291cmNlIjoyMCwidGFyZ2V0IjoyMH0sInN0eWxlIjp7ImJvZHkiOnsibmFtZSI6Im5vbmUifSwiaGVhZCI6eyJuYW1lIjoibm9uZSJ9fX1dLFsxOSwxNiwiKCopIiwxLHsic2hvcnRlbiI6eyJzb3VyY2UiOjIwLCJ0YXJnZXQiOjIwfSwic3R5bGUiOnsiYm9keSI6eyJuYW1lIjoibm9uZSJ9LCJoZWFkIjp7Im5hbWUiOiJub25lIn19fV0sWzEyLDIwLCJcXHVwbF97Yyd9IiwxLHsic2hvcnRlbiI6eyJzb3VyY2UiOjIwLCJ0YXJnZXQiOjIwfSwic3R5bGUiOnsiYm9keSI6eyJuYW1lIjoibm9uZSJ9LCJoZWFkIjp7Im5hbWUiOiJub25lIn19fV1d
\[\begin{tikzcd}[ampersand replacement=\&,column sep=3.15em]
	A \& {A'} \& {A''} \\
	{A'} \& {A'} \& {A''} \\
	{A'} \&\& {A''} \\
	{A''} \&\& {A''}
	\arrow[""{name=0, anchor=center, inner sep=0}, "{e_c}"', from=1-1, to=2-1]
	\arrow[""{name=1, anchor=center, inner sep=0}, "\id"', from=2-1, to=3-1]
	\arrow[""{name=2, anchor=center, inner sep=0}, "{e_{c'}}"', from=3-1, to=4-1]
	\arrow[""{name=3, anchor=center, inner sep=0}, "\id", from=2-3, to=3-3]
	\arrow["{r(A')}", "\shortmid"{marking}, no head, from=2-1, to=2-2]
	\arrow["{c'}", "\shortmid"{marking}, no head, from=2-2, to=2-3]
	\arrow[""{name=4, anchor=center, inner sep=0}, "{\push_{c'}(\delre_c)}", from=1-3, to=2-3]
	\arrow["c", "\shortmid"{marking}, no head, from=1-1, to=1-2]
	\arrow["{c'}", "\shortmid"{marking}, no head, from=1-2, to=1-3]
	\arrow[""{name=5, anchor=center, inner sep=0}, "{\delre_c}", from=1-2, to=2-2]
	\arrow[""{name=6, anchor=center, inner sep=0}, "{\delre_{c'} }", from=3-3, to=4-3]
	\arrow["{c'}", "\shortmid"{marking}, no head, from=3-1, to=3-3]
	\arrow["{r(A'')}"', "\shortmid"{marking}, no head, from=4-1, to=4-3]
	\arrow["{\upl_c}"{description}, draw=none, from=0, to=5]
	\arrow["{(**)}"{description}, draw=none, from=1, to=3]
	\arrow["{(*)}"{description}, draw=none, from=5, to=4]
	\arrow["{\upl_{c'}}"{description}, draw=none, from=2, to=6]
\end{tikzcd}\]

      The square $(*)$ exists by the push-pull property for $c'$, and the square
      $(**)$ exists by the downward closure of $c'$.

      \item $\upr$ is the following square: % https://q.uiver.app/#q=WzAsMTAsWzAsMCwiQSJdLFswLDEsIkEiXSxbMCwyLCJBIl0sWzIsMCwiQSJdLFsyLDEsIkEnIl0sWzIsMiwiQSciXSxbMCwzLCJBIl0sWzIsMywiQScnIl0sWzEsMiwiQSciXSxbMSwzLCJBJyJdLFswLDEsIlxcZGVsbGVfYyIsMl0sWzEsMiwiXFxpZCIsMl0sWzIsNiwiXFxwdWxsX2MoXFxkZWxsZV97Yyd9KSIsMl0sWzQsNSwiXFxpZCJdLFszLDQsImVfYyJdLFs1LDcsImVfe2MnfSJdLFswLDMsInIoQSkiLDAseyJzdHlsZSI6eyJib2R5Ijp7Im5hbWUiOiJiYXJyZWQifSwiaGVhZCI6eyJuYW1lIjoibm9uZSJ9fX1dLFsxLDQsImMiLDAseyJzdHlsZSI6eyJib2R5Ijp7Im5hbWUiOiJiYXJyZWQifSwiaGVhZCI6eyJuYW1lIjoibm9uZSJ9fX1dLFsyLDgsImMiLDAseyJzdHlsZSI6eyJib2R5Ijp7Im5hbWUiOiJiYXJyZWQifSwiaGVhZCI6eyJuYW1lIjoibm9uZSJ9fX1dLFs4LDUsInIoQScpIiwwLHsic3R5bGUiOnsiYm9keSI6eyJuYW1lIjoiYmFycmVkIn0sImhlYWQiOnsibmFtZSI6Im5vbmUifX19XSxbNiw5LCJjIiwyLHsic3R5bGUiOnsiYm9keSI6eyJuYW1lIjoiYmFycmVkIn0sImhlYWQiOnsibmFtZSI6Im5vbmUifX19XSxbOSw3LCJjJyIsMix7InN0eWxlIjp7ImJvZHkiOnsibmFtZSI6ImJhcnJlZCJ9LCJoZWFkIjp7Im5hbWUiOiJub25lIn19fV0sWzgsOSwiXFxkZWxsZV97Yyd9IiwyXSxbMTEsMTMsIigqKSIsMSx7InNob3J0ZW4iOnsic291cmNlIjoyMCwidGFyZ2V0IjoyMH0sInN0eWxlIjp7ImJvZHkiOnsibmFtZSI6Im5vbmUifSwiaGVhZCI6eyJuYW1lIjoibm9uZSJ9fX1dLFsxMCwxNCwiXFx1cHJfYyIsMSx7InNob3J0ZW4iOnsic291cmNlIjoyMCwidGFyZ2V0IjoyMH0sInN0eWxlIjp7ImJvZHkiOnsibmFtZSI6Im5vbmUifSwiaGVhZCI6eyJuYW1lIjoibm9uZSJ9fX1dLFsyMiwxNSwiIiwxLHsic2hvcnRlbiI6eyJzb3VyY2UiOjIwLCJ0YXJnZXQiOjIwfSwic3R5bGUiOnsiYm9keSI6eyJuYW1lIjoibm9uZSJ9LCJoZWFkIjp7Im5hbWUiOiJub25lIn19fV0sWzEyLDIyLCIoKiopIiwxLHsic2hvcnRlbiI6eyJzb3VyY2UiOjIwLCJ0YXJnZXQiOjIwfSwic3R5bGUiOnsiYm9keSI6eyJuYW1lIjoibm9uZSJ9LCJoZWFkIjp7Im5hbWUiOiJub25lIn19fV0sWzIyLDE1LCJcXHVwcl97Yyd9IiwxLHsic2hvcnRlbiI6eyJzb3VyY2UiOjIwLCJ0YXJnZXQiOjIwfSwic3R5bGUiOnsiYm9keSI6eyJuYW1lIjoibm9uZSJ9LCJoZWFkIjp7Im5hbWUiOiJub25lIn19fV1d
\[\begin{tikzcd}[ampersand replacement=\&,column sep=3.15em]
	A \&\& A \\
	A \&\& {A'} \\
	A \& {A'} \& {A'} \\
	A \& {A'} \& {A''}
	\arrow[""{name=0, anchor=center, inner sep=0}, "{\delle_c}"', from=1-1, to=2-1]
	\arrow[""{name=1, anchor=center, inner sep=0}, "\id"', from=2-1, to=3-1]
	\arrow[""{name=2, anchor=center, inner sep=0}, "{\pull_c(\delle_{c'})}"', from=3-1, to=4-1]
	\arrow[""{name=3, anchor=center, inner sep=0}, "\id", from=2-3, to=3-3]
	\arrow[""{name=4, anchor=center, inner sep=0}, "{e_c}", from=1-3, to=2-3]
	\arrow[""{name=5, anchor=center, inner sep=0}, "{e_{c'}}", from=3-3, to=4-3]
	\arrow["{r(A)}", "\shortmid"{marking}, no head, from=1-1, to=1-3]
	\arrow["c", "\shortmid"{marking}, no head, from=2-1, to=2-3]
	\arrow["c", "\shortmid"{marking}, no head, from=3-1, to=3-2]
	\arrow["{r(A')}", "\shortmid"{marking}, no head, from=3-2, to=3-3]
	\arrow["c"', "\shortmid"{marking}, no head, from=4-1, to=4-2]
	\arrow["{c'}"', "\shortmid"{marking}, no head, from=4-2, to=4-3]
	\arrow[""{name=6, anchor=center, inner sep=0}, "{\delle_{c'}}"', from=3-2, to=4-2]
	\arrow["{(*)}"{description}, draw=none, from=1, to=3]
	\arrow["{\upr_c}"{description}, draw=none, from=0, to=4]
	\arrow[draw=none, from=6, to=5]
	\arrow["{(**)}"{description}, draw=none, from=2, to=6]
	\arrow["{\upr_{c'}}"{description}, draw=none, from=6, to=5]
\end{tikzcd}\]
    \end{itemize}

    To show $c \comp c'$ is quasi-right-representable we define
    \begin{itemize}
      \item $p_{c \comp c'} = p_c \circ p_{c'}$
      \item $\dellp_{c \comp c'} = \dellp_c \cdot \pull_c(\dellp_{c'})$
      \item $\delrp_{c \comp c'} = \push_{c'}(\delrp_c) \cdot \delrp_{c'}$
      \item $\dnr$ is the following square: % https://q.uiver.app/#q=WzAsMTAsWzAsMCwiQiJdLFsyLDAsIkInJyJdLFsyLDEsIkInIl0sWzIsMiwiQiciXSxbMiwzLCJCIl0sWzAsMSwiQiJdLFswLDIsIkIiXSxbMCwzLCJCIl0sWzEsMCwiQiciXSxbMSwxLCJCJyJdLFswLDgsImQiLDAseyJzdHlsZSI6eyJib2R5Ijp7Im5hbWUiOiJiYXJyZWQifSwiaGVhZCI6eyJuYW1lIjoibm9uZSJ9fX1dLFs4LDEsImQnIiwwLHsic3R5bGUiOnsiYm9keSI6eyJuYW1lIjoiYmFycmVkIn0sImhlYWQiOnsibmFtZSI6Im5vbmUifX19XSxbNiwzLCJkIiwwLHsic3R5bGUiOnsiYm9keSI6eyJuYW1lIjoiYmFycmVkIn0sImhlYWQiOnsibmFtZSI6Im5vbmUifX19XSxbNyw0LCJyKEIpIiwyLHsic3R5bGUiOnsiYm9keSI6eyJuYW1lIjoiYmFycmVkIn0sImhlYWQiOnsibmFtZSI6Im5vbmUifX19XSxbMSwyLCJwX3tkJ30iXSxbMiwzLCJcXGlkIl0sWzMsNCwicF9kIl0sWzAsNSwiXFxwdWxsX2QoXFxkZWxscF97ZCd9KSIsMl0sWzUsNiwiXFxpZCIsMl0sWzYsNywiXFxkZWxscF9kIiwyXSxbNSw5LCJkIiwwLHsic3R5bGUiOnsiYm9keSI6eyJuYW1lIjoiYmFycmVkIn0sImhlYWQiOnsibmFtZSI6Im5vbmUifX19XSxbOSwyLCJyKEInKSIsMCx7InN0eWxlIjp7ImJvZHkiOnsibmFtZSI6ImJhcnJlZCJ9LCJoZWFkIjp7Im5hbWUiOiJub25lIn19fV0sWzgsOSwiXFxkZWxscF97ZCd9IiwyXSxbMTksMTYsIlxcZG5yX2QiLDEseyJzaG9ydGVuIjp7InNvdXJjZSI6MjAsInRhcmdldCI6MjB9LCJzdHlsZSI6eyJib2R5Ijp7Im5hbWUiOiJub25lIn0sImhlYWQiOnsibmFtZSI6Im5vbmUifX19XSxbMTcsMjIsIigqKSIsMSx7InNob3J0ZW4iOnsic291cmNlIjoyMCwidGFyZ2V0IjoyMH0sInN0eWxlIjp7ImJvZHkiOnsibmFtZSI6Im5vbmUifSwiaGVhZCI6eyJuYW1lIjoibm9uZSJ9fX1dLFsyMiwxNCwiXFxkbnJfe2QnfSIsMSx7InNob3J0ZW4iOnsic291cmNlIjoyMCwidGFyZ2V0IjoyMH0sInN0eWxlIjp7ImJvZHkiOnsibmFtZSI6Im5vbmUifSwiaGVhZCI6eyJuYW1lIjoibm9uZSJ9fX1dXQ==
\[\begin{tikzcd}[ampersand replacement=\&,column sep=3.15em]
	B \& {B'} \& {B''} \\
	B \& {B'} \& {B'} \\
	B \&\& {B'} \\
	B \&\& B
	\arrow["d", "\shortmid"{marking}, no head, from=1-1, to=1-2]
	\arrow["{d'}", "\shortmid"{marking}, no head, from=1-2, to=1-3]
	\arrow["d", "\shortmid"{marking}, no head, from=3-1, to=3-3]
	\arrow["{r(B)}"', "\shortmid"{marking}, no head, from=4-1, to=4-3]
	\arrow[""{name=0, anchor=center, inner sep=0}, "{p_{d'}}", from=1-3, to=2-3]
	\arrow["\id", from=2-3, to=3-3]
	\arrow[""{name=1, anchor=center, inner sep=0}, "{p_d}", from=3-3, to=4-3]
	\arrow[""{name=2, anchor=center, inner sep=0}, "{\pull_d(\dellp_{d'})}"', from=1-1, to=2-1]
	\arrow["\id"', from=2-1, to=3-1]
	\arrow[""{name=3, anchor=center, inner sep=0}, "{\dellp_d}"', from=3-1, to=4-1]
	\arrow["d", "\shortmid"{marking}, no head, from=2-1, to=2-2]
	\arrow["{r(B')}", "\shortmid"{marking}, no head, from=2-2, to=2-3]
	\arrow[""{name=4, anchor=center, inner sep=0}, "{\dellp_{d'}}"', from=1-2, to=2-2]
	\arrow["{\dnr_d}"{description}, draw=none, from=3, to=1]
	\arrow["{(*)}"{description}, draw=none, from=2, to=4]
	\arrow["{\dnr_{d'}}"{description}, draw=none, from=4, to=0]
\end{tikzcd}\]
      
      Here the square $(*)$ exists by the push-pull property for $c$.
      
      \item $\dnl$ is the following square: % https://q.uiver.app/#q=WzAsMTAsWzAsMCwiQScnIl0sWzIsMCwiQScnIl0sWzAsMSwiQSciXSxbMCwyLCJBJyJdLFswLDMsIkEiXSxbMiwxLCJBJyciXSxbMiwyLCJBJyciXSxbMiwzLCJBJyciXSxbMSwzLCJBJyJdLFsxLDIsIkEnIl0sWzAsMiwicF97Yyd9IiwyXSxbMiwzLCJcXGlkIiwyXSxbMyw0LCJwX2MiLDJdLFsxLDUsImlfe0EnJ30oXFxkZWxycF97Yyd9KSJdLFs1LDYsIlxcaWQiXSxbNiw3LCJpX3tBJyd9KFxccHVzaF97Yyd9KFxcZGVscnBfYykpIl0sWzAsMSwicihBJycpIiwwLHsic3R5bGUiOnsiYm9keSI6eyJuYW1lIjoiYmFycmVkIn0sImhlYWQiOnsibmFtZSI6Im5vbmUifX19XSxbMiw1LCJjJyIsMCx7InN0eWxlIjp7ImJvZHkiOnsibmFtZSI6ImJhcnJlZCJ9LCJoZWFkIjp7Im5hbWUiOiJub25lIn19fV0sWzMsOSwicihBJykiLDAseyJzdHlsZSI6eyJib2R5Ijp7Im5hbWUiOiJiYXJyZWQifSwiaGVhZCI6eyJuYW1lIjoibm9uZSJ9fX1dLFs5LDYsImMnIiwwLHsic3R5bGUiOnsiYm9keSI6eyJuYW1lIjoiYmFycmVkIn0sImhlYWQiOnsibmFtZSI6Im5vbmUifX19XSxbNCw4LCJjIiwyLHsic3R5bGUiOnsiYm9keSI6eyJuYW1lIjoiYmFycmVkIn0sImhlYWQiOnsibmFtZSI6Im5vbmUifX19XSxbOCw3LCJjJyIsMix7InN0eWxlIjp7ImJvZHkiOnsibmFtZSI6ImJhcnJlZCJ9LCJoZWFkIjp7Im5hbWUiOiJub25lIn19fV0sWzksOCwiXFxkZWxycF9jIl0sWzEwLDEzLCJcXGRubF97Yyd9IiwxLHsic2hvcnRlbiI6eyJzb3VyY2UiOjIwLCJ0YXJnZXQiOjIwfSwic3R5bGUiOnsiYm9keSI6eyJuYW1lIjoibm9uZSJ9LCJoZWFkIjp7Im5hbWUiOiJub25lIn19fV0sWzIyLDE1LCIoKikiLDEseyJzaG9ydGVuIjp7InNvdXJjZSI6MjAsInRhcmdldCI6MjB9LCJzdHlsZSI6eyJib2R5Ijp7Im5hbWUiOiJub25lIn0sImhlYWQiOnsibmFtZSI6Im5vbmUifX19XSxbMTIsMjIsIlxcZG5sX2MiLDEseyJzaG9ydGVuIjp7InNvdXJjZSI6MjAsInRhcmdldCI6MjB9LCJzdHlsZSI6eyJib2R5Ijp7Im5hbWUiOiJub25lIn0sImhlYWQiOnsibmFtZSI6Im5vbmUifX19XV0=
\[\begin{tikzcd}[ampersand replacement=\&]
	{A''} \&\& {A''} \\
	{A'} \&\& {A''} \\
	{A'} \& {A'} \& {A''} \\
	A \& {A'} \& {A''}
	\arrow["{r(A'')}", "\shortmid"{marking}, no head, from=1-1, to=1-3]
	\arrow[""{name=0, anchor=center, inner sep=0}, "{p_{c'}}"', from=1-1, to=2-1]
	\arrow[""{name=1, anchor=center, inner sep=0}, "{i_{A''}(\delrp_{c'})}", from=1-3, to=2-3]
	\arrow["{c'}", "\shortmid"{marking}, no head, from=2-1, to=2-3]
	\arrow["\id"', from=2-1, to=3-1]
	\arrow["\id", from=2-3, to=3-3]
	\arrow["{r(A')}", "\shortmid"{marking}, no head, from=3-1, to=3-2]
	\arrow[""{name=2, anchor=center, inner sep=0}, "{p_c}"', from=3-1, to=4-1]
	\arrow["{c'}", "\shortmid"{marking}, no head, from=3-2, to=3-3]
	\arrow[""{name=3, anchor=center, inner sep=0}, "{\delrp_c}", from=3-2, to=4-2]
	\arrow[""{name=4, anchor=center, inner sep=0}, "{i_{A''}(\push_{c'}(\delrp_c))}", from=3-3, to=4-3]
	\arrow["c"', "\shortmid"{marking}, no head, from=4-1, to=4-2]
	\arrow["{c'}"', "\shortmid"{marking}, no head, from=4-2, to=4-3]
	\arrow["{\dnl_{c'}}"{description}, draw=none, from=0, to=1]
	\arrow["{\dnl_c}"{description}, draw=none, from=2, to=3]
	\arrow["{(*)}"{description}, draw=none, from=3, to=4]
\end{tikzcd}\]
    \end{itemize}

    % 2.
    \item Same as (1), replacing predomains/morphisms/relations/squares with error domains.
\end{proof}


The next two lemmas concern quasi-equivalence and the functors $\li$ and
$U$.

\begin{lemma}\label{lem:Fcc-equiv-FcFc'}
  
  Let $A$, $A'$, and $A''$ be value objects and let $c : A \rel A'$ and $c' : A'
  \rel A''$ be predomain relations. Then we have $\li(c \comp c') \bisim \li c
  \comp \li c'$.
\end{lemma}
\begin{proof}
  First, we claim that $\li(c \comp c')$ and $\li c \comp \li c'$ are both
  quasi-left-represented by $\li e_{c'} \circ \li e_c$. Indeed, we have by part
  (1) of Lemma \ref{lem:representation-comp} that $e_c' \circ e_c$
  quasi-left-represents $c \comp c'$, and then by Lemma
  \ref{lem:representation-U-F} we have $\li(e_{c'} \circ e_c) = \li e_{c'} \circ \li e_c$
  quasi-left-represents $\li (c \comp c')$.
  %
  On the other hand, we also know that $\li e_c$ quasi-left-represents $\li c$ and
  $\li e_{c'}$ quasi-left-represents $\li c'$ again by Lemma
  \ref{lem:representation-U-F}. Then by part (2) of Lemma
  \ref{lem:representation-comp}, their composition quasi-left-represents $\li c
  \comp \li c'$.

  The result now follows by Lemma \ref{lem:left-rep-by-same-morphism}.
\end{proof}

\begin{lemma}\label{lem:Udd-equiv-UdUd'}
  Let $B$, $B'$, and $B''$ be value objects and let $d : B \rel B'$ and $d' : B'
  \rel B''$ be error domain relations. Then we have $U(d \comp d') \bisim Ud \comp Ud'$.
\end{lemma}
\begin{proof}
  Analogous to the proof of the previous lemma.
\end{proof}

Now we combine the above lemmas to show a result about the functors $\li$ and
$U$ applied to a composition of relations:

\begin{lemma}\label{lem:representation-comp-F-U}
  Let $A$, $A'$, and $A''$ be value objects, and $B$, $B'$, and
  $B''$ be computation objects. Let $c : A \rel A'$ and $c' : A' \rel A''$ be
  predomain relations with push-pull structures, and let $d : B \rel B'$ and $d'
  : B' \rel B''$ be error domain relations with push-pull structures.
  \begin{enumerate}
    \item If $\li c$ and $\li c'$ are quasi-right-representable, then $\li (c
    \comp c')$ is quasi-right-representable.
    
    \item If $Ud$ and $Ud'$ are quasi-left-representable, then $U(d \comp d')$
    is quasi-left-representable.

    % \item Given left-representation structures $\rho^L_c$ for $c$ and $\rho^L_{c'}$ for $c'$,
    % we can define a left-representation structure for the composition $c \comp c'$
    % \item Given right-representation structures $\rho^R_d$ for $d$ and $\rho^R_{d'}$ for $d'$,
    % we can define a right-representation structure for the composition $d \comp d'$
    % \item Given right-representation structures $\rho^R_{Fc}$ for $Fc$ and $\rho^R_{Fc'}$ for $Fc'$,
    % we can define a right-representation structure $\rho^R_{F(c \comp c')}$ for $F(c \comp c')$.
    % \item Given left-representation structures $\rho^L_{Ud}$ for $Ud$ and $\rho^L_{Ud'}$ for $Ud'$,
    % we can define a left-representation structure $\rho^L_{U(d \comp d')}$ for $U(d \comp d')$.
  \end{enumerate}
  
\end{lemma}
\begin{proof}
  \begin{enumerate}
    % 1.
    \item We show that $\li (c \comp c')$ is quasi-right-representable.
    
    First, by Lemma \ref{lem:Fcc-equiv-FcFc'} there is a square $\alpha$ of the form

    % https://q.uiver.app/#q=WzAsNSxbMCwwLCJcXGxpIEEiXSxbMiwwLCJcXGxpIEEnJyJdLFswLDEsIlxcbGkgQSJdLFsxLDEsIlxcbGkgQSciXSxbMiwxLCJcXGxpIEEnJyJdLFswLDEsIlxcbGkgKGMgXFxjb21wIGMnKSIsMCx7InN0eWxlIjp7ImJvZHkiOnsibmFtZSI6ImJhcnJlZCJ9LCJoZWFkIjp7Im5hbWUiOiJub25lIn19fV0sWzIsMywiXFxsaSBjIiwwLHsic3R5bGUiOnsiYm9keSI6eyJuYW1lIjoiYmFycmVkIn0sImhlYWQiOnsibmFtZSI6Im5vbmUifX19XSxbMyw0LCJcXGxpIGMnIiwwLHsic3R5bGUiOnsiYm9keSI6eyJuYW1lIjoiYmFycmVkIn0sImhlYWQiOnsibmFtZSI6Im5vbmUifX19XSxbMCwyLCJpX3tcXGxpIEF9KFxcZGVsdGFebCkiLDJdLFsxLDQsImlfe1xcbGkgQScnfShcXGRlbHRhXnIpIl0sWzgsOSwiXFxhbHBoYSIsMSx7InNob3J0ZW4iOnsic291cmNlIjoyMCwidGFyZ2V0IjoyMH0sInN0eWxlIjp7ImJvZHkiOnsibmFtZSI6Im5vbmUifSwiaGVhZCI6eyJuYW1lIjoibm9uZSJ9fX1dXQ==
    \[\begin{tikzcd}[ampersand replacement=\&]
      {\li A} \&\& {\li A''} \\
      {\li A} \& {\li A'} \& {\li A''}
      \arrow["{\li (c \comp c')}", "\shortmid"{marking}, no head, from=1-1, to=1-3]
      \arrow[""{name=0, anchor=center, inner sep=0}, "{i_{\li A}(\delta^l)}"', from=1-1, to=2-1]
      \arrow[""{name=1, anchor=center, inner sep=0}, "{i_{\li A''}(\delta^r)}", from=1-3, to=2-3]
      \arrow["{\li c}", "\shortmid"{marking}, no head, from=2-1, to=2-2]
      \arrow["{\li c'}", "\shortmid"{marking}, no head, from=2-2, to=2-3]
      \arrow["\alpha"{description}, draw=none, from=0, to=1]
    \end{tikzcd}\]
    where $\delta^l$ and $\delta^r$ are syntactic perturbations.

    We define the projection $p_{\li (c \comp c')}$ to be $p_{\li c \comp \li c'} \circ i_{A''}(\delta^r)$.
    We define $\dellp_{\li (c \comp c')}$ to be $\dellp_{\li c \comp \li c'} \cdot \delta^l$.

    Then we can build the $\dnr$ square by pasting the square $\alpha$ on top of
    the $\dnr$ square for the composition $\li c \comp \li c'$, as shown below:

    % https://q.uiver.app/#q=WzAsNyxbMCwwLCJcXGxpIEEiXSxbMiwwLCJcXGxpIEEnJyJdLFswLDEsIlxcbGkgQSJdLFsxLDEsIlxcbGkgQSciXSxbMiwxLCJcXGxpIEEnJyJdLFswLDIsIlxcbGkgQSJdLFsyLDIsIlxcbGkgQSJdLFswLDEsIlxcbGkgKGMgXFxjb21wIGMnKSIsMCx7InN0eWxlIjp7ImJvZHkiOnsibmFtZSI6ImJhcnJlZCJ9LCJoZWFkIjp7Im5hbWUiOiJub25lIn19fV0sWzIsMywiXFxsaSBjIiwwLHsic3R5bGUiOnsiYm9keSI6eyJuYW1lIjoiYmFycmVkIn0sImhlYWQiOnsibmFtZSI6Im5vbmUifX19XSxbMyw0LCJcXGxpIGMnIiwwLHsic3R5bGUiOnsiYm9keSI6eyJuYW1lIjoiYmFycmVkIn0sImhlYWQiOnsibmFtZSI6Im5vbmUifX19XSxbNSw2LCJyKFxcbGkgQSkiLDIseyJzdHlsZSI6eyJib2R5Ijp7Im5hbWUiOiJiYXJyZWQifSwiaGVhZCI6eyJuYW1lIjoibm9uZSJ9fX1dLFswLDIsImlfe1xcbGkgQX0oXFxkZWx0YV5sKSIsMl0sWzIsNSwiaV9BKFxcZGVsbHBfe1xcbGkgYyBcXGNvbXAgXFxsaSBjJ30pIiwyXSxbMSw0LCJpX3tcXGxpIEEnJ30oXFxkZWx0YV5yKSJdLFs0LDYsInBfe1xcbGkgYyBcXGNvbXAgXFxsaSBjJ30iXSxbMTIsMTQsIlxcZG5yX3tcXGxpIGMgXFxjb21wIFxcbGkgYyd9IiwxLHsic2hvcnRlbiI6eyJzb3VyY2UiOjIwLCJ0YXJnZXQiOjIwfSwic3R5bGUiOnsiYm9keSI6eyJuYW1lIjoibm9uZSJ9LCJoZWFkIjp7Im5hbWUiOiJub25lIn19fV0sWzExLDEzLCJcXGFscGhhIiwzLHsic2hvcnRlbiI6eyJzb3VyY2UiOjIwLCJ0YXJnZXQiOjIwfSwic3R5bGUiOnsiYm9keSI6eyJuYW1lIjoibm9uZSJ9LCJoZWFkIjp7Im5hbWUiOiJub25lIn19fV1d
    \[\begin{tikzcd}[ampersand replacement=\&]
      {\li A} \&\& {\li A''} \\
      {\li A} \& {\li A'} \& {\li A''} \\
      {\li A} \&\& {\li A}
      \arrow["{\li (c \comp c')}", "\shortmid"{marking}, no head, from=1-1, to=1-3]
      \arrow[""{name=0, anchor=center, inner sep=0}, "{i_{\li A}(\delta^l)}"', from=1-1, to=2-1]
      \arrow[""{name=1, anchor=center, inner sep=0}, "{i_{\li A''}(\delta^r)}", from=1-3, to=2-3]
      \arrow["{\li c}", "\shortmid"{marking}, no head, from=2-1, to=2-2]
      \arrow[""{name=2, anchor=center, inner sep=0}, "{i_A(\dellp_{\li c \comp \li c'})}"', from=2-1, to=3-1]
      \arrow["{\li c'}", "\shortmid"{marking}, no head, from=2-2, to=2-3]
      \arrow[""{name=3, anchor=center, inner sep=0}, "{p_{\li c \comp \li c'}}", from=2-3, to=3-3]
      \arrow["{r(\li A)}"', "\shortmid"{marking}, no head, from=3-1, to=3-3]
      \arrow["\alpha"{marking, allow upside down}, draw=none, from=0, to=1]
      \arrow["{\dnr_{\li c \comp \li c'}}"{description}, draw=none, from=2, to=3]
    \end{tikzcd}\]

    We define $\delrp_{\li (c \comp c')}$ to be $\delrp_{\li c \comp \li c'} \cdot \delta^r$.
    For $\dnl$, we paste the identity square $\delta^r \ltdyn \delta^r$ on top of
    the $\dnl$ square for the composition $\li c \comp \li c'$, and below that we paste
    the square $\id \ltdyn_{\li (c \comp c')}^{\li c \comp \li c'} \id$ which we get from
    the fact that $\li$ is lax.

    % https://q.uiver.app/#q=WzAsOSxbMCwwLCJcXGxpIEEnJyJdLFsyLDAsIlxcbGkgQScnIl0sWzAsMSwiXFxsaSBBJyciXSxbMiwxLCJcXGxpIEEnJyJdLFswLDIsIlxcbGkgQSJdLFsyLDIsIlxcbGkgQScnIl0sWzAsMywiXFxsaSBBIl0sWzIsMywiXFxsaSBBJyciXSxbMSwyLCJcXGxpIEEnIl0sWzAsMSwicihcXGxpIEEnJykiLDAseyJzdHlsZSI6eyJib2R5Ijp7Im5hbWUiOiJiYXJyZWQifSwiaGVhZCI6eyJuYW1lIjoibm9uZSJ9fX1dLFsyLDMsInIoXFxsaSBBJycpIiwwLHsic3R5bGUiOnsiYm9keSI6eyJuYW1lIjoiYmFycmVkIn0sImhlYWQiOnsibmFtZSI6Im5vbmUifX19XSxbNiw3LCJcXGxpIChjIFxcY29tcCBjJykiLDIseyJzdHlsZSI6eyJib2R5Ijp7Im5hbWUiOiJiYXJyZWQifSwiaGVhZCI6eyJuYW1lIjoibm9uZSJ9fX1dLFswLDIsImlfe0EnJ30oXFxkZWx0YV5yKSIsMl0sWzIsNCwicF97XFxsaSBjIFxcY29tcCBcXGxpIGMnfSIsMl0sWzQsNiwiXFxpZCIsMl0sWzEsMywiaV97QScnfShcXGRlbHRhXnIpIl0sWzMsNSwiaV97QScnfShcXGRlbHJwX3tcXGxpIGMgXFxjb21wIFxcbGkgYyd9KSJdLFs1LDcsIlxcaWQiXSxbNCw4LCJcXGxpIGMiLDAseyJzdHlsZSI6eyJib2R5Ijp7Im5hbWUiOiJiYXJyZWQifSwiaGVhZCI6eyJuYW1lIjoibm9uZSJ9fX1dLFs4LDUsIlxcbGkgYyciLDAseyJzdHlsZSI6eyJib2R5Ijp7Im5hbWUiOiJiYXJyZWQifSwiaGVhZCI6eyJuYW1lIjoibm9uZSJ9fX1dLFsxMiwxNSwiXFxpZCIsMyx7InNob3J0ZW4iOnsic291cmNlIjoyMCwidGFyZ2V0IjoyMH0sInN0eWxlIjp7ImJvZHkiOnsibmFtZSI6Im5vbmUifSwiaGVhZCI6eyJuYW1lIjoibm9uZSJ9fX1dLFsxMywxNiwiXFxkbmxfe1xcbGkgYyBcXGNvbXAgXFxsaSBjJ30iLDEseyJzaG9ydGVuIjp7InNvdXJjZSI6MjAsInRhcmdldCI6MjB9LCJzdHlsZSI6eyJib2R5Ijp7Im5hbWUiOiJub25lIn0sImhlYWQiOnsibmFtZSI6Im5vbmUifX19XV0=
    \[\begin{tikzcd}[ampersand replacement=\&]
      {\li A''} \&\& {\li A''} \\
      {\li A''} \&\& {\li A''} \\
      {\li A} \& {\li A'} \& {\li A''} \\
      {\li A} \&\& {\li A''}
      \arrow["{r(\li A'')}", "\shortmid"{marking}, no head, from=1-1, to=1-3]
      \arrow[""{name=0, anchor=center, inner sep=0}, "{i_{A''}(\delta^r)}"', from=1-1, to=2-1]
      \arrow[""{name=1, anchor=center, inner sep=0}, "{i_{A''}(\delta^r)}", from=1-3, to=2-3]
      \arrow["{r(\li A'')}", "\shortmid"{marking}, no head, from=2-1, to=2-3]
      \arrow[""{name=2, anchor=center, inner sep=0}, "{p_{\li c \comp \li c'}}"', from=2-1, to=3-1]
      \arrow[""{name=3, anchor=center, inner sep=0}, "{i_{A''}(\delrp_{\li c \comp \li c'})}", from=2-3, to=3-3]
      \arrow["{\li c}", "\shortmid"{marking}, no head, from=3-1, to=3-2]
      \arrow["\id"', from=3-1, to=4-1]
      \arrow["{\li c'}", "\shortmid"{marking}, no head, from=3-2, to=3-3]
      \arrow["\id", from=3-3, to=4-3]
      \arrow["{\li (c \comp c')}"', "\shortmid"{marking}, no head, from=4-1, to=4-3]
      \arrow["\id"{marking, allow upside down}, draw=none, from=0, to=1]
      \arrow["{\dnl_{\li c \comp \li c'}}"{description}, draw=none, from=2, to=3]
    \end{tikzcd}\]


    % 2.
    \item The proof that $U(d \comp d')$ is quasi-left-representable is analogous.

  \end{enumerate}
\end{proof}

% Want to show: U(d \comp d') is weakly equivalent to U(d) \comp U(d').
% This holds because both are quasi-representable by the same projection
% \begin{lemma}
%   Let $\mathcal M$ be a step-3 intensional model, and let
%   $d : B \rel B'$ and $d' : B' \rel B''$.
%   Let $\rho_\text{comp}$ be an arbitrary right-representation for $d \comp d'$.
%   Then the projection $p_{d \circ d'}$ is weakly equivalent to
% \end{lemma}




%%%%%%%%%%%%
% Products %
%%%%%%%%%%%%
\begin{lemma}\label{lem:representation-product} 
  Let $c_1 : A_1 \rel A_1'$ and $c_2 : A_2 \rel A_2'$. 
  \begin{enumerate}
    \item If $c_1$ and $c_2$ are
    quasi-left-representable, then $c_1 \times c_2$ is quasi-left-representable.
    \item If $\li c_1$ and $\li c_2$ are quasi-right-representable, then so
    is $\li (c_1 \times c_2)$.
  \end{enumerate}

  % Let $\rho^L_{c_1}$ be a left-representation structure for $c_1$, and
  % let $\rho^L_{c_2}$ be a left-representation structure for $c_2$.
  % Then we can define a left-representation structure for $c_1 \times c_2$.

  % Likewise, let $\rho^R_{Fc_1}$ and $\rho^R_{Fc_2}$ be right-representation
  % structures for $Fc_1$ and $Fc_2$ respectively.
  % Then we can define a right-representation structure for $F(c_1 \times c_2)$.
\end{lemma}
\begin{proof}
  \begin{enumerate}
    \item To show $c_1 \times c_2$ is quasi-left-representable, we define
    \begin{itemize}
      \item $e_{c_1 \times c_2} = e_{c_1} \times e_{c_2}$
      \item $\delre_{c_1 \times c_2} = \inl (\delre_{c_1}) \cdot \inr
      (\delre_{c_2})$ (recall from Section \ref{sec:perturbation-constructions}
      that $M_{A_1' \times A_2'}$ is the coproduct $M_{A_1'} \oplus M_{A_2'}$)
      \item $\delle_{c_1 \times c_2}$ is defined similarly
      \item The $\upl$ square is constructed using the functorial action of
      $\times$ and the corresponding squares for $c_1$ and $c_2$:
      % https://q.uiver.app/#q=WzAsNixbMCwwLCJBXzEgXFx0aW1lcyBBXzIiXSxbMCwxLCJBXzEnIFxcdGltZXMgQV8yIl0sWzAsMiwiQV8xJyBcXHRpbWVzIEFfMiciXSxbMSwwLCJBXzEnIFxcdGltZXMgQV8yJyJdLFsxLDEsIkFfMScgXFx0aW1lcyBBXzInIl0sWzEsMiwiQV8xJyBcXHRpbWVzIEFfMiciXSxbMCwzLCJjXzEgXFx0aW1lcyBjXzIiLDAseyJzdHlsZSI6eyJib2R5Ijp7Im5hbWUiOiJiYXJyZWQifSwiaGVhZCI6eyJuYW1lIjoibm9uZSJ9fX1dLFsxLDQsInIoQV8xJykgXFx0aW1lcyBjXzIiLDAseyJzdHlsZSI6eyJib2R5Ijp7Im5hbWUiOiJiYXJyZWQifSwiaGVhZCI6eyJuYW1lIjoibm9uZSJ9fX1dLFsyLDUsInIoQV8xJykgXFx0aW1lcyByKEFfMicpIiwyLHsic3R5bGUiOnsiYm9keSI6eyJuYW1lIjoiYmFycmVkIn0sImhlYWQiOnsibmFtZSI6Im5vbmUifX19XSxbMCwxLCJlX3tjXzF9IFxcdGltZXMgXFxpZCIsMl0sWzEsMiwiXFxpZCBcXHRpbWVzIGVfe2NfMn0iLDJdLFszLDQsImlfe0FfMSd9KFxcZGVscmVfe2NfMX0pIFxcdGltZXMgXFxpZCJdLFs0LDUsIlxcaWQgXFx0aW1lcyBpX3tBXzInfShcXGRlbHJlX3tjXzJ9KSJdLFs5LDExLCJcXHVwbF97Y18xfSBcXHRpbWVzIFxcaWQiLDEseyJzaG9ydGVuIjp7InNvdXJjZSI6MjAsInRhcmdldCI6MjB9LCJzdHlsZSI6eyJib2R5Ijp7Im5hbWUiOiJub25lIn0sImhlYWQiOnsibmFtZSI6Im5vbmUifX19XSxbMTAsMTIsIlxcaWQgXFx0aW1lcyBcXHVwbF97Y18yfSIsMSx7InNob3J0ZW4iOnsic291cmNlIjoyMCwidGFyZ2V0IjoyMH0sInN0eWxlIjp7ImJvZHkiOnsibmFtZSI6Im5vbmUifSwiaGVhZCI6eyJuYW1lIjoibm9uZSJ9fX1dXQ==
        \[\begin{tikzcd}[ampersand replacement=\&]
          {A_1 \times A_2} \& {A_1' \times A_2'} \\
          {A_1' \times A_2} \& {A_1' \times A_2'} \\
          {A_1' \times A_2'} \& {A_1' \times A_2'}
          \arrow["{c_1 \times c_2}", "\shortmid"{marking}, no head, from=1-1, to=1-2]
          \arrow[""{name=0, anchor=center, inner sep=0}, "{e_{c_1} \times \id}"', from=1-1, to=2-1]
          \arrow[""{name=1, anchor=center, inner sep=0}, "{i_{A_1'}(\delre_{c_1}) \times \id}", from=1-2, to=2-2]
          \arrow["{r(A_1') \times c_2}", "\shortmid"{marking}, no head, from=2-1, to=2-2]
          \arrow[""{name=2, anchor=center, inner sep=0}, "{\id \times e_{c_2}}"', from=2-1, to=3-1]
          \arrow[""{name=3, anchor=center, inner sep=0}, "{\id \times i_{A_2'}(\delre_{c_2})}", from=2-2, to=3-2]
          \arrow["{r(A_1') \times r(A_2')}"', "\shortmid"{marking}, no head, from=3-1, to=3-2]
          \arrow["{\upl_{c_1} \times \id}"{description}, draw=none, from=0, to=1]
          \arrow["{\id \times \upl_{c_2}}"{description}, draw=none, from=2, to=3]
        \end{tikzcd}\]
      \item The $\upr$ square is defined similarly
    \end{itemize}

    \item To show that $\li(c_1 \times c_2)$ is quasi-right-representable, we define
    \begin{itemize}
      \item $p_{\li (c_1 \times c_2)} = (p_{\li c_1} \timesk A_2) \circ (A_1' \timesk p_{\li c_2})$
      \item $\dellp_{\li (c_1 \times c_2)} = (\dellp_{\li c_1} \timesk \id) \cdot (\id \timesk \dellp_{\li c_2})$ 
      using the Kleisli product actions on syntactic perturbations in Definition \ref{def:kleisli-product-perturbations}
      \item $\delrp_{\li (c_1 \times c_2)} = (\delrp_{\li c_1} \timesk \id) \cdot (\id \timesk \delrp_{\li c_2})$
      \item The squares are obtained via the functorial action of $\timesk$ on the squares for $\li c_1$ and $\li c_2$.
    \end{itemize}
  \end{enumerate}
\end{proof}


%%%%%%%%%
% Arrow %
%%%%%%%%%
\begin{lemma}\label{lem:representation-arrow}
  Let $c : A \rel A'$ and $d : B \rel B'$.
  %
  \begin{enumerate}
    \item If $c$ is quasi-left-representable and $d$ is quasi-right-representable, then
    $c \arr d$ is quasi-right-representable.
  
    \item If $\li c$ is quasi-right-representable and $Ud$ is
    quasi-left-representable, then $U(c \arr d)$ is quasi-left-representable.
  \end{enumerate}
  
  % Let $\rho^L_c$ be a left-representation structure for $c$,
  % and let $\rho^R_d$ be a right-representation structure for $d$.
  % Then we can define a right-representation structure for $c \arr d$.

  % Likewise, let $\rho^R_{Fc}$ be a right-representation structure for $Fc$,
  % and let $\rho^L_{Ud}$ be a left-representation structure for $Ud$.
  % Then we can define a left-representation structure for $U(c \arr d)$.
\end{lemma}
\begin{proof}
  \begin{enumerate}
    \item 
      To show $c \arr d$ is quasi-right-representable, we define:
      \begin{itemize}
        \item $p_{c \arr d} = e_c \arr p_d$
        \item $\dellp_{c \arr d} = \inl(\delle_c) \cdot \inr(\dellp_d)$
        \item $\delrp_{c \arr d} = \inl(\delre_c) \cdot \inr(\delrp_d)$
        \item The squares $\dnr$ and $\dnl$ are obtained via the functorial action of $\arr$ on squares.
        
      \end{itemize}

    % TODO check this
    \item
      We show $U(c \arr d)$ is quasi-left-representable as follows:
      \begin{itemize}
        \item $e_{U(c \arr d)} = (p_{\li c} \tok B') \circ (A \tok e_{Ud})$
        \item $\delre_{U(c \arr d)} = (\delrp_{\li c} \tok B') \cdot (A' \tok
              \delre_{Ud})$ using the Kleisli arrow actions on syntactic
              perturbations in Definition \ref{def:kleisli-arrow-perturbations}.
        \item $\delle_{U(c \arr d)} = (\dellp_{\li c} \tok B) \cdot (A \tok \delle_{Ud})$
        \item The squares $\upl$ and $\upr$ are obtained via the functorial action of $\tok$ on squares.
        For instance, $\upl$ is given by the following square:

        % https://q.uiver.app/#q=WzAsNixbMCwwLCJVKEEgXFx0byBCKSJdLFsyLDAsIlUoQScgXFx0byBCJykiXSxbMiwxLCJVKEEnIFxcdG8gQicpIl0sWzIsMiwiVShBJyBcXHRvIEInKSJdLFswLDEsIlUoQSBcXHRvIEInKSJdLFswLDIsIlUoQScgXFx0byBCJykiXSxbMCwxLCJVKGMgXFx0byBkKSJdLFs0LDIsIlUoYyBcXHRvIHIoQicpKSJdLFs1LDMsIlUocihBJykgXFx0byByKEInKSkiXSxbNCw1LCJwX3tGY30gXFx0b2sgQiciLDJdLFsxLDIsIkEnIFxcdG9rIFxcZGVscmVfe1VkfSJdLFsyLDMsIlxcZGVscnBfe0ZjfSBcXHRvayBCJyJdLFswLDQsIkEgXFx0b2sgZV97VWR9IiwyXSxbMTIsMTAsIlxcaWRfe0ZjfSBcXHRvayBcXHVwbF97VWR9IiwxLHsic2hvcnRlbiI6eyJzb3VyY2UiOjIwLCJ0YXJnZXQiOjIwfSwic3R5bGUiOnsiYm9keSI6eyJuYW1lIjoibm9uZSJ9LCJoZWFkIjp7Im5hbWUiOiJub25lIn19fV0sWzksMTEsIlxcZG5sX3tGY30gXFx0b2sgXFxpZF97cihCJyl9IiwxLHsic2hvcnRlbiI6eyJzb3VyY2UiOjIwLCJ0YXJnZXQiOjIwfSwic3R5bGUiOnsiYm9keSI6eyJuYW1lIjoibm9uZSJ9LCJoZWFkIjp7Im5hbWUiOiJub25lIn19fV1d
        \[\begin{tikzcd}[ampersand replacement=\&,row sep=large]
          {U(A \to B)} \&\& {U(A' \to B')} \\
          {U(A \to B')} \&\& {U(A' \to B')} \\
          {U(A' \to B')} \&\& {U(A' \to B')}
          \arrow["{U(c \to d)}", from=1-1, to=1-3]
          \arrow["{U(c \to r(B'))}", from=2-1, to=2-3]
          \arrow["{U(r(A') \to r(B'))}", from=3-1, to=3-3]
          \arrow[""{name=0, anchor=center, inner sep=0}, "{p_{Fc} \tok B'}"', from=2-1, to=3-1]
          \arrow[""{name=1, anchor=center, inner sep=0}, "{A' \tok \delre_{Ud}}", from=1-3, to=2-3]
          \arrow[""{name=2, anchor=center, inner sep=0}, "{\delrp_{Fc} \tok B'}", from=2-3, to=3-3]
          \arrow[""{name=3, anchor=center, inner sep=0}, "{A \tok e_{Ud}}"', from=1-1, to=2-1]
          \arrow["{\id_{Fc} \tok \upl_{Ud}}"{description}, draw=none, from=3, to=1]
          \arrow["{\dnl_{Fc} \tok \id_{r(B')}}"{description}, draw=none, from=0, to=2]
        \end{tikzcd}\]
      \end{itemize}

    The construction of $\upr$ is similar.
\end{enumerate}
\end{proof}


\subsection{Model Construction: Composition and Functorial Actions on Relations}

We now show that our notions of value and computation relations compose.
\begin{definition}[composition of value (resp. computation) relations]\label{def:value-computation-rel-comp}
  Let $A$, $A'$ and $A''$ be value objects and let $c$ and $c'$ be \emph{value
  relations} between $A$ and $A'$ and $A'$ and $A''$ respectively. Then we
  define their composition $cc'$ as follows:

  \begin{itemize}
    \item The predomain relation is the composition of the underlying predomain
    relations $cc'$
    \item The push-pull structure is given by Lemma \ref{lem:push-pull-comp}.
    \item Quasi-left-representability of $cc'$ follows from Lemma
    \ref{lem:representation-comp} and the fact that $c$ and $c'$ are
    quasi-left-representable
    \item Quasi-right-representability of $\li(cc')$ holds by Lemma
    \ref{lem:representation-comp-F-U} and the quasi-right-representability of
    $\li c$ and $\li c'$
  \end{itemize}

  Likewise, let $B$, $B'$ and $B''$ be computation objects and let $d$ and $d'$ be
  \emph{computation relations} between $B$ and $B'$ and $B'$ and $B''$
  respectively. We define their composition $dd'$ as follows:
  
  \begin{itemize}
    \item The error domain relation is the composition of the underlying error
    domain relations $dd'$
    \item The push-pull structure is given by Lemma \ref{lem:push-pull-comp}.
    \item Quasi-right-representability of $dd'$ follows from Lemma
    \ref{lem:representation-comp} and the fact that $d$ and $d'$ are
    quasi-right-representable
    \item Quasi-left-representability of $U(dd')$ holds by Lemma
    \ref{lem:representation-comp-F-U} and the quasi-left-representability of
    $Ud$ and $Ud'$
  \end{itemize}
\end{definition}


We can define the functorial actions of $\li$, $U$, $\times$, and $\arr$ on
value and computation relations as follows:
\begin{definition}[functorial actions on relations]\label{def:functorial-actions-on-relations}
  
  \begin{enumerate}

    \item Let $A$ and $A'$ be value objects and $c : A \rel A'$ a value relation.
          Then we define the computation relation $\li c$ as follows:
      \begin{itemize}
        \item The error domain relation is given by the action of $\li$ on the
        underlying predomain relation of $c$
        \item The push-pull structure is given by Lemma \ref{lem:push-pull-U-F}
        \item Quasi-right-representability of $\li c$ holds because it is part
        of the definition of the value relation $c$
        \item Quasi-left-representability of $U \li c$ follows from Lemma
        \ref{lem:representation-U-F} and the quasi-left-representability of $c$
      \end{itemize}
          
    \item Let $B$ and $B'$ be computation objects and $d : B \rel B'$ a computation relation.
          We define $Ud$ as follows:
          \begin{itemize}
            \item The predomain relation is given by the action of $U$ on the
            error domain relation of $d$
            \item The push-pull structure is given by Lemma
            \ref{lem:push-pull-U-F}
            \item Quasi-left-representability of $Ud$ holds because it is part
            of the definition of the computation relation $d$
            \item Quasi-right-representability of $\li(Ud)$ follows from Lemma
            \ref{lem:representation-U-F} and the quasi-right-representability of
            $d$
          \end{itemize}

    \item Let $c_1 : A_1 \rel A_1'$ and $c_2 : A_2 \rel A_2'$ be value relations.
          We define $c_1 \times c_2$ as follows:
          \begin{itemize}
            \item The relation is given by the action of $\times$ on the
            predomain relations of $c_1$ and $c_2$
            \item The push-pull structure is given by Lemma
            \ref{lem:push-pull-times}
            \item Quasi-left-representability of $c_1 \times c_2$ and
            quasi-right-representabiity of $\li(c_1 \times c_2)$ are given by
            Lemma \ref{lem:representation-product}
          \end{itemize}

    \item Let $c : A \rel A'$ be a value relation and $d : B \rel B'$ be a computation relation.
          We define $c \arr d$ as follows:
          \begin{itemize}
            \item The error domain relation is given by the action of $\arr$ on
            the underlying relations of $c$ and $d$
            \item The push-pull structure is given by Lemma
            \ref{lem:push-pull-arrow}
            \item Quasi-right-representability of $c \arr d$, and
                  quasi-left-representability of $U(c \arr d)$ follow from Lemma
                  \ref{lem:representation-arrow}
          \end{itemize}

  \end{enumerate}
\end{definition}

Lastly, we prove quasi-equivalence of relations involving composition and the
functors $U$, $\li$, $\to$ and $\times$.

\begin{lemma}\label{lem:quasi-order-equiv-functors}
  The following hold:
  \begin{itemize}
    \item $U(d \comp d') \qordeq U(d) \comp U(d')$
    \item $\li(c \comp c') \qordeq \li (c) \comp \li (c')$
    \item $(c \comp c') \to (d \comp d') \qordeq (c \to d) \comp (c' \to d')$
    \item $(c_1 \comp c_1') \times (c_2 \comp c_2') \qordeq (c_1 \times c_2) \comp (c_1'\times c_2')$
  \end{itemize}
\end{lemma}
\begin{proof}
  (1) and (2) were already shown in Lemmas \ref{lem:Fcc-equiv-FcFc'} and
  \ref{lem:Udd-equiv-UdUd'}. (3) follows from observing that both relations are
  right-represented by the morphism $e_c'e_c \arr p_{d}p_{d'}$ and (4) from the
  fact that both are left-represented by the morphism $e_{c_1'} e_{c_1} \times
  e_{c_2'} e_{c_2}$
\end{proof}

% We now establish the quasi-order-equivalence for the functors.
% We already showed that $U(d \comp d') \qordeq U(d)U(d')$ and $F(c \comp c'') \qordeq F(c)F(c')$
% in the proof of Lemma \ref{lem:representation-comp}.
% The other two properties
% $(cc') \to (dd') \qordeq (c \to d)(c' \to d')$ and
% $(c_1c_1') \times (c_2c_2') \qordeq (c_1 \times c_2)(c_1'\times c_2')$
% are proved similarly, noting in both cases that the both relations are
% quasi-represented by the same morphism.

\subsection{Definitions of Error Ordering and Weak Bisimilarity for the Delay Monad}
\label{sec:relations-on-delay-monad}

We define a notion of lock-step error ordering and weak bisimilarity relation
for the coinductive delay monad $\delay(\mathbb{N} + {\mho})$:

The lock-step error ordering is defined coinductively by the following rules:

%
\begin{mathpar}
  \inferrule*[]
  { }
  {\tnow (\inr\, 1) \ledelay d}

  \inferrule*[]
  {x_1 \le_X x_2}
  {\tnow (\inl\, x_1) \ledelay \tnow (\inl\, x_2)}

  \inferrule*[]
  {d_1 \ledelay d_2}
  {\tlater\, d_1 \ledelay \tlater\, d_2}
\end{mathpar}
%
And we similarly define by coinduction a ``weak bisimilarity'' relation on
$\delay(\mathbb{N} + {\mho})$. This uses a relation $d \Da x_?$ between
$\delay(\mathbb{N} + {\mho})$ and $\mathbb{N} + {\mho}$ that is defined as $d
\Da n_? := \Sigma_{i \in \mathbb{N}} d = \tlater^i(\tnow\, n_?)$. Then weak
bisimilarity for the delay monad is defined coinductively by the rules
%
\begin{mathpar}
  \inferrule*[]
  {n_? \bisim_{\mathbb{N} + {\mho}} m_?}
  {\tnow\, n_? \bisimdelay \tnow\, m_? }

  \inferrule*[leftskip=1.5em]
  {d_1 \Da n_? \and n_? \bisim_{\mathbb{N} + {\mho}} m_?}
  {\tlater\, d_1 \bisimdelay \tnow\, m_? }

  \inferrule*[leftskip=1.5em]
  {d_2 \Da m_? \and n_? \bisim_{\mathbb{N} + {\mho}} m_?}
  {\tnow\, n_? \bisimdelay \tlater\, d_2}

  \inferrule*[leftskip=1.5em]
  {d_1 \bisimdelay d_2}
  {\tlater\, d_1 \bisimdelay \tlater\, d_2 }
  %
  % \inferrule*[]
  % {d_1 \Da x_? \and d_2 \Da y_? \and x_? \bisim_{X + 1} y_?}
  % {d_1 \bisimdelay d_2}
  %
  % \inferrule*[]
  % {d_1 \bisimdelay d_2}
  % {\tlater d_1 \bisimdelay \tlater d_2 }
\end{mathpar}



% %%%%%%%%%%%%%%%%%%%%%%
% % Model Construction %
% %%%%%%%%%%%%%%%%%%%%%%
% Now we can give the proof of the main lemma:
%   % Write 
%   % %
%   % \[ \mathcal M = (\vf, \vsq, \ef, \esq, \Ff, \Fsq, \Uf, \Usq, \arrf, \arrsq). \] 
%   % %

%   We define a step-3 model $\mathcal M'$ as follows:
%   \begin{itemize}
%     \item The objects of $\mathcal M'$ are defined to be the same as the objects of $\mathcal M$.
%     \item The value and computation morphisms in $\mathcal M'$ are the same as those of $\mathcal M$.
%     \item A value relation is defined to be a tuple $(c, \rho^L_c, \rho^R_{Fc})$ with
%     \begin{itemize}
%       \item $c$ a value relation in $\mathcal M$,
%       \item $\rho^L_c$ a left-representation structure for $c$, and
%       \item $\rho^R_{Fc}$ a right-representation structure for $Fc$
%     \end{itemize}
%     \item Likewise, a computation relation is defined to be a tuple $(d, \rho^R_d, \rho^L_{Ud})$ with
%     \begin{itemize}
%       \item $d$ a computation relation in $\mathcal M$,
%       \item $\rho^R_d$ a right-representation structure for $d$, and
%       \item $\rho^L_{Ud}$ a left-representation structure for $Ud$.
%     \end{itemize}
%     \item Morphisms of value relations (i.e., the value squares) are defined by simply
%     ignoring the representation structures. That is, a morphism of value relations
%     $\alpha \in \vsq'((c, \rho^L_c, \rho^R_{Fc}), (c' \rho^L_{c'}, \rho^R_{Fc'}))$ is simply a morphism of value
%     relations in $\vsq(c, c')$. Likewise for computations.
%   \end{itemize}

% We define horizontal composition of relations and squares as follows:
% Let $c : A \rel A'$ and $c' : A' \rel A''$. We define
% %
% \begin{align*} 
%   ((c, \rho^L_c, &\rho^R_{Fc}) \comp (c', \rho^L_{c'}, \rho^R_{Fc'})) =
%   (c \comp c', \rho^L_{c \comp c'}, \rho^R_{F(c \comp c')}), 
% \end{align*}
% %
% and
% %
% \begin{align*}
%   ((d, \rho^R_d, &\rho^L_{Ud}) \comp (d', \rho^R_{d'}, \rho^L_{Ud'})) =
%   (d \comp d', \rho^R_{d \comp d'}, \rho^L_{U(d \comp d')}),
% \end{align*}
% %
% where $\rho^L_{c \comp c'}, \rho^R_{F(c \comp c')}, \rho^R_{d \comp d'},$
% and $\rho^L_{U(d \comp d')}$ are as defined in Lemma \ref{lem:representation-comp}.

% Now we define the functors $F$, $U$, $\times$, and $\arr$.
% On objects, the behavior is the same as the respective functors in $\mathcal M$.
% For relations, we define

% \[ F(c, \rho^L_c, \rho^R_{Fc}) = 
%   (F c, \rho^R_{Fc}, \rho^L_{UF(c)}), \]
% and
% \[ U(d, \rho^R_d, \rho^L_{Ud}) = 
%   (U d, \rho^L_{Ud}, \rho^R_{FU(d)}), \]

% where $\rho^L_{UF(c)}$ and $\rho^R_{FU(d)}$ are as defined in the proof of Lemma
% \ref{lem:representation-UF-FU}.

% We define \[ (c_1, \rho^L_{c_1}, \rho^R_{Fc_1}) \times (c_2, \rho^L_{c_2}, \rho^R_{Fc_2}) = 
%              (c_1 \times c_2, \rho^L_{c_1 \times c_2}, \rho^R_{F(c_1 \times c_2)}), \]

% where $\rho^L_{c_1 \times c_2}$ and $\rho^R_{F(c_1 \times c_2)}$ are as defined in the
% proof of Lemma \ref{lem:representation-product}.

% Lastly, we define

% \[ (c, \rho^L_c, \rho^R_{Fc}) \arr (d, \rho^R_d, \rho^L_{Ud}) =
%     (c \arr d, \rho^R_{c \arr d}, \rho^L_{U(c \arr d)}), \]

% where $\rho^R_{c \arr d}$ and $\rho^L_{U(c \arr d)}$ are as defined in the proof
% of Lemma \ref{lem:representation-arrow}.

% % We define $(c, \rho^L_c) \arr (d, \rho^R_d) = (c \arr d, \rho^R_{c \arr d})$.




%%%%%%%%%%%%%%%%%%%%%%%%%%%%%%%%%%%%%%%%%%%%%%%%%%%%%%%%%%%%%%%%%%%%%%%%%%%%%%


\section{Constructing a Concrete Model}\label{sec:concrete-model}

In this section, we complete the construction of a relational model of gradual
typing. To manage the complexity, we carry out the construction in phases,
beginning with a relational model consisting of partially-ordered sets with a
notion of bisimilarity and then adding requirements for a structured set of
delays on each type as well as a notion of representability for relations that
models the cast rules.

\subsection{Phase One: Predomains and Error Domains}

In the previous section, we introduced the notion of weak bisimilarity as a
relation on the guarded lift monad. Recall that weak bisimilarity on $\li A$ is
parameterized by a relation on $A$. Thus, in order to define a compositional
model for all types, we need to equip the objects (i.e. the posets and error
domains) a reflexive, symmetric bisimilarity relation. This leads to our first
new definition: 

\begin{definition}
A \textbf{predomain} $A$ consists of a set $A$ along with two relations:
\begin{itemize}
    \item A partial order $\le_A$.
    \item A reflexive, symmetric ``bisimilarity'' relation $\bisim_A$.
\end{itemize}
\end{definition}

We will generally write $A$ for both a predomain and its underlying set; if we
need to emphasize the difference we will write $|A|$ for the underlying set of
$A$.

Given a predomain $A$, we can form the predomain $\later A$. The underlying set
is $\later |A|$ and the relation is defined in the obvious way, i.e., $\tilde{x}
\le_{\later A} \tilde{x'}$ iff $\later_t(\tilde{x}_t \le_A \tilde{x'}_t)$.
Likewise for bisimilarity.
%
We also give a predomain structure to the natural numbers $\mathbb{N}$, where
both the ordering and the bisimilarity relation are equality.
%
Morphisms of predomains are functions between the underlying sets that preserve
the ordering and the bisimilarity relation. More formally:
%
\begin{definition}
Let $A$ and $A'$ be predomains.
A morphism $f : A \to A'$ is a function between the underlying sets such that for all $x, x'$,
if $x \le_A x'$, then $f(x) \le f(x')$, and if $x \bisim_A x'$, then $f(x) \bisim_{A'} f(x')$.
\end{definition}

The definition of error domain is similar to that given in the previous section,
except that now, an error domain also has a notion of bisimilarity. We
additionally impose a condition relating the $\theta$ morphism to the
bisimilarity relation. More concretely:
%
\begin{definition}
An \textbf{error domain} $B$ consists of a predomain $B$ along with the following data:
\begin{itemize}
    \item A distinguished ``error" element $\mho_B \in B$
    \item A morphism of predomains $\theta_B \colon \later B \to B$
    \item For all $x : B$ we have $\theta_B(\nxt\, x) \bisim_B x$
\end{itemize}
\end{definition}
%
For an error domain $B$, we define the predomain morphism $\delta_B := \theta_B
\circ \nxt$.
%
As before, a morphism of error domains is simply a morphism of the underlying
predomains that preserves the error element and the $\theta$ map.

%
The definition of monotone relations on predomains is the same as that for
posets introduced in the previous section, as the bisimilarity relations of the
predomains are not involved in the definition of monotone relation. Likewise,
the definition of error domain relations is unchanged.
%
The definitions of squares for predomains and error domains are also unchanged
from the previous section, except of course that the morphisms on the left and
right, being morphisms of predomains, must also preserve the bisimilarity
relation.

\begin{comment}
We define a (monotone) relation on predomains $A$ and $A'$ to be a relation on
the underlying sets that is downward-closed under $\le_A$ and upward-closed
under $\le_{A'}$. More formally:
%
\begin{definition}
Let $A$ and $A'$ be predomains. A \emph{predomain relation} between $A$ and $A'$
is a relation $R$ between the underlying sets such that:
\begin{enumerate}
    \item (Downward closure): For all $x_1, x_2 \in A$ and $y \in A'$,
    if $x_1 \le_A x_2$ and $x_2 \mathbin{R} y$, then $x_1 \mathbin{R} y$.
    \item (Upward closure): For all $x \in A$ and $y_1, y_2 \in A'$,
    if $x \mathbin{R} y_1$ and $y_1 \le_{A'} y_2$, then $x \mathbin{R} y_2$.
\end{enumerate}
\end{definition}
%
Composition of relations on predomains is the usual relational composition
(which is truncated to be propositional). Similarly, we define a (monotone)
relation on error domains to be a relation on the underlying predomains that
respects error and preserves $\theta$:
%
\begin{definition}
    Let $B$ and $B'$ be error domains. An \emph{error domain relation} between
    $B$ and $B'$ is a relation $R$ between the underlying predomains such that
    \begin{enumerate}
       \item (Respects error): For all $y \in B'$, we have $\mho_B \mathbin{R} y$.
       \item (Preserves $\theta$): For all $\tilde{x}$ in $\later B$ and $\tilde{y} \in \later B'$,
       if $\later_t( \tilde{x}_t \mathbin{R} \tilde{y}_t )$ then
       $\theta_B(\tilde{x}) \mathbin{R} \theta_{B'}(\tilde{y})$.
    \end{enumerate}
\end{definition}
%
Because of a technical restriction involving the interaction between
propositional truncation and the later modality, we cannot define the
composition of error domain relations as simply the composition of their
underlying predomain relations. Instead, we define composition of error domain
relations $d$ on $B_1$ and $B_2$ and $d'$ on $B_2$ and $B_3$ to be the least
relation containing $d$ and $d'$ that is downward closed under $\le_{B_1}$,
upward-closed under $\le_{B_3}$, respects error, and preserves $\theta$.
Specifically, it is defined inductively by the following rules:
%
\begin{mathpar}
    \inferrule*[right = Comp]
    {b_1 \mathbin{d} b_2 \and b_2 \mathbin{d'} b_3}
    {b_1 \mathbin{d \relcomp d'} b_3}

    \inferrule*[right = DnClosed]
    {b_1' \le_{B_1} b_1 \and b_1 \mathbin{d \relcomp d'} b_3}
    {b_1' \mathbin{d \relcomp d'} b_3}

    \inferrule*[right = UpClosed]
    {b_1 \mathbin{d \relcomp d'} b_3 \and b_3 \le_{B_3} b_3'}
    {b_1 \mathbin{d \relcomp d'} b_3'}

    \inferrule*[right = PresErr]
    { }
    {\mho_{B_1} \mathbin{d \relcomp d'} b_3}

    \inferrule*[right = PresTheta]
    {\later_t( \tilde{b_1} \mathbin{d \relcomp d'} \tilde{b_3} ) }
    {\theta_{B_1}(\tilde{b_1}) \mathbin{d \relcomp d'} \theta_{B_3}(\tilde{b_3}) }
\end{mathpar}
%
% We note that this composition has the following universal property.
%



We now describe the squares. Suppose we are given predomains $A_i, A_o, A_i'$,
and $A_o'$, relations $c_i : A_i \rel A_i'$ and $c_o : A_o \rel A_o'$, and
morphisms $f : A_i \to A_o, f' : A_i' \to A_o'$ Given a square with these
morphisms and relations, we say that the square commutes, written $f \le f'$, if
for all $x \in A_i$ and $x' \in A_i'$ with $(x, x') \in c_i$, we have $(f(x),
f'(x')) \in c_o$. We make the analogous definition for error domains.

\end{comment}

\subsubsection{Guarded Lift Monad and Free-Forgetful Adjunction}\label{sec:guarded-lift-monad}

% Lift monad
The functor $F$ takes a predomain $A$ to the free error domain on $A$. We first
define the predomain $UFA$. The underlying set of $UFA$ is defined to be $\li
|A|$ (the guarded lift monad applied to $|A|$). The ordering relation is the
lock-step error ordering introduced in the previous section (taking $A = A'$ in
the definition), and the bisimilarity relation is the weak bisimilarity relation
on $\li |A|$ also defined in the previous section. With these relations, the
constructors $\eta$ and $\theta$ of the guarded lift monad are in fact morphisms
of predomains, i.e., they are monotone and preserve bisimilarity.

We observe that this predomain structure extends to an error domain structure by
noting that the required error element is given by the constructor $\mho$ and
the required $\theta$ map is the constructor $\theta$. Lastly, it can be shown
that the delay morphism $\delta = \theta \circ \nxt$ satisfies $x \bisim
\delta\, x$ for all $x$ as is required by the definition of error domain.

The monadic bind operation takes a predomain morphism $f : A \to UB$ to an error
domain morphism $FA \to B$; with this we define the action of $F$ on morphisms:
Given $f : A \to A'$ we observe that $\eta \circ f : A \to UFA'$ and so we define
$Ff = \ext{\eta \circ f}{} : FA \to FA'$.

% The monad $\li$ decomposes into a free-forgetful adjunction $F \dashv U$. Given
% a predomain $A$, we define the error domain $FA$ whose underlying predomain is
% $\li A$, whose error element is $\mho$, and whose $\theta$ map is the
% constructor $\theta$. 


% It is easily verified that $\li A$ is the free error- and later-algebra on the
% predomain $A$, so we have that $\li$ is left-adjoint to $U$.
 

% i.e., error domain morphisms from $\li A$ to $B$ are in one-to-one correspondence with
% predomain morphisms from $A$ to $UB$.

% We define $\delta : U(\li A) \To U(\li A)$ by $\delta(x) = \theta(\nxt x)$.
% We define $\hat{\delta} : \li A \arr \li A$ by $\hat{\delta} = \ext{(\delta \circ \eta)}{}$.
% Note that by definition of $\text{ext}$, we have that $\hat{\delta}$ is a morphism of error domains.

%\subsubsection{Lock-Step Error Ordering}\label{sec:lock-step}


% We define the action of $\li$ on a relation $R$ between $A$ and $A'$ to be the
% ``heterogeneous" version of the lock-step error ordering.

% TODO action of \li on commuting squares

\subsubsection{Functors}

The functors $F$, $U$, $\times$, and $\arr$ act on morphisms, relations, and
squares. Most of these actions have already been described in the previous
section; the only difference is the added presence of the bisimilarity relation.
For example, the functor $U$ from error domains to predomains simply returns the
underlying predomain. On morphisms, it returns the underlying morphism of
predomains, and on relations and squares it returns the underlying predomain
relation and predomain square respectively. We define the action of $F$ on a
relation $c$ to be the lock-step error ordering between $UFA$ and $UFA'$ and
that it by definition satisfies the additional requirements of being an error
domain relation between the error domains $FA$ and $FA'$.

\begin{comment}

% internal hom for predomains and error domains
Given predomains $A$ and $A'$, we can form the predomain of
predomain morphisms from $A$ to $A'$, denoted $A \To A'$.
\begin{itemize}
    % Should we give the definition involving x and x'?
    \item The ordering is defined by $f \le_{A \To A'} f'$ iff for all
    $x \in A$, we have $f(x) \le_{A'} f'(x)$.
    \item The bisimilarity relation is defined by $f \bisim_{A \To A'} f'$ iff
    for all $x, x' \in A$ with $x \bisim_{A} x'$, we have $f(x) \bisim_{A'} f'(x')$. 
\end{itemize}
%
Given $f : A_1' \to A_1$ and $g : A_2 \to A_2'$ we define the predomain morphism
$f \To g : (A_1 \To A_2) \to (A_1' \To A_2')$ by $\lambda h. \lambda x'. g(h(f(x')))$.

% TODO: include this?
% The monadic extension operation $\ext{\cdot}{} : (A_1 \To U (\li A_2)) \To (\li A_1 \To U(\li A_2))$
% is a morphism of predomains from $A_1 \To U(\li A_2)$ to $U(\li A_1) \To U(\li A_2)$, i.e.,
% it preserves the ordering and bisimilarity relations.


% Given a predomain $A$ and error domain $B$, we define $A \arr B := A \To UB$.
We note that $A \To UB$ carries a natural error domain structure
(in the below, the lambda is a meta-theoretic notation):
\begin{itemize}
    \item The error is given by $\lambda x . \mho_B$
    \item The $\theta$ operation is defined by
      \[ \theta_{A \To UB}(\tilde{f}) = \lambda x . \theta_B(\lambda t . \tilde{f}_t(x)). \]
\end{itemize}

Given a predomain $A$ and error domain $B$, we define
$A \arr B$ to be the error domain such that $U(A \arr B) = A \To UB$,
and whose error and $\theta$ operations are as defined above.
We can define the functorial action of $\arr$ on morphisms
$f \arr \phi$ in the obvious way.
%
It is easily verified that $A \arr B$ is an exponential of $UB$ by $A$
in the category of predomains and their morphisms.

\end{comment}

\begin{comment}
Lastly, given a relation of predomains $c$ between $A$ and $A'$, and a relation
of error domains $d$ between $B$ and $B'$, we define the relation $c \arr d$
between $A \arr B$ and $A' \arr B'$ in the obvious way, i.e., $f \in A \arr B$
is related to $g \in A' \arr B'$ iff for all $x \in A$ and $x' \in A'$ with
$(x, x') \in c$, we have $(f(x), g(x')) \in Ud$.
%
One can verify that this relation is indeed a relation of error domains
in that it respects error and preserves $\theta$.
\end{comment}

% With all of the above data, we can form a step-1 intensional model of gradual typing
% (See Definition \ref{def:step-1-model}).


\subsection{Phase 2: Perturbations and Quasi-Representable Relations}

We now introduce the additional definitions needed to complete the construction
of our model. Recall from Section \ref{sec:adjusting-cast-rules} that in order
to establish the DnL rule for the downcast of $\iarr$, we needed to adjust the
rule by inserting a ``delay'' on the right-hand side. A similar adjustment is
needed for the DnR rule for $\iarr$. Moreover, the need to insert delays impacts
the semantics of the cast rules for \emph{all} relations, because of the
functorial nature of casts. That is, the upcast at a function type $c_i \ra c_o$
involves a downcast in the domain and an upcast in the codomain. This has two
consequences: first, the squares corresponding to the rules UpL and UpR may
also require the insertion of a delay. Second, we need to be able to insert
``higher-order'' delays in a way that follows the structures of the casts.


To accomplish this, we equip every predomain $A$ with a monoid $M_A$ of
\emph{syntactic perturbations}, as well as a means of \emph{interpreting} these
perturbations as actual endomorphisms on $A$. Moreover, we want the resulting
endomorphisms to be extensionally equivalent to the identity function, a
condition we express by saying that they are weakly bisimilar to the identity
morphism on $A$. More formally, for every $A$ we require a homomorphism of
monoids $i_A : M_A \to \{ f : A \to A \mid f \bisim \id_A \}$. We likewise equip
every error domain $B$ with a monoid $M_B$ and interpretation homomorphism
$i_B$. These concepts deserve their own definitions: We call the triple $(A,
M_A, iA)$ a \textbf{value object} as it will be the denotation of value types in
our final model. Likewise, the triple $(B, M_B, i_B)$ will be called a
\textbf{computation object}.

We must now impose an additional condition on predomain relations $c$ and error
domain relations $d$ that specify how they interact with these perturbations. In
particular, we need to be able to ``push'' and ``pull'' perturbations along
relations $c$ and $d$. The intuition for this requirement comes from the
construction of the square corresponding to the original UpL rule from the
square for the simplified version as was shown in Section
\ref{sec:modeling-term-precision}. Recall that we horizontally composed the
simplified UpL for $c$ square with the identity square for $c'$ on the right.
Now that the square for UpL involves a delay, this construction doesn't work
unless we can ``push'' the delay on $A_2$ to a delay on $A_3$. That is, given a
relation $c : A \rel A'$ and a perturbation $m_A \in M_A$, we need to be able to
turn it into a perturbation $m_{A'} \in M_{A'}$ such that the resulting
endomorphisms obtained by the respective homomorphisms $i_A$ and $i_{A'}$ form a
square. This is made formal by the following definition:
%
\begin{definition}
    Let $(A, M_A, i_A)$ and $(A', M_{A'}, and i_{A'})$ be value objects, and let
    $c : A \rel A'$ be a predomain relation. A \emph{push-pull structure} for
    $c$ consists of a homomorphism $\push : M_A \to M_A'$ and $\pull : M_A' \to
    M_A$ such that for all $m_A \in M_A$ there is a square $i_A(m_A) \ltsq{c}{c}
    i_{A'}(\push\, m_A)$ and for all $m_{A'} \in M_{A'}$ there is a square
    $i_A(\pull\, m_{A'}) \ltsq{c}{c} i_{A'}(m_{A'})$.
\end{definition}
%
We make the analogous definition for computation objects.

With the notion of perturbation defined, we can now discuss the final aspect of
the model, namely the representability of relations.
%
\begin{definition}[quasi-left-representable relations]
Let $(A, M_A, i_A)$ and $(A', M_{A'}, i_{A'})$ be value objects, and let $c : A
\rel A'$ be a predomain relation. We say that $c$ is
\emph{quasi-left-representable} by a predomain morphism $e : A \to A'$ if there
are perturbations $\delle_c \in M_A$ and $\delre_c \in M_{A'}$ such that the
following two squares exist: (UpL) $e \ltsq{c}{r(A')} i_{A'}(\delre_c)$, and
(UpR) $i_A(\delle_c) \ltsq{r(A)}{c} e$.
\end{definition}
%
Observe that this generalizes the previous notion of representability, since
under that definition the perturbations were always the identity.
%
We make the analogous definition of quasi-left-representability for error domain
relations $d$.
%
Likewise, we define quasi-right-representability as follows:
%
\begin{definition}
Let $(B, M_B, i_B)$ and $(B', M_{B'}, i_{B'})$ be computation objects, and let
$d : B \rel B'$ be an error domain relation. We say that $d$ is
\emph{quasi-right-representable} by an error domain morphism $p : B' \to B$ if
there are perturbations $\dellp_d \in M_B$ and $\delrp_d \in M_{B'}$ such that
the following two squares exist: 
(DnL) $p \ltsq{r(B')}{d} i_{B'}(\delrp_d)$ and
(DnR) $i_B(\dellp_d) \ltsq{d}{r(B)} p$.
\end{definition}
%
We make the analogous definition of quasi-right-representability for predomain
relations $c$.

Although it makes sense to refer to the quasi-representability of an arbitrary
predomain or error domain relation, quasi-representable relations do not in
general compose. In order to construct the needed squares for the composition of
two relations, we require that the relations have push-pull structures.

We can now give the final definition of value and computation relations for our
model:
%
\begin{definition}[value relations]
A \textbf{value relation} between value objects $(A, M_A, i_A)$ and $(A', M_{A'},
i_{A'})$ is a predomain relation $c : A \rel A'$ that has a push-pull structure
is quasi-left-representable by a morphism $e_c : A \to A'$, and is such that
$Fc$ is quasi-right-representable by an error domain morphism $p_c : FA' \to
FA$.
\end{definition}

\begin{definition}[computation relations]
A \textbf{computation relation} between computation objects $(B, M_B, i_B)$ and $(B',
M_{B'}, i_{B'})$ is an error domain relation $d$ that has a push-pull structure,
is quasi-right-representable by a morphism $p_d : B' \to B$, and is such that
$Ud$ is quasi-left-representable by a morphism $e_d : UB \to UB'$
\end{definition}

Because we have augmented the definitions of object and relation for our model,
we need to specify the actions of the functors $F$, $U$, $\times$, and $\arr$ on
these objects and relations. For instance, the functor $F$ now acts on
\emph{value objects} $(A, M_A, i_A)$, which means we need to specify the monoid
$M_{FA}$ and interpretation $i_{FA}$ corresponding to the error domain $FA$. We
leave these constructions to the appendix, as they are technical and not crucial
to understanding our main result.




\subsection{The Dynamic Type}
Lastly, we describe the denotation of the dynamic type. The predomain
representing the dynamic type is defined using guarded recursion as the solution
to the equation
% \footnote{In this section, we write $F$ instead of $\li$ so that the notation follows that of the abstract model section.}
%
\[ D \cong \mathbb{N}\, + (D \times D)\, + \laterhs U(D \arr FD). \]
%
% Note that the operators in the above equation are all combinators for predomains, so
% this also defines the ordering and the bisimilarity relation for $D$.

For the sake of clarity, we name the ``constructors" $\text{nat}$,
$\text{times}$, and $\text{fun}$, respectively.
%
We define $e_\mathbb{N} : \mathbb{N} \to D$ to be the $\text{nat}$ constructor,
$e_\times : D \times D \to D$ to be $\text{times}$, and $e_\to : U(D \arr F D)$
to be the morphism $\nxt$ followed by $\text{fun}$.

Explicitly, the ordering on $D$ is given by:
%
\begin{align*}
    \tnat(n) \le \tnat(n') 
        &\iff n = n' \\
    \ttimes (d_1, d_2) \le \ttimes (d_1', d_2')
        &\iff d_1 \le d_2 \text{ and } d_1' \le d_2'\\
    \tfun(\tilde{f}) \le \tfun(\tilde{f'}) 
        &\iff \later_t(\tilde{f}_t \le \tilde{f'}_t)
\end{align*}
%
We define a relation $\inat : \mathbb{N} \rel D$ by $(n, d) \in \inat$ iff
$e_\mathbb{N}(n) \le_D d$. We similarly define $\itimes : D \times D \rel D$ by
$((d_1, d_2), d) \in \itimes$ iff $e_\times(d_1, d_2) \le_D d$, and we define
$\text{inj}_\to : U(D \arr F D) \rel D$ by $(f, d) \in \iarr$ iff $e_\to(f)
\le_D d$.

Now we define the perturbations for $D$. Recall that for each value type $A$ we
associate a monoid $P_A$ of perturbations and a homomorphism into the monoid of
endomorphisms bisimilar to the identity, and likewise for computation types. We
define the perturbations for $D$ via least-fixpoint in the category of monoids
as
%
\( P_D \cong (P_{D \times D}) \times P_{U(D \to FD)}. \)
%
Unfolding the definitions, this is the same as
%
\( P_D \cong (P_D \times P_D) \times (\mathbb{N} \times P_D^{op} \times \mathbb{N} \times P_D). \)
%
We now explain how to interpret these perturbations as endomorphisms.
We define $\ptb_D : P_D \to \{ f : D \to D \mid f \bisim \id \}$ below,
% via the universal property of the coproduct of monoids, giving a case for each of the generators.
% In the below, note the use of the functorial action of $\arr$ on morphisms.
%
% \[ P_D \cong (P_D \times P_D) \times (P_D \times (\mathbb{N} \times P_D)), \]
%
\begin{align*}
 \ptb_D(p_{\text{times}}, p_{\text{fun}}) &= \lambda d.\text{case $d$ of}  \\
 &\alt \tnat(m) \mapsto \tnat(m) \\
    &\alt \ttimes(d_1, d_2) \mapsto {\ttimes(\ptb_{D \times D}(p_\text{times})(d_1, d_2))} \\
    &\alt \tfun(\tilde{f}) \mapsto {\tfun(\lambda t. \ptb_{U(D \to FD)}()(\tilde{f}_t))}
\end{align*}
    % \item $\ptb_D(1)$
    % \item $\ptb_D(\delta^K_D)$ is defined similarly to the previous but has
    % \[ \id \arr i^K(\delta^K_D) \] instead.
    % \item $\ptb_D(\delta^K_D) = \lambda d.\text{case $d$ of}$
    %   \begin{align*} 
    %     &\alt \ttimes(d_1, d_2) \mapsto {\ttimes(i^K(\delta^K_D)(d_1), d_2)} \\
    %     &\alt d' \to d'
    %   \end{align*}
    % \item $\ptb_D(\delta^K_D)$ is defined similarly to the previous but has 
    % \[ (d_1, \ptb_D(\delta^K_D)(d_2)) \] instead.
One can verify that this defines a homomorphism from $P_D \to \{ f : D \to D : f \bisim \id \}$.
We claim that the three relations $\inat$, $\itimes$, and $\iarr$
%and their lifted versions 
satisfy the push-pull property.
As an illustrative case, we establish the push-pull property for the relation $\iarr$.
We define $\pull_{\iarr} : P_D \to P_{U(D \arr FD)}$ by \( \pull_{\iarr}(p_{\text{times}}, p_{\text{fun}}) = p_{\text{fun}}, \)
% (recall that $P_{U(D \arr FD)} = \mathbb{N} \times P_D^{op} \times \mathbb{N} \times P_D$).
%
i.e., we simply forget the other perturbation.
%
We define $\push_{\iarr} : P_{U(D \arr FD)} \to P_D$ by \( \push_{\iarr}(p_{\text{fun}}) = (\id, p_{\text{fun}}) \).
%
Showing that the relevant squares commute is straightforward.
%
% To do so, consider an arbitrary perturbation $p$ on $D$. 
% Let $(f, d) \in \iarr$. This means that $d$ must be of the form $\tfun{\tilde{f}}$. 
% When we interpret the perturbation on $D$, obtaining and endomorphism to which we then apply $d$,
% we will be perform the action that was performed by the perturbation on the other side and thus
% we will be done by our assumption that $f$ is related to $d$.


We next claim that the relations $\inat$, $\itimes$, and $\iarr$ are quasi-left-representable,
and that their lifts are quasi-right-representable.
Indeed, since the relations are functional, it is easy to see that they are quasi-left-representable
where the perturbations are taken to be the identity.
%
For quasi-right-representability, the most interesting case is $\li(\iarr)$.
Defining the projection $p_{\iarr} : FD \to FU(D \to FD)$ is equivalent to defining
$p' : D \to UFU(D \to FD)$. We define
\begin{align*}
 p' = \lambda d.\text{case $d$ of}   &\alt \tnat(m) \mapsto \mho \\
    &\alt \ttimes(d_1, d_2) \mapsto \mho \\
    &\alt \tfun(\tilde{f}) \mapsto \theta (\lambda t. \eta(\tilde{f}_t)).
\end{align*}
%
We define $\dellp_D = \delrp_D = \delta_{FD}$.
Then it is easy to show using the definition of $\iarr$ that the squares for $\dnl$ and $\dnr$ commute.
% Then for $\dnr$, we need to show that if $(f, d) \in \iarr$ then
%
It is also straightforward to establish the retraction property for
each of these three relations. In the case of $\iarr$, we
have that the property holds up to a delay.

% \subsection{Obtaining an Extensional Model}

Now that we have defined an intensional model with an interpretation for the dynamic type, we can apply
the abstract constructions introduced in Section \ref{sec:extensional-model-construction}.
Doing so, we obtain an extensional model of gradual typing, where the squares are given by the
``bisimilarity closure'' of the intensional error ordering.

\subsection{Adequacy}\label{sec:adequacy}

In this section, we prove an adequacy result for the concrete extensional model of GTT we obtained in the previous section.
applying the abstract constructions introduced in Section
\ref{sec:extensional-model-construction} to the concrete model built in the previous section.
%\ref{sec:concrete-model}.

First we establish some notation. Fix a morphism $f : 1 \to \li \mathbb{N} \cong \li \mathbb{N}$.
We write $f \da n$ to mean that there exists $m$ such that $f = \delta^m(\eta n)$
and $f \da \mho$ to mean that there exists $m$ such that $f = \delta^m(\mho)$.

Recall that $\ltls$ denotes the relation on value morphisms defined as the bisimilarity-closure
of the intensional error-ordering on morphisms.
That is, we have $f \ltls g$ iff there exists $f'$ and $g'$ with
%
\[ f \bisim_{\li \mathbb{N}} f' \le_{\li \mathbb{N}} g' \bisim_{\li \mathbb{N}} g. \]
%
Here $\le_{\li \mathbb{N}}$ is the lock-step error ordering, and
$\bisim_{\li \mathbb{N}}$ is weak bisimilarity.
First observe that in this ordering, the semantics of error is not equivalent to
the semantics of the diverging term.
The main result we would like to show is as follows:
\begin{lemma}
If $f \ltls g : \li \mathbb{N}$, then:
\begin{itemize}
  \item If $f \da n$ then $g \da n$.
  \item If $g \da \mho$ then $f \da \mho$.
  \item If $g \da n$ then $f \da n$ or $f \da \mho$.
\end{itemize}
\end{lemma}
%
Unfortunately, this is actually not provable!
Roughly speaking, the issue is that this is a ``global'' result, and it is not possible
to prove such results inside of the guarded setting. 
In particular, if we tried to prove the above result in the guarded
setting, we would run into a problem where we would have a natural number
``stuck'' under a $\later$, with no way to get out the underlying number.
%
Thus, to prove our adequacy result, we need to leave the guarded setting and pass back
to the more familiar, set-theoretic world with no internal notion of step-indexing.
We can do this using a process known as \emph{clock quantification}.
Recall that all of the constructions we have made in SGDT take place in the context of a clock $k$.
All of our uses of the later modality and guarded recursion have taken place with respect to this clock.
For example, recall the definition of the lift monad by guarded recursion.
% We define the lift monad $\li^k X$ as the guarded fixpoint of $\lambda \tilde{T}. X + 1 + \later^k_t (\tilde{T}_t)$.
We can view this definition as being parameterized by a clock $k$: $\li^k : \type \to \type$.
Then for $X$ satisfying a certain technical requirement known as \emph{clock-irrelevance},
\footnote{A type $X$ is clock-irrelevant if there is an isomorphism $\forall k.X \cong X$.}
we can define the ``global lift'' monad as $\li^{gl} X := \forall k. \li^k X$.

It can be shown that there is an isomorphism between the global lift monad and the
delay monad of Capretta \cite{lmcs:2265}.
Recall that, given a type $X$, the delay monad $\text{Delay}(X)$ is defined as the coinductive
type generated by 
$\tnow : X \to \delay(X)$ and $\tlater : \delay(X) \to \delay(X)$.

% solution to the equation

% \[ \text{Delay}(X) \cong X + \text{Delay}(X). \]

It can be shown that for a clock-irrelevant type $X$, $\li^{gl} X$ is a final
coalgebra of the functor $F(Y) = X + 1 + Y$ (For example, this follows from Theorem 4.3 in
\cite{kristensen-mogelberg-vezzosi2022}.) 
\footnote{The proof relies on the existence of an operation  
$\mathsf{force} : \forall k. \later^k A \to \forall k. X$ that
allows us to eliminate the later operator under a clock quantifier.
This must be added as an axiom in guarded type theory.}
Since $\delay(X + 1)$ is also a final coalgebra
of this functor, then we have $\li^{gl} X \cong \delay(X + 1)$.

Given a predomain $X$ on a clock-irrelevant type, we can define a
``global'' version of the lock-step error ordering and the
weak bisimilarity relation on elements of the global lift; the former is defined by
%
\( x \le^{gl}_X y := \forall k. x[k] \le y[k], \)
%
and the latter is defined by
%
\( x \bisim^{gl}_X y := \forall k. x[k] \bisim y[k]. \)
%
On the other hand, we can define coinductively a ``lock-step error ordering"
relation on $\delay(X + 1)$:
%
\begin{mathpar}
  \inferrule*[]
  { }
  {\tnow (\inr\, 1) \ledelay d}

  \inferrule*[]
  {x_1 \le_X x_2}
  {\tnow (\inl\, x_1) \ledelay \tnow (\inl\, x_2)}

  \inferrule*[]
  {d_1 \ledelay d_2}
  {\tlater\, d_1 \ledelay \tlater\, d_2}
\end{mathpar}
%
And we similarly define by coinduction a ``weak bisimilarity'' relation on $\delay(X + 1)$, which uses
a relation $d \Da x_?$ between $\delay(X+1)$ and $X+1$ that is defined as 
$d \Da x_? := \Sigma_{m \in \mathbb{N}} d = \tlater^m(\tnow\, x_?)$.
Then weak bisimilarity is defined by the rules
%
\begin{mathpar}
  \inferrule*[]
  {x_? \bisim_{X + 1} y_?}
  {\tnow\, x_? \bisimdelay \tnow\, y_? }

  \inferrule*[]
  {d_1 \Da x_? \and x_? \bisim_{X + 1} y_?}
  {\tlater\, d_1 \bisimdelay \tnow\, y_? }

  \inferrule*[]
  {d_2 \Da y_? \and x_? \bisim_{X + 1} y_?}
  {\tnow\, x_? \bisimdelay \tlater\, d_2}

  \inferrule*[]
  {d_1 \bisimdelay d_2}
  {\tlater\, d_1 \bisimdelay \tlater\, d_2 }

  % \inferrule*[]
  % {d_1 \Da x_? \and d_2 \Da y_? \and x_? \bisim_{X + 1} y_?}
  % {d_1 \bisimdelay d_2}

  % \inferrule*[]
  % {d_1 \bisimdelay d_2}
  % {\tlater d_1 \bisimdelay \tlater d_2 }

\end{mathpar}
%
Note the similarity of these definitions to the corresponding guarded definitions.
By adapting the aforementioned theorem to the setting of inductively-defined relations,
we can show that both the global lock-step error ordering and the global weak bisimilarity
admit coinductive definitions. In particular, modulo the above isomorphism
between $\li^{gl} X$ and $\delay(X+1)$, the global version of the lock-step
error ordering is equivalent to the lock-step error ordering on $\delay(X + 1)$,
and likewise, the global version of the weak bisimilarity relation is equivalent to the
weak bisimilarity relation on $\delay(X + 1)$.

This implies that the global version of the extensional term precision semantics for
$\li^{gl} X$ agrees with the corresponding notion for $\delay(X + 1)$.
Then adequacy follows by proving the corresponding
result for $\delay(X + 1)$ which in turn follows from the definitions of the relations.


% We have been writing the type as $\li X$, but it is perhaps more accurate to write it as $\li^k X$ to
% emphasize that the construction is parameterized by a clock $k$.

% Need : nat is clock irrelevant, as well as the inputs and outputs of effects
% Axioms about forcing clock
% Adapt prior argument to get that the defining of the global bisim
% and global lock-step error ordering are coinductive

\section{Constructing a Concrete Model}

In this section, we build a concrete extensional model of gradual typing.
We begin by defining a step-1 intensional model, and then
apply the abstract constructions outlined in the previous section
to obtain an extensional model.

We begin with some definitions. These concepts were first introduced in the section
on the term semantics (Section \ref{sec:domain-theory}), but for completeness we repeat them here.
% for ease of reference when instantiating the definition of step-1 model.

\begin{definition}
A \textbf{predomain} $A$ consists of a set $A$ along with two relations:
\begin{itemize}
    \item A partial order $\le_A$.
    \item A reflexive, symmetric ``bisimilarity'' relation $\bisim_A$.
\end{itemize}
\end{definition}

% TODO: later of a predomain
Given a predomain $A$, we can form the predomain $\later A$.
The underlying set is $\later |A|$ and the relation is defined in the obvious way,
i.e., $\tilde{x} \le_{\later A} \tilde{x'}$ iff $\later_t(\tilde{x}_t \le_A \tilde{x'}_t)$.

Morphisms of predomains are functions between the underlying sets that preserve the ordering
and the bisimilarity relation. More formally:
%
\begin{definition}
Let $A$ and $A'$ be predomains.
A morphism $f : A \to A'$ is a function between the underlying sets such that for all $x, x'$,
if $x \le_A x'$, then $f(x) \le f(x')$, and if $x \bisim_A x'$, then $f(x) \bisim_{A'} f(x')$.
\end{definition}

\begin{definition}
An \textbf{error domain} $B$ consists of a predomain $B$ along with the following data:
\begin{itemize}
    \item A distinguished ``error" element $\mho_B \in B$
    \item A morphism of predomains $\theta_B \colon \later B \to B$
\end{itemize}
\end{definition}


Morphisms of error domains are morphisms of the underlying predomains that preserve the
algebraic structure. More formally:
%
\begin{definition}
Let $B$ and $B'$ be error domains.
A morphism $\phi : B \wand B'$ is a morphism between the underlying predomains such that:
\begin{enumerate}
    \item $\phi(\mho_B) = \mho_{B'}$
    \item $\phi(\theta_B(\tilde{x})) = \theta_{B'}(\lambda t. \phi(\tilde{x}_t))$
\end{enumerate}
\end{definition}


We define a (monotone) relation on predomains $A$ and $A'$ to be a relation on the
underlying sets that is downward-closed under $\le_A$ and upward-closed under $\le_{A'}$.
More formally:

\begin{definition}
Let $A$ and $A'$ be predomains. A \emph{predomain relation} between $A$ and $A'$
is a relation $R$ between the underlying sets such that:
\begin{enumerate}
    \item (Downward closure): For all $x_1, x_2 \in A$ and $y \in A'$,
    if $x_1 \le_A x_2$ and $x_2 \mathbin{R} y$, then $x_1 \mathbin{R} y$.
    \item (Upward closure): For all $x \in A$ and $y_1, y_2 \in A'$,
    if $x \mathbin{R} y_1$ and $y_1 \le_{A'} y_2$, then $x \mathbin{R} y_2$.
\end{enumerate}
\end{definition}

Composition of relations on predomains is the usual relational composition.

Similarly, we define a (monotone) relation on error domains to be a relation on the
underlying predomains that respects error and preserves $\theta$.
%
\begin{definition}
    Let $B$ and $B'$ be error domains. An \emph{error domain relation} between
    $B$ and $B'$ is a relation $R$ between the underlying predomains such that
    \begin{enumerate}
       \item (Respects error): For all $y \in B'$, we have $\mho_B \mathbin{R} y$.
       \item (Preserves $\theta$): For all $\tilde{x}$ in $\later B$ and $\tilde{y} \in \later B'$,
       if 
       \[ \later_t( \tilde{x}_t \mathbin{R} \tilde{y}_t ), \]
       then
       \[ \theta_B(\tilde{x}) \mathbin{R} \theta_{B'}(\tilde{y}). \]
    \end{enumerate}
\end{definition}


We define composition of error domain relations $R$ on $B_1$ and $B_2$ and $S$
on $B_2$ and $B_3$ to be the least relation containing $R$ and $S$ that respects
error and preserves $\theta$.
Specifically, it is defined inductively by the following rules:

\begin{mathpar}
    \inferrule*[right = Comp]
    {b_1 \mathbin{R} b_2 \and b_2 \mathbin{S} b_3}
    {b_1 \mathbin{R \relcomp S} b_3}

    \inferrule*[right = PresErr]
    { }
    {\mho_{B_1} \mathbin{R \relcomp S} b_3}

    \inferrule*[right = PresTheta]
    {\later_t( \tilde{b_1} \mathbin{R \relcomp S} \tilde{b_3} ) }
    {\theta_{B_1}(\tilde{b_1}) \mathbin{R \relcomp S} \theta_{B_3}(\tilde{b_3}) }
\end{mathpar}

% We note that this composition has the following universal property.


We now describe the ``commuting squares".
Suppose we are given predomains $A_i, A_o, A_i'$, and $A_o'$,
relations $R_i$ and $R_o$, and morphisms $f, f'$ as shown below.

% https://q.uiver.app/#q=WzAsNCxbMCwwLCJBX2kiXSxbMSwwLCJBX2knIl0sWzAsMSwiQV9vIl0sWzEsMSwiQV9vJyJdLFswLDIsImYiLDJdLFsxLDMsImYnIl0sWzAsMSwiUl9pIiwwLHsic3R5bGUiOnsiYm9keSI6eyJuYW1lIjoiYmFycmVkIn0sImhlYWQiOnsibmFtZSI6Im5vbmUifX19XSxbMiwzLCJSX28iLDIseyJzdHlsZSI6eyJib2R5Ijp7Im5hbWUiOiJiYXJyZWQifSwiaGVhZCI6eyJuYW1lIjoibm9uZSJ9fX1dLFs0LDUsIlxcbHRkeW4iLDEseyJzaG9ydGVuIjp7InNvdXJjZSI6MjAsInRhcmdldCI6MjB9LCJzdHlsZSI6eyJib2R5Ijp7Im5hbWUiOiJub25lIn0sImhlYWQiOnsibmFtZSI6Im5vbmUifX19XV0=
\[\begin{tikzcd}[ampersand replacement=\&]
	{A_i} \& {A_i'} \\
	{A_o} \& {A_o'}
	\arrow[""{name=0, anchor=center, inner sep=0}, "f"', from=1-1, to=2-1]
	\arrow[""{name=1, anchor=center, inner sep=0}, "{f'}", from=1-2, to=2-2]
	\arrow["{R_i}", "\shortmid"{marking}, no head, from=1-1, to=1-2]
	\arrow["{R_o}"', "\shortmid"{marking}, no head, from=2-1, to=2-2]
	\arrow["\ltdyn"{description}, draw=none, from=0, to=1]
\end{tikzcd}\]

We say that the above square commutes, written $f \le f'$, if for all
$x \in A_i$ and $x' \in A_i'$ with $x \mathbin{R_i} x'$, we have
$f(x) \mathbin{R_o} f'(x')$.

We make the analogous definition for error domains.



% Lift monad
The guarded lift monad $\li$ takes a predomain $A$ to the error domain $\li A$;
the precise definition is given in Section \ref{sec:lift-monad}.
The partial order $\le_{\li A}$ is the lock-step error ordering defined in Section
\ref{sec:lock-step}, and the bisimilarity relation $\bisim_{\li A}$ is the weak bisimilarity
relation defined in Section \ref{sec:weak-bisimilarity}.
We define the action of $\li$ on a relation $R$ between $A$ and $A'$ to be the
``heterogeneous" version of the lock-step error lifting of $R$ (see Section \ref{sec:lift-monad}).

% TODO action of \li on commuting squares

There is a functor $U$ from error domains to predomains that on objects simply takes the
underlying predomain, and on morphisms takes the underlying morphism of predomains.
%
% TODO check this
One can verify that there is a free-forgetful adjunction between the functors $\li$ and $U$;
this holds because $\li A$ is the initial/free error- and later-algebra on the predomain $A$,
i.e., error domain morphisms from $\li A$ to $B$ are in one-to-one correspondence with
predomain morphisms from $A$ to $UB$.

% internal hom for predomains and error domains
Given predomains $A$ and $A'$, we can form the predomain of
predomain morphisms from $A$ to $A'$, denoted $A \To A'$.
\begin{itemize}
    % Should we give the definition involving x and x'?
    \item The ordering is defined by $f \le_{A \To A'} f'$ iff for all
    $x \in A$, we have $f(x) \le_{A'} f'(x)$.
    \item The bisimilarity relation is defined by $f \bisim_{A \To A'} f'$ iff
    for all $x, x' \in A$ with $x \bisim_{A} x'$, we have $f(x) \bisim_{A'} f'(x')$. 
\end{itemize}

Given a predomain $A$ and error domain $B$, we define $A \arr B := A \To UB$.
We note that $A \arr B$ carries a natural error domain structure
(in the below, the lambda is a meta-theoretic notation):
\begin{itemize}
    \item The error is given by $\lambda x . \mho_B$
    \item The $\theta$ operation is defined by
      \[ \theta_{A \arr B}(\tilde{f}) = \lambda x . \theta_B(\lambda t . \tilde{f}_t(x)). \]
\end{itemize}

It is easily verified that $A \arr B$ is an exponential of $UB$ by $A$
in the category of predomains and their morphisms.

Lastly, given a relation of predomains $R$ between $A$ and $A'$, and a relation
of error domains $S$ between $B$ and $B'$, we can form a relation $R \arr S$
between $A \arr B$ and $A' \arr B'$ in the obvious way, i.e., $f \in A \arr B$
is related to $g \in A' \arr B'$ iff for all $x \in A$ and $x' \in A'$ with
$x \mathrel{R} x'$, we have $f(x) \mathrel{S} g(x')$.


With all of the above data, we can form a step-1 intensional model of gradual typing
(See Definition \ref{def:step-1-model}).

\subsection{The Dynamic Type}


% dynamic type
The underlying set of the dynamic type will be defined using guarded recursion as the
solution to the equation

\[ D \cong \mathbb{N} + \later (D \to \li D). \]

The perturbation monoid for $D$ is defined inductively as the free monoid $M$ equipped
with an operation $\To : M \times M \to M$, i.e., we take the set defined inductively by

\begin{mathpar}
    \inferrule*[]
    {}
    {e \in M}

    \inferrule*[]
    {p \in M \and p' \in M}
    {p \odot p' \in M}

    \inferrule*[]
    {p \in M \and p' \in M}
    {p \To p' \in M}
\end{mathpar}

and quotient by the necessary monoid equations.

% perturbations and quasi-representability



\subsection{Obtaining an Extensional Model}

Now that we have defined an intensional model, we can apply
the abstract construction introduced in Section \ref{sec:extensional-model-construction}.
Doing so, we obtain an extensional model of gradual typing.

 

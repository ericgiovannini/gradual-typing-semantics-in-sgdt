\section{Syntactic Theory of Gradually Typed Lambda Calculus}\label{sec:GTLC}

Here we give an overview of a fairly standard cast calculus for
gradual typing along with its (in-)equational theory that capture our
desired notion of type-based reasoning and graduality. The main
departure from prior work is our explicit treatment of type precision
derivations and an equational theory of those derivations.

\max{TODO: ordinary cbv syntax}

\begin{figure}
  \begin{mathpar}
    \begin{array}{rcl}
    \text{Types } A &::=& \nat \alt \,\dyn \alt A \ra A' \alt A \times A'\\
    \text{Type Precision } c &::=& r(A) \alt \iarr \alt \inat \alt \itimes \alt c \ra c' \alt c \times c'\\
    \text{Terms } M,N &::=& \err\alt \upc c M \alt \dnc c M \alt \zro \alt \suc\, M \alt \lda{x}{M} \\ 
     &&\alt M\, N \alt (M,N) \alt \textrm{let } (x,y) = M \textrm{ in } N\\
    \text{Contexts } \Gamma &::= &\cdot \alt \Gamma, x : A \\
    \text{Ctx Precision } \Delta &::=& \cdot\alt \Delta,x:c
  \end{array}
  \end{mathpar}
  \caption{GTLC Cast Calculus Syntax}
\end{figure}

The graduality theorem is formulated in terms of \emph{type} and
\emph{term precision}, usually presented as binary relations on types
and terms. Here we will, like \citet{who?} use an explicit term
assignment for type precision.

\max{TODO: prose about type precision derivations and equivalence
  of type precision derivations}
\begin{figure}
  \begin{mathpar}
    \inferrule{}{r(A) : A \ltdyn A}\and
    \inferrule{c : A \ltdyn A' \and c' : A' \ltdyn A''}{cc' : A \ltdyn A''}\and
    \inferrule{}{\iarr : D \ra D \ltdyn D}\and
    \inferrule{}{\inat : \nat \ltdyn D}\and
    \inferrule{}{\iarr : D \times D \ltdyn D}\and
    \inferrule{c_i : A_i \ltdyn A_i' \and c_o : A_o \ltdyn A_o'}{c_i \ra c_o : (A_i \ra A_o) \ltdyn (A_i' \ra A_o')}\and
    \inferrule{c_1 : A_1 \ltdyn A_1' \and c_2 : A_2 \ltdyn A_2'}{c_1 \times c_2 : (A_1 \times A_2) \ltdyn (A_1' \times A_2')}\and
     r(A)c \equiv c\and
     c \equiv cr(A')\and
     c(c'c'') \equiv (cc')c''\and
     r(A_i \ra A_o) \equiv r(A_i) \ra r(A_o)\and
     r(A_1\times A_2) \equiv r(A_1) \times r(A_2)\and
     (c_i \ra c_o)(c_i' \ra c_o')\equiv (c_ic_i' \ra c_oc_o') \and
     (c_1\times c_2)(c_1'\times c_2')\equiv (c_1c_1' \times c_2c_2')
  \end{mathpar}
  \caption{Type Precision Derivations and equational theory}
\end{figure}

In addition to congruence rules and CBV $\beta\eta$ equality, we have
the following rules.\max{TODO: more about beta-eta}
\begin{figure}
  \begin{mathpar}
  \inferrule
  {M \ltdyn M' : c \and c \equiv c'}
  {M \ltdyn M' : c'}

  \inferrule
  {}
  {\dnc {c} \upc {c} M \equidyn M}

  \inferrule
  {}
  {\upc{c}\upc{c'}M \equidyn \upc{cc'}M}

  \inferrule
  {}
  {\dnc{c'}\dnc{c}M \equidyn \dnc{cc'}M}

  \inferrule
  {M \ltdyn M' : cc_r}
  {\upc {c} M \ltdyn M' : c_r}

  \inferrule
  {M \ltdyn M' : c_l}
  {M \ltdyn \upc {c} M' : c_lc}

  \inferrule
  {M \ltdyn M' : c_r}
  {\dnc {c} M \ltdyn M' : cc_r}

  \inferrule
  {M \ltdyn M' : c_lc}
  {M \ltdyn \dnc {c} M' : c_l}
  \end{mathpar}
  \caption{Term Precision Rules (Selected)}
\end{figure}

%% Here we describe the syntax and typing for the gradually-typed lambda calculus.
%% We also give the rules for syntactic type and term precision.
%% % We define four separate calculi: the normal gradually-typed lambda calculus, which we
%% % call the extensional or \emph{step-insensitive} lambda calculus ($\extlc$),
%% % as well as an \emph{intensional} lambda calculus
%% % ($\intlc$) whose syntax makes explicit the steps taken by a program.

%% Before diving into the details, let us give a brief overview of what we will define.
%% We begin with a gradually-typed lambda calculus $(\extlc)$, which is similar to
%% the normal call-by-value gradually-typed lambda calculus, but differs in that it
%% is actually a fragment of call-by-push-value specialized such that there are no
%% non-trivial computation types. We do this for convenience, as either way
%% we would need a distinction between values and effectful terms; the framework of
%% of call-by-push-value gives us a convenient language to define what we need.

%% We then show that composition of type precision derivations is admissible, as is
%% heterogeneous transitivity for term precision, so it will suffice to consider a new
%% language ($\extlcm$) in which we don't have composition of type precision derivations
%% or heterogeneous transitivity of term precision.

%% We then observe that all casts, except those between $\nat$ and $\dyn$
%% and between $\dyn \ra \dyn$ and $\dyn$, are admissible.
%% % (we can define the cast of a function type functorially using the casts for its domain and codomain).
%% This means it will be sufficient to consider a new language ($\extlcmm$) in which
%% instead of having arbitrary casts, we have injections from $\nat$ and
%% $\dyn \ra \dyn$ into $\dyn$, and case inspections from $\dyn$ to $\nat$ and
%% $\dyn$ to $\dyn \ra \dyn$.

%% From here, we define a \emph{step-sensitive} (also called \emph{intensional}) GSTLC,
%% so-named because it makes the intensional stepping behavior of programs explicit in the syntax.
%% This is accomplished by adding a syntactic ``later'' type and a
%% syntactic $\theta$ that maps terms of type later $A$ to terms of type $A$.
%% Finally, we define a \emph{quotiented} version of the step-sensitive language where
%% we add a rule that equates terms that are the same up to their stepping behavior.

%% % ---------------------------------------------------------------------------------------
%% % ---------------------------------------------------------------------------------------

%% \subsection{Syntax}

%% The language is based on Call-By-Push-Value \cite{levy01:phd}, and as such it has two kinds of types:
%% \emph{value types}, representing pure values, and \emph{computation types}, representing
%% potentially effectful computations.
%% In the language, all computation types have the form $\Ret A$ for some value type $A$.
%% Given a value $V$ of type $A$, the term $\ret V$ views $V$ as a term of computation type $\Ret A$.
%% Given a term $M$ of computation type $B$, the term $\bind{x}{M}{N}$ should be thought of as
%% running $M$ to a value $V$ and then continuing as $N$, with $V$ in place of $x$.


%% We also have value contexts and computation contexts, where the latter can be viewed
%% as a pair consisting of (1) a stoup $\Sigma$, which is either empty or a hole of type $B$,
%% and (2) a (potentially empty) value context $\Gamma$.

%% \begin{align*} % TODO is hole a term?
%%   &\text{Value Types } A := \nat \alt \,\dyn \alt (A \ra A') \\
%%   &\text{Computation Types } B := \Ret A \\
%%   &\text{Value Contexts } \Gamma := \cdot \alt (\Gamma, x : A) \\
%%   &\text{Computation Contexts } \Delta := \cdot \alt \hole B \alt \Delta , x : A \\
%%   &\text{Values } V :=  \zro \alt \suc\, V \alt \lda{x}{M} \alt \up{A}{B} V \\ 
%%   &\text{Terms } M, N := \err_B \alt \matchnat {V} {M} {n} {M'} \\ 
%%   &\quad\quad \alt \ret {V} \alt \bind{x}{M}{N} \alt V_f\, V_x \alt \dn{A}{B} M 
%% \end{align*}

%% The value typing judgment is written $\hasty{\Gamma}{V}{A}$ and 
%% the computation typing judgment is written $\hasty{\Delta}{M}{B}$.

%% \begin{comment}
%% We define substitution for value contexts by the following rules:

%% \begin{mathpar}
%%   \inferrule*
%%   { \gamma : \Gamma' \to \Gamma \and 
%%     \hasty{\Gamma'}{V}{A}}
%%   { (\gamma , V/x ) \colon \Gamma' \to \Gamma , x : A }

%%   \inferrule*
%%   {}
%%   {\cdot \colon \cdot \to \cdot}
%% \end{mathpar}

%% We define substitution for computation contexts by the following rules:

%% \begin{mathpar}
%%     \inferrule*
%%     { \delta : \Delta' \to \Delta \and 
%%       \hasty{\Delta'|_V}{V}{A}}
%%     { (\delta , V/x ) \colon \Delta' \to \Delta , x : A }

%%     \inferrule*
%%     {}
%%     {\cdot \colon \cdot \to \cdot}

%%     \inferrule*
%%     {\hasty{\Delta'}{M}{B}}
%%     {M \colon \Delta' \to \hole{B}}
%% \end{mathpar}
%% \end{comment}

%% The typing rules are as expected, with a cast between $A$ to $B$ allowed only when $A \ltdyn B$.
%% Notice that the upcast of a value is a value, since it always succeeds, while the downcast
%% of a value is a computation, since it may fail.

%% \begin{mathpar}
%%     % Var
%%     \inferrule*{ }{\hasty {\cdot, \Gamma, x : A, \Gamma'} x A}

%%     % Err
%%     \inferrule*{ }{\hasty {\cdot, \Gamma} {\err_B} B} 
  
%%     % Zero and suc
%%     \inferrule*{ }{\hasty \Gamma \zro \nat}
  
%%     \inferrule*{\hasty \Gamma V \nat} {\hasty \Gamma {\suc\, V} \nat}

%%     % Match-nat
%%     \inferrule*
%%     {\hasty \Gamma V \nat \and 
%%      \hasty \Delta M B \and \hasty {\Delta, n : \nat} {M'} B}
%%     {\hasty \Delta {\matchnat {V} {M} {n} {M'}} B}
  
%%     % Lambda
%%     \inferrule* 
%%     {\hasty {\cdot, \Gamma, x : A} M {\Ret A'}} 
%%     {\hasty \Gamma {\lda x M} {A \ra A'}}
  
%%     % App
%%     \inferrule*
%%     {\hasty \Gamma {V_f} {A \ra A'} \and \hasty \Gamma {V_x} A}
%%     {\hasty {\cdot , \Gamma} {V_f \, V_x} {\Ret A'}}

%%     % Ret
%%     \inferrule*
%%     {\hasty \Gamma V A}
%%     {\hasty {\cdot , \Gamma} {\ret\, V} {\Ret A}}
%%     % TODO should this involve a Delta?

%%     % Bind
%%     \inferrule*
%%     {\hasty \Delta M {\Ret A} \and \hasty{\cdot , \Delta|_V , x : A}{N}{B} } % Need x : A in context
%%     {\hasty {\Delta} {\bind{x}{M}{N}} {B}}

%%     % Upcast
%%     \inferrule*
%%     {A \ltdyn A' \and \hasty \Gamma V A}
%%     {\hasty \Gamma {\up A {A'} V} {A'} }

%%     % Downcast
%%     % \inferrule*
%%     % {A \ltdyn A' \and \hasty {\Gamma} V {A'}}
%%     % {\hasty {\cdot, \Gamma} {\dn A {A'} V} {\Ret A}}

%%     \inferrule* % TODO is this correct?
%%     {B \ltdyn B' \and \hasty {\Delta} {M} {B'}}
%%     {\hasty {\Delta} {\dn B {B'} M} {B}}

%% \end{mathpar}


%% In the equational theory, we have $\beta$ and $\eta$ laws for function type,
%% as well a $\beta$ and $\eta$ law for $\Ret A$.

%% % TODO do we need to add a substitution rule here?
%% \begin{mathpar}
%%   % Function Beta and Eta
%%   \inferrule*
%%   {\hasty {\cdot, \Gamma, x : A} M {\Ret A'} \and \hasty \Gamma V A}
%%   {(\lda x M)\, V = M[V/x]}

%%   \inferrule*
%%   {\hasty \Gamma V {A \ra A}}
%%   {\Gamma \vdash V = \lda x {V\, x}}

%%   % Ret Beta and Eta
%%   \inferrule*
%%   {}
%%   {(\bind{x}{\ret\, V}{N}) = N[V/x]}

%%   \inferrule*
%%   {\hasty {\hole{\Ret A} , \Gamma} {M} {B}}
%%   {\hole{\Ret A}, \Gamma \vdash M = (\bind{x}{\bullet}{M[\ret\, x]})}

%%   % Match-nat Beta
%%   \inferrule*
%%   {\hasty \Delta M B \and \hasty {\Delta, n : \nat} {M'} B}
%%   {\matchnat{\zro}{M}{n}{M'} = M}

%%   \inferrule*
%%   {\hasty \Gamma V \nat \and 
%%    \hasty \Delta M B \and \hasty {\Delta, n : \nat} {M'} B}
%%   {\matchnat{\suc\, V}{M}{n}{M'} = M'}

%%   % Match-nat Eta
%%   % This doesn't build in substitution
%%   \inferrule*
%%   {\hasty {\Delta , x : \nat} M A}
%%   {M = \matchnat{x} {M[\zro / x]} {n} {M[(\suc\, n) / x]}}



%% \end{mathpar}

%% % ---------------------------------------------------------------------------------------
%% % ---------------------------------------------------------------------------------------

%% \subsection{Type Precision}

%% The type precision rules specify what it means for a type $A$ to be more precise than $A'$.
%% We have reflexivity rules for $\dyn$ and $\nat$, as well as rules that $\nat$ is more precise than $\dyn$
%% and $\dyn \ra \dyn$ is more precise than $\dyn$.
%% We also have a transitivity rule for composition of type precision,
%% and also a rule for function types stating that given $A_i \ltdyn A'_i$ and $A_o \ltdyn A'_o$, we can prove
%% $A_i \ra A_o \ltdyn A'_i \ra A'_o$.
%% Finally, we can lift a relation on value types $A \ltdyn A'$ to a relation $\Ret A \ltdyn \Ret A'$ on
%% computation types.

%% \begin{mathpar}
%%   \inferrule*[right = \dyn]
%%     { }{\dyn \ltdyn\, \dyn}

%%   \inferrule*[right = \nat]
%%     { }{\nat \ltdyn \nat}

%%   \inferrule*[right = $\ra$]
%%     {A_i \ltdyn A'_i \and A_o \ltdyn A'_o }
%%     {(A_i \ra A_o) \ltdyn (A'_i \ra A'_o)}

%%   \inferrule*[right = $\textsf{Inj}_\nat$]
%%     { }{\nat \ltdyn\, \dyn}

%%   \inferrule*[right=$\textsf{Inj}_{\ra}$]
%%     { }
%%     {(\dyn \ra \dyn) \ltdyn\, \dyn}

%%   \inferrule*[right=ValTrans]
%%     {A \ltdyn A' \and A' \ltdyn A''}
%%     {A \ltdyn A''}

%%   \inferrule*[right=CompTrans]
%%     {B \ltdyn B' \and B' \ltdyn B''}
%%     {B \ltdyn B''}

%%   \inferrule*[right=$\Ret{}$]
%%     {A \ltdyn A'}
%%     {\Ret {A} \ltdyn \Ret {A'}}

%%     % TODO are there other rules needed for computation types?

  
%% \end{mathpar}

%% % Type precision derivations
%% Note that as a consequence of this presentation of the type precision rules, we
%% have that if $A \ltdyn A'$, there is a unique precision derivation that witnesses this.
%% As in previous work, we go a step farther and make these derivations first-class objects,
%% known as \emph{type precision derivations}.
%% Specifically, for every $A \ltdyn A'$, we have a derivation $c : A \ltdyn A'$ that is constructed
%% using the rules above. For instance, there is a derivation $\dyn : \dyn \ltdyn \dyn$, and a derivation
%% $\nat : \nat \ltdyn \nat$, and if $c_i : A_i \ltdyn A_i$ and $c_o : A_o \ltdyn A'_o$, then
%% there is a derivation $c_i \ra c_o : (A_i \ra A_o) \ltdyn (A'_i \ra A'_o)$. Likewise for
%% the remaining rules. The benefit to making these derivations explicit in the syntax is that we
%% can perform induction over them.
%% Note also that for any type $A$, we use $A$ to denote the reflexivity derivation that $A \ltdyn A$,
%% i.e., $A : A \ltdyn A$.
%% Finally, observe that for type precision derivations $c : A \ltdyn A'$ and $c' : A' \ltdyn A''$, we
%% can define (via the rule ValComp) their composition $c \relcomp c' : A \ltdyn A''$.
%% The same holds for computation type precision derivations.
%% This notion will be used below in the statement of transitivity of the term precision relation.

%% % ---------------------------------------------------------------------------------------
%% % ---------------------------------------------------------------------------------------

%% \subsection{Term Precision}

%% We allow for a \emph{heterogeneous} term precision judgment on terms values $V$ of type
%% $A$ and $V'$ of type $A'$ provided that $A \ltdyn A'$ holds. Likewise, for computation
%% types $B \ltdyn B'$, if $M$ has type $B$ and $M'$ has type $B'$, we can form the judgment
%% that $M \ltdyn M'$.

%% % Type precision contexts
%% % TODO should we include the formal definitions of value and computation type precision contexts?
%% In order to deal with open terms, we will need the notion of a type precision \emph{context}, which we denote
%% $\gamlt$. This is similar to a normal context but instead of mapping variables to types,
%% it maps variables $x$ to related types $A \ltdyn A'$, where $x$ has type $A$ in the left-hand term
%% and $B$ in the right-hand term. We may also write $x : d$ where $d : A \ltdyn A'$ to indicate this.
%% Similarly, we have computation type precision contexts $\Delta^\ltdyn$. Similar to ``normal'' computation
%% type precision contexts $\Delta$, these consist of (1) a stoup $\Sigma$ which is either empty or
%% has a hole $\hole{d}$ for some computation type precision derivation $d$, and (2) a value type precision context
%% $\Gamma^\ltdyn$.

%% % An equivalent way of thinking of type precision contexts is as a pair of ``normal" typing
%% % contexts $\Gamma, \Gamma'$ with the same domain such that $\Gamma(x) \ltdyn \Gamma'(x)$ for
%% % each $x$ in the domain.
%% % We will write $\gamlt : \Gamma \ltdyn \Gamma'$ when we want to emphasize the pair of contexts.
%% % Conversely, if we are given $\gamlt$, we write $\gamlt_l$ and $\gamlt_r$ for the normal typing contexts on each side.

%% An equivalent way of thinking of a type precision context $\gamlt$ is as a
%% pair of ``normal" typing contexts, $\gamlt_l$ and $\gamlt_r$, with the same
%% domain and such that $\gamlt_l(x) \ltdyn \gamlt_r(x)$ for each $x$ in the domain.
%% We will write $\gamlt : \gamlt_l \ltdyn \gamlt_r$ when we want to emphasize the pair of contexts.

%% As with type precision derivations, we write $\Gamma$ to mean the ``reflexivity" type precision context
%% $\Gamma : \Gamma \ltdyn \Gamma$.
%% Concretely, this consists of reflexivity type precision derivations $\Gamma(x) \ltdyn \Gamma(x)$ for
%% each $x$ in the domain of $\Gamma$.
%% Similarly, we also have reflexivity for computation type precision contexts.
%% %
%% Furthermore, we write $\gamlt_1 \relcomp \gamlt_2$ to denote the ``composition'' of $\gamlt_1$ and $\gamlt_2$
%% --- that is, the precision context whose value at $x$ is the type precision derivation
%% $\gamlt_1(x) \relcomp \gamlt_2(x)$. This of course assumes that each of the type precision
%% derivations is composable, i.e., that the RHS of $\gamlt_1(x)$ is the same as the left-hand side of $\gamlt_2(x)$.
%% We define the same for computation type precision contexts $\deltalt_1$ and $\deltalt_2$,
%% provided that both the computation type precision contexts have the same ``shape'', which is defined as
%% (1) either the stoup is empty in both, or the stoup has a hole in both, say $\hole{d}$ and $\hole{d'}$
%% where $d$ and $d'$ are composable, and (2) their value type precision contexts are composable as described above.

%% The rules for term precision come in two forms. We first have the \emph{congruence} rules,
%% one for each term constructor. These assert that the term constructors respect term precision.
%% The congruence rules are as follows:

%% \begin{mathpar}

%%   \inferrule*[right = Var]
%%     { c : A \ltdyn B \and \gamlt(x) = (A, B) } 
%%     { \etmprec {\gamlt} x x c }

%%   \inferrule*[right = Zro]
%%     { } {\etmprec \gamlt \zro \zro \nat }

%%   \inferrule*[right = Suc]
%%     { \etmprec \gamlt V {V'} \nat } {\etmprec \gamlt {\suc\, V} {\suc\, V'} \nat}

%%   \inferrule*[right = MatchNat]
%%   {\etmprec \gamlt V {V'} \nat \and 
%%     \etmprec \deltalt M {M'} d \and \etmprec {\deltalt, n : \nat} {N} {N'} d}
%%   {\etmprec \deltalt {\matchnat {V} {M} {n} {N}} {\matchnat {V'} {M'} {n} {N'}} d}

%%   \inferrule*[right = Lambda]
%%     { c_i : A_i \ltdyn A'_i \and 
%%       c_o : A_o \ltdyn A'_o \and 
%%       \etmprec {\cdot , \gamlt , x : c_i} {M} {M'} {\Ret c_o} } 
%%     { \etmprec \gamlt {\lda x M} {\lda x {M'}} {(c_i \ra c_o)} }

%%   \inferrule*[right = App]
%%     { c_i : A_i \ltdyn A'_i \and
%%       c_o : A_o \ltdyn A'_o \\\\
%%       \etmprec \gamlt {V_f} {V_f'} {(c_i \ra c_o)} \and
%%       \etmprec \gamlt {V_x} {V_x'} {c_i}
%%     } 
%%     { \etmprec {\cdot , \gamlt} {V_f\, V_x} {V_f'\, V_x'} {\Ret {c_o}}}

%%   \inferrule*[right = Ret]
%%     {\etmprec {\gamlt} V {V'} c}
%%     {\etmprec {\cdot , \gamlt} {\ret\, V} {\ret\, V'} {\Ret c}}

%%   \inferrule*[right = Bind]
%%     {\etmprec {\deltalt} {M} {M'} {\Ret c} \and 
%%      \etmprec {\cdot , \deltalt|_V , x : c} {N} {N'} {d} }
%%     {\etmprec {\deltalt} {\bind {x} {M} {N}} {\bind {x} {M'} {N'}} {d}}
%% \end{mathpar}

%% We then have additional equational axioms, including transitivity, $\beta$ and $\eta$ laws, and
%% rules characterizing upcasts as least upper bounds, and downcasts as greatest lower bounds.

%% We write $M \equidyn N$ to mean that both $M \ltdyn N$ and $N \ltdyn M$.

%% % TODO adapt these for value/computation distinction
%% % TODO substitution rules for values and terms?
%% \begin{mathpar}
%%   \inferrule*[right = $\err$]
%%     { \hasty {\deltalt_l} M B }
%%     {\etmprec {\Delta} {\err_B} M B}

%%   \inferrule*[right = Transitivity]
%%     { d : B \ltdyn B' \and d' : B' \ltdyn B'' \\\\
%%      \etmprec {\deltalt_1} {M} {M'} {d} \and
%%      \etmprec {\deltalt_2} {M'} {M''} {d'} } 
%%     {\etmprec {\deltalt_1 \relcomp \deltalt_2} {M} {M''} {d \relcomp d'} }


%%   \inferrule*[right = $\beta$-fun]
%%     { \hasty {\cdot, \Gamma, x : A_i} M {\Ret A_o} \and
%%       \hasty {\Gamma} V {A_i} } 
%%     { \etmequidyn {\cdot, \Gamma} {(\lda x M)\, V} {M[V/x]} {\Ret A_o} }

%%   \inferrule*[right = $\eta$-fun]
%%     { \hasty {\Gamma} {V} {A_i \ra A_o} } 
%%     { \etmequidyn \Gamma {\lda x (V\, x)} V {A_i \ra A_o} }

%%   % Match-nat beta and eta



%%   \inferrule*[right = $\beta$-ret]
%%     {}
%%     {\bind{x}{\ret\, V}{N} \equidyn N[V/x]}

%%   \inferrule*[right = $\eta$-ret]
%%     {\hasty {\hole{\Ret A} , \Gamma} {M} {B}}
%%     {\hole{\Ret A}, \Gamma \vdash M \equidyn \bind{x}{\bullet}{M[\ret\, x]}}
    

%%   % Could specify \gamlt : \Gamma \ltdyn \Gamma'
%%   % and then we wouldn't need to say l and r

%%   \inferrule*[right = UpR]
%%     { d : A \ltdyn A' \and 
%%       \hasty {\Delta} {M} {A} } 
%%     { \etmprec {\Delta} {M} {\up {A} {A'} M} {d}  }

%%   \inferrule*[right = UpL]
%%     { d : A \ltdyn A' \and
%%       \etmprec {\deltalt} {M} {N} {d} } 
%%     { \etmprec {\deltalt} {\up {A} {A'} M} {N} {A'} }

%%   \inferrule*[right = DnL]
%%     { d : B \ltdyn B' \and 
%%       \hasty {\Delta} {M} {B'} } 
%%     { \etmprec {\Delta} {\dn {B} {B'} M} {M} {d} }

%%   \inferrule*[right = DnR]
%%     { d : B \ltdyn B' \and
%%       \etmprec {\deltalt} {M} {N} {d} } 
%%     { \etmprec {\deltalt} {M} {\dn {B} {B'} N} {B} }
%% \end{mathpar}

%% % TODO explain the least upper bound/greatest lower bound rules
%% The rules UpR, UpL, DnL, and DnR were introduced in \cite{new-licata18} as a means
%% of cleanly axiomatizing the intended behavior of casts in a way that
%% doesn't depend on the specific constructs of the language.
%% Intuitively, rule UpR says that the upcast of $M$ is an upper bound for $M$
%% in that $M$ may error more, and UpL says that the upcast is the \emph{least}
%% such upper bound, in that it errors more than any other upper bound for $M$.
%% Conversely, DnL says that the downcast of $M$ is a lower bound, and DnR says
%% that it is the \emph{greatest} lower bound.
%% % These rules provide a clean axiomatization of the behavior of casts that doesn't
%% % depend on the specific constructs of the language.

%% % ---------------------------------------------------------------------------------------
%% % ---------------------------------------------------------------------------------------
%% \subsection{Removing Transitivity as a Primitive}

%% The first observation we make is that transitivity of type precision, and heterogeneous
%% transitivity of term precision, are admissible. That is, consider a related language which
%% is the same as $\extlc$ except that we have removed the composition rule for type precision and
%% the heterogeneous transitivity rule for type precision. Denote this language by $\extlcm$.
%% We claim that in this new language, the rules we removed are derivable from the remaining rules.

%% To see this, suppose $\gamlt : \Gamma \ltdyn \Gamma'$ and $d : A \ltdyn A'$, and that
%%  $\etmprec {\gamlt} {V} {V'} {d}$, as shown in the diagram below:

%% % https://q.uiver.app/?q=WzAsNCxbMCwwLCJcXEdhbW1hIl0sWzAsMSwiXFxHYW1tYSciXSxbMSwwLCJBIl0sWzEsMSwiQSciXSxbMCwxLCJcXGx0ZHluIiwzLHsic3R5bGUiOnsiYm9keSI6eyJuYW1lIjoibm9uZSJ9LCJoZWFkIjp7Im5hbWUiOiJub25lIn19fV0sWzIsMywiXFxsdGR5biIsMyx7InN0eWxlIjp7ImJvZHkiOnsibmFtZSI6Im5vbmUifSwiaGVhZCI6eyJuYW1lIjoibm9uZSJ9fX1dLFswLDIsIlYiXSxbMSwzLCJWJyJdLFs2LDcsIlxcbHRkeW4iLDMseyJzaG9ydGVuIjp7InNvdXJjZSI6MjAsInRhcmdldCI6MjB9LCJzdHlsZSI6eyJib2R5Ijp7Im5hbWUiOiJub25lIn0sImhlYWQiOnsibmFtZSI6Im5vbmUifX19XV0=
%% \[\begin{tikzcd}[ampersand replacement=\&]
%% 	\Gamma \& A \\
%% 	{\Gamma'} \& {A'}
%% 	\arrow["\ltdyn"{marking}, draw=none, from=1-1, to=2-1]
%% 	\arrow["\ltdyn"{marking}, draw=none, from=1-2, to=2-2]
%% 	\arrow[""{name=0, anchor=center, inner sep=0}, "V", from=1-1, to=1-2]
%% 	\arrow[""{name=1, anchor=center, inner sep=0}, "{V'}", from=2-1, to=2-2]
%% 	\arrow["\ltdyn"{marking}, draw=none, from=0, to=1]
%% \end{tikzcd}\]

%% Now note that this is equivalent, by the cast rule UpL, to
%% $\etmprec {\Gamma'} {\up{A}{A'} V} {V'} {A'}$,
%% where as noted above, $\Gamma'$ refers to the context $\Gamma'$ viewed as a reflexivity
%% precision context and likewise the $A'$ at the end refers to the reflexivity derivation $A' \ltdyn A'$.

%% % https://q.uiver.app/?q=WzAsMixbMCwwLCJcXEdhbW1hJyJdLFsxLDAsIkEnIl0sWzAsMSwiXFx1cCB7QX0ge0EnfSBWIiwwLHsiY3VydmUiOi0yfV0sWzAsMSwiViciLDIseyJjdXJ2ZSI6Mn1dLFsyLDMsIlxcbHRkeW4iLDMseyJzaG9ydGVuIjp7InNvdXJjZSI6MjAsInRhcmdldCI6MjB9LCJzdHlsZSI6eyJib2R5Ijp7Im5hbWUiOiJub25lIn0sImhlYWQiOnsibmFtZSI6Im5vbmUifX19XV0=
%% \[\begin{tikzcd}[ampersand replacement=\&]
%% 	{\Gamma'} \& {A'}
%% 	\arrow[""{name=0, anchor=center, inner sep=0}, "{\up {A} {A'} V}", curve={height=-12pt}, from=1-1, to=1-2]
%% 	\arrow[""{name=1, anchor=center, inner sep=0}, "{V'}"', curve={height=12pt}, from=1-1, to=1-2]
%% 	\arrow["\ltdyn"{marking}, draw=none, from=0, to=1]
%% \end{tikzcd}\]

%% Now consider the situation shown below:

%% % https://q.uiver.app/?q=WzAsNixbMCwwLCJcXEdhbW1hIl0sWzAsMSwiXFxHYW1tYSciXSxbMCwyLCJcXEdhbW1hJyciXSxbMiwwLCJBIl0sWzIsMSwiQSciXSxbMiwyLCJBJyciXSxbMiw1LCJWJyciXSxbMSw0LCJWJyJdLFswLDMsIlYiXSxbMyw0LCJcXGx0ZHluIiwzLHsic3R5bGUiOnsiYm9keSI6eyJuYW1lIjoibm9uZSJ9LCJoZWFkIjp7Im5hbWUiOiJub25lIn19fV0sWzQsNSwiXFxsdGR5biIsMyx7InN0eWxlIjp7ImJvZHkiOnsibmFtZSI6Im5vbmUifSwiaGVhZCI6eyJuYW1lIjoibm9uZSJ9fX1dLFswLDEsIlxcbHRkeW4iLDMseyJzdHlsZSI6eyJib2R5Ijp7Im5hbWUiOiJub25lIn0sImhlYWQiOnsibmFtZSI6Im5vbmUifX19XSxbMSwyLCIiLDEseyJzdHlsZSI6eyJib2R5Ijp7Im5hbWUiOiJub25lIn0sImhlYWQiOnsibmFtZSI6Im5vbmUifX19XSxbMSwyLCJcXGx0ZHluIiwzLHsic3R5bGUiOnsiYm9keSI6eyJuYW1lIjoibm9uZSJ9LCJoZWFkIjp7Im5hbWUiOiJub25lIn19fV0sWzgsNywiXFxsdGR5biIsMyx7InNob3J0ZW4iOnsic291cmNlIjoyMCwidGFyZ2V0IjoyMH0sInN0eWxlIjp7ImJvZHkiOnsibmFtZSI6Im5vbmUifSwiaGVhZCI6eyJuYW1lIjoibm9uZSJ9fX1dLFs3LDYsIlxcbHRkeW4iLDMseyJzaG9ydGVuIjp7InNvdXJjZSI6MjAsInRhcmdldCI6MjB9LCJzdHlsZSI6eyJib2R5Ijp7Im5hbWUiOiJub25lIn0sImhlYWQiOnsibmFtZSI6Im5vbmUifX19XV0=
%% \[\begin{tikzcd}[ampersand replacement=\&]
%% 	\Gamma \&\& A \\
%% 	{\Gamma'} \&\& {A'} \\
%% 	{\Gamma''} \&\& {A''}
%% 	\arrow[""{name=0, anchor=center, inner sep=0}, "{V''}", from=3-1, to=3-3]
%% 	\arrow[""{name=1, anchor=center, inner sep=0}, "{V'}", from=2-1, to=2-3]
%% 	\arrow[""{name=2, anchor=center, inner sep=0}, "V", from=1-1, to=1-3]
%% 	\arrow["\ltdyn"{marking}, draw=none, from=1-3, to=2-3]
%% 	\arrow["\ltdyn"{marking}, draw=none, from=2-3, to=3-3]
%% 	\arrow["\ltdyn"{marking}, draw=none, from=1-1, to=2-1]
%% 	\arrow[draw=none, from=2-1, to=3-1]
%% 	\arrow["\ltdyn"{marking}, draw=none, from=2-1, to=3-1]
%% 	\arrow["\ltdyn"{marking}, draw=none, from=2, to=1]
%% 	\arrow["\ltdyn"{marking}, draw=none, from=1, to=0]
%% \end{tikzcd}\]


%% Using the above observation, we have that the above is equivalent to

%% % https://q.uiver.app/?q=WzAsNCxbMCwwLCJcXEdhbW1hJyJdLFswLDEsIlxcR2FtbWEnJyJdLFsyLDAsIkEnIl0sWzIsMSwiQScnIl0sWzAsMiwiXFx1cCB7QX0ge0EnfSBWIiwwLHsiY3VydmUiOi0yfV0sWzAsMiwiViciLDIseyJjdXJ2ZSI6Mn1dLFsxLDMsIlYnJyIsMix7ImN1cnZlIjoyfV0sWzAsMSwiXFxsdGR5biIsMyx7InN0eWxlIjp7ImJvZHkiOnsibmFtZSI6Im5vbmUifSwiaGVhZCI6eyJuYW1lIjoibm9uZSJ9fX1dLFsyLDMsIlxcbHRkeW4iLDMseyJzdHlsZSI6eyJib2R5Ijp7Im5hbWUiOiJub25lIn0sImhlYWQiOnsibmFtZSI6Im5vbmUifX19XSxbNCw1LCJcXGx0ZHluIiwzLHsic2hvcnRlbiI6eyJzb3VyY2UiOjIwLCJ0YXJnZXQiOjIwfSwic3R5bGUiOnsiYm9keSI6eyJuYW1lIjoibm9uZSJ9LCJoZWFkIjp7Im5hbWUiOiJub25lIn19fV0sWzUsNiwiXFxsdGR5biIsMyx7InNob3J0ZW4iOnsic291cmNlIjoyMCwidGFyZ2V0IjoyMH0sInN0eWxlIjp7ImJvZHkiOnsibmFtZSI6Im5vbmUifSwiaGVhZCI6eyJuYW1lIjoibm9uZSJ9fX1dXQ==
%% \[\begin{tikzcd}[ampersand replacement=\&]
%% 	{\Gamma'} \&\& {A'} \\
%% 	{\Gamma''} \&\& {A''}
%% 	\arrow[""{name=0, anchor=center, inner sep=0}, "{\up {A} {A'} V}", curve={height=-12pt}, from=1-1, to=1-3]
%% 	\arrow[""{name=1, anchor=center, inner sep=0}, "{V'}"', curve={height=12pt}, from=1-1, to=1-3]
%% 	\arrow[""{name=2, anchor=center, inner sep=0}, "{V''}"', curve={height=12pt}, from=2-1, to=2-3]
%% 	\arrow["\ltdyn"{marking}, draw=none, from=1-1, to=2-1]
%% 	\arrow["\ltdyn"{marking}, draw=none, from=1-3, to=2-3]
%% 	\arrow["\ltdyn"{marking}, draw=none, from=0, to=1]
%% 	\arrow["\ltdyn"{marking}, draw=none, from=1, to=2]
%% \end{tikzcd}\]

%% % TODO finish the explanation
  

%% % ---------------------------------------------------------------------------------------
%% % ---------------------------------------------------------------------------------------

%% \subsection{Removing Casts as Primitives}

%% % We now observe that all casts, except those between $\nat$ and $\dyn$
%% % and between $\dyn \ra \dyn$ and $\dyn$, are admissible, in the sense that
%% % we can start from $\extlcm$, remove casts except the aforementioned ones,
%% % and in the resulting language we will be able to derive the other casts.

%% We now observe that all casts, except those between $\nat$ and $\dyn$
%% and between $\dyn \ra \dyn$ and $\dyn$, are admissible.
%% That is, consider a new language ($\extlcmm$) in which
%% instead of having arbitrary casts, we have injections from $\nat$ and
%% $\dyn \ra \dyn$ into $\dyn$, and case inspections from $\dyn$ to $\nat$ and
%% $\dyn$ to $\dyn \ra \dyn$. We claim that in $\extlcmm$, all of the casts
%% present in $\extlcm$ are derivable.
%% It will suffice to verify that casts for function type are derivable.
%% This holds because function casts are constructed inductively from the casts
%% of their domain and codomain. The base case is one of the casts involving $\nat$
%% or $\dyn \ra \dyn$ which are present in $\extlcmm$ as injections and case inspections.


%% The resulting calculus $\extlcmm$ now lacks transitivity of type precision,
%% heterogeneous transitivity of term precision, and arbitrary casts as primitive
%% notions.

%% \begin{align*}
%%   &\text{Value Types } A := \nat \alt \dyn \alt (A \ra A') \\
%%   &\text{Computation Types } B := \Ret A \\
%%   &\text{Value Contexts } \Gamma := \cdot \alt (\Gamma, x : A) \\
%%   &\text{Computation Contexts } \Delta := \cdot \alt \hole B \alt \Delta , x : A \\
%%   &\text{Values } V :=  \zro \alt \suc\, V \alt \lda{x}{M} \alt \injnat V \alt \injarr V \\ 
%%   &\text{Terms } M, N := \err_B \alt \ret {V} \alt \bind{x}{M}{N}
%%     \alt V_f\, V_x \alt
%%     \\ & \quad\quad \casenat{V}{M_{no}}{n}{M_{yes}} 
%%     \alt \casearr{V}{M_{no}}{f}{M_{yes}}
%% \end{align*}

%% In this setting, rather than type precision, it makes more sense to
%% speak of arbitrary \emph{monotone relations} on types, which we denote by $A \rel A'$.
%% We have relations on value types, as well as on computation types. We also have
%% value relation contexts and computation relation contexts, analogous to the value type
%% precision contexts and computation type precision contexts from before.

%% \begin{align*}
%%   &\text{Value Relations } R := \nat \alt \dyn \alt (R \ra R) \alt\, \dyn\, R(V_1, V_2)\\
%%   &\text{Computation Relations } S := \li R \\
%%   &\text{Value Relation Contexts } \Gamma^{\rel} := \cdot \alt \Gamma^{\rel} , A^{\rel} (x_l : A_l , x_r : A_r)\\
%%   &\text{Computation Relation Contexts } \Delta^{\rel} := \cdot \alt \hole{B^{\rel}} \alt 
%%     \Delta^{\rel} , A^{\rel} (x_l : A_l , x_r : A_r)   \\
%% \end{align*}

%% % TODO rules for relations
%% The forms for relations are as follows:

%% \begin{align*}
%%   A^{\rel}      &\colon A_l      \rel A_r \\
%%   \Gamma^{\rel} &\colon \Gamma_l \rel \Gamma_r \\
%%   B^{\rel}      &\colon B_l      \rel B_r \\
%%   \Delta^{\rel} &\colon \Delta_l \rel \Delta_r
%% \end{align*}



%% Figure \ref{fig:relation-rules} shows the rules for relations. We show only those for value types;
%% the corresponding computation type relation rules are analogous.
%% The rules for relations are as follows. First, we require relations to be reflexive.
%% We also require that they are \emph{profunctorial}, in the sense that a relation between
%% $A$ and $A'$ is closed under the ``homogeneous'' relations on both sides.
%% We also require that they satisfy a substitution principle.

%% \begin{figure}
%%   \begin{mathpar}
%%     \inferrule*[right = Reflexivity]
%%     {\hasty \Gamma V A}
%%     {\refl(\Gamma) \vdash \refl(A)(V, V)}

%%     \inferrule*[right = Profunctoriality]
%%     { \refl(\Gamma^{\rel}_l) \vdash  \refl(A^{\rel}_l) (V_l' , V_l) \\\\ 
%%         \Gamma^{\rel}    \vdash    A^{\rel}    (V_l  , V_r) \\\\
%%       \refl(\Gamma^{\rel}_r) \vdash  \refl(A^{\rel}_r) (V_r  , V_r')
%%     }
%%     {\Gamma^{\rel} \vdash A^{\rel} (V_l', V_r')}

%%     \inferrule*[right = Subst]
%%     { \Gamma'^{\rel} \vdash \Gamma^{\rel} (\gamma_l, \gamma_r) \\\\
%%       \Gamma^{\rel} A^{\rel} (V_l, V_r)
%%     }
%%     {\Gamma'^{\rel} \vdash A^{\rel} (V_l[\gamma_l] , V_r[\gamma_r]) }

%%     % \inferrule*[right = TermSubst]
%%     % { \Delta'^{\rel} \vdash \Delta^{\rel} (\delta_l, \delta_r) \\\\
%%     %   \Delta^{\rel} B^{\rel} (M_l, M_r)
%%     % }
%%     % {\Delta'^{\rel} \vdash B^{\rel} (M_l[\delta_l] , M_r[\delta_r]) }

%%   \end{mathpar}
%%   \caption{Rules for value type relations. The rules for computation type relations are analogous.}
%%   \label{fig:relation-rules}
%% \end{figure}

%% We also have a rule for the restriction of a relation along a function,
%% and we have a rule characterizing relation at function type. The latter states that
%% if under the assumption that $x$ is related to $x'$ by $A^{\rel}$, we can show that $M$
%% is related to $M'$ by $\li A'^{\rel}$, then we have that $\lda{x}{M}$ is related to
%% $\lda{x'}{M'}$ by $A^{\rel} \ra A'^{\rel}$.

%% \begin{mathpar}
%%   \mprset{fraction={===}}

%%   % \inferrule*[]
%%   % { A^{\rel}  (x_l, x_r) \vdash A^{\rel} (V_l, V_r) }
%%   % { A'^{\rel} (x_l, x_r) \vdash A^{\rel} (V_l, V_r)(x_l, x_r) }

%%   \inferrule*[right = Restriction]
%%   { \Gamma^{\rel} \vdash A^{\rel} (V_l (V_l'), V_r (V_r')) }
%%   { \Gamma^{\rel} \vdash (A^{\rel} (V_l, V_r)) (V_l', V_r') }

%%   \inferrule*[right = $\text{Rel}_\ra$]
%%   { A^{\rel} (x, x') \vdash (\li A'^{\rel})(M , M') }
%%   {  \vdash (A^{\rel} \ra A'^{\rel}) (\lda{x}{M}) , (\lda{x'}{M'})}

%% \end{mathpar}



%% % New rules
%% Figure \ref{fig:extlc-minus-minus-typing} shows the new typing rules,
%% and Figure \ref{fig:extlc-minus-minus-eqns} shows the equational rules
%% for case-nat (the rules for case-arrow are analogous).

%% \begin{figure}
%%   \begin{mathpar}
%%       % inj-nat
%%       \inferrule*
%%       {\hasty \Gamma M \nat}
%%       {\hasty \Gamma {\injnat M} \dyn}

%%       % inj-arr 
%%       \inferrule*
%%       {\hasty \Gamma M (\dyn \ra \dyn)}
%%       {\hasty \Gamma {\injarr M} \dyn}

%%       % Case nat
%%       \inferrule*
%%       {\hasty{\Delta|_V}{V}{\dyn} \and 
%%         \hasty{\Delta , x : \nat }{M_{yes}}{B} \and 
%%         \hasty{\Delta}{M_{no}}{B}}
%%       {\hasty {\Delta} {\casenat{V}{M_{no}}{n}{M_{yes}}} {B}}
    
%%       % Case arr
%%       \inferrule*
%%       {\hasty{\Delta|_V}{V}{\dyn} \and 
%%         \hasty{\Delta , x : (\dyn \ra \dyn) }{M_{yes}}{B} \and 
%%         \hasty{\Delta}{M_{no}}{B}}
%%       {\hasty {\Delta} {\casearr{V}{M_{no}}{f}{M_{yes}}} {B}}
%%   \end{mathpar}
%%   \caption{New typing rules for $\extlcmm$.}
%%   \label{fig:extlc-minus-minus-typing}
%% \end{figure}


%% \begin{figure}
%%   \begin{mathpar}
%%      % Case-nat Beta
%%      \inferrule*
%%      {\hasty \Gamma V \nat}
%%      {\casenat {\injnat {V}} {M_{no}} {n} {M_{yes}} = M_{yes}[V/n]}

%%      \inferrule*
%%      {\hasty \Gamma V {\dyn \ra \dyn} }
%%      {\casenat {\injarr {V}} {M_{no}} {n} {M_{yes}} = M_{no}}

%%      % Case-nat Eta
%%      \inferrule*
%%      {}
%%      {\Gamma , x :\, \dyn \vdash M = \casenat{x}{M}{n}{M[(\injnat{n}) / x]} }


%%      % Case-arr Beta


%%      % Case-arr Eta


%%   \end{mathpar}
%%   \caption{New equational rules for $\extlcmm$ (rules for case-arrow are analogous
%%            and hence are omitted).}
%%   \label{fig:extlc-minus-minus-eqns}
%% \end{figure}



%% % TODO : Updated term precision rules



%% \subsection{The Step-Sensitive Lambda Calculus}\label{sec:step-sensitive-lc}

%% % \textbf{TODO: Subject to change!}

%% Rather than give a semantics to $\extlcmm$ directly, we first introduce another intermediary
%% language, a \emph{step-sensitive} (also called \emph{intensional}) calculus.
%% As mentioned, this language makes the intensional stepping behavior of programs
%% explicit in the syntax. We do this by adding a syntactic ``later'' type and a
%% syntactic $\theta$ that takes terms of type later $A$ to terms of type $A$.

%% % In the step-sensitive syntax, we add a type constructor for later, as well as a
%% % syntactic $\theta$ term and a syntactic $\nxt$ term.
%% We add rules for these new constructs, and also modify the rules for inj-arr and
%% case-arrow, since now the function is not $\Dyn \ra \Dyn$ but rather $\later (\Dyn \ra \Dyn)$.
%% We also add congruence relations for $\later$ and $\nxt$.

%% % TODO show changes

%% \noindent Modified syntax:
%% \begin{align*}
%%   &\text{Value Types } A := \nat \alt \dyn \alt (A \ra A') \alt {\color{red} \later A} \\
%%   &\text{Values } V :=  \zro \alt \suc\, V \alt \lda{x}{M} \alt \injnat V \alt \injarr V 
%%     \alt {\color{red} \nxt\, V} \alt {\color{red} \mathbf{\theta}}
%% \end{align*}

%% \noindent Additional typing rules:
%% \begin{mathpar}
%%   \inferrule
%%   {\hasty \Gamma V A}
%%   {\hasty \Gamma {\nxt\, V} {\later A}}

%%   \inferrule
%%   {}
%%   {\hasty \Gamma \theta {\later A \ra A}}

%%   % \theta(\nxt x) = \theta(y); \texttt{ret}\, x
%% \end{mathpar}

%% \noindent Modified typing rules:
%% \begin{mathpar}

%%   % inj-arr 
%%   \inferrule*
%%   {\hasty \Gamma M {\color{red} \later (\dyn \ra \dyn)}}
%%   {\hasty \Gamma {\injarr M} \dyn}

%%   % Case arr
%%   % TODO if the extensional version is incorrect and needs to change, make
%%   % sure to change this one accordingly
%%   \inferrule*
%%   {\hasty{\Delta|_V}{V}{\dyn} \and 
%%     \hasty{\Delta , x \colon {\color{red} \later (\dyn \ra \dyn)} }{M_{yes}}{B} \and 
%%     \hasty{\Delta}{M_{no}}{B}}
%%   {\hasty {\Delta} {\casearr{V}{M_{no}}{\tilde{f}}{M_{yes}}} {B}}  
%% \end{mathpar}

%% \noindent Additional relations:
%% \begin{mathpar}
%%   \inferrule*[]
%%   {A^{\rel} : A_l \rel A_r}
%%   {\later A^{\rel} : \later A_l \rel \later A_r}

%%   \inferrule*[]
%%   {A^{\rel} (V_l, V_r)}
%%   {\later A^{\rel} (\nxt\, V_l, \nxt\, V_r)}

%% \end{mathpar}

%% % TODO what about the relation for theta? Or is that automatic since it's a function symbol?

%% % TODO beta rule for theta

%% We define the term $\delta$ to be the function $\lda {x} {\theta\, (\nxt\, x)}$.

%% % We define an erasure function from step-sensitive syntax to step-insensitive syntax
%% % by induction on the step-sensitive types and terms.
%% % The basic idea is that the syntactic type $\later A$ erases to $A$,
%% % and $\nxt$ and $\theta$ erase to the identity.



%% \subsection{Quotienting by Syntactic Bisimilarity}

%% We now define a quotiented variant of the above step-sensitive calculus,
%% which we denote by $\intlcbisim$.
%% In this syntax, we add a rule saying, roughly speaking, that 
%% $\theta \circ \nxt$ is the identity. This causes terms that differ only in
%% their intensional behavior to become equal.
%% Note that a priori, this is not the same language as the step-insensitive
%% calculus on which we based the insensitive calculus.

%% Formally, the equational theory for the quotiented syntax is the same as
%% that of the original step-sensitive language, with the addition of the following
%% rule:

%% % TODO is this correct?
%% \begin{mathpar}
%%   \inferrule*
%%   { }
%%   { \theta\, (\nxt\, x) = \bind{y}{(\theta\, V')}{\ret\, x}  }
%% \end{mathpar}

%% This states that the application of $\theta$ to $\nxt\, x$ is equivalent to
%% the computation that applies $\theta$ to $V'$ to obtain a variable $y$, and
%% then simply returns $x$.



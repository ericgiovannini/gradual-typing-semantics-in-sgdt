\section{Towards a Full Relational Model}\label{sec:towards-relational-model}


% In the previous section, we gave a denotational semantics to the terms of the
% gradually-typed lambda calculus using the tools of SGDT. We defined a big-step
% semantics for closed terms of type $\nat$, as functions from $\mathbb{N} \to
% ((\mathbb{N} + 1) + 1)$.

\eric{Wording?}
In the previous section, we gave a denotational semantics to the terms of the
gradually-typed lambda calculus using the tools of SGDT. In this section, we
build on the ideas used in defining a term model with the goal of defining a
model of the full gradually-typed lambda calculus, i.e., one that has constructs
that model type and term precision.
%
% The purpose of this section is to motivate the need for many of the
% seemingly complicated features of the abstract model introduced in
% Section \ref{sec:abstract-models}.

We will not give the complete description of the concrete model in this section;
rather we seek to develop just enough to motivate the need for the non-standard
aspects in the definitions of abstract model introduced in Section
\ref{sec:abstract-models}. After defining the notion of abstract model and
describing a modular way of constructing such a model, we will return in Section
\ref{sec:concrete-model} to the complete construction of a concrete model.  

\subsection{A First Attempt}

% TODO should we frame this in terms of the idealized categorical models
% i.e. say what the vertical and horizontal morphisms and squares are
% introduced previously? What is the proper denotation of values versus terms?
% We don't distinguish between value types and computation types in the syntax

As a first attempt at giving a semantics to the term precision relation, we
could try to model types as sets equipped with an ordering that models term
precision. Since term precision is reflexive and transitive, and since we
identify terms that are equi-precise, we could model value types as
partially-ordered sets and values $\Gamma \vdash V : A$ as \emph{monotone}
functions. Analogously, every error domain is now equipped with a partial
ordering for which the error element is the bottom, and the map $\theta_B
\laterhs B \to B$ is now required to be monotone. 
%
\footnote{This requires us to define a notion of ``later'' for partially-ordered
sets $X$, but this can be done. The elements are simply ``later''-elements of
$X$, and the ordering is the obvious one, i.e., two later-elements are related
if after one time step, the elements they step to are related.}
%
% \eric{What about computation
% types? We need to extend the error domains mentioned in the term model section
% to also have an ordering such that the error element is the least element and
% the theta morphism is monotone. Is this best introduced here, or should we wait
% until after we have introduced the revised notions of abstract model?}
%
We model type precision $c : A \ltdyn B$ as a \emph{montone relation}, i.e., a
relation $R$ that is upward-closed under the relation on $B$ and downward-closed
under the relation on $A$. We could then model the term precision ordering $M
\ltdyn N : A \ltdyn B$ as an ordering relation between the monotone functions
denoted by $M$ and $N$.

However, it turns out that modeling term precision is not as straightforward as
one might hope. A first attempt might be to define an ordering $\semltbad$
between $\li X$ and $\li Y$ that allows for computations that may take different
numbers of steps to be related. We call this the \emph{step-insensitive error
ordering}. The ordering is parameterized by an ordering relation $\le$ between
$X$ and $Y$. The definition is by guarded recursion and is shown in Figure
\ref{fig:step-insensitive-error-ordering}.

\begin{figure*}
    \begin{align*}
        &\eta\, x \semltbad \eta\, y \text{ if } 
            x \semlt y \\
        %		
        &\mho \semltbad l \\
        %
        &\theta\, \tilde{l} \semltbad \theta\, \tilde{l'} \text{ if } 
            \later_t (\tilde{l}_t \semltbad \tilde{l'}_t) \\
        %	
        &\theta\, \tilde{l} \semltbad \mho \text{ if } 
            \theta\, \tilde{l} = \delta^n(\mho) \text { for some $n$ } \\
        %	
        &\theta\, \tilde{l} \semltbad \eta\, y \text{ if }
            (\theta\, \tilde{l} = \delta^n(\eta\, x))
        \text { for some $n$ and $x : \ty{X}$ such that $x \le y$ } \\
        %
        &\mho \semltbad \theta\, \tilde{l'} \text { if } 
            \theta\, \tilde{l'} = \delta^n(\mho) \text { for some $n$ } \\
        %	
        &\eta\, x \semltbad \theta\, \tilde{l'} \text { if }
            (\theta\, \tilde{l'} = \delta^n (\eta\, y))
        \text { for some $n$ and $y : \ty{Y}$ such that $x \le y$ }
    \end{align*}
    \caption{The step-insensitive error ordering.}
    \label{fig:step-insensitive-error-ordering}
\end{figure*}

Two computations that immediately return $(\eta)$ are related if the returned
values are related in the underlying ordering. The computation that errors
$(\mho)$ is the least term in the ordering. If both sides step (i.e., both sides
are $\theta$), then we allow one time step to pass and compare the resulting
terms (this is where use the relation defined ``later'' and is why we employ
guarded recursion to define the relation).
%
Lastly, if one side steps and the other immediately returns a value, then in
order for these terms to be related, the side that steps must terminate with a
value in some finite number of steps $n$, and that value must be related to
the value returned by the other side. Likewise, if the LHS steps and the RHS
immediately errors, then the LHS must eventually terminate with error.

\subsection{Problem: Failure of Transitivity}

The problem with the above definition is that the relation is \emph{not}
transitive, which we explain below. We begin with a general ``no-go'' theorem
about relations defined on the guarded lift monad:

\begin{theorem}[No-go theorem for relations on $\li X$]\label{thm:no-go}
    Let $R$ be a relation on the guarded lift monad satisfying the following
    three properties:
    \begin{enumerate}
        \item Transitivity
        
        \item Congruence with respect to $\theta$: If $\later_t (\tilde{x}_t
        \binrel{R} \tilde{y}_t)$, then $\theta(\tilde{x}) \binrel{R}
        \theta(\tilde{y})$

        \item Closure under delay on both sides: If $x \binrel{R} y$, then $x
        \binrel{R} \delta y$ and $\delta x  \binrel{R} y$
    \end{enumerate}

    Then $R$ is trivial, i.e., $x \binrel{R} y$ for all $x : \li X$ and $y : \li
    Y$.
    
\end{theorem}

This theorem follows from two ``one-sided'' lemmas where we assume that the
relation $R$ is closed under delay on the left (resp. right) and show that
$(\fix\, \theta) \binrel{R} y$ for all $y$ (resp. $x \binrel{R} (\fix\, \theta)$
for all $x$). These lemmas in turn are proved by \lob-induction.

The theorem implies that the relation $\semltbad$ defined on $\li X \times \li
Y$ is \emph{not} transitive whenever the underlying relation $\semlt$ on $X$ and
$Y$ is not trivial, i.e., when there are $x_0$ and $y_0$ that are not related in
the relation $\semlt$ on $X$ and $Y$. To show this, first observe that
$\semltbad$ satisfies properties (2) and (3) in the above theorem. Congruence
with respect to $\theta$ is immediate from the definition, while closure under
delay can be proved using \lob-induction.
% TODO the proof of this uses tick irrelevance
Then since $x_0$ is not related to $y_0$, we have that $\eta\, x_0$ is not
related to $\eta\, y_0$ in $\semltbad$, i.e., $\semltbad$ is non-trivial. But if
it were transitive, then by the theorem, it would be trivial, a contradiction.
Thus, we are able to refute transitivity of $\semltbad$ provided the underlying
relation is non-trivial.

% It can be shown (in Clocked Cubical Type
% Theory) that if $R$ is a relation on $\li X$ satisfying three specific
% properties, one of which is transitivity, then that relation is trivial. The
% other two properties are that the relation is a congruence with respect to
% $\theta$, and that the relation is closed under delays $\delta = \theta \circ
% \nxt$ on either side. Since the above relation \emph{does} satisfy the other
% two properties, we conclude that it must not be transitive.

\eric{I'm not sure that this is a satisfactory explanation...}
At this point, we could give up on transitivity and instead of requiring types
to be modeled by partially-ordered sets, we could model types as sets equipped
with a reflexive, anti-symmetric ordering relation. However, this is not
convenient for carrying out the constructions necessary for proving graduality.
For example, in order to model the UpL/UpR/DnL/DnR rules for casts, in the
presence of transitivity it suffices to model simpler rules that do not build in
composition. We can then use transitivity to derive the actual, more complicated
rules we need to show.
% For example, in order to model the UpL/UpR/DnL/DnR rules for casts, prior work
% has taken as primitive simpler rules that do not involve composition, and then
% uses transitivity to establish the more complicated versions that have
% composition ``built in''.
Without transitivity, however, this is not possible, and so we must instead take as
primitive the more complicated versions of the cast rules. Then in the model, we
must verify that those rules hold ``from scratch''. So, although this approach
would suffice for proving graduality, it lacks the compositionality that we seek
in a reusable framework for the semantics of gradual typing.

%Because it is convenient to make use of transitive reasoning in proving graduality, ...
We are therefore led to wonder whether we can formulate an ordering
relation on $\li X \times \li Y$ that \emph{is} transitive. The theorem above
tells us, however, that any such relation must either not be a congruence with
respect to $\theta$, or must not be closed under delay.
\footnote{Technically, we only need to give up closure under delay on \emph{one}
side, as the above theorem requires $R$ to be closed on both sides. We could
define two separate ``step-semi-sensitive'' relations on $\li X \times \li Y$,
each closed under delays on the left and right respectively. The denotation of
term precision would then involve the conjunction of those relations. We do not
pursue this approach in this paper, because (1) it is not clear if the
step-semi-sensitive relations would be transitive, and (2) we would need to
reason about the two ordering relations separately for every pair of terms.}
%
We cannot forego the property of being a $\theta$-congruence, as without this we
would not be able to prove properties of the relation using \lob-induction,
e.g., that the monadic bind for the guarded lift monad is a \emph{monotone}
function with respect to the ordering.
%and moreover this would mean that $\theta$ would not be a monotone function.

Thus, we choose to sacrifice closure under delays. That is, we will define an
ordering relation that requires terms to be in ``lock-step''. In order for two
computations to be related in this ordering, they must have the same stepping
behavior (or the left-hand side must be an error). To deal with terms that take
differing numbers of steps, we then define a separate relation called \emph{weak
bisimilarity} that relates terms that are extensionally equal and may differ by
a finite number of delays. Then, the semantics of the error ordering for the
guarded lift monad will involve a combination of these two relations: the
``closure'' of the lock-step error ordering under weak bisimilarity on both
sides.
% Although the combined relation will not be transitive (for the same reason that
% $\semltbad$ is not transitive), this...
This decomposition has the advantage that we can recover some transitive
reasoning and push much of the reasoning about stepping to the margins of the
development.

% Doing so, we define a \emph{lock-step} error ordering, where roughly speaking,
% in order for computations to be related, they must have the same stepping
% behavior. We then formulate a separate relation, \emph{weak bisimilarity}, that
% relates computations that are extensionally equal and may differ only in their
% stepping behavior. % up to a finite number of \delta's
% Finally, the semantics of term precision will involve a combination of these two
% relations, a sort of closure of the lock-step ordering under weak bisimilarity
% on both sides. This decomposition has the advantage that we can recover some
% transitive reasoning and push the parts involving stepping to the margins of the
% development.

\subsection{Solution: Splitting into Two Relations}\label{sec:lock-step-and-weak-bisim}

As mentioned, to recover transitive reasoning we will decompose the error
ordering on the guarded lift monad into two separate relations. The first is the
\emph{lock-step error-ordering} (also called the ``step-sensitive error
ordering''), the idea being that two computations $l$ and $l'$ are related if
they are in lock-step with regard to their intensional behavior, up to $l$
erroring. This will be the partial ordering with which we equip the guarded lift monad.
Figure \ref{fig:lock-step-error-ordering} gives the definition.

\begin{figure}
    \fbox{$l_1 \ltls l_2$}
    \begin{align*}
        &\eta\, x \ltls \eta\, y \text{ if } 
            x \mathbin{R} y \\
        %		
        &\mho \ltls l' \\
        %
        &\theta\, \tilde{l} \ltls \theta\, \tilde{l'} \text{ if } 
            \later_t (\tilde{l}_t \ltls \tilde{l'}_t)
    \end{align*}
    \caption{The lock-step error ordering.}
    \label{fig:lock-step-error-ordering}
\end{figure}

When both sides are $\eta$, then we check that the returned values are related
in $R$. The error term $\mho$ is below everything. Lastly, if both sides step
(i.e., are a $\theta$) then we compare the resulting computations one time step
later.
%
It is straightforward to prove using \lob-induction that this relation is
reflexive, transitive and anti-symmetric given that the underyling relation $R$
has those properties. The lock-step ordering is therefore the partial ordering
we will associate with $\li X$.
%
We similarly define a heterogeneous version of this ordering between the lifts
of two different types $X$ and $Y$, parameterized by a relation $R$ between $X$
and $Y$.

The second relation we define on the guarded lift monad is \emph{weak bisimilarity}.
% For a type $X$, we define a relation on $\li X$, called ``weak bisimilarity",
% written $l \bisim l'$. 
Intuitively, we say $l \bisim l'$ if they are the same ``up to delays''.
%
Weak bisimilarity on $\li X$ is parameterized by an underlying relation $R$ on $X$
and is defined by guarded fixpoint as shown in Figure \ref{fig:weak-bisimilarity}.

\begin{figure}
    \fbox{$l_1 \bisim l_2$}
    \begin{align*}
        &\mho \bisim \mho \\
      %
        &\eta\, x \bisim \eta\, y \text{ if } 
          x \mathbin{R} y \\
      %		
        &\theta\, \tilde{x} \bisim \theta\, \tilde{y} \text{ if } 
          \later_t (\tilde{x}_t \bisim \tilde{y}_t) \\
      %	
        &\theta\, \tilde{x} \bisim \mho \text{ if } 
          \theta\, \tilde{x} = \delta^n(\mho) \text { for some $n$ } \\
      %	
        &\theta\, \tilde{x} \bisim \eta\, y \text{ if }
          (\theta\, \tilde{x} = \delta^n(\eta\, x))
        \text { for some $n$ and $x : X$ such that $x \mathbin{R} y$ } \\
      %
        &\mho \bisim \theta\, \tilde{y} \text { if } 
          \theta\, \tilde{y} = \delta^n(\mho) \text { for some $n$ } \\
      %	
        &\eta\, x \bisim \theta\, \tilde{y} \text { if }
          (\theta\, \tilde{y} = \delta^n (\eta\, y))
        \text { for some $n$ and $y : X$ such that $x \mathbin{R} y$ }
      \end{align*}
    \caption{The weak bisimilarity relation.}
    \label{fig:weak-bisimilarity}
\end{figure}

When both sides are $\eta$, then we ensure that the underlying values are
bisimilar in the underlying bisimilarity relation on $A$. When one side is a
$\theta$ and the other is $\eta x$ (i.e., one side steps), we stipulate that the
$\theta$-term runs to $\eta y$ where $x$ is related to $y$. When one side is
$\theta$ and the other $\mho$, we require that the $\theta$-term runs to $\mho$.
If both sides step, then we allow one time step to pass and compare the
resulting terms. This definition captures the intuition of two terms being
related up to delays.

It can be shown (by \lob-induction) that weak bisimilarity is reflexive and
symmetric. We can also show that this relation is \emph{not} transitive. The
argument is the same as that used to show that the step-insensitive error
ordering $\semltbad$ described above is not transitive. In particular, we
observe that weak bisimilarity is a congruence with respect to $\theta$, and we
can show by $\lob$-induction that it is closed under delay on both sides. Thus,
we may appeal to the no-go theorem as before.
% We appeal to the no-go theorem given above

% This issue of non-transitivity will be resolved when we consider the relation's \emph{globalization}.

Now we can define the denotation of term precision for closed terms of type
$\nat$ as follows:
%
\[ l_1 \leq l_2 := 
  \Sigma_{l_1', l_2' \in \li \mathbb{N}}. l_1 \bisim l_1' \ltls l_2' \bisim l_2. \]

\eric{Does this belong in the next section?}
Importantly, we note that as with the original ordering $\semltbad$ defined at
the beginning of this section, the relation just defined is not transitive
(again as a result of the above no-go theorem). It may therefore seem that we
have not solved the original issue we faced. In a sense, this is true. Indeed,
we will need to weaken our abstract notion of categorical model to no longer
require transitivity of term precision. On the other hand, we now have the
ability to apply transitive reasoning when working with the lock-step ordering.
This allows us to carry out complex proofs compositionally, a key goal of our
framework.

\subsection{Failure of the Cast Rules}

We have seen how we can recover transitive reasoning by working with the
lock-step error ordering. However, we cannot completely isolate the reasoning
about steps from the reasoning about the ordering.
% Unfortunately, by defining the partial order on the guarded lift monad to be
% the lock-step error ordering, we introduce a new issue.
In particular, the cast rules required for proving graduality no longer hold if
interpreted using the lock-step ordering. The source of the problem is that the
downcast from the dynamic type to a function takes a step, i.e., involves a
$\theta$. Consider a simplified version of the DnL rule shown below (see Section
\ref{sec:GTLC} for background):
%
\begin{mathpar}
  \inferrule*{c : A \ltdyn B \and M \ltdyn_i N : B}
             {\dnc{c}{M} \ltdyn_i N : c}
\end{mathpar}
%
Suppose that $c$ is $\iarr \colon \dyntodyn \ltdyn\, \dyn$, and recall the
definition of the semantics of a downcast given in Section
\ref{sec:term-interpretation}. When we downcast $M$ from $dyn$ to $\dyntodyn$,
semantically this will involve a $\theta$ because the value of type $D$ in the
model will be a \emph{later} function $\tilde{f}$.
%
Thus, in order for the downcast to be related to the term $N$ in the lock-step
error ordering on $\li (D \to \li D)$, the right-hand side must be of the form
$\theta(\dots)$, and moreover, after one time step, the argument to $\theta$ on
the LHS must be related to the argument of $\theta$ on the RHS. We can achieve
this by inserting a ``delay'' on the right, i.e., wrapping the RHS in $\delta =
\theta \circ \nxt$. Notice that this delay function is weakly bisimlar to the
identity, and thus it will disappear when we apply the bisimilarity-closure
construction described above. That is, this delay makes no difference in the
extensional setting, but its presence is crucial in the intensional setting.

% TODO Maybe show the correct form of the rule in the special case $c = inj-arr$.


\subsection{Extending the Model to Higher-Order Types}

% At this point, we could carry out the full construction of a concrete relational
% model with these additional features. 

So far, we have been focusing on closed terms of base type, for which the above
techniques give a semantic interpretation of term precision. We now consider how
to extend these constructions to \emph{all} types. In particular, in order to
model higher-order data types (i.e., functions) we need to equip \emph{every}
semantic object with not only a partial ordering relation but also a
``bisimilarity'' relation that is reflexive and symmetric. We must similarly
equip every object with a structured set of delays that can be inserted to
ensure the appropriate cast rules hold at all higher-order types. For example,
the upcast for a derivation $c \ra d$ involves the downcast corresponding to $c$
and an upcast corresponding to $d$. We must be able to insert delays in a
functorial manner to mimic the structure of the casts.

For the sake of reusability and modularity, rather than carry out this
construction in the concrete setting developed here, we will instead return to
the abstract setting and adjust our definition of model to account for these
requirements. We will break the construction into smaller steps and isolate the
pieces that require the techniques of SGDT from those that do not. Then with
this framework at our disposal, we will return to the construction of a concrete
model in Section \ref{sec:concrete-model}.
% , taking as a starting point the definitions introduced in the current section.

% Thus in the next section we define revised notions of a model of
% gradual typing based on the lessons learned in the present section. 
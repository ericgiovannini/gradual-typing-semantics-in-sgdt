\section{Discussion}

\subsection{Related Work}
% Discuss Joey Eremondi's thesis on gradual dependent types

% Discuss Jeremy Siek's work on graduality in Agda

\subsection{Mechanization}
% Discuss Guarded Cubical Agda and mechanization efforts
We will use Guarded Cubical Agda to prove graduality in the syntax of 
GTLC, which involves the construction of the abstract model described in 
\ref{sec:concrete-model} and the extensional model with external dynamic 
type. We also plan to formalize the adequacy result in \ref{sec:appendix-adequacy}.

% step-1
Currently, we have defined the step-1 intensional model in Cubical 
Agda. For value types, we constructed predomain which holds ordering and 
bisimilarity, and morphisms between predmains. For computation types, we 
constructed error domain which contains its underlying predomain, 
elements of $\mho$ and $\theta$ and morphisms between error domains. We 
also defined two functors $U$ and $F$ which support conversions between 
predomains and error domains.

% step-2 
Then we plan to construct the step-2 intensional model. Besides all the 
data in step-1, we need to include perturbations, functors $\times$, $\arr$, $U$, and $F$ that preserve 
perturbations and push/pull properties for all morphisms on value and 
computation types. Notice that for any object $A$ which has value type, 
we will take not only the monoid of perturbations $P^V_A$ and the monoid 
homomorphism $\ptbv_A : \pv_A \to \vf(A,A)$ on itself, but also $P^C_{F A}
$ and $\ptbe_{F A} : \pe_{F A} \to \ef(F A,F A)$ on $F A$, which have 
computation types. Similarly, for any computation object $B$, we will 
construct the perturbations on $U B$ besides the monoid $P^C_B$ and 
monoid homomorphism $\ptbe_B : \pe_B \to \ef(B,B)$. Also, for functors 
that preserves perturbations, we need to include the ones in the context 
of Kleisli category. For this part, we need to define the perturbation on 
not only the objects itself, but also the global lift and delay of objects, 
which requires us to provide each piece of supporting constructor. This step 
and futher steps towards to the model construction are still 
work-in-progress, but once it's finished, we will provide a complete 
framework which takes formalization on an explicit type and obtains an 
extensional model.

% step-3
In the step-3 intensional model, we will enhance it with 
quasi-representability. For any value relation $c : A \rel A'$, we need 
to show that there exists a left-representation structure for $c$ and a 
right-representation structure for $F\ c$. Correspondingly, for any 
computation relation $d : B \rel B'$, we will show there exists a 
right-representation structure for $d$ and a left-representation 
structure for $U\ d$. As we define the quasi-representability for value 
and computation relation, we will construct the quasi-representability on 
the function and product of the relation, which makes it necessary to 
have the dual version of quasi-representability.

% step-4 construct a concrete dynamic type and apply it to the abstract model
After defining the abstract model and its interface, we will model GTLC 
by providing explicit construction triples of dynamic type at each step, 
which includes defining Dyn as a predomain, its pure and Kleisli 
perturbation monoids, push/pull property for pure and Kleisli 
perturbation, as well as quasi-representability. The 
quasi-representability involves explicit rules which show that Nat is 
more precise than Dyn (Inj-Nat) and Dyn $\to$ Dyn is more precise than 
Dyn (Inj-Arr). Currently, we have formalized the concrete construction of 
Dyn in Cubical Agda and it was more challenging than expected because we 
define Dyn using the technique of guarded recursion and fixed point, which 
means that every time we analyze the case inside of Dyn, we need to unfold 
it and add corresponding proof. 

% adequacy
Besides the abstract model and its concrete construction on dynamic type, 
we will also formalize the adequacy result in \ref{sec:appendix-adequacy}, 
which involves clock quantification of the lift monad, the weak bisim 
relation, and the lock-step error ordering. In order to prove adequacy, 
we will first prove that the global lift of X is isomorphic to Delay(1 + X)
whether X is clock-irrelevant or not. Then, we aim to prove the equivalence 
between the global lock-step error ordering and the error ordering observed 
in Delay(1 + X) and equivalence between the global weak bisimilarity 
relation and the weak bisimilarity relation on Delay(1 + X). We have 
finished some prerequisite proofs on clock quantification and postulated 
some theorems on clock globalization.

\subsection{Benefits and Drawbacks of the Synthetic Approach}

\subsection{Synthetic Ordering}

A key to managing the complexity of our concrete construction is in
using a \emph{synthetic} approach to step-indexing rather than working
analytically with presheaves. This has helped immensely in our ongoing
mechanization in cubical Agda as it sidesteps the need to formalize
these constructions internally. 
%
However, there are other aspects of the model, the bisimilarity and
the monotonicity, which are treated analytically and are .
%
It may be possible to utilize further synthetic techniques to reduce
this burden as well, and have all type intrinsically carry a notion of
bisimilarity and ordering relation, and all constructions to
automatically preserve them.
%
A synthetic approach to ordering is common in (non-guarded) synthetic
domain theory and has also been used for synthetic reasoning for cost
models \cite{synthetic-domain-theory,decalf}.

\subsection{Future Work}

% Cite GrEff paper

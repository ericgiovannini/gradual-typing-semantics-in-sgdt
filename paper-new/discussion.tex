\section{Discussion}\label{sec:discussion}

\eric{I moved this from the intro. We may need to revise it.}

\subsection{Limitations of Prior Work}

We give an overview of current approaches to proving graduality of
languages and why they do not meet our criteria of a reusable semantic
framework.

\subsubsection{From Static to Gradual}

Current approaches to constructing languages that satisfy the
graduality property include the methods of Abstracting Gradual Typing
\cite{garcia-clark-tanter2016} and the formal tools of the Gradualizer
\cite{cimini-siek2016}.  These allow the language developer to start
with a statically typed language and derive a gradually typed language
that satisfies the graduality. The main downside to these
approaches lies in their inflexibility: since the process in entirely
mechanical, the language designer must adhere to the predefined
framework.  Many gradually typed languages do not fit into either
framework, e.g., Typed Racket \cite{tobin-hochstadt06,
  tobin-hochstadt08} and the semantics produced is not always the
desired one.
%
Furthermore, while these frameworks do prove graduality of the
resulting languages, they do not show the correctness of the
equational theory, which is equally important to sound gradual typing.

\subsubsection{Double Categorical Semantics}

New and Licata \cite{new-licata18} developed an axiomatic account of
the graduality relation on a call-by-name cast calculus terms and
showed that the graduality proof could be modeled using semantics in
certain kinds of \emph{double categories}, categories internal to the
category of categories. A double category extends a category with a
second notion of morphism, often a notion of ``relation'' to be paired
with the notion of functional morphism, as well as a notion of
functional morphisms preserving relations. In gradual typing the
notion of relation models type precision and the squares model the
term precision relation. This approach was influenced by the semantics
of parametricity using reflexive graph categories
\cite{ma-reynolds,dunphythesis,reynoldsprogramme}: reflexive graph
categories are essentially double categories without a notion of
relational composition. In addition to capturing the notions of type
and term precision, the double categorical approach allows for a
\emph{universal property} for casts: upcasts are the \emph{universal}
way to turn a relation arrow into a function in a forward direction
and downcasts are the dual universal arrow.  Later, New, Licata and
Ahmed \cite{new-licata-ahmed2019} extended this axiomatic treatment from
call-by-name to call-by-value as well by giving an axiomatic theory of
type and term precision in call-by-push-value.

%% With this notion of abstract categorical model in hand, denotational
%% semantics is then the work of constructing concrete models that
%% exhibit the categorical construction. New and Licata
%% \cite{new-licata18} present such a model using categories of
%% $\omega$-CPOs, and this model was extended by Lennon-Bertrand,
%% Maillard, Tabareau and Tanter to prove graduality of a gradual
%% dependently typed calculus $\textrm{CastCIC}^{\mathcal G}$. This
%% domain-theoretic approach meets our criteria of being a semantic
%% framework for proving graduality, but suffers from the limitations of
%% classical domain theory: the inability to model viciously
%% self-referential structures such as higher-order extensible state and
%% similar features such as runtime-extensible dynamic types. Since these
%% features are quite common in dynamically typed languages, we seek a
%% new denotational framework that can model these type system features.

The standard alternative to domain theory that scales to essentially
arbitrary self-referential definitions is \emph{step-indexing} or its
synthetic form of \emph{guarded recursion}. A series of works
\cite{new-ahmed2018, new-licata-ahmed2019, new-jamner-ahmed19}
developed step-indexed logical relations models of gradually typed
languages based on operational semantics. Unlike classical domain
theory, such step-indexed techniques are capable of modeling
higher-order store and runtime-extensible dynamic types
\cite{appelmcallester01,ahmed06,neis09,new-jamner-ahmed19}. However,
their proof developments are highly repetitive and technical, with
each development formulating a logical relation from first-principles
and proving many of the same tedious lemmas without reusable
mathematical abstractions. This is addressed somewhat by Siek and Chen
\cite{siek-chen2021}, who give a proof in Agda of graduality for an
operational semantics who use a \emph{guarded logic of propositions}
shallowly embedded in Agda. The guarded logic simplifies the treatment
of step-indexed logical relations, but the approach is still
fundamentally operational, and so the main lemmas of the work are
still tied to the particular operational syntactic calculus being
modeled. A further advantage of the denotational approach is that it
easily validates equational reasoning, not just graduality, and it is
completely independent of any particular syntax of gradual typing.

% Discuss Joey Eremondi's thesis on gradual dependent types
Eremondi \cite{Eremondi_2023} uses guarded type theory to
define a syntactic model for a gradually-typed source
language with dependent types. By working in guarded type theory, they are
able to give an exact semantics to non-terminating computations in a language
which is logically consistent, allowing for metatheoretic results about the
source language to be proven.
%
Similarly to our approach, they define a guarded lift monad to model potentially-
nonterminating computations and use guarded recursion to model the dynamic type.
However, they do not give a denotational semantics to term precision and it is unclear
how to prove the graduality in this setting.
The work includes a formalization of the syntactic model in Guarded Cubical Agda.

% Discuss Jeremy Siek's work on graduality in Agda

\subsection{Mechanization}
% Discuss Guarded Cubical Agda and mechanization efforts
In parallel with developing the theory discussed in this paper, we have
developed a partial formalization of our results in Guarded Cubical Agda
\cite{veltri-vezzosi2020}.
%
We have formalized the major components of the definition of the concrete model
described in Section \ref{sec:concrete-relational-model}: predomains, error
domains, their morphisms, relations, and squares, and we used the guarded
features to define the free error domain, the lock-step error ordering, and weak
bisimilarity. We also formalized the no-go theorem from Section
\ref{sec:concrete-term-model}.

Building on these definitions, we formalized the notion of semantic
perturbations and push-pull structures as well as quasi-representable relations,
culminating in the definition of value and computation types and relations as
introduced in Section \ref{sec:concrete-relational-model}.
%
% culminating in the definition of value and computation types as predomains
% (resp. error domains) equipped with a monoid of syntactic perturbations with an
% interpretation homomorphism into the semantic perturbations. 
%
We constructed the predomain for the dynamic type via mixed induction and
guarded recursion, and defined its monoid of perturbations. We also defined the
relations $\inat$, $\itimes$, and $\iarr$, and proved that they are
quasi-representable.
%
We have not yet formalized the constructions involving value and computation
relations: showing that these relations compose, and defining the actions of the
functors $\li$, $U$, $\times$, and $\arr$ on the relations. We have also not yet
formalized the syntax-to-semantics translation.

% \max{need to say what's not formalized}

% We plan to formalize the construction of perturbations and 
% quasi-representable relations, but we have yet to decide
% whether to follow the approach we take in this work and define
% the abstract notion of intensional model and formalize the constructions in that setting,
% and then apply those abstract constructions to the concrete model.
% Alternatively, it may be better from a mechanization standpoint
% to carry out those abstract constructions explicitly in the concrete model,
% i.e., our representation of objects in the concrete model of predomains
% would include a field for the perturbations and our notion of relations
% would include fields for the push-pull property and quasi-representability.
% We leave this investigation to future work.

Lastly, we have formalized the big-step term semantics discussed in Section
\ref{sec:big-step-term-semantics} and the adequacy of the relational model
discussed in Section \ref{sec:adequacy}. This required us to add axioms about
clock quantification as well as axioms asserting the \emph{clock-irrelevance} of
booleans and natural numbers since as of this writing these axioms are not
built-in to Guarded Cubical Agda. These axioms are discussed in prior work on
guarded type theory \cite{atkey-mcbride2013, kristensen-mogelberg-vezzosi2022}.
The mechanization of adequacy also required us to formalize some essential
lemmas involving clocks and clock-irrelevance; we are considering later
refactoring these as part of a ``standard library'' for Guarded Cubical
Agda.
% \max{say these axioms are taken from prior work and cite that}

\begin{comment}
% prove graduality in the syntax of 
% GTLC, which involves the construction of the abstract model described in 
% \ref{sec:concrete-model} and the extensional model with external dynamic 
% type. We also plan to formalize the adequacy result in \ref{sec:appendix-adequacy}.

% step-2 
Then we plan to construct the step-2 intensional model. Besides all the 
data in step-1, we need to include perturbations, functors $\times$, $\arr$, $U$, and $F$ that preserve 
perturbations and push/pull properties for all morphisms on value and 
computation types. Notice that for any object $A$ which has value type, 
we will take not only the monoid of perturbations $P^V_A$ and the monoid 
homomorphism $\ptbv_A : \pv_A \to \vf(A,A)$ on itself, but also $P^C_{F A}
$ and $\ptbe_{F A} : \pe_{F A} \to \ef(F A,F A)$ on $F A$, which have 
computation types. Similarly, for any computation object $B$, we will 
construct the perturbations on $U B$ besides the monoid $P^C_B$ and 
monoid homomorphism $\ptbe_B : \pe_B \to \ef(B,B)$. Also, for functors 
that preserves perturbations, we need to include the ones in the context 
of Kleisli category. For this part, we need to define the perturbation on 
not only the objects itself, but also the global lift and delay of objects, 
which requires us to provide each piece of supporting constructor. This step 
and futher steps towards to the model construction are still 
work-in-progress, but once it's finished, we will provide a complete 
framework which takes formalization on an explicit type and obtains an 
extensional model.

% step-3
In the step-3 intensional model, we will enhance it with 
quasi-representability. For any value relation $c : A \rel A'$, we need 
to show that there exists a left-representation structure for $c$ and a 
right-representation structure for $F\ c$. Correspondingly, for any 
computation relation $d : B \rel B'$, we will show there exists a 
right-representation structure for $d$ and a left-representation 
structure for $U\ d$. As we define the quasi-representability for value 
and computation relation, we will construct the quasi-representability on 
the function and product of the relation, which makes it necessary to 
have the dual version of quasi-representability.

% step-4 construct a concrete dynamic type and apply it to the abstract model
After defining the abstract model and its interface, we will model GTLC 
by providing explicit construction triples of dynamic type at each step, 
which includes defining Dyn as a predomain, its pure and Kleisli 
perturbation monoids, push/pull property for pure and Kleisli 
perturbation, as well as quasi-representability. The 
quasi-representability involves explicit rules which show that Nat is 
more precise than Dyn (Inj-Nat) and Dyn $\to$ Dyn is more precise than 
Dyn (Inj-Arr). Currently, we have formalized the concrete construction of 
Dyn in Cubical Agda and it was more challenging than expected because we 
define Dyn using the technique of guarded recursion and fixed point, which 
means that every time we analyze the case inside of Dyn, we need to unfold 
it and add corresponding proof. 

% adequacy
Besides the abstract model and its concrete construction on dynamic type, 
we will also formalize the adequacy result in \ref{sec:appendix-adequacy}, 
which involves clock quantification of the lift monad, the weak bisim 
relation, and the lock-step error ordering. In order to prove adequacy, 
we will first prove that the global lift of X is isomorphic to Delay(1 + X)
whether X is clock-irrelevant or not. Then, we aim to prove the equivalence 
between the global lock-step error ordering and the error ordering observed 
in Delay(1 + X) and equivalence between the global weak bisimilarity 
relation and the weak bisimilarity relation on Delay(1 + X). We have 
finished some prerequisite proofs on clock quantification and postulated 
some theorems on clock globalization.
\end{comment}

\subsection{Comparison to Embedding-projection pairs}

A line of work by New, Licata and Ahmed proves graduality by
interpreting type precision $c : A \ltdyn A'$ as an
\emph{embedding-projection pair}, that is a pure function $e : A \to
A'$ and a possibly erroring projection $p : \li A' \multimap \li A$
such that $p \circ e = \id$ and $e \circ p \ltdyn \id$
\cite{newahmed,newlicata,newlicataahmed}. At first glance, our model
of precision looks somewhat different: it is a \emph{relation}
$\sem{c} : A \relto A'$ with a quasi-representability structure that
gives us something like $e$ and $p$ in an embedding-projection
pair. The relationship between these models is that \emph{if} the
relation were \emph{truly} representable rather than only
quasi-representable we would have that $e$ and $p$ form a \emph{Galois
connection} $e \circ p \ltdyn \id$ and $\id \ltdyn p \circ
e$. However, we have dropped the stronger property of retraction from
our analysis in this work. With true representability, the relation
$c$ is \emph{uniquely determined} by the embedding $e$, but since we
only have quasi-representability, we need to keep the relation around
explicitly. So the extra complexity of managing explicit relations is
a cost of the intensional reasoning that guarded type theory
introduces.

\subsection{Comparison to Logical Relations Models}
Working internally to guarded type theory reduces the overhead of
needing to carry around the step-indices in the proofs as is required
when using explicit step-indexing. Additionally, the logical relations
constructed to prove graduality in prior work \cite{new-ahmed2018,new-licata-ahmed2019,new-giovannini-licata-2022} suffer
from technical complications of requiring two separate expression
relations, one that counts steps on the left and the other on the
right, and there is no analogue of this in our approach. However,
using two expression relations allows some but not all transitive
reasoning of term precision to be recovered. In the future we aim to
explore if this approach is feasible in guarded semantics.

\subsection{Synthetic Ordering}
\max{cut this subsection if we need space}
A key to managing the complexity of our concrete construction is in
using a \emph{synthetic} approach to step-indexing rather than working
analytically with presheaves. This has helped immensely in our ongoing
mechanization in cubical Agda as it sidesteps the need to formalize
these constructions internally. 
%
However, there are other aspects of the model, the bisimilarity and
the monotonicity, which are treated analytically and are similarly
tedious.
%
It may be possible to utilize further synthetic techniques to reduce
this burden as well, and have all type intrinsically carry a notion of
bisimilarity and ordering relation, and all constructions to
automatically preserve them.
%
A synthetic approach to ordering is common in (non-guarded) synthetic
domain theory and has also been used for synthetic reasoning for cost
models \cite{fiore_1997,GrodinNSH24}.

\subsection{Future Work}

\max{Expand on this, e.g. what some of the challenges might be and what would be reusable}
In the future, we plan to apply our approach to give a denotational
semantics for languages that feature higher-order state or
runtime-extensible dynamic typing
\cite{DBLP:journals/corr/abs-2210-02169} as well as richer type
disciplines such as gradual dependent types and effect systems.

%% to gradually-typed
%% languages with algebraic effects, building on prior work on gradual typing for effect handlers
%% \cite{greff}. In particular, that work proves graduality via a complicated step-indexed logical relation,
%% and we hope to prove their results by building a denotational model for GrEff.
%% This would serve as a step towards applying our techniques to prove graduality for languages
%% with other advanced features.

%% The extensional model we construct differs from the usual notion of extensional
%% model considered in prior work on gradual typing in that it lacks horizontal composition of squares.
%% We would like to clarify the relationship between our notion of model and prior extensional models,
%% with the aim of determining whether our approach could allow for the construction of such a model.

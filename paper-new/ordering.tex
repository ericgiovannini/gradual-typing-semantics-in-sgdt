\section{Extending the Semantics to Precision}\label{sec:gtlc-precision}

In this section, we extend the set-theoretic semantics for terms given in
the previous section to a semantics for the type and term precision relations
of the gradually-typed lambda calculus. We first introduce the type and term precision
relations, then show how to give them a semantics using SGDT.

% TODO mention intensional syntax


\subsection{Term Precision for GTLC}\label{sec:gtlc-term-precision-axioms}

% ---------------------------------------------------------------------------------------
% ---------------------------------------------------------------------------------------

%\subsubsection{Term Precision}\label{sec:term-precision}

We allow for a \emph{heterogeneous} term precision judgment on values $V$ of type
$A$ and $V'$ of type $A'$ provided that $A \ltdyn A'$ holds. Likewise, for producers,
if $M$ has type $A$ and $M'$ has type $A'$, we can form the judgment that $M \ltdyn M'$.
We use the same notation for the precision relation on both values and producers.

% Type precision contexts
In order to deal with open terms, we will need the notion of a type precision \emph{context}, which we denote
$\gamlt$. This is similar to a normal context but instead of mapping variables to types,
it maps variables $x$ to related types $A \ltdyn A'$, where $x$ has type $A$ in the left-hand term
and $A'$ in the right-hand term. We may also write $x : d$ where $d : A \ltdyn A'$ to indicate this.

% An equivalent way of thinking of type precision contexts is as a pair of ``normal" typing
% contexts $\Gamma, \Gamma'$ with the same domain such that $\Gamma(x) \ltdyn \Gamma'(x)$ for
% each $x$ in the domain.
% We will write $\gamlt : \Gamma \ltdyn \Gamma'$ when we want to emphasize the pair of contexts.
% Conversely, if we are given $\gamlt$, we write $\gamlt_l$ and $\gamlt_r$ for the normal typing contexts on each side.

An equivalent way of thinking of a type precision context $\gamlt$ is as a
pair of ``normal" typing contexts, $\gamlt_l$ and $\gamlt_r$, with the same
domain and such that $\gamlt_l(x) \ltdyn \gamlt_r(x)$ for each $x$ in the domain.
We will write $\gamlt : \gamlt_l \ltdyn \gamlt_r$ when we want to emphasize the pair of contexts.

As with type precision derivations, we write $\Gamma$ to mean the ``reflexivity" type precision context
$\Gamma : \Gamma \ltdyn \Gamma$.
Concretely, this consists of reflexivity type precision derivations $\Gamma(x) \ltdyn \Gamma(x)$ for
each $x$ in the domain of $\Gamma$.

Furthermore, we write $\gamlt_1 \relcomp \gamlt_2$ to denote the ``composition'' of $\gamlt_1$ and $\gamlt_2$
--- that is, the precision context whose value at $x$ is the type precision derivation
$\gamlt_1(x) \relcomp \gamlt_2(x)$. This of course assumes that each of the type precision
derivations is composable, i.e., that the RHS of $\gamlt_1(x)$ is the same as the left-hand side of $\gamlt_2(x)$.

% We define the same for computation type precision contexts $\deltalt_1$ and $\deltalt_2$,
% provided that both the computation type precision contexts have the same ``shape'', which is defined as
% (1) either the stoup is empty in both, or the stoup has a hole in both, say $\hole{d}$ and $\hole{d'}$
% where $d$ and $d'$ are composable, and (2) their value type precision contexts are composable as described above.

The rules for term precision come in two forms. We first have the \emph{congruence} rules,
one for each term constructor. These assert that the term constructors respect term precision.
The congruence rules are shown in Figure \ref{fig:term-prec-congruence-rules}.

\begin{figure*}
\begin{mathpar}

  \inferrule*[right = Var]
    { c : A \ltdyn B \and \gamlt(x) = (A, B) } 
    { \etmprec {\gamlt} x x c }

  \inferrule*[right = Zro]
    { } {\etmprec \gamlt \zro \zro \nat }

  \inferrule*[right = Suc]
    { \etmprec \gamlt V {V'} \nat } {\etmprec \gamlt {\suc\, V} {\suc\, V'} \nat}

  \inferrule*[right = MatchNat]
  {\etmprec \gamlt V {V'} \nat \and 
    \etmprec \deltalt M {M'} d \and \etmprec {\deltalt, n : \nat} {N} {N'} d}
  {\etmprec \deltalt {\matchnat {V} {M} {n} {N}} {\matchnat {V'} {M'} {n} {N'}} d}

  \inferrule*[right = Lambda]
    { c_i : A_i \ltdyn A'_i \and 
      c_o : A_o \ltdyn A'_o \and 
      \etmprec {\gamlt, x : c_i} {M} {M'} {c_o} } 
    { \etmprec \gamlt {\lda x M} {\lda x {M'}} {(c_i \ra c_o)} }

  \inferrule*[right = App]
    { c_i : A_i \ltdyn A'_i \and
      c_o : A_o \ltdyn A'_o \\\\
      \etmprec \gamlt {V_f} {V_f'} {(c_i \ra c_o)} \and
      \etmprec \gamlt {V_x} {V_x'} {c_i}
    } 
    { \etmprec {\gamlt} {V_f\, V_x} {V_f'\, V_x'} {{c_o}}}

  \inferrule*[right = Ret]
    {\etmprec {\gamlt} V {V'} c}
    {\etmprec {\gamlt} {\ret\, V} {\ret\, V'} {c}}

  \inferrule*[right = Bind]
    {\etmprec {\gamlt} {M} {M'} {c} \and 
     \etmprec {\gamlt, x : c} {N} {N'} {d} }
    {\etmprec {\gamlt} {\bind {x} {M} {N}} {\bind {x} {M'} {N'}} {d}}
\end{mathpar}
\caption{Term precision: congruence rules.}
\label{fig:term-prec-congruence-rules}
\end{figure*}

We then have additional equational axioms as shown in Figure \ref{fig:term-prec-additional-rules}.
In particular, we have a rule stating that the error term is the bottom element,
$\beta$ and $\eta$ laws as order-equivalences, and lastly rules characterizing upcasts
as least upper bounds, and downcasts as greatest lower bounds.
% For the sake of familiarity, we formulate the cast rules using arbitrary casts; later we
% will state the analogous versions for the version of the calculus without arbitrary casts.

We write $M \equidyn N$ to mean that both $M \ltdyn N$ and $N \ltdyn M$.

\begin{figure*}
\begin{mathpar}
  \inferrule*[right = $\err$]
    {\phasty {\Gamma} M B }
    {\etmprec {\Gamma} {\err_B} M B}

  \inferrule*[right = $\beta$-fun]
    { \phasty {\Gamma, x : A_i} M {A_o} \and
      \vhasty {\Gamma} V {A_i} } 
    { \etmequidyn {\Gamma} {(\lda x M)\, V} {M[V/x]} {A_o} }

  \inferrule*[right = $\eta$-fun]
    { \vhasty {\Gamma} {V} {A_i \ra A_o} } 
    { \etmequidyn \Gamma {\lda x (V\, x)} V {A_i \ra A_o} }

  \inferrule*[right = UpR]
    { c : A \ltdyn B \and d : B \ltdyn C \and 
      \etmprec {\gamlt} {M} {N} {c} } 
    { \etmprec {\gamlt} {M} {\up {B} {C} N} {c \circ d}  }

  \inferrule*[right = UpL]
    { c : A \ltdyn B \and d : B \ltdyn C \and
      \etmprec {\gamlt} {M} {N} {c \circ d} } 
    { \etmprec {\gamlt} {\up {A} {B} M} {N} {d} }

  \inferrule*[right = DnL]
    { c : A \ltdyn B \and d : B \ltdyn C \and
      \etmprec {\gamlt} {M} {N} {d} } 
    { \etmprec {\gamlt} {\dn {A} {B} M} {N} {c \circ d} }

  \inferrule*[right = DnR]
    { c : A \ltdyn B \and d : B \ltdyn C \and
      \etmprec {\gamlt} {M} {N} {c \circ d} } 
    { \etmprec {\gamlt} {M} {\dn {B} {C} N} {c} }
\end{mathpar}
\caption{Term precision: additional rules.}
\label{fig:term-prec-additional-rules}
\end{figure*}

% TODO explain the least upper bound/greatest lower bound rules
The rules \textsc{UpR}, \textsc{UpL}, \textsc{DnL}, and \textsc{DnR}
were introduced in \cite{new-licata18} as a means
of cleanly axiomatizing the intended behavior of casts in a way that
doesn't depend on the specific constructs of the language.
Intuitively, rule \textsc{UpR} says that the upcast of $M$ is an upper bound for $M$
in that $M$ may error more, and \textsc{UpL} says that the upcast is the \emph{least}
such upper bound, in that it errors more than any other upper bound for $M$.
Conversely, \textsc{DnL} says that the downcast of $M$ is a lower bound, and \textsc{DnR} says
that it is the \emph{greatest} lower bound.
% These rules provide a clean axiomatization of the behavior of casts that doesn't
% depend on the specific constructs of the language.



\subsection{Semantics for Precision}

As a first attempt at giving a semantics to the ordering, we could try to model types as
sets equipped with an ordering that models term precision. Since term precision is reflexive
and transitive, and since we identify terms that are equi-precise, we choose to model types
as partially-ordered sets. We model the term precision ordering $M \ltdyn N : A \ltdyn B$ as an
ordering relation between the posets denoted by $A$ and $B$.

However, it turns out that modeling term precision by a relation defined by guarded fixpoint
is not as straightforward as one might hope.
A first attempt might be to define an ordering $\semltbad$ between $\li X$ and $\li Y$
that allows for computations that may take different numbers of steps to be related.
The relation is parameterized by a relation $\le$ between $X$ and $Y$, and is defined
by guarded fixpoint as follows:
% simultaneously captures the notions of error approximation and equivalence up to stepping behavior:

\begin{align*}
  &\eta\, x \semltbad \eta\, y \text{ if } 
    x \semlt y \\
%		
  &\mho \semltbad l \\
%
  &\theta\, \tilde{l} \semltbad \theta\, \tilde{l'} \text{ if } 
    \later_t (\tilde{l}_t \semltbad \tilde{l'}_t) \\
%	
  &\theta\, \tilde{l} \semltbad \mho \text{ if } 
    \theta\, \tilde{l} = \delta^n(\mho) \text { for some $n$ } \\
%	
  &\theta\, \tilde{l} \semltbad \eta\, y \text{ if }
    (\theta\, \tilde{l} = \delta^n(\eta\, x))
  \text { for some $n$ and $x : \ty{X}$ such that $x \le y$ } \\
%
  &\mho \semltbad \theta\, \tilde{l'} \text { if } 
    \theta\, \tilde{l'} = \delta^n(\mho) \text { for some $n$ } \\
%	
  &\eta\, x \semltbad \theta\, \tilde{l'} \text { if }
    (\theta\, \tilde{l'} = \delta^n (\eta\, y))
  \text { for some $n$ and $y : \ty{Y}$ such that $x \le y$ }
\end{align*}

Two computations that immediately return $(\eta)$ are related if the underlying
values are related in the underlying ordering. 
%
The computation that errors $(\mho)$ is below everything else.
%
If both sides step (i.e., both sides are $\theta$),
then we allow one time step to pass and compare the resulting terms.
(This is where use the relation defined ``later''.)
%
Lastly, if one side steps and the other returns a value, the side that steps should
terminate with a value in some finite number of steps $n$, and that value should
be related to the value returned by the other side.
Likewise, if one side steps and the other errors, then the side that steps
should terminate with error.

The problem with this definition is that the resulting relation is \emph{provably} not
transitive: it can be shown (in Clocked Cubical Type Theory) that if $R$ is a
relation on $\li X$ satisfying three specific properties, one of which is
transitivity, then that relation is trivial.
(The other two properties are that the relation is a congruence with respect to $\theta$,
and that the relation is closed under delays $\delta = \theta \circ \nxt$ on either side.)
Since the above relation \emph{does} satisfy the other two properties, we conclude
that it must not be transitive.

%But having a non-transitive relation to model term precision presents a problem
%for...

We are therefore led to wonder whether we can formulate a version of the relation
that \emph{is} transitive.
It turns out that we can, by sacrificing another of the three properties from
the above lemma. Namely, we give up on closure under delays. Doing so, we end up
with a \emph{lock-step} error ordering, where, roughly speaking, in order for
computations to be related, they must have the same stepping behavior.
%
We then formulate a separate relation, \emph{weak bisimilarity}, that relates computations
that are extensionally equal and may only differ in their stepping behavior.

% As a result, we instead separate the semantics of term precision into two relations:
% an intensional, step-sensitive \emph{error ordering} and a \emph{bisimilarity relation}.


\subsubsection{Double Posets}\label{sec:predomains}

As discussed above, there are two relations that we would like to define
in the semantics: a step-sensitive error ordering, and weak bisimilarity of computations.
%
The semantic objects that interpret our types should therefore be equipped with
two relations. We call these objects ``double posets''.
A double poset $A$ is a set with two relations: an partial order $\semlt_A$ on $A$, and
a reflexive, symmetric relation $\bisim_A$ on $A$.
We write the underling set of $A$ as $\ty{A}$.

We define morphisms of double posets as functions that preserve both
the ordering and the bisimilarity relation. Given double posets
$A$ and $B$, we write $f \colon A \monto B$ to indicate that $f$ is a morphism
from $A$ to $B$, i.e, the following hold:
(1) for all $a_1 \semlt_A a_2$, we have $f(a_1) \semlt_{B} f(a_2)$, and
(2) for all $a_1 \bisim_A a_2$, we have $f(a_1) \bisim_{B} f(a_2)$.


%%%%% RESUME HERE

We define an ordering on morphisms of double posets as
$f \le g$ if for all $a$ in $\ty{A_i}$, we have $f(a) \le_{A_o} g(a)$,
and similarly bisimilarity extends to morphisms via
$f \bisim g$ if for all $a$ in $\ty{A_i}$, we have $f(a) \bisim_{A_o} g(a)$.

For each type we want to represent, we define a double poset for the corresponding semantic
type. For instance, we define a double poset for natural numbers, for the
dynamic type, for functions, and for the lift monad. We
describe each of these below.

\begin{itemize}
  \item There is a double poset $\Nat$ for natural numbers, where the ordering and the
  bisimilarity relations are both equality.
  
  % TODO explain that there is a theta operator for posets?
  \item There is a double poset $\Dyn$ to represent the dynamic type. The underlying type
  for this double poset is defined by guarded fixpoint to be such that
  $\ty{\Dyn} \cong \mathbb{N}\, +\, \later (\ty{\Dyn} \monto \li \ty{\Dyn})$.
  This definition is valid because the occurrences of Dyn are guarded by the $\later$.
  The ordering is defined via guarded recursion by cases on the argument, using the
  ordering on $\mathbb{N}$ and the ordering on monotone functions described above.

  \item For a double poset $A$, there is a double poset $\li A$ for the ``lift'' of $A$
  using the lift monad. We use the same notation for $\li A$ when $A$ is a type
  and $A$ is a double poset, since the context should make clear which one we are referring to.
  The underling type of $\li A$ is simply $\li \ty{A}$, i.e., the lift of the underlying
  type of $A$.
  The ordering on $\li A$ is the ``lock-step error-ordering'' which we describe in
  \ref{subsec:lock-step}. The bismilarity relation is the ``weak bisimilarity''
  described in Section \ref{}

  \item For double posets $A_i$ and $A_o$, we form the double poset of monotone functions
  from $A_i$ to $A_o$, which we denote by $A_i \To A_o$.

  \item Given a double poset $A$, we can form the double poset $\later A$ whose underlying
  type is $\later \ty{A}$. We define $\tilde{x} \le_{\later A} \tilde{y}$ to be
  $\later_t (\tilde{x}_t \le_A \tilde{y}_t)$.
\end{itemize}

\subsubsection{Step-Sensitive Error Ordering}\label{subsec:lock-step}

As mentioned, the ordering on the lift of a double poset $A$ is called the
\emph{step-sensitive error-ordering} (also called ``lock-step error ordering''),
the idea being that two computations $l$ and $l'$ are related if they are in
lock-step with regard to their intensional behavior, up to $l$ erroring.
Formally, we define this ordering as follows:

\begin{itemize}
  \item 	$\eta\, x \ltls \eta\, y$ if $x \le_A y$.
  \item 	$\mho \ltls l$ for all $l$ 
  \item   $\theta\, \tilde{r} \ltls \theta\, \tilde{r'}$ if
          $\later_t (\tilde{r}_t \ltls \tilde{r'}_t)$
\end{itemize}

We also define a heterogeneous version of this ordering between the lifts of two
different double posets $A$ and $B$, parameterized by a relation $R$ between $A$ and $B$.

\subsubsection{Weak Bisimilarity Relation}

For a double poset $A$, we define a relation on $\li A$, called ``weak bisimilarity",
written $l \bisim l'$. Intuitively, we say $l \bisim l'$ if they are equivalent
``up to delays''.
% We introduce the notation $x \sim_A y$ to mean $x \le_A y$ and $y \le_A x$.
% TODO if A is a poset, then we can just say that x = y
%
The weak bisimilarity relation is defined by guarded fixpoint as follows:

\begin{align*}
  &\mho \bisim \mho \\
%
  &\eta\, x \bisim \eta\, y \text{ if } 
    x \bisim_A y \\
%		
  &\theta\, \tilde{x} \bisim \theta\, \tilde{y} \text{ if } 
    \later_t (\tilde{x}_t \bisim \tilde{y}_t) \\
%	
  &\theta\, \tilde{x} \bisim \mho \text{ if } 
    \theta\, \tilde{x} = \delta^n(\mho) \text { for some $n$ } \\
%	
  &\theta\, \tilde{x} \bisim \eta\, y \text{ if }
    (\theta\, \tilde{x} = \delta^n(\eta\, x))
  \text { for some $n$ and $x : \ty{A}$ such that $x \sim_A y$ } \\
%
  &\mho \bisim \theta\, \tilde{y} \text { if } 
    \theta\, \tilde{y} = \delta^n(\mho) \text { for some $n$ } \\
%	
  &\eta\, x \bisim \theta\, \tilde{y} \text { if }
    (\theta\, \tilde{y} = \delta^n (\eta\, y))
  \text { for some $n$ and $y : \ty{A}$ such that $x \sim_A y$ }
\end{align*}

When both sides are $\eta$, then we ensure that the underlying values are bisimilar
in the underlying bisimilarity relation on $A$.
When one side is a $\theta$ and the other is $\eta x$ (i.e., one side steps),
we stipulate that the $\theta$-term runs to $\eta y$ where $x$ is related to $y$.
Similarly when one side is $\theta$ and the other $\mho$.
If both sides step, then we allow one time step to pass and compare the resulting terms.
In this way, the definition captures the intuition of terms being equivalent up to
delays.

It can be shown (by \lob-induction) that the step-sensitive relation is symmetric.
However, it can also be shown that this relation is \emph{not} transitive:
The argument is the same as that used to show that the step-insensitive error
ordering $\semltbad$ described above is not transitive. Namely, we show that
if it were transitive, then it would have to be trivial in that $l \bisim l'$ for all $l, l'$.
that if this relation were transitive, then in fact it would be trivial in that
%This issue will be resolved when we consider the relation's \emph{globalization}.

\subsection{The Cast Rules}

Unfortunately, the four cast rules defined above do not hold in
the intensional setting where we are tracking the steps taken by terms.
The source of the problem is that the downcast from the dynamic type to
a function involves a delay, i.e., a $\theta$.
So in order to keep the other term in lock-step, we need to insert a ``delay"
that is extensionally equivalent to the identity function.
More concretely, consider a simplified version of the DnL rule shown below:

\begin{mathpar}
  \inferrule*{M \ltdyn_i N : B}
             {\dnc{c}{M} \ltdyn_i N : c}
\end{mathpar}

If $c$ is inj-arr, then when we downcast $M$ from $dyn$ to $\dyntodyn$,
semantically this will involve a $\theta$ because the value of type $dyn$
in the semantics will contain a \emph{later} function $\tilde{f}$.
Thus, in order for the right-hand side to be related to the downcast,
the right-hand side must be of the form $\theta(\dots)$, and the argument to
$\theta$ on the LHS must be related to the argument of $\theta$ on the RHS.
We can accomplish this by inserting a ``delay" on the right, i.e., wrapping the
RHS in $\delta = \theta \circ \nxt$.
% Maybe show the correct form of the rule in the special case $c = inj-arr$.
%
The need for delays affects the cast rules involving upcasts as well, because
the upcast for functions involves a downcast on the domain:

\[ \up{A_i \ra A_o}{B_i \ra B_o}{M} \equiv \lambda (x : B_i). \up{A_o}{B_o}(M\, (\dn {A_i}{B_i} x)). \]

Thus, the correct versions of the cast rules involve delays on the side that was not casted.

The specific manner in which these delays must be inserted for a given type precision
derivation $c$ is best examined in a more abstract setting, as it is independent of the
particular concrete model under consideration.
Thus, we postpone this discussion to the section on the abstract intensional categorical models of
gradual typing (Section \ref{sec:abstract-intensional-models}).


\begin{comment}
\subsubsection{Perturbations}

We can describe precisely how the delays are inserted for any type precision
derivation $c$.

To do so, we first define simultaneously an inductive type of \emph{perturbations}
for embeddings $\perte$ and for projections $\pertp$ by the following rules:

\begin{mathpar}

\inferrule{}{\id : \perte A}

\inferrule{}{\id : \pertp A}

\inferrule
  {\delta_c : \pertp A \and \delta_d : \perte B}
  {\delta_c \ra \delta_d : \perte (A \ra B)}

\inferrule
  {\delta_c : \perte A \and \delta_d : \pertp B}
  {\delta_c \ra \delta_d : \pertp (A \ra B)}

\inferrule
  {\delta_\nat : \perte \nat \and \delta_f : \perte (\dyntodyn)}
  {\pertdyn{\delta_\nat}{\delta_f} : \perte \dyn}

\inferrule
  {\delta_\nat : \pertp \nat \and \delta_f : \pertp (\dyntodyn)}
  {\pertdyn{\delta_\nat}{\delta_f} : \pertp \dyn}

\end{mathpar}

The structure of embedding perturbations is designed to follow the structure
of the corresponding embeddings, and likewise for the projection perturbations.
Thus, in the function case, an embedding perturbation consists of a \emph{projection}
perturbation for the domain and an \emph{embedding} perturbation for the codomain.
The opposite holds for the projection perturbation for functions.

Another way in which the two kinds of perturbations differ is that there is an additional
projection perturbation for delaying $\delaypert{\delta}$.
This corresponds to the actual delay term $\delta = \theta \circ \nxt$ in the semantics,
and it is the generator/source of all non-trivial perturbations.

Given a perturbation $\delta$, we can turn it into a term, which we also write as
$\delta$ unless there is opportunity for confusion.

% Delays for function types and for inj-arr(c)

\end{comment}

\section{Denotational Semantics}

First, we define a denotational semantics of types and terms of the
cast calculus by giving a standard monadic denotational semantics in
the cartesian closed category of preorders and monotone functions,
extended to model the primitives of gradual typing: the dynamic type,
errors and type casts. The most interesting part of this semantics is
the construction of the monad and the dynamic type.



\section{Domain-Theoretic Constructions}\label{sec:domain-theory}

In this section, we discuss the fundamental objects of the model into which we will embed
the step-sensitive lambda calculus $\intlc$ and inequational theory. It is important to remember that
the constructions in this section are entirely independent of the syntax described in the
previous section; the notions defined here exist in their own right as purely mathematical
constructs. In the next section, we will link the syntax and semantics via a semantic interpretation
function.

\subsection{The Lift Monad}

When thinking about how to model intensional gradually-typed programs, we should consider
their possible behaviors. On the one hand, we have \emph{failure}: a program may fail
at run-time because of a type error. In addition to this, a program may ``think'',
i.e., take a step of computation. If a program thinks forever, then it never returns a value,
so we can think of the idea of thinking as a way of intensionally modelling \emph{partiality}.

With this in mind, we can describe a semantic object that models these behaviors: a monad
for embedding computations that has cases for failure and ``thinking''.
Previous work has studied such a construct in the setting of the latter, called the lift
monad \cite{mogelberg-paviotti2016}; here, we augment it with the additional effect of failure.

For a type $A$, we define the \emph{lift monad with failure} $\li A$, which we will just call
the \emph{lift monad}, as the following datatype:

\begin{align*}
  \li A &:= \\
  &\eta \colon A \to \li A \\
  &\mho \colon \li A \\
  &\theta \colon \later (\li A) \to \li A
\end{align*}

Unless otherwise mentioned, all constructs involving $\later$ or $\fix$
are understood to be with respect to a fixed clock $k$. So for the above, we really have for each
clock $k$ a type $\li^k A$ with respect to that clock.

Formally, the lift monad $\li A$ is defined as the solution to the guarded recursive type equation

\[ \li A \cong A + 1 + \later \li A. \]

An element of $\li A$ should be viewed as a computation that can either (1) return a value (via $\eta$),
(2) raise an error and stop (via $\mho$), or (3) think for a step (via $\theta$).
%
Notice there is a computation $\fix \theta$ of type $\li A$. This represents a computation
that thinks forever and never returns a value.

Since we claimed that $\li A$ is a monad, we need to define the monadic operations
and show that they respect the monadic laws. The return is just $\eta$, and extend
is defined via by guarded recursion by cases on the input.
% It is instructive to give at least one example of a use of guarded recursion, so
% we show below how to define extend:
% TODO
%
%
Verifying that the monadic laws hold requires \lob-induction and is straightforward.

\begin{comment}
% Since this mentions the ordering on B, it should be introduced after introducing predomains.
The lift monad has the following universal property. Let $f$ be a function from $A$ to $B$,
where $B$ is a $\later$-algebra, i.e., there is $\theta_B \colon \later B \to B$.
Further suppose that $B$ is also an ``error-algebra'', that is, there is an error element
$\mho_B$ such that $\mho_B \le_B y$ for all $y \in B$.

% Further suppose that $B$ is also an algebra of the
% constant functor $1 \colon \text{Type} \to \text{Type}$ mapping all types to Unit.
% This latter statement amounts to saying that there is a map $\text{Unit} \to B$, so $B$ has a
% distinguished ``error element" $\mho_B \colon B$ such that $\mho_B \le_B y$ for all $y \in B$.

Then there is a unique homomorphism of algebras $f' \colon \li A \to B$ such that
$f' \circ \eta = f$. The function $f'(l)$ is defined via guarded fixpoint by cases on $l$. 
In the $\mho$ case, we simply return $\mho_B$.
In the $\theta(\tilde{l})$ case, we will return

\[\theta_B (\lambda t . (f'_t \, \tilde{l}_t)). \]

Recalling that $f'$ is a guarded fixpoint, it is available ``later'' and by
applying the tick we get a function we can apply ``now''; for the argument,
we apply the tick to $\tilde{l}$ to get a term of type $\li A$.
\end{comment}

%\subsubsection{Model-Theoretic Description}
%We can describe the lift monad in the topos of trees model as follows.


\subsection{Predomains}\label{sec:predomains}

The next important construction is that of a \emph{predomain}. A predomain is intended to
model the notion of error ordering that we want terms to have. Thus, we define a predomain $A$
as a partially-ordered set, which consists of a type which we denote $\ty{A}$ and a reflexive,
transitive, and antisymmetric relation $\le_P$ on $A$.

We define monotone functions between predomain as expected. Given predomains
$A$ and $B$, we write $f \colon A_i \monto A_o$ to indicate that $f$ is a monotone
function from $A_i$ to $A_o$, i.e, for all $a_1 \le_{A_i} a_2$, we have $f(a_1) \le_{A_o} f(a_2)$.
We also define an ordering on monotone functions as
$f \le g$ if for all $a$ in $\ty{A_i}$, we have $f(a) \le_{A_o} g(a)$.

For each type we want to represent, we define a predomain for the corresponding semantic
type. For instance, we define a predomain for natural numbers, a predomain for the
dynamic type, a predomain for functions, and a predomain for the lift monad. We
describe each of these below.

\begin{itemize}
  \item There is a predomain $\Nat$ for natural numbers, where the ordering is equality.
  
  \item There is a predomain $\Dyn$ to represent the dynamic type. The underlying type
  for this predomain is defined by guarded fixpoint to be such that
  $\ty{\Dyn} \cong \mathbb{N}\, +\, \later (\ty{\Dyn} \monto \li \ty{\Dyn})$.
  This definition is valid because the occurrences of Dyn are guarded by the $\later$.
  The ordering is defined via guarded recursion by cases on the argument, using the
  ordering on $\mathbb{N}$ and the ordering on monotone functions described above.

  \item For a predomain $A$, there is a predomain $\li A$ for the ``lift'' of $A$
  using the lift monad. We use the same notation for $\li A$ when $A$ is a type
  and $A$ is a predomain, since the context should make clear which one we are referring to.
  The underling type of $\li A$ is simply $\li \ty{A}$, i.e., the lift of the underlying
  type of $A$.
  The ordering on $\li A$ is the ``step-sensitive error-ordering'' which we describe in
  \ref{subsec:lock-step}.

  \item For predomains $A_i$ and $A_o$, we form the predomain of monotone functions
  from $A_i$ to $A_o$, which we denote by $A_i \To A_o$.

  \item Given a preomain $A$, we can form the predomain $\later A$ whose underlying
  type is $\later \ty{A}$. We define $\tilde{x} \le_{\later A} \tilde{y}$ to be
  $\later_t (\tilde{x}_t \le_A \tilde{y}_t)$.
\end{itemize}



\subsection{Step-Sensitive Error Ordering}\label{subsec:lock-step}

As mentioned, the ordering on the lift of a predomain $A$ is called the
\emph{step-sensitive error-ordering} (also called ``lock-step error ordering''),
the idea being that two computations $l$ and $l'$ are related if they are in
lock-step with regard to their intensional behavior, up to $l$ erroring.
Formally, we define this ordering as follows:

\begin{itemize}
  \item 	$\eta\, x \ltls \eta\, y$ if $x \le_A y$.
  \item 	$\mho \ltls l$ for all $l$ 
  \item   $\theta\, \tilde{r} \ltls \theta\, \tilde{r'}$ if
          $\later_t (\tilde{r}_t \ltls \tilde{r'}_t)$
\end{itemize}

We also define a heterogeneous version of this ordering between the lifts of two
different predomains $A$ and $B$, parameterized by a relation $R$ between $A$ and $B$.

\subsection{Step-Insensitive Bisimilarity Relation}

We define another ordering on $\li A$, called ``step-insensitive bisimilarity"
or ``weak bisimilarity" written $l \bisim l'$.
Intuitively, we say $l \bisim l'$ if they are equivalent ``up to delays''.
We introduce the notation $x \sim_A y$ to mean $x \le_A y$ and $y \le_A x$.
% TODO if A is a poset, then we can just say that x = y
%
The step-insensitive bisimilarity relation is defined by guarded fixpoint as follows:

\begin{align*}
  &\mho \bisim \mho \\
%
  &\eta\, x \bisim \eta\, y \text{ if } 
    x \sim_A y \\
%		
  &\theta\, \tilde{x} \bisim \theta\, \tilde{y} \text{ if } 
    \later_t (\tilde{x}_t \bisim \tilde{y}_t) \\
%	
  &\theta\, \tilde{x} \bisim \mho \text{ if } 
    \theta\, \tilde{x} = \delta^n(\mho) \text { for some $n$ } \\
%	
  &\theta\, \tilde{x} \bisim \eta\, y \text{ if }
    (\theta\, \tilde{x} = \delta^n(\eta\, x))
  \text { for some $n$ and $x : \ty{A}$ such that $x \sim_A y$ } \\
%
  &\mho \bisim \theta\, \tilde{y} \text { if } 
    \theta\, \tilde{y} = \delta^n(\mho) \text { for some $n$ } \\
%	
  &\eta\, x \bisim \theta\, \tilde{y} \text { if }
    (\theta\, \tilde{y} = \delta^n (\eta\, y))
  \text { for some $n$ and $y : \ty{A}$ such that $x \sim_A y$ }
\end{align*}

When both sides are $\eta$, then we ensure that the underlying values are related.
When one side is a $\theta$ and the other is $\eta x$ (i.e., one side steps),
we stipulate that the $\theta$-term runs to $\eta y$ where $x$ is related to $y$.
Similarly when one side is $\theta$ and the other $\mho$.
If both sides step, then we allow one time step to pass and compare the resulting terms.
In this way, the definition captures the intuition of terms being equivalent up to
delays.

It can be shown (by \lob-induction) that the step-sensitive relation is symmetric.
However, it can also be shown that this relation is \emph{not} transitive:
One can prove within Clocked Cubical Type Theory
that if this relation were transitive, then in fact it would be trivial in that
$l \bisim l'$ for all $l, l'$.
This issue will be resolved when we consider the relation's \emph{globalization}.



\subsection{Error Domains}

While value types will be interpreted as predomains, we also need a semantics
for computation types. This will be in the form of \emph{error domains}, of which the
Lift monad is a prototypical example. For a fixed clock $k$, an error domain $A$
consists of a predomain (which we also denote by $A$ when there is no risk of confusion),
along with a bottom element $\mho_A$ and a $\later$-algebra $\theta_A \colon \later^k A \monto A$.

% TODO function space as error domain?

\subsection{Globalization}\label{sec:globalization}

Recall that in the above definitions, any occurrences of $\later$ were with
respect to a fixed clock $k$. Intuitively, this corresponds to a step-indexed set.
It will be necessary to consider the ``globalization'' of these definitions,
i.e., the ``global'' behavior of the type over all potential time steps.
This is accomplished in the type theory by \emph{clock quantification} \cite{atkey-mcbride2013},
whereby given a type $X$ parameterized by a clock $k$, we consider the type
$\forall k. X[k]$. This corresponds to leaving the step-indexed world and passing to
the usual semantics in the category of sets.


\section{Semantics}\label{sec:semantics}

\subsection{Step-indexed Semantics}

We give a semantics to the step-sensitive lambda calculus $\intlc$ we defined
in Section \ref{sec:step-sensitive-lc}.
%
Much of the semantics is similar to a normal call-by-value denotational semantics;
we highlight the differences.
Recall that we will interpret value types as predomains, and computation types
as error domains. Value type contexts $\Gamma = x_1 \colon A_1, \dots, x_n \colon A_n$
will be interpreted as the product $\sem{A_1} \times \cdots \times \sem{A_n}$, and
computation type contexts $\Delta = \Delta_\Sigma , \Delta|_V$ will be interpreted as a pair
$(\delta_\Sigma, \delta_V)$ where $\delta_\Sigma$ is either empty or $\sem{B}$.


The semantics of the dynamic type $\dyn$ is the predomain $\Dyn$ introduced in Section
\ref{sec:predomains}.
%
The interpretation of a value $\hasty {\Gamma} V A$ will be a monotone function from
$\sem{\Gamma}$ to $\sem{A}$. Likewise, a term $\hasty {\Delta} M {\Ret{A}}$ will be interpreted
as a monotone function from $\sem{\Delta}$ to $\sem{\Ret{A}} = \li \sem{A}$.

Recall that $\Dyn$ is isomorphic to $\Nat\, + \later (\Dyn \monto \li \Dyn)$.
Thus, the semantics of $\injnat{\cdot}$ and $\injarr{\cdot}$ are simply the
injections $\inl$ and $\inr$.

The interpretation of $\lda{x}{M}$ works as follows. Recall by the typing rule for
lambda that $\hasty {\cdot, \Gamma, x : A_i} M {\Ret {A_o}}$, so the interpretation of $M$
has type $\{*\} \times (\sem{\Gamma} \times \sem{A_i})$ to $\sem{A_o}$.
The interpretation of lambda is thus a function (in the ambient type theory) that takes
a value $a$ representing the argument and applies it (along with $\gamma$) as argument to
the interpretation of $M$.
%
The interpretation of bind and of application both make use the monadic extend function on $\li A$.
%
The interpretation of case-nat and case-arrow is simply a case inspection on the
interpretation of the scrutinee, which has type $\Dyn$.


\vspace{2ex}


\noindent Types:
\begin{align*}
  \sem{\nat} &= \Nat \\
  \sem{\dyn} &= \Dyn \\
  \sem{A \ra A'} &= \sem{A} \To \sem{A'} \\
  \sem{\later A} &=\, \later \sem{A} \\
  \sem{\Ret A} &= \li \sem{A}
\end{align*}

% Contexts:

% TODO check these, especially the semantics of bind, case-nat, and case-arr
% with respect to their context argument
\noindent Values and terms:
\begin{align*}
  \sem{\zro}         &= \lambda \gamma . 0 \\
  \sem{\suc\, V}     &= \lambda \gamma . (\sem{V}\, \gamma) + 1 \\
  \sem{x \in \Gamma} &= \lambda \gamma . \gamma(x) \\
  \sem{\lda{x}{M}}   &= \lambda \gamma . \lambda a . \sem{M}\, (*,\, (\gamma , a))  \\
  \sem{\injnat{V_n}} &= \lambda \gamma . \inl\, (\sem{V_n}\, \gamma) \\
  \sem{\injarr{V_f}} &= \lambda \gamma . \inr\, (\sem{V_f}\, \gamma) \\[2ex]
  \sem{\nxt\, V}     &= \lambda \gamma . \nxt (\sem{V}\, \gamma) \\
  \sem{\theta}       &= \lambda \gamma . \theta \\
%
  \sem{\err_B}         &= \lambda \delta . \mho \\
  \sem{\ret\, V}       &= \lambda \gamma . \eta\, \sem{V} \\
  \sem{\bind{x}{M}{N}} &= \lambda \delta . \ext {(\lambda x . \sem{N}\, (\delta, x))} {\sem{M}\, \delta} \\
  \sem{V_f\, V_x}      &= \lambda \gamma . \ext {(\lambda f . (\ext {f} {\sem{V_x}\, \gamma}))} {(\sem{V_f}\, \gamma)} \\
  \sem{\casenat{V}{M_{no}}{n}{M_{yes}}}         &= 
    \lambda \delta . \text{case $(\sem{V}\, \delta)$ of} \\ 
    &\quad\quad\quad\quad \alt \inl(n) \to \sem{M_{yes}}(n) \\
    &\quad\quad\quad\quad \alt \inr(\tilde{f}) \to \sem{M_{no}} \\
  \sem{\casearr{V}{M_{no}}{\tilde{f}}{M_{yes}}} &= 
    \lambda \delta . \text{case $(\sem{V}\, \delta)$ of} \\ 
    &\quad\quad\quad\quad \alt \inl(n) \to \sem{M_{no}} \\
    &\quad\quad\quad\quad \alt \inr(\tilde{f}) \to \sem{M_{yes}}(\tilde{f})
\end{align*}

% TODO
% \noindent Relations:
% \begin{align*}
% %
% \end{align*}


\begin{comment}
\subsection{Global Semantics}

Having defined the above step-indexed semantics, we now pass to a ``global''
semantics that does not involve any step-indexing. The resulting semantics is still
intensional in that terms that produce the same value in a different number of steps
will be distinct.
We define 

\[ \semgl{\cdot} = \forall (\kpa \colon \Clock) .\, \sem{\cdot}[\kpa]. \]

Note that for a term $M$ of type $\Ret{A}$, the semantics has type
$\sem{M} \colon \sem{\Delta} \monto \sem{\Ret{A}} = \sem{\Delta} \monto \li \sem{A}$.
In the case where $\Delta$ is the empty context, i.e., when $M$ is a closed term,
then this is equivalent to $\li \sem{A}$.
Then the global semantics in this case is $\forall \kappa . \liclk {\kappa} \sem{A}$.
We can show in Clocked Cubical Type Theory this type satisfies a coinductive unfolding property

\[ \forall \kappa . \li \sem{A} \cong \sem{A} + 1\, + (\forall \kappa.\li \sem{A}). \]
\end{comment}


% Machines

% We then define a relation $\Dwn^n$ between terms of type $T$ and $\Machine {\sem{T}}$ by

% \subsection{Extensional Collapse}


% \subsection{Relational Semantics}

% \subsubsection{Term Precision via the Step-Sensitive Error Ordering}
% Homogeneous vs heterogeneous term precision

% \subsection{Logical Relations Semantics}

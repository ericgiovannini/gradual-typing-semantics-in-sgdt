\section{Domain-Theoretic Constructions}\label{sec:domain-theory}

In this section, we discuss the fundamental objects of the model into which we will embed
the intensional lambda calculus and inequational theory. It is important to remember that
the constructions in this section are entirely independent of the syntax described in the
previous section; the notions defined here exist in their own right as purely mathematical
constructs. In the next section, we will link the syntax and semantics via an interpretation
function.

\subsection{The Lift Monad}

When thinking about how to model intensional gradually-typed programs, we must consider
their possible behaviors. On the one hand, we have \emph{failure}: a program may fail
at run-time because of a type error. In addition to this, a program may ``think'',
i.e., take a step of computation. If a program thinks forever, then it never returns a value,
so we can think of the idea of thinking as a way of intensionally modelling \emph{partiality}.

With this in mind, we can describe a semantic object that models these behaviors: a monad
for embedding computations that has cases for failure and ``thinking''.
Previous work has studied such a construct in the setting of the latter, called the lift
monad \cite{mogelberg-paviotti2016}; here, we augment it with the additional effect of failure.

For a type $A$, we define the \emph{lift monad with failure} $\li A$, which we will just call
the \emph{lift monad}, as the following datatype:

\begin{align*}
  \li A &:= \\
  &\eta \colon A \to \li A \\
  &\mho \colon \li A \\
  &\theta \colon \later (\li A) \to \li A
\end{align*}

Unless otherwise mentioned, all constructs involving $\later$ or $\fix$
are understood to be with repsect to a fixed clock $k$. So for the above, we really have for each
clock $k$ a type $\li^k A$ with respect to that clock.

Formally, the lift monad $\li A$ is defined as the solution to the guarded recursive type equation

\[ \li A \cong A + 1 + \later \li A. \]

An element of $\li A$ should be viewed as a computation that can either (1) return a value (via $\eta$),
(2) raise an error and stop (via $\mho$), or (3) think for a step (via $\theta$).
%
Notice there is a computation $\fix \theta$ of type $\li A$. This represents a computation
that thinks forever and never returns a value.

Since we claimed that $\li A$ is a monad, we need to define the monadic operations
and show that they repect the monadic laws. The return is just $\eta$, and extend
is defined via by guarded recursion by cases on the input.
% It is instructive to give at least one example of a use of guarded recursion, so
% we show below how to define extend:
% TODO
%
%
Verifying that the monadic laws hold requires \lob-induction and is straightforward.

The lift monad has the following universal property. Let $f$ be a function from $A$ to $B$,
where $B$ is a $\later$-algebra, i.e., there is $\theta_B \colon \later B \to B$.
Further suppose that $B$ is also an ``error-algebra'', that is, an algebra of the
constant functor $1 \colon \text{Type} \to \text{Type}$ mapping all types to Unit.
This latter statement amounts to saying that there is a map $\text{Unit} \to B$, so $B$ has a
distinguished ``error element" $\mho_B \colon B$.

Then there is a unique homomorphism of algebras $f' \colon \li A \to B$ such that
$f' \circ \eta = f$. The function $f'(l)$ is defined via guarded fixpoint by cases on $l$. 
In the $\mho$ case, we simply return $\mho_B$.
In the $\theta(\tilde{l})$ case, we will return

\[\theta_B (\lambda t . (f'_t \, \tilde{l}_t)). \]

Recalling that $f'$ is a guaded fixpoint, it is available ``later'' and by
applying the tick we get a function we can apply ``now''; for the argument,
we apply the tick to $\tilde{l}$ to get a term of type $\li A$.


%\subsubsection{Model-Theoretic Description}
%We can describe the lift monad in the topos of trees model as follows.


\subsection{Predomains}

The next important construction is that of a \emph{predomain}. A predomain is intended to
model the notion of error ordering that we want terms to have. Thus, we define a predomain $A$
as a partially-ordered set, which consists of a type which we denote $\ty{A}$ and a reflexive,
transitive, and antisymmetric relation $\le_P$ on $A$.

For each type we want to represent, we define a predomain for the corresponding semantic
type. For instance, we define a predomain for natural numbers, a predomain for the
dynamic type, a predomain for functions, and a predomain for the lift monad. We
describe each of these below.

We define monotone functions between predomain as expected. Given predomains
$A$ and $B$, we write $f \colon A \monto B$ to indicate that $f$ is a monotone
function from $A$ to $B$, i.e, for all $a_1 \le_A a_2$, we have $f(a_1) \le_B f(a_2)$.

\begin{itemize}
  \item There is a predomain Nat for natural numbers, where the ordering is equality.
  
  \item There is a predomain Dyn to represent the dynamic type. The underlying type
  for this predomain is defined by guarded fixpoint to be such that
  $\ty{\Dyn} \cong \mathbb{N}\, +\, \later (\ty{\Dyn} \monto \ty{\Dyn})$.
  This definition is valid because the occurrences of Dyn are guarded by the $\later$.
  The ordering is defined via guarded recursion by cases on the argument, using the
  ordering on $\mathbb{N}$ and the ordering on monotone functions described below.

  \item For a predomain $A$, there is a predomain $\li A$ for the ``lift'' of $A$
  using the lift monad. We use the same notation for $\li A$ when $A$ is a type
  and $A$ is a predomain, since the context should make clear which one we are referring to.
  The underling type of $\li A$ is simply $\li \ty{A}$, i.e., the lift of the underlying
  type of $A$.
  The ordering on $\li A$ is the ``step-sensitive error-ordering'' which we describe in
  \ref{subsec:lock-step}.

  \item For predomains $A_i$ and $A_o$, we form the predomain of monotone functions
  from $A_i$ to $A_o$, which we denote by $A_i \To A_o$.
  The ordering is such that $f$ is below $g$ if for all $a$ in $\ty{A_i}$, we have
  $f(a)$ is below $g(a)$ in the ordering for $A_o$. 
\end{itemize}



\subsection{Step-Sensitive Error Ordering}\label{subsec:lock-step}

As mentioned, the ordering on the lift of a predomain $A$ is called the
\emph{step-sensitive error-ordering} (also called ``lock-step error ordering''),
the idea being that two computations $l$ and $l'$ are related if they are in
lock-step with regard to their intensional behavior, up to $l$ erroring.
Formally, we define this ordering as follows:

\begin{itemize}
  \item 	$\eta\, x \ltls \eta\, y$ if $x \le_A y$.
  \item 	$\mho \ltls l$ for all $l$ 
  \item   $\theta\, \tilde{r} \ltls \theta\, \tilde{r'}$ if
          $\later_t (\tilde{r}_t \ltls \tilde{r'}_t)$
\end{itemize}

We also define a heterogeneous version of this ordering between the lifts of two
different predomains $A$ and $B$, parameterized by a relation $R$ between $A$ and $B$.

\subsection{Step-Insensitive Relation}

We define another ordering on $\li A$, called the ``step-insensitive ordering'' or
``weak bisimilarity'', written $l \bisim l'$.
Intuitively, we say $l \bisim l'$ if they are equivalent ``up to delays''.
We introduce the notation $x \sim_A y$ to mean $x \le_A y$ and $y \le_A x$.

The weak bisimilarity relation is defined by guarded fixpoint as follows:

\begin{align*}
  &\mho \bisim \mho \\
%
  &\eta\, x \bisim \eta\, y \text{ if } 
    x \sim_A y \\
%		
  &\theta\, \tilde{x} \bisim \theta\, \tilde{y} \text{ if } 
    \later_t (\tilde{x}_t \bisim \tilde{y}_t) \\
%	
  &\theta\, \tilde{x} \bisim \mho \text{ if } 
    \theta\, \tilde{x} = \delta^n(\mho) \text { for some $n$ } \\
%	
  &\theta\, \tilde{x} \bisim \eta\, y \text{ if }
    (\theta\, \tilde{x} = \delta^n(\eta\, x))
  \text { for some $n$ and $x : \ty{A}$ such that $x \sim_A y$ } \\
%
  &\mho \bisim \theta\, \tilde{y} \text { if } 
    \theta\, \tilde{y} = \delta^n(\mho) \text { for some $n$ } \\
%	
  &\eta\, x \bisim \theta\, \tilde{y} \text { if }
    (\theta\, \tilde{y} = \delta^n (\eta\, y))
  \text { for some $n$ and $y : \ty{A}$ such that $x \sim_A y$ }
\end{align*}

\subsection{Error Domains}


\subsection{Globalization}

Recall that in the above definitions, any occurrences of $\later$ were with
repsect to a fixed clock $k$. Intuitively, this corresponds to a step-indexed set.
It will be necessary to consider the ``globalization'' of these definitions,
i.e., the ``global'' behavior of the type over all potential time steps.
This is accomplished in the type theory by \emph{clock quantification} \cite{atkey-mcbride2013},
whereby given a type $X$ parameterized by a clock $k$, we consider the type
$\forall k. X[k]$. This corresponds to leaving the step-indexed world and passing to
the usual semantics in the category of sets.


\section{Semantics}\label{sec:semantics}


\subsection{Relational Semantics}

\subsubsection{Term Precision via the Step-Sensitive Error Ordering}
% Homogeneous vs heterogeneous term precision

% \subsection{Logical Relations Semantics}
\section{Idealized Double Categorical Models of Graduality}
\label{sec:cbpv}

In order to organize our construction of denotational models we first
develop sufficient \emph{abstract} categorical semantics of gradually
typed languages. We start by modeling the type and term structure of
gradual typing and then extend this to type and term precision.
%
Gradually typed languages inherently involve computational effects of
errors and non-termination and typically in practice many others such
as mutable state and I/O.
%
To model this cleanly categorically, we follow New, Licata and Ahmed's
GTT calculus and base our models off of Levy's Call-by-push-value
(CBPV) calculus which is a standard model of effectful programming
\cite{levycbpv}.
%
There are several notions of model of CBPV from the literature with
varying requirements of which connectives are present
\cite{levy,fioremm,eec}, we will use a variant which models precisely
the connectives we require and no more
($1,\times,F,U,\to$)\footnote{It is essential in this case that we do
not require a cartesian closed category of values as there is no way
to implement casts for an exponential in general.}.

\begin{enumerate}
\item A cartesian category $\mathcal V$
\item A category $\mathcal E$
\item An action of $\mathcal V^{op}$ (with the $\mathcal V$ cartesian
  product as monoidal structure) on $\mathcal E$. We write this with
  an arrow $A \arr B$.

  This means we have 
  \[ \alpha : {A_1 \times A_2} \arr B \cong A_2 \arr (A_1 \arr B) \]
  and
  \[ i : 1 \arr B \cong B \]
  satisfying coherence isomorphisms.
\item A functor $U : \mathcal E \to \mathcal V$ that ``preserves
  powering'' in that every $U(A \arr B)$ is an exponential of $UB$ by $A$
  and that $U\alpha$ and $Ui$ are mapped to the canonical isomorphisms
  for exponentials.
\item A left adjoint $F \dashv U$
\end{enumerate}

\begin{example}
  Given a strong monad $T$ on a bicartesian closed category $\mathcal
  V$, we can extend this to a CBPV model by defining $\mathcal E$ to
  be the category $\mathcal V^T$ of algebras of the monad, defining $A
  \to B$ as the powering of algebras, $F$ as the free algebra and $U$
  as the underlying object functor.
\end{example}

To additionally model the error terms, we add a requirement that there
is a natural transformation $\mho : 1 \to U$. The naturality
requirement encodes that strict morphisms (e.g., the denotations of
evaluation contexts) preserve errors.

We can then model CBV terms and types in a straightforward adaptation
of Levy's interpretation of CBV in CBPV. We interpret types $A$ as
objects $A \in \mathcal V$ and CBV terms $\Gamma \vdash M : A$ as
morphism of any of the equivalent forms $\mathcal E(F(\times\Gamma),
F(A)) \cong \mathcal V(\times\Gamma \vdash UF(A)) \cong \mathcal
E(F(1), \Gamma \to F(A))$. The most interesting type translation is
the CBV function type: $A \ra A' = U(A \to F A')$.
%
Such a model validates all type-based equational reasoning, i.e.,
$\beta\eta$ equality, and models the introduction and elimination
rules for CBV.
%
Thus a CBPV model is sufficient to interpret the CBV term language. We
will require additional structure to interpret the precision and type
casts.

\subsection{Double Categorical Semantics of Graduality}

New and Licata modeled the graduality and type casts for call-by-name
gradual typing using \emph{double categories}, which are defined to be
categories internal to the category of categories. That is, a double
category $\mathcal C$ consists of a category $\mathcal C_o$ of
``objects and function morphisms'' and a category $\mathcal C_{sq}$ of
``relation morphisms and squares'' with functors (reflexive relation)
$r : C_o \to C_{sq}$ and (source and target) $s,t : C_{sq} \to C_o$
satisfying $sr = tr = \id$ as well as a composition operation $c :
C_{sq} \times_{s,t} C_{sq} \to C_{sq}$ respecting source and
target. This models an abstract notion of functions and relations. For
notation, we write function morphisms as $f : A \to B$ and relation
morphisms as $c : A \rel B$ where $c \in C_{sq}$ and $s(c) = A$ and
$t(c) = B$. Finally a morphism $\alpha$ from $c$ to $d$ with
$s(\alpha) = f$ and $s(\beta) = g$ is visualized as
% https://q.uiver.app/#q=WzAsNCxbMCwwLCJBIl0sWzEsMCwiQiJdLFswLDEsIkEnIl0sWzEsMSwiQiciXSxbMCwyLCJmIiwyXSxbMSwzLCJnIl0sWzAsMSwiYyIsMCx7InN0eWxlIjp7ImJvZHkiOnsibmFtZSI6ImJhcnJlZCJ9LCJoZWFkIjp7Im5hbWUiOiJub25lIn19fV0sWzIsMywiZCIsMix7InN0eWxlIjp7ImJvZHkiOnsibmFtZSI6ImJhcnJlZCJ9LCJoZWFkIjp7Im5hbWUiOiJub25lIn19fV1d
\[\begin{tikzcd}[ampersand replacement=\&]
	A \& B \\
	{A'} \& {B'}
	\arrow["f"', from=1-1, to=2-1]
	\arrow["g", from=1-2, to=2-2]
	\arrow["c", "\shortmid"{marking}, no head, from=1-1, to=1-2]
	\arrow["d"', "\shortmid"{marking}, no head, from=2-1, to=2-2]
\end{tikzcd}\]
And is thought of as an abstraction of the notion of relatedness of
functions: functions take related inputs to related outputs. The
composition operations and functoriality give us a notion of
composition of relations as well as functions and vertical and
horizontal composition of squares. In this work we will be chiefly
interested in \emph{locally thin} double categories, that is, double
categories where there is at most one square for any $f,c,g,d$. In
this case we use the notation $f \leq_{c,d} g$ to mean that a square
like the above exists.

New, Licata and Ahmed \cite{newlicataahmed} extended the axiomatic
syntax to call-by-push-value but did not analyze the structure
categorically. We fill in this missing analysis now: a model of the
congruence rules of their system can be given by a locally thin
``double CBPV model'', which we define as a category internal to the
category of CBPV models and \emph{strict} homomorphisms of CBPV
models\footnote{it may be possible to also define this as a notion of
CBPV model internal to some structured $2$-category of categories, but
the authors are not aware of any such definition of an internal CBPV
model}. A strict homomorphism of CBPV models from $(\mathcal
V,\mathcal E,\ldots)$ to $(\mathcal V', \mathcal E',\ldots)$ consists
of functors $G_v : \mathcal V \to \mathcal V'$ and $G_e : \mathcal E
\to \mathcal E'$ that strictly preserve all CBPV constructions, see
the appendix for a more detailed definition.
Some of that data can be visualized as follows:
% https://q.uiver.app/#q=WzAsNCxbMCwwLCJcXHZzcSJdLFsyLDAsIlxcZXNxIl0sWzAsMiwiXFx2ZiJdLFsyLDIsIlxcZWYiXSxbMiwzLCJcXEZmIiwwLHsiY3VydmUiOi0yfV0sWzMsMiwiXFxVZiIsMCx7ImN1cnZlIjotMn1dLFswLDEsIlxcRnNxIiwwLHsiY3VydmUiOi0yfV0sWzEsMCwiXFxVc3EiLDAseyJjdXJ2ZSI6LTJ9XSxbMiwwLCJcXHJ2Il0sWzAsMiwiXFxzdiIsMCx7ImN1cnZlIjotMn1dLFswLDIsIlxcdHYiLDIseyJjdXJ2ZSI6Mn1dLFsxLDMsIlxcc2UiLDAseyJjdXJ2ZSI6LTJ9XSxbMSwzLCJcXHRlIiwyLHsiY3VydmUiOjJ9XSxbMywxLCJcXHJlIl0sWzQsNSwiXFxib3QiLDEseyJzaG9ydGVuIjp7InNvdXJjZSI6MjAsInRhcmdldCI6MjB9LCJzdHlsZSI6eyJib2R5Ijp7Im5hbWUiOiJub25lIn0sImhlYWQiOnsibmFtZSI6Im5vbmUifX19XSxbNiw3LCJcXGJvdCIsMSx7InNob3J0ZW4iOnsic291cmNlIjoyMCwidGFyZ2V0IjoyMH0sInN0eWxlIjp7ImJvZHkiOnsibmFtZSI6Im5vbmUifSwiaGVhZCI6eyJuYW1lIjoibm9uZSJ9fX1dXQ==
\[\begin{tikzcd}[ampersand replacement=\&]
	\vsq \&\& \esq \\
	\\
	\vf \&\& \ef
	\arrow[""{name=0, anchor=center, inner sep=0}, "\Ff", curve={height=-12pt}, from=3-1, to=3-3]
	\arrow[""{name=1, anchor=center, inner sep=0}, "\Uf", curve={height=-12pt}, from=3-3, to=3-1]
	\arrow[""{name=2, anchor=center, inner sep=0}, "\Fsq", curve={height=-12pt}, from=1-1, to=1-3]
	\arrow[""{name=3, anchor=center, inner sep=0}, "\Usq", curve={height=-12pt}, from=1-3, to=1-1]
	\arrow["\rv", from=3-1, to=1-1]
	\arrow["\sv", curve={height=-12pt}, from=1-1, to=3-1]
	\arrow["\tv"', curve={height=12pt}, from=1-1, to=3-1]
	\arrow["\se", curve={height=-12pt}, from=1-3, to=3-3]
	\arrow["\te"', curve={height=12pt}, from=1-3, to=3-3]
	\arrow["\re", from=3-3, to=1-3]
	\arrow["\bot"{description}, draw=none, from=0, to=1]
	\arrow["\bot"{description}, draw=none, from=2, to=3]
\end{tikzcd}\]
Type precision $A \ltdyn A'$ is interpreted as a relation morphism
$c_A : A \rel A'$ in $\mathcal V_{sq}$, and term precision $\Gamma
\ltdyn \Gamma' \vdash M \ltdyn M' : A \ltdyn A'$ is interpreted as a
square $M \ltdyn_{c_\Gamma,UF c_A} M'$. The fact that $t,r$ and the
composition are all given by strict CBPV homomorphisms says that all
the type constructors lift to precision (monotonicity of type
constructors) as well as all term constructors (congruence). Further,
$r$ and composition being strict homomorphisms implies that all type
constructors strictly preserve the identity relation (identity
extension) and composition.

Next, to model type casts, their model further requires that every
value relation $c : A \rel A'$ is \emph{left representable} by a
function $u_c : A \to A'$ and every computation relation $d : B \rel
B'$ is \emph{right representable} by a function $d_c : B' \to B$. In a locally thin double category, these are defined as follows:
\begin{definition}
  $c : A \rel B$ is left representable by $f : A \to B$ if $f \ltsq{c}{r(B)} \id$ and $\id \ltsq{r(A)}{c} f $.

  Dually, $c : A \rel B$ is right representable by $g : B \to A$ if
  $\id \ltsq{c}{r(A)} g$ and $g \ltsq{r(B)}{c} \id$.
\end{definition}
These rules are sufficient to model the UpL/UpR/DnL/DnR rules for
casts. Additionally, since representable morphisms compose and so the
compositionality of casts comes for free. However, the retraction
property must be added as an additional axiom to the model.  To model
the error being a least element we add the requirement that $\mho
\circ ! \ltsq{r(A)}{r(UB)} f$ holds for all $f : \mathcal V(A,B)$.
Finally, the dynamic type can be modeled as an arbitrary value type $D$ with
arbitrary relations $\nat \rel D$ and $D \ra D \rel D$ and $D \times D
\rel D$ (or whatever basic type cases are required).

\begin{example}
  (Adapted from \cite{new-licata}): Define a double CBPV model where
  $\mathcal V$ is the category of predomain preorders: sets with an
  $\omega$-CPO structure $\leq$ as well as a poset structure
  $\ltdyn$. Functional morphisms are given by $\leq$-continuous and
  $\ltdyn$-monotone functions. Then define $\mathcal E$ to be the
  category of pointed domain preorders which are domain preorders with
  least elements $\bot$ for $\leq$ and $\mho$ for $\ltdyn$ such that
  $\bot$ is $\ltdyn$-maximal, and morphisms are as before but preserve
  $\bot$ and $\mho$. This can be extended to a CBPV model with
  forgetful functor $U : \mathcal E \to \mathcal V$. $D$ can be
  defined by solving a domain equation.

  This can be extended to a double CBPV model by defining a value
  relation $A \rel A'$ to be a $\ltdyn$-\emph{embedding}: a morphism
  $e : A \to A'$ that is injective and such that $F e : F A \to F A'$
  has a right adjoint (with respect to $\ltdyn$) and a square $f
  \ltsq{e}{e'} f' = f \circ e \ltdyn e' \circ f$. Similarly
  computation relations $B \rel B'$ are defined to be
  \emph{projections}: morphisms $p : B' \to B$ that are surjective and
  $Up$ has a left adjoint, with squares defined similarly. A suitable
  dynamic type can be constructed by solving a domain equation $D
  \cong \nat + U(D \to FD) + (D \times D)$.
\end{example}

\subsection{Weakening the Double Category Semantics}

While the double categorical semantics can be satisfied with classical
domain theoretic models, there are obstructions to developing a
semantics based on \emph{guarded} recursion. While guarded type theory
makes construction of arbitrary guarded recursive definitions
possible, it comes at a significant cost to reasoning: unfolding of
recursive definitions is explicit. In a denotational semantics, this
means that non-well-founded recursion must perform a kind of
\emph{observable step}. This is a difficulty in proving graduality,
which is a property that is oblivious to the number of steps that a
program takes. Therefore to prove graduality we must impose a kind of
weak bisimilarity relation on our programs to reason about
step-independent relational properties. However, here we arrive at the
fundamental issue with double categorical semantics using
representable relations:

\begin{enumerate}
\item When reasoning up to weak bisimilarity, transitive reasoning is
  not possible, and so horizontal pasting of squares is not a valid
  reasoning principle.
\item When reasoning with observable computation steps, certain casts
  take observable steps, and can no longer be modeled using
  representable morphisms.
\end{enumerate}

Our solution to this dilemma comes in two parts. Since graduality
ignores computational steps, the syntactic theory of graduality must
be modeled up to weak bisimilarity, where transitive reasoning is not
valid. For this purpose we develop a notion of \emph{extensional}
model which weakens from double categories to a reflexive graph
categories, dropping the operation of horizontal pasting of squares,
but still maintaining a form of representability of
relations.

However, transitive reasoning is essential for compositional semantic
constructions, and so we need to work with an intermediate notion of
\emph{intensional} model which is based on double categories, but
where representability has to be weakened to
``quasi-representability'', a kind of representability \emph{up to}
observable steps. To reason up to observable steps without using weak
bisimilarity, we develop a notion we call \emph{perturbations},
certain terms that are bisimilar to the identity but can be
manipulated explicitly in constructions. We then show that the
compositional construction of casts from domain theoretic models can
be adapted to this guarded setting by incorporating some explicit
manipulation of perturbations. Finally, we show that taking an
``extensional collapse'' by bisimilarity provides a model of the
extensional theory, which can then be used to model the global
graduality property.

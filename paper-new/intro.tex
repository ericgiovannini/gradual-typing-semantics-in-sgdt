\section{Introduction}
  
% gradual typing, graduality
\subsection{Gradual Typing and Graduality}
In programming language design, there is a tension between \emph{static} typing
and \emph{dynamic} typing disciplines.
With static typing, the code is type-checked at compile time, while in dynamic typing,
the type checking is deferred to run-time. Both approaches have benefits: with static 
typing, the programmer is assured that if the program passes the type-checker, their
program is free of type errors, and moreover, soundness of the equational theory implies
that program optimizations are valid.
Meanwhile, dynamic typing allows the programmer to rapidly prototype
their application code without needing to commit to fixed type signatures for their
functions. Most languages choose between static or dynamic typing and as a result, programmers that initially write their code in a dynamically typed language
in order to benefit from faster prototyping and development time need to rewrite
some or all of their codebase in a static language if they would like to receive the benefits of static
typing once their codebase has matured.

\emph{Gradually-typed languages} \cite{siek-taha06, tobin-hochstadt06} seek to resolve this tension
by allowing for both static and dynamic typing disciplines to be used in the same codebase,
and by supporting smooth interoperability between statically-typed and dynamically-typed code.
This flexibility allows programmers to begin their projects in a dynamic style and
enjoy the benefits of dynamic typing related to rapid prototyping and easy modification
while the codebase ``solidifies''. Over time, as parts of the code become more mature
and the programmer is more certain of what the types should be, the code can be
\emph{gradually} migrated to a statically typed style without needing to
rewrite the project in a completely different language.

%Gradually-typed languages should satisfy two intuitive properties.
% The following two properties have been identified as useful for gradually typed languages.

In order for this to work as expected, gradually-typed languages should allow for
different parts of the codebase to be in different places along the spectrum from
dynamic to static, and allow for those different parts to interact with one another.
Moreover, gradually-typed languages should support the smooth migration from
dynamic typing to static typing, in that the programmer can initially leave off the
typing annotations and provide them later without altering the meaning of the program.
% Sound gradual typing
Furthermore, the parts of the program that are written in a dynamic
style should soundly interoperate with the parts that are written in a
static style.  That is, the interaction between the static and dynamic
components of the codebase should preserve, to the extent possible,
the guarantees made by the static types.  In particular, while
statically-typed code can error at runtime in a gradually-typed
language, such an error can always be traced back to a
dynamically-typed term that violated the typing contract imposed by
statically typed code. Further, static type assertions are sound in
the static portion, and should enable type-based reasoning and
optimization.

% Moreover, gradually-typed languages should allow for
% different parts of the codebase to be in different places along the spectrum from
% dynamic to static, and allow for those different parts to interact with one another.
% In a \emph{sound} gradually-typed language,
% this interaction should respect the guarantees made by the static types.

% Graduality property
One of the fundamental theorems for gradually typed languages is
\emph{graduality}, also known as the \emph{dynamic gradual guarantee},
originally defined by Siek, Vitousek, Cimini, and Boyland
\cite{siek_et_al:LIPIcs:2015:5031, new-ahmed2018}.
%
Informally, graduality says that going from a dynamic to static style should
only allow for the introduction of static or dynamic type errors, and not
otherwise change the behavior of the program.
%
This is a way to capture programmer intuition that increasing type
precision corresponds to a generalized form of runtime assertions in
that there are no observable behavioral changes up to the point of the
first dynamic type error\footnote{once a dynamic type error is raised,
in languages where the type error can be caught, program behavior may
then further diverge, but this is typically not modeled in gradual
calculi.}.
%
Fundamentally, this property comes down to the behavior of
\emph{runtime type casts}, which implement these generalized runtime
assertions.

Additionally, gradually typed languages should offer some of the
benefits of static typing. While classical type soundness, that
well-typed programs are free from runtime errors, is not compatible
with runtime type errors, we can instead focus on \emph{type-based
reasoning}. For instance, while dynamically typed $\lambda$ calculi
only satisfy $\beta$ equality for their type formers, statically typed
$\lambda$ calculi additionally satisfy type-dependent $\eta$
properties that ensure that functions are determined by their behavior
under application and that pattern matching on data types
is safe and exhaustive.

More concretely, consider a gradually typed language whose only
effects are gradual type errors and divergence. Then if we fix a
result type of natural numbers, a whole program semantics is a partial
function from closed programs to either natural numbers or errors:
\[ -\Downarrow : \{M \,|\, \cdot \vdash M : \nat \} \rightharpoonup \mathbb{N} \cup \{\mho\} \]
where $\mho$ is notation for a runtime type error. We write $M
\Downarrow n$ and $M\Downarrow \mho$ to mean this semantics is defined
as a number or error, and $M\Uparrow$ to mean the semantics is
undefined, representing divergence.
%
A well-behaved semantics should then satisfy several properties. First, it
should be \emph{adequate}: natural number constants should step to
themselves. Second it should validate type based reasoning. To
formalize type based reasoning we give a typed equational theory for
terms of the language $M \equiv N$ for when two terms should be
considered equivalent. Then we want to verify that the big step
semantics respects this equational theory: if closed programs $M \equiv
N$ are equivalent in the equational theory then they have the same
semantics, $M \Downarrow n \iff N \downarrow n$ and $M\Uparrow \iff N
\Uparrow$ and $M \Downarrow \mho \iff N \Downarrow \mho$.
%
Lastly, the graduality property is defined by giving an
\emph{inequational} theory called term precision, where $M \ltdyn N$
roughly means that $M$ and $N$ have the same type erasure and $M$ has
at each point in the program a more precise/static type than $N$.
%
Then, the graduality property states that if $M \ltdyn N$ are whole
programs then $M$ must either have the same behavior as $N$ or error:
Either $M\Downarrow \mho$ or $M \Downarrow n $ and $N \Downarrow n$ or
$M \Uparrow $ and $N \Uparrow$\footnote{we use a slightly more complex
definition of this relation in our technical development below that is
classically equivalent but constructively weaker}.

\subsection{Denotational Semantics in Guarded Domain Theory}

Our goal in this work is to provide an \emph{expressive},
\emph{reusable}, \emph{compositional} semantic framework for defining
such well-behaved semantics of programs.
%
Our approach to achieving this goal is to provide a compositional
\emph{denotational semantics}, mapping types to a kind of semantic
domain, terms to functions and relations such as term precision to
proofs of semantic relations between the denoted functions.
%
Since the denotational constructions are all syntax-independent, the
constructions we provide may be reused for similar languages. Since it
is compositional, components can be mixed and matched depending on
what source language features are present.
%
Providing this semantics for gradual typing is inherently complicated
in that it involves: (1) recursion and recursive types through the
presence of dynamic types, (2) effects in the form of divergence and
errors (3) relational models in capturing the graduality
property. Recursion and recursive types must be handled using some
flavor of domain theory. Effects can be modeled using monads in the
style of Moggi, or adjunctions in the style of Levy. Relational
properties and their verification lead naturally to the use of
reflexive graph categories or double categories.

The only prior denotational semantics for gradual typing was given by
New and Licata and is based on a classical Scott-style \emph{domain
theory} \cite{new-licata18}. The fundamental idea is to equip
$\omega$-CPOs with an additional ``error ordering'' $\ltdyn$ which
models the graduality ordering, and for casts to arise from
\emph{embedding-projection pairs}. Then the graduality property
follows as long as all language constructs can be interpreted using
constructions that are monotone with respect to the error ordering.
%
This framework has the benefit of being compositional, and was
expressive enough to be extended to model dependently typed gradual
typing \cite{10.1145/3495528}.
%
However, an approach based on classical domain theory has fundamental
limitations: domain theory is incapable of modeling certain perversely
recursive features of programming languages such as dynamic type tag
generation and higher-order references, which are commonplace in
real-world gradually typed systems as well as gradual calculi. Our
long-term goal is to develop a semantics that will scale up to these
advanced features, and so we begin with this work by departing from
classical domain theory.

The main denotational alternative to classical domain theory that can
successfully model these advanced features is \emph{guarded domain
theory}, which we adopt in this work. While classical domain theory is
based on modeling types as ordered sets with certain joins, guarded
domain theory is based on an entirely different foundations, sometimes
(ultra)metric spaces but more commonly as ``step-indexed sets'', i.e.,
objects in the \emph{topos of trees}, i.e., presheaves on the poset of
natural numbers\cite{birkedal-mogelberg-schwinghammer-stovring2011}. We think of such a
$\mathbb{N}$-indexed set as modeling a kind of ``sequence of
approximations'' to the true type being modeled.
%
Key to guarded domain theory is that there is a ``later'' operator
$\triangleright$ on types. Thinking of types as a sequence of
approximations, the later operator delays the approximation by one
step. Then the crucial axiom of guarded domain theory is that any
guarded equation $X \cong F(\triangleright X)$ has a unique
solution. This allows guarded domain theory to model essentially
\emph{any} recursive concept, with the caveat that it is guarded by a
later.
%
This caveat is the main source of difficulty in adapting a semantic
approach based on classical domain theory to guarded domain theory:
classical domain theory has limitations in what it can model, but it
provides \emph{exact} solutions to domain equations when it
applies. When adapting the New-Licata approach to guarded domain
theory, this means we must work with an \emph{intensional} semantics,
one where unfolding the dynamic type requires an observable
computational step: e.g., a function that pattern matches on an
element of the dynamic type and then returns it unchanged is
\emph{not} equal to the identity function, because it ``costs'' a step
to perform the pattern match.
%
This leads to the main departure of our semantics from the New-Licata
approach: while the New-Licata semantics equips $\omega$-CPOs with an
error ordering, we work with types equipped with not just an error
ordering but also a \emph{bisimilarity} relation that reflects when
elements differ only in the number of abstract computational steps
that they take.
%
Because bisimilarity, unlike equality, is not transitive this
necessitates further changes to the core concepts of our semantics.

\subsection{Synthetic Guarded Domain Theory}

While guarded domain theory can be presented analytically using
ultrametric spaces or the topos of trees, in practice it is
considerably simpler to work \emph{synthetically} by working in a
non-standard foundation such as guarded type theory where the later
modality is taken as a primitive operation on types, and we take as an
axiom that guarded domain equations have a (necessarily unique)
solution. This has the added benefit that providing a denotational
semantics of gradually typed terms is as simple as a semantics of an
effectful language using a monad on the category of sets. Not only
does this make on-paper reasoning about guarded domain theory easier,
it also enables a simpler avenue to verification in a proof
assistant. Whereas formalizing analytic guarded domain theory would
require significant theory of presheaves and making sure that all
constructions are functors on categories of presheaves, formalizing
synthetic guarded domain theory can be done by directly adding the
later modality and the guarded fixed point property axiomatically.
%
% \max{TODO: more here, specifically this is where we should talk about adequacy I think}

Because the solutions we obtain working in guarded domain theory are constructed
as a sequence of increasingly better approximations, when we want to establish a
property that holds in the limit, we must reason about all finite
approximations. In the setting of the topos of trees, this manifests as the need
to consider \emph{global elements} of the presheaves, i.e., a family of elements
$x_i \in X_i$ compatible with the restriction maps. For example, consider the
set of programs that may take a computational step or return unit. In the topos
of trees, we can model this set as the indexed family of sets $X_n =
\{0,1,\dots,n-1, \bot\}$ for $n \ge 1$, where $i$ denotes a program that steps
$i$ times and then returns, and $\bot$ denotes a program that fails to terminate
in $n-1$ steps or fewer. For any fixed $n$, the set $X_n$ fails to distinguish a
program that takes $n$ steps and then terminates from a program that never
terminates. On the other hand, if we instead consider the denotation of a
program to be a global element of the presheaf $\{X_n\}$ -- a family of
elements $x_i \in X_i$ for all $i \in \mathbb{N}$ -- then the denotation of the
diverging program will be distinct from that of a program that terminates,
regardless of the number of steps it takes.

In the setting of synthetic guarded domain theory, this same idea applies. To
construct a global solution we must use an additional construct, e.g., the
$\square$ modality \cite{10.1007/978-3-319-08918-8_8}, whose semantics in the
topos of trees model is to compute the global elements of a presheaf. A related
approach involves objects known as \emph{clocks}, whereby
\emph{clock-quantification} \cite{atkey-mcbride2013} provides a means to obtain
a global solution.

This has important ramifications for defining an adequate denotational semantics
in guarded domain theory, as we seek to do in this paper: we do not want our
denotational semantics to conflate a diverging program with a program that fails
to terminate in sufficiently few steps. Thus, establishing adequacy of a
semantics in guarded domain theory will require a means of globalizing,
a point we will return to in Section \ref{sec:big-step-term-semantics}.


% For example, when establishing the adequacy of a semantics defined using
% guarded type theory, we must reason about all finite approximations.

% In this paper, we develop an adequate denotational semantics that satisfies
% graduality and soundness of the equational theory of cast calculi using
% synthetic guarded domain theory.  


\subsection{Contributions}

The main contribution of this work is a compositional denotational
semantics for gradually typed languages that validates $\beta\eta$
equality and satisfies a graduality theorem. A great deal of the work
has further been verified in Guarded Cubical Agda \cite{veltri-vezzosi2020}, 
demonstrating that the semantics is readily mechanizable.

\begin{enumerate}
\item First, we give a simple concrete term semantics where we show
  how to model the dynamic type as a solution to a guarded domain equation.
\item Next, we identify where prior work on classical domain theoretic
  semantics of gradual typing breaks down when using guarded semantics
  of recursive types.
\item We develop a key new concept of \emph{syntactic perturbations},
  which allow us to recover enough extensional reasoning to model the
  graduality property compositionally.
\item We combine this insight together with an abstract categorical
  model of gradual typing using reflexive graph categories and
  call-by-push-value to give a compositional construction of our
  denotational model.
\item We prove that the resulting denotational model provides a
  well-behaved semantics as defined above by proving \emph{adequacy},
  respect for an equational theory and the graduality property.
\end{enumerate}

The paper is laid out as follows:
\begin{enumerate}
\item In Section \ref{sec:GTLC} we fix our input language, a fairly
  typical gradually typed cast calculus.
\item In Section \ref{sec:concrete-term-model} we develop a first
  denotational semantics in synthetic guarded domain theory that
  satisfies adequacy and respects the equational theory, but does not
  validate graduality. We use this to introduce some of our main
  technical tools: modeling recursive types in guarded type theory and
  modeling effects using call-by-push-value.
\item In Section \ref{sec:towards-relational-model} we show where the
  New-Licata classical domain theoretic approach fails to adapt
  cleanly to the guarded setting and explore the difficulties of
  proving graduality in an intensional model.
\item In Section \ref{sec:concrete-relational-model} we describe the resulting
  model in detail. We then prove that the resulting denotational model is
  \emph{adequate} for the graduality property: a closed term precision $M \ltdyn
  N : \nat$ has the expected semantics, i.e., that $M$ errors or $M$ and $N$ have the
  same extensional behavior.
\item In Section \ref{sec:discussion} we discuss prior work on proving
  graduality, our partial mechanization of these results and discuss
  future directions for denotational semantics of gradual typing.
\end{enumerate}

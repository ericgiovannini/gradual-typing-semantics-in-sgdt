\section{Introduction}
  
% gradual typing, graduality
\subsection{Gradual Typing and Graduality}
One of the principal ways of categorizing type systems of programming languages is
whether they are \emph{static} or \emph{dynamic}.
In static typing, the code is type-checked at compile time, while in dynamic typing,
the type checking is deferred to run-time. Both approaches have benefits: with static typing,
the programmer is assured that if the program passes the type-checker, their program
is free of type errors. Meanwhile, dynamic typing allows the programmer to rapidly prototype
their application code without needing to commit to fixed type signatures for their functions.

\emph{Gradually-typed languages} \cite{siek-taha06} allow for both static and dynamic typing disciplines
to be used in the same codebase, and support smooth interoperability between statically-typed
and dynamically-typed code.
This flexibility allows programmers to begin their projects in a dynamic style and
enjoy the benefits of dynamic typing related to rapid prototyping and easy modification
while the codebase ``solidifies''. Over time, as parts of the code become more mature
and the programmer is more certain of what the types should be, the code can be
\emph{gradually} migrated to a statically typed style without needing to start the project
over in a completely differnt language.

Gradually-typed languages should satisfy two intuitive properties.
First, the interaction between the static and dynamic components of the codebase
should be safe -- i.e., should preserve the guarantees made by the static types.
In particular, while statically-typed code can error at runtime in a gradually-typed language,
such an error can always be traced back to a dynamically-typed term that
violated the typing contract imposed by statically typed code.
% In other words, in the statically-typed portions of the codebase, type soundness must be preserved.
Second, gradually-typed langugaes should support the smooth migration from dynamic typing
to static typing, in that the programmer can initially leave off the
typing annotations and provide them later without altering the meaning of the
program.

% Graduality property
Formally speaking, gradually typed languages should satisfy the 
\emph{dynamic gradual guarantee}, originally defined by Siek, Vitousek, Cimini,
and Boyland \cite{siek_et_al:LIPIcs:2015:5031}.
This property is also referred to as \emph{graduality}, by analogy with parametricity.
Intuitively, graduality says that going from a dynamic to static style should not
introduce changes in the meaning of the program.
More specifically, making the types more precise by adding typing annotations
% (replacing $\dyn$ with a more specific type
%such as integers)
will either result in the same behavior as the original, less precise program,
or will result in a type error.


\subsection{Current Approaches to Proving Graduality}
Current approaches to proving graduality include the methods of Abstracting Gradual Typing
\cite{garcia-clark-tanter2016} and the formal tools of the Gradualier \cite{cimini-siek2016}.
These allow the language developer to start with a statically typed langauge and derive a
gradually typed language that satisfies the gradual guarantee. The downside to these approaches
is that the semantics of the resulting languages are too lazy: the frameworks consider only
the $\beta$ rules and not the $\eta$ equalities. Furthermore, while these frameworks do prove
graduality of the resulting languages, they do not show the correctness of the equational theory,
which is equally important to sound gradual typing. For example, programmers often refactor their
code without thinking about whether the refactoring has broken the semantics of the program. 
It is the validity of the laws in the equational theory that guarantees that such refactorings are sound. 
Similarly, correctness of compiler optimizations rests on the validity of the corresponding equations
from the equational theory. It is therefore important that the langages that claim to be gradually typed
have provably correct equational theories.

%The downside is that
%not all gradually typed languages can be derived from these frameworks, and moreover, in both
%approaches the semantics is derived from the static type system as opposed to the alternative
%in which the semantics determines the type checking. Without a clear semantic interpretation of type
%dynamism, it becomes difficult to extend these techniques to new language features such as polymorphism.

% Proving graduality via LR
New and Ahmed \cite{new-ahmed2018}
have developed a semantic approach to specifying type dynamism in terms of
\emph{embedding-projection pairs}, which allows for a particularly elegant formulation of the
gradual guarantee.
Moreover, their axiomatic account of program equivalence allows for type-based reasoning
about program equivalences.
%
In this approach, a logical relation is constructed and shown to be sound with respect to
the notion of observational approximation that specifies when one program is more precise than another.
The downside of this approach is that each new language requires a different logical relation
to prove graduality. Furthermore, the logical relations tend to be quite complicated due
to a technical requirement known as \emph{step-indexing}.
As a result, developments using this approach tend to require vast effort, with the
corresponding technical reports having 50+ pages of proofs.

An alternative approach, which we investigate in this paper, is provided by
\emph{synthetic guarded domain theory}.
The tecnhiques of synthetic guarded domain theory allow us to internalize the
step-index reasoning normally required in logical relations proofs of graduality,
ultimately allowing us to specify the logical relation in a manner that looks nearly
identical to a typical, non-step-indexed logical relation.

In this paper, we report on work we have done to mechanize proofs of graduality and
correctness of equational theories using SGDT techniques in Agda.
Our goal in this work is to mechanize these proofs in a reusable way,
thereby providing a framework to use to more easily and
conveniently prove that existing languages satsify graduality and have
sound equational theories. Moreover, the aim is for designers of new languages
to utlize the framework to facilitate the design of new provably-correct
gradually-typed languages with nontrivial features.


\subsection{Proving Graduality in SGDT}
\textbf{TODO:} This section should probably be moved to after the relevant background has been introduced?

% TODO introduce the idea of cast calculus and explicit casts?

In this paper, we will utilize SGDT techniques to prove graduality for a particularly
simple gradually-typed cast calculus, the gradually-typed lambda calculus.
This is the usual simply-typed lambda calculus with a dynamic type $\dyn$ such that
$A \ltdyn\, \dyn$ for all types $A$, as well as upcasts and downcasts between any types
$A$ and $B$ such that $A \ltdyn B$. The complete definition will be provided in
Section \ref{sec:GTLC}.
The graduality theorem is shown below.


% \begin{theorem}[Graduality]
%   If $M \ltdyn N : \nat$, then either:
%   \begin{enumerate}
%     \item $M = \mho$
%     \item $M = N = \zro$
%     \item $M = N = \suc n$ for some $n$
%     \item $M$ and $N$ diverge
%   \end{enumerate}
% \end{theorem}


\begin{theorem}[Graduality]
  If $\cdot \vdash M \ltdyn N : \nat$, then
  \begin{enumerate}
    \item $M \Downarrow$ iff $N \Downarrow$
    \item If $M \Downarrow v_?$ and $N \Downarrow v'_?$ then either $v_? = \mho$, or $v_? = v'_?$.
  \end{enumerate}
\end{theorem}

Details can be found in later sections, but we provide a brief explanation of the terminology and notation:

\begin{itemize}
  \item $M \ltdyn N : \nat$ means $M$ and $N$ are terms of type $\nat$ such that
  $M$ is ``syntactically more precise'' than $N$, or equivalently, $N$ is 
  ``more dynamic'' than $M$. Intuitively this means that $M$ and $N$ are the
  same except that in some places where $M$ has explicit typing annotations,
  $N$ has $\dyn$ instead.
  \item $\cdot \Downarrow$ is a relation on terms that is defined such that $M \Downarrow$ means
  that $M$ terminates, either with a run-time error or a value $n$ of type $\nat$.
  \item $\mho$ is a syntactic representation of a run-time type error, which
  happens, for example, when a programmer tries to call a function with a value whose type
  is found to be incompatible with the argument type of the function.
  \item $v_?$ is shorthand for the syntactic representation of a term that is either equal to
  $\mho$, or equal to the syntactic representation of a value $n$ of type $\nat$.
\end{itemize}

% We also should be able to show that $\mho$, $\zro$, and $\suc\, N$ are not equal.

Our first step toward proving graduality is to formulate an \emph{step-sensitive}, or \emph{intensional},
gradual lambda calculus, which we call $\intlc$, in which the computation steps taken by a term are made explicit.
The ``normal'' gradual lambda calculus for which we want to prove graduality will be called the
\emph{step-insensitive}, or \emph{extensional}, gradual lambda calculus, denoted $\extlc$.
We will define an erasure function $\erase{\cdot} : \intlc \to \extlc$ which takes a program in the intensional lambda calculus
and ``forgets'' the syntactic information about the steps to produce a term in the extensional calculus.

Every term $M_e$ in $\extlc$ will have a corresponding program $M_i$ in $\intlc$ such that
$\erase{M_i} = M_e$. Moreover, we will show that if $M_e \ltdyn_e M_e'$ in the extensional theory,
then there exists terms $M_i$ and $M_i'$ such that $\erase{M_i}=M_e$, $\erase{M_i'}=M_e'$ and
$M_i \ltdyn_i M_i'$ in the intensional theory.

We formulate and prove an analogous graduality theorem for the intensional lambda calculus.
We define an interpretation of the intensional lambda calculus into a model in which we prove
various results. Using the observation above, given $M_e \ltdyn M_e' : \nat$, we can find
intensional programs $M_i$ and $M_i'$ that erase to them and are such that $M_i \ltdyn M_i'$.
We will then apply the intensional graduality theorem to $M_i$ and $M_i'$, and translate the result
back to $M_e$ and $M_e'$.

\subsection{Contributions}
Our main contribution is a reusable framework in Guarded Cubical Agda for developing machine-checked proofs
of graduality of a cast calculus.
To demonstrate the feasability and utility of our approach, we have used the framework to prove
graduality for the simply-typed gradual lambda calculus. Along the way, we have developed an ``intensional" theory
of graduality that is of independent interest.


\subsection{Overview of Remainder of Paper}

In Section \ref{sec:technical-background}, we provide technical background on gradually typed languages and
on synthetic guarded domain theory.
% 
In Section \ref{sec:GTLC}, we introduce the gradually-typed cast calculus
for which we will prove graduality. Important here are the notions of syntactic
type precision and term precision. We introduce both the \emph{extensional} gradual lambda calculus
($\extlc$) and the \emph{intensional} gradual lambda calculus ($\intlc$).
%
In Section \ref{sec:domain-theory}, we define several fundamental constructions
internal to SGDT that will be needed when we give a denotational semantics to
our intensional lambda calculus.
This includes the notion of Predomains as well as the concept
of EP-Pairs.
%
In Section \ref{sec:semantics}, we define the denotational semantics for the
intensional gradually-typed lambda calculus using the domain theoretic constructions in the
previous section.
%
In Section \ref{sec:graduality}, we outline in more detail the proof of graduality for the
extensional gradual lambda calculus, which will make use of prove properties we prove about
 the intensional gradual lambda calculus.
%
In Section \ref{sec:discussion}, we discuss the benefits and drawbacks to our approach in comparison
to the traditional step-indexing approach, as well as possibilities for future work.


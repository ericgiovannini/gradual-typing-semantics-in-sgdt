\section{Introduction}
  
% gradual typing, graduality
\subsection{Gradual Typing and Graduality}
In programming language design, there is a tension between \emph{static} typing
and \emph{dynamic} typing disciplines. With static typing, the code is
type-checked at compile time, while in dynamic typing, the type checking is
deferred to run-time. Both approaches have benefits and excel in different
scenarios, with static typing offering compile-time assurance of a program's
type safety and type-based reasoning principles that justify program
optimizations, and dynamic typing allowing for rapid prototyping of a codebase
without committing to fixed type signatures.
%
Most languages choose between static or dynamic typing and as a result,
programmers that initially write their code in a dynamically typed language need
to rewrite some or all of their codebase in a static language if they would like
to receive the benefits of static typing once their codebase has matured.

\emph{Gradually typed languages} \cite{siek-taha06, tobin-hochstadt06} seek to
resolve this tension by allowing for both static and dynamic typing disciplines
to be used in the same codebase. These languages support smooth interoperability
between statically-typed and dynamically-typed styles, allowing the programmer to
begin with fully dynamically-typed code and \emph{gradually} migrate portions of the
codebase to a statically typed style without needing to rewrite the project in a
completely different language.

%Gradually-typed languages should satisfy two intuitive properties.
% The following two properties have been identified as useful for gradually typed languages.

% \eric{This paragraph could be deleted}
%% In order for this to work as expected, gradually-typed languages should allow for
%% different parts of the codebase to be in different places along the spectrum from
%% dynamic to static, and allow for those different parts to interact with one another.
%% Moreover, gradually-typed languages should support the smooth migration from
%% dynamic typing to static typing, in that the programmer can initially leave off the
%% typing annotations and provide them later without altering the meaning of the program.
%% % Sound gradual typing
%% Furthermore, the parts of the program that are written in a dynamic
%% style should soundly interoperate with the parts that are written in a
%% static style.  That is, the interaction between the static and dynamic
%% components of the codebase should preserve, to the extent possible,
%% the guarantees made by the static types.  In particular, while
%% statically-typed code can error at runtime in a gradually-typed
%% language, such an error can always be traced back to a
%% dynamically-typed term that violated the typing contract imposed by
%% statically typed code. Further, static type assertions are sound in
%% the static portion, and should enable type-based reasoning and
%% optimization.

% Moreover, gradually-typed languages should allow for
% different parts of the codebase to be in different places along the spectrum from
% dynamic to static, and allow for those different parts to interact with one another.
% In a \emph{sound} gradually-typed language,
% this interaction should respect the guarantees made by the static types.

% Graduality property
One of the fundamental theorems for gradually typed languages is
\emph{graduality}, also known as the \emph{dynamic gradual guarantee},
originally defined by Siek, Vitousek, Cimini, and Boyland
\cite{siek_et_al:LIPIcs:2015:5031, new-ahmed2018}.
%
Informally, graduality says that migrating code from dynamic to
static typing should only allow for the introduction of static or
dynamic type errors, and not otherwise change the behavior of the
program.
%
This is a way to capture programmer intuition that increasing type
precision corresponds to a generalized form of runtime assertions in
that there are no observable behavioral changes up to the point of the
first dynamic type error\footnote{once a dynamic type error is raised,
in languages where the type error can be caught, program behavior may
then further diverge, but this is typically not modeled in gradual
calculi.}.
%
Fundamentally, this property comes down to the behavior of
\emph{runtime type casts}, which implement these generalized runtime
assertions.

Additionally, gradually typed languages should offer some of the
benefits of static typing. While standard type soundness, that
well-typed programs are free from runtime errors, is not compatible
with runtime type errors, it is possible instead to prove that
gradually typed languages validate \emph{type-based reasoning}. For
instance, while dynamically typed $\lambda$ calculi only satisfy
$\beta$ equality for their type formers, statically typed $\lambda$
calculi additionally satisfy type-dependent $\eta$ properties that
ensure that functions are determined by their behavior under
application and that pattern matching on data types is safe and
exhaustive. A gradually typed calculus that validates these
type-dependent $\eta$ laws then provides some of the type-based
reasoning that dynamic languages lack.

% moved the remaining paragraphs to a new section later in the intro

\subsection{Denotational Semantics in Guarded Domain Theory}

Our goal in this work is to provide an \emph{expressive},
\emph{reusable}, \emph{compositional} semantic framework for defining
such well-behaved semantics of gradually typed programs.
%
Our approach to achieving this goal is to provide a compositional
\emph{denotational semantics}, mapping types to a kind of semantic
domain, terms to functions and relations such as term precision to
proofs of semantic relations between the denoted functions.
%
Since the denotational constructions are all syntax-independent, the
constructions we provide may be reused for similar languages. Since it
is compositional, components can be mixed and matched depending on
what source language features are present.

Providing a semantics for gradual typing is inherently complicated in
that it involves: (1) recursion and recursive types through the
presence of dynamic types, (2) effects in the form of divergence and
errors (3) relational models in capturing the graduality
property. Recursion and recursive types must be handled using some
flavor of domain theory. Effects can be modeled using monads in the
style of Moggi, or adjunctions in the style of
Levy\cite{moggi,levy}. Relational properties and their verification
lead naturally to the use of reflexive graph categories or double
categories\cite{reflgraphcats,doublecats}.

The only prior denotational semantics for gradual typing was given by
New and Licata and is based on a classical Scott-style \emph{domain
theory} \cite{new-licata18}. The fundamental idea is to equip
$\omega$-CPOs with an additional ``error ordering'' $\ltdyn$ which
models the graduality ordering, and for casts to arise from
\emph{embedding-projection pairs}. Then the graduality property
follows as long as all language constructs can be interpreted using
constructions that are monotone with respect to the error ordering.
%
This framework has the benefit of being compositional, and was
expressive enough to be extended to model dependently typed gradual
typing \cite{gradualizing-cic}.
%
However, an approach based on classical domain theory has fundamental
limitations: domain theory appears incapable of modeling certain perversely
recursive features of programming languages such as dynamic type tag
generation and higher-order references \cite{Birkedal-Stovring-Thamsborg-2009}, which are commonplace in
real-world gradually typed systems as well as gradual calculi.
%
Our long-term goal is to develop a denotational approach that can
scale up to these advanced features, and so we must abandon classical
domain theory as the foundation in order to make progress. In this
work, we provide first steps towards this goal by adapting work on a
simple gradually typed lambda calculus to \emph{guarded domain
theory}. Our goal is for this model to scale to dynamic type tag
generation and higher-order references in future work.

\emph{Guarded} domain theory is the main denotational alternative to
classical domain theory that can successfully model these advanced
features. While classical domain theory is based on modeling types as
ordered sets with certain joins, guarded domain theory is based on an
entirely different foundations, sometimes (ultra)metric spaces but
more commonly as ``step-indexed sets'', i.e., objects in the
\emph{topos of trees}
\cite{birkedal-mogelberg-schwinghammer-stovring2011}.  Such an object
consists of a family $\{X_n\}_{n \in \mathbb{N}}$ of sets along with
restriction functions $r_n : X_{n+1} \to X_n$ for all $n$.  (in
category theoretic terminology, these are presheaves on the poset of
natural numbers.)  We think of a $\mathbb{N}$-indexed set as an
infinite sequence of increasingly precise approximations to the true
type being modeled.
%
Key to guarded domain theory is that there is an operator
$\triangleright$ on step-indexed sets called ``later''. In terms of
sequences of approximations, the later operator delays the
approximation by one step. Then the crucial axiom of guarded domain
theory is that any guarded domain equation $X \cong F(\triangleright
X)$ has a unique solution. This allows guarded domain theory to model
essentially \emph{any} recursive concept, with the caveat that the
recursion is \emph{guarded} by a use of the later operator.

% The theorems we want are about the whole sequence of approximations
% whereas working with presheaves is about each finite approximation
% The statement of graduality is in terms of global elements
% \eric{This next paragraph could be moved to the section on adequacy, but since
% the example is in the analytic setting it also makes sense to keep it here. I'm
% not sure which option makes more sense.} While the definitions in guarded domain
% theory are constructed as sequences of approximations, results about semantics
% of programs, e.g., graduality, are actually statements concerning entire
% \emph{sequences} of approximations. As a simple example, consider the set of
% programs that may take a computational step or return unit.\footnote{This
% example is adapted from \cite{mogelberg-paviotti2016}.} In the topos of trees,
% we can model the set of these programs as the indexed family of sets $\{X_n\}_{n
% \in \mathbb{N}}$ where for each $n \ge 1$ we define $X_n = \{0,1,\dots,n-1,
% \bot\}$. Here, $i$ denotes a program that steps $i$ times and then returns, and
% $\bot$ denotes a program that fails to terminate in $n-1$ steps or fewer. For
% any fixed $n$, the set $X_n$ fails to distinguish a program that takes $n$ steps
% and then terminates from a program that never terminates. On the other hand, if
% we define the denotation of a program to be a \emph{sequence} of elements $x_i
% \in X_i$ for all $i \in \mathbb{N}$, then the denotation of the diverging
% program will be distinct from that of a program that terminates, regardless of
% the number of steps it takes.


\subsection{Synthetic Guarded Domain Theory}

While guarded domain theory can be presented analytically using
ultrametric spaces or the topos of trees, in practice it is
considerably simpler to work \emph{synthetically} by working in a
non-standard foundational system such as \emph{guarded type
theory}. In guarded type theory later is taken as a primitive
operation on types, and we take as an axiom that guarded domain
equations have a (necessarily unique) solution. The benefit of this
synthetic approach is that when working in the non-standard
foundation, we don't need to model an object language type as a
step-indexed set, but instead simply as a set, and object-language
terms can be modeled in the Kleisli category of a simple monad defined
using guarded recursion. Not only does this make on-paper reasoning
about guarded domain theory easier, it also enables a simpler avenue
to verification in a proof assistant. Whereas formalizing analytic
guarded domain theory would require a significant theory of presheaves
and making sure that all constructions are functors on categories of
presheaves, formalizing synthetic guarded domain theory can be done by
directly adding the later modality and the guarded fixed point
property axiomatically.
%
% \max{TODO: more here, specifically this is where we should talk about adequacy I think}
%
In this paper, we work informally in a guarded type theory which is described in
more detail in Section \ref{sec:guarded-type-theory}.

% The move to synthetic setting is useful, but now you're stuck in the guarded setting
The simplicity afforded by working in the synthetic setting comes with a
drawback: by changing our ambient foundations so that all types represent
step-indexed sets, it is no longer possible to define a type that denotes an
ordinary set. Roughly speaking, what is needed is an ``escape hatch'' to leave
the guarded setting and make definitions that are interpreted as constructions
in ordinary mathematics. This is typically accomplished by adding a modality to
the guarded type theory. The specific approach we take in this paper is the
technique of \emph{clock-quantification} \cite{atkey-mcbride2013}. In
particular, the guarded type theory in which we work has a notion of clocks that
index all of our guarded constructions. This is discussed in more detail in
Section \ref{sec:big-step-term-semantics}.

%
% As in the setting of analytic guarded domain theory, in the synthetic setting it
% is also possible to construct global solutions. The specific approach we use in
% this work is that of \emph{clocks} and \emph{clock-quantification}
% \cite{atkey-mcbride2013}.

% \max{shouldn't we say we use clock quantification?}
% In the setting of synthetic guarded domain theory, there is an analogue of
% constructing global solutions. To construct a global solution we must use an
% additional construct, e.g., the $\square$ modality
% \cite{10.1007/978-3-319-08918-8_8}, whose semantics in the topos of trees model
% is to compute the global elements of a presheaf. A related approach involves
% objects known as \emph{clocks}, whereby \emph{clock-quantification}
% \cite{atkey-mcbride2013} provides a means to obtain a global solution.


\subsection{Adapting the New-Licata Model to the Guarded Setting}

% Working in a step-counting model, but graduality is independent of steps
% Casts take stpes, so we need to reason up to weak bisimilarity, but it is not transitive
% New-Licata freely uses transitivity, but we can't adapt their proof methodology because it uses transitivity pervasively
% Thus the old proofs from New-Licata that use transitivity don't apply.
% Work with a combination of weak bisimilarity and step-counting reasoning
% Need a tiny bit of weak bisimilarity which are perturbations, explicit synchronizations that we manipulate syntactically

Since guarded domain theory only provides solutions to guarded domain equations,
there is no systematic way to convert a classical domain-theoretic semantics to
a guarded one.  Classical domain theory has limitations in what it can model,
but it provides \emph{exact} solutions to domain equations when it applies. When
adapting the New-Licata approach to guarded domain theory, the presence of later
in the semantics makes it \emph{intensional}: unfolding the dynamic type
requires an observable computational step. For example, a function that pattern
matches on an element of the dynamic type and then returns it unchanged is
\emph{not} equal to the identity function, because it ``costs'' a step to
perform the pattern match.

% - The essential role of transitivity
% - We can't have transitivity + extensional reasoning
% - Solution: split the ordering into two, and introduce perturbations
%   to recover some amount of transitive reasoning

The intensional nature of guarded domain theory leads to the main departure of
our semantics from the New-Licata approach. Because casts involve computational
steps, the graduality property must be insensitive to the steps taken by terms.
This means that the model must allow for reasoning \emph{up to weak
bisimilarity}, where two terms are weakly bisimilar if they differ only in their
number of computational steps. However, the weak bisimilarity relation is not
transitive in the guarded setting, which follows from a no-go theorem we
establish (Theorem \ref{thm:no-go}). The New-Licata proofs freely use
transitivity, and we argue in Section \ref{sec:towards-relational-model} that
some amount of transitive reasoning is nessecary for defining a
syntax-independent model of gradual typing. As a result, the lack of
transitivity of weak bisimilarity presents a challenge in adapting the
New-Licata model to the guarded setting. Our solution is to work with a
combination of weak bisimilarity and step-sensitive reasoning, where the latter
notion \emph{is} transitive. To deal with the fact that casts take computational
steps, we introduce the novel concept of \emph{syntactic perturbations}, which
are explicit synchronizations that we manipulate syntactically. The combination
of lock-step reasoning with perturbations is transitive and is able to account
for the step-insensitive nature of the graduality property.

% These can then be used to explicitly \emph{synchronize} elements to ensure
% that they are in lock-step.
% %
% In the end, we have come to a resolution of the issue: the lock-step error
% ordering where we employ syntactic perturbations to synchronize elements is
% still transitive, while simultaneously being able to relate terms that involve
% an explicit, known pattern of computational steps.


\subsection{Adequacy}

% Our approach to proving graduality by constructing a denotational model in
% guarded type theory consists of two main steps. First, there is the construction
% of the model itself. This gives us an interpretation of the terms as well as the
% axioms that model the graduality property. 

Having defined the model of gradual typing in guarded type theory, we need to
show that the guarded model induces a sensible set-theoretic semantics in
ordinary mathematics. We cannot hope to derive such a semantics for \emph{all}
types (e.g., the dynamic type), but for the subset of closed terms of base type,
we should be able to extract a well-behaved semantics for which the graduality
property holds.

More concretely, consider a gradually typed language whose only effects are
gradual type errors and divergence (errors arise from failing casts; divergence
arises because we can use the dynamic type to encode untyped lambda calculus;
see Section \ref{sec:GTLC}). If we fix a result type of natural numbers, a
``big-step'' semantics is a partial function from closed programs to either natural
numbers or errors:
%
\[ -\Downarrow : \{M \,|\, \cdot \vdash M : \nat \} \rightharpoonup \mathbb{N}
\cup \{\mho\} \] 
%
where $\mho$ is notation for a runtime type error. We write $M \Downarrow n$ and
$M\Downarrow \mho$ to mean this semantics is defined as a number or error, and
$M\Uparrow$ to mean the semantics is undefined, representing divergence.
%
A well-behaved semantics should then satisfy several properties. First, it
should be \emph{adequate}: natural number constants should evaluate to
themselves: $n \Downarrow n$. Second, it should validate type based reasoning.
To formalize type based reasoning, languages typically have an equational theory
$M \equiv N$ specifying when two terms should be considered equivalent.
Then we want to verify that the big step semantics respects this equational
theory: if closed programs $M \equiv N$ are equivalent in the equational theory
then they have the same semantics, $M \Downarrow n \iff N \Downarrow n$ and
$M\Uparrow \iff N \Uparrow$ and $M \Downarrow \mho \iff N \Downarrow \mho$.

To define the big-step semantics, we apply clock quantification to obtain
an interpretation of closed programs of base type as partial functions.
We describe the process in more detail in Section
\ref{sec:big-step-term-semantics}. The resulting term semantics is adequate, and
furthemore it validates the equational theory, since equivalent terms in the
equational theory denote equal terms in the big-step semantics.

Lastly, the semantics should be adequate for the graduality property. Graduality
is typically axiomatized by giving an \emph{inequational} theory called term
precision, where $M \ltdyn N$ roughly means that $M$ and $N$ have the same type
erasure and $M$ has at each point in the program a more precise/static type than
$N$. Then adequacy says that if $M$ and $N$ are whole programs returning type
$\nat$ and $M \ltdyn N$, then the denotations of $M$ and $N$ as big-step terms
should be related in the expected way: either $M$ errors, or $M$ and $N$ have
the same extensional behavior. That is, either $M\Downarrow \mho$ or $M
\Downarrow n $ and $N \Downarrow n$ or $M \Uparrow $ and $N
\Uparrow$\footnote{we use a slightly more complex definition of this relation in
our technical development below that is classically equivalent but
constructively weaker}.
%
To prove that the semantics is adequate for graduality, we again apply clock
quantification, this time to the relation that denotes the term precision
ordering. We show that a term precision ordering $M \ltdyn N$ implies the
corresponding ordering on the partial functions denoted by $M$ and $N$. The
details of the proof are given in Section \ref{sec:adequacy}.


% For example, when establishing the adequacy of a semantics defined using
% guarded type theory, we must reason about all finite approximations.

% In this paper, we develop an adequate denotational semantics that satisfies
% graduality and soundness of the equational theory of cast calculi using
% synthetic guarded domain theory.  


\subsection{Contributions and Outline}

The main contribution of this work is a compositional denotational semantics in
guarded type theory for a simple gradually typed language that validates
$\beta\eta$ equality and satisfies a graduality theorem.
%
% In particular, the notion of syntactic perturbations stands out as a key
% technical contribution and is the most significant and novel feature of our
% model that allowed us to successfully adapt the denotational approach of New and
% Licata to the guarded setting.
% In particular, the notion of syntactic perturbations stands out as a key feature
% of our model, and is our most significant technical contribution allowing us to
% successfully adapt the denotational approach of New and Licata to the guarded
% setting.
%
Within our semantics, the notion of syntactic perturbation is our most
significant technical contribution, allowing us to successfully adapt the
denotational approach of New and Licata to the guarded setting.
%
Most of our work has further been verified in Guarded Cubical Agda
\cite{veltri-vezzosi2020}, demonstrating that the semantics is readily
mechanizable. We provide an overview of the mechanization effort, and what
remains to be formalized, in Section \ref{sec:mechanization}.
%


% Syntactic perturbations 
%
% The notion of \emph{syntactic perturbations} allow us to encode the steps
% taken by a term in a form that we can manipulate. This allows us to recover
% enough extensional reasoning to model the graduality property compositionally.

% Syntactic perturbations give us a type-directed way of imposing delays on terms.

\begin{comment}
\begin{enumerate}
\item First, we give a simple concrete term semantics where we show
  how to model the dynamic type as a solution to a guarded domain equation.
\item Next, we identify where prior work on classical domain theoretic
  semantics of gradual typing breaks down when using guarded semantics
  of recursive types.
\item We develop a key new concept of \emph{syntactic perturbations},
  which allow us to recover enough extensional reasoning to model the
  graduality property compositionally.
\item We combine this insight together with an abstract categorical
  model of gradual typing using reflexive graph categories and
  call-by-push-value to give a compositional construction of our
  denotational model.
\item We prove that the resulting denotational model provides a
  well-behaved semantics as defined above by proving \emph{adequacy},
  respect for an equational theory and the graduality property.
\end{enumerate}
\end{comment}

The paper is laid out as follows:
% \max{I think there's too much detail in this outline, especially parts 4 and 5. Can we instead expand those into the Intro?}
\begin{enumerate}
\item In Section \ref{sec:GTLC} we fix our input language, a fairly
  typical gradually typed cast calculus.
\item In Section \ref{sec:concrete-term-model} we develop a
  denotational semantics in synthetic guarded domain theory for the
  \emph{terms} of the gradual lambda calculus.  The model is adeqauate
  and validates the equational theory, but it does not satisfy
  graduality. We use this to introduce some of our main technical
  tools: modeling recursive types in guarded type theory and modeling
  effects using call-by-push-value.
\item In Section \ref{sec:towards-relational-model} we show where the New-Licata
  classical domain theoretic approach fails to adapt cleanly to the guarded
  setting and explore the difficulties of proving graduality in an intensional
  model. We prove a no-go theorem about extensional, transitive relations,
  introduce the lock-step error ordering and weak bisimilarity relation, and
  motivate the need for perturbations.
\item In Section \ref{sec:concrete-relational-model} we describe the
  construction of the model in detail, and discuss the
  translation of the syntax and axioms of the gradually typed cast calculus into the model.
  Lastly, we prove that the model is adequate for the graduality property.
  % The resulting model validates the axioms for type and term precision specified
  % in Section \ref{sec:GTLC}. The final step is to prove that the model is
  % \emph{adequate} for the graduality property: a closed term precision $M \ltdyn
  % N : \nat$ has the expected semantics, i.e., that $M$ errors or $M$ and $N$
  % have the same extensional behavior. We do so by extending the globalization
  % techniques used in defining the big-step term model to account for the
  % interpretation of term precision.
\item In Section \ref{sec:discussion} we discuss prior work on proving
  graduality, the partial mechanization of our results in Agda, and
  future directions for denotational semantics of gradual typing.
\end{enumerate}

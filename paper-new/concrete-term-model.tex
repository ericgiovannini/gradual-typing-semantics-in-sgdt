\section{A Concrete Term Model}\label{sec:concrete-term-model}

\eric{Wording?}
The principal goal of the next two sections is to ground the refined notions of
abstract model of gradual typing described in Section \ref{sec:abstract-models}
in a concrete setting. We aim to introduce the key issues whose resolutions in
the concrete setting will inform the refined definitions of abstract model.
%
In the present section, we use the techniques of SGDT to give a set-theoretic
denotational semantics to the terms of the gradually-typed lambda calculus
introduced in Section \ref{sec:GTLC}. This serves two important purposes: First,
it acts as a warm-up for employing the tools of SGDT to define a denotational
semantics. Second, constructing this semantic model establishes the validity of
the $\beta$ and $\eta$ principles for the gradually-typed lambda calculus. It
also acts as a stepping stone into the next section, where we extend the
techniques employed here to accommodate the type and term precision orderings of
the gradually-typed lambda calculus.
%
% In Section \ref{sec:towards-relational-model}, we will discuss how to extend the
% denotational semantics to accommodate the type and term precision orderings. 
%
%necessitate the abstract components that are found in the definition of the model.
%
%\subsection{Semantic Constructions}\label{sec:domain-theory}
%
We begin by discussing the objects of the model into which we will embed the
types and terms of the gradually-typed lambda calculus. Recall that we are
working informally in a guarded type theory with clocks and ticks (see Section
\ref{sec:technical-background}) for details. The constructions in this section
will make significant use of the primitives of guarded type theory.

\subsection{Value and Computation Types}

% TODO there aren't actually computation types in the syntax. They
% arise because we are embedding call-by-value into CBPV.

% TODO should we introduce the lift monad first, and then introduce error
% domains as algebras of the lift monad?

We will give an embedding of our call-by-value syntax into a call-by-push-value
model based on a free-forgetful adjunction $F \dashv U$. Value types will denote
sets. Computation types will denote sets with additional algebraic structure
corresponding to the possible behaviors of a term. Recall that we are modelling
gradually-typed programs. Thus, one such behavior is \emph{failure}: a program
may fail at run-time because of a type error. In addition to this, a program may
fail to terminate. This is possible because the dynamic type includes function
types and so is recursive. We will discuss the dynamic type further in a
subsequent section, but for now it suffices to know that we can express a
diverging computation in our syntax. Therefore, it will be necessary to track
the ``steps'' taken by a term; we can regard this as a way of intensionally
modelling \emph{partiality}.

The semantic objects modelling computations will be sets $B$ that are equipped
with a distinguished error element $\mho_B$ and a map $\theta_B \colon \laterhs
B \to B$. We call such sets \emph{error domains}. A \emph{morphism of error
domains} from $B_1$ to $B_2$ is a function $\phi$ on the underlying sets that
respects the error and $\theta$ maps, i.e., for which 
%
(1) $\phi(\mho_{B_1}) = \mho_{B_2}$ and
%
(2) $\phi(\theta_{B_1}(\tilde{x})) = \theta_{B_2}(\later (\phi(\tilde{x})))$
(using the functorial action of $\later$).

\subsection{The Lift + Error Monad}\label{sec:lift-monad}

% When thinking about how to model gradually-typed programs semantically, we
% should consider their possible behaviors. On the one hand, we have
% \emph{failure}: a program may fail at run-time because of a type error. In
% addition to this, a program may fail to terminate. This is possible because the
% dynamic type includes function types and so is recursive. We will discuss the
% dynamic type further in the next section, but for now it suffices to know that
% we can express a diverging computation in our syntax. Therefore, it will be
% useful to track the ``steps'' taken by a term; we can regard this as a way of
% intensionally modelling \emph{partiality}.

% With this in mind, we can describe a semantic object that models these
% behaviors: a monad for embedding computations that has cases for failure and
% ``stepping''. Previous work has studied such a construct in the setting of the
% latter, called the lift monad \cite{mogelberg-paviotti2016}; here, we augment it
% with the additional effect of failure.

We now describe the monad that decomposes into the free-forgetful adjunction
between sets and error domains. We base the construction on the \emph{guarded
lift monad} described in \cite{mogelberg-paviotti2016}. Here, we augment the
guarded lift monad to accommodate the additional effect of failure. For a type
$A$, we define the \emph{guarded lift monad with failure} $\li A$, which we will
just call the \emph{guarded lift monad} or simply the \emph{lift monad}.
%
\begin{align*}
  \li A :=\quad
  &\alt \eta \colon A \to \li A \\
  &\alt \mho \colon \li A \\
  &\alt \theta \colon \laterhs (\li A) \to \li A
\end{align*}
%
Unless otherwise mentioned, all constructs involving $\later$ or $\fix$ are
understood to be with respect to a fixed clock $k$. So for the above, we really
have for each clock $k$ a type $\li^k A$ with respect to that clock.
%
Formally, the lift monad $\li A$ is defined as the solution to the guarded
recursive type equation
%
\[ \li A \cong A + 1\, + \laterhs \li A. \]
%
An element of $\li A$ should be viewed as a computation that can either (1)
return a value (via $\eta$), (2) raise an error and stop (via $\mho$), or (3)
take a step (via $\theta$).
%
Notice there is a computation $\fix\, \theta$ of type $\li A$. This represents a
computation that runs forever and never returns a value.
%
Since we claimed that $\li A$ is a monad, we need to define the monadic
operations and show that they respect the monadic laws. The return is just
$\eta$, and the monadic extension is defined via guarded recursion by cases on
the input.
% It is instructive to give at least one example of a use of guarded recursion, so
% we show below how to define extend:
% TODO
%
%
Verifying that the monadic laws hold uses \lob-induction and is straightforward.
%
% TODO mention that error domains are algebras of the lift monad?
We observe that the guarded lift monad constructs the free error domain on a set
$A$, i.e., we have $\li = UF$ where $F : \Set \to \errdom$ and $U : \errdom \to
\Set$ and $F \dashv U$. Given a set $A$ and an error domain $B$, the set of
functions $A \to UB$ is naturally isomorphic to the set of error domain
morphisms $FA \to B$.

% The function type A \ra A' will be modeled as \sem{A} \to \li \sem{A'}


\subsection{Modeling the Dynamic Type}\label{sec:dynamic-type}

We now discuss how to model the dynamic type $\dyn$.
% For simplicity, we omit the
% case of products in the dynamic type, as from a semantic viewpoint it is the
% function case that is problematic.
We will define a set $D$ that satisfies the isomorphism
%
% TODO make sure we deal with the product correctly
\[ D \cong \Nat + (D \times D) + (D \to \li D). \]
%
where the $\li D$ represents the fact that in the case where the value of type
$\dyn$ is a function, the function when called may return an error or take a
step of computation. Unfortunately, this equation does not have inductive or
coinductive solutions. The usual way of dealing with such equations is via
domain theory, by which we can obtain an exact solution. However, the machinery
of domain theory can be difficult for language designers to learn and apply in
the mechanized setting. Instead, we will leverage the tools of guarded type
theory. Consider the following similar-looking equation:
%
\[ D \cong \Nat + (D \times D)\, + \laterhs (D \to \li D). \]
%
Since the negative occurrence of $D$ is guarded under a later, this equation has
a solution constructed via a combination of least fixed-point and guarded
recursion. Specifically, consider the parameterized inductive type
%
\[ D'\, X := \mu T. \Nat + (T \times T) + X. \]
%
Now consider the function $f$ defined by
%
\[ \lambda (\tilde{D} \colon \laterhs \type) . D' (\laterhs_t (\tilde{D}_t \to (\li (\tilde{D}_t)))). \]
%
Recall that the tick $t : \tick$ is evidence that time has passed, and since
$\tilde{D}$ has type $\later \type$, i.e. $\tick \to \type$, then $\tilde{D}_t$
has type $\type$.
%
Finally, define 
\[ D := \fix f. \]
%
% TODO refer to technical background?
Observe that the unfolding equality characterizing the guarded fixpoint implies
that $D = D' (\later D \to \li D)$, and by the definition of $D'$ as a least
fixpoint we see that this is equal to $\Nat + (D \times D) + \later (D \to \li
D)$.


\subsection{Term Semantics}\label{sec:term-interpretation}

We can now give a semantics to the types and terms of the gradually-typed lambda
calculus. 
%
Much of the semantics is similar to a normal call-by-value denotational
semantics, so we focus only on the cast semantics. We interpret types as sets;
the interpretation of types is given in Figure \ref{fig:type-interpretation}.
Contexts $\Gamma = x_1 \colon A_1, \dots, x_n \colon A_n$ are interpreted as the
product $\sem{A_1} \times \cdots \times \sem{A_n}$. The semantics of the dynamic
type $\dyn$ is the set $D$ introduced in Section \ref{sec:dynamic-type}. The
product type $A \times A'$ is interpreted as the Cartesian product of the
denotations of $A$ and $A'$. The function type $A \ra A'$ is interpreted as the
set of functions from $\sem{A}$ to $\li (\sem {A'})$.
%
The interpretation of a value $\hasty {\Gamma} V A$ is a function from
$\sem{\Gamma}$ to $\sem{A}$. Likewise, a term $\hasty {\Gamma} M {{A}}$ is
interpreted as a function from $\sem{\Gamma}$ to $\li \sem{A}$.

We now discuss the semantics of the cast terms, shown in Figure
\ref{fig:term-semantics}. Upcasts are pure, so the upcast ${\upc c}$, where $c :
A \ltdyn A'$, denotes an ordinary function from $\sem{A}$ to $\sem{A'}$.
Downcasts are effectful, so a downcast ${\dnc c}$ is interpreted as a morphism
of error domains between $F\sem{A'}$ and $F\sem{A}$. By definition of the
adjunction this is equivalent to an ordinary function between $\sem{A'}$ and
$UF\sem{A}$, i.e., $\sem{A'} \to \li \sem{A}$.

Recall the definition of type precision derivations given in Figure
\ref{fig:typrec}. The semantics of the up- and downcasts for a type precision
derivation $c$ are defined by induction on $c$. For $r(A)$ the up- and downcasts
are simply the identity function. For the composition $c \comp c'$ we simply
compose the cast for $c$ and the cast for $c'$ as functions. For $c_i \ra c_o$
we apply the casts for $c_i$ and $c_o$ in the domain and codomain respectively.
More concretely, for the upcast we are given a function $V_f : A_i \to \li A_o$
and we must define a function $A_i' \to \li A_o'$. The function first downcasts
its argument according to $c_i$, resulting in an element of $\li A_i$. It then
applies $\ext{V_f}{}$ to this value to obtain an element of $\li A_o$. Finally, it
maps the upcast of $c_o$ over the lift monad, obtaining an element of $\li A_o'$
as needed. For the downcast from $A_i' \to \li A_o'$ to $A_i \to \li A_o$, the
resulting function first applies the upcast by $c_i$ to its argument, then
applies the original function, and finally applies the downcast by $c_o$ to the
result. Likewise, for $c_1 \times c_2$, we apply the cast for $c_1$ on the left
and the cast for $c_2$ on the right.

The ``base cases" for the casts are  $\inat$, $\itimes$, and $\iarr$. Recall
that $D$ is isomorphic to $\Nat\, + (D \times D)\, + \laterhs (D \to \li D)$
(say that the sum is left-associative).
%
For $\inat$, the upcast is simply $\inl \circ \inl$. The downcast has type $D
\to \li D$ and performs a case inspection on the sum type. If it is a natural
number $n$, then we return $\eta\, n$. Otherwise, we return $\mho$, modelling
the fact that a run-time error occurs.
%
For $\itimes$, the upcast is $\inl \circ \inr$, and the downcast performs a case
inspection analogously to the natural number downcast.
%
% TODO explain this further
For $\iarr$, the upcast is given by $\inr \circ \nxt$. The downcast performs a
case inspection, and in the $\inr$ case, it uses a $\theta$ in order to gain
access to the function under the $\later$.


%
% TODO write this out explicitly?

% TODO should we include function casts?

\begin{figure*}
  \begin{align*}
    \sem{\nat} &= \Nat \\
    \sem{\dyn} &= D \\
    \sem{A \times A'} &= \sem{A} \times \sem{A'} \\ 
    \sem{A \ra A'} &= \sem{A} \To \li \sem{A'} \\
  \end{align*}
  \caption{Interpretation of types}
  \label{fig:type-interpretation}
\end{figure*}
  
\begin{figure*}
  \eric{Check this}
  
  % TODO check these
  \begin{align*}
    % \sem{\zro}         &= \lambda \gamma . 0 \\
    % \sem{\suc\, V}     &= \lambda \gamma . (\sem{V}\, \gamma) + 1 \\
    % \sem{x \in \Gamma} &= \lambda \gamma . \gamma(x) \\
    % \sem{\lda{x}{M}}   &= \lambda \gamma . \lambda a . \sem{M}\, (*,\, (\gamma , a))  \\
    % \sem{V_f\, V_x}    &= \lambda \gamma . {({(\sem{V_f}\, \gamma)} \, {(\sem{V_x}\, \gamma)})} \\
    % \sem{\err_B}       &= \lambda \gamma . \mho \\
    %
    % upcasts
    \sem{\upc{r(A)}} &= \id \\
    \sem{\upc{c \comp c'}(V)} &= \sem{\upc{c'}} \circ \sem{\upc{c}} \\
    \sem{\upc{c_i \ra c_o}} &=
      \lambda V_f . \text{map } \upc{c_o} \circ \ext{V_f}{} \circ \dnc{c_i} \\
    \sem{\upc{c_1 \times c_2}} &= 
      \lambda (V_1, V_2) . ((\sem{\upc{c_1} V_1}) , 
                            (\sem{\upc{c_2} V_2})) \\
    \sem{\upc{\inat}} &= 
      \lambda V_n . \inl\, \inl\, V_n \\
    \sem{\upc{\itimes}} &= 
      \lambda (V_1, V_2) . \inl\, \inr\, (V_1 , V_2) \\
    \sem{\upc{\iarr}} &= 
      \lambda V_f . \inr\, \nxt V_f \\[2ex]
    %
    % downcasts
    \sem{\dnc{\inat}} &=
      \lambda V_d . \text{case $({V_d})$ of}
      \{ \inl\, (\inl\, n) \to \eta\, n
         \alt \text{otherwise} \to \mho \} \\
    \sem{\dnc{\itimes}} &=
      \lambda V_d . \text{case $({V_d})$ of}
      \{ \inl\, (\inr\, (d_1, d_2)) \to \eta\, (d_1, d_2)
         \alt \text{otherwise} \to \mho \} \\
    \sem{\dnc{\iarr}} &=
      \lambda V_d . \text{case $({V_d})$ of}
      \{ \inr\, \tilde{f} \to \theta\, (\lambda t. \eta (\tilde{f}_t))
         \alt \text{otherwise} \to \mho \} \\
    % \sem{\ret\, V}       &= \lambda \gamma . \eta\, \sem{V} \\
    % \sem{\bind{x}{M}{N}} &= \lambda \delta . \ext {(\lambda x . \sem{N}\, (\delta, x))} {\sem{M}\, \delta} \\
  \end{align*}

  \caption{Semantics of casts.}
  \label{fig:term-semantics}
\end{figure*}






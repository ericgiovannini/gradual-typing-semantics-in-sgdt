\documentclass{article}
\usepackage{amssymb}
\usepackage{amsmath}
\usepackage{amsthm}
\usepackage{tikz-cd}
\usepackage{mathpartir}
\usepackage{stmaryrd}
\usepackage{parskip}
\usepackage{tikz-cd}


\newtheorem{theorem}{Theorem}[section]
\newtheorem{nonnum-theorem}{Theorem}[section]
\newtheorem{corollary}{Corollary}[section]
\newtheorem{definition}{Definition}[section]
\newtheorem{lemma}{Lemma}[section]

\newcommand{\To}{\Rightarrow}
\newcommand{\inl}{\mathsf{inl}}
\newcommand{\inr}{\mathsf{inr}}
\newcommand{\alt}{\mathrel{\bf \,\mid\,}}


\newcommand{\extlc}{\text{Ext-}\lambda}
\newcommand{\extlcm}{\text{Ext-}\lambda^{-\text{trans}}}
\newcommand{\extlcmm}{\text{Ext-}\lambda^{-\text{trans}-\text{cast}}}
\newcommand{\extlcprime}{\text{Ext-}\lambda'}
\newcommand{\intlc}{\text{Int-}\lambda}
\newcommand{\intlcbisim}{\text{Int$_\approx$-}\lambda}
\newcommand{\erase}[1]{\lfloor {#1} \rfloor}


\newcommand{\uarrowl}{\mathrel{\rotatebox[origin=c]{-30}{$\leftarrowtail$}}}
\newcommand{\uarrowr}{\mathrel{\rotatebox[origin=c]{60}{$\leftarrowtail$}}}
\newcommand{\darrowl}{\mathrel{\rotatebox[origin=c]{30}{$\twoheadleftarrow$}}}
\newcommand{\darrowr}{\mathrel{\rotatebox[origin=c]{120}{$\twoheadleftarrow$}}}
\newcommand{\vuarrow}{\mathrel{\rotatebox[origin=c]{-90}{$\leftarrowtail$}}}
\newcommand{\vdarrow}{\mathrel{\rotatebox[origin=c]{90}{$\twoheadleftarrow$}}}

% Types, terms, and precision 

\newcommand{\dyn}{?}
\newcommand{\nat}{\text{Nat}}
\newcommand{\bool}{\text{Bool}}
\newcommand{\ra}{\rightharpoonup}
\newcommand{\Ret}[1]{\mathsf{Ret\,}{#1}}
\newcommand{\hole}[1]{\bullet \colon {#1}}
\newcommand{\dyntodyn}{\dyn \ra\, \dyn}


\newcommand{\up}[2]{\langle{#2}\uarrowl{#1}\rangle}
\newcommand{\dn}[2]{\langle{#1}\darrowl{#2}\rangle}

\newcommand{\upc}[2]{\text{up}\,{#1}\,{#2}}
\newcommand{\dnc}[2]{\text{dn}\,{#1}\,{#2}}


\newcommand{\ret}{\mathsf{ret}}
\newcommand{\err}{\mho}
\newcommand{\zro}{\textsf{zro}}
\newcommand{\suc}{\textsf{suc}}
\newcommand{\lda}[2]{\lambda {#1} . {#2}}
\newcommand{\injarr}[1]{\textsf{Inj}_\ra ({#1})}
\newcommand{\injnat}[1]{\textsf{Inj}_\text{nat} ({#1})}
\newcommand{\casenat}[4]{\text{Case}_\text{nat} ({#1}) \{ \text{no} \to {#2} \alt \text{nat}({#3}) \to {#4} \}}
\newcommand{\casearr}[4]{\text{Case}_\ra ({#1}) \{ \text{no} \to {#2} \alt \text{fun}({#3}) \to {#4} \}}
\newcommand{\casedyn}[5]{\text{Case}_\Dyn ({#1}) \{ \text{nat}({#2}) \to {#3} \alt \text{fun}({#4}) \to {#5} \}}
\newcommand{\bind}[3]{\text{var } {#1} = {#2} \text{ in } {#3}}
\newcommand{\matchnat}[4]{\text{match } {#1} \text{ with } \{ \textsf {zero} \Rightarrow {#2} \alt \textsf{ suc } {#3} \Rightarrow {#4} \}}

\newcommand{\refl}{\text{refl}}

\newcommand{\Lift}{\text{Lift}}

%\newcommand{\rel}{\circ\hspace{-4px}-\hspace{-4px}\bullet}
\newcommand{\rel}{\mathrel{\circ\mkern-6mu-\mkern-6mu\bullet}}
\newcommand{\ltdyn}{\sqsubseteq}
\newcommand{\gtdyn}{\sqsupseteq}
\newcommand{\equidyn}{\mathrel{\gtdyn\ltdyn}}
\newcommand{\gamlt}{\Gamma^\ltdyn}
\newcommand{\deltalt}{\Delta^\ltdyn}
\newcommand{\relcomp}{\odot}

\newcommand{\hasty}[3]{{#1} \vdash {#2} \colon {#3}}
\newcommand{\vhasty}[3]{{#1} \vdash^v {#2} \colon {#3}}
\newcommand{\phasty}[3]{{#1} \vdash^p {#2} \colon {#3}}
\newcommand{\etmprec}[4]{{#1} \vdash {#2} \ltdyn_e {#3} \colon {#4}}
\newcommand{\itmprec}[4]{{#1} \vdash {#2} \ltdyn_i {#3} \colon {#4}}
\newcommand{\etmequidyn}[4]{{#1} \vdash {#2} \equidyn_e {#3} \colon {#4}}
\newcommand{\itmequidyn}[4]{{#1} \vdash {#2} \equidyn_i {#3} \colon {#4}}
\newcommand{\synbisim}{\approx_{\text{syn}}}

\newcommand{\Dwn}{\Downarrow}
\newcommand{\qte}[1]{\text{quote}({#1})}

\newcommand{\elab}[1]{\text{Elab}({#1})}

% Perturbations
\newcommand{\pertp}{\text{Pert}^\text{P}}
\newcommand{\perte}{\text{Pert}^\text{E}}
\newcommand{\pertdyn}[2]{\text{pert-dyn}({#1}, {#2})}
\newcommand{\delaypert}[1]{\text{delay-pert}({#1})}

\newcommand{\pertc}{\text{Pert}_{\text{C}}}
\newcommand{\pertv}{\text{Pert}_{\text{V}}}


% SGDT and Intensional Stuff

\newcommand{\later}{\vartriangleright}
\newcommand{\type}{\texttt{Type}}
\newcommand{\lob}{\text{L\"{o}b}}
\newcommand{\tick}{\mathsf{tick}}
\newcommand{\nxt}{\mathsf{next}}
\newcommand{\fix}{\mathsf{fix}}
\newcommand{\kpa}{\kappa}

% Model-related stuff
\newcommand{\calC}{\mathcal{C}}
\newcommand{\Set}{\mathsf{Set}}
\newcommand{\Yo}{\mathsf{Yo}}
\newcommand{\Hom}{\mathsf{Hom}}
\newcommand{\calS}{\mathcal{S}}
\newcommand{\gfix}{\texttt{gfix}}
\newcommand{\calU}{\mathcal{U}}
\newcommand{\laterhat}{\widehat{\later}}
\newcommand{\El}{\mathsf{El}}
\newcommand{\Clock}{\mathsf{Clock}}

\newcommand{\Machine}[1]{\mathsf{Machine}\, {#1}}


% Predomains and EP pairs
\newcommand{\Nat}{\mathsf{Nat}}
\newcommand{\Dyn}{\mathsf{Dyn}}
\newcommand{\ty}[1]{\langle {#1} \rangle}
\newcommand{\li}{L_\mho}
\newcommand{\liclk}[1]{L_\mho [{#1}]}

\newcommand{\ext}[2]{\text{ext}\,{#1}\,{#2}}
\newcommand{\map}[2]{\text{map}\,{#1}\,{#2}}

\newcommand{\ltls}{\lesssim}
\newcommand{\bisim}{\approx}
\newcommand{\semlt}{\le}
\newcommand{\semltbad}{\lesssim}

%\newcommand{\injarr}{\textsf{Inj}_\to}
%\newcommand{\injnat}{\textsf{Inj}_\mathbb{N}}

\newcommand{\id}{\mathsf{id}}
\newcommand{\ep}{\leadsto}

\newcommand{\emb}[2]{\mathsf{emb}_{#1}({#2})}
\newcommand{\proj}[2]{\mathsf{proj}_{#1}({#2})}

\newcommand{\monto}{\to_m}


% Notation for wait functions
\newcommand{\wre}{w_r^e}
\newcommand{\wle}{w_l^e}
\newcommand{\wrp}{w_r^p}
\newcommand{\wlp}{w_l^p}

\newcommand{\sem}[1]{\llbracket {#1} \rrbracket}
\newcommand{\semgl}[1]{{\sem{#1}}^\text{gl}}


% Denotational model
\newcommand{\ptb}{\text{ptb}}
\newcommand{\push}{\text{push}}
\newcommand{\pull}{\text{pull}}


\begin{document}

\title{Technical Outline}
\author{Eric Giovannini}

\maketitle

\section{The Overall Plan}

The goal is to show graduality of the extensional gradual lambda calculus.
The main theorem is the following:

\begin{theorem}[Graduality]
If $M \ltdyn M'$, then

\begin{itemize}
    \item $M \Downarrow$ iff $M' \Downarrow$.
    \item $M \Downarrow v_?$ iff $M' \Downarrow v_?'$, where $v_? = \err$ or $v_? = v_?'$.
\end{itemize}
\end{theorem}


\section{Syntax}

    % TODO details on admissibility: we can translate equations of the original logic
    % into the new one and show that they hold there

    We begin with a gradually-typed lambda calculus $(\extlc)$, which is similar to
    the normal call-by-value gradually-typed lambda calculus, but differs in that it
    is actually a fragment of call-by-push-value specialized such that there are no
    non-trivial computation types. We do this for convenience, as either way
    we would need a distinction between values and effectful terms; the framework of
    of call-by-push-value gives us a convenient langugae to define what we need.

    We then observe that composition of type precision derivations is admissible, 
    as is heterogeneous transitivity for term
    precision (via casts), so it will suffice to consider a new language ($\extlcm$)
    in which we don't have composition of type precision derivations or transitivity
    of term precision.

    We further observe that all casts, except those between $\nat$ and $\dyn$
    and between $\dyn \ra \dyn$ and $\dyn$, are admissible
    (we can define the cast of a function type
    functorially using the casts for its domain and codomain).

    This means it is sufficient to consider a new language ($\extlcmm$) in which
    instead of having arbitrary casts, we have injections from $\nat$ and
    $\dyn \ra \dyn$ into $\dyn$, and case inspections from $\dyn$ to $\nat$ and
    $\dyn$ to $\dyn \ra \dyn$.

    From here, we define an intensional GSTLC, so-named because it makes the
    intensional stepping behavior of programs explicit in the syntax.
    This is acocmplished by adding a syntactic ``later'' type and a
    syntactic $\theta$ that maps terms of type later $A$ to terms of type $A$.
    
    \subsection{Extensional ($\extlcmm$)}

    \begin{align*}
        &\text{Value Types } A := \nat \alt \dyn \alt (A \ra B) \\
        &\text{Computation Types } B := \Ret A \\
        &\text{Value Contexts } \Gamma := \cdot \alt (\Gamma, x : A) \\
        &\text{Computation Contexts } \Delta := \cdot \alt \hole B \alt \Delta , x : A \\
        &\text{Values } V :=  \zro \alt \suc\, V \alt \injnat V \alt \injarr V \\ 
        &\text{Terms } M, N := \err_B \alt \Ret {V} \alt \bind{x}{M}{N} % \alt \suc\, M, 
          \alt \lda{x}{M} \alt V_f\, V_x \alt
          %\\ & \quad\quad \injnat M \alt \injarr M
          \\ & \quad\quad \casenat{V}{M_{no}}{n}{M_{yes}} 
          \alt \casearr{V}{M_{no}}{f}{M_{yes}}
      \end{align*}

    % TODO is err a value? What about injnat and injarr?
    % TODO should suc be a term and a value? What about injnat and injarr?

    The value typing judgment is written $\hasty{\Gamma}{V}{A}$ and 
    the computation typing judgment is written $\hasty{\Delta}{M}{B}$.

    We define substitution for computation contexts as follows:
    \begin{mathpar}
        \inferrule*
        { \delta : \Delta' \to \Delta \and 
          \hasty{\Delta'|_V}{V}{A}}
        { (\delta , V/x ) \colon \Delta' \to \Delta , x : A }

        \inferrule*
        {}
        {\cdot \colon \cdot \to \cdot}

        \inferrule*
        {\hasty{\Delta'}{M}{B}}
        {M \colon \Delta' \to \hole{B}}
    \end{mathpar}

    The typing rules are as follows:
    \begin{mathpar}
        % Err
        \inferrule*{ }{\hasty {\cdot, \Gamma} {\err_B} B} 
      
        % Zero and suc
        \inferrule*{ }{\hasty \Gamma \zro \nat}
      
        \inferrule*{\hasty \Gamma M \nat} {\hasty \Gamma {\suc\, M} \nat}
      
        % Lambda
        \inferrule* 
        {\hasty {\cdot, \Gamma, x : A} M {\Ret A'}} 
        {\hasty \Gamma {\lda x M} {A \ra A'}}
      
        % App
        \inferrule*
        {\hasty \Gamma {V_f} {A \ra A'} \and \hasty \Gamma {V_x} A}
        {\hasty {\cdot , \Gamma} {M \, N} {\Ret A'}}

        % Ret
        \inferrule*
        {\hasty \Gamma V A}
        {\hasty {\cdot , \Gamma} {\ret V} {\Ret A}}
      
        % Bind
        \inferrule*
        {\hasty \Delta M {\Ret A} \and \hasty{\cdot , \Delta|_V , x : A}{N}{B} } % Need x : A in context
        {\hasty{\Delta}{\bind{x}{M}{N}}{B}}

        % inj-nat
        \inferrule*
        {\hasty \Gamma M \nat}
        {\hasty \Gamma {\injnat M} \dyn}

        % inj-arr 
        \inferrule*
        {\hasty \Gamma M (\dyn \ra \dyn)}
        {\hasty \Gamma {\injarr M} \dyn}

        % Case nat
        \inferrule*
        {\hasty{\Gamma}{V}{\dyn} \and 
          \hasty{\Delta , x : \nat }{M_{yes}}{B} \and 
          \hasty{\Delta}{M_{no}}{B}}
        {\hasty {\Delta} {\casenat{M}{M_{no}}{n}{M_{yes}}} {B}}
      
        % Case arr
        \inferrule*
        {\hasty{\Gamma}{V}{\dyn} \and 
          \hasty{\Delta , x : (\dyn \ra \dyn) }{M_{yes}}{B} \and 
          \hasty{\Delta}{M_{no}}{B}}
        {\hasty {\Delta} {\casearr{M}{M_{no}}{f}{M_{yes}}} {B}}
    \end{mathpar}


    The equational theory is as follows:
    \begin{mathpar}
        % Function Beta and Eta
        \inferrule*
        {\hasty {\cdot, \Gamma, x : A} M {\Ret A'} \and \hasty \Gamma V A}
        {(\lda x M)\, V = M[V/x]}

        \inferrule*
        {\hasty \Gamma V {A \ra A}}
        {\Gamma \vdash V = \lda x {V\, x}}

        % Case-nat Beta
        \inferrule*
        {\hasty \Gamma V \nat}
        {\casenat {\injnat {V}} {M_{no}} {n} {M_{yes}} = M_{yes}[V/n]}

        \inferrule*
        {\hasty \Gamma V {\dyn \ra \dyn} }
        {\casenat {\injarr {V}} {M_{no}} {n} {M_{yes}} = M_{no}}

        % Case-nat Eta
        \inferrule*
        {}
        {\Gamma , x :\, \dyn \vdash M = \casenat{x}{M}{n}{M[(\injnat{n}) / x]} }


        % Case-arr Beta


        % Case-arr Eta


        % Ret Beta and Eta
        \inferrule*
        {}
        {\bind{x}{\ret\, V}{N} = N[V/x]}

        \inferrule*
        {\hasty {\hole{\Ret A} , \Gamma} {M} {B}}
        {\hole{\Ret A}, \Gamma \vdash M = \bind{x}{\bullet}{M[\ret\, x]}}

    \end{mathpar}

    Equivalent terms in the equational theory are considered equal.

    
    We now discuss type and term precision. In our language, we do not have
    ``term precision'' but rather arbitrary monotone relations on types,
    which we denote by $A \rel B$. We have relations on value types, as well
    as on computation types.
    %
    In addition, because we don't have casts in our language, we do not have
    the usual cast rules specifying that casts are least upper bounds and greatest
    lower bounds. Instead, we have rules for our injection and case terms.

    \begin{align*}
        &\text{Value Relations } R := \nat \alt \dyn \alt (R \ra R) \alt \alpha_1 \leq_A \alpha_2 \\
        &\text{Computation Relations } S := \Lift R \\
    \end{align*}

    % We define 

    % \begin{mathpar}
    %     \inferrule*
    %     {\alpha_1 : A_1 \mid \alpha_2 : A_2 \vdash R \text{ rel} \and 
    %      \alpha_1' : A_1' \vdash V_1 : A_1 \and
    %      \alpha_2' : A_2' \vdash V_2 : A_2 }
    %     {\alpha_1' : A_1' \mid \alpha_2' : A_2' \vdash R (V_1/\alpha_1, V_2/\alpha_2)}
    % \end{mathpar}

    % TODO term precision


    \subsection{Intensional}
    In the intensional syntax, we add a type constructor for later, as well as a
    syntactic $\theta$ term and a syntactic $\nxt$ term.
    We add rules for each of these, and also modify the rules for inj-arr and
    case-arr, since now
    the function is not $\Dyn \ra \Dyn$ but rather $\later (\Dyn \ra \Dyn)$.

    % TODO show changes

    We define an erasure function from intensional syntax to extensional syntax
    by induction on the intensional types and terms.
    The basic idea is that the syntactic type $\later A$ erases to $A$,
    and $\nxt$ and $\theta$ erase to the identity.

    


\section{Semantics}
To give a semantics of the extensional, quotiented syntax ($\extlcmm$), we proceed
in a few steps:

\begin{enumerate}
    \item We first give a denotational semantics of the intensional syntax $\intlc$
    using synthetic guarded domain theory, i.e., working internal to the topos of trees.
    The denotation of a value type $A$ will be a poset $\sem{A}$, and the denotation of a computation
    type $\Ret A$ will be the lift $\li \sem{A}$, (i.e., the free error domain on $\sem{A}$). % Ret A is our only computation type
    We refer to this semantics as the \emph{step-indexed semantics}.

    %The denotation of a closed term of type $B$ will be as an element of $\li \sem{B}$.

    \item We then show how to go from the above step-indexed semantics to a set-based
    semantics which is still intensional in that the denotations of terms that differ only in
    their number of ``steps'' will not be equal. The denotations here are a type of
    coinductive ``Machine'' that at each step of computation can either return a result,
    error, or continue running. We call this the \emph{Machine semantics}.
    %
    The passage from step-indexed semantics to the coinductive semantics
    will make use of clock quantification; see below for details.

    \item Finally, we collapse the above intensional Machine-based semantics to an
    extensional semantics, so-called because at this point the information about the
    precise number of steps a Machine has taken has been lost. We care only whether the
    Machine runs to a value or error, or whether it diverges. Of course, we cannot
    in general decide whether a given Machine will halt, so the semantic objects here
    are pairs of a proposition $P$ and a function taking a proof of $P$ to an element of
    the poset.

\end{enumerate}
    
The benefit to working synthetically in step 1 above is that the construction of the logical
relation that proves canonicity for the intensional syntax can be carried out much like
any normal, non-step-indexed logical relation.


\section{Unary Canonicity}

We first want to show soundness:

\begin{lemma}[Soundness]
    In the Machine semantics, none of the following are equal: $\mho, \zro, \suc M, \omega$,
    where $\omega$ represents a diverging program.
\end{lemma}

Next we use a logical relations argument to establish the following:

\begin{lemma}
    $M_i \Dwn^n v_?$ iff $M_i = \delta^n(\qte{v_?})$
\end{lemma}

We will need the following key property relating erasure to the semantics:

\begin{lemma}
    If $\erase{M_i} = M$ and $\erase{M_i'} = M'$, and $M = M'$, 
    then $| \sem{M_i} | = | \sem{M_i'} |$.
\end{lemma}

The idea of the proof is that since $M_i$ and $M_i'$ have equal erasures
(in the extensional equational theory), we can recover from the proof that their
erasures are equal a series of equational rules (e.g., $\beta$ equality), and
apply the intensional analogues of those rules to $M_i$ to ensure that $M_i$ and $M_i'$
differ syntactically only in their number of $\theta$'s.

    

\section{Graduality}






\end{document}
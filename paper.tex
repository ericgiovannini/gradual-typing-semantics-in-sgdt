\documentclass[acmsmall,screen]{acmart}
\usepackage{mathpartir}
\usepackage{stmaryrd}
\usepackage{tikz-cd}

%% Rights management information.  This information is sent to you
%% when you complete the rights form.  These commands have SAMPLE
%% values in them; it is your responsibility as an author to replace
%% the commands and values with those provided to you when you
%% complete the rights form.
\setcopyright{acmcopyright}
\copyrightyear{2018}
\acmYear{2018}
\acmDOI{10.1145/1122445.1122456}

%% These commands are for a PROCEEDINGS abstract or paper.
\acmConference[Woodstock '18]{Woodstock '18: ACM Symposium on Neural
  Gaze Detection}{June 03--05, 2018}{Woodstock, NY}
\acmBooktitle{Woodstock '18: ACM Symposium on Neural Gaze Detection,
  June 03--05, 2018, Woodstock, NY}
\acmPrice{15.00}
\acmISBN{978-1-4503-XXXX-X/18/06}


\begin{document}

\newcommand{\sem}[1]{\llbracket{}{#1}\rrbracket{}}
\newcommand{\opsemGMNV}[4]{#1 \vDash #2 \Downarrow^{#3} #4}
\newcommand{\opsemMV}[2]{\opsemGMNV \gamma {#1} n {#2}}
\newcommand{\opsemMNV}[3]{\opsemGMNV \gamma {#1} {#2} {#3}}

\title{Compositional Mechanized Metatheory for Gradual Typing using Synthetic Guarded Domain Theory}
\author{Eric Giovannini}
\author{Max S. New}

\begin{abstract}
  We develop a semantics for a gradually typed lambda calculus with
  effects that and prove the validity of type-based equational
  reasoning as well as the graduality theorem. Improving on prior
  work, we use a compositional denotational semantics in category of
  preordered sets and monotone functions to develop the theory
  modularly. This means that nearly all theorems and constructions
  come from general facts about the category of preordered sets and
  almost no metatheory is specific to gradually typed languages.

  The main challenge is in developing a suitable model for the dynamic
  type, which requires solving a domain equation. We avoid the
  complexities of both traditional order-theoretic domain theory and
  explicit step-indexed models by using the recently developed
  technique of synthetic guarded domain theory in a guarded type
  theory. This makes constructing solutions to guarded domain
  equations trivial, and allows us to carry out the equivalent of a
  step-indexed logical relation without using explicit step-indexing
  while still maintaining the full expressivity of dependent type
  theory as our metatheoretic setting.

  Finally, the entire semantics has been mechanized in the proof
  assistant Guarded Cubical Agda, with much of the semantics entirely
  in reusable extensions to the standard category theory library. This
  provides a foundation for reusable modular mechanized metatheory of
  gradual typing and other effectful denotational semantics.
\end{abstract}

\section{Introduction}

Gradually typed languages provide a unified setting in which dynamic
and statically typed programming styles can be accommodated along with
free interoperability between static and dynamically typed components.

The metatheory of gradually typed languages is considered especially
challenging, as it inherently involves (1) general recursion and at
least an error effect distinct from non-termination (2) the
verification of type-based equational reasoning to demonstrate the
soundness of static typing and (3) proving graduality, a relational
property of program semantics that establishes that increasing the
amount of static typing in a program results in a refinement of
program behavior. Most prior work to this metatheory relies on purely
operational approaches: an operational semantics for the language is
developed and a simulation-based argument is given for
graduality. These have the advantage of being fairly elementary and so
amenable to direct mechanization in a proof assistant but at the cost
of lacking reusability. Furthermore, equational aspects of the
language are for the most part ignored. A second line of prior work is
to develop generic operational frameworks in which the above can be
carried out: the gradualizer as well as Abstracting Gradual
Typing. These are reusable techniques that apply across languages but
still do not address equational theory and . Finally a third line of
prior work gives a semantic theory of gradual typing based on the idea
that casts form embedding-projection pairs in a 2-categorical
semantics. This approach has been formalized in two ways: one using
classical denotational semantics based on complete partial orders
(CPOs) and another based on step-indexed logical relations over an
untyped operational semantics. This approach has similar benefits of
abstraction to the AGT framework, but suffers from a lack of
reusability. First, classical domain theory is believed to be unable
to model higher-order references, a feature of all practical gradually
typed languages. Second, the proofs based on step-indexed logical
relations, while known to scale to higher-order references are
relatively unstructured and complex. While inspired by the
embedding-projection pair semantics, several components of the proof
that work in the CPO-based semantics break down in the step-indexed
setting for unclear reasons. This work also tends to have a high level
of tedious proof overhead in practice as evidenced that most papers
using this approach include tech reports of repetitive paper proofs
that typically dwarf the size of the paper itself.

In this current work it is our goal to combine the benefits of the
previous approaches: we desire the compositionality and reusability of
the domain theoretic semantics, the genericity of abstracting gradual
typing and the gradualizer, the easy mechanizability of the elementary
operational techniques as well as the generality of step-indexed
approaches.

\section{Call-by-value Gradual Cast Calculus: Syntax and Axioms}

\subsection{}

\section{A Compositional ``Operational'' Semantics}

We start by giving a concrete description of the semantics our
denotational approach provides in the style of a monadic
interpreter.

The values of a syntactic type $A$ in the GCC are interpreted as
elements of a set $\sem A$: values of number type are interpreted as
numbers, values of function type are interpreted as functions with a
monadic output and value of the dynamic type are interpreted in a complex
set that contains numbers and functions as subsets.

For now for concreteness we will work with a simple monad that
supports only errors and divergence with observable computational
steps. This can be described as the composition of two monads $T X =
L(X_\mho)$ where $LX$ is Capretta's coinductive delay monad and
$X_\mho$ is simply the maybe monad where we call the additional
element $\mho$ to connote that it is intended to represent a runtime
error.
%
Capretta's delay monad is a coinductive set defined as the greatest
solution to the equation $L X \cong \textrm{Done}(X)+
\textrm{Running}(LX)$. That is an element is either done with a
produced $X$ or takes an observable computational step. Taking the
greatest fixed point rather than least fixed point allows for the
possibility of a program that runs forever.
%
We think of this as a computational process that requires external
``fuel'' to execute: either the process is complete and we receive a
final value or the computation requires an additional fuel to continue
executing.
%

For each computation $\Gamma \vdash M : A$ we define a 4-place
relation $\opsemGMNV \gamma M n V$ where $\gamma \in
\sem{\Gamma}$, $n \in \mathbb N$ and $V \in \sem{A}$. Further this
relation is a decidable partial function when considering $M,\gamma,n$
as inputs and $V$ as an output and is a partial function when
considering $M,\gamma$ as inputs and $n$ as an output.

\begin{mathpar}
  \opsemMNV {\texttt{zro}} 0 0

  \inferrule
      {\opsemMNV M n m}
      {\opsemMNV {\texttt{suc}(M)} n {m+1}}

  \inferrule
  {R(V_i,n,V_o) \iff \opsemGMNV {(\gamma, x\mapsto V_i)} M n {V_o}}
  {\opsemMNV {\lambda x. M} 0 R}
\end{mathpar}
As well as two 

of values, computations, upcasts and downcasts in GCC mutually recursively 

%% This can be equivalently described without
%% mention of coinduction as follows. First, let $X_\ohm$ be the maybe
%% monad but where we call the additional element $\ohm$ to connote that
%% the extra element represents divergence rather than error. Then an
%% element of the delay monad $l \in L X$ is a function $l : \mathbb N
%% \to X_\ohm$ that is defined on at most one input. The intuition is
%% that an element $l$ represents a computational procedure that when
%% executed may possibly diverge or run in some finite number of steps to
%% a value in $X$. Applying $l$ to a number $n$ is asking the question 

%% : an element of $L_\mho X$ is a
%% function $\mathbb N \to X_{\ohm}$ 

\subsection{Abstracting GCC Semantics}





% gradual typing, graduality

% synthetic guarded domain theory, denotational semantics therein

% Difficulties in prior semantics

\section{Discussion}

\subsection{Synthetic Ordering}

While the use of synthetic guarded domain theory allows us to very
conveniently work with non-well-founded recursive constructions while
abstracting away the precise details of step-indexing, we do work with
the error ordering in a mostly analytic fashion in that gradual types
are interpreted as sets equipped with an ordering relation, and all
terms must be proven to be monotone.
%
It is possible that a combination of synthetic guarded domain theory
with \emph{directed} type theory would allow for an a synthetic
treatment of the error ordering as well.

\end{document}
